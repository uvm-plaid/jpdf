\section{Confidentiality Types for $\minicat$}


\newcommand{\sx}[2]{\elab{\secret{#1}}{#2}}
\newcommand{\mx}[2]{\elab{\mesg{#1}}{#2}} 
%\newcommand{\px}[2]{\elab{\rvl{#1}}{#2}} 
\newcommand{\rx}[2]{\elab{\flip{#1}}{#2}}
\newcommand{\ox}[2]{\elab{\out{#1}}{#2}}
\newcommand{\signals}{\leadsto}

\newcommand{\tj}[5]{#1,#2 \vdash_{#3} #4 : #5}
\newcommand{\cty}[2]{c(#1,#2)}
\newcommand{\setit}[1]{\{ #1 \}}
\newcommand{\ty}{T}
\newcommand{\eqs}{C}
\newcommand{\toeq}[1]{\lfloor #1 \rfloor}
\newcommand{\autheq}[1]{\phi_{\mathrm{auth}}(#1)}
\newcommand{\upgrade}[1]{\uparrow #1}

\begin{definition}
  A \emph{ciphertype} is of the form $\cty{x}{\ty}$ where $\ty$ is a \emph{type} which is a set
  of either either variables $x$ or ciphertypes. A \emph{type environment} $\Gamma$ is a finite
  list of type bindings. 
\end{definition}

Intuitively, a type is the set of variables on which a given value may
depend, and a ciphertype $\cty{x}{\ty}$ denotes a value with type
$\ty$ that has been encrypted using $x$ as a one-time-pad. So for
example, the following is the type of $\mx{s1}{2}$ in the example in Section
\ref{section-lang-example}:
$$
\setit{\cty{\elab{\locflip}{1}}{\setit{\cty{\rx{x}{1}}{\setit{\sx{1}{1}}}}}}
$$
and the following is the type of $\mx{s1}{3}$:
$$
\setit{\rx{x}{1}}
$$

Further, the union of the types of multiple values represents their
joint dependencies. We make the conservative approximation that any
dependency on a one-time-pad $x$ ``unlocks'' dependencies of a value
encrypted with $x$. To capture a notion of closure under this
interpretation we define the type rewrite relation $\signals$.
\begin{mathpar}
  (\setit{x,\cty{x}{\ty'}} \cup \ty) \signals (\setit{x} \cup \ty' \cup \ty)

  \inferrule
  {\ty_1 \signals \ty_2 \\ \ty_2 \signals \ty_3}
  {\ty_1 \signals \ty_3}

  \ty \signals \ty
\end{mathpar}

Then, a set of values is considered high security if it has any
dependencies on honest secrets revealed by $\signals$. This allows us
to associate \emph{security labels} $\hilab$ and $\lolab$ with types.
Note that this implicitly assumes that encrypted values, random
samples, and corrupt secrets are all low security.
$$
\seclab(\ty) =
\begin{cases}
  \hilab \text{\ if\ } \ty \signals \{ \sx{w}{\cid} \} \cup \ty' \text{\ and\ } \cid \in H \\
  \lolab \text{\ otherwise}
\end{cases}
$$

Expression type judgements are of the form
$\tj{\Gamma}{R}{\cid}{\be}{\ty}$, where $\Gamma$ is a classical type
environment and $R$ is a linear type environment.  The latter ensures
that a random variable $\elab{\flip{w}}{\cid}$ is only used once for
this purpose, though it may be used more than once as a classical type
that reflects its dependency. To help enforce this in the analysis We
write $R_1;R_2$ to denote the union of $R_1$ and $R_2$ if $R_1$ and
$R_2$ are disjoint, otherwise $R_1;R_2$ is undefined. The type of
programs $\Gamma,R \vdash \prog : \Gamma'$ can be viewed as a Hoare
Type where $\Gamma$ represents dependency preconditions in memory and
$\Gamma'$ are dependencies resulting from execution of $\prog$.

The most interesting rule is $\TirName{Share}$ which generates
a ciphertype, given a random sample used as a one-time-pad that
from which the encrypted value is either added or subtracted.
Additionally, the linear use of the sample is registered in
the judgement. 
\begin{mathpar}
  \inferrule[Value] {}
            {\tj{\Gamma}{\varnothing}{\cid}{v}{\varnothing}}
  
  \inferrule[Secret]
  {}
  {\tj{\Gamma}{\varnothing}{\cid}{\secret{w}}{\setit{\sx{w}{\cid}}}}
    
  \inferrule[Rando]
  {}
  {\tj{\Gamma}{\varnothing}{\cid}{\flip{w}}{\setit{\rx{w}{\cid}}}}
  
  \inferrule[Mesg]
  {}
  {\tj{\Gamma}{\varnothing}{\cid}{\mesg{w}}{\Gamma(\mx{w}{\cid})}}
    
  \inferrule[PubM]
  {}
  {\tj{\Gamma}{\varnothing}{\cid}{\rvl{w}}{\Gamma(\rvl{w})}}

  \inferrule[Share]
  {\tj{\Gamma}{R}{\cid}{\be}{\ty} \\ \oplus \in \{ \fplus,\fminus \}}
  {\tj{\Gamma}{R;\rx{w}{\cid}}{\cid}{\be \oplus \flip{w}}{\setit{\cty{\rx{w}{\cid}}{\ty}}}}

  \inferrule[Binop]
  {\tj{\Gamma}{R_1}{\cid}{\be_1}{\ty_1} \\ \tj{\Gamma}{R_1}{\cid}{\be_1}{\ty_1} \\
              \oplus \in \{ \fplus,\fminus,\ftimes \}}
  {\tj{\Gamma}{R_1;R_2}{\cid}{\be_1 \oplus \be_2}{\ty_1 \cup \ty_2}}

  \inferrule[Command]
  {\tj{\Gamma}{R_1}{\cid}{\be}{\ty} \\ \Gamma; x : \ty,R_2 \vdash \prog : \Gamma'}
  {\Gamma,R_1;R_2 \vdash x := \elab{\be}{\cid};\prog : \Gamma'} 

  \inferrule[Term]
  {}
  {\Gamma,\varnothing \vdash \varnothing: \Gamma}          
\end{mathpar}

Then a main result is that protocols where the collective messages received by the
adversary are low security satisfy gradual release.
\begin{theorem}[Confidentiality Type Safety]
  Given $\Gamma_1,R \vdash \prog : \Gamma_2$ where $\Gamma_1$ is sound for
  preprocessing and:
  $$\seclab(\bigcup_{\mx{w}{\cid} \in X} \Gamma_2(\mx{w}{\cid}) \cup \flips(\prog)_C) = \lolab$$
  for $X \defeq \dom(\Gamma_2)_C$.
  Then $\prog$ satisfies gradual release.
\end{theorem}

This definition accounts for preprocessing of memories. In the default
case where initial memories just contain secrets and the random tape,
we can take $\Gamma_1$ to be $\varnothing$ In cases of nontrivial
pre-processing we need to define $\Gamma_1$ \cnote{and also a formal
  notion of soundness}. For example, in a 2-client Beaver Triple
scheme, we imagine that client 1's secret $x$ is shared as
messages $\mx{x}{1}$ and $\mx{x}{x}$ in the initial memory, obtained
via the ``0-time-pad'' use of the random value $\rx{x}{1}$.
So in this case $\Gamma_1$ would contain:
\begin{mathpar}
\mx{x}{1} : \setit{\rx{x}{1}}

\mx{x}{2} : \setit{\cty{\rx{x}{1}}{\setit{\sx{x}{1}}}}
\end{mathpar}
Similarly, we imagine that clients 1 and 2 possess respective
shares of random values $a$ and $b$ on their random tape:
$$
\rx{a}{1}\quad  \rx{a}{2}\quad \rx{b}{1}\quad \rx{b}{2} 
$$
as well as shares $\mx{c}{1}$ and $\mx{c}{2}$ of a value
$c \defeq a * b$, i.e.:
$$c \defeq
(\rx{a}{1} \fplus \rx{a}{2}) \ftimes (\rx{b}{1} \fplus \rx{b}{2})$$
so $\Gamma_1$ contains:
\begin{mathpar}
  \mx{c}{1} : \setit{\rx{c}{1}}

\mx{c}{2} : \setit{ \cty{\rx{c}{1}}{\setit{\rx{a}{1},\rx{a}{2},\rx{b}{1},\rx{b}{2}}} }
\end{mathpar}

\section{Confidentiality and Integrity Types for $\minifed$}


\begin{mathpar}
  \inferrule[Value] {}
            {\tj{\Gamma}{\varnothing}{\cid}{v}{\varnothing}}
  
  \inferrule[Secret]
  {}
  {\tj{\Gamma}{\varnothing}{\cid}{\secret{w}}{\setit{\sx{w}{\cid}}}}
    
  \inferrule[Rando]
  {}
  {\tj{\Gamma}{\varnothing}{\cid}{\flip{w}}{\setit{\rx{w}{\cid}}}}
  
  \inferrule[Mesg]
  {}
  {\tj{\Gamma}{\varnothing}{\cid}{\mesg{w}}{\Gamma(\mx{w}{\cid})}}
    
  \inferrule[PubM]
  {}
  {\tj{\Gamma}{\varnothing}{\cid}{\rvl{w}}{\Gamma(\rvl{w})}}

  \inferrule[Share]
  {\tj{\Gamma}{R}{\cid}{\be}{\ty} \\ \oplus \in \{ \fplus,\fminus \}}
  {\tj{\Gamma}{R;\rx{w}{\cid}}{\cid}{\be \oplus \flip{w}}{\setit{\cty{\rx{w}{\cid}}{\ty}}}}

  \inferrule[Binop]
  {\tj{\Gamma}{R_1}{\cid}{\be_1}{\ty_1} \\ \tj{\Gamma}{R_1}{\cid}{\be_1}{\ty_1} \\
              \oplus \in \{ \fplus,\fminus,\ftimes \}}
  {\tj{\Gamma}{R_1;R_2}{\cid}{\be_1 \oplus \be_2}{\ty_1 \cup \ty_2}}

  \inferrule[Command]
  {\tj{\Gamma}{R_1}{\cid}{\be}{\ty} \\ \Gamma; x : \ty,R_2, \eqs \cup \{ x = \toeq{\elab{\be}{\cid}} \} \vdash \prog : \Gamma'}
  {\Gamma,R_1;R_2, \eqs \vdash x := \elab{\be}{\cid};\prog : \Gamma'} 

  \inferrule[Assert]
  {\eqs \vdash \toeq{\elab{\be_1}{\cid}} = \toeq{\elab{\be_2}{\cid}} 
              \\ \Gamma; R, \eqs \vdash \prog : \Gamma'}
  {\Gamma,R,\eqs \vdash \elab{\assert{\be_1 = \be_2}}{\cid};\prog : \Gamma'} 

  \inferrule[Auth]
  {\eqs \vdash \toeq{\autheq{\mesg{w}}}
              \\ \Gamma; \elab{\mesg{w}}{\cid} : \upgrade{\Gamma(\elab{\mesg{w}}{\cid})}, R, \eqs \vdash \prog : \Gamma'}
  {\Gamma,R,\eqs \vdash \elab{\assert{\autheq{\mesg{w}}}}{\cid};\prog : \Gamma'} 

  \inferrule[Term]
  {}
  {\Gamma,\varnothing \vdash \varnothing: \Gamma,\varnothing}          
\end{mathpar}


