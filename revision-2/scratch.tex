\newcommand{\sx}[2]{\elab{\secret{#1}}{#2}}
\newcommand{\mx}[2]{\elab{\mesg{#1}}{#2}} 
%\newcommand{\px}[2]{\elab{\rvl{#1}}{#2}} 
\newcommand{\rx}[2]{\elab{\flip{#1}}{#2}}
\newcommand{\ox}[2]{\elab{\out{#1}}{#2}}
\newcommand{\signals}[1]{\stackrel{#1}{\leadsto}}

\newcommand{\tj}[6]{#1,#2,#3 \vdash_{#4} #5 : #6}
\newcommand{\ej}[6]{#1,#2,#3 \vdash #4 : #5,#6}
\newcommand{\cty}[2]{c(#1,#2)}
\newcommand{\setit}[1]{\{ #1 \}}
\newcommand{\ty}{T}
\newcommand{\lty}[2]{#1 \cdot #2}
\newcommand{\eqs}{\mathit{E}}
\newcommand{\toeq}[1]{\lfloor #1 \rfloor}
\newcommand{\eop}{\equiv}
\newcommand{\autheq}[1]{\phi_{\mathrm{auth}}(#1)}
\newcommand{\upgrade}[1]{\uparrow #1}
\newcommand{\seclev}{\mathcal{L}}

\newcommand{\eqj}[4]{#1,#2 \vdash #3 : #4}
\newcommand{\cpj}[4]{#1,#2 \vdash #3 : #4}
\newcommand{\leakj}[3]{#1,#2 \vdash #3}

\renewcommand{\redx}{\Rightarrow}
\renewcommand{\redxs}{\redx}
\newcommand{\abort}{\bot}
\newcommand{\pre}[1]{\ttt{pre}(#1)}
\newcommand{\post}[1]{\ttt{post}(#1)}
\newcommand{\eqflag}{\mathit{sw}}
\newcommand{\eqon}{\ttt{on}}
\newcommand{\eqoff}{\ttt{off}}
\newcommand{\eqtrans}[1]{\lfloor #1 \rfloor}
\newcommand{\mc}[4]{(#1,#2,#3,#4)}
\newcommand{\cmd}{\instr}

\section{$\minicat$ Syntax and Semantics}

$$
    \begin{array}{rcl@{\hspace{2mm}}r}
      \multicolumn{4}{l}{v \in \mathbb{F}_p,\ w \in \mathrm{String},\ \cid \in \mathrm{Clients} \subset  \mathbb{N} }\\[2mm] %, \bop \in \{ \eand, \eor, \exor \}} \\[2mm]
      \be &::=& \flip{w} \mid \secret{w} \mid \mesg{w} \mid \rvl{w} \mid & \textit{expressions}\\
      & & v \mid \be \fminus \be \mid \be \fplus \be \mid \be \ftimes \be \\[2mm]
      x &::=& \elab{\flip{w}}{\cid} \mid \elab{\secret{w}}{\cid} \mid \elab{\mesg{w}}{\cid} \mid \rvl{w} \mid \out{\cid} & \textit{variables} \\[2mm]
      %& &  \select{\be}{\be}{\be} \mid \ctxt{v}{k} \mid \key{w} \mid \sk{\be}(\be) \mid \pk{\be}{\be}(\be) \mid \pk{\be}{\be} \\[2mm]
      %& &  \select{\fp(\be)}{\be}{\be} \ctxt{v,\be}{k}  \mid \sk{\be}(\be) \mid \pk{\be}{\be}(\be) \mid \pk{\be}{\be} \\[2mm]
      \prog &::=& \eassign{\mesg{w}}{\cid}{\be}{\cid} \mid
      \reveal{w}{e}{\cid} \mid \pubout{\cid}{\be}{\cid} \mid \prog;\prog & \textit{protocols} 
    \end{array}
$$

\bigskip
    
 $$
  %\begin{array}{c@{\hspace{5mm}}c}
  \begin{array}{rcl}
    \lcod{\store, v}{\cid} &=& v\\
    \lcod{\store, \be_1 \fplus \be_2}{\cid} &=& \fcod{\lcod{\store, \be_1}{\cid} \fplus \lcod{\store, \be_2}{\cid}}\\ 
    \lcod{\store, \be_1 \fminus \be_2}{\cid} &=& \fcod{\lcod{\store, \be_1}{\cid} \fminus \lcod{\store, \be_2}{\cid}}\\ 
    \lcod{\store, \be_1 \ftimes \be_2}{\cid} &=& \fcod{\lcod{\store, \be_1}{\cid} \ftimes \lcod{\store, \be_2}{\cid}}\\
  %\end{array} 
  %\begin{array}{rcl}
    \lcod{\store, \flip{w}}{\cid} &=& \store(\elab{\flip{w}}{\cid})\\
    \lcod{\store, \secret{w}}{\cid} &=& \store(\elab{\secret{w}}{\cid})\\
    \lcod{\store, \mesg{w}}{\cid} &=& \store(\elab{\mesg{w}}{\cid})\\
    \lcod{\store, \rvl{w}}{\cid} &=& \store(\rvl{w})\\
    %\lcod{\store, f\,\be_1\,\cdots\, \be_n}{\cid} &=& \delta(f,\lcod{\store, \be_1}{\cid},\ldots,\lcod{\store,\be_n}{\cid})
  \end{array}
  %\end{array}
  $$

\bigskip

  \begin{mathpar}
    (\store, \xassign{x}{\be}{\cid}) \redx \extend{\store}{x}{\lcod{\store,\be}{\cid}}

    \inferrule
    {(\store_1,\prog_1) \redx \store_2 \\ (\store_2,\prog_2) \redx \store_3 }
    {(\store_1,\prog_1;\prog_2) \redx \store_3}
    %(\store, \eassign{\mesg{w}}{\cid_1}{\be}{\cid_2};\prog) \redx (\extend{\store}{\mesg{w}_{\cid_1}}{\lcod{\store,\be}{\cid_2}}, \prog)    
    %(\store, \reveal{w}{\be}{\cid};\prog) \redx (\extend{\store}{\rvl{w}}{\lcod{\store,\be}{\cid}}, \prog)   
    %(\store, \pubout{\cid}{\be}{\cid};\prog) \redx (\extend{\store}{\out{\cid}}{\lcod{\store,\be}{\cid}}, \prog)
  \end{mathpar}

\section{$\minicat$ Adversarial Semantics}

$$
\begin{array}{rclr}
  (\store, \xassign{x}{\be}{\cid}) &\aredx&
  \extend{\store}{x}{\lcod{\store,\be}{\cid}} & \cid \in H\\
  (\store, \xassign{x}{\be}{\cid}) &\aredx&
  \extend{\store}{x}{\lcod{\arewrite(\store_C,\be)}{\cid}} & \cid \in C
\end{array}
$$

$$
\begin{array}{rcl@{\qquad}r}
  (\store,\elab{\assert{\be_1 = \be_2}}{\cid}) &\aredx& \store & \text{if\ }
  \lcod{\store,\be_1}{\cid} = \lcod{\store,\be_2}{\cid}  \text{\ or\ } \cid \in C\\
  (\store,\elab{\assert{\phi(\be)}}{\cid}) &\aredx& \abort & \text{if\ } \neg\phi(\store,\lcod{\store,\be}{\cid})
\end{array}
$$

\begin{mathpar}
  \inferrule
      {(\store_1,\prog_1) \aredx \store_2 \\ (\store_2,\prog_2) \aredx \store_3 }
      {(\store_1,\prog_1;\prog_2) \aredx \store_3}

  \inferrule
      {(\store_1,\prog_1) \aredx \abort}
      {(\store_1,\prog_1;\prog_2) \aredx \abort}
      
  \inferrule
      {(\store_1,\prog_1) \aredx \store_2 \\ (\store_2,\prog_2) \aredx \abort }
      {(\store_1,\prog_1;\prog_2) \aredx \abort}
\end{mathpar}

\section{$\minicat$ Constraint Typing}

$$
\begin{array}{rcl}
  \phi &::=& x \mid \phi \fplus \phi \mid \phi \fminus \phi \mid \phi \ftimes \phi \\
  \eqs &::=& \phi \eop \phi \mid \eqs \wedge \eqs 
\end{array}
$$

We write $\eqs_1 \models \eqs_2$ iff every model of $E_1$ is a model of $E_2$. Note that
this relation is reflexive and transitive.

\begin{mathpar}
  \toeq{x} = x

  \toeq{\elab{\be_1 \fplus \be_2}{\cid}} = \toeq{\elab{\be_2}{\cid}} \fplus \toeq{\elab{\be_1}{\cid}}

  \toeq{\elab{\be_1 \fminus \be_2}{\cid}} = \toeq{\elab{\be_2}{\cid}} \fminus \toeq{\elab{\be_1}{\cid}}

  \toeq{\elab{\be_1 \ftimes \be_2}{\cid}} = \toeq{\elab{\be_2}{\cid}} \ftimes \toeq{\elab{\be_1}{\cid}}
\end{mathpar}

\begin{mathpar}
  \toeq{\xassign{x}{\be}{\cid}} = x \eop \toeq{\elab{\be}{\cid}}

  \toeq{\prog_1;\prog_2} = \toeq{\prog_1} \wedge \toeq{\prog_2} 
\end{mathpar}

The motivating idea is that we can interpret any protocol $\prog$ as a set
of equality constraints $\toeq{\prog}$ and use an SMT solver to verify
properties relevant to correctness, confidentiality, and integrity.
Further, we can leverage entailment relation is critical for efficiency--
we can use annotations to obtain a weakened precondition for relevant properties.
That is, given $\prog$, program annotations or other cues can be used
to find a minimal $\eqs$ with $\toeq{\prog} \models \eqs$ for verifying
correctness and security.

\subsection{Confidentiality Types}

\begin{mathpar}
  \inferrule[DepTy]
  {}
  {\eqj{\varnothing}{\eqs}{\phi}{\vars(\phi)}}
  
  \inferrule[Encode]
  {\eqs \models \phi \eop \phi' \oplus \rx{w}{\cid} \\
   \oplus \in \{ \fplus,\fminus \}\\
   \eqj{R}{\eqs}{\phi'}{\ty}}
  {\eqj{R;\{ \rx{w}{\cid} \}}{\eqs}{\phi}{\setit{\cty{\rx{w}{\cid}}{\ty}}}}
\end{mathpar}

\begin{mathpar}
  \inferrule[Send]
            {\eqj{R}{\eqs}{\toeq{\elab{\be}{\cid}}}{\ty}}
            {\cpj{R}{\eqs}{\xassign{x}{\be}{\cid}}{x : \ty}}
            
  \inferrule[Seq]
            {\cpj{R_1}{\eqs}{\prog_1}{\Gamma_1}\\
             \cpj{R_2}{\eqs}{\prog_2}{\Gamma_2}}
            {\cpj{R_1;R_2}{\eqs}{\prog_1;\prog_2}{\Gamma_1;\Gamma_2}}
\end{mathpar}

\begin{definition}
  $\cpj{R}{\eqs}{\prog}{\Gamma}$ is \emph{valid} iff it is derivable and $\toeq{\prog} \models \eqs$.
\end{definition}

\begin{mathpar}
  \inferrule
      {\cid \in C}
      {\leakj{\Gamma}{C}{\Gamma(\mx{w}{\cid})}}

  \inferrule
      {\leakj{\Gamma}{C}{T_1 \cup T_2}}
      {\leakj{\Gamma}{C}{T_1}}

  \inferrule
      {\leakj{\Gamma}{C}{\setit{ \mx{w}{\cid} }}}
      {\leakj{\Gamma}{C}{\Gamma(\mx{w}{\cid})}}

  \inferrule
      {\leakj{\Gamma}{C}{\setit{ \rx{w}{\cid} }} \\ \leakj{\Gamma}{C}{\setit{ \cty{\rx{w}{\cid}}{\ty} }} }
      {\leakj{\Gamma}{C}{\ty}}
\end{mathpar}

\begin{theorem}
  If $\cpj{R}{\eqs}{\prog}{\Gamma}$ is valid and for all $H,C$
  it is not the case that $\leakj{\Gamma}{C}{\setit{\sx{w}{\cid}}}$ for $\cid \in H$,
  then $\prog$ satisfies gradual release.
\end{theorem}

\subsection{Integrity Types}

\begin{mathpar}
  \inferrule[Value]
  {}
  {\tj{\Gamma}{\varnothing}{\eqs}{\cid}{v}{\lty{\varnothing}}{\hilab}}
  
  \inferrule[Secret]
  {}
  {\tj{\Gamma}{\varnothing}{\eqs}{\cid}{\secret{w}}{\lty{\setit{\sx{w}{\cid}}}{\seclev(\cid)}}}
  
  \inferrule[Rando]
  {}
  {\tj{\Gamma}{\varnothing}{\eqs}{\cid}{\flip{w}}{\lty{\setit{\rx{w}{\cid}}}{\seclev(\cid)}}}  
  
  \inferrule[Mesg]
  {}
  {\tj{\Gamma}{\varnothing}{\eqs}{\cid}{\mesg{w}}{\Gamma(\mx{w}{\cid})}}
    
  \inferrule[PubM]
  {}
  {\tj{\Gamma}{\varnothing}{\eqs}{\cid}{\rvl{w}}{\Gamma(\rvl{w})}}

  \inferrule[IntegrityWeaken]
  {\tj{\Gamma}{R}{\eqs}{\cid}{\be}{\lty{\ty}{\latel_1}} \\ \latel_1 \sle \latel_2}
  {\tj{\Gamma}{R}{\eqs}{\cid}{\be}{\lty{\ty}{\latel_2}}}

  \inferrule[Encode]
  {\tj{\Gamma}{\varnothing}{\eqs}{\cid}{\be}{\lty{\ty}{\latel}} \\
    \eqs \models \toeq{\elab{\be}{\cid}} = \phi \oplus \rx{w}{\cid'} \\ \oplus \in \{ \fplus,\fminus \}}
  {\tj{\Gamma}{\rx{w}{\cid}}{\eqs}{\cid}{\be}{\lty{\setit{\cty{\rx{w}{\cid'}}{\Gamma(\phi)}}}}{\latel}}
  
  %\inferrule[RandoDeduce]
  %{\tj{\Gamma}{\varnothing}{\eqs}{\cid}{\be}{\lty{\ty}{\latel}} \\ \eqs \models
  %            \toeq{\elab{\be}{\cid}} = \rx{w}{\cid}'}
  %{\tj{\Gamma}{\varnothing}{\eqs}{\cid}{\be}{\lty{\setit{\rx{w}{\cid}}}{\latel}}}
  %
  %\inferrule[Encode]
  %{\tj{\Gamma}{R_1}{\eqs}{\cid}{\be_1}{\lty{\ty}{\latel}} \\
  % \tj{\Gamma}{R_2}{\eqs}{\cid}{\be_2}{\lty{\setit{\rx{w}{\cid}}}{\latel}} \\ \oplus \in \{ \fplus,\fminus \}}
  %{\tj{\Gamma}{R1;R_2;\rx{w}{\cid}}{\eqs}{\cid}{\be_1 \oplus \be_2}{\lty{\setit{\cty{\rx{w}{\cid}}{\ty}}}}{\latel}}
  %
  \inferrule[Binop]
  {\tj{\Gamma}{R_1}{\eqs}{\cid}{\be_1}{\lty{\ty_1}{\latel}} \\
   \tj{\Gamma}{R_2}{\eqs}{\cid}{\be_2}{\lty{\ty_2}{\latel}} \\ \oplus \in \{ \fplus,\fminus,\ftimes \}}
  {\tj{\Gamma}{R_1;R_2}{\eqs}{\cid}{\be_1 \oplus \be_2}{\lty{\ty_1 \cup \ty_2}}{\latel}}
\end{mathpar}

\begin{mathpar}
  \inferrule[Send]
            {\tj{\Gamma}{R}{\eqs}{\cid}{\be}{\lty{\ty}{\seclev(\cid)}} \\
             \eqs' \models \eqs \wedge x = \toeq{\elab{\be}{\cid}}}
            {\ej{\Gamma}{R}{\eqs}{\xassign{x}{\be}{\cid}}{\Gamma; x : \lty{\ty}{\seclev(\cid)}}
              {\eqs'}}
             
  \inferrule[Assert]
            {\eqs \models \toeq{\elab{\be_1}{\cid}} = \toeq{\elab{\be_2}{\cid}}}
            {\ej{\Gamma}{R}{\eqs}{\elab{\assert{\be_1 = \be_2}}{\cid}}{\Gamma}{\eqs}}
            
  \inferrule[Seq]
            {\ej{\Gamma_1}{R_1}{\eqs_1}{\prog_1}{\Gamma_2}{\eqs_2}\\
             \ej{\Gamma_2}{R_2}{\eqs_2}{\prog_2}{\Gamma_3}{\eqs_3}}
            {\ej{\Gamma_1}{R_1;R_2}{\eqs_1}{\prog_1;\prog_2}{\Gamma_3}{\eqs_3}}

  \inferrule[Constraint]
      {\ej{\Gamma_1}{R}{\eqs_1}{\prog}{\Gamma_2}{\eqs_2} \\ \eqs_1' \models \eqs_1' \\ \eqs_2 \models \eqs_2'}
      {\ej{\Gamma_1}{R}{\eqs_1'}{\prog}{\Gamma_2}{\eqs_2'}}
             
  \inferrule[MAC]
            {\eqs \models 
              \mx{w\ttt{m}}{\cid} = \mx{w\ttt{k}}{\cid} \fplus \ttt{(}\mx{\ttt{delta}}{\cid} \ftimes
                  \mx{w\ttt{s}}{\cid}\ttt{)} \\
              \Gamma(\mx{w\ttt{s}}{\cid}) = \lty{\ty}{\latel}}
            {\ej{\Gamma}{R}{\eqs}{
                \elab{\assert{\mesg{w\ttt{m}} = \mesg{w\ttt{k}} \fplus \ttt{(}\mesg{\ttt{delta}} \ftimes
                  \mesg{w\ttt{s}}\ttt{)}}}{\cid}}{\Gamma;\mx{w\ttt{s}}{\cid}:\lty{\ty}{\hilab}}{\eqs}}
\end{mathpar}


\section{$\metaprot$ Syntax and Semantics}

$$
\begin{array}{rcl}
  \multicolumn{3}{l}{\flab \in \mathrm{Field},\   y \in \mathrm{EVar}, \  f \in \mathrm{FName}}\\[1mm]
  %x &\in& \mathrm{EVar}\\
  %f &\in& \mathrm{FName}\\[2mm]
  e &::=& \mv \mid \flip{e} \mid \secret{e} \mid \mesg{e} \mid \rvl{e} \mid e \bop e \mid
  \elet{y}{e}{e} \mid \\
  & & f(e,\ldots,e) \mid \{ \flab = e; \ldots; \flab = e \} \mid e.\flab \\
  %  & \textit{expressions}\\
  \cmd &::=& \msend{e}{e}{e}{e} \mid \reveal{e}{e}{e} \mid \pubout{e}{e}{e} \mid
      \elab{\assert{e = e}}{e} \mid \\
  & & f(e,\ldots,e) \mid  \cmd;\cmd \mid \pre{\eqs} \mid \post{\eqs} \\[1mm]
  \bop &::=& \fplus \mid \fminus \mid \ftimes \mid \concat  \\[1mm]% \textit{operators}\\[2mm]
  \mv &::=& w \mid \cid \mid \be \mid \{ \flab = \mv;\ldots;\flab = \mv \} 
  \\ % \mid \ttt{()} \\[1mm] %& \textit{values}\\[2mm]
  \mathit{fn} &::=& f(y,\ldots,y) \{ e \} \mid  f(y,\ldots,y) \{ \cmd \} \\[1mm]%& \textit{functions}
  \phi &::=& \elab{\flip{e}}{e} \mid \elab{\secret{e}}{e} \mid \elab{\mesg{e}}{e} \mid \rvl{e} \mid \out{e} \mid \phi \fplus \phi \mid \phi \fminus \phi \mid \phi \ftimes \phi \\
  \eqs &::=& \phi \eop \phi \mid \eqs \wedge \eqs 
\end{array}
$$

\begin{mathpar}
  \inferrule
      {e[\mv/y] \redx \mv'}
      {\elet{y}{\mv}{e} \redx \mv'}

  \inferrule
      {\codebase(f) = y_1,\ldots,y_n,\ e \\ e_1 \redx \mv_1 \cdots e_n \redx \mv_n \\
        e[\mv_1/y_1]\cdots[\mv_n/y_n] \redx \mv}
      {f(e_1,\ldots,e_n) \redx \mv}

  \inferrule
      {e_1 \redx \mv_1 \cdots e_n \redx \mv_n }
      {\{ \flab_1 = e_1; \ldots; \flab_n = e_n \} \redx \{ \flab_1 = \mv_1; \ldots; \flab_n = \mv_n \} }

  \inferrule
      {e \redx \{\ldots; \flab = \mv; \ldots\}}
      {e.\flab \redx \mv}

  \inferrule
      {e_1 \redx w_1 \\ e_2 \redx w_2}
      {e_1 \concat e_2 \redx w_1w_2}
\end{mathpar}

\begin{mathpar}
  \inferrule
      {e_1 \redx \be_1 \\ e_2 \redx \be_2 \\ e \redx \cid}
      {\mc{\prog}{(\eqs_1,\eqs_2)}{\eqon}{\elab{\assert{e_1 = e_2}}{e}} \redx
        (\prog;\elab{\assert{\be_1 = \be_2}}{\cid},(\eqs_1, \eqs_2 \wedge \eqtrans{\elab{\be_1}{\cid}} = \eqtrans{\elab{\be_2}{\cid}}),\eqon)}
      
  \inferrule
      {e_1 \redx \be_1 \\ e_2 \redx \be_2 \\ e \redx \cid}
      {\mc{\prog}{(\eqs_1,\eqs_2)}{\eqoff}{\elab{\assert{e_1 = e_2}}{e}} \redx
        (\prog;\elab{\assert{\be_1 = \be_2}}{\cid},(\eqs_1, \eqs_2,\eqoff)}
      
  \inferrule
      {e_1 \redx w \\ e_2 \redx \cid_1 \\ e_3 \redx \be \\ e_4 \redx \cid_2}
      {\mc{\prog}{(\eqs_1,\eqs_2)}{\eqon}{\msend{e_1}{e_2}{e_3}{e_4}} \redx
        (\prog;\msend{w}{\cid_1}{\be}{\cid_2},
        (\eqs_1 \wedge \mx{w}{\cid_1} = \eqtrans{\elab{\be}{\cid_2}}, \eqs_2),\eqon)}
      
  \inferrule
      {e_1 \redx w \\ e_2 \redx \cid_1 \\ e_3 \redx \be \\ e_4 \redx \cid_2}
      {\mc{\prog}{(\eqs_1,\eqs_2)}{\eqoff}{\msend{e_1}{e_2}{e_3}{e_4}} \redx
        (\prog;\msend{w}{\cid_1}{\be}{\cid_2},
        (\eqs_1, \eqs_1),\eqoff)}

  \inferrule
      {}
      {\mc{\prog}{(\eqs_1,\eqs_2)}{\eqon}{\pre{\eqs}} \redx (\prog,\eqs_1,\eqs_2 \wedge \eqs,\eqoff)}

  \inferrule
      {}
      {\mc{\prog}{(\eqs_1,\eqs_2)}{\eqoff}{\post{\eqs}} \redx (\prog,(\eqs_1 \wedge \eqs,\eqs_2),\eqon)}
            
  \inferrule
      {\mc{\prog_1}{(\eqs_{11},\eqs_{12})}{\eqflag_1}{\cmd_1} \redx
        (\prog_2,(\eqs_{21},\eqs_{22}),\eqflag_2) \\
       \mc{\prog_2}{(\eqs_{21},\eqs_{22})}{\eqflag_2}{\cmd_2} \redx
        (\prog_3,(\eqs_{31},\eqs_{32}),\eqflag_3)}
      {\mc{\prog_1}{(\eqs_{11},\eqs_{12})}{\eqflag_1}{\cmd_1;\cmd_2} \redx
        (\prog_3,(\eqs_{31},\eqs_{32}),\eqflag_3)}

  \inferrule
      {\codebase(f) = y_1,\ldots,y_n,\ \cmd \\ e_1 \redx \mv_1 \cdots e_n \redx \mv_n \\
        \mc{\prog_1}{(\eqs_{11},\eqs_{12})}{\eqflag_1}{\cmd[\mv_1/y_1,]\cdots[\mv_n/y_n]} \redx
        (\prog_2,(\eqs_{21},\eqs_{22}),\eqflag_2)}
      {\mc{\prog_1}{(\eqs_{11},\eqs_{12})}{\eqflag_1}{f(e_1,\ldots,e_n)} \redx
        (\prog_2,(\eqs_{21},\eqs_{22}),\eqflag_2)}
\end{mathpar}

\section{Examples}

\begin{verbatimtab}
    encodegmw(in, i1, i2) {
      m[in]@i2 := (s[in] xor r[in])@i2;
      m[in]@i1 := r[in]@i2
    }
    
    andtablegmw(b1, b2, r) {
      let r11 = r xor (b1 xor true) and (b2 xor true) in
      let r10 = r xor (b1 xor true) and (b2 xor false) in
      let r01 = r xor (b1 xor false) and (b2 xor true) in
      let r00 = r xor (bl xor false) and (b2 xor false) in
      { row1 = r11; row2 = r10; row3 = r01; row4 = r00 }
    }
    
    andgmw(z, x, y) {
      pre();
      let r = r[z] in
      let table = andtablegmw(m[x],m[y],r) in
      m[z]@2 := OT4(m[x],m[y],table,2,1);
      m[z]@1 := r@1;
      post(m[z]@1 xor m[z]@2 == (m[x]@1 xor m[x]@2) and (m[y]@1 xor m[y]@2))
    }
    
    xorgmw(z, x, y) {
      m[z]@1 := (m[x] xor m[y])@1; m[z]@2 := (m[x] xor m[y])@2;
    }
    
    decodegmw(z) {
      p["1"] := m[z]@1; p["2"] := m[z]@2;
      out@1 := (p["1"] xor p["2"])@1;
      out@2 := (p["1"] xor p["2"])@2
    }

    encodegmw("x",2,1);
    encodegmw("y",2,1);
    encodegmw("z",1,2);
    andgmw("g1","x","z");
    xorgmw("g2","g1","y");
    decodegmw("g2")
    pre();
    post(out@1 == (s["x"]@1 and s["z"]@2) xor s["y"]@1)
    
\end{verbatimtab}


\begin{verbatimtab}
  secopen(w1,w2,w3,i1,i2) {
      pre(m[w1++"m"]@i2 == m[w1++"k"]@i1 + (m["delta"]@i1 * m[w1++"s"]@i2 /\
          m[w1++"m"]@i2 == m[w1++"k"]@i1 + (m["delta"]@i1 * m[w1++"s"]@i2));
      let locsum =  macsum(macshare(w1), macshare(w2)) in
      m[w3++"s"]@i1 := (locsum.share)@i2;
      m[w3++"m"]@i1 := (locsum.mac)@i2;
      auth(m[w3++"s"],m[w3++"m"],mack(w1) + mack(w2),i1);
      m[w3]@i1 := (m[w3++"s"] + (locsum.share))@i1
  }

  
  _open(x,i1,i2){
    m[x++"exts"]@i1 := m[x++"s"]@i2;
    m[x++"extm"]@i1 := m[x++"m"]@i2;
    assert(m[x++"extm"] == m[x++"k"] + (m["delta"] * m[x++"exts"]));
    m[x]@i1 := (m[x++"exts"] + m[x++"s"])@i2
  }`
  
  _sum(z, x, y,i1,i2) {
      pre(m[x++"m"]@i2 == m[x++"k"]@i1 + (m["delta"]@i1 * m[x++"s"]@i2 /\
          m[y++"m"]@i2 == m[y++"k"]@i1 + (m["delta"]@i1 * m[y++"s"]@i2));
      m[z++"s"]@i2 := (m[x++"s"] + m[y++"s"])@i2;
      m[z++"m"]@i2 := (m[x++"m"] + m[y++"m"])@i2;
      m[z++"k"]@i1 := (m[x++"k"] + m[y++"k"])@i1;
      post(m[z++"m"]@i2 == m[z++"k"]@i1 + (m["delta"]@i1 * m[z++"s"]@i2)
  }

  sum(z,x,y) { _sum(z,x,y,1,2);_sum(z,x,y,2,1) }

  open(x) { _open(x,1,2); _open(x,2,1) } 


  sum("a","x","d");
  open("d");
  sum("b","y","e");
  open("e");
  let xys =
      macsum(macctimes(macshare("b"), m["d"]),
             macsum(macctimes(macshare("a"), m["e"]),
                    macshare("c")))
  let xyk = mack("b") * m["d"] + mack("a") * m["e"] + mack("c")
                    
  secopen("a","x","d",1,2);
    secopen("a","x","d",2,1);
    secopen("b","y","e",1,2);
    secopen("b","y","e",2,1);
    let xys =
      macsum(macctimes(macshare("b"), m["d"]),
             macsum(macctimes(macshare("a"), m["e"]),
                    macshare("c")))
    in
    let xyk = mack("b") * m["d"] + mack("d") * m["d"] + mack("c")               
    in
    secreveal(xys,xyk,"1",1,2);
    secreveal(maccsum(xys,m["d"] * m["e"]),
              xyk - m["d"] * m["e"],
              "2",2,1);
    out@1 := (p[1] + p[2])@1;
    out@2 := (p[1] + p[2])@2;
\end{verbatimtab}


