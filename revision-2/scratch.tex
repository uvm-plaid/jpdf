\newcommand{\sx}[2]{\elab{\secret{#1}}{#2}}
\newcommand{\mx}[2]{\elab{\mesg{#1}}{#2}} 
%\newcommand{\px}[2]{\elab{\rvl{#1}}{#2}} 
\newcommand{\rx}[2]{\elab{\flip{#1}}{#2}}
\newcommand{\ox}[2]{\elab{\out{#1}}{#2}}
\newcommand{\signals}{\leadsto}

\newcommand{\tj}[5]{#1,#2 \vdash_{#3} #4 : #5}
\newcommand{\cty}[2]{c(#1,#2)}
\newcommand{\setit}[1]{\{ #1 \}}
\newcommand{\ty}{T}
\newcommand{\eqs}{\mathit{E}}
\newcommand{\toeq}[1]{\lfloor #1 \rfloor}
\newcommand{\autheq}[1]{\phi_{\mathrm{auth}}(#1)}
\newcommand{\upgrade}[1]{\uparrow #1}

\renewcommand{\redx}{\Rightarrow}
\renewcommand{\redxs}{\redx}
\newcommand{\abort}{\bot}
\newcommand{\pre}[1]{\ttt{pre}(#1)}
\newcommand{\post}[1]{\ttt{post}(#1)}
\newcommand{\eqflag}{\mathit{sw}}
\newcommand{\eqon}{\ttt{on}}
\newcommand{\eqoff}{\ttt{off}}
\newcommand{\eqtrans}[1]{\lfloor #1 \rfloor}
\newcommand{\mc}[4]{(#1,#2,#3,#4)}
\newcommand{\cmd}{\instr}

$$
    \begin{array}{rcl@{\hspace{2mm}}r}
      \multicolumn{4}{l}{v \in \mathbb{F}_p,\ w \in \mathrm{String},\ \cid \in \mathrm{Clients} \subset  \mathbb{N} }\\[2mm] %, \bop \in \{ \eand, \eor, \exor \}} \\[2mm]
      \be &::=& \flip{w} \mid \secret{w} \mid \mesg{w} \mid \rvl{w} \mid & \textit{expressions}\\
      & & v \mid \be \fminus \be \mid \be \fplus \be \mid \be \ftimes \be \\[2mm]
      x &::=& \elab{\flip{w}}{\cid} \mid \elab{\secret{w}}{\cid} \mid \elab{\mesg{w}}{\cid} \mid \rvl{w} \mid \out{\cid} & \textit{variables} \\[2mm]
      %& &  \select{\be}{\be}{\be} \mid \ctxt{v}{k} \mid \key{w} \mid \sk{\be}(\be) \mid \pk{\be}{\be}(\be) \mid \pk{\be}{\be} \\[2mm]
      %& &  \select{\fp(\be)}{\be}{\be} \ctxt{v,\be}{k}  \mid \sk{\be}(\be) \mid \pk{\be}{\be}(\be) \mid \pk{\be}{\be} \\[2mm]
      \prog &::=& \eassign{\mesg{w}}{\cid}{\be}{\cid} \mid
      \reveal{w}{e}{\cid} \mid \pubout{\cid}{\be}{\cid} \mid \prog;\prog & \textit{protocols} 
    \end{array}
$$

\bigskip
    
 $$
  %\begin{array}{c@{\hspace{5mm}}c}
  \begin{array}{rcl}
    \lcod{\store, v}{\cid} &=& v\\
    \lcod{\store, \be_1 \fplus \be_2}{\cid} &=& \fcod{\lcod{\store, \be_1}{\cid} \fplus \lcod{\store, \be_2}{\cid}}\\ 
    \lcod{\store, \be_1 \fminus \be_2}{\cid} &=& \fcod{\lcod{\store, \be_1}{\cid} \fminus \lcod{\store, \be_2}{\cid}}\\ 
    \lcod{\store, \be_1 \ftimes \be_2}{\cid} &=& \fcod{\lcod{\store, \be_1}{\cid} \ftimes \lcod{\store, \be_2}{\cid}}\\
  %\end{array} 
  %\begin{array}{rcl}
    \lcod{\store, \flip{w}}{\cid} &=& \store(\elab{\flip{w}}{\cid})\\
    \lcod{\store, \secret{w}}{\cid} &=& \store(\elab{\secret{w}}{\cid})\\
    \lcod{\store, \mesg{w}}{\cid} &=& \store(\elab{\mesg{w}}{\cid})\\
    \lcod{\store, \rvl{w}}{\cid} &=& \store(\rvl{w})\\
    %\lcod{\store, f\,\be_1\,\cdots\, \be_n}{\cid} &=& \delta(f,\lcod{\store, \be_1}{\cid},\ldots,\lcod{\store,\be_n}{\cid})
  \end{array}
  %\end{array}
  $$

\bigskip

  \begin{mathpar}
    (\store, \xassign{x}{\be}{\cid}) \redx \extend{\store}{x}{\lcod{\store,\be}{\cid}}

    \inferrule
    {(\store_1,\be_1) \redx \store_2 \\ (\store_2,\be_2) \redx \store_3 }
    {(\store_1,\be_1;\be_2) \redx \store_3}
    %(\store, \eassign{\mesg{w}}{\cid_1}{\be}{\cid_2};\prog) \redx (\extend{\store}{\mesg{w}_{\cid_1}}{\lcod{\store,\be}{\cid_2}}, \prog)    
    %(\store, \reveal{w}{\be}{\cid};\prog) \redx (\extend{\store}{\rvl{w}}{\lcod{\store,\be}{\cid}}, \prog)   
    %(\store, \pubout{\cid}{\be}{\cid};\prog) \redx (\extend{\store}{\out{\cid}}{\lcod{\store,\be}{\cid}}, \prog)
  \end{mathpar}


$$
\begin{array}{rclr}
  (\store, \xassign{x}{\be}{\cid}) &\aredx&
  \extend{\store}{x}{\lcod{\store,\be}{\cid}} & \cid \in H\\
  (\store, \xassign{x}{\be}{\cid}) &\aredx&
  \extend{\store}{x}{\lcod{\arewrite(\store_C,\be)}{\cid}} & \cid \in C
\end{array}
$$

$$
\begin{array}{rcl@{\qquad}r}
  (\store,\elab{\assert{\be_1 = \be_2}}{\cid}) &\aredx& \store & \text{if\ }
  \lcod{\store,\be_1}{\cid} = \lcod{\store,\be_2}{\cid}  \text{\ or\ } \cid \in C\\
  (\store,\elab{\assert{\phi(\be)}}{\cid}) &\aredx& \abort & \text{if\ } \neg\phi(\store,\lcod{\store,\be}{\cid})
\end{array}
$$

\begin{mathpar}
  (\store, \xassign{x}{\be}{\cid}) \redx \extend{\store}{x}{\lcod{\store,\be}{\cid}}
  
  \inferrule
      {(\store_1,\be_1) \redx \abort}
      {(\store_1,\be_1;\be_2) \redx \abort}
\end{mathpar}

$$
\begin{array}{rcl}
  \multicolumn{3}{l}{\flab \in \mathrm{Field},\   y \in \mathrm{EVar}, \  f \in \mathrm{FName}}\\[1mm]
  %x &\in& \mathrm{EVar}\\
  %f &\in& \mathrm{FName}\\[2mm]
  e &::=& \mv \mid \flip{e} \mid \secret{e} \mid \mesg{e} \mid \rvl{e} \mid e \bop e \mid
  \elet{y}{e}{e} \mid \\
  & & f(e,\ldots,e) \mid \{ \flab = e; \ldots; \flab = e \} \mid e.\flab \\
  %  & \textit{expressions}\\
  \cmd &::=& \msend{e}{e}{e}{e} \mid \reveal{e}{e}{e} \mid \pubout{e}{e}{e} \mid
      \elab{\assert{e = e}}{e} \mid \\
  & & f(e,\ldots,e) \mid  \cmd;\cmd \mid \pre{\eqs} \mid \post{\eqs} \\[1mm]
  \bop &::=& \fplus \mid \fminus \mid \ftimes \mid \concat  \\[1mm]% \textit{operators}\\[2mm]
  \mv &::=& w \mid \cid \mid \be \mid \{ \flab = \mv;\ldots;\flab = \mv \} 
  \\ % \mid \ttt{()} \\[1mm] %& \textit{values}\\[2mm]
  \mathit{fn} &::=& f(y,\ldots,y) \{ e \} \mid  f(y,\ldots,y) \{ \cmd \} \\[1mm]%& \textit{functions}
  \phi &::=& \elab{\flip{e}}{e} \mid \elab{\secret{e}}{e} \mid \elab{\mesg{e}}{e} \mid \rvl{e} \mid \out{e} \mid \phi \fplus \phi \mid \phi \fminus \phi \mid \phi \ftimes \phi \\
  \eqs &::=& \phi = \phi \mid \eqs \wedge \eqs 
\end{array}
$$

\begin{mathpar}
  \inferrule
      {e[\mv/y] \redx \mv'}
      {\elet{y}{\mv}{e} \redx \mv'}

  \inferrule
      {\codebase(f) = y_1,\ldots,y_n,\ e \\ e_1 \redx \mv_1 \cdots e_n \redx \mv_n \\
        e[\mv_1/y_1]\cdots[\mv_n/y_n] \redx \mv}
      {f(e_1,\ldots,e_n) \redx \mv}

  \inferrule
      {e_1 \redx \mv_1 \cdots e_n \redx \mv_n }
      {\{ \flab_1 = e_1; \ldots; \flab_n = e_n \} \redx \{ \flab_1 = \mv_1; \ldots; \flab_n = \mv_n \} }

  \inferrule
      {e \redx \{\ldots; \flab = \mv; \ldots\}}
      {e.\flab \redx \mv}

  \inferrule
      {e_1 \redx w_1 \\ e_2 \redx w_2}
      {e_1 \concat e_2 \redx w_1w_2}
\end{mathpar}

\begin{mathpar}
  \inferrule
      {}
      {\mc{\prog}{(\eqs_1,\eqs_2)}{\eqflag}{}}
\end{mathpar}
