\section{The $\metaprot$ Metalanguage}
\label{section-metalang}

Practical MPC computations protocols are
typically composed of compositional units. Examples include GMW circuits
and Yao's Garbled Circuits (YGC), that are composed of so-called
garbled gates. Languages such as Fairplay \cite{269581} provide gates as
units of abstraction that are ``wired'' together by the programmer to
generate a complete circuit.

The $\fedprot$ language is low-level and does not include abstractions
for defining composable elements. So in this Section we introduce the
$\metaprot$ language which that includes structured data and function
definitions for defining composable protocol elements at a higher
level of abstraction.  The $\metaprot$ language is a
\emph{metalanguage}, where $\fedprot$ protocols are the residuum of
computation. In addition to these declarative benefits of $\metaprot$,
component definitions support compositional verification of larger
protocols as we will discuss with examples in Sections
\ref{section-example-gmw} and \ref{section-example-bdoz}.

\metaprotfig

\subsection{Syntax}

The syntax of $\metaprot$ is defined in Figure
\ref{fig-metaprot}.  It includes a syntax of function
definitions and records, and values $\mv$ include client ids $\cid$, identifier
strings $w$, expressions $\be$ in field $\mathbb{F}_p$, and the unit value $()$.
Expression forms allow dynamic construction of field expressions and $\minifed$ commands.
The construction of a command has the side-effect of adding to the residual
$\minifed$ protocol. Formally, we consider a complete metaprogram to include both a
codebase and a ``main'' program that uses the codebase. 
\begin{definition}
A \emph{codebase} $\codebase$ is a list of function 
declarations. We write $ \codebase(f) = y_1,\ldots,y_n,\ e$
iff $f(y_1,\ldots,y_n) \{ e \} \in \codebase$.
A \emph{metaprogram}, aka \emph{metaprotocol}  is a pair of a 
codebase and expression $\codebase, e$. We may omit
$\codebase$ if it is clear from context.  
\end{definition}

When we consider larger examples, our
focus will be on developing a codebase that can be used to define
arbitrary circuits, i.e., complete and concrete protocols. Since
strings and identifiers can be constructed manually, and expressions
can occur inside assignments and field expression forms, function
definitions can generalize over $\fedprot$-level patterns to obtain
composable program units.
\subsection{Semantics}

We define a small-step evaluation aka reduction relation $\redx$ in
Figure \ref{fig-metaprot}.  We write $\redxs$ to denote the
reflexive, transitive closure of $\redx$. Reduction is defined on
\emph{configurations} which are pairs of the form $\config{\prog}{e}$,
where $\prog$ is the $\minifed$ program accumulated during evaluation.
In this definition we write $e[\mv/y]$ to denote the substitution of $\mv$
for free occurrences of $y$ in $e$. The rules are mostly standard,
except when a concrete $\minifed$ assignment is encountered it is added
to the end of $\prog$.

The rules rely on a definition of \emph{evaluation contexts} $E$
allowing computation within a larger program context, where $E[e]$
denotes an expression with $e$ in the hole $[]$ of $E$. The syntax
of $E$ imposes a left-to-right order of evaluation of subexpressions
for all forms.
