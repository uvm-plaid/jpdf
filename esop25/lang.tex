\section{$\minicat$ Syntax and Operational Semantics}
\label{section-lang}

The $\minifed$ language establishes a basic model of synchronous
protocols between a federation of \emph{clients} exchanging values in
the binary field. We identify clients by natural numbers and federations- finite sets of
clients- are always given statically.  Our threat model assumes a
partition of the federation into \emph{honest} $H$ and \emph{corrupt}
$C$ subsets. We model probabilistic programming via a \emph{random
tape} semantics. That is, we will assume that programs can make
reference to values chosen from a uniform random distributions defined
in the initial program memory.  Programs aka protocols execute
deterministically given the random tape.

\subsection{Syntax}

\minicatsyntaxfig

The syntax of $\minifed$, defined in Figure \ref{fig-minicat-syntax},
includes values $v$ and standard operations of addition, subtraction,
and multiplication in a finite field $\mathbb{F}_p$ where $p$ is some
prime.  Protocols are given input secret values $\secret{w}$ as well
as random samples $\flip{w}$ on the input tape, implemented using a
\emph{memory} as described below (Section
\ref{section-lang-semantics}) where $w$ is a distinguishing 
identifier string. Protocols are sequences of assignment commands of three
different forms:
\begin{itemize}
\item $\eassign{\mesg{w}}{\cid_2}{\be}{\cid_1}$: This
  is a \emph{message send} where expression $\be$ is computed
  by client $\cid_1$ and sent to client $\cid_2$ as message
  $\mesg{w}$.
\item $\reveal{w}{\be}{\cid}$: This
  is a \emph{public reveal} where expression $\be$ is computed
  by client $\cid$ and broadcast to the federation, typically
  to communicate intermediate results for use in final output
  computations.
\item $\pubout{\cid}{\be}{\cid}$: This
  is an \emph{output} where expression $\be$ is computed
  by client $\cid$ and reported as its output. As a
  sanity condition we disallow commands
  $\pubout{\cid_1}{\be}{\cid_2}$ where $\cid_1\ne\cid_2$.
\end{itemize}
The distinction between
messages and broadcast public reveal is consistent with previous
formulations, e.g., \cite{6266151}. To identify and distinguish
between collections of variables in protocols we introduce the
following notation.
\begin{definition}
We let $x$ range over \emph{variables} which are identifiers where
client ownership is specified- e.g.,
$\elab{\mesg{\mathit{foo}}}{\cid}$ is a message $\mathit{foo}$ that
was sent to $\cid$. We let $X$ range over sets of variables, and more
specifically, $S$ ranges over sets of secret variables
$\elab{\secret{w}}{\cid}$, $R$ ranges over sets of random variables
$\elab{\flip{w}}{\cid}$, $M$ ranges over sets of message variables
$\elab{\mesg{w}}{\cid}$, $P$ ranges over sets of public variables
$\rvl{w}$, and $O$ ranges over sets of output variables $\out{\cid}$.
Given a program $\prog$, we write $\iov(\prog)$ to denote the
particular set $S \cup M \cup P \cup O$ of variables in $\prog$ and
$\secrets(\prog)$ to denote $S$, and we write $\flips(\prog)$ to
denote the particular set $R$ of random samplings in $\prog$. We write
$\vars(\prog)$ to denote $\iov(\prog) \cup \flips(\prog)$. For any set
of variables $X$ and clients $I$, we write $X_I$ to denote the subset
of $X$ owned by any client $\cid \in I$, in particular we write $X_H$
and $X_C$ to denote the subsets belonging to honest and corrupt
parties, respectively.
\end{definition}
In all cases we disallow overwriting of variables, since assignment is
intended to represent distinct message sends. An example of a
$\minicat$ protocol, for additive secret sharing, is discussed in
Section \ref{section-smt-toeq}.

\subsection{Semantics}
\label{section-lang-semantics}

\minicatredxfig

\emph{Memories} are fundamental to the semantics of $\fedcat$ and
provide random tape and secret inputs to protocols, and also record
message sends, public broadcast, and client outputs.
\begin{definition}
Memories $\store$ are finite (partial) mappings from variables $x$ to
values $v \in \mathbb{F}_p$.  The \emph{domain} of a memory is written
$\dom(\store)$ and is the finite set of variables on which the memory
is defined.  We write $\store\{ x \mapsto v\}$ for
$x\not\in\dom(\store)$ to denote the memory $\store'$ such that
$\store'(x) = v$ and otherwise $\store'(y) = \store(y)$ for all $y \in
\dom(\store)$. We write $\store \subseteq \store'$ iff $\dom(\store)
\subseteq \dom(\store')$ and $\store(x) = \store'(x)$ for all $x \in
\dom(\store)$. Given any $\store$ and $\store'$ with $\store(x) =
\store'(x)$ for all $x \in \dom(\store) \cap \dom(\store')$, we write
$\store \uplus \store'$ to denote the memory with domain $X =
\dom(\store) \cup \dom(\store')$ such that:
$$
\forall x \in X .
(\store \uplus \store')(x) =
\begin{cases} \store(x) \text{\ if\ } x\in\dom(\store) \\ \store'(x) \text{\ otherwise\ }\end{cases} 
$$
%We write $\store \subseteq \store'$ iff $\store \uplus \store' = \store$.
\end{definition}
In our subsequent presentation we will often want to consider arbitrary
memories that range over particular variables and to restrict
memories to particular subsets of their domain:
\begin{definition}
  Given a set of variables $X$ and memory $\store$, we write
  $\store_X$ to denote the memory with $\dom(\store_X) = X$ and
  $\store_X(x) = \store(x)$ for all $x \in X$. We define $\mems(X)$ as
  the set of all memories with domain $X$:
  $$
  \mems(X) \defeq \{ \store \mid \dom(\store) = X \}
  $$
\end{definition}
So for example, the set of all random tapes for
for a protocol $\prog$ is $\mems(\flips(\prog))$, and the memory $\store_{\secrets(\prog)}$
is $\store$ restricted to the secrets in $\prog$.
%We let $\stores$ range
%over sets of memories with the same domain, and abusing notation
%we write $\dom(\stores)$ to denote the common domain,
%and $\stores_X \defeq \{ \store_X | \store \in \stores \}$.

Given a variable-free expression $\be$, we write $\cod{\be}$ to denote
the standard interpretation of $\be$ in the arithmetic field
$\mathbb{F}_{p}$. With the introduction of variables to expressions,
we need to interpret variables with respect to a specific memory, and
all variables used in an expression must belong to a specified client.
Thus, we denote interpretation of expressions $\be$ computed on a
client $\cid$ as $\lcod{\store,\be}{\cid}$. This interpretation is
defined in Figure \ref{fig-minicat-redx}, along with the big-step
reduction relation $\redx$ to evaluate commands. Reduction is a mapping
from \emph{configurations} $(\store,\prog)$ to final stores
where all three command forms- message send, broadcast, and
output- are implemented as updates to the memory $\store$. 


\paragraph{Oblivious Transfer} A passive secure oblivious transfer (OT) protocol
based on previous work \cite{barthe2019probabilistic} can be defined
in $\minicat$, albeit assuming some shared randomness. Alternatively,
a simple passive secure OT can be defined with the addition of public
key cryptography as a primitive. But given the diversity of approaches
to OT, we instead assume that OT is abstract with respect to its
implementation, and any use is of the form
$\mx{w}{\cid_1} := \elab{\OT{\be_1}{\cid_1}{\be_2}{\be_3}}{\cid_2}$ --
given a \emph{choice bit} $\be_1$ provided by a receiver $\cid_1$, the
sender $\cid_2$ sends either $\be_2$
or $\be_3$.  Critically, $\cid_2$ learns nothing about $\be_1$ and
$\cid_1$ learns nothing about the unselected value. We return to
the formal interpretation of OT in Section \ref{section-smt}.

\subsection{$\minicat$ Adversarial Semantics}
  
\minicataredxfig

In the malicious model we assume that corrupt clients are in
thrall to an adversary $\adversary$ who does not necessarily follow
the rules of the protocol.  We model this by positing a $\arewrite$
function which is given a corrupt memory $\store_C$ and expression
$\be$, and returns a rewritten expression that can be interpreted to
yield a corrupt input. We define the evaluation relation that
incorporates the adversary in Figure \ref{fig-minicat-aredx}.

A key technical distinction of the malicious setting is that it
typically incorporates ``abort''. Honest parties implement strategies
to detect rule-breaking-- aka \emph{cheating}-- by using, e.g.,
message authentication codes with semi-homomorphic properties as in
BDOZ/SPDZ \cite{10.1007/978-3-030-68869-1_3}. If cheating is detected,
the protocol is aborted. To model this, we extend $\minifed$ with an
\ttt{assert} command:
$$
    \begin{array}{rcl@{\hspace{2mm}}r}
      \prog &::=& \cdots \mid \elab{\assert{\be = \be}}{\cid}
    \end{array}
$$
As we will discuss, this form of asserting equality between expressions
is sufficient to capture MAC checking and support integrity in protocols.
The big-step adversarial semantics relation $\aredx$ is then defined
in Figure \ref{fig-minicat-aredx} as a mapping from configurations to
$\stores$ or $\bot$.

\subsubsection{Protocol Runs}

Our semantics require that random tapes contain values for all program
values $\elab{\flip{w}}{\cid}$ sampled from a uniform distribution
over $\mathbb{F}_p$. Input memories also contain input secret values
and possibly other initial view elements as a result of
pre-processing, e.g., Beaver triples for efficient multiplication,
and/or MACed share distributions as in BDOZ/SPDZ
\cite{evans2018pragmatic,10.1007/978-3-030-68869-1_3}. We define
$\runs(\prog)$ as the set of final memories resulting from execution
of $\prog$ given any initial memory, possibly augmented with a
preprocessed memory (a memory that satisfies a predicate $\preproc$),
and $\aruns(\prog)$ is the analog in the malicious setting. Since we
constrain programs to not overwrite any variable, we are assured that
final memories contain both a complete record of all initial secrets
as well as views resulting from communicated information. This
formulation will be fundamental to our consideration of probability
distributions and hyperproperties of protocols.
%In this
%setting, given a program $\prog$ with $\iov(\prog) = S \cup V$ and
%$\flips(\prog) = F$ we will consider all $\store \in \mems(S \cup V
%\cup F)$ such that $ \config{\store_{S \cup F}}{\prog} \redxs
%\config{\store_}{\varnothing} $ to be equally probable, establishing
%the basic distribution of the program. %From this, we can immediately
%derive the marginal distribution of $S \cup V$ to reason about
%dependencies between secrets and views.
\begin{definition}
  \label{def-runs}
  Given $\prog$ with $\secrets(\prog) = S$ and $\flips(\prog) = R$ and
  predicate $\preproc$ on memories, define:
  $$
  \begin{array}{c}
    \runs(\prog) \defeq 
    \{ \store \mid \exists \store_1 \in \mems(S \cup R) . 
    \exists \store_2 . \preproc(\store_2) \wedge
    %(\dom(\store) = \iov(\prog) \cup R) \wedge
    (\store_1 \uplus \store_2,\prog) \redxs \store \}
  \end{array}
  $$
  And similarly for any $\adversary$ define:
  $$
  \begin{array}{c}
    \aruns(\prog) \defeq 
    \{ \store \mid \exists \store_1 \in \mems(S \cup R) . 
    \exists \store_2 . \preproc(\store_2) \wedge
    %(\dom(\store) = \iov(\prog) \cup R) \wedge
    (\store_1 \uplus \store_2,\prog) \aredxs \store \}
  \end{array}
  $$
\end{definition}

