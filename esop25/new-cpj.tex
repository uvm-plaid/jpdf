\renewcommand{\unity}[2]{U(#1, #2)}
\renewcommand{\ty}{T}
\newcommand{\eset}{D}
\newcommand{\dty}[2]{\langle #1, #2 \rangle}
\newcommand{\sharet}[1]{\mathrm{shares}(#1)}
\newcommand{\cvec}[1]{[#1]}
\newcommand{\secopent}{\mathrm{secopen}}
\newcommand{\declasst}[1]{\mathrm{declass}(#1)}
\renewcommand{\view}{\mathit{view}}
\newcommand{\revealed}{\mathit{revealed}}
\newcommand{\sty}{\varsigma}
\newcommand{\noise}{\mathrm{noise}}
\newcommand{\owners}{\mathrm{owners}}
\newcommand{\federation}{I_{\mathit{fed}}}
\newcommand{\tjoin}{\bowtie}

\section{$\minifed$ Declassification Typing}

$$
\begin{array}{rcl@{\hspace{4mm}}r}
\eset &\in& 2^{\phi} & \gdesc{Dependency Approximations}\\ 
\ty &::=& \dty{R}{\eset} & \gdesc{Probabilistic Types} \\
\Gamma &::=& \varnothing \mid \Gamma; x : \ty & \gdesc{Type Environments}
\end{array}
$$

The types of expressions $\elab{\be}{\cid}$ reflect their dependencies which
may include variables from clients beside $\cid$, so typings are in terms
of interpretations $\toeq{\elab{\be}{\cid}}$. Types are of the form $\dty{R}{\eset}$,
where $\eset$ is a set of expressions $\phi$, and intuitively the typing
$\toeq{\elab{\be}{\cid}} : \dty{R}{\eset}$ means that the marginal distribution
of its variable dependencies is equivalent to the (uniform) marginal distribution of
variables in $R$. If $R$ is empty this means that the distribution is dependent
on the marginal distribution of variables in $\eset$. So for example:
\begin{mathpar}
  \sx{w}{\cid} : \dty{\varnothing}{\setit{\sx{w}{\cid}}}

  \rx{w}{\cid} : \dty{\setit{\rx{w}{\cid}}}{\varnothing}
  
  \sx{w}{\cid} - \rx{w}{\cid} : \dty{\setit{\rx{w}{\cid}}}{\setit{\sx{w}{\cid}}}
\end{mathpar}
The typing rules are an approximation of dependencies, so wherever dependencies
may be revealed, the system (conservatively) reflects this. So for example:
\begin{mathpar}
  \sx{w}{\cid} \ftimes \rx{w}{\cid} : \dty{\varnothing}{\setit{\rx{w}{\cid}, \sx{w}{\cid}}}
\end{mathpar}
\subsection{Primitive Typing Rules}

The primitive typing rules are precise and expressive and reflect no particular
implementation strategies, they just track dependencies as reflected in types.
The most important rules are for binary operators, where dependencies merge
in interesting ways especially with summation. In particular, if two
randomized values are summed, they remain randomized modulo disjointness
of randomness. Otherwise, overlapping randomness is merged into the dependencies.
For example:
\begin{eqnarray*}
& (\sx{w}{\cid_1} - \rx{w}{\cid_1}) + (\sx{w}{\cid_2} - \rx{w}{\cid_2}) : \\
& \dty{\setit{\rx{w}{\cid_1},\rx{w}{\cid_2}}}{\setit{\sx{w}{\cid_2},\sx{w}{\cid_1}}}
\end{eqnarray*}
but:
$$
(\sx{w}{\cid_1} - \rx{w}{\cid_1}) + \rx{w}{\cid_1} :
\dty{\varnothing}{\setit{\rx{w}{\cid_1},\sx{w}{\cid_1}}}
$$
To join types appropriately, we make the following definition.
\begin{definition}[Type Join]
  Given $\dty{R_1}{\eset_1}$ and $\dty{R_2}{\eset_2}$, define
  $\dty{R_1}{\eset_1} \tjoin \dty{R_2}{\eset_2} \defeq \dty{R}{\eset}$ where
  $R = (R_1 - (R_2 \cup \eset_2) \cup (R_2 - (R_1 \cup \eset_1))$ 
  and $\eset = (R_1 \cup R_2 \cup \eset_1 \cup \eset_2) - R $.
  %\begin{eqnarray*}
  %  R &\defeq& (R_1 - (R_2 \cup \eset_2) \cup (R_2 - (R_1 \cup \eset_1))\\
  %  \eset &\defeq& (R_1 \cup R_2 \cup \eset_1 \cup \eset_2) - R 
  %\end{eqnarray*}
\end{definition}
Then the rules for typing are as follows. The most interesting are the rules
for summation that allow joining types and possibly retaining uniformity,
and the rule for binary operations (including multiplication) that
revert to the approximation of distributions of dependent expressions.

\begin{mathpar}
  \inferrule[Mesg]
            {}
            {\Gamma, \eqs \vdash \mx{w}{\cid} : \Gamma(\mx{w}{\cid})}
            
  \inferrule[Rand]
            {}
            {\Gamma, \eqs \vdash \rx{w}{\cid} : \dty{\setit{\rx{w}{\cid}}}{\varnothing}}
            
  \inferrule[Ideal]
            {}
            {\Gamma, \eqs \vdash f(\sx{w_1}{\cid_1},\ldots,\sx{w_n}{\cid_n}) :
              \dty{\varnothing}{\setit{f(\sx{w_1}{\cid_1},\ldots,\sx{w_n}{\cid_n})}}}

  \inferrule[Invert]
            {\Gamma, \eqs \vdash \phi : \ty}
            {\Gamma, \eqs \vdash -\phi : \ty}

  \inferrule[Join]
      {\Gamma, \eqs \vdash \phi_1 : \dty{R_1}{\eset_1}\\
       \Gamma, \eqs \vdash \phi_2 : \dty{R_2}{\eset_2}}%\\
        %R \defeq (R_1 - (R_2 \cup \eset_2) \cup (R_2 - (R_1 \cup \eset_1)) \\ 
        %\eset \defeq (R_1 \cup R_2 \cup \eset_1 \cup \eset_2) - R }
      {\Gamma, \eqs \vdash \phi_1 \fplus \phi_2 : \dty{R_1}{\eset_1} \tjoin \dty{R_2}{\eset_2}}

  \inferrule[Smear]
      {\Gamma, \eqs \vdash \phi_1 : \dty{R_1}{\eset_1}\\
       \Gamma, \eqs \vdash \phi_2 : \dty{R_2}{\eset_2}}
      {\Gamma, \eqs \vdash \phi_1 \bop \phi_2 : \dty{\varnothing}{R_1 \cup R_2 \cup \eset_1 \cup \eset_2}}

  \inferrule[Eq]
            {\eqs \models \phi \eop \phi' \\ \Gamma, \eqs \vdash \phi' : \ty}
            {\Gamma, \eqs \vdash \phi : \ty}          
\end{mathpar}

\begin{mathpar}
  \inferrule[Mesg]
      {\Gamma, \eqs \vdash \toeq{\elab{e}{\cid}} : \ty}
      {\Gamma, \eqs \vdash \xassign{x}{e}{\cid} : (\Gamma;x:\ty)}

  \inferrule[Seq]
      {\Gamma_1,\eqs \vdash \prog_1 : \Gamma_2 \\
        \Gamma_2,\eqs \vdash \prog_2 : \Gamma_3}
      {\Gamma_1, \eqs \vdash \prog_1;\prog_2 : \Gamma_3}
\end{mathpar}
We note that typing tracks uniformity of marginal distributions of independent messages,
but not their joint distributions. In order to capture this, we define a notion of
dependency closures in views:  
\begin{definition}[Views]
  Given $\prog, \eqs, \Gamma$ with $\toeq{\prog} \models \eqs$ and $\varnothing, \eqs \vdash \prog : \Gamma$.
  Define:
  $$
  \view(\Gamma,\eqs,X) \defeq \bigcup_{X' \in (\pow(X) - \varnothing)} \ty
     \qquad \text{where}\ \Gamma,\eqs \vdash \Sigma_{x \in X'} x : \ty
  $$
\end{definition}
The goal of the analysis is to show that views contain only noise and the
result of the ideal functionality. To this end we define a notion of revelation, where
functions of honest secrets are available in the clear in adversarial
views:
\begin{definition}[Revelation]
  $$
  \begin{array}{ll}
    \revealed(\ty,H) \defeq & \setit{f(\sx{w_1}{\cid_1},\ldots,\sx{w_n}{\cid_n}) \mid  \\
      & \qquad \dty{\varnothing}{\eset} \in \ty\ \wedge\\
      & \qquad f(\sx{w_1}{\cid_1},\ldots,\sx{w_n}{\cid_n}) \in \eset\ \wedge \\
      & \qquad \exists 1 \le i \le n . \cid_i \in H} 
  \end{array}
  $$
\end{definition}
Now we can state our security typing correctness result, which says that the only
information revealed in adversarial views (besides noise) is the ideal functionality. 
\begin{theorem}[Security Typing]
  Given $\prog, \eqs, \Gamma$ with $\toeq{\prog} \models \eqs$ and $\varnothing, \eqs \vdash \prog : \Gamma$.
  Then $\prog$ satisfies {NIMO} for ideal functionality $\idealf$ if for all $H,C$, we have:
  $$
  \revealed(\view(\Gamma,\eqs,\vars(\prog)_C),H) = \setit{\idealf}
  $$
\end{theorem}

\subsection{Strategy Typings}

The type system presented above is in a logical form and is not syntax-directed.
Also, the definition of $\view$ involves a powerset construction with exponential
complexity in the size of the program.

As a step towards an algorithm, we can define specific strategies and analysis
based on our type system that both reduces overall complexity and leverages SMT
solving. This will form the foundation of a decidable and scalable type analysis
for $\metaprot$.  We define \emph{vectors} of messages $\nu$ that participate in strategies,
e.g., the collective shares of secret values.
\begin{definition}
  A \emph{message vector} $\nu$ is either a vector of distinct
  broadcast messages $\cvec{\px{w_1},\ldots,\px{w_n}}$, or of unicast
  messages $\cvec{\mx{w_1}{\cid_1},\ldots,\mx{w_n}{\cid_n}}$ where
  each $\cid_i$ for $i \in \setit{\cid_1,\ldots,\cid_n}$ is distinct.  
\end{definition}

\emph{Strategy typings} and environments are then defined via the following syntax.
$$
\begin{array}{rcl@{\hspace{4mm}}r}
\sty &::=& \sharet{\phi} \mid \declasst{\phi} \mid \secopent & \gdesc{Strategy Types} \\[1mm]
\Delta &::=& \varnothing \mid \Delta; \nu : \sty & \gdesc{Strategy Type Environments}
\end{array}
$$

$$
\inferrule
    {\Gamma, \eqs \vdash \phi : \dty{R}{\eset} \\ R - R_I \neq \varnothing}
    {\Gamma, \eqs \vdash \phi : \noise(R,I)}
$$

\begin{mathpar}
  \inferrule[Sharing]
      {
        \eqs \models \mx{w_1}{\cid_1} \fplus \cdots \fplus \mx{w_n}{\cid_n} \eop \phi \\
        \\
        \exists R_2\ .\ (\forall M \in \pow(\setit{\mx{w_1}{\cid_1},\ldots,\mx{w_n}{\cid_n}})\ .\ 0 < |M| < n \implies \\
          \qquad \Gamma, \eqs \vdash \Sigma_{x \in M} x : \noise(R,\owners(M)) \ \wedge \ R \subseteq R_2)
      }   
      {\Gamma,\eqs \vdash R_1,\Delta \leadsto R_1;R_2, \Delta; \cvec{\mx{w_1}{\cid_1},\ldots,\mx{w_n}{\cid_n}} : \sharet{\phi}}

  \inferrule[Homomorphic Addition]
      {
       \Delta(\cvec{\mx{w^1_1}{\cid_1},\ldots,\mx{w^1_n}{\cid_n}}) = \sharet{\phi_1} \\
       \Delta(\cvec{\mx{w^2_1}{\cid_1},\ldots,\mx{w^2_n}{\cid_n}}) = \sharet{\phi_2} \\
       \eqs \models \mx{w_1}{\cid_1} \eop \mx{w^1_1}{\cid_1} \fplus \mx{w^2_1}{\cid_1} 
       \ \ldots\ 
       \eqs \models \mx{w_n}{\cid_n} \eop \mx{w^1_n}{\cid_n} \fplus \mx{w^2_n}{\cid_n} 
      }
      {
        \Gamma, \eqs \vdash R,\Delta \leadsto R,\Delta;\cvec{\mx{w_1}{\cid_1},\ldots,\mx{w_n}{\cid_n}} : \sharet{\phi_1 \fplus \phi_2}
      }
      
 \inferrule[Secure Opening]
      {
        \Delta(\cvec{\mx{w_1}{\cid_1},\ldots,\mx{w_n}{\cid_n}}) = \sharet{\phi} \\
        \eqs \models \px{w_1} \eop \mx{w_1}{\cid_1}
        \ \ldots\
        \eqs \models \px{w_n} \eop \mx{w_n}{\cid_n}\\
        \Gamma,\eqs \vdash \phi : \noise(R_2, \federation)
      }
      {
        \Gamma, \eqs \vdash R_1;\Delta \leadsto R_1;R_2, \Delta; \cvec{\px{w_1},\ldots,\px{w_n}} : \secopent
      }
      
 \inferrule[Declassification]
      {
        \Delta(\cvec{\mx{w_1}{\cid_1},\ldots,\mx{w_n}{\cid_n}}) =  \sharet{\phi} \\
        \eqs \models \px{w_1} \eop \mx{w_1}{\cid_1}
        \ \ldots\
        \eqs \models \px{w_n} \eop \mx{w_n}{\cid_n}
      }
      {
        \Gamma, \eqs \vdash R, \Delta \leadsto R, \Delta;\cvec{\px{w_1},\ldots,\px{w_n}} : \declasst{\phi}
      }
\end{mathpar}

\begin{theorem}
  Given $\prog, \eqs, \Gamma$ with $\toeq{\prog} \models \eqs$ and $\varnothing, \eqs \vdash \prog : \Gamma$.
  Suppose also $\Gamma,\eqs \vdash \varnothing,\varnothing \Rightarrow \Delta$, and define:
  $$\eset \defeq \{ \phi \mid \exists \nu \in \dom(\Delta) . \Delta(\nu) \mapsto \declasst{\phi} \}$$
  Let $\iov(\prog) = S \cup M \cup P$. 
  Then for any $H,C$, if for all $x \in (M \cup P)_C$ there exists
  $\nu \in \dom(\Delta)$ with $x \in \nu$, we have:
  $$
  D = \revealed(\view(\Gamma,\eqs,\vars(\prog)_C),H)
  $$
\end{theorem}

\begin{lemma}
  Given $\prog, \eqs, \Gamma$ with $\toeq{\prog} \models \eqs$ and $\varnothing, \eqs \vdash \prog : \Gamma$.
  Suppose also $\Gamma,\eqs \vdash \varnothing,\varnothing \Rightarrow \Delta$, and define:
  $$\eset = \{ \phi \mid \exists \nu \in \dom(\Delta) . \Delta(\nu) \mapsto \declasst{\phi} \}$$
  Then for any $H,C$, with:
  $$
  X \defeq \{ x \mid x \in \vars(\prog)_C \wedge \neg \exists \nu \in \dom(\Delta) . x \in \nu \}
  $$
  we have:
  $$
  \eset \cup  \revealed(\view(\Gamma,\eqs,X),H) = \revealed(\view(\Gamma,\eqs,\vars(\prog)_C),H)
  $$
\end{lemma}

