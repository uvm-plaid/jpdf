\section{$\minicat$ Constraint Verification}
\label{section-smt}

In previous work it was shown that the semantics of $\minicat$ in
$\mathbb{F}_2$ can be implemented with Datalog
\cite{skalka-near-ppdp24}, which is a form of constraint
programming. In this paper we extend this idea to arbitrary prime
fields by using a more general form of SMT constraint programming,
specifically Satisfiability Modulo Finite Fields \cite{SMFF}. As we
will show in subsequent sections, this interpretation supports
correctness guarantees, and also static type analyses for enforcing
confidentiality and integrity properties.

\subsection{Constraint Satisfiability Modulo Finite Fields}

We introduce the following syntax of SMT-style constraints
for $\minicat$:
$$
\begin{array}{rcl}
  \phi ::= x \mid \phi \fplus \phi \mid \phi \fminus \phi \mid \phi \ftimes \phi  \qquad
  \eqs ::= \phi \eop \phi \mid \eqs \wedge \eqs \mid \neg\eqs
\end{array}
$$
%
%In Figure \ref{fig-eqs} we define a syntax of SMT-style constraints
%$\eqs$, where
Note that constraint terms $\phi$ are similar to expressions $\be$
except that $\phi$ can ``cross party lines''. This turns out to be very useful
to express correctness properties-- for example, in the
Goldreich-Micali-Wigderson (GMW) protocol wire values in circuits are
represented by reconstructive shares \cite{evans2018pragmatic}.  If by
convention shares of values $n$ are represented by $\mesg{n}$ on
clients, then assuming two clients $\setit{1,2}$ the reconstructed
value can be expressed as $\mx{n}{1} + \mx{n}{2}$.  So, while summing
values across clients is disallowed in $\minicat$ protocols, we can
express properties of shared values via constraints.

%\begin{fpfig}[t]{Syntax of constraints}{fig-eqs}
%$$
%\begin{array}{rcl}
%  \phi ::= x \mid \phi \fplus \phi \mid \phi \fminus \phi \mid \phi \ftimes \phi  \qquad
%  \eqs ::= \phi \eop \phi \mid \eqs \wedge \eqs \mid \neg\eqs
%\end{array}
%$$
%
%\begin{mathpar}
%  \store(\phi_1 \fplus \phi_2) = \cod{\store(\phi_1) \fplus \store(\phi_2)}
%  
%  \store(\phi_1 \ftimes \phi_2) = \cod{\store(\phi_1) \ftimes \store(\phi_2)}
%  
%  \store(\phi_1 \fminus \phi_2) = \cod{\store(\phi_1) \fminus \store(\phi_2)}
%\end{mathpar}
%\end{fpfig}

This language of constraints has an obvious direct interpretation in the Finite
Fields theory of cvc5 \cite{SMFF}. We can leverage this to implement a
critical \emph{entailment} property, written $\eqs_1 \models \eqs_2$.
Our entailment relation is based on satisfiability in a standard
sense, where we represent models as memories $\store$ (mappings
from variables to values).
\begin{definition}
  We write $\store \models \eqs$ if $\store$ satisfies, aka is a model
  of, $\eqs$. We write $\eqs_1 \models
  \eqs_2$ iff  $\store \models E_1$ implies $\store \models
  E_2$ for all $\store$, and note this relation is reflexive and transitive.
\end{definition}
Given this Definition, the following Theorem is well-known and a fundamental
technique in SMT to implement our (common) notion of entailment. 
\begin{theorem}
  $\eqs \models \eqs'$ iff $\eqs \wedge \neg\eqs'$ is
  not satisfiable.
\end{theorem}

\subsection{Programs as Constraint Systems}
\label{section-smt-toeq}

\begin{fpfig}[t]{Interpretation of $\minicat$ expressions (T) and programs (B) as
  constraints}{fig-toeq}
\small{
\begin{mathpar}
  \toeq{x} = x

  \toeq{\elab{\be_1 \fplus \be_2}{\cid}} = \toeq{\elab{\be_2}{\cid}} \fplus \toeq{\elab{\be_1}{\cid}}

  \toeq{\elab{\be_1 \fminus \be_2}{\cid}} = \toeq{\elab{\be_2}{\cid}} \fminus \toeq{\elab{\be_1}{\cid}}

  \toeq{\elab{\be_1 \ftimes \be_2}{\cid}} = \toeq{\elab{\be_2}{\cid}} \ftimes \toeq{\elab{\be_1}{\cid}}
\end{mathpar}
\begin{mathpar}
  \toeq{\xassign{x}{\be}{\cid}} = x \eop \toeq{\elab{\be}{\cid}}
  
  \toeq{\elab{\assert{\be_1 = \be_2}}{\cid}} =  \toeq{\elab{\be_1}{\cid}} \eop \toeq{\elab{\be_2}{\cid}}

  \toeq{\prog_1;\prog_2} = \toeq{\prog_1} \wedge \toeq{\prog_2} 
\end{mathpar}}
\end{fpfig}

A central idea of our approach is that we can interpret any protocol
$\prog$ as a set of equality constraints (denoted $\toeq{\prog}$) and use an SMT
solver to verify properties relevant to correctness, confidentiality,
and integrity. Further, we can leverage entailment relation is critical for efficiency--
we can use annotations to obtain a weakened precondition for relevant properties.
That is, given $\prog$, program annotations or other cues can be used
to find a minimal $\eqs$ with $\toeq{\prog} \models \eqs$ for verifying
correctness and security.

The mapping $\toeq{\cdot}$ from programs $\prog$ to contraints
is defined in Figure \ref{fig-toeq}. The interpretion of OT is ommitted
from this figure which is general-- for $\mathbb{F}_2$ the interpretation
is:
\begin{mathpar}
  \toeq{\elab{\OT{\be_1}{\cid_1}{\be_2}{\be_3}}{\cid_2}} =
  (\toeq{\elab{\be_1}{\cid_1}} \ftimes \toeq{\elab{\be_3}{\cid_2}}) \fplus
  (\neg\toeq{\elab{\be_1}{\cid_1}} \ftimes \toeq{\elab{\be_2}{\cid_2}}) 
\end{mathpar}
In general we will assume that any preprocessing predicate is defined
via a constraint $\eqspre$, i.e., $\preproc(\store) \iff \store
\models \eqspre$ for all $\store$.  The correctness of the
SMT interpretation of programs is characterized as follows.
\begin{theorem}
  \label{theorem-toeq}
  $\store$ is a model of $\eqspre \wedge \toeq{\prog}$ iff $\store \in \runs(\prog)$.
\end{theorem}

\subsubsection{Example: Correctness of 3-Party Addition}
Consider the following 3-party additive secret sharing protocol.
{\footnotesize
$$
\begin{array}{c@{\qquad}c}
\begin{array}{lll}
  \elab{\mesg{s1}}{2} &:=& \elab{(\secret{1} \fminus \locflip \fminus \flip{x})}{1} \\ 
  \elab{\mesg{s1}}{3} &:=& \elab{\flip{x}}{1} \\ 
  \elab{\mesg{s2}}{1} &:=& \elab{(\secret{2} \fminus \locflip \fminus \flip{x})}{2} \\ 
  \elab{\mesg{s2}}{3} &:=& \elab{\flip{x}}{2} \\ 
  \elab{\mesg{s3}}{1} &:=& \elab{(\secret{3} \fminus \locflip \fminus \flip{x})}{3} \\ 
  \elab{\mesg{s3}}{2} &:=& \elab{\flip{x}}{3}
\end{array} & 
\begin{array}{lll}
  \rvl{1} &:=& \elab{(\locflip \fplus \mesg{s2} \fplus \mesg{s3})}{1} \\ 
  \rvl{2} &:=& \elab{(\mesg{s1} \fplus \locflip \fplus \mesg{s3})}{2} \\
  \rvl{3} &:=& \elab{(\mesg{s1} \fplus \mesg{s2} \fplus \locflip)}{3} \\
  \out{1} &:=& \elab{(\rvl{1} \fplus \rvl{2} + \rvl{3})}{1}\\
  \out{2} &:=& \elab{(\rvl{1} \fplus \rvl{2} + \rvl{3})}{2}\\
  \out{3} &:=& \elab{(\rvl{1} \fplus \rvl{2} + \rvl{3})}{3}
\end{array}
\end{array}
$$}
Letting $\prog$ be this protocol, we can verify correctness
as follows-- the output is the sum of the three input secrets:
$$
\toeq{\prog}\ \models\ \out{1} \eop \sx{1}{1} \fplus \sx{2}{2} \fplus \sx{3}{3}
$$

