\section{Share Types}

\newcommand{\sharety}{\mathrm{shares}}
\newcommand{\pubty}{\mathrm{public}}
\newcommand{\genshares}{\mathrm{genshares}}
\newcommand{\recon}{\mathrm{recon}}
\newcommand{\xvec}{\upsilon}
\newcommand{\xdefs}{\mathcal{D}}
\newcommand{\kwas}{\ \mathrm{as}\ }
\newcommand{\shtrip}[3]{#1 \Vdash #2 \rhd #3}
% \newcommand{\shtrip}{\htrip}
\newcommand{\redxsub}{\varrho}

\subsection{Logical Rules for $\minifed$}

The logical typing rules enforce confidentiality. They impose
coordination across clients and are defined in terms of vectors of
expressions $\xvec$. Critically \emph{\bf all rules are defined modulo
equivalence of vectors given $\eqs$}. We start by defining share
generation and reconstruction. While we assume additive shares in this
work, the type system is generally (intentionally) agnostic wrt to
these abstractions, and so could be swapped out for, e.g., Shamir
shares.

\begin{eqnarray*}
  \genshares(\phi,y,\cid) &\defeq& \setit{\rx{y}{\cid}}, [\phi - \rx{y}{\cid}, \rx{y}{\cid}]\\
  \recon(\phi_1,\phi_2) &\defeq& \phi_1 \fplus \phi_2
\end{eqnarray*}

Generally the type system works more like an information flow type system
than our previous work. The following rules define allowable interactions
between clients. The idea is that all interactive messages must be categorized
as either shares or public values in order for the protocol to be well-typed.
\begin{mathpar}
  \inferrule[Share]
      {\eqs \models \genshares(\phi_1,y,\cid) \eop R_2,[\mx{z}{\cid},\phi_2] }
      {\Gamma,R_1,\eqs \vdash [\mx{z}{\cid}] : R_1;R_2, \Gamma; [\mx{z}{\cid},\phi_2] : \sharety}

  \inferrule[Open]
      {\Gamma, \eqs \vdash [\px{y},\px{z}] : \sharety}
      {\Gamma,R,\eqs \vdash [\px{y},\px{z}] : R, \Gamma; [\recon(\px{y},\px{z})] : \pubty}

  \inferrule[Seq]
      {\Gamma_1,\eqs,R_1 \vdash \xvec_1 : R_2,\Gamma_2 \\
        \Gamma_2,\eqs,R_2 \vdash \xvec_2 : R_3,\Gamma_3}
      {\Gamma_1, \eqs,R_1 \vdash \xvec_1\xvec_2 : R_3,\Gamma_3}
\end{mathpar}

In addition to the interactive rules, we have rules for local computation.
Again, these are all modulo equivalence of vectors under $\eqs$. These rules
allow standard homorphic encryption operations, and in particular ensure
that all openings are secure in a NIMO sense. These can be combined with
the Hoare logic to obtain a precise bound on information release, e.g.,
that outputs are secure openings of the ideal functionality.
\begin{mathpar}
  \inferrule[Hyp]
      {}
      {\Gamma,\eqs \vdash \xvec : \Gamma(\xvec)}

  \inferrule[HEAdd]
      {\Gamma,\eqs \vdash [\phi^1_1,\phi^1_2] : \sharety\\
        \Gamma,\eqs \vdash [\phi^2_1,\phi^2_2] : \sharety}
      {\Gamma,\eqs \vdash [\phi^1_1\fplus\phi^2_1,\phi^1_2\fplus\phi^2_2] : \sharety}

  \inferrule[HEAddPub]
      {\Gamma,\eqs \vdash [\phi] : \pubty\\
        \Gamma,\eqs \vdash [\phi_1,\phi_2] : \sharety}
      {\Gamma,\eqs \vdash [\phi_1\fplus\phi,\phi_2] : \sharety}
            
  \inferrule[HEMultPub]
      {\Gamma,\eqs \vdash [\phi] : \pubty\\
        \Gamma,\eqs \vdash [\phi_1,\phi_2] : \sharety}
      {\Gamma,\eqs \vdash [\phi_1\ftimes\phi,\phi_2\ftimes\phi] : \sharety}

  \inferrule[PubOp]
      {\Gamma,\eqs \vdash [\phi_1] : \pubty\\
        \Gamma,\eqs \vdash [\phi_2] : \pubty}
      {\Gamma,\eqs \vdash [\phi_1\bop\phi_2] : \pubty}
\end{mathpar}

\subsection{Algorithmic Rules for $\metaprot$}

The $\minifed$ type system is undecidable given validity up to equivalence of expression
vectors, so as in previous work we want a decidable algorithm for $\metaprot$ that must
rely on type annotations and explicit type preconditions. For example, a GMW or BGW
sum gate should assume that the inputs are shares, and assert that the outputs are shares.
To this end we extend the syntax of commands to include type annotations:
$$
\cmd ::= \cdots \mid [e,e] \kwas \genshares(\notg{\phi},e,e) \mid [e,e] : \sharety \mid
     [e,e] : \pubty \mid [e] : \pubty
$$
These annotations need to be verified, and in general may rely on an SMT equivalence.
Currently the only examples of the latter I've found are for the genshares case, but
coercion forms for others could easily be added. The relevant equivalence for
genshares can be inferred as a precondition in the Hoare logic, as follows.
\begin{mathpar}
  %\inferrule[Assert]
  %    {}
  %    {\htrip{\mx{e^1_1}{e^1_2} \eop \phi - \rx{y}{\cid} \wedge
  %            \mx{e^2_1}{e^2_2} \eop \rx{y}{\cid}}
  %           {[\mx{e^1_1}{e^1_2},\mx{e^2_1}{e^2_2}] \kwas \genshares(\phi,y,\cid)}{\eqtrue}}
  \inferrule[CoerceShares]
      {}
      {\htrip{e_1 \eop \notg{\phi} - \rx{e}{e'} \wedge e_2 \eop \rx{e}{e'}}
             {[e_1,e_2] \kwas \genshares(\notg{\phi},e,e')}{\eqtrue}}
      
  %\inferrule[CoerceShares]
  %    {}
  %    {\htrip{e \eop e_1 \fplus e_2}{[e] \kwas \recon(e_1,e_2)}{\eqtrue}} 
\end{mathpar}
The typing rules are then given as follows. First, the type annotations, which
rely on local computation typing which is purely syntax-directed given an empty
constraint. $\xdefs$ is a mapping from $x$s to $\phi$s (their ``definitions'')
which is just the explicit definition of $x$ in the protocol (see rule
\TirName{Defn}).
\begin{mathpar}
  \inferrule[CoerceGenshare]
      {e_1 \redx \mx{w_1}{\cid_1} \\ e_2 \redx \mx{w_2}{\cid_2} \\ e \redx w \\ e' \redx \cid}
      {\shtrip{\Gamma, R, \xdefs}
        {[e_1,e_2] \kwas \genshares(\notg{\phi},e,e')}
        {\Gamma;[\mx{w_1}{\cid_1},\mx{w_2}{\cid_2}] : \sharety, R;\setit{\rx{w}{\cid}}, \xdefs} }

  \inferrule[LocalShare]
      {e_1 \redx \mx{w_1}{\cid_1} \\ e_2 \redx \mx{w_2}{\cid_2}  \\\\
        \Gamma,\eqtrue \vdash [\xdefs(\mx{w_1}{\cid_1} ), \xdefs(\mx{w_2}{\cid_2})] : \sharety}
      {\shtrip{\Gamma, R, \xdefs}{[e_1,e_2] : \sharety}{\Gamma;[\mx{w_1}{\cid_1},\mx{w_2}{\cid_2}] : \sharety,
          R, \xdefs}}

  \inferrule[Open]
      {e_1 \redx \px{w_1} \\ e_2 \redx \px{w_2}  \\
        \Gamma,\eqtrue \vdash [\xdefs(\px{w_1}), \xdefs(\px{w_2})] : \sharety}
      {\shtrip{\Gamma, R, \xdefs}{[e_1,e_2] : \pubty}{\Gamma;[\px{w_1} \fplus \px{w_2}] : \pubty,
          R, \xdefs}}

  \inferrule[Public]
      {e \redx \mx{w}{\cid} \\ \Gamma,\eqtrue \vdash [\xdefs(\mx{w}{\cid})] : \pubty}
      {\shtrip{\Gamma, R, \xdefs}{[e] : \pubty}{\Gamma;[\mx{w}{\cid}] : \pubty,
          R, \xdefs}}

  \inferrule[Defn]
      {e_1 \redx \mx{w}{\cid_1} \\ e_2 \redx \elab{\be}{\cid_2}}
      {\shtrip{\Gamma, R, \xdefs}{e_1 := e_2}{\Gamma, R, \xdefs[\mx{w}{\cid_1} \mapsto \toeq{\elab{\be}{\cid_2}}]}}
\end{mathpar}
The remaining rules are as follows, with the rule for application
being the most interesting as usual. Since we allow dependent
generalization in function types, as in previous work (and as in the
Hoare logic) we equate expressions up to the result of their
evaluation. To make rules a little more succint we define ``evaluating
substitution'' $\redxsub(\notg{R}),\redxsub(\notg{\Gamma})$ to also
impose evaluation \cnote{probably should do this for $\peq$ as
  well}. I.e., given $\redxsub = [e_1/y_1 \cdots e_n/y_n]$:
    $$
    \redxsub(\notg{\Gamma}) = \Gamma \ \text{where}\ \notg{\Gamma}[e_1/y_1 \cdots e_n/y_n] \redx \Gamma
    $$
and so on.:
\begin{mathpar}
  \inferrule[App]
      {f : \Pi y_1,\ldots,y_n. \notg{\Gamma_1} \rightarrow (\notg{R'},\notg{\Gamma_2}) \\
       \redxsub = [e_1/y_1 \cdots e_n/y_n]}
      {\shtrip{\Gamma;\redxsub(\notg{\Gamma_1}), R, \xdefs}{f(e_1,\ldots,e_n)}{\Gamma;\redxsub(\notg{\Gamma_2}), R;\redxsub(\notg{R'}), \xdefs}}

  \inferrule[Seq]    
      {\shtrip{\Gamma_1,R_1,\xdefs_1}{\cmd_1}{\Gamma_2,R_2,\xdefs_2} \\
       \shtrip{\Gamma_2,R_2,\xdefs_2}{\cmd_2}{\Gamma_3,R_3,\xdefs_3}}
      {\shtrip{\Gamma_1,R_1,\xdefs_1}{\cmd_1;\cmd_2}{\Gamma_3,R_3,\xdefs_3}}

  \inferrule[Let]
      {\shtrip{\Gamma_1,R_1,\xdefs_1}{\cmd[e/y]}{\Gamma_2,R_2,\xdefs_2}}
      {\shtrip{\Gamma_1,R_1,\xdefs_1}{\elet{y}{e}{\cmd}}{\Gamma_2,R_2,\xdefs_2}}
\end{mathpar}
The \TirName{App} rule relies on function typings, which are a dependent type form
$\Pi y_1,\ldots,y_n. \notg{\Gamma_1} \rightarrow (\notg{R},\notg{\Gamma_2})$ where
$y_1,\ldots,y_n$ are the expression parameters, $\notg{\Gamma_1}$ is the
generalized typing precondition, and $(\notg{R},\notg{\Gamma_2})$ is the generalized
typing postconditions including any randomness consumed by OTP encryption.
Function typings are computed as follows (compositionally):
$$
\inferrule[Fn]
    {\codebase(f) = y_1 : \tau_1, \ldots, y_n : \tau_n, \cmd \\ \mv_1,\ldots,\mv_n = \fresh(\tau_1,\ldots,\tau_n) \\
      \redxsub = [\mv_1/y_1 \cdots \mv_n/y_n] \\
     \shtrip{\redxsub(\notg{\Gamma_1}), \varnothing, \varnothing}{\cmd}{\redxsub(\notg{\Gamma_2}), \redxsub(\notg{R}),\xdefs}}
    {f : \Pi y_1,\ldots,y_n. \notg{\Gamma_1} \rightarrow (\notg{R},\notg{\Gamma_2})}
 $$

Note that in rule $\TirName{Fn}$, the precondition $\notg{\Gamma_1}$
and the typing annotations that may contribute to
$(\notg{R},\notg{\Gamma_2})$ need to be specified somewhere, a style
choice. Currently they are being written as additional clauses to the
\ttt{precondition} and \ttt{postcondition} of functions, e.g.:
\begin{verbatimtab}
precondition: ([m[x++"s"]@1,m[x++"s"]@2] : shares;
               [m[y++"s"]@1,m[y++"s"]@2] : shares)
sum(z:string, x:string, y:string){ _sum(z,x,y,1); _sum(z,x,y,2) }
postcondition: ( [m[z++"s"]@1,m[z++"s"]@2] : shares),
               ( SHARED(z) == SHARED(x) + SHARED(y) )
\end{verbatimtab}
This example illustrates how we can write both confidentiality and correctness properties--
or not include either as desired. That is, the confidentiality type system will ensure
gradual release, without any SMT pre or postconditions. 
