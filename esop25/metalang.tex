\section{Compositional Type Verification in $\metaprot$}

Practical MPC computations protocols are
typically composed of compositional units. Examples include GMW circuits
and Yao's Garbled Circuits (YGC), that are composed of so-called
garbled gates. Languages such as Fairplay \cite{269581} provide gates as
units of abstraction that are ``wired'' together by the programmer to
generate a complete circuit.

The $\fedprot$ language is low-level and does not include abstractions
for defining composable elements. So in this Section we introduce the
$\metaprot$ language that includes structured data and function
definitions for defining composable protocol elements at a higher
level of abstraction.  The $\metaprot$ language is a
\emph{metalanguage}, where $\fedprot$ protocols are the residuum of
computation. In addition to these declarative benefits of $\metaprot$,
component definitions support compositional verification of larger
protocols as we will discuss with examples in Section
\ref{section-examples}.

\metaprotsyntaxfig

\subsection{Syntax and Semantics}

The syntax of $\metaprot$ is defined in Figure
\ref{fig-metaprotsyntax}.  It includes a syntax of values $\mv$ that
include client ids $\cid$, identifier strings $w$, expressions $\be$
in field $\mathbb{F}_p$, record values, and $\minicat$ variables
$x$. $\metaprot$ expression forms allow dynamic construction of these
values. $\metaprot$ \emph{instruction} forms allow dynamic
construction of $\minicat$ protocols $\prog$, incorporating expression
evaluation. The syntax also supports definitions of functions that
compute values $\mv$ and, as a distinct form, functions that compute
protocols.  Formally, we consider a complete metaprogram to include
both a codebase and a ``main'' program that uses the codebase.
\begin{definition}
A \emph{codebase} $\codebase$ is a list of function 
declarations. We write $ \codebase(f) = y_1,\ldots,y_n,\ e$
if $f(y_1,\ldots,y_n) \{ e \} \in \codebase$, and we
write  $ \codebase(f) = y_1,\ldots,y_n,\ \instr$
if $f(y_1,\ldots,y_n) \{ \instr \} \in \codebase$.
%A \emph{metaprogram}, aka \emph{metaprotocol}  is a pair of a 
%codebase and expression $\codebase, e$. We may omit
%$\codebase$ if it is clear from context.  
\end{definition}

\metaprotexprsemanticsfig

We define a big-step evaluation relation $\redx$ in Figures
\ref{fig-metaprotexprsemantics} and \ref{fig-metaprotinstrsemantics}
for expressions and instructions, respectively.  In this definition we
write $e[\mv/y]$ and $\instr[\mv/y]$ to denote the substitution of
$\mv$ for free occurrences of $y$ in $e$ or $\instr$ respectively. The
rules are mostly standard. Note that we do not include an evaluation
rule for the form $\eqcast{e}{e}{\notg{\phi}}$ which only has
relevance for the type system and which we assume is erased from
programs prior to evaluation. We defer discussion of this form,
as well as the syntactic category $\notg{\phi}$, to the next
Section.


\metaprotinstrsemanticsfig

\subsection{Dependent Hoare Type Theory}

Our first goal in the $\metaprot$ type theory is to define an
algorithmic system that is sound for both confidentiality and
integrity typings as defined in Sections \ref{section-cpj} and
\ref{section-ipj}. Returning to the example in Section
\ref{section-cpj}, note that the key equivalence of the $\ttt{mux}$
expression with a one-time-pad encryption we observe there
is not trivial. To ensure that this sort of equivalence
can be picked up by the type system, we introduce a
an annotation form that allows the programmer to provide
the needed hint.

Consider the following $\ttt{encode}$ function definition that
generalizes this encoding for any identifier $y$ and sender, receiver
pair $s,r$. The second line in the body is a hint that
the message $\mx{y}{r}$ can be considered equivalent to
$\neg\sx{y}{s} \fplus \rx{y}{s}$:
$$
\begin{array}{l}
\ttt{encode}(y,s,r) \{\\
\quad \xassign{\mx{y}{r}}{\mux{\secret{y}}{\neg\flip{y}}{\flip{y}}}{s};\\
\quad \eqcast{\mx{y}{r}}{\neg\sx{y}{s} \fplus \rx{y}{s}}\\
\}
\end{array}
$$
This hint can be validated using SMT, and then the
syntactic structure of $\neg\sx{y}{s} \fplus \rx{y}{s}$
allows its immediate interpretation as a one-time-pad encoding.

Our second goal in the $\metaprot$ type system is to minimize the
amount of SMT solving needed for type verification.  Returning to the
$\ttt{encode}$ example, type checking in our systems only requires the
hint to be verified once, with the guarantee that $\ttt{encode}$ can
be applied anywhere without needing to re-verify the hint in
application instances. To verify the hint in $\ttt{encode}$, we can
just choose arbitrary values $w,\cid_1,\cid_2$ for $y$, $s$, and $r$,
evaluate $\ttt{encode}(w,\cid_1,\cid_2) \redx \prog$, and then verify:
$$\toeq{\prog} \models \mx{w}{\cid_2} \eop
\neg\sx{w}{\cid_1} \fplus \rx{w}{\cid_1}$$ 
Since $\ttt{encode}$ is closed, validity is guaranteed for
any instance of $y$, $s$, and $r$.

We generalize these benefits of compositional verification by allowing
pre- and post-condition annotations on $\metaprot$ functions. For
example, consider a GMW-style, 2-party ``and-gate'' function
$\ttt{andgmw}(x,y,z)$. In this protocol, each party $\cid
\in \{1,2\}$ holds an additive secret share $\mx{x}{\cid}$ of
values identified by $x$ and $y$, and at the end of the
protocol each hold a secret share $\mx{z}{\cid}$, where:
$$
\mx{z}{1} \fplus \mx{z}{2} \eop (\mx{x}{1} \fplus \mx{x}{2}) \ftimes (\mx{y}{1} \fplus \mx{y}{2})
$$
We provide details of $\ttt{andgmw}$ and other GMW protocol
elements in Section \ref{section-examples}.  Our point now is that,
similarly to the $\ttt{encode}$ example, we can verify this
post-condition once as a correctness property for $\ttt{andgmw}$, and
then integrate instances of it into circuit correctness properties
with the guarantee that each instance also holds for any
$\ttt{andgmw}$ gate.  Since the program logic of $\ttt{andgmw}$ is
non-trivial, and typical circuits can use up to thousands of gates,
this has significant practical benefits by greatly reducing SMT overhead
in whole-program analysis.

\subsubsection{$\minicat$ expression type algorithm.}

A core element of $\metaprot$ type checking is type checking
of $\minicat$ expressions. The integrity type system presented in Section
\ref{section-ipj} is already algorithmic and ready to use. But confidentiality
typing presented in Section \ref{section-cpj} is not syntax-directed
due to the \TirName{Encode} rule. But as described above, by introducing
hint annotations we can ``cast'' any relevant expression form into
the simplest equivalent syntactic form for one-time-pad encoding.
Thus, in Figure \ref{fig-atj} we present the algorithm for
confidentiality type judgements, where we eliminate the need for
integrated SMT solving by assuming this sort of casting. 

\atjfig

The following Lemma establishes correctness of algorithmic confidentiality
type checking, and makes explicit that SMT checking is eliminated in the
judgement.
\begin{lemma}
  If $\atj{\toeq{\phi}}{R}{\ty}$ then $\eqj{R}{\eqs}{\phi}{\ty}$ for any $\eqs$.
\end{lemma}

\subsubsection{Dependent Hoare types for instructions.}

Hoare-style types for instructions have the following form:
$$
\hty{\eqs_1}{\Gamma}{R}{\Delta}{\eqs_2}
$$
Here, $\eqs_1$ and $\eqs_2$ are the pre- and post- conditions
respectively, $\Gamma$ is the confidentiality type environment of the
protocol resulting from execution of the instruction, which is sound
wrt confidentiality typing, $R$ are the one-time-pads consumed in the
confidentialty typing, and $\Delta$ is the sound integrity type
environment.

\notgfig

To obtain high precision in the type system, and to generalize
functions typings, we also introduce a form of type dependency,
specifically dependence on $\metaprot$ expressions. Dependent
$\Pi$ types have the form:
$$
\dht{y_1,\ldots,y_n}{\notg{\eqs_1}}{\notg{\Gamma}}{\notg{R}}{\notg{\Delta}}{\notg{\eqs_2}}
$$
where $y_1,\ldots,y_n$ range over values $\mv$ and each of
$\notg{\eqs_1}$ etc.~may contain expressions with free occurences
of $y_1,\ldots,y_n$-- the syntax of these forms is in Figure
\ref{fig-notg}. These $\Pi$ types are assigned to functions
and instantiated at application points. 

For example, here is a valid typing of $\ttt{encode}$.  It says that
under any precondition, evaluating $\ttt{encode}$ results in a cipher
type for the encoded message $\mx{x}{r}$, which consumes the
one-time-pad $\rx{x}{s}$, and the integrity of $\mx{x}{r}$ is
determined by the integrity of $\rx{x}{s}$ and $\sx{x}{s}$. The
postcondition expresses the key confidentiality property of
$\mx{x}{r}$, but also may be practically useful for correctness
properties since it is a simpler expression form than the $\ttt{mux}$
form:
$$
\ttt{encode} :
\begin{array}[t]{ll}
  \Pi x,s,r . & \{ \} \quad \mx{x}{r} : \cty{\rx{x}{s}}{\sx{x}{s}}, \{ \rx{x}{s} \} \\
  & \phantom{\{ \}} \quad \mx{x}{r} : \ity{s}{\{ \rx{x}{s}, \sx{x}{s} \}} \\
  & \{ \mx{x}{r} \eop \neg\sx{y}{s} \fplus \rx{y}{s} \}
\end{array}
$$


\begin{mathpar}
  \inferrule[Mesg]
            {e_1 \redx x \\ e_2 \redx \be \\ e_3 \redx \cid \\ \atj{\toeq{\elab{\be}{\cid}}}{R_2}{\ty} \\
              \itj{\cid}{\be}{V}}
            {\mtj{\xassign{e_1}{e_2}{e_3}}{\eqs}{(x:\ty)}{R_1;R_2}{(x : \ity{\cid}{V})}{\eqs \wedge x \eop \toeq{\elab{\be}{\cid}}}}

  \inferrule[Encode]
            {e_1 \redx w \\ e_2 \redx \cid \\ \notg{\phi} \redx \phi \\
              \eqs \models \toeq{\elab{\be}{\cid}} \eop \phi\\
              \atj{\phi}{R}{\ty}}
            {\mtj{\eqcast{\mx{e_1}{e_2}}{\notg{\phi}}}{\eqs}{(\mx{w}{\cid}:\ty)}{R}{\varnothing}{\eqs}}

  \inferrule[App]
            {\tsig(f) = \dht{y_1,\ldots,y_n}{\notg{\eqs_1}}{\notg{\Gamma}}{\notg{R}}{\notg{\Delta}}{\notg{\eqs_2}} \\
              e_1 \redx \mv_1\ \cdots\ e_n \redx \mv_n \\
              \subn = [\mv_1/y_1]\cdots[\mv_n/y_n] \\
              \subn(\hty{\notg{\eqs_1}}{\notg{\Gamma}}{\notg{R}}{\notg{\Delta}}{\notg{\eqs_2}}) \redx
                    \hty{\eqs_1}{\Gamma}{R}{\Delta}{\eqs_2} \\
              \eqs \models \eqs_1}
            {\mtj{f(e_1,\ldots,e_n)}{\eqs}{\Gamma}{R}{\Delta}{\eqs \wedge \eqs_2}}

  \inferrule[Seq]          
            {\mtj{\prog_1}{\eqs_1}{\Gamma_1}{R_1}{\Delta_1}{\eqs_2} \\
             \mtj{\prog_2}{\eqs_2}{\Gamma_2}{R_2}{\Delta_2}{\eqs_3}}
            {\mtj{\prog_1;\prog_2}{\eqs_1}{\Gamma_1;\Gamma_2}{R_1;R_2}{\Delta_1;\Delta_2}{\eqs_3}}
\end{mathpar}

\begin{mathpar}
  \inferrule[Sig]
            {\codebase(f) = y_1,\ldots,y_n, \instr \\
              \subn = [\mv_1/y_1]\cdots[\mv_n/y_n] \\
              \subn(\hty{\notg{\eqs_1}}{\notg{\Gamma}}{\notg{R}}{\notg{\Delta}}{\notg{\eqs_2}}) \redx
                    \hty{\eqs_1}{\Gamma}{R}{\Delta}{\eqs_2} \\
              \mtj{\subn(\instr)}{\eqs_1}{\Gamma}{R}{\Delta}{\eqs} \\
              \eqs \models \eqs_2}
            {f : \dht{y_1,\ldots,y_n}{\notg{\eqs_1}}{\notg{\Gamma}}{\notg{R}}{\notg{\Delta}}{\notg{\eqs_2}}}
\end{mathpar}

\begin{definition}
  $\tsig$ is \emph{verified} iff $f : \tsig(f)$ is valid for all $f \in \dom(\tsig)$.
\end{definition}

\begin{theorem}
  Given preprocessing predicate $\eqspre$, program $\instr$, and verified $\tsig$, if
  the judgement $\mtj{\instr}{\eqspre}{\Gamma}{R}{\Delta}{\eqs}$ is derivable then
  $\instr \redx \prog$ and:
  \begin{enumerate}
  \item $\cpj{R}{\eqs}{\eqspre \wedge \toeq{\prog}}{\Gamma}$ is valid.
  \item $\ipj{\eqs}{\prog}{\Delta}$ is valid.
  \end{enumerate}
\end{theorem}
