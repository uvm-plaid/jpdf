\section{Compositional Type Verification in $\metaprot$}
\label{section-metalang}

The $\metaprot$ language \cite{skalka-near-ppdp24} includes structured
data and function definitions for defining composable protocol
elements at a higher level of abstraction than $\minicat$.  The
$\metaprot$ language is a \emph{metalanguage} aka metaprogramming
language, where a $\fedprot$ protocols is the result of
computation. In addition to these declarative benefits of $\metaprot$,
component definitions support compositional verification of larger
protocols. Separate verification of well-designed components results
in confidentiality and integrity properties to be recorded in their
types, allowing for significant reduction of SMT verification in whole
program analysis, as we will discuss with extended examples in Section
\ref{section-examples}.

\metaprotsyntaxfig

\subsection{Syntax and Semantics}

The syntax of $\metaprot$ is defined in Figure
\ref{fig-metaprotsyntax}.  It includes a syntax of values $\mv$ that
include client ids $\cid$, identifier strings $w$, expressions $\be$
in field $\mathbb{F}_p$, record values, and $\minicat$ variables
$x$. $\metaprot$ expression forms allow dynamic construction of these
values. $\metaprot$ \emph{instruction} forms allow dynamic
construction of $\minicat$ protocols $\prog$, incorporating expression
evaluation. The syntax also supports definitions of functions that
compute values $\mv$ and, as a distinct form, functions that compute
protocols $\prog$.  Formally, we consider a complete metaprogram to include
both a codebase and a ``main'' program that uses the codebase.
We disallow recursion, mainly to guarantee decidability
of type dependence (Section \ref{section-metalangty}).
\begin{definition}
A \emph{codebase} $\codebase$ is a list of function 
declarations. We write $ \codebase(f) = y_1,\ldots,y_n,\ e$
if $f(y_1,\ldots,y_n) \{ e \} \in \codebase$, and we
write  $ \codebase(f) = y_1,\ldots,y_n,\ \instr$
if $f(y_1,\ldots,y_n) \{ \instr \} \in \codebase$.
%A \emph{metaprogram}, aka \emph{metaprotocol}  is a pair of a 
%codebase and expression $\codebase, e$. We may omit
%$\codebase$ if it is clear from context.  
\end{definition}

\metaprotexprsemanticsfig

We define a big-step evaluation relation $\redx$ in Figures
\ref{fig-metaprotexprsemantics} and \ref{fig-metaprotinstrsemantics}
for expressions and instructions, respectively.  In this definition we
write $e[\mv/y]$ and $\instr[\mv/y]$ to denote the substitution of
$\mv$ for free occurrences of $y$ in $e$ or $\instr$ respectively. The
rules are mostly standard. Note that we do not include an evaluation
rule for the form $\eqcast{\mx{e}{e}}{\notg{\phi}}$ which is a type
annotation that we assume is erased from programs prior to
evaluation. We defer discussion of this form, as well as the syntactic
category $\notg{\phi}$, to the next Section.

\metaprotinstrsemanticsfig

\subsection{Dependent Hoare Type Theory}
\label{section-metalangty}

Our first goal in the $\metaprot$ type theory is to define an
algorithmic system that is sound for both confidentiality and
integrity typings as defined in Sections \ref{section-cpj} and
\ref{section-ipj}. Returning to the YGC secret encoding example in
Section \ref{section-cpj}, note that the key equivalence of the
$\ttt{mux}$ expression with a one-time-pad encryption we observe there
is not trivial. To ensure that this sort of equivalence can be picked
up by the type system, we introduce an $\ttt{as}$ annotation form that
allows the programmer to provide the needed hint.

Consider the following $\ttt{encode}$ function definition that
generalizes this YGC encoding for any identifier $y$ and sender, receiver
pair $s,r$. The second line in the body is a hint that
the message $\mx{y}{r}$ can be considered equivalent to
$\neg\sx{y}{s} \fplus \rx{y}{s}$:
$$
\begin{array}{l}
\ttt{encode}(y,s,r) \{\\
\quad \xassign{\mx{y}{r}}{\mux{\secret{y}}{\neg\flip{y}}{\flip{y}}}{s};\\
\quad \eqcast{\mx{y}{r}}{\neg\sx{y}{s} \fplus \rx{y}{s}}\\
\}
\end{array}
$$
This hint can be validated using SMT, and then the
syntactic structure of $\neg\sx{y}{s} \fplus \rx{y}{s}$
allows its immediate interpretation as a one-time-pad encoding.

Our second goal in the $\metaprot$ type system is to minimize the
amount of SMT solving needed for type verification.  Returning to the
$\ttt{encode}$ example, type checking in our systems only requires the
hint to be verified once, with the guarantee that $\ttt{encode}$ can
be applied anywhere without needing to re-verify the hint in
application instances. To verify the hint in $\ttt{encode}$, we can
just choose arbitrary ``fresh'' values $w,\cid_1,\cid_2$ for $y$, $s$,
and $r$, evaluate $\ttt{encode}(w,\cid_1,\cid_2) \redx \prog$, and
then verify:
$$\toeq{\prog} \models \mx{w}{\cid_2} \eop
\neg\sx{w}{\cid_1} \fplus \rx{w}{\cid_1}$$ 
Since $\ttt{encode}$ is closed, validity is guaranteed for
any instance of $y$, $s$, and $r$.

We generalize these benefits of compositional verification by allowing
pre-~and postcondition annotations on $\metaprot$ functions. For
example, consider a GMW-style, 2-party ``and-gate'' function
$\ttt{andgmw}(x,y,z)$. In this protocol, each party $\cid
\in \{1,2\}$ holds an additive secret share $\mx{x}{\cid}$ of
values identified by $x$ and $y$, and at the end of the
protocol each hold a secret share $\mx{z}{\cid}$, where:
$$
\mx{z}{1} \fplus \mx{z}{2} \eop (\mx{x}{1} \fplus \mx{x}{2}) \ftimes (\mx{y}{1} \fplus \mx{y}{2})
$$
We provide details of $\ttt{andgmw}$ and other GMW protocol
elements in Section \ref{section-examples}.  Our point now is that,
similarly to the $\ttt{encode}$ example, we can verify this
postcondition once as a correctness property for $\ttt{andgmw}$, and
then integrate instances of it into circuit correctness properties
with the guarantee that each instance also holds for any
$\ttt{andgmw}$ gate.  Since the program logic of $\ttt{andgmw}$ is
non-trivial, and typical circuits can use up to thousands of gates,
this has significant practical benefits by greatly reducing SMT overhead
in whole-program analysis.

\atjfig

\subsubsection{$\minicat$ expression type algorithm.}

A core element of $\metaprot$ type checking is type checking
of $\minicat$ expressions. The integrity type system presented in Section
\ref{section-ipj} is already algorithmic and ready to use. But confidentiality
typing presented in Section \ref{section-cpj} is not syntax-directed
due to the \TirName{Encode} rule. But as described above, by introducing
hint annotations we can ``coerce'' any relevant expression form into
the simplest equivalent syntactic form for one-time-pad encoding.
Thus, in Figure \ref{fig-atj} we present the algorithm for
confidentiality type judgements, where we eliminate the need for
integrated SMT solving by assuming this sort of casting. 

The following Lemma establishes correctness of algorithmic confidentiality
type checking, and makes explicit that SMT checking is eliminated in the
judgement.
\begin{lemma}
  \label{lemma-atj-sound}
  If $\atj{\toeq{\phi}}{R}{\ty}$ then $\eqj{R}{\eqs}{\phi}{\ty}$ for any $\eqs$.
\end{lemma}

\subsubsection{Dependent Hoare types for instructions.}

Hoare-style types for instructions have the following form:
$$
\hty{\eqs_1}{\Gamma}{R}{\Delta}{\eqs_2}
$$
Here, $\eqs_1$ and $\eqs_2$ are the pre- and postconditions
respectively, $\Gamma$ is the confidentiality type environment of the
protocol resulting from execution of the instruction, which is sound
wrt confidentiality typing, $R$ are the one-time-pads consumed in the
confidentiality typing, and $\Delta$ is the sound integrity type
environment.

\notgfig

To obtain high precision in the type system, and to generalize
functions typings, we also introduce a form of type dependency,
specifically dependence on $\metaprot$ expressions. Dependent
$\Pi$ types have the form:
$$
\dht{y_1,\ldots,y_n}{\notg{\eqs_1}}{\notg{\Gamma}}{\notg{R}}{\notg{\Delta}}{\notg{\eqs_2}}
$$
where $y_1,\ldots,y_n$ range over values $\mv$ and each of
$\notg{\eqs_1}$ etc.~may contain expressions with free occurrences
of $y_1,\ldots,y_n$-- the syntax of these forms is in Figure
\ref{fig-notg}. These $\Pi$ types are assigned to functions
and instantiated at application points. 

For example, here is a valid typing of $\ttt{encode}$.  It says that
under any precondition, evaluating $\ttt{encode}$ results in a cipher
type for the encoded message $\mx{x}{r}$, which consumes the
one-time-pad $\rx{x}{s}$, and the integrity of $\mx{x}{r}$ is
determined by the security level of $s$. The
postcondition expresses the key confidentiality property of
$\mx{x}{r}$, but also may be practically useful for correctness
properties since it is a simpler expression form than the $\ttt{mux}$
form:
$$
\ttt{encode} :
\begin{array}[t]{ll}
  \Pi x,s,r . & \{ \}\quad \mx{x}{r} : \cty{\rx{x}{s}}{\sx{x}{s}}, \{ \rx{x}{s} \} \ \cdot\\
  & \phantom{\{ \}} \quad \mx{x}{r} : \ity{s}{\setit{}} \\
  & \{ \mx{x}{r} \eop \neg\sx{y}{s} \fplus \rx{y}{s} \}
\end{array}
$$
A key property of this example and our type system generally is that once
the $\Pi$ type is verified, typing any application of it requires verification
of the precondition instance, but not the postcondition instance.

\subsection{Algorithmic Type Validity in $\metaprot$}

\mtjfig

We equate types up to evaluation of subexpressions as defined in
Figure \ref{fig-notg}. Since expression evaluation is total we can
just evaluate types of closed instructions, which are guaranteed to
have closed types, to obtain Hoare typings for instructions. In Figure
\ref{fig-mtj} we define the type derivation rules for $\metaprot$. In
the \TirName{Mesg} rule we use algorithmic confidentiality and
integrity type checking in a straightforward manner.  In the
\TirName{Encode} rule, we verify the hint given in the type
annotation, and then use it for type checking.  In the \TirName{App}
rule we verify preconditions of the given function type and
instantiate the type and postconditions with the given expression
arguments.  The \TirName{Sig} rule we specify how to verify $\Pi$
types for functions. Note that the values $\mv_1,\ldots,\mv_n$ chosen
to instantiate the function parameters are chosen ``fresh'', meaning
that any strings and client identifiers occurring as substrings in
$\mv_1,\ldots,\mv_n$ are unique and not used in any program source
code. This ensures generality of verification as demonstrated in Lemma
\ref{lemma-eqs-notg} in the Appendix.

\mtjfnfig

We allow the programmer to specify a type signature $\tsig$ that
maps functions in $\codebase$ to valid dependent Hoare
typings. Define:
\begin{definition}
  $\tsig$ is \emph{verified} iff $f : \tsig(f)$ is valid for all $f \in \dom(\tsig)$.
\end{definition}
In practice, to define $\tsig$ the programmer only needs to specify
pre-~and postconditions, as we illustrate in Section
\ref{section-examples}, since the rest of the type is reconstructed by
type checking.

The following Theorem establishes our main result, that the
$\metaprot$ type checking is sound with respect to both
confidentiality and integrity typings. Since these typings imply
security hyperproperties of confidentiality and integrity, $\metaprot$
type checking implicitly enforces them. Proof details are given in the
Appendix.
\begin{theorem}[Soundness of $\metaprot$ type checking]
  \label{theorem-mtj}
  Given preprocessing predicate $\eqspre$, program $\instr$, and verified $\tsig$, if
  the judgement: $$\mtj{\instr}{\eqspre}{\Gamma}{R}{\Delta}{\eqs}$$ is derivable then
  $\instr \redx \prog$ and:
  \begin{enumerate}
  \item $\cpj{R}{\eqs}{\eqspre \wedge \toeq{\prog}}{\Gamma}$ is valid.
  \item $\ipj{\eqs}{\prog}{\Delta}$ is valid.
  \end{enumerate}
\end{theorem}

