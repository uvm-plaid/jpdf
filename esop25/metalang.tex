\section{Compositional Type Verification in $\metaprot$}

Practical MPC computations protocols are
typically composed of compositional units. Examples include GMW circuits
and Yao's Garbled Circuits (YGC), that are composed of so-called
garbled gates. Languages such as Fairplay \cite{269581} provide gates as
units of abstraction that are ``wired'' together by the programmer to
generate a complete circuit.

The $\fedprot$ language is low-level and does not include abstractions
for defining composable elements. So in this Section we introduce the
$\metaprot$ language that includes structured data and function
definitions for defining composable protocol elements at a higher
level of abstraction.  The $\metaprot$ language is a
\emph{metalanguage}, where $\fedprot$ protocols are the residuum of
computation. In addition to these declarative benefits of $\metaprot$,
component definitions support compositional verification of larger
protocols as we will discuss with examples in Section
\ref{section-examples}.

\metaprotsyntaxfig

\subsection{Syntax and Semantics}

The syntax of $\metaprot$ is defined in Figure
\ref{fig-metaprotsyntax}.  It includes a syntax of values $\mv$ that
include client ids $\cid$, identifier strings $w$, expressions $\be$
in field $\mathbb{F}_p$, record values, and $\minicat$ variables
$x$. $\metaprot$ expression forms allow dynamic construction of these
values. $\metaprot$ \emph{instruction} forms allow dynamic
construction of $\minicat$ protocols $\prog$, incorporating expression
evaluation. The syntax also supports definitions of functions that
compute values $\mv$ and, as a distinct form, functions that compute
protocols.  Formally, we consider a complete metaprogram to include
both a codebase and a ``main'' program that uses the codebase.
\begin{definition}
A \emph{codebase} $\codebase$ is a list of function 
declarations. We write $ \codebase(f) = y_1,\ldots,y_n,\ e$
if $f(y_1,\ldots,y_n) \{ e \} \in \codebase$, and we
write  $ \codebase(f) = y_1,\ldots,y_n,\ \instr$
if $f(y_1,\ldots,y_n) \{ \instr \} \in \codebase$.
%A \emph{metaprogram}, aka \emph{metaprotocol}  is a pair of a 
%codebase and expression $\codebase, e$. We may omit
%$\codebase$ if it is clear from context.  
\end{definition}

We define a big-step evaluation relation $\redx$ in Figures
\ref{fig-metaprotexprsemantics} and \ref{fig-metaprotinstrsemantics}
for expressions and instructions, respectively.  In this definition we
write $e[\mv/y]$ and $\instr[\mv/y]$ to denote the substitution of
$\mv$ for free occurrences of $y$ in $e$ or $\instr$ respectively. The
rules are mostly standard. Note that we do not include an evaluation
rule for the form $\eqcast{e}{e}{\notg{\phi}}$ which only has
relevance for the type system and which we assume is erased from
programs prior to evaluation. We defer discussion of this form,
as well as the syntactic category $\notg{\phi}$, to the next
Section.

\metaprotexprsemanticsfig

\metaprotinstrsemanticsfig

\subsection{Dependent Hoare Type Theory}

Our first goal with a $\metaprot$ type theory is to define an
algorithmic system that is sound for both confidentiality and
integrity typings as defined in Sections \ref{section-cpj} and
\ref{section-ipj}. Returning to the example in Section
\ref{section-cpj}, note that the key equivalence of the $\ttt{mux}$
expression with a one-time-pad encryption we observe there
is not trivial. To ensure that this sort of equivalence
can be picked up by the type system, we introduce a
an annotation form that allows the programmer to provide
the needed hint.

Consider the following $\ttt{encode}$ function definition that
generalizes this encoding for any identifier $y$ and sender, receiver
pair $s,r$:
$$
\begin{array}{l}
\ttt{encode}(y,s,r) \{\\
\quad \xassign{\mx{y}{r}}{\mux{\secret{y}}{\neg\flip{y}}{\flip{y}}}{s};\\
\quad \eqcast{\mx{y}{r}}{\neg\sx{y}{s} \fplus \rx{y}{s}}\\
\}
\end{array}
$$

Furthermore, we want to minimize the amount of SMT solving needed for
type verification.  Our approach is to support pre- and post-condition
annotations on $\metaprot$ functions that can be independently
verified, allowing significant reduction of SMT overhead in
whole-program analysis.

\subsubsection{$\minicat$ expression type algorithm}

\atjfig

\begin{lemma}
  If $\atj{\toeq{\phi}}{R}{\ty}$ then $\eqj{R}{\eqs}{\phi}{\ty}$ for any $\eqs$.
\end{lemma}


$$
\cmd ::= \cdots \mid \eqcast{\mx{e}{e}}{\notg{\phi}}
$$

$$
\begin{array}{rcl}
  \notg{\phi} &::=& e \mid \notg{\phi} \fplus \notg{\phi} \mid \notg{\phi} \fminus \notg{\phi} \mid \notg{\phi} \ftimes \notg{\phi} \\
  \notg{\eqs} &::=& \notg{\phi} \eop \notg{\phi} \mid \notg{\eqs} \wedge \notg{\eqs} \\
  \notg{t} &::=& e \mid \cty{e}{\notg{\ty}} \\
  \notg{\ty} & \in & 2^{\notg{t}}\\
  \notg{\Gamma} &::=& \varnothing \mid \notg\Gamma; e : \notg{\ty}\\
  \notg\Delta &::=& \varnothing \mid \notg\Delta; e : \ity{e}{\notg{V}}\\
  \notg{X} &\in& 2^{e}
\end{array}
$$

\begin{mathpar}
  \inferrule
      {\notg{\phi_1} \redx \phi_1 \\ \notg{\phi_2} \redx \phi_2}     
      {\notg{\phi_1} \ftimes \notg{\phi_2} \redx \phi_1 \ftimes \phi_2}

  \inferrule
      {\notg{\phi_1} \redx \phi_1 \\ \notg{\phi_2} \redx \phi_2}
      {\notg{\phi_1} \eop \notg{\phi_2} \redx \phi_1 \eop \phi_2}

  \inferrule
      {\notg{\eqs_1} \redx \eqs_1 \\ \notg{\eqs_2} \redx \eqs_2 }
      {\notg{\eqs_1} \wedge \notg{\eqs_2} \redx \eqs_1 \wedge \eqs_2}
\end{mathpar}

\begin{mathpar}
  \inferrule
      {e \redx x \\ \notg{\ty} \redx \ty}
      {\cty{e}{\notg{\ty}} \redx \cty{x}{\ty}}
      
  \inferrule
      {\notg{t_1} \redx t_1 \\ \cdots \\ \notg{t_n} \redx t_n}
      {\setit{\notg{t_1},\ldots,\notg{t_n}} \redx \setit{ t_1,\ldots,t_n }}

  \inferrule
      {\notg{\Gamma} \redx \Gamma \\ e \redx x \\ \notg{\ty} \redx \ty }
      {\notg{\Gamma}; e : \notg{\ty} \redx \Gamma; x : \ty }

  \inferrule
      {\notg{\Delta} \redx \Delta \\ e_1 \redx x  \\ e_2 \redx \cid \\ \notg{V} \redx V}
      {\notg{\Delta}; e_1 : \ity{e_2}{\notg{V}} \redx \Delta; x : \ity{\cid}{V} }

  \inferrule
      {\notg{\eqs_1} \redx \eqs_1 \\ \notg{\Gamma} \redx \\ \notg{R} \redx R
        \\ \notg{\Delta} \redx \Delta \\ \notg{\eqs_2} \redx \eqs_2}
      {\hty{\notg{\eqs_1}}{\notg{\Gamma}}{\notg{R}}{\notg{\Delta}}{\notg{\eqs_2}} \redx
        \hty{\eqs_1}{\Gamma}{R}{\Delta}{\eqs_2}}
\end{mathpar}



\begin{mathpar}
  \inferrule[Mesg]
            {e_1 \redx x \\ e_2 \redx \be \\ e_3 \redx \cid \\ \atj{\toeq{\elab{\be}{\cid}}}{R_2}{\ty} \\
              \itj{\cid}{\be}{V}}
            {\mtj{\xassign{e_1}{e_2}{e_3}}{\eqs}{(x:\ty)}{R_1;R_2}{(x : \ity{\cid}{V})}{\eqs \wedge x \eop \toeq{\elab{\be}{\cid}}}}

  \inferrule[Encode]
            {e_1 \redx w \\ e_2 \redx \cid \\ \notg{\phi} \redx \phi \\
              \eqs \models \toeq{\elab{\be}{\cid}} \eop \phi\\
              \atj{\phi}{R}{\ty}}
            {\mtj{\eqcast{\mx{e_1}{e_2}}{\notg{\phi}}}{\eqs}{(\mx{w}{\cid}:\ty)}{R}{\varnothing}{\eqs}}

  \inferrule[App]
            {\tsig(f) = \dht{y_1,\ldots,y_n}{\notg{\eqs_1}}{\notg{\Gamma}}{\notg{R}}{\notg{\Delta}}{\notg{\eqs_2}} \\
              e_1 \redx \mv_1\ \cdots\ e_n \redx \mv_n \\
              \subn = [\mv_1/y_1]\cdots[\mv_n/y_n] \\
              \subn(\hty{\notg{\eqs_1}}{\notg{\Gamma}}{\notg{R}}{\notg{\Delta}}{\notg{\eqs_2}}) \redx
                    \hty{\eqs_1}{\Gamma}{R}{\Delta}{\eqs_2} \\
              \eqs \models \eqs_1}
            {\mtj{f(e_1,\ldots,e_n)}{\eqs}{\Gamma}{R}{\Delta}{\eqs \wedge \eqs_2}}

  \inferrule[Seq]          
            {\mtj{\prog_1}{\eqs_1}{\Gamma_1}{R_1}{\Delta_1}{\eqs_2} \\
             \mtj{\prog_2}{\eqs_2}{\Gamma_2}{R_2}{\Delta_2}{\eqs_3}}
            {\mtj{\prog_1;\prog_2}{\eqs_1}{\Gamma_1;\Gamma_2}{R_1;R_2}{\Delta_1;\Delta_2}{\eqs_3}}
\end{mathpar}

\begin{mathpar}
  \inferrule[Sig]
            {\codebase(f) = y_1,\ldots,y_n, \instr \\
              \subn = [\mv_1/y_1]\cdots[\mv_n/y_n] \\
              \subn(\hty{\notg{\eqs_1}}{\notg{\Gamma}}{\notg{R}}{\notg{\Delta}}{\notg{\eqs_2}}) \redx
                    \hty{\eqs_1}{\Gamma}{R}{\Delta}{\eqs_2} \\
              \mtj{\subn(\instr)}{\eqs_1}{\Gamma}{R}{\Delta}{\eqs} \\
              \eqs \models \eqs_2}
            {f : \dht{y_1,\ldots,y_n}{\notg{\eqs_1}}{\notg{\Gamma}}{\notg{R}}{\notg{\Delta}}{\notg{\eqs_2}}}
\end{mathpar}

\begin{definition}
  $\tsig$ is \emph{verified} iff $f : \tsig(f)$ is valid for all $f \in \dom(\tsig)$.
\end{definition}

\begin{theorem}
  Given preprocessing predicate $\eqspre$, program $\instr$, and verified $\tsig$, if
  the judgement $\mtj{\instr}{\eqspre}{\Gamma}{R}{\Delta}{\eqs}$ is derivable then
  $\instr \redx \prog$ and:
  \begin{enumerate}
  \item $\cpj{R}{\eqs}{\eqspre \wedge \toeq{\prog}}{\Gamma}$ is valid.
  \item $\ipj{\eqs}{\prog}{\Delta}$ is valid.
  \end{enumerate}
\end{theorem}
