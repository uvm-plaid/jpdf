\section{Introduction}

Data privacy is a critical concern in distributed applications
including privacy-preserving machine learning
\cite{li2021privacy,knott2021crypten,koch2020privacy,liu2020privacy}
and blockchains
\cite{ishai2009zero,lu2019honeybadgermpc,gao2022symmeproof,tomaz2020preserving}.
Secure multiparty computation (MPC) is a key software-based enabling
technology for these applications. MPC protocols provide both confidentiality
and integrity properties, formulated as real/ideal (aka simulator)
security and universal composability (UC) \cite{evans2018pragmatic},
with well-studied manual proof methods \cite{Lindell2017}.  MPC
security semantics has also been reformulated as
\emph{hyperproperties}
\cite{8429300,10.1145/3453483.3454074,skalka-near-ppdp24} to support
automated language-based proof methods. Recent research in the SMT
community has developed theories of finite fields \cite{SMFF} with
clear relevance to verification of cryptographic schemes that rely on
field arithmetic. The goal of this paper is to combine these two
approaches in a decidable type system for verifying correctness and
security properties of MPC protocols.

Our focus is on low-level MPC protocols. Previous high-level
MPC-enabled languages such as Wysteria \cite{rastogi2014wysteria} and
Viaduct \cite{10.1145/3453483.3454074} are designed to provide
effective programming of full applications, and incorporate
sophisticated compilation techniques such as orchestration
\cite{viaduct-UC} to guarantee high-level security properties. But
these approaches rely on \emph{libraries} of low-level MPC protocols,
such as binary and arithmetic circuits. These low-level protocols are
probabilistic, encapsulate abstractions such as secret sharing and
semi-homomorphic encryption, and are complementary to high-level
language design.

\subsection{Overview and Contributions}

Our work is based on the $\metaprot$/$\minicat$ framework developed in
previous work \cite{skalka-near-ppdp24}. The prior work developed the 
language and semantics, plus two mechanisms for verifying security: an
automated approach for protocols in $\mathbb{F}_2$ (the binary field),
based on enumerating all possible executions of the protocol, that scales 
only to very small protocols; and a manual approach based on a program logic.
%
In Section \ref{section-lang} we recall the $\minicat$ language model and key definitions. In
Section \ref{section-model} we recall formulations of program distributions,
and key probabilistic hyperproperties of MPC in $\minicat$. 

Together, our contributions \emph{automate} the verification of security
properties in $\metaprot$/$\minicat$. In this work, we build on~\cite{skalka-near-ppdp24}
by developing automatic verification approaches that work for arbitrary
finite fields and scale to large protocols. To accomplish this, we develop new
type systems for compositional verification, and use SMT to automate the checks
that were accomplished by enumeration in prior work. Specifically, we make the
following contributions:
%
\begin{enumerate}[i.]
\item In Section \ref{section-smt}, an interpretation of protocols in
  $\minicat$ as SMT constraints (Theorem \ref{theorem-toeq}) and
  methodology for verifying correctness properties.
\item In Section \ref{section-cpj}, a confidentiality type system for
  $\minicat$ protocols with a soundness property guaranteeing that
  information is released only through explicit declassifications
  (Theorem \ref{theorem-cpj}).
\item In Section \ref{section-ipj}, an integrity type system for
  $\minicat$ protocols with a soundness property guaranteeing
  robustness in malicious settings (Theorem \ref{theorem-ipj}).
\item In Section \ref{section-metalang}, a dependent Hoare type system and
  algorithm for the $\metaprot$ language that is sound for the
  confidentiality and integrity type systems (Theorem
  \ref{theorem-mtj}), with independently verified, compositional $\Pi$
  types for functions.
\end{enumerate}
In all cases, our type systems are boosted by verification mechanisms
provided by Satisfiability Modulo Finite Fields \cite{SMFF}. In
Section \ref{section-examples} we develop and discuss extended example
applications of our type analyses to real protocols including
the Goldreich-Micali-Wigderson (GMW), Bendlin-Damgard-Orland-Zakarias
(BDOZ), and Yao's Garbled Circuits (YGC) protocols.

\subsection{Related Work}
\label{section-related-work}

The goal of the $\minicat/\metaprot$ framework is to automate
correctness and security verification of low-level protocols.  Other
prior work has considered automated verification of high-level
protocols and manual verification of low-level protocols.
We summarize this comparison in Figure \ref{fig-comp-wrap}.

% We consider related systems in several
% dimensions, including whether they are aimed at low-level design with
% probabilistic features, whether they support reasoning about
% conditional probabilities which are central to real/ideal security,
% whether they consider MPC security through the lens of
% hyperproperties, and whether they consider passive and/or malicious
% security models.

The most closely related work is \cite{skalka-near-ppdp24}, where
foundations for $\minicat/\metaprot$ are introduced including a
verification algorithm. However, this algorithm only works for
$\mathbb{F}_2$-- e.g., binary circuits-- and does not scale due to
exponential complexity, necessitating semi-automated proof
techniques. In contrast, our type systems scale to arbitrary prime
fields and whole program analysis for larger circuits (protocols). So
our work is a significant advancement of $\minicat/\metaprot$.  Our
work is also closely related to Satisfiability Modulo Finite Fields
\cite{SMFF} but only in the sense that we apply their method as we
discuss in Section \ref{section-smt} and subsequently.

Other low-level languages with probabilistic features for privacy
preserving protocols have been proposed. The
$\lambda_{\mathrm{obliv}}$ language \cite{darais2019language} uses a
type system to automatically enforce so-called probabilistic trace
obliviousness.  But similar to previous work on oblivious data
structures \cite{10.1145/3498713}, obliviousness is related to pure
noninterference, not the conditional form related to passive MPC
security (Definition \ref{definition-NIMO}). The Haskell-based security
type system in \cite{6266151} enforces a version of noninterference
that is sound for passive security, but does not consider malicious
security. And properties of real/ideal passive and malicious security
for a probabilistic language have been (manually) formulated in
EasyCrypt \cite{8429300}.

\compwrapfig

Several high-level languages have been developed for writing MPC
applications. Previous work on analysis for the
SecreC language \cite{almeida2018enforcing,10.1145/2637113.2637119} is
concerned with properties of complex MPC circuits, in particular a
user-friendly specification and automated enforcement of
declassification bounds in programs that use MPC in subprograms. The
Wys$^\star$ language \cite{wysstar}, based on Wysteria
\cite{rastogi2014wysteria}, has similar goals and includes a
trace-based semantics for reasoning about the interactions of MPC
protocols. Their compiler also guarantees that underlying
multi-threaded protocols enforce the single-threaded source language
semantics. These two lines of work were focused on passive
security. The Viaduct language \cite{10.1145/3453483.3454074} has a
well-developed information flow type system that automatically
enforces both confidentiality and integrity through hyperproperties
such as robust declassification, in addition to rigorous compilation
guarantees through orchestration \cite{viaduct-UC}. However, these
high level languages lack probabilistic features and other
abstractions of low-level protocols, the implementation and security
of which are typically assumed as a selection of library components.

Program logics for probabilistic languages and specifically reasoning
about properties such as joint probabilistic independence is also
important related work. Probabilistic Separation Logic (PSL)
\cite{barthe2019probabilistic} develops a logical framework for
reasoning about probabilistic independence (aka separation) in
programs, and they consider several (hyper)properties, such as perfect
secrecy of one-time-pads and indistinguishability in secret sharing,
that are critical to MPC. Lilac \cite{li2023lilac} extends this line
of work with formalisms for conditional independence which is 
also particularly important for MPC \cite{skalka-near-ppdp24}.

