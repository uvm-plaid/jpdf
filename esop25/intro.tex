\section{Introduction}

Secure Multi-Party Computation (MPC) protocols support data privacy in
important modern, distributed applications such as privacy-preserving
machine learning
\cite{li2021privacy,knott2021crypten,koch2020privacy,liu2020privacy}
and Zero-Knowledge proofs in blockchains
\cite{ishai2009zero,lu2019honeybadgermpc,gao2022symmeproof,tomaz2020preserving}.
The security semantics of MPC include both confidentiality and
integrity properties incorporated into models such as real/ideal (aka
simulator) security and universal composability (UC), developed
primarily by the cryptography community \cite{evans2018pragmatic}.
Related proof methods are well-studied \cite{Lindell2017} but mostly
manual. To support automated reasoning, recent work has reformulated
MPC security semantics as \emph{hyperproperties}
\cite{8429300,10.1145/3453483.3454074,skalka-near-ppdp24}.
Separately, recent research in the SMT community has developed
theories of finite fields \cite{SMFF} with clear relevance to
verification of cryptographic schemes that rely on field
arithmetic. The goal of this paper is to combine these two approaches
in an automated type system for verifying correctness and security
properties of MPC protocols.

The distinction between high- and low-level languages for MPC is
important. High-level languages such as Wysteria
\cite{rastogi2014wysteria} and Viaduct \cite{10.1145/3453483.3454074}
are designed to provide effective programming of full
applications. These language designs incorporate sophisticated
verified compilation techniques such as orchestration
\cite{viaduct-UC} to guarantee high-level security properties, and
they rely on \emph{libraries} of low-level MPC protocols, such as
binary and arithmetic circuits. These low-level protocols encapsulate
abstractions such as secret sharing and semi-homomorphic encryption,
and must be verified by hand. So, low-level MPC
programming and protocol verification remains a distinct challenge and both
critical to the general challenge of PL design for MPC and
complementary to high-level language design.

\subsection{Overview and Contributions}

\begin{enumerate}
\item A \textbf{low-level language} for defining MPC primitives
  (Section \ref{section-lang}) with an associated \textbf{metalanguage}
  to ease programming (Section \ref{section-metalang})\footnote{By metalanguage
  we mean a multi-stage aka metaprogramming language where code is a value, as
  in, e.g., MetaML \cite{TAHA2000211}.}. 
\item A \textbf{fully-automated verification method} for low-level MPC
  primitives in $\mathbb{F}_2$ (Section \ref{section-bruteforce}).
\item A \textbf{partially-automated verification method} for
  MPC protocols, which leverages automated proofs for
  low-level primitives (Section \ref{section-example-gmw}).
\end{enumerate}

\subsection{Related Work}
\label{section-related-work}

\cnote{This sections needs updating with, including with
  \cite{skalka-near-ppdp24,SMFF}}.
Our main focus is on PL design and automated and semi-automated
reasoning about security properties of low-level MPC protocols. Prior
work has considered \textbf{automated} verification of
\textbf{high-level} protocols and \textbf{manual} verification of
\textbf{low-level} protocols---but none offers the combination of
automation and low-level support we consider.
%
We summarize this comparison in Figure
\ref{fig-comp-wrap}, with the caveat that works vary in the degree of
development in each dimension.

% We consider related systems in several
% dimensions, including whether they are aimed at low-level design with
% probabilistic features, whether they support reasoning about
% conditional probabilities which are central to real/ideal security,
% whether they consider MPC security through the lens of
% hyperproperties, and whether they consider passive and/or malicious
% security models.

\compwrapfig

As mentioned above, several high-level languages have been developed
for writing MPC applications, and frequently exploit the connection
between hyperproperties and MPC security. Previous work on analysis
for the SecreC language
\cite{almeida2018enforcing,10.1145/2637113.2637119} is concerned with
properties of complex MPC circuits, in particular a user-friendly
specification and automated enforcement of declassification bounds in
programs that use MPC in subprograms. The Wys$^\star$ language
\cite{wysstar}, based on Wysteria \cite{rastogi2014wysteria}, has
similar goals and includes a trace-based semantics for reasoning about
the interactions of MPC protocols. Their compiler also guarantees that
underlying multi-threaded protocols enforce the single-threaded source
language semantics. These two lines of work were focused on passive
security. The Viaduct language
\cite{10.1145/3453483.3454074} has a well-developed
information flow type system that automatically enforces both
confidentiality and integrity through hyperproperties such as robust
declassification, in addition to rigorous compilation guarantees
through orchestration \cite{viaduct-UC}. However, these high level
languages lack probabilistic features and other abstractions of
low-level protocols, the implementation and security of which are
typically assumed as a selection of library components.

Various related low-level languages with probabilistic features have
also been developed. The $\lambda_{\mathrm{obliv}}$ language
\cite{darais2019language} uses a type system to automatically
enforce so-called probabilistic trace obliviousness.  But similar to
previous work on oblivious data structures \cite{10.1145/3498713},
obliviousness is related to pure noninterference, not the relaxed form
related to passive MPC security. The Haskell-based security type
system in \cite{6266151} enforces a version of noninterference that is
sound for passive security, but does not verify the correctness of
declassifications and does not consider malicious security. And
properties of real/ideal passive and malicious security for a
probabilistic language have been formulated in EasyCrypt
\cite{8429300}-- though their proof methods, while mechanized, are
fully manual, and their formulation of malicious security is not as
clearly related to robust declassification as is the one we present in
Section \ref{section-hyper}.

Program logics for probabilistic languages and specifically reasoning
about properties such as joint probabilistic independence is also
important related work. Probabilistic Separation Logic (PSL)
\cite{barthe2019probabilistic} develops a logical framework for
reasoning about probabilistic independence (aka separation) in
programs, and they consider several (hyper)properties, such as perfect
secrecy of one-time-pads and indistinguishability in secret sharing,
that are critical to MPC. However, their methods are manual, and
don't include conditional independence (separation). This
latter issue has been addressed in Lilac \cite{li2023lilac}. The
application of Lilac-style reasoning to MPC protocols has not
previously been explored, as we do in Section
\ref{section-example-gmw}.

Our work also shares many ideas with probabilistic programming
languages designed to perform (exact or approximate) statistical
inference~\cite{holtzen2020scaling,carpenter2017stan,wood2014new,bingham2019pyro,albarghouthi2017fairsquare,de2007problog,pfeffer2009figaro,saad2021sppl}. Our setting, however, requires
verifying properties beyond inference, including conditional
statistical independence. Recent work by Li et al.~\cite{li2023lilac} proposes a
manual approach for proving such properties, but does not provide
automation.

