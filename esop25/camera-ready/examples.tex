\section{Extended Example: GMW}
\label{section-examples}

In Section \ref{section-smt} we gave the example of 3-party additive
sharing for any field $\fieldp{p}$. In Sections \ref{section-ipj} and
\ref{section-metalangty} we discussed the example of YGC input encoding.
Now we consider examples that illustrate the versatility of the
$\TirName{Encode}$ typing rule for confidentiality, the use of
postconditions for correctness, and the use of both pre-~and
postconditions for integrity.

To evaluate our system especially regarding performance, using the
Satisfiability Modulo Fields theory in cvc5 on an Apple M1 processor,
we have verified correctness of additive sharing protocols in fields
up to size $\fieldp{2^{31} - 1}$ (an approximation of 32-bit integers)
which takes $\sim$7ms to compute for 3 parties and not significantly more
than for, e.g., $\fieldp{2}$. We have also verified the semi-homomorphic
encryption properties leveraged in the BDOZ circuit library functions
(Figures \ref{fig-bdozsum} and \ref{fig-bdozmult}) in fields up to
size $\fieldp{2^{31} - 1}$ which take $\sim$8ms for the property used for
the postcondition of the $\ttt{mult}$ gate. This is categorically
better performance than can be achieved using previous brute-force
methods in $\fieldp{2}$ \cite{skalka-near-ppdp24}.

\begin{fpfig}[t]{2-Party GMW circuit library with And gate (T), and type annotations (B).}{fig-gmw}
{\footnotesize
  \begin{verbatimtab}
    encodegmw(in1,in2) {
      v[1, in1++"out"] := s[1,in1] xor flip[1,in1];
      v[2, in1++"out"] := flip[1,in1];
      v[1, in2++"out"] := s[2,in2] xor flip[2,in2];
      v[2, in2++"out"] := flip[2,in2]
      { shares1 = { c1 = v[1, in1++"out"]; c2 = v[2, in1++"out"] };
        shares2 = { c1 = v[1, in2++"out"]; c2 = v[2, in2++"out"]} } 
    }
    
    andtablegmw(b1, b2) {
      let r11 = (b1 xor true) and (b2 xor true) in
      let r10 = (b1 xor true) and (b2 xor false) in
      let r01 = (b1 xor false) and (b2 xor true) in
      let r00 = (bl xor false) and (b2 xor false) in
      { v1 = r11; v2 = r10; v3 = r01; v4 = r00 }
    }
    
    andgmw(g, shares1, shares2) {
      let r = flip[1,g++".r"] in
      let table = andtablegmw(shares1.c1, shares2.c1) in
      let r11 =  r xor table.v1 in
      let r10 =  r xor table.v2 in
      let r01 =  r xor table.v3 in
      let r00 =  r xor table.v4 in
        v[2,g++"out"] := OT4(shares1.c2, shares2.c2, r11, r10, r01, r00);
      v[1,g++"out"] := r;
      { c1 = v[1,g++"out"]; c2 = v[2,g++"out"]}
    }
    
    decodegmw(shares) { v[0,"output"] := shares.c1 xor shares.c2 }   \end{verbatimtab}
}
\end{fpfig}


In the 2-party GMW protocol \cite{evans2018pragmatic}, another garbled
binary circuit protocol, values are encrypted in a manner similar to
BDOZ as described in Section \ref{section-ipj}. In our definition of
GMW we use the convention that shared values $\macgv{\mesg{w}}$ are
identified by strings $w$ and encoded as shares $\mx{w}{1}$ and
$\mx{w}{2}$.  As for YGC, ciphertypes reflect the confidentiality of
GMW input encodings as defined in the $\ttt{encodegmw}$ function in
Figure \ref{fig-gmw}. No programmer annotation is needed given the
syntax of the function body. More interestingly, the $\ttt{andgmw}$
gate definition shows how a programmer hint can express the relevant
confidentiality property of the output share $\mx{z}{2}$ on client 2,
using the $\ttt{as}$ casting. The non-trivial equivalence can be
verified by SMT once during verification, and subsequently
confidentiality is expressed in the dependent type of the function (as
for $\ttt{encode}$).

Note also that the $\ttt{andgmw}$ function is decorated with a
postcondition that expresses the correctness property of the
function. Although not strictly necessary for confidentiality, this
annotation can at least help eliminate bugs, and also can be used
compositionally for whole-program correctness properties. As for
confidentiality hints, this postcondition needs to be verified only
once at the point of definition and subsequently is guaranteed to hold
for any application.
