\section{Extended Examples}
\label{section-examples}

In Section \ref{section-smt} we gave the example of 3-party additive
sharing for any field $\fieldp{p}$. In Sections \ref{section-ipj} and
\ref{section-metalangty} we discussed the example of YGC input encoding.
Now we consider examples that illustrate the versatility of the
$\TirName{Encode}$ typing rule for confidentiality, the use of
postconditions for correctness, and the use of both pre-~and
postconditions for integrity.

Using the Satisfiability Modulo Fields theory in cvc5 on an Apple M1
processor, we have verified correctness of additive sharing protocols
in fields up to size $\fieldp{2^{31} - 1}$ (an approximation of 32-bit
integers) which takes ~7ms to compute for 3 parties and not
significantly more than for, e.g., $\fieldp{2}$. We have also verified
the semi-homorphic encryption properties leveraged in the BDOZ circuit
library functions (Figures \ref{fig-bdozsum} and \ref{fig-bdozmult}) in
fields up to size $\fieldp{2^{31} - 1}$ which take ~8ms for the
property used for the postcondition of the $\ttt{mult}$ gate. This is
categorically better performance than can be achieved using previous
brute-force methods in $\fieldp{2}$ \cite{skalka-near-ppdp24}.

\subsection{Confidentiality Examples}

\begin{fpfig}[t]{2-Party GMW circuit library with And gate (T), and type annotations (B).}{fig-gmw}
{\footnotesize
  \begin{verbatimtab}
    encodegmw(in1,in2) {
      v[1, in1++"out"] := s[1,in1] xor flip[1,in1];
      v[2, in1++"out"] := flip[1,in1];
      v[1, in2++"out"] := s[2,in2] xor flip[2,in2];
      v[2, in2++"out"] := flip[2,in2]
      { shares1 = { c1 = v[1, in1++"out"]; c2 = v[2, in1++"out"] };
        shares2 = { c1 = v[1, in2++"out"]; c2 = v[2, in2++"out"]} } 
    }
    
    andtablegmw(b1, b2) {
      let r11 = (b1 xor true) and (b2 xor true) in
      let r10 = (b1 xor true) and (b2 xor false) in
      let r01 = (b1 xor false) and (b2 xor true) in
      let r00 = (bl xor false) and (b2 xor false) in
      { v1 = r11; v2 = r10; v3 = r01; v4 = r00 }
    }
    
    andgmw(g, shares1, shares2) {
      let r = flip[1,g++".r"] in
      let table = andtablegmw(shares1.c1, shares2.c1) in
      let r11 =  r xor table.v1 in
      let r10 =  r xor table.v2 in
      let r01 =  r xor table.v3 in
      let r00 =  r xor table.v4 in
        v[2,g++"out"] := OT4(shares1.c2, shares2.c2, r11, r10, r01, r00);
      v[1,g++"out"] := r;
      { c1 = v[1,g++"out"]; c2 = v[2,g++"out"]}
    }
    
    decodegmw(shares) { v[0,"output"] := shares.c1 xor shares.c2 }   \end{verbatimtab}
}
\end{fpfig}


In the 2-party GMW protocol \cite{evans2018pragmatic}, another garbled
binary circuit protocol, values are encrypted in a manner similar to
BDOZ as described in Section \ref{section-ipj}. In our definition of
GMW we use the convention that shared values $\macgv{\mesg{w}}$ are
identified by strings $w$ and encoded as shares $\mx{w}{1}$ and
$\mx{w}{2}$.  As for YGC, ciphertypes reflect the confidentiality of
GMW input encodings as defined in the $\ttt{encodegmw}$ function in
Figure \ref{fig-gmw}. No programmer annotation is needed given the
syntax of the function body. More interestingly, the $\ttt{andgmw}$
gate definition shows how a programmer hint can express the relevant
confidentiality property of the output share $\mx{z}{2}$ on client
2. The non-trivial equivalence can be verified by SMT once during
verification, and subsequently confidentiality is expressed in the
dependent type of the function (as for $\ttt{encode}$).

Note also that the $\ttt{andgmw}$ function is decorated with a
postcondition that expresses the correctness property of the
function. Although not strictly necessary for confidentiality, this
annotation can at least help eliminate bugs, and also can be used
compositionally for whole-program correctness properties. As for
confidentiality hints, this postcondition needs to be verified only
once at the point of definition and subsequently is guaranteed to hold
for any application.

\subsection{Integrity Examples}

\begin{fpfig}[t]{2-party BDOZ protocol library.}{fig-bdoz}
{\footnotesize{
\begin{verbatimtab}
auth(s,m,k,i) { assert(m == k + (m["delta"] * s))@i; }
  
sum_she(z,x,y,i) {
  m[z++"s"]@i := (m[x++"s"] + m[y++"s"])@i;
  m[z++"m"]@i := (m[x++"m"] + m[y++"m"])@i;
  m[z++"k"]@i := (m[x++"k"] + m[y++"k"])@i
}

open(x,i1,i2){
  m[x++"exts"]@i1 := m[x++"s"]@i2;
  m[x++"extm"]@i1 := m[x++"m"]@i2;
  auth(m[x++"exts"], m[x++"extm"], m[x++"k"], i1);
  m[x]@i1 := (m[x++"exts"] + m[x++"s"])@i1
}
\end{verbatimtab}
}}
\end{fpfig}


%%%%%%%%% OLD VERSION BELOW
\begin{comment}

\begin{fpfig}[t]{2-Party BDOZ Protocol Library.}{fig-bdoz}
{\footnotesize{
  \begin{verbatimtab}
    macsum(s1,s2)
    { { share = s1.share + s2.share; mac = s1.mac + s2.mac } }
    
    maccsum(s,c)
    { { share = s.share + c; mac = s.mac + c } }
    
    macctimes(s,c)
    { { share = s.share * c; mac = s.mac * c } }
    
    macshare(w) { {  share = m[w]; mac = m[w++"mac"] } }

    mack(w) { m[w++"k"] }
    
    auth(s,m,k,i) { assert(m = k + m["delta"] * s)@i }
    
    secopen(w1,w2,w3,i1,i2)
    {
      let locsum =  macsum(macshare(w1), macshare(w2)) in
      m[w3++"s"]@i1 := (locsum.share)@i2;
      m[w3++"smac"]@i1 := (locsum.mac)@i2;
      auth(m[w3++"s"],m[w3++"smac"],mack(w1) + mack(w2),i1);
      m[w3]@i1 := (m[w3++"s"] + (locsum.share))@i1
    }

    secreveal(s,k,w,i1,i2)
    {
      p[w] = s.share@i2;
      p[w++"mac"] = s.mac@i2;
      auth(p[w],p[w++"mac"],k,i1)    
    } \end{verbatimtab}
}}
\end{fpfig}

\end{comment}
    


Returning to the example of malicious-secure 2-party BDOZ arithmetic
circuits begun in Section \ref{section-ipj}, in Figure \ref{fig-bdozsum} we
define $\ttt{sum}$ and $\ttt{open}$ functions. The latter implements
``secure opening''-- each party sends it's local share of a
global value $\macgv{\mesg{w}}$ along with its MAC to the other party,
which is authenticated via the $\macbdoz{w}$ check (in \ttt{\_open}),
and then each party reconstructs $\macgv{\mesg{w}}$. 

The $\ttt{sum}$ function implements an addition gate. Note that this
is done non-interactively-- each party just sums its local shares
of the two values. The pre- and postcondition annotations
on $\ttt{sum}$ express the additive homomorphism associated
with this encryption scheme-- the sum of MACs of the input
shares is a valid MAC for the output share on each client,
which can be checked using the sum of keys of the input shares. 

In BDOZ a pre-processing phase is assumed where initial input secrets
are shared along with their associated MACs and keys. This can be
expressed in $\eqspre$ for input secrets $\sx{\ttt{"x"}}{1}$ and
$\sx{\ttt{"y"}}{2}$, for example, which subsume the following
constraints on shares:
{\footnotesize$$
\begin{array}{l}
\mx{\ttt{"xs"}}{2} \eop \sx{\ttt{"x"}}{1} \fminus \rx{\ttt{"x"}}{1} \wedge 
\mx{\ttt{"xs"}}{1} \eop \rx{\ttt{"x"}}{1} \wedge \\
\mx{\ttt{"ys"}}{1} \eop \sx{\ttt{"y"}}{2} \fminus \rx{\ttt{"y"}}{2} \wedge 
\mx{\ttt{"ys"}}{2} \eop \rx{\ttt{"y"}}{2} 
\end{array}
$$}
and the following constraint on keys and MACs for authentication
of $\sx{\ttt{"x"}}{1}$ (and similarly for $\sx{\ttt{"y"}}{2}$):
{\footnotesize$$
\begin{array}{l}
\mx{\ttt{"delta"}}{1} \eop \rx{\ttt{"delta"}}{1} \wedge
\mx{\ttt{"xk"}}{1} \eop \rx{\ttt{"xk"}}{1} \wedge\\
\mx{\ttt{"xm"}}{2} \eop \mx{\ttt{"xk"}}{1} \fplus (\mx{\ttt{"delta"}}{1} * \mx{\ttt{"xs"}}{2})
\end{array}
$$}
Given this, a malicious secure opening of $\sx{\ttt{"x"}}{1} +
\sx{\ttt{"y"}}{2}$ would be obtained as
$\ttt{sum}(\ttt{"z"},\ttt{"x"},\ttt{"y"}); \ttt{open}(\ttt{"z"})$.

A common approach to implementing multiplication gates in a BDOZ
setting is to use \emph{Beaver Triples}. Recall that Beaver triples
are values $a,b,c$ with $a$ and $b$ chosen randomly and $c = a * b$,
unique per multiplication gate, that are secret shared with clients
during pre-processing.  In our encoding we assume the additional
convention that each gate output identifier distinguishes the Beaver
triple, so for example the share of the $a$ value for a gate
$\ttt{"g1"}$ is identified by $\ttt{"g1as"}$, etc., and $\eqspre$
subsumes the following constraints for the $a,b,c$ values of gate
$\ttt{"g1"}$.  {\footnotesize$$
\begin{array}{l}
\mx{\ttt{"g1as"}}{1} \eop \rx{\ttt{"g1as"}}{1}\ \wedge 
\mx{\ttt{"g1bs"}}{1} \eop \rx{\ttt{"g1bs"}}{1}\ \wedge\\
\mx{\ttt{"g1as"}}{2} \eop \rx{\ttt{"g1as"}}{2}\ \wedge 
\mx{\ttt{"g1bs"}}{2} \eop \rx{\ttt{"g1bs"}}{2}\ \wedge\\
\mx{\ttt{"g1cs"}}{1} \eop \\
\qquad ((\mx{\ttt{"g1as"}}{1} \fplus \mx{\ttt{"g1bs"}}{2})\ \ftimes\\
\qquad\phantom{(}(\mx{\ttt{"g1bs"}}{1} \fplus \mx{\ttt{"g1bs"}}{2})) \fminus \rx{\ttt{"g1cs"}}{2}\ \wedge\\
\mx{\ttt{"g1cs"}}{2} \eop \rx{\ttt{"g1cs"}}{2}
\end{array}
$$}
The definition of the $\ttt{mult}$ gate is presented in Figure
\ref{fig-bdozmult}, where we present just client 1's side of the
protocol $\ttt{\_mult1}$ (client 2's side is nearly symmetric but with
a small variation-- refer to
\cite{evans2018pragmatic,10.1007/978-3-030-68869-1_3} for more
details).  We could call this function on input secrets
$\sx{\ttt{"x"}}{1}$ and $\sx{\ttt{"y"}}{2}$ in gate $\ttt{"g1"}$ as
$\ttt{mult}(\ttt{"g1"},\ttt{"x"},\ttt{"y"})$, for example, or embed
this gate internally in a circuit.

As for $\ttt{\_sum}$, the postcondition of the $\ttt{\_mult1}$
function expresses the relevant semi-homorphic encryption property of
the resulting share, MAC, and key after gate execution-- specifically,
it preserves the BDOZ authentication property. The precondition that
input shares must possess this property is enforced by the two calls
to $\ttt{sum}$ within the body of $\ttt{mult}$ and need not be made
explicit. Finally, the postcondition of $\ttt{mult}$ expresses the
correctness property of the multiplication gate. In any case,
integrity of any circuit constructed from the $\ttt{sum}$ and
$\ttt{mult}$ library functions will require little MPC.

%%%%%%%%%%%%%%%%%%
%%% ARCHIVAL BELOW
%%%%%%%%%%%%%%%%%%
\begin{comment}
\begin{verbatimtab}
andtableygc(g,x,y)
{
   let table = (~r[g],~r[g],~r[g],r[g])
   in permute4(r[x],r[y],table)
}

m[x]@1 := s2(s[x],r[x],~r[x])@2;
m[x]@1 as s[x]@2 xor r[x]@2;

// m[x]@1 : { c(r[x]@2, { s[x]@2 }) } 

m[y]@1 := OT(s[y]@1,r[y],~r[y])@2;
m[y]@1 as s[y]@1 xor r[y]@2;

// m[y]@1 : { c(r[y]@2, { s[y]@1 }) } 
	      
m[ag]@1 := OT4(m[x]@1, m[y]@1, andtable(ag,r[x],r[y]))@2;
m[ag]@1 as  ~((r[x]@2 = m[x]@1) and (r[y]@2 = m[y]@1)) xor r[ag]@2;

// m[ag]@1 : { c(r[ag]@2, {r[x]@2, r[y]@2, m[x]@1,  m[y]@1} }

p[o] := OT2(m[ag]@1, perm2(r[ag],(false,true)))@2

// p[o] : { c(r[ag]@2, {r[x]@2, r[y]@2,  m[x]@1,  m[y]@1}), r[ag]@2  } 

out@1 := p[o]@1

// out@1 == s[x] and s[y]
\end{verbatimtab}

\begin{verbatimtab}
    encodegmw(in, i1, i2) {
      m[in]@i2 := (s[in] xor r[in])@i1;
      m[in]@i1 := r[in]@i1
    }
    
    andtablegmw(x, y, z) {
      let r11 = r[z] xor (m[x] xor true) and (m[y] xor true) in
      let r10 = r[z] xor (m[x] xor true) and (m[y] xor false) in
      let r01 = r[z] xor (m[x] xor false) and (m[y] xor true) in
      let r00 = r[z] xor (m[x] xor false) and (m[y] xor false) in
      { row1 = r11; row2 = r10; row3 = r01; row4 = r00 }
    }
    
    andgmw(z, x, y) {
      let table = andtablegmw(x,y,z) in
      m[z]@2 := OT4(m[x],m[y],table,2,1);
      m[z]@2 as ~((m[x]@1 xor m[x]@2) and (m[y]@1 xor m[y]@2)) xor r[z]@1);
      m[z]@1 := r[z]@1
    }

    // and gate correctness postcondition
    {} andgmw { m[z]@1 xor m[z]@2 == (m[x]@1 xor m[x]@2) and (m[y]@1 xor m[y]@2) }

    // and gate type
    andgmw :
     Pi z,x,y .
     {}
     { { r[z]@1 },
       (m[z]@1 : { r[z]@1 }; m[z]@2 : {c(r[z]@1, { m[x]@1, m[x]@2, m[y]@1, m[y]@2 })} ),
       m[z]@1 xor m[z]@2 == (m[x]@1 xor m[x]@2) and (m[y]@1 xor m[y]@2)}
    
    xorgmw(z, x, y) {
      m[z]@1 := (m[x] xor m[y])@1; m[z]@2 := (m[x] xor m[y])@2;
    }
    
    decodegmw(z) {
      p["1"] := m[z]@1; p["2"] := m[z]@2;
      out@1 := (p["1"] xor p["2"])@1;
      out@2 := (p["1"] xor p["2"])@2
    }

    prot() {
      encodegmw("x",2,1);
      encodegmw("y",2,1);
      encodegmw("z",1,2);
      andgmw("g1","x","z");
      xorgmw("g2","g1","y");
      decodegmw("g2")
    }

    {} prot { out@1 == (s["x"]@1 and s["z"]@2) xor s["y"]@1 }
\end{verbatimtab}

\subsection{Integrity Examples}

\begin{verbatimtab}
  secopen(w1,w2,w3,i1,i2) {
      pre(m[w1++"m"]@i2 == m[w1++"k"]@i1 + (m["delta"]@i1 * m[w1++"s"]@i2 /\
          m[w1++"m"]@i2 == m[w1++"k"]@i1 + (m["delta"]@i1 * m[w1++"s"]@i2));
      let locsum =  macsum(macshare(w1), macshare(w2)) in
      m[w3++"s"]@i1 := (locsum.share)@i2;
      m[w3++"m"]@i1 := (locsum.mac)@i2;
      auth(m[w3++"s"],m[w3++"m"],mack(w1) + mack(w2),i1);
      m[w3]@i1 := (m[w3++"s"] + (locsum.share))@i1
  }

  
  _open(x,i1,i2){
    m[x++"exts"]@i1 := m[x++"s"]@i2;
    m[x++"extm"]@i1 := m[x++"m"]@i2;
    assert(m[x++"extm"] == m[x++"k"] + (m["delta"] * m[x++"exts"]));
    m[x]@i1 := (m[x++"exts"] + m[x++"s"])@i2
  }`
  
  _sum(z, x, y,i1,i2) {
      pre(m[x++"m"]@i2 == m[x++"k"]@i1 + (m["delta"]@i1 * m[x++"s"]@i2 /\
          m[y++"m"]@i2 == m[y++"k"]@i1 + (m["delta"]@i1 * m[y++"s"]@i2));
      m[z++"s"]@i2 := (m[x++"s"] + m[y++"s"])@i2;
      m[z++"m"]@i2 := (m[x++"m"] + m[y++"m"])@i2;
      m[z++"k"]@i1 := (m[x++"k"] + m[y++"k"])@i1;
      post(m[z++"m"]@i2 == m[z++"k"]@i1 + (m["delta"]@i1 * m[z++"s"]@i2)
  }

  sum(z,x,y) { _sum(z,x,y,1,2);_sum(z,x,y,2,1) }

  open(x) { _open(x,1,2); _open(x,2,1) } 


  sum("a","x","d");
  open("d");
  sum("b","y","e");
  open("e");
  let xys =
      macsum(macctimes(macshare("b"), m["d"]),
             macsum(macctimes(macshare("a"), m["e"]),
                    macshare("c")))
  let xyk = mack("b") * m["d"] + mack("a") * m["e"] + mack("c")
                    
  secopen("a","x","d",1,2);
    secopen("a","x","d",2,1);
    secopen("b","y","e",1,2);
    secopen("b","y","e",2,1);
    let xys =
      macsum(macctimes(macshare("b"), m["d"]),
             macsum(macctimes(macshare("a"), m["e"]),
                    macshare("c")))
    in
    let xyk = mack("b") * m["d"] + mack("d") * m["d"] + mack("c")               
    in
    secreveal(xys,xyk,"1",1,2);
    secreveal(maccsum(xys,m["d"] * m["e"]),
              xyk - m["d"] * m["e"],
              "2",2,1);
    out@1 := (p[1] + p[2])@1;
    out@2 := (p[1] + p[2])@2;
\end{verbatimtab}


\end{comment}
