\section{Integrity Types}

\subsection{BDOZ MAC Integrity}

In a malicious setting, detecting cheating by adding
information-theoretic secure MAC codes to shares is a fundamental
technique pioneered in systems such as BDOZ and SPDZ
\cite{SPDZ1,SPDZ2,BDOZ,10.1007/978-3-030-68869-1_3}.  These protocols
assume a pre-processing phase that distributes shares with MAC codes
to clients.  This integrates well with pre-processed Beaver Triples to
implement malicious secure, and relatively efficient, multiplication
\cite{evans2018pragmatic}.

A field value $v$ is secret shared among 2 clients in BDOZ with
accompanying MAC values.  Each client $\cid$ gets a pair of values
$(v_\cid,m_\cid)$, where $v_\cid$ is a share of $v$ reconstructed by
addition, i.e., $v = \fcod{v_1 \fplus v_2}$, and $m_\cid$ is a MAC of
$v_\cid$.  More precisely, $m_\cid = k + (k_\Delta * v_\cid)$ where
\emph{keys} $k$ and $k_\Delta$ are known only to $\cid' \ne \cid$ and
$k_\Delta$. The \emph{local key} $k$ is unique per MAC, while the
\emph{global key} $k_\Delta$ is common to all MACs authenticated by
$\cid'$. This is a semi-homomorphic encryption scheme that supports
addition of shares and multiplication of shares by a constant-- for
more details the reader is referred to Orsini
\cite{10.1007/978-3-030-68869-1_3}. We note that while we restrict
values $v$, $m$, and $k$ to the same field in this presentation for
simplicity, in general $m$ and $k$ can be in extensions of
$\mathbb{F}_p$.

As in \cite{skalka-near-ppdp24}, we use $\macgv{\mesg{w}}$ to refer to
secret-shared values reconstructed with addition, and by
convention each share of $\macgv{\mesg{w}}$ is represented as
$\elab{\mesg{w\ttt{s}}}{\cid}$, the MAC of which is represented as a
$\elab{\mesg{w\ttt{m}}}{\cid}$ for all $\cid$, and each client holds a
key $\elab{\mesg{w\ttt{k}}}{\cid}$ for authentication of the other's
share. Each client also holds a global key $\mesg{\ttt{delta}}$. For
any such share identified by string $w$, the BDOZ MAC scheme is defined 
by the equality predicate $\macbdoz{w}$:
$$
\macbdoz{w} \defeq
\mesg{w\ttt{m}} = \mesg{w\ttt{k}} \fplus \ttt{(}\mesg{\ttt{delta}} \ftimes
\mesg{w\ttt{s}}\ttt{)}
$$

\subsection{Typing and Integrity Labeling}

\ipjfig

The integrity of values during protocol executions depends on the
partitioning $H,C$, since any value received from a member of
$C$ is initially considered to be low integrity. But as for
confidentiality types, we want our basic type analysis to be
generalizable to arbitrary $H,C$\cnote{Add related text to that
  section.}. So integrity type judgements are of the form
$\ipj{\eqs}{\prog}{\Delta}$, where $\eqs$ plays a similar role as in
confidentiality types, as a possibly weakened expression of algebraic
properties of $\prog$, and encironments $\Delta$ record integrity
information. Environments are ordered lists, since order of evaluation
is important for tracking integrity-- validating low
integrity information through MAC checking allows \emph{subsequent}
treatment of it as high integrity, but not \emph{before}.

The typing of protocols depends on a type judgement for expressions
of the form $\itj{\cid}{\be}{\ity{\cid}{V}}$, which records
messages and reveals $V$ used in the construction of $\be$.
In an assignment, this source is recorded in $\Delta$-- so for
example, letting:
$$
\prog\ \defeq\ (\xassign{\mx{x}{1}}{\secret{x} - \flip{x}}{2};\ \rvl{x} := \mx{x}{1})
$$
the following judgement is valid:
$$
\ipj{\toeq{\prog}}{\prog}{(\mx{x}{1} : \ity{2}{\varnothing}; \rvl{x} : \ity{1}{\mx{x}{1}})}
$$
More formally, we defined validity as follows:
\begin{definition}
  Given pre-processing predicate $\eqspre$ and protocol $\prog$, 
  we say $\ipj{\eqs}{\prog}{\Delta}$ is \emph{valid}
  iff it is derivable and $\eqspre \wedge \toeq{\prog} \models E$.
\end{definition}

In the above example, note that environment implicitly records
that $\rvl{x}$ depends on data from client 1, which in turn
depends on data from client 2. The rule for MAC
checking allows an overwrite of the dependency. We explore
the application of this in more detail in Section
\ref{section-examples}.

\cheatjfig

In all cases, we want to determine the integrity level of computed
values based on assumed partitionins $H,C$. In the previous example,
we want to say that $\rvl{x}$ has low integrity if either client 1 or
2 is corrupt.  To this end we define \emph{security labelings}
$\seclev$ which are mappings from variables $x$ to security levels
$\latel$. By default, given a partitioning $H,C$ this labeling maps
messages to the integrity level of the owner.
\begin{definition}  
  Given $H,C$,
  define $\seclev_{H,C}$ such that $\seclev_{H,C}(\mx{w}{\cid} ) = \hilab$
  if $\cid \in H$  and $\lolab$
  otherwise for all $\mx{w}{\cid}$.
\end{definition}
During execution of a protocol, the integrity of messages and reveals
affected by values sent by other parties, as recorded in $\Delta$
obtained from its typing. We additionally need to determine
whether this impacts integrity, which we accomplish through
the inductively defined rewrite relation $\cheatj{\Delta}{H,C}{\seclev}$,
axiomatized in Figure \ref{fig-cheatj}.
Our main type safety result for integrity is then given as follows.
\begin{theorem}
  Given pre-processing predicate $\eqspre$ and protocol $\prog$ with
  $\views(\prog) = V$, if
  $\ipj{\eqs}{\prog}{\Delta}$ is valid
  and for all $H,C$ with $\cheatj{\Delta}{H,C}{\seclev}$ 
  we have $\seclev(x) = \hilab$ for all $x \in \houtputs$, then cheating
  is detectable in $\prog$.
\end{theorem}
