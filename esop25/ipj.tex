\section{Integrity Types}
\label{section-ipj}

Our integrity type system is essentially a taint analysis, with
machinery to allow consideration of any $H,C$ partitioning.  We also
include rules to consider MAC codes to prevent cheating. We consider a
semi-homomorphic encryption scheme that is used in BDOZ and SPDZ
\cite{SPDZ1,SPDZ2,BDOZ,10.1007/978-3-030-68869-1_3} and is applicable
to various species of circuits. However other MAC schemes could be
considered with a modified derivation rule incorporated with our taint
analysis framework. BDOZ/SPDZ assumes a pre-processing phase that
distributes shares with MAC codes to clients.  This integrates well
with pre-processed Beaver Triples to implement malicious secure, and
relatively efficient, multiplication \cite{evans2018pragmatic}.

In BDOZ, a field value $v$ is secret shared among 2 clients 
with accompanying MAC values.  Each client $\cid$ gets a pair of
values $(v_\cid,m_\cid)$, where $v_\cid$ is a share of $v$
reconstructed by addition, i.e., $v = \fcod{v_1 \fplus v_2}$, and
$m_\cid$ is a MAC of $v_\cid$.  More precisely, $m_\cid = k +
(k_\Delta * v_\cid)$ where \emph{keys} $k$ and $k_\Delta$ are known
only to $\cid' \ne \cid$ and $k_\Delta$. The \emph{local key} $k$ is
unique per MAC, while the \emph{global key} $k_\Delta$ is common to
all MACs authenticated by $\cid'$. This is a semi-homomorphic
encryption scheme that supports addition of shares and multiplication
of shares by a constant-- for more details the reader is referred to
Orsini \cite{10.1007/978-3-030-68869-1_3}. We note that while we
restrict values $v$, $m$, and $k$ to the same field in this
presentation for simplicity, in general $m$ and $k$ can be in
extensions of $\mathbb{F}_p$.

\ipjfig

As in \cite{skalka-near-ppdp24}, we use $\macgv{\mesg{w}}$ to refer to
secret-shared values reconstructed with addition, and by
convention each share of $\macgv{\mesg{w}}$ is represented as
$\elab{\mesg{w\ttt{s}}}{\cid}$, the MAC of which is represented as a
$\elab{\mesg{w\ttt{m}}}{\cid}$ for all $\cid$, and each client holds a
key $\elab{\mesg{w\ttt{k}}}{\cid}$ for authentication of the other's
share. Each client also holds a global key $\mesg{\ttt{delta}}$. For
any such share identified by string $w$, the BDOZ MAC scheme is defined 
by the equality predicate $\macbdoz{w}$:
$$
\macbdoz{w} \defeq
\mesg{w\ttt{m}} = \mesg{w\ttt{k}} \fplus \ttt{(}\mesg{\ttt{delta}} \ftimes
\mesg{w\ttt{s}}\ttt{)}
$$

\subsection{Typing and Integrity Labeling}

The integrity of values during protocol executions depends on the
partitioning $H,C$, since any value received from a member of $C$ is
initially considered to be low integrity. But as for confidentiality
types, we want our basic type analysis to be generalizable to
arbitrary $H,C$. So integrity type judgements, defined in Figure
\ref{fig-ipj}, are of the form $\ipj{\eqs}{\prog}{\Delta}$, where
$\eqs$ plays a similar role as in confidentiality types, as a possibly
weakened expression of algebraic properties of $\prog$, and
environments $\Delta$ record dependency information that can be
specialized to a given $H,C$. Integrity environments are ordered lists
since order of evaluation is important for tracking integrity--
validating low integrity information through MAC checking allows
subsequent treatment of it as high integrity, but not before. Note
that MAC checking is \emph{not} a form of endorsement as it is usually
defined (as a dual of declassification
\cite{sabelfeld2009declassification}), since a successful check
guarantees no deviation from the protocol.

The typing of protocols depends on a type judgement for expressions
of the form $\itj{\cid}{\be}{V}$, which records
messages and reveals $V$ used in the construction of $\be$.
This source is recorded in $\Delta$ for message sends-- so for
example, for some string $w$ letting:
$$
\prog\ \defeq\ (\xassign{\mx{w}{1}}{\secret{w} - \flip{w}}{2};\ \rvl{w} := \mx{w}{1})
$$
the following judgement is valid:
$$
\ipj{\toeq{\prog}}{\prog}{(\mx{w}{1} : \ity{2}{\varnothing}; \rvl{w} : \ity{1}{\mx{w}{1}})}
$$
More formally, we define validity as follows:
\begin{definition}
  Given pre-processing predicate $\eqspre$ and protocol $\prog$, 
  we say $\ipj{\eqs}{\prog}{\Delta}$ is \emph{valid}
  iff it is derivable and $\eqspre \wedge \toeq{\prog} \models E$.
\end{definition}

In the above example, note that the environment implicitly records
that $\rvl{w}$ depends on data from client 1, which in turn
depends on data from client 2. The rule for MAC
checking allows an overwrite of the dependency. We explore
the application of this in more detail in Section
\ref{section-examples}.

\cheatjfig

\subsection{Assigning Integrity Labels}

In all cases, we want to determine the integrity level of computed
values based on assumed partitionins $H,C$. In the previous example,
we want to say that $\rvl{w}$ has low integrity if either client 1 or
2 is corrupt.  To this end we define \emph{security labelings}
$\seclev$ which are mappings from variables $x$ to security levels
$\latel$. By default, given a partitioning $H,C$ this labeling maps
messages to the integrity level of the owner.
\begin{definition}  
  Given $H,C$,
  define $\seclev_{H,C}$ such that $\seclev_{H,C}(\mx{w}{\cid} ) = \hilab$
  if $\cid \in H$  and $\lolab$
  otherwise for all $\mx{w}{\cid}$.
\end{definition}
But the values of messages and reveals may be affected by values sent
by other parties, which is recorded in the $\Delta$ obtained from its
typing. To determine whether this impacts integrity for any given
$H,C$, we apply the inductively defined rewrite relation
$\cheatj{\Delta}{H,C}{\seclev}$, axiomatized in Figure
\ref{fig-cheatj}. The composition of typing and this latter rewriting
obtains essentially a traditional taint analysis, modulo MAC checking.
Our main type soundness result for integrity is then given as follows.
Proof details are given in the Appendix.
\begin{theorem}
  \label{theorem-ipj}
  Given  $\eqspre$ and $\prog$ with
  $\iov(\prog) = S \cup V \cup O$, if
  $\ipj{\eqs}{\prog}{\Delta}$ is valid
  and for all $H,C$ with $\cheatj{\Delta}{H,C}{\seclev}$ 
  we have $\seclev(x) = \hilab$ for all $x \in \houtputs \cup O_H$, then $\prog$
  has integrity for all $H,C$.
\end{theorem}
