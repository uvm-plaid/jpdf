\section{Confidentiality Types}
\label{section-cpj}

It is well-known that adding or subtracting a sample from a uniform
distribution in a finite field yields a value in a uniform
distribution, meaning that samples can be used as one-time-pads with
perfect secrecy \cite{barthe2019probabilistic,darais2019language}.
In our type system for confidentiality, we aim to approximate
distributions by tracking which program variables an expression
may depend on. But we also want to capture this fundamental mechanism
of encryption.

For example, given some string $y$ and a message send such as:
$$
\xassign{\mx{w}{1}}{\secret{w} \ftimes \flip{w}}{2}
$$
we would assign the following type to denote that $\mx{w}{1}$ is dependent on
both $\sx{w}{2}$ and $\rx{w}{2}$:
$$
\mx{w}{1} : \setit{\sx{w}{2}, \rx{w}{2}}
$$
But in the case of 2-party reconstructive sharing:
$$
\xassign{\mx{w}{1}}{\secret{w} \fminus \flip{w}}{2}
$$
we want the type of $\mx{w}{1}$ to be an ``encrypted'' type
that is by itself independent of $\sx{w}{2}$. However, we also
want to track the dependence of $\mx{w}{2}$ and $\rx{w}{2}$.
In short, we assign $\mx{w}{1}$ a ``ciphertext type'':
$$
\mx{w}{1} : \cty{\rx{w}{2}}{\setit{\sx{w}{2}}}
$$
indicating syntactically that $\mx{w}{1}$ is an encryption of a
value dependent on $\sx{w}{2}$ with $\rx{w}{2}$ and as long as client
1 does not subsequently receive messages that are dependent on
$\rx{w}{2}$, the value $\sx{w}{2}$ remains hidden.

Of course, there are other methods for encrypting values in MPC
protocols-- but we can observe that many are algebraically equivalent
to this fundamental schema. For example, in Yao's Garbled Circuits
(YGC) in $\mathbb{F}_{2}$, the ``garbler'' represents encrypted wire
values as a random sample (for 1) or its negation (for 0), and the
``evaluator'' obtains the wire value for its input via OT. The
evaluator shares the initial encrypted wire value of one of its input
secrets as a selection bit-- assuming that client 2 is the garbler and
client 1 is the evaluator, we can write this as:
$$
\xassign{\mx{w}{1}}{\mux{\secret{w}}{\neg\flip{w}}{\flip{w}}}{2}
$$
where for all $\be_1,\be_2,\be_3$:
$$
\mux{\be_1}{\be_2}{\be_3} \defeq (\neg\be_1 \ftimes \be_2) \fplus (\be_1 \ftimes \be_3)
$$
and letting this protocol be $\prog$ the following is valid:
\begin{mathpar}
%\toeq{\xassign{\mx{w}{1}}{\OT{\secret{w}}{1}{\neg\flip{w}}{\flip{w}}}{2}}
   \toeq{\prog} \models \mx{w}{1} \eop \neg\sx{w}{2} \fplus \rx{w}{2}
\end{mathpar}
This allows us to assign the same ciphertext type to $\mx{w}{1}$ as
above-- that is, $\cty{\rx{w}{2}}{\setit{\sx{w}{2}}}$. Leveraging the
equivalence this way has been previously well-studied
\cite{barthe2019probabilistic}.  Since this is a one-time-pad scheme,
it is also important that we ensure that samples are used at most once
for encryption. As in previous work \cite{darais2019language} we can
use a form of type linearity to this end.

\cpjfig

The syntax of types $\ty$ and type environments $\Gamma$ are presented
in Figure \ref{fig-cpj}\cnote{need to fix set notations}. We also
define type judgements for expressions and programs by direct
interpretation as terms $\phi$ and constraints $\eqs$ respectively,
via the encoding $\toeq{cdot}$. The \TirName{Encode} rule captures
the use ``fresh'' samples for encryption discussed above. Derivations
enforce linearity by requiring that variables used for encryption
are added to judgements, and guaranteed to not be used elsewhere
via a form of union that ensures disjointness of combined sets. 
\begin{definition}
  $R_1;R_2 = R_1 \cup R_2$ iff $R_1 \cap R_2 = \varnothing$.
\end{definition}
The \TirName{DepTy} rule handles all other syntactic forms and just
records the dependencies on variables in the term.  The \TirName{Send}
rule records the types of messages, reveals, and outputs in
enviroments, and the \TirName{Seq} rule combines the types of
sequenced programs.  Given this we define validity of program type
judgements as follows. \cnote{Define $\eqspre$.}
\begin{definition}
  Given preprocessing predicate $\eqspre$ and protocol $\prog$ we say
  $\cpj{R}{\eqs}{\eqspre \wedge \toeq{\prog}}{\Gamma}$ is \emph{valid} iff it is derivable and
  $\eqspre \wedge \toeq{\prog} \models \eqs$.
\end{definition}
It is important to note that this presentation of derivations is
logical, not algorithmic-- in particular it is not syntax-directed
due to the \TirName{Encode} rule.

\subsection{Detecting Leakage Through Dependencies}

Given a valid typing, we can leverage structural type information to
conservatively approximate the information accessible by the
adversary who has control over all corrupt clients in $C$ given
a partitioning $H,C$. In particular, if the adversary has access to an
encypted value and another value that is dependent on its one-time-pad
key-- i.e., access to values with types of the forms:
$$
\setit{\cty{\rx{w}{\cid}}{\ty}} \qquad \setit{\rx{w}{\cid}, \ldots} 
$$
then we can conservatively assume that the adversary can decrypt the
value of type $\ty$ and thus have access to those values.

In Figure \ref{fig-leakj} we define derivation rules for the
``leakage'' judgement $\leakj{\Gamma}{C}{\ty}$ which intuitively says
that given a protocol with type environment $\Gamma$ and corrupt
clients $C$, the adversary may be able to observe dependencies on
values of type $\ty$ in their adversarial messages.  The real aim of
this relation is to determine if
$\leakj{\Gamma}{C}{\setit{\sx{w}{\cid}}}$ with $\cid \in H$-- that is,
if the adversary may observe dependencies on honest secrets.

\subsubsection{Type Soundness}

\cnote{Define (conditional) inputs in hyperproperties section.}

\begin{lemma}
  \label{lemma-eqsprogsound}
  If $\toeq{\prog} \models \eqs$ and $\eqs \models \eqs'$, then
  for all $\store \in \runs(\prog)$ we have $\store \models \eqs'$.
\end{lemma}

\begin{definition}
  \label{definition-sound}
  Given $\prog$ with $\iov(\prog) = S \cup M \cup P \cup O$
  and $\flips(\prog) = R$, we say that
  \emph{$\Gamma$ is sound for $\prog$} iff for all $M' \subseteq M$
      and $X \subseteq S \cup R$, the following
      conditions hold:
  \begin{enumerate}[\hspace{5mm}i.]
    \item  if $X$ are inputs to $M'$ then
      there exists $\ty$ with $\leakj{\Gamma}{M'}{\ty}$ and $X \subseteq \ty$.
    \item  for all  $R' \subseteq R$, if
      $X$ are $R'$ are conditional inputs to $M'$ then $\leakj{\Gamma}{M'}{\ty}$
      and $\leakclose{\Gamma}{\ty \cup R'}{\ty'}$ and $X \subseteq \ty'$.
  \end{enumerate}
\end{definition}

\begin{lemma}
  \label{lemma-cpjsound}
  If $\cpj{R}{\eqs}{\eqspre \wedge \toeq{\prog}}{\Gamma}$ is valid then $\Gamma$ is
  sound for $\prog$.
\end{lemma}
\begin{proof}
  Unrolling Definition \ref{definition-sound}, proceed by induction on $M'$.
  In case $M' = \varnothing$ the result follows trivially. Otherwise
  $M' = M_0 \cup \{ \mx{w}{\cid} \}$ and wlog $\Gamma = \Gamma_0; \mx{w}{\cid} : \ty_0$
  and $\prog = \prog_0;\xassign{\mx{w}{\cid}}{\be}{\cid_0}$, where $R = R_0;R_1$ and:
  \begin{mathpar}
    \cpj{R_0}{\eqs}{\eqspre \wedge \toeq{\prog_0}}{\Gamma_0}

    \eqj{R_1}{\eqs}{\toeq{\elab{\be}{\cid}}}{\ty_0}
  \end{mathpar}
  are both valid, and by the induction hypothesis $\Gamma_0$ is sound for $\prog_0$.
  We can then proceed by a second induction on the derivation of $\ty_0$, where there
  are two subcases. XXX
\end{proof}

Our main type soundness result combines typing and leakage judgements,
and guarantees gradual release of protocols-- that is, that adversarial
messages reflect no dependencies on honest secrets.
\cnote{Need to define gradual release... should probably have a section
  where we summarize hyperproperties.}
\begin{theorem}
  Given $\prog$ with $\views(\prog) = M \cup P$, if $\cpj{R}{\eqs}{\eqspre \wedge \toeq{\prog}}{\Gamma}$
  is valid and for all $H,C$ we have $\leakj{\Gamma}{M_C}{T}$ with $T$ closed
  and $S_H \cap T = \varnothing$, then $\prog$ satisfies gradual release.
\end{theorem}
\begin{proof}
  Immediate by Lemma \ref{lemma-cpjsound}. \qed
\end{proof}

\leakjfig

Of course, it is important to recall from \cite{skalka-near-ppdp24}
that gradual release does \emph{not} imply passive security, since it
only concerns adversarial messages, but not public reveals of outputs
where dependencies on input secrects can occur even in secure
protocols.

\begin{comment}
\subsection{Examples}

\begin{verbatimtab}
m[s1]@2 := (s[1] - r[local] - r[x])@1
m[s1]@3 := r[x]@1

// m[s1]@2 : { c(r[x]@1, { c(r[local]@1, {s[1]@1} ) }
// m[s1]@3 : { r[x]@1 }
\end{verbatimtab}

\begin{verbatimtab}
m[x]@1 := s2(s[x],-r[x],r[x])@2

// m[x]@1 == s[x]@2 + -r[x]@2 
// m[x]@1 : { c(r[x]@2, { s[x]@2 }) } 

m[y]@1 := OT(s[y]@1,-r[y],r[y])@2

// m[y]@1 == s[y]@1 + -r[y]@2
// m[y]@1 : { c(r[y]@2, { s[y]@1 }) } 
\end{verbatimtab}
\end{comment}
