\section{SMT Verification and Hoare Logic}

\subsection{Corrections to $\metaprot$ Generalized Constraint Forms and Evaluation}

The syntax of $\notg{\eqs}$ in the ESOP paper was too complicated. It should simply be:
$$
\begin{array}{rcl@{\hspace{4mm}}r}
\notg{\eqs} &::=& e \eop e \mid \notg{\eqs} \wedge \notg{\eqs} 
\end{array}
$$
The semantics of generalized constraint evaluation can also be greatly simplified, as
follows:
\begin{mathpar}
  \inferrule
      {e_1 \redx \be_1 \\ e_2 \redx \be_2}
      {e_1 \eop e_2 \redx \toeq{\be_1} \eop \toeq{\be_2}}

  \inferrule
      {\notg{\eqs_1} \redx \eqs_1 \\ \notg{\eqs_2} \redx \eqs_2 }
      {\notg{\eqs_1} \wedge \notg{\eqs_2} \redx \eqs_1 \wedge \eqs_2}
\end{mathpar}
Given these changes, we also have some corrections to the $\TirName{Assert}$ and $\TirName{Mesg}$
rules below in the Hoare Logic, which referred to a syntactic form ``$\toeq{e}$'' which now seems meaningless.

The upshot is that evaluation of generalized constraints, mainly in the $\TirName{GenEntails}$
rule, can be defined very directly in terms of evaluation of $\metaprot$ expressions.

\subsection{Hoare Logic}

\newcommand{\htrip}[3]{\{ #1 \}\ #2\ \{ #3 \}}
\newcommand{\peq}{\notg{\eqs}}
\newcommand{\ptoeq}[1]{\toeq{#1}}
\newcommand{\eqtrue}{\mathrm{true}}

Adding basic types to command function declarations to support generalize pre/post conditions. 

$$
\begin{array}{rcl@{\hspace{4mm}}r}
\tau &::=& \fieldp{p} \mid \mathit{string} \mid \mathit{cid} \mid \{ \flab_1 : \tau_1; \ldots; \flab_n : \tau_n \} & \gdesc{basic types}\\[2mm]
\mathit{fn} &::=& f(y,\ldots,y) \{ e \} \mid  f(y : \tau,\ldots,y : \tau) \{ \cmd \} & \textit{functions}
\end{array}
$$
\medskip

\noindent This is the generalized rule for entailment:

$$
\inferrule[GenEntails]
          {\mv_1,\ldots,\mv_n = \fresh(\tau_1,\ldots,\tau_n) \\ \peq_1[\mv_1/y_n \cdots \mv_n/y_n] \redx \eqs_1 \\
           \peq_2[\mv_1/y_n \cdots \mv_n/y_n] \redx \eqs_2 \\ \eqs_1 \models \eqs_2}
          {\forall y_1:\tau_1,\ldots,y_n:\tau_n . \peq_1 \models \peq_2 }
$$

The following are generalized Hoare triple computations. Hard and soft pack rules allow for compositional
verification. Hardpack eliminates program variables, while softpack may require less programmer overhead. 
\begin{mathpar}
  \inferrule[Mesg]
            {}
            {\htrip{\eqtrue}{\xassign{e_1}{e_2}{e_3}}{e_1 \eop \elab{e_2}{e_3}}}

  %\inferrule[Encode]
  %          {\mx{e_1}{e_2} \redx x \\ \notg{\phi} \redx \phi \\
  %            \eqs \models x \eop \phi\\
  %            \atj{\phi}{R}{\ty}}
  %          {\mtj{\eqcast{\mx{e_1}{e_2}}{\notg{\phi}}}{\eqs}{(x : \ty)}{R}{\varnothing}{\eqs}}
  %
  \inferrule[Assert]
            {}
            {\htrip{\elab{e_1}{e_3} \eop \elab{e_2}{e_3}}{\elab{\assert{e_1 = e_2}}{e_3}}{\eqtrue}}

  \inferrule[Seq]          
            {\htrip{\peq_1^1}{\cmd_1}{\peq_2^1} \\ \htrip{\peq_1^2}{\cmd_2}{\peq_2^2}}
            {\htrip{\peq_1^1 \wedge \peq_1^2}{\cmd_1;\cmd_2}{\peq_2^1 \wedge \peq_2^2}}

  \inferrule[Let]
            {\htrip{\peq_1}{\cmd[e/y]}{\peq_2}}
            {\htrip{\peq_1}{\elet{y}{e}{\cmd}}{\peq_2}}

  \inferrule[App]
            {\htrip{\peq_1}{f(y_1:\tau_1,\ldots,y_n:\tau_n)}{\peq_2}}
            {\htrip{\peq_1[e_1/y_1 \cdots e_n/y_n]}{f(e_1,\ldots,e_n)}{\peq_2[e_1/y_1 \cdots e_n/y_n]}}

  \inferrule[Fn]
            {\codebase(f) = y_1 : \tau_1, \ldots, y_n : \tau_n, \cmd \\ \htrip{\peq_1}{\cmd}{\peq_2}}
            {\htrip{\peq_1}{f(y_1:\tau_1,\ldots,y_n:\tau_n)}{\peq_2}}

  \inferrule[Hardpack]
            {\precond(f) = \peq_1 \\ \postcond(f) = \peq_2 \\
              \htrip{\peq_1'}{f(y_1:\tau_1,\ldots,y_n:\tau_n)}{\peq_2'} \\
              \forall y_1:\tau_1,\ldots,y_n:\tau_n . \peq_1 \wedge \peq_2' \models \peq_1' \wedge \peq_2 }
            {\htrip{\peq_1}{f(y_1:\tau_1,\ldots,y_n:\tau_n)}{\peq_2}}

  \inferrule[Softpack]
            {\precond(f) = \peq_1 \\ \postcond(f) = \peq_2 \\
              \htrip{\peq_1'}{f(y_1:\tau_1,\ldots,y_n:\tau_n)}{\peq_2'} \\
              \forall y_1:\tau_1,\ldots,y_n:\tau_n . \peq_1 \wedge \peq_1' \wedge \peq_2' \models \peq_2 }
            {\htrip{\peq_1' \wedge \peq_1}{f(y_1:\tau_1,\ldots,y_n:\tau_n)}{\peq_2' \wedge \peq_2}}
\end{mathpar}


% https://tchajed.github.io/sys-verif-fa24/notes/hoare.html#example-consequences-of-bind-rule
