\section{Confidentiality Types}

$$
\begin{array}{rcl@{\hspace{2mm}}r}
  t &::=& x \mid \cty{x}{T} \\
  \ty & \in & 2^{t}\\
  \Gamma &::=& \varnothing \mid \Gamma; x : \ty
\end{array} 
$$

\begin{definition}
  $R_1;R_2 = R_1 \cup R_2$ iff $R_1 \cap R_2 = \varnothing$.
\end{definition}

\begin{mathpar}
  \inferrule[DepTy]
  {}
  {\eqj{\varnothing}{\eqs}{\phi}{\vars(\phi)}}
  
  \inferrule[Encode]
  {\eqs \models \phi \eop \phi' \oplus \rx{w}{\cid} \\
   \oplus \in \{ \fplus,\fminus \}\\
   \eqj{R}{\eqs}{\phi'}{\ty}}
  {\eqj{R;\{ \rx{w}{\cid} \}}{\eqs}{\phi}{\setit{\cty{\rx{w}{\cid}}{\ty}}}}
\end{mathpar}

\begin{mathpar}
  \inferrule[Send]
            {\eqj{R}{\eqs}{\phi}{\ty}}
            {\cpj{R}{\eqs}{x \eop \phi}{(x : \ty)}}
            
  \inferrule[Seq]
            {\cpj{R_1}{\eqs}{\phi_1}{\Gamma_1}\\
             \cpj{R_2}{\eqs}{\phi_2}{\Gamma_2}}
            {\cpj{R_1;R_2}{\eqs}{\phi_1 \wedge \phi_2}{\Gamma_1;\Gamma_2}}
\end{mathpar}

\begin{definition}
  Given preprocessing predicate $\eqspre$ and protocol $\prog$ we say
  $\cpj{R}{\eqs}{\eqspre \wedge \toeq{\prog}}{\Gamma}$ is \emph{valid} iff it is derivable and
  $\eqspre \wedge \toeq{\prog} \models \eqs$.
\end{definition}

\begin{mathpar}
  \inferrule
      {\cid \in C}
      {\leakj{\Gamma}{C}{\Gamma(\mx{w}{\cid})}}

  \inferrule
      {\leakj{\Gamma}{C}{T_1 \cup T_2}}
      {\leakj{\Gamma}{C}{T_1}}

  \inferrule
      {\leakj{\Gamma}{C}{\setit{ \mx{w}{\cid} }}}
      {\leakj{\Gamma}{C}{\Gamma(\mx{w}{\cid})}}

  \inferrule
      {\leakj{\Gamma}{C}{\setit{ \rx{w}{\cid} }} \\ \leakj{\Gamma}{C}{\setit{ \cty{\rx{w}{\cid}}{\ty} }} }
      {\leakj{\Gamma}{C}{\ty}}
\end{mathpar}

\begin{theorem}
  If $\cpj{R}{\eqs}{\eqspre \wedge \toeq{\prog}}{\Gamma}$ is valid and there exists no $H,C$ 
  and  $\sx{w}{\cid}$ for $\cid \in H$ with $\leakj{\Gamma}{C}{\setit{\sx{w}{\cid}}}$,
  then $\prog$ satisfies gradual release.
\end{theorem}

\subsection{Examples}

\begin{verbatimtab}
m[s1]@2 := (s[1] - r[local] - r[x])@1
m[s1]@3 := r[x]@1

// m[s1]@2 : { c(r[x]@1, { c(r[local]@1, {s[1]@1} ) }
// m[s1]@3 : { r[x]@1 }
\end{verbatimtab}

\begin{verbatimtab}
m[x]@1 := s2(s[x],-r[x],r[x])@2

// m[x]@1 == s[x]@2 + -r[x]@2 
// m[x]@1 : { c(r[x]@2, { s[x]@2 }) } 

m[y]@1 := OT(s[y]@1,-r[y],r[y])@2

// m[y]@1 == s[y]@1 + -r[y]@2
// m[y]@1 : { c(r[y]@2, { s[y]@1 }) } 
\end{verbatimtab}
