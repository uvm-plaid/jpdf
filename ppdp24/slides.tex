% Inbuilt themes in beamer
\documentclass{beamer}

% Theme choice:
\usetheme{CambridgeUS}
\usecolortheme{default}

\input{slide-macros}

% Title page details: 
\title{Language-Based Security for Low-Level MPC}

\author{Christian Skalka and Joe Near}

\date{September 10, 2024}

\logo{
\includegraphics[width=3cm]{UVM_Logo_Primary_Horiz_G.png}
}

\beamertemplatenavigationsymbolsempty

\begin{document}

% Title page frame
\begin{frame}
    \titlepage 
\end{frame}

% Remove logo from the next slides
\logo{}


% % Outline frame
% \begin{frame}{Outline}
%     \tableofcontents
% \end{frame}

\section{What's an MPC Protocol}
\begin{frame}

  A \cemph{protocol} $\prog$ implements an \emph{ideal functionality} $\idealf$,
  defined in terms of operations in a prime field $\mathbb{F}_p$, where
  the inputs are the secrets of clients.

  \medskip

  Low-level, probabilistic protocols are defined in terms of basic variable types:
  \begin{itemize}
  \item $\secret{x}$: secret values in $\mathbb{F}_p$.
  \item $\mesg{x}$: messages received during the protocol.
  \item $\reveal{x}$: broadcast messages.
  \item $\flip{x}$: a uniformly random sample from $\mathbb{F}_p$ (implemented with
    a random tape semantics).
  \item $\pubout$: a protocol output
  \end{itemize}
  We can localize variables and expressions to \emph{clients}, and any information
  exchanged between clients must be through messaging.
  \begin{block}{Example}
    $$
    \mx{x}{1} := \elab{(\secret{x} - \flip{x})}{2}
    $$
  \end{block}
\end{frame}

\section{Ideal Functionality}
\begin{frame}{Ideal Functionality}
  $$
  \idealf(\sx{1}{1},\sx{2}{2},\sx{3}{3}) =
  \sx{1}{1}\ \exor\ \sx{2}{2}\ \exor\ \sx{3}{3}
  $$
  Suppose the output of the protocol is 1 and that party 3 is corrupted, so
  the adversary knows that $\sx{3}{3} = 0$.
  
  \medskip

  This unavoidably declassifies some information about the secrets of
  parties 1 and 2-- either:
  $$
  \sx{1}{1} = 0 \wedge \sx{2}{2} = 1
  $$
  or:
  $$
  \sx{1}{1} = 1 \wedge \sx{2}{2} = 0
  $$
  with equal probability.
\end{frame}

\section{Conditional Noninterference}
\begin{frame}{Conditional Noninterference}
  
  \begin{block}{Bounded Declassification}
    The ideal functionality defines a declassification ``limit''. Any protocol $\prog$
    should not reveal any more through broadcasts and DMs.
  \end{block}

  \begin{block}{Adversarial Views}
    In a run of the protocol, in addition to the output and secrets of corrupt
    parties, the adversary has access to all broadcasts and messages received
    by corrupt parties.
  \end{block}

  \begin{alertblock}{Conditional Noninterference}
    The probability of honest secrets conditioned on the protocol output
    should not be changed by conditioning on adversarial views. 
  \end{alertblock}
  
\end{frame}

\section{Secret Sharing}
\begin{frame}{Secret Sharing}

  \begin{block}{Reconstructive Sharing}
    $$
    \begin{array}{lll}
      \elab{\mesg{s1}}{2} &:=& \elab{(\secret{1} \fminus \locflip \fminus \flip{x})}{1} \\ 
      \elab{\mesg{s1}}{3} &:=& \elab{\flip{x}}{1} \\ 
      \elab{\mesg{s2}}{1} &:=& \elab{(\secret{2} \fminus \locflip \fminus \flip{x})}{2} \\ 
      \elab{\mesg{s2}}{3} &:=& \elab{\flip{x}}{2} \\ 
      \elab{\mesg{s3}}{1} &:=& \elab{(\secret{3} \fminus \locflip \fminus \flip{x})}{3} \\ 
      \elab{\mesg{s3}}{2} &:=& \elab{\flip{x}}{3}
    \end{array}
    $$
  \end{block}
  
\end{frame}

% Lists frame
\section{Lists in Beamer}
\begin{frame}{Lists in Beamer}

This is an unordered list:
\begin{itemize}
    \item Item 1
    \item Item 2
    \item Item 3
\end{itemize}

and this is an ordered list:
\begin{enumerate}
    \item Item 1
    \item Item 2
    \item Item 3
\end{enumerate}

\end{frame}

% Blocks frame
\section{Blocks in Beamer}
\begin{frame}{Blocks in Beamer}
    \begin{block}{Standard Block}
        This is a standard block.
    \end{block}
    \begin{alertblock}{Alert Message}
        This block presents alert message.
    \end{alertblock}
    \begin{exampleblock}{An example of typesetting tool}
        Example: MS Word, \LaTeX{}
    \end{exampleblock}
\end{frame} 

\end{document}
