% Inbuilt themes in beamer
\documentclass{beamer}

% Theme choice:
\usetheme{CambridgeUS}
\usecolortheme{default}

\usepackage[style=verbose,backend=biber]{biblatex}
\addbibresource{secure-computation-bibliography.bib}

\input{slide-macros}

% Title page details: 
\title{Language-Based Security for Low-Level MPC}

\author{Christian Skalka and Joe Near}

\date{September 10, 2024}

\logo{
\includegraphics[width=3cm]{UVM_Logo_Primary_Horiz_G.png}
}

\beamertemplatenavigationsymbolsempty

\begin{document}

% Title page frame
\begin{frame}
    \titlepage 
\end{frame}

% Remove logo from the next slides
\logo{}


% % Outline frame
% \begin{frame}{Outline}
%     \tableofcontents
% \end{frame}

\section{Overview}

\begin{frame}{Background}
  
  \cemph{Secure Multiparty Computation (MPC)} is a paradigm for
  \cemph{confidential computing}.
  \begin{itemize}
  \item Guarantees data privacy.
  \item Software-based protocols (vs., e.g., enclaves).
  \item No trusted 3rd party.
  \end{itemize}
  Many existing deployments: voting, auctions, satellite trajectories, COVID exposure notification, ...\footnote{https://mpc.cs.berkeley.edu}

  \medskip

  Various language frameworks, including MPyC, Fairplay, Viaduct.
\end{frame}

\begin{frame}{Contributions}

  Existing HLLs rely on low-level MPC protocols with correctness
  assumed.
  \begin{itemize}
  \item Low-level protocols possess distinct features, e.g., probabilistic sampling.
  \item Existing proof techniques manual, error prone, complex security model.
  \end{itemize}
  Our contributions:
  \begin{itemize}
  \item Language design for low-level protocol definitions.
  \item (Re)formulation of security mode as \cemph{hyperproperties} (e.g., noninterference).
  \item Semi-automated proof techniques.
    \begin{itemize}
    \item Automatically verify small program components, manually combine for whole-program proofs. 
    \end{itemize}
  \end{itemize}
  
\end{frame}

\section{What's an MPC Protocol}

\begin{frame}{The Ideal Functionality}
  A \emph{protocol} $\prog$ implements an \emph{ideal functionality} $\idealf$,
  defined in terms of operations in a prime field $\mathbb{F}_p$.
  \begin{itemize}
  \item The inputs are \cemph{secret values} of \cemph{clients} in a \emph{federation}
    $\setit{1,...,j}$.
  \item The \cemph{output} is made \emph{publicly available} by $\prog$
  \end{itemize}
  The output may leak secret information to an \cemph{adversary} trying to
  guess secret values.

  \begin{exampleblock}{Consider:}
  $$
  \idealf(x,y) = x * y
  $$
  If the output is odd, the adversary knows both $x$ and $y$ must be, even
  if $x$ and $y$ remain hidden.
  \end{exampleblock}
  
\end{frame}
 
\begin{frame}{Protocol Implementation in Overture}

  Low-level, probabilistic protocols are defined in terms of basic variable types
  (identifiers $x$ distinguish them):
  \begin{itemize}
  \item $\secret{x}$: secret values in $\mathbb{F}_p$.
  \item $\mesg{x}$: messages received during the protocol.
  \item $\flip{x}$: uniformly random samples from $\mathbb{F}_p$ (implemented with
    a random tape semantics).
  \end{itemize}
  We can localize variables and expressions to \emph{clients}, and any information
  exchanged between clients must be through messaging.
  \begin{exampleblock}{Toy Example:}
    $$
    \begin{array}{rcl@{\qquad}l}
      \mx{x}{1} &:=& \elab{(\secret{x} - \flip{x})}{2}  & \textit{(unicast)}\\
      \mx{y}{*} &:=& \elab{(\flip{y} + \mesg{x})}{1}  & \textit{(broadcast)}
    \end{array}
    $$ 
  \end{exampleblock}
\end{frame}

\begin{frame}{(Passive) Threat Model}
  The adversary wants to guess secrets. In addition to knowing the public
  output, the adversary may also corrupt \emph{any} subset of clients $C$:
  \begin{itemize}
  \item adversary has access to the secrets of clients in $C$.
  \item adversary has access to messages received by members of $C$ during execution of $\prog$,
    aka \cemph{adversarial views}.
  \item adversary cannot break the rules of the protol\footnote{Malicious security also
  considered separately in our work.}.
  \end{itemize}

  \begin{alertblock}{Bounded Declassification}
    The ideal functionality defines a declassification ``limit'', any protocol $\prog$
    should not reveal any more through adversarial views.
  \end{alertblock}
\end{frame}

\begin{frame}{Declassification of the Ideal Functionality}

  \begin{exampleblock}{Example: 3-Party $\fplus$ in $\mathbb{F}_2$ (where $\fplus$ is $\exor$ and $\ftimes$ is $\wedge$)}
  $$
  \idealf(\sx{1}{1},\sx{2}{2},\sx{3}{3}) =
  \sx{1}{1}\ \fplus\ \sx{2}{2}\ \fplus\ \sx{3}{3}
  $$
  Suppose in a run of $\prog$ the output of $\prog$ is 1 and $C = \{ 3 \}$ and
  the adversary knows $\sx{3}{3} = 0$.
  Therefore the adversary knows that either:
  $$
  \sx{1}{1} = 0 \text{\ and\ } \sx{2}{2} = 1
  $$
  or:
  $$
  \sx{1}{1} = 1 \text{\ and\ } \sx{2}{2} = 0
  $$
  with equal joint probability of .5. \cemph{Adversarial views from the run of $\prog$ should not
    allow the adversary better guesses.}
  \end{exampleblock}
  
\end{frame}

\section{Conditional Probabilistic Noninterference}
\begin{frame}{Conditional Probabilistic Noninterference}

  \begin{itemize}
    \item The set of all protocol runs under all possible secrets and random tapes
      induces a probability distribution of program variables.
    \item Conditioning on the output and corrupt secrets =  declassification
      bounds.
    \item Conditioning on adversarial views tells us whether they leak information--
      if it changes the distribution of honest secrets.
  \end{itemize}
  
  \begin{alertblock}{Conditional Noninterference}
    The probability of honest secrets conditioned on the protocol output
    should not be changed by conditioning on adversarial views. 
  \end{alertblock}

  \begin{itemize}
  \item Sound for the standard \emph{passive simulator security} model.
  \item A \emph{hyperproperty} that accommodates traditional PL-based security
    mechanisms and theory. 
  \end{itemize}
  
\end{frame}

\section{Example: Shamir Addition}
\begin{frame}{Example: Shamir Addition}

  To implement 3-party summation in any $\mathbb{F}_p$, we can use \emph{Shamir addition}.
  Each party first generates \emph{reconstructive shares}.
  
  \begin{block}{Reconstructive Sharing}
    $$
    \begin{array}{lll}
      \elab{\mesg{s1}}{2} &:=& \elab{(\secret{1} \fminus \locflip \fminus \flip{x})}{1} \\ 
      \elab{\mesg{s1}}{3} &:=& \elab{\flip{x}}{1} \\ 
      \elab{\mesg{s2}}{1} &:=& \elab{(\secret{2} \fminus \locflip \fminus \flip{x})}{2} \\ 
      \elab{\mesg{s2}}{3} &:=& \elab{\flip{x}}{2} \\ 
      \elab{\mesg{s3}}{1} &:=& \elab{(\secret{3} \fminus \locflip \fminus \flip{x})}{3} \\ 
      \elab{\mesg{s3}}{2} &:=& \elab{\flip{x}}{3}
    \end{array}
    $$
  \end{block}
  \begin{itemize}
  \item $\elab{\mesg{s1}}{2}, \elab{\mesg{s1}}{3}, \elab{\locflip}{1}$ are the \emph{shares} of $\sx{1}{1}$,
    etc.
  \item \emph{algebraically} $\sx{1}{1} = \elab{\mesg{s1}}{2} + \elab{\mesg{s1}}{3} + \elab{\locflip}{1}$.
  \item \emph{probabilistically} any 2 out of three shares are in a uniform random distribution.
  \end{itemize}
  
\end{frame}

\begin{frame}{Example contd.: Public Broadcast Reveal}

  \begin{itemize}
   \item \emph{algebraically} the sum of all shares = $\sx{1}{1} + \sx{2}{2} + \sx{3}{3}$    
  \end{itemize}

  \begin{block}{Broadcast Sum-of-Shares}
    $$
    \begin{array}{lll}
      \mx{ss1}{*} &:=& \elab{(\locflip \fplus \mesg{s2} \fplus \mesg{s3})}{1} \\ 
      \mx{ss2}{*} &:=& \elab{(\mesg{s1} \fplus \locflip \fplus \mesg{s3})}{2} \\
      \mx{ss3}{*} &:=& \elab{(\mesg{s1} \fplus \mesg{s2} \fplus \locflip)}{3} 
    \end{array}
    $$
  \end{block}

  \begin{itemize}
  \item Messages $\mx{ss1}$ through $\mx{ss3}$ are \cemph{not probabilistically independent of inputs}, but ...
  \item They \emph{are} \cemph{independent of secrets conditioned on the output}!
  \end{itemize}
  
\end{frame}

\begin{frame}{Example contd.: Output}

  \begin{itemize}
    \item Any $\mx{ss1}{*},\mx{ss2}{*},\mx{ss3}{*}$ such that:
    $$
    \mx{ss1}{*} + \ \mx{ss2}{*} + \ \mx{ss3}{*} = \sx{1}{1} + \sx{2}{2}  + \sx{3}{3}
    $$
    are \cemph{equally likely}.
  \end{itemize}

  \begin{block}{Client Outputs}
    $$
    \begin{array}{lll}
      \out{1} &:=& \elab{(\mesg{ss1} \fplus \mesg{ss2} + \mesg{ss3})}{1}\\
      \out{2} &:=& \elab{(\mesg{ss1} \fplus \mesg{ss2} + \mesg{ss3})}{2}\\
      \out{3} &:=& \elab{(\mesg{ss1} \fplus \mesg{ss2} + \mesg{ss3})}{3}
    \end{array}
    $$
  \end{block}
  
  %\begin{exampleblock}{Example Bad Protocol}
  %  If 1 is honest and accidentally broadcasts its local share:
  %  $$
  %  \mx{leak}{*} := \elab{\locflip}{1}
  %  $$
  %  conditioning on adversarial views now yields probability 1 for
  %  either $\sx{1}{1} = 0 \wedge \sx{2}{2} = 1$ or
  %  $\sx{1}{1} = 1 \wedge \sx{2}{2} = 0$.
  %\end{exampleblock}
  
\end{frame}

\section{Models of a Protocol}

\begin{frame}{Computing Protocol Distributions}

  The probability distribution of a protocol $\prog$ (as a pmf) can be precisely computed
  by interpretation as a \cemph{constraint program}.
  \begin{itemize}
  \item $\mathbb{F}_2$ interpretation as a Datalog program-- set of Least Herbrand Models.
    \begin{itemize}
      \item Random tapes represented as fact base.
    \end{itemize}
  \item $\mathbb{F}_p$ interpretation as SMT constraints\footnote{Ozdemir, A., Kremer, G., Tinelli, C., Barrett, C. (2023). \emph{Satisfiability Modulo Finite Fields}. In: Enea, C., Lal, A. (eds) Computer Aided Verification. CAV 2023. }-- set of all models.
  \end{itemize}
  Set of models can be queried to enforce hyperproperties-- but poor scaling.

  \begin{alertblock}{Hypothesis: Composable Properties}
    Compositional properties of protocol components can be independently verified, tractably.
  \end{alertblock}
  
\end{frame}

\section{Compositional Properties of Circuits}

\begin{frame}{Prelude and Component Definitions}

  The Prelude language allows modular definition of Overture protocol components.
  \begin{itemize}
  \item Function definitions, function calls, structured data.
  \item A \emph{staged} language that generates a complete Overture protocol
    as a side effect.
  \end{itemize}
  
  \begin{exampleblock}{Example: Local Summation}
    $$
    \ttt{sum}(x,y,z,i) \{ \mx{z}{i} := \elab{(\mesg{x} \fplus \mesg{y})}{i} \} 
    $$
  \end{exampleblock}

  In addition to programmer convenience, Prelude supports verification:
  \begin{itemize}
  \item Prelude functions define Overture (sub)protocols that can be automatically verified.
  \item Large protocols can be verified manually by composing function properties. 
  \end{itemize}

\end{frame}

\begin{frame}{Circuits and Gates}

  Low-level protocols such as YGC and GMW implement $\idealf$ as a \cemph{circuit}.
  \begin{itemize}
  \item Composed by combing-- aka ``wiring''-- \cemph{gates}.
  \item Each gate standard definition, supports primitive $\fplus$ and $\ftimes$.
  \item Two input wires, one output wire, values remain encrypted on wire. 
  \end{itemize}

  \begin{exampleblock}{Example: 2-Party GMW in $\mathbb{F}_2$}
    Rep.~invariant: each client in $\setit{1,2}$ holds a share of wire values.

    \medskip
  
    $\ttt{andgmw}(x,y,z)$:
    
    \begin{itemize}
    \item Input shares $\mx{x}{1}$, $\mx{x}{2}$ and $\mx{y}{1}$, $\mx{y}{2}$--
      wire value for $x$ is $\mx{x}{1} + \mx{x}{2}$, etc.
    \item Output shares $\mx{z}{1}$, $\mx{z}{2}$ received during gate execution.
      $$
      \mx{z}{1} + \mx{z}{2} = (\mx{x}{1} + \mx{x}{2}) * (\mx{y}{1} + \mx{y}{2})
      $$
    \end{itemize}

  \end{exampleblock}

\end{frame}

\begin{frame}
  
  \begin{exampleblock}{Semi-Automatically Verifying 2-Party GMW}

    Picking dummy $x,y,z$, we can execute $\ttt{andgmw}(x,y,z)$
    to generate a generic ``and'' gate $\prog_{\wedge}$ in Overture.
    \begin{itemize}
    \item Automatically calculate the pmf of $\prog_{\wedge}$.
    \item Pmf can be queried to verify critical properties-- e.g., the
      uniform distribution of output shares under any input conditions. 
    \end{itemize}
    Once properties of gates are verified, we can prove that rep.~invariants
    are maintained in any circuit.
    \begin{itemize}
    \item Manual proofs, but automated tactics and applicability to any circuit.
    \item Leverage logics for reasoning about probabilistic programming: Probabilistic
      Separation Logic, Lilac.
    \end{itemize}
 
  \end{exampleblock}
  
\end{frame}

\section{Conclusion and Future Work}

\logo{
\includegraphics[width=3cm]{UVM_Logo_Primary_Horiz_G.png}
}

\begin{frame}{Future Work}

  \begin{itemize}
  \item Generalize verification to arbitrary prime fields.
  \item Automated type system to eliminate manual proof elements.
  \item Support for concurrency.
  \end{itemize}

\end{frame}

\end{document}
