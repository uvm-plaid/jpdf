% Inbuilt themes in beamer
\documentclass{beamer}

% Theme choice:
\usetheme{CambridgeUS}
\usecolortheme{default}

\usepackage[style=verbose,backend=biber]{biblatex}
\addbibresource{secure-computation-bibliography.bib}

\newcommand{\evals}{\Downarrow}
\newcommand{\diverges}{\Uparrow}
\newcommand{\intt}{\mathrm{int}}
\newcommand{\unitt}{\mathrm{unit}}
\newcommand{\boolt}{\mathrm{bool}}
\newcommand{\floatt}{\mathrm{float}}
\newcommand{\stringt}{\mathrm{string}}
\newcommand{\chart}{\mathrm{char}}
\newcommand{\vb}[1]{\verb+#1+}
\newcommand{\evalexmp}[2]{\texttt{#1}\ \ensuremath{\evals}\ \texttt{#2}}
\newcommand{\texmp}[2]{\texttt{#1\ :\ #2}}
\newcommand{\skipper}{\bigskip\\}
\newcommand{\fyi}{\noindent\textbf{\textit{fyi:}}\ }
\newcommand{\NB}{\noindent\textbf{NB:\ }}
\newcommand{\const}{\ensuremath{\mathbf{c}}}
\newcommand{\defn}{\heading{definition}}
\newcommand{\defeq}{\triangleq}
\newcommand{\nat}{\mathbb{N}}
\newcommand{\atom}{\texttt{const}}
\def\squareforqed{\hbox{\rlap{$\sqcap$}$\sqcup$}}
\def\qed{\ifmmode\squareforqed\else{\unskip\nobreak\hfil
\penalty50\hskip1em\null\nobreak\hfil\squareforqed
\parfillskip=0pt\finalhyphendemerits=0\endgraf}\fi}
\newcommand{\exampletab}[1]{\skipper\begin{tabular}{lll}#1\end{tabular}\skipper}
\newcommand{\verbtab}[1]{\skipper\begin{verbatimtab}{#1}\end{verbatimtab}\skipper}
\newcommand{\eqntab}[1]{\skipper\begin{tabular}{rcl}#1\end{tabular}\skipper}
\newcommand{\recdefn}[1]{\{#1\}}
\newcommand{\ttt}[1]{\texttt{#1}}
\newcommand{\gdesc}[1]{\text{\textit{#1}}}
\newcommand{\true}{\mathrm{true}}
\newcommand{\false}{\mathrm{false}}
\newcommand{\etrue}{\texttt{true}}
\newcommand{\efalse}{\texttt{false}}
\newcommand{\reval}{\Rightarrow}
\newcommand{\Dand}{\ \mathrm{and}\ }
\newcommand{\Dor}{\ \mathrm{or}\ }
\newcommand{\Dxor}{\ \mathrm{xor}\ }
\newcommand{\Dnot}{\mathrm{not}\ }
\newcommand{\cod}[1]{\llbracket #1 \rrbracket}
\newcommand{\lcod}[2]{\llbracket #1 \rrbracket_{#2}}
\newcommand{\Dplus}{\mathrm{Plus}}
\newcommand{\Dminus}{\mathrm{Minus}}
\newcommand{\Dequal}{\mathrm{=}}
\newcommand{\Dabs}[2]{(\mathrm{Function}\ #1 \rightarrow #2)}
\newcommand{\Dfix}[3]{(\mathrm{Fix}\ #1 . #2 \rightarrow #3)}
\newcommand{\Dite}[3]{\mathrm{If}\ #1\ \mathrm{Then}\ #2\ \mathrm{Else}\ #3}
\newcommand{\dotminus}{\stackrel{.}{-}}
\newcommand{\Dlet}[3]{\mathrm{Let}\ #1 = #2\ \mathrm{In}\ #3}
\newcommand{\Dletrec}[4]{\mathrm{Let\ Rec}\ #1\ #2 = #3\ \mathrm{In}\ #4}
\newcommand{\Dfst}{\mathrm{left}}
\newcommand{\Dsnd}{\mathrm{right}}
\newcommand{\labset}{\mathit{Lab}}
\newcommand{\Drec}[1]{\{ #1 \}}
\newcommand{\linfer}[3]{\inferrule*[right=(\TirName{#1})]{#2}{#3}}
\newcommand{\lab}[1]{\mathrm{#1}}
\newcommand{\loc}{\ell}
\newcommand{\Dref}[1]{\mathrm{Ref}\,#1}
%\newcommand{\store}{\mathcal{M}}
\newcommand{\store}{m}
%\newcommand{\stores}{\bar{\store}}
\newcommand{\stores}{\Sigma}
\newcommand{\config}[2]{\langle #1,#2 \rangle}
\newcommand{\configf}[2]{\begin{array}[t]{l}\langle #1\\,\\ #2 \rangle \end{array}}
\newcommand{\extend}[3]{#1\{#2 \mapsto #3\}}
\newcommand{\emptystore}{\{\}}
\newcommand{\storedefn}[1]{\{#1\}}
\newcommand{\Dret}[1]{\mathrm{Return}\,#1}
\newcommand{\Draise}[1]{\mathrm{Raise}\,#1}
\newcommand{\Dexn}[2]{\#\!#1\,#2}
\newcommand{\Dtry}[3]{\mathrm{Try}\,#1\,\mathrm{With}\,#2 \rightarrow #3}
\newcommand{\xname}{\mathit{exn}}
\newcommand{\Dboolt}{\mathrm{Bool}}
\newcommand{\Dreft}[1]{#1\,\mathrm{ref}}
\newcommand{\reft}[1]{#1\,\mathrm{ref}}
\newcommand{\Dintt}{\mathrm{Int}}
%\newcommand{\tjudge}[3]{#1 \vdash #2 : #3}
\newcommand{\textend}[3]{#1;#2:#3}
\newcommand{\fnty}[2]{#1 \rightarrow #2}
\newcommand{\TDabs}[3]{(\mathrm{Function}\ (#1 : #2) \rightarrow #3)}
\newcommand{\TDfix}[5]{(\mathrm{Fix}\ #1 . (#2 : #3) : #4 \rightarrow #5)}
\newcommand{\TDletrec}[6]{\Dletrec{#1}{#2 : #3}{#4 : #5}{#6}}
\newcommand{\emptyenv}{\varnothing}
\newcommand{\tfail}{\mathbf{fail}}
\newcommand{\tcheck}{\mathrm{TC}}
\newcommand{\tcheckfail}{\mathbf{TypeMismatch}}
\newcommand{\algtab}[1]
{
\vspace*{-3mm}
\begin{tabbing}
\hspace*{12mm}\=\hspace{9mm}\=\hspace{9mm}\=\hspace{6mm}\=\hspace{6mm}\=
\hspace{6mm}\=
#1
\end{tabbing}
}
\newcommand{\eassign}[2]{#1 := #2}
\newcommand{\ederef}[1]{\,!#1}
\newcommand{\declass}[2]{\mathrm{declassify}_{#2}(#1)}
\newcommand{\eendorse}[2]{\mathrm{endorse}_{#2}(#1)}
\newcommand{\lt}{\left\{}
\newcommand{\rt}{\right\}}
\newcommand{\Lt}{\left\{\!\!\right.}
\newcommand{\Rt}{\left.\!\!\right\}}
\newcommand{\tinfer}{\mathit{PT}}
\newcommand{\unify}{\mathit{unify}}
\newcommand{\tsubn}{\varphi}
\newcommand{\scheme}[2]{\forall #1 . #2}
\newcommand{\Dself}{\mathrm{this}}
\newcommand{\Dsuper}{\mathrm{super}}
\newcommand{\Dsend}[3]{#1.#2(#3)}
\newcommand{\Dselect}[2]{#1.#2}
\newcommand{\Demptyclass}{\mathrm{EmptyClass}}
%\newcommand{\Dclass}[3]{\mathrm{Class}\ \mathrm{Extends}\ #1\ \mathrm{Inst}
%\ #2\ \mathrm{Meth}\ #3}
\newcommand{\Dclass}[2]{\mathrm{Class}\ \mathrm{Inst} \ #1\ \mathrm{Meth}\ #2}
\newcommand{\Dobj}[2]{\mathrm{Object}\ \mathrm{Inst}\ #1\ \mathrm{Meth}\ #2}
%\newcommand{\Dclassf}[3]{
%\begin{array}[t]{l}
%\mathrm{Class}\ \mathrm{Extends}\ #1 \\
%\quad \mathrm{Inst}\\
%\qquad #2 \\ 
%\quad \mathrm{Meth}\\
%\qquad #3
%\end{array}
%}
\newcommand{\Dclassf}[3]{
\begin{array}[t]{l}
\mathrm{Class}\\
\quad \mathrm{Inst}\\
\qquad #1 \\ 
\quad \mathrm{Meth}\\
\qquad #2
\end{array}
}
\newcommand{\Dobjf}[2]{
\begin{array}[t]{l}
\mathrm{Object}\\
\quad \mathrm{Inst}\\
\qquad #1 \\ 
\quad \mathrm{Meth}\\
\qquad #2
\end{array}
}
\newcommand{\Dnew}[1]{\mathrm{New}\ #1}
\newcommand{\vtab}[1]{\begin{verbatimtab}[4]#1\end{verbatimtab}}

\newcounter{topiccounter}
\setcounter{topiccounter}{1}
\newcommand{\topic}[1]
    {\noindent \textbf{Topic \arabic{topiccounter}.\ \textit{#1}. } \stepcounter{topiccounter}}


\newcommand{\lcalc}{$\lambda$-calculus}
\newcommand{\redx}{\rightarrow}
\newcommand{\redxs}{\redx^*}
\newcommand{\idfn}{\mathit{ID}}
\newcommand{\mlfn}[2]{\mathrm{fun}\, #1 \rightarrow #2}
\newcommand{\mlrecfn}[3]{\mathrm{fix}\,#1.#2 \rightarrow #3}
\newcommand{\mlfix}{\mathrm{fix}}
\newcommand{\eite}[3]{\mathrm{if}\ #1\ \mathrm{then} \ #2\ \mathrm{else} \ #3\ }
\newcommand{\esucc}[1]{\texttt{succ}\ #1}
\newcommand{\epred}[1]{\texttt{pred}\ #1}
\newcommand{\eiszero}[1]{\texttt{iszero}\ #1}
\newcommand{\ezero}{\texttt{0}}
\newcommand{\elet}[3]{\mathrm{let}\ #1 = #2\ \mathrm{in}\ #3}
\newcommand{\eletrec}[3]{\mathrm{letrec}\ #1 = #2\ \mathrm{in}\ #3}
\newcommand{\fv}{\mathrm{fv}}
\newcommand{\ourml}{\mathit{ML}_{\mathit{Cat}}}
\newcommand{\raisexn}{\mathrm{raise}}
\newcommand{\handler}[3]{\mathrm{try}\, #1\, \mathrm{with}\, \exn(#2) \Rightarrow #3}
\newcommand{\exn}{\mathit{exn}}
\newcommand{\dom}{\mathrm{dom}}
\newcommand{\rdom}{\mathrm{rdom}}
\newcommand{\efst}{\mathrm{fst}}
\newcommand{\esnd}{\mathrm{snd}}
\newcommand{\natt}{\textrm{Nat}}
\newcommand{\earray}{\mathrm{array}}
\newcommand{\varray}{\alpha}
\newcommand{\length}{\mathit{length}}
\newcommand{\arrayml}{\ourml^{\earray}}
\newcommand{\stackml}{\ourml^{\mathit{stack}}}
\newcommand{\flowml}{\ourml^{\mathit{flow}}}
\newcommand{\taintml}{\ourml^{\mathit{taint}}}
\newcommand{\secfail}{\mathbf{secfail}}
\newcommand{\tr}{\theta}
\newcommand{\rewrite}[1]{\mathcal{R}(#1)}
%\newcommand{\secprop}{\mathcal{P}}
%\newcommand{\trprop}{\hat{\secprop}}
\newcommand{\Prop}{\mathbf{P}}
\newcommand{\Hprop}{\mathbf{H}}
\newcommand{\secprop}{\phi}
\newcommand{\hyprop}{\eta}
\newcommand{\trprop}{\gamma}
\newcommand{\tracess}{\Sigma}
\newcommand{\trsprop}{\sigma}
\newcommand{\traces}{\Psi}
\newcommand{\fpkeyword}[1]{\mathrm{#1}}
\newcommand{\ebinop}[2]{#1\,\mathit{binop}\,#2}
\newcommand{\eenablepriv}[2]{\fpkeyword{enable}\ #1\ \fpkeyword{for}\ #2}
\newcommand{\echeckpriv}[2]{\fpkeyword{check}\ #1\ \fpkeyword{then}\ #2}
\newcommand{\esigned}[2]{#1.#2}
\newcommand{\enabled}{\mathit{enabledprivs}}
%\newcommand{\acl}{\mathcal{A}}
\newcommand{\priv}{\pi}
\newcommand{\privs}{\mathit{R}}
\newcommand{\prin}{p}
\newcommand{\nobody}{\mathit{nobody}}
\newcommand{\po}{\preceq}
\newcommand{\seclattice}{\mathcal{S}}
\newcommand{\binsl}{\mathcal{S}_{\mathrm{bin}}}
\newcommand{\seclevs}{\mathcal{L}}
\newcommand{\latel}{\varsigma}
\newcommand{\hilab}{H}
\newcommand{\lolab}{L}
\newcommand{\hiloc}{\mathit{hi}}
\newcommand{\loloc}{\mathit{low}}
\newcommand{\labty}[2]{#1 \cdot #2}
\newcommand{\labval}[2]{#1 \cdot #2}
\newcommand{\mi}[1]{\mathit{#1}}
\newcommand{\pc}{\latel_{\mathit{pc}}}
\newcommand{\cfnty}[3]{#1 \rightarrow_{#2} #3}
\newcommand{\pow}{\mathrm{pow}}

%\newcommand{\tr}{\theta}


\newcommand{\chash}{\mathcal{H}}
\newcommand{\acl}{\mathit{Auth}}
\newcommand{\opn}{\mathit{op}}
\newcommand{\egid}{\mathit{egid}}
\newcommand{\euid}{\mathit{euid}}
\newcommand{\suid}{\ttt{suid}}
\newcommand{\sgid}{\ttt{sgid}}
\newcommand{\uxroot}{\ttt{root}}
\newcommand{\fowner}[1]{\mathit{owner}_{#1}}
\newcommand{\fgroup}[1]{\mathit{group}_{#1}}
\newcommand{\gprivs}[1]{\mathit{Privs_{#1}}.\mathit{group}}
\newcommand{\uprivs}[1]{\mathit{Privs_{#1}}.\mathit{owner}}
\newcommand{\oprivs}[1]{\mathit{Privs_{#1}}.\mathit{other}}
\newcommand{\uxprivs}[1]{\mathit{Privs_{#1}}}

\newcommand{\seclab}{\mathcal{L}}
\newcommand{\sle}{\preceq}
\newcommand{\ile}{\preceq_I}
\newcommand{\ilab}{\seclab_I}

\newcommand{\minifed}{\mathit{Overture}}
\newcommand{\minicat}{\minifed}
\newcommand{\fedprot}{\minifed}
\newcommand{\metaprot}{\mathit{Prelude}}
\newcommand{\mlscat}{\mathit{mlscat}}
\newcommand{\flowcat}{\mathit{flowcat}}
\newcommand{\dflowcat}{\mathit{dflowcat}}
\newcommand{\minicatde}{\mathit{minicat}_{\mathit{de}}}
\newcommand{\minicatexp}{\mathit{minicat}_{\mathit{taint}}}
%\newcommand{\prog}{\mathcal{P}}
\newcommand{\prog}{\pi}
\newcommand{\main}{\mathit{main}}
\renewcommand{\reval}{\redx}
\renewcommand{\Dite}{\eite}


%\renewcommand{\labty}[2]{#2}
\newcommand{\fnsty}{\Sigma}
\newcommand{\secty}{\latel}

\newcommand{\tc}{\mathrm{TC}}
\newcommand{\validate}{\mathrm{validate}}


\newcommand{\mlsid}[1]{\mathrm{mls}(#1)}
\newcommand{\mlsredx}[1]{\redx_{\mlsid{#1}}}
\newcommand{\confid}{\mathit{flow}}
\newcommand{\taintid}{\mathit{dflow}}
\newcommand{\credx}{\redx_{\confid}}
\newcommand{\tredx}{\redx_{\taintid}}
\newcommand{\ccod}[1]{\lcod{\confid}{#1}}
\newcommand{\tcod}[1]{\lcod{\taintid}{#1}}
\renewcommand{\mod}{\ \textrm{mod}\ }


\newcommand{\mtrace}[1]{\mathit{trace}_{#1}}
\newcommand{\mtraces}[1]{\mathit{traces}_{#1}}
\newcommand{\head}{\mathit{hd}}
\newcommand{\memt}{\mathit{mems}}

\newcommand{\bop}{\ \mathit{binop}\ }
\newcommand{\ak}{K}
\newcommand{\kernel}[2]{#1^{-1}(#2)}
%\newcommand{\deassign}[2]{\eassign{#1}{\mathrm{declassify}(#2)}
\newcommand{\deassign}[2]{#1 :=  [#2]_\wedge }
%\newcommand{\deassign}[2]{#1\ \wedge\!\,=  #2}

\newcommand{\mems}{\mathit{mems}}
\newcommand{\mto}{\mapsto}
\newcommand{\pdf}[1]{D_{#1}}
\newcommand{\margd}[2]{{#1}_{#2}}
\newcommand{\condd}[3]{#1_{({#2}|{#3})}}
\newcommand{\progd}{\mathrm{PD}}
\newcommand{\progtt}{\mathrm{BD}}
\newcommand{\vars}{\mathit{vars}}
\newcommand{\iov}{\mathit{iovars}}
\newcommand{\flips}{\mathit{flips}}
\newcommand{\keys}{\mathit{keys}}
\newcommand{\fedcat}{\minifed}

\newcommand{\sx}[2]{\elab{\secret{#1}}{#2}}
\newcommand{\mx}[2]{\elab{\mesg{#1}}{#2}} 
%\newcommand{\px}[2]{\elab{\rvl{#1}}{#2}} 
\newcommand{\rx}[2]{\elab{\flip{#1}}{#2}}
\newcommand{\ox}[1]{\out{#1}}

\newcommand{\IF}[1]{#1_{\mathit{i}}}
\newcommand{\idealf}{\mathcal{F}}
\newcommand{\SIM}{\mathrm{Sim}}
\newcommand{\prob}{\mathrm{Pr}}
\newcommand{\dist}{\mathrm{D}}

\def\TirName#1{\text{\sc #1}}

\newcommand{\srct}{\tau}
\newcommand{\cidty}[1]{\ttt{cid(}#1\ttt{)}}
\newcommand{\stringty}[1]{\ttt{string(}#1\ttt{)}}
\newcommand{\unity}{\mathtt{unit}}
\newcommand{\jpdty}[2]{\mathtt{jpd}(#1,#2)}
\newcommand{\viewst}{\mathcal{V}}
\newcommand{\tjudge}[5]{#1, #2 \vdash #3 : #4,#5}
\newcommand{\bet}[1]{\ttt{bool[}#1\ttt{]}}
\newcommand{\tas}{\mathcal{A}}


\newcommand{\flip}[2]{\ttt{flip[}#1\ttt{,}#2\ttt{]}}
\newcommand{\secret}[2]{\ttt{s[}#1\ttt{,}#2\ttt{]}}
\newcommand{\view}[2]{\ttt{v[}#1\ttt{,}#2\ttt{]}}
\newcommand{\oracle}[1]{\ttt{H[}#1\ttt{]}}
\newcommand{\Oracle}{H}
\renewcommand{\etrue}{\ttt{true}}
\renewcommand{\efalse}{\ttt{false}}
\newcommand{\enot}{\ttt{not}}
\newcommand{\eand}{\ttt{and}}
\newcommand{\eor}{\ttt{or}}
\newcommand{\exor}{\ttt{xor}}
\renewcommand{\elet}[3]{\ttt{let}\ #1\ \ttt{=}\ #2\ \ttt{in}\ #3}
\newcommand{\vloc}[2]{#1@#2}
\renewcommand{\redx}{\xrightarrow{}}
\renewcommand{\redxs}{\xrightarrow{}^{*}}
\newcommand{\lredx}[1]{\xrightarrow{#1}}
\newcommand{\mem}{M}
\newcommand{\randos}{R}
\newcommand{\tape}{\randos}
\newcommand{\secrets}{\mathit{secrets}}
\newcommand{\outputs}{\mathit{outputs}}
\newcommand{\clients}{C}
\newcommand{\views}{\mathit{views}}
\newcommand{\str}{\varsigma}
\newcommand{\cid}{\iota}
\newcommand{\send}[2]{#1\ \ttt{:=}\ #2}
\newcommand{\msend}[4]{\elab{\mesg{#1}}{#2}\ \ttt{:=}\ \elab{#3}{#4}}
\newcommand{\OT}[3]{\ttt{OT(} #1 \ttt{,}\ #2 \ttt{,}\ #3 \ttt{)}}
\newcommand{\select}[3]{\ttt{select(} #1 \ttt{,}\ #2 \ttt{,}\ #3 \ttt{)}}
\newcommand{\codebase}{\mathcal{C}}
\newcommand{\interp}[1]{\llbracket #1 \rrbracket}
\newcommand{\finterp}[2]{\llbracket #1 \rrbracket_{#2}}
\newcommand{\prot}{\prog}
\newcommand{\Tapes}{\mathcal{R}}
\newcommand{\outloc}{\mathit{output}}
\newcommand{\pdist}{\mathit{pd}}
\newcommand{\genpdf}{\mathrm{PD}}
\newcommand{\card}[1]{|#1|}
\newcommand{\setdefn}[2]{\{#1\ |\ #2 \}}
\newcommand{\tapes}{\mathit{tapes}}
\newcommand{\nimo}{\mathit{NIMO}}
\newcommand{\pni}{\mathit{PNI}}
\newcommand{\passec}{PS}
\newcommand{\parties}{\mathcal{P}}
\newcommand{\iout}{\mathit{output}}
\newcommand{\kideal}{k_i}
\newcommand{\jpdf}{\mathrm{pdf}}
\newcommand{\leakproof}{\mathit{LP}}
\newcommand{\flab}{\ell}
\newcommand{\be}{\varepsilon}
\newcommand{\instr}{\mathbf{c}}
\newcommand{\solvealg}{\mathit{models}}
\newcommand{\solve}[3]{\solvealg\ #1\ #2\ #3}
\newcommand{\itv}{\mathit{it}}
\newcommand{\outv}{\mathit{out}}
\newcommand{\NIMO}{\mathit{NIMO}}
\newcommand{\gNIMO}{\mathit{gNIMO}}
\newcommand{\gates}{\mathit{gates}}
\newcommand{\owl}{\mathit{owl}}
\newcommand{\logit}[1]{\lfloor #1 \rfloor}
\newcommand{\runs}{\mathit{runs}}
\newcommand{\cruns}{\hat{\mathit{runs}}}
\newcommand{\cprogd}{\hat{\progd}}
\newcommand{\cprogtt}{\hat{\progtt}}
\newcommand{\datalog}{\mathit{datalog}}
%\newcommand{\concat}{\ttt{|\!|}}
\newcommand{\concat}{\ttt{++}}
\newcommand{\wired}{\mathit{wired}}
\newcommand{\gc}[3]{\mathit{goc}(#1,#2,#3)}
\newcommand{\vc}[3]{#1 \vdash #2 \sim #3}
\newcommand{\detx}[1]{\mathbf{D}(#1)}
\newcommand{\unix}[1]{\mathbf{U}(#1)}
\newcommand{\sep}[3]{#1 \vdash #2 * #3}
\newcommand{\condp}[3]{#1|#2 \vdash #3}
\newcommand{\conddetx}[3]{\condp{#1}{#2}{\detx{#3}}}
\newcommand{\condsep}[4]{\condp{#1}{#2}{#3 * #4}}
\newcommand{\condunix}[3]{\condp{#1}{#2}{\unix{#3}}}
\newcommand{\gtab}{\mathit{table}}
\newcommand{\vdefs}{\mathit{vdefs}}
\newcommand{\funcVar}{\$}
%\newcommand{\pmf}{\mathrm{Pr}}
\newcommand{\pmf}{\mathit{P}}

%%%% REVISION DEFS

\renewcommand{\flip}[1]{\ttt{r[}#1\ttt{]}}
\newcommand{\locflip}{\ttt{r[}\mathtt{local}\ttt{]}}
\renewcommand{\secret}[1]{\ttt{s[}#1\ttt{]}}
\newcommand{\ke}[1]{\ttt{k[}#1\ttt{]}}
\newcommand{\mesg}[1]{\ttt{m[}#1\ttt{]}}
\newcommand{\out}[1]{\elab{\ttt{out}}{#1}}
\newcommand{\rvl}[1]{\ttt{p[}#1\ttt{]}}
\renewcommand{\oracle}[1]{\ttt{H[}#1\ttt{]}}
%\newcommand{\elab}[2]{#1_{#2}}
\newcommand{\elab}[2]{#1\ttt{@}#2}
\renewcommand{\eassign}[4]{\elab{#1}{#2} := \elab{#3}{#4}}
\newcommand{\xassign}[3]{#1 := \elab{#2}{#3}}
\newcommand{\pubout}[3]{\out{#1} := \elab{#2}{#3}}
\newcommand{\reveal}[3]{\rvl{#1} := \elab{#2}{#3}}
\newcommand{\sk}[1]{\mathrm{sk}[#1]}
\newcommand{\pk}[2]{\mathrm{pk}[#1,#2]}
\newcommand{\kgen}[1]{\mathit{kgen}(#1)}
\newcommand{\adversary}{\mathcal{A}}
\newcommand{\aredx}{\redx_{\adversary}}
\newcommand{\aredxs}{\redxs_{\adversary}}
\newcommand{\arewrite}{\mathit{rewrite}_{\adversary}}
\newcommand{\cinputs}{V_{C \rhd H}}
\newcommand{\houtputs}{V_{H \rhd C}}
\newcommand{\aruns}{\mathit{runs}_\adversary}
\newcommand{\botruns}{\mathit{runs}_{\adversary,\bot}}
\newcommand{\att}{\mathrm{AD}}
\newcommand{\support}{\mathit{support}}
\renewcommand{\store}{\sigma}
\newcommand{\ctxt}[2]{\{ #1 \}_{#2}}
\newcommand{\cpub}{\mathit{pub}}
\renewcommand{\runs}{\mathit{runs}}
\newcommand{\pattern}[1]{\lfloor #1 \rfloor}
\newcommand{\fcod}[1]{\lcod{#1}{}}
\renewcommand{\flips}{\mathit{rands}}
\newcommand{\kmat}{\kappa}
\renewcommand{\Oracle}{\mathbb{O}}
\newcommand{\afilter}{\mathit{afilter}}
\renewcommand{\select}[3]{\mathtt{if}\ #1\ \mathtt{then}\ #2\ \mathtt{else}\ #3}
\newcommand{\fp}{\mathit{P}}
\newcommand{\ftimes}{*}
\newcommand{\fplus}{+}
\newcommand{\fminus}{-}
\newcommand{\mactimes}{\,\hat{\ftimes}\,}%{\otimes}
\newcommand{\macplus}{\,\hat{\fplus}\,}%\oplus}
\newcommand{\macminus}{\,\hat{\fminus}\,}%{\ominus}
\newcommand{\macgv}[1]{\langle #1 \rangle}
\newcommand{\macv}{\hat{v}}
\newcommand{\macx}[2]{\macgv{\elab{ #1 }{#2}}}
\newcommand{\mack}[2]{#1.\ttt{k}_{#2}}
\newcommand{\macshare}[1]{\langle #1 \rangle.\ttt{share}}
\newcommand{\macopen}{\mathrm{open}}
\newcommand{\macauth}{\mathrm{auth}}
\newcommand{\fieldty}{\mathrm{F}}
\newcommand{\cipherty}{\mathit{c}}
\newcommand{\macty}{\hat{\fieldty}}%_{\mathit{mac}}}}
\renewcommand{\unity}[1]{\mathit{U}(#1)}
\renewcommand{\labty}[3]{#1^{#2}_{#3}}
\newcommand{\memenv}{\mathcal{M}}
\newcommand{\tensor}{\multimap}
\newcommand{\lib}{\mathcal{L}}
\newcommand{\okt}{\mathit{OK}}
\newcommand{\vty}{t}
\newcommand{\disty}{\dot{\vty}}
\newcommand{\tlev}[1]{\mathcal{T}(#1)}
\newcommand{\otp}{\mathrm{sum}}
\newcommand{\macotp}{\hat{\mathrm{minus}}}
\newcommand{\preproc}{\mathit{preproc}}
\newcommand{\assert}[1]{\ttt{assert(}#1\ttt{)}}
\newcommand{\mv}{\nu}
\newcommand{\andgmw}{\ttt{andgmw}}
\newcommand{\decodegmw}{\ttt{decodegmw}}
\newcommand{\bodies}{\mathit{bodies}}

\long\def\cnote#1{{\small\textbf{\textit{\color{red}(*#1 -- Chris*)}}}}
\long\def\jnote#1{{\small\textbf{\textit{\color{brown}(*#1 -- Joe*)}}}}


\newcommand{\mynote}[2]
    {{\color{red} \fbox{\bfseries\sffamily\scriptsize#1}
    {\small$\blacktriangleright$\textsf{\emph{#2}}$\blacktriangleleft$}}~}
%\newcommand{\mynote}[2]{}
\newcommand{\todo}[1]{\mynote{TODO}{#1}}

\newcommand{\cemph}[1]{\textcolor{darkred}{\emph{#1}}}
\newcommand{\setit}[1]{\{ #1 \}}


% Title page details: 
\title{Language-Based Security for Low-Level MPC}

\author{Christian Skalka and Joe Near}

\date{September 10, 2024}

\logo{
\includegraphics[width=3cm]{UVM_Logo_Primary_Horiz_G.png}
}

\beamertemplatenavigationsymbolsempty

\begin{document}

% Title page frame
\begin{frame}
    \titlepage 
\end{frame}

% Remove logo from the next slides
\logo{}


% % Outline frame
% \begin{frame}{Outline}
%     \tableofcontents
% \end{frame}

\section{Overview}

\begin{frame}{Background}
  
  \cemph{Secure Multiparty Computation (MPC)} is a paradigm for
  \cemph{confidential computing}.
  \begin{itemize}
  \item Guarantees data privacy.
  \item Software-based protocols (vs., e.g., enclaves).
  \item No trusted 3rd party.
  \end{itemize}
  Many existing deployments: voting, auctions, satellite trajectories, COVID exposure notification, ...\footnote{https://mpc.cs.berkeley.edu}

  \medskip

  Various language frameworks, including MPyC, Fairplay, Viaduct.
\end{frame}

\begin{frame}{Contributions}

  Existing HLLs rely on low-level MPC protocols with correctness
  assumed.
  \begin{itemize}
  \item Low-level protocols possess distinct features, e.g., probabilistic sampling.
  \item Existing proof techniques manual, error prone, complex security model.
  \end{itemize}
  Our contributions:
  \begin{itemize}
  \item Language design for low-level protocol definitions.
  \item (Re)formulation of security model as \cemph{hyperproperties} (e.g., noninterference).
  \item Semi-automated proof techniques.
    \begin{itemize}
    \item Automatically verify small program components, manually combine for whole-program proofs. 
    \end{itemize}
  \end{itemize}
  
\end{frame}

\section{What's an MPC Protocol}

\begin{frame}{The Ideal Functionality}
  A \emph{protocol} $\prog$ implements an \emph{ideal functionality} $\idealf$,
  defined in terms of operations in a prime field $\mathbb{F}_p$.
  \begin{itemize}
  \item The inputs are \cemph{secret values} of \cemph{clients} in a \emph{federation}
    $\setit{1,...,j}$.
  \item The \cemph{output} is made \emph{publicly available} by $\prog$
  \end{itemize}
  \cemph{The output may leak secret information to an adversary} trying to
  guess secret values.

  \begin{exampleblock}{Consider:}
  $$
  \idealf(x,y) = x * y
  $$
  If the output is odd, the adversary knows both $x$ and $y$ must be, even
  if $x$ and $y$ remain hidden.
  \end{exampleblock}
  
\end{frame}
 
\begin{frame}{Protocol Implementation in Overture}

  Low-level, probabilistic protocols are defined in terms of basic variable types
  (identifiers $x$ distinguish them):
  \begin{itemize}
  \item $\secret{x}$: secret values in $\mathbb{F}_p$.
  \item $\mesg{x}$: messages received during the protocol.
  \item $\flip{x}$: uniformly random samples from $\mathbb{F}_p$ (implemented with
    a random tape semantics).
  \end{itemize}
  We can localize variables and expressions to \emph{clients}, and any information
  exchanged between clients must be through messaging.
  \begin{exampleblock}{Toy Example:}
    $$
    \begin{array}{rcl@{\qquad}l}
      \mx{x}{1} &:=& \elab{(\secret{x} - \flip{x})}{2}  & \textit{(unicast)}\\
      \mx{y}{*} &:=& \elab{(\flip{y} + \mesg{x})}{1}  & \textit{(broadcast)}
    \end{array}
    $$ 
  \end{exampleblock}
\end{frame}

\begin{frame}{(Passive) Threat Model}
  The adversary wants to guess secrets. In addition to knowing the public
  output, the adversary may also corrupt \emph{any} subset of clients $C$:
  \begin{itemize}
  \item adversary has access to the secrets of clients in $C$.
  \item adversary has access to messages received by members of $C$ during execution of $\prog$,
    aka \cemph{adversarial views}.
  \item adversary cannot break the rules of the protocol\footnote{Malicious security also
  considered separately in our work.}.
  \end{itemize}

  \begin{alertblock}{Bounded Declassification}
    The ideal functionality defines a declassification ``limit'', any protocol $\prog$
    should not reveal any more through adversarial views.
  \end{alertblock}
\end{frame}

\begin{frame}{Declassification of the Ideal Functionality}

  \begin{exampleblock}{Example: 3-Party $\fplus$} %in $\mathbb{F}_2$ (where $\fplus$ is $\exor$ and $\ftimes$ is $\wedge$)}
  $$
  \idealf(\sx{1}{1},\sx{2}{2},\sx{3}{3}) =
  \sx{1}{1}\ \fplus\ \sx{2}{2}\ \fplus\ \sx{3}{3}
  $$
  Suppose in a run of $\prog$ the output of $\prog$ is 1 and $C = \{ 3 \}$ and
  the adversary knows $\sx{3}{3} = 0$.
  Therefore the adversary knows that either:
  $$
  \sx{1}{1} = 0 \text{\ and\ } \sx{2}{2} = 1
  $$
  or:
  $$
  \sx{1}{1} = 1 \text{\ and\ } \sx{2}{2} = 0
  $$
  with equal joint probability of .5. \cemph{Adversarial views from the run of $\prog$ should not
    allow the adversary better guesses.}
  \end{exampleblock}
  
\end{frame}

\section{Conditional Probabilistic Noninterference}
\begin{frame}{Conditional Probabilistic Noninterference}

  \begin{itemize}
    \item The set of all protocol runs under all possible secrets and random tapes
      induces a probability distribution of program variables.
    \item Conditioning on the output and corrupt secrets =  declassification
      bounds.
    \item Conditioning on adversarial views tells us whether they leak information--
      if it changes the distribution of honest secrets.
  \end{itemize}
  
  \begin{alertblock}{Conditional Noninterference}
    The probability of honest secrets conditioned on the protocol output
    should not be changed by conditioning on adversarial views. 
  \end{alertblock}

  \begin{itemize}
  \item Sound for the standard \emph{passive simulator security} model.
  \item A \emph{hyperproperty} that accommodates traditional PL-based security
    mechanisms and theory. 
  \end{itemize}
  
\end{frame}

\section{Example: Additive Sharing}
\begin{frame}{Example: Additive Secret Shares}

  To implement 3-party summation in any $\mathbb{F}_p$, we can use \emph{additive sharing}.
  Each party first generates \emph{reconstructive shares}.
  
  \begin{block}{Reconstructive Sharing}
    $$
    \begin{array}{lll}
      \elab{\mesg{s1}}{2} &:=& \elab{(\secret{1} \fminus \locflip \fminus \flip{x})}{1} \\ 
      \elab{\mesg{s1}}{3} &:=& \elab{\flip{x}}{1} \\ 
      \elab{\mesg{s2}}{1} &:=& \elab{(\secret{2} \fminus \locflip \fminus \flip{x})}{2} \\ 
      \elab{\mesg{s2}}{3} &:=& \elab{\flip{x}}{2} \\ 
      \elab{\mesg{s3}}{1} &:=& \elab{(\secret{3} \fminus \locflip \fminus \flip{x})}{3} \\ 
      \elab{\mesg{s3}}{2} &:=& \elab{\flip{x}}{3}
    \end{array}
    $$
  \end{block}
  \begin{itemize}
  \item $\elab{\mesg{s1}}{2}, \elab{\mesg{s1}}{3}, \elab{\locflip}{1}$ are the \emph{shares} of $\sx{1}{1}$,
    etc.
  \item \emph{algebraically} $\sx{1}{1} = \elab{\mesg{s1}}{2} + \elab{\mesg{s1}}{3} + \elab{\locflip}{1}$.
  \item \emph{probabilistically} any 2 out of three shares are in a uniform random distribution.
  \end{itemize}
  
\end{frame}

\begin{frame}{Example contd.: Public Broadcast Reveal}

  \begin{itemize}
   \item \emph{algebraically} the sum of all shares = $\sx{1}{1} + \sx{2}{2} + \sx{3}{3}$    
  \end{itemize}

  \begin{block}{Broadcast Sum-of-Shares}
    $$
    \begin{array}{lll}
      \mx{ss1}{*} &:=& \elab{(\locflip \fplus \mesg{s2} \fplus \mesg{s3})}{1} \\ 
      \mx{ss2}{*} &:=& \elab{(\mesg{s1} \fplus \locflip \fplus \mesg{s3})}{2} \\
      \mx{ss3}{*} &:=& \elab{(\mesg{s1} \fplus \mesg{s2} \fplus \locflip)}{3} 
    \end{array}
    $$
  \end{block}

  \begin{itemize}
  \item Messages $\mx{ss1}$ through $\mx{ss3}$ are \cemph{not probabilistically independent of inputs}, but ...
  \item They \emph{are} \cemph{independent of secrets conditioned on the output}!
  \end{itemize}
  
\end{frame}

\begin{frame}{Example contd.: Output}

  \begin{itemize}
    \item Any $\mx{ss1}{*},\mx{ss2}{*},\mx{ss3}{*}$ such that:
    $$
    \mx{ss1}{*} + \ \mx{ss2}{*} + \ \mx{ss3}{*} = \sx{1}{1} + \sx{2}{2}  + \sx{3}{3}
    $$
    are \cemph{equally likely}.
  \end{itemize}

  \begin{block}{Client Outputs}
    $$
    \begin{array}{lll}
      \out{1} &:=& \elab{(\mesg{ss1} \fplus \mesg{ss2} + \mesg{ss3})}{1}\\
      \out{2} &:=& \elab{(\mesg{ss1} \fplus \mesg{ss2} + \mesg{ss3})}{2}\\
      \out{3} &:=& \elab{(\mesg{ss1} \fplus \mesg{ss2} + \mesg{ss3})}{3}
    \end{array}
    $$
  \end{block}
  
  %\begin{exampleblock}{Example Bad Protocol}
  %  If 1 is honest and accidentally broadcasts its local share:
  %  $$
  %  \mx{leak}{*} := \elab{\locflip}{1}
  %  $$
  %  conditioning on adversarial views now yields probability 1 for
  %  either $\sx{1}{1} = 0 \wedge \sx{2}{2} = 1$ or
  %  $\sx{1}{1} = 1 \wedge \sx{2}{2} = 0$.
  %\end{exampleblock}
  
\end{frame}

\section{Models of a Protocol}

\begin{frame}{Computing Protocol Distributions}

  The probability distribution of a protocol $\prog$ (as a pmf) can be precisely computed
  by interpretation as a \cemph{constraint program}.
  \begin{itemize}
  \item $\mathbb{F}_2$ interpretation as a Datalog program-- set of Least Herbrand Models.
    \begin{itemize}
      \item Random tapes represented as fact base.
    \end{itemize}
  \item $\mathbb{F}_p$ interpretation as SMT constraints\footnote{Ozdemir, A., Kremer, G., Tinelli, C., Barrett, C. (2023). \emph{Satisfiability Modulo Finite Fields}. In: Enea, C., Lal, A. (eds) Computer Aided Verification. CAV 2023. }-- set of all models.
  \end{itemize}
  Set of models can be queried to enforce hyperproperties-- but poor scaling.

  \begin{alertblock}{Hypothesis: Composable Properties}
    Compositional properties of protocol components can be independently verified, tractably.
  \end{alertblock}
  
\end{frame}

\section{Compositional Properties of Circuits}

\begin{frame}{Prelude and Component Definitions}

  The Prelude language allows modular definition of Overture protocol components.
  \begin{itemize}
  \item Function definitions, function calls, structured data.
  \item A \emph{staged} language that generates a complete Overture protocol
    as a side effect.
  \end{itemize}
  
  \begin{exampleblock}{Example: Local Summation}
    $$
    \ttt{sum}(x,y,z,i) \{ \mx{z}{i} := \elab{(\mesg{x} \fplus \mesg{y})}{i} \} 
    $$
  \end{exampleblock}

  In addition to programmer convenience, Prelude supports verification:
  \begin{itemize}
  \item Prelude functions define Overture (sub)protocols that can be automatically verified.
  \item Large protocols can be verified manually by composing function properties. 
  \end{itemize}

\end{frame}

\begin{frame}{Circuits and Gates}

  Low-level protocols such as YGC and GMW implement $\idealf$ as a \cemph{circuit}.
  \begin{itemize}
  \item Composed by combing-- aka ``wiring''-- \cemph{gates}.
  \item Each gate standard definition, supports primitive $\fplus$ and $\ftimes$.
  \item Two input wires, one output wire, values remain encrypted on wire. 
  \end{itemize}

  \begin{exampleblock}{Example: 2-Party GMW in $\mathbb{F}_2$}
    Rep.~invariant: each client in $\setit{1,2}$ holds a share of wire values.

    \medskip
  
    $\ttt{andgmw}(x,y,z)$:
    
    \begin{itemize}
    \item Input shares $\mx{x}{1}$, $\mx{x}{2}$ and $\mx{y}{1}$, $\mx{y}{2}$--
      wire value for $x$ is $\mx{x}{1} + \mx{x}{2}$, etc.
    \item Output shares $\mx{z}{1}$, $\mx{z}{2}$ received during gate execution.
      $$
      \mx{z}{1} + \mx{z}{2} = (\mx{x}{1} + \mx{x}{2}) * (\mx{y}{1} + \mx{y}{2})
      $$
    \end{itemize}

  \end{exampleblock}

\end{frame}

\begin{frame}
  
  \begin{exampleblock}{Semi-Automatically Verifying 2-Party GMW}

    Picking dummy $x,y,z$, we can execute $\ttt{andgmw}(x,y,z)$
    to generate a generic ``and'' gate $\prog_{\wedge}$ in Overture.
    \begin{itemize}
    \item Automatically calculate the pmf of $\prog_{\wedge}$.
    \item Pmf can be queried to verify critical properties-- e.g., the
      uniform distribution of output shares under any input conditions. 
    \end{itemize}
    Once properties of gates are verified, we can prove that rep.~invariants
    are maintained in any circuit.
    \begin{itemize}
    \item Manual proofs, but automated tactics and applicability to any circuit.
    \item Leverage logics for reasoning about probabilistic programming: Probabilistic
      Separation Logic, Lilac.
    \end{itemize}
 
  \end{exampleblock}
  
\end{frame}

\section{Conclusion and Future Work}

\logo{
\includegraphics[width=3cm]{UVM_Logo_Primary_Horiz_G.png}
}

\begin{frame}{Future Work}

  \begin{itemize}
  \item Generalize verification to arbitrary prime fields.
  \item Automated type system to eliminate manual proof elements.
  \item Support for concurrency.
  \end{itemize}

\end{frame}

\end{document}
