\subsection{2-Party BDOZ and Integrity Enforcement}
\label{section-example-bdoz}

\begin{fpfig}[t]{2-party BDOZ circuit library: sum gates and secure opening.}{fig-bdozsum}
{\footnotesize
\begin{verbatimtab}
  _open(x,i1,i2){
    m[x++"exts"]@i1 := m[x++"s"]@i2;
    m[x++"extm"]@i1 := m[x++"m"]@i2;
    assert(m[x++"extm"] = m[x++"k"] + (m["delta"] * m[x++"exts"]))@i1;
    m[x]@i1 := (m[x++"exts"] + m[x++"s"])@i1
  }
  
  pre: { m[x++"m"]@i2 == m[x++"k"]@i1 + (m["delta"]@i1 * m[x++"s"])@i2 /\
         m[y++"m"]@i2 == m[y++"k"]@i1 + (m["delta"]@i1 * m[y++"s"])@i2 }
  _sum(z, x, y,i1,i2) {
      m[z++"s"]@i2 := (m[x++"s"] + m[y++"s"])@i2;
      m[z++"m"]@i2 := (m[x++"m"] + m[y++"m"])@i2;
      m[z++"k"]@i1 := (m[x++"k"] + m[y++"k"])@i1
  }
  post: { m[z++"m"]@i2 == m[z++"k"]@i1 + (m["delta"]@i1 * m[z++"s"]@i2 } 
  
  sum(z,x,y) { _sum(z,x,y,1,2); _sum(z,x,y,2,1) }

  open(x) { _open(x,1,2); _open(x,2,1) }
\end{verbatimtab}
}
\end{fpfig}

\begin{fpfig}[t]{2-party BDOZ circuit library: multiplication gates.}{fig-bdozmult}
{\footnotesize
\begin{verbatimtab}
  _mult1(z, x, y) {
      m[z++"s"]@1 :=
        (m[z++"bs"] * m[z++"d"] + m[z++"as"] * m[z++"e"] + m[z++"cs"])@1;
      m[z++"m"]@1 :=
        (m[z++"bm"] * m[z++"d"] + m[z++"am"] * m[z++"e"] + m[z++"cm"])@1;
      m[z++"k"]@2 :=
        (m[z++"bk"] * m[z++"d"] + m[z++"ak"] * m[z++"e"] + m[z++"ck"])@2    
  }
  post: { m[z++"m"]@2 == m[z++"k"]@1 + (m["delta"]@1 * m[z++"s"]@1 }

  mult(z,x,y) {
      sum(z++"a", x, z++"d");
      open(z++"d");
      sum(z++"b", y, z++"e");
      open(z++"e"); 
      _mult1(z,x,y); _mult2(z,x,y)
  }
  post: {  m[z++"s"]@1 + m[z++"s"]@2 ==
          (m[x++"s"]@1 + m[x++"s"]@2) * (m[y++"s"]@1 + m[y++"s"]@2)} 
  
\end{verbatimtab}
}
\end{fpfig}

% hints for confidentiality
\begin{comment}
      m[z++"ds"]@1 as m[x++"s"]@2 + r[z++"as"]@2;
      m[z++"ds"]@2 as m[x++"s"]@1 + r[z++"as"]@1;
      m[z++"ms"]@2 as m[z++"k"]@1 + (m["delta"]@1 * m[z++"ds"]@2);
      m[z++"ms"]@1 as m[z++"k"]@2 + (m["delta"]@2 * m[z++"ds"]@1);
\end{comment}


In a malicious setting, ``detecting cheating'' by adding
information-theoretic secure MAC codes to shares is a fundamental
technique pioneered in systems such as BDOZ and SPDZ
\cite{SPDZ1,SPDZ2,BDOZ,10.1007/978-3-030-68869-1_3}.  These protocols
assume a pre-processing phase that distributes shares with MAC codes
to clients.  This integrates well with pre-processed Beaver Triples to
implement malicious secure, and relatively efficient, multiplication
\cite{evans2018pragmatic}. Recall that Beaver triples are values $a,b,c$ with
$\fcod{a * b} = c$, unique per multiplication gate, that are secret
shared with clients during pre-processing. Here we consider the
2-party version.

A field value $v$ is secret shared among 2 clients in BDOZ in the same
manner as in GMW, but with the addition of a separate MAC values.  Each
client $\cid$ gets a pair of values $(v_\cid,m_\cid)$, where $v_\cid$
is a share of $v$ reconstructed by addition, i.e., $v = \fcod{v_1
  \fplus v_2}$, and $m_\cid$ is a MAC of $v_\cid$.  More precisely,
$m_\cid = k + (k_\Delta * v_\cid)$ where \emph{keys} $k$ and
$k_\Delta$ are known only to $\cid' \ne \cid$ and $k_\Delta$. The
\emph{local key} $k$ is unique per MAC, while the \emph{global key}
$k_\Delta$ is common to all MACs authenticated by $\cid'$. This is a
semi-homomorphic encryption scheme that supports addition of shares
and multiplication of shares by a constant-- for more details the
reader is referred to Orsini \cite{10.1007/978-3-030-68869-1_3}. We
note that while we restrict values $v$, $m$, and $k$ to the same field
in this presentation for simplicity, in general $m$ and $k$ can be in
extensions of $\mathbb{F}_p$.

We can capture both the preliminary distribution of Beaver triples and
BDOZ shares as a pre-processing predicate that establishes conditions
for initial memories (see Definition \ref{def-aprogtt}).  Here we
assume two input secrets $\elab{\secret{x}}{1}$ and
$\elab{\secret{y}}{2}$ and a single Beaver Triple to compute
$\elab{\secret{x}}{1} \ftimes \elab{\secret{y}}{2}$, but we can extend
this for additional gates.  As for GMW, we use $\macgv{\mesg{w}}$ to
refer to secret-shared values reconstructed with addition, but for
BDOZ by convention each share of $\macgv{\mesg{w}}$ is represented as
$\elab{\mesg{w\ttt{s}}}{\cid}$, the MAC of which is represented as a
$\elab{\mesg{w\ttt{m}}}{\cid}$ for all $\cid$, 
and each client holds a key $\elab{\mesg{w\ttt{k}}}{\cid}$ for
authentication of the other's share.
\begin{definition}[BDOZ preprocessing]
  Define:
  \begin{mathpar}
    \mathit{shares} \defeq
    \{ \elab{\mesg{w\ttt{s}}}{\cid}\ |\ \cid \in \{ 1, 2 \} \wedge w \in \{ a,b,c,x,y \}  \}

    \mathit{macs} \defeq  \{ \elab{\mesg{w\ttt{m}}}{\cid}\ |\ \cid \in \{ 1, 2 \} \wedge w \in \{ a,b,c,x,y \}  \}

    \mathit{keys} \defeq  \begin{array}{l}\{ \elab{\mesg{w\ttt{k}}}{\cid}\ |\ \cid \in \{ 1, 2 \} \wedge w \in
    \{ a,b,c,x,y \}  \} \cup \\ \{ \elab{\mesg{\ttt{delta}}}{\cid}\ |\ \cid \in \{ 1, 2 \} \} \end{array}
  \end{mathpar}
  Then a memory $\store$ satisfies BDOZ preprocessing iff:
  $$\dom(\store) = \{ \elab{\secret{x}}{1}, \elab{\secret{y}}{2} \} \cup \mathit{shares}
  \cup \mathit{macs} \cup \mathit{keys}$$
  and, writing $\store(\macgv{\mesg{w}})$ to denote
  $\fcod{\store(\elab{\mesg{w\ttt{s}}}{1}) + \store(\elab{\mesg{w\ttt{s}}}{2})}$,
  the following conditions hold:
  \begin{mathpar}
    \store(\macgv{\mesg{x}}) = \store(\elab{\secret{x}}{1})
    
    \store(\macgv{\mesg{y}}) = \store(\elab{\secret{y}}{2})
    
    \fcod{\store(\macgv{\mesg{a}}) * \store(\macgv{\mesg{b}})} = \store(\macgv{\mesg{c}})
  \end{mathpar}
  and for all $\cid,\cid' \in \{1,2\}$ with $\cid \ne \cid'$ and $w \in \{ a,b,c,x,y\}$:
  $$
  \store(\mx{w\ttt{m}}{\cid}) =
    \fcod{\store(\mx{w\ttt{k}}{\cid'}) + \store(\mx{\ttt{delta}}{\cid'}) * \store(\mx{w\ttt{s}}{\cid})}
  $$
  %$$\lcod{\store, \mesg{w\ttt{m}}}{\cid} =
  %\fcod{\lcod{\store, \mesg{w\ttt{k}} + \mesg{\ttt{delta}} * \mesg{w\ttt{m}}}{\cid'}$$
\end{definition}

With these conventions, a BDOZ library is defined in Figure
\ref{fig-bdoz} and a multiplication protocol to compute $\sx{x}{1} *
\sx{y}{2}$ is defined in Figure \ref{fig-beaver}. The latter is
malicious secure assuming that client 1 is trusted. Initially, in
calls to $\ttt{sum\_she}$ each party computes the sum of their shares
of $\sx{x}{1} + \macgv{\mesg{a}}$ and $\sx{y}{2} + \macgv{\mesg{b}}$,
and non-interactively computes the associated MACs and keys in the
SHE authentication scheme. These values are then
securely opened on client 1, which authenticates client
2's share via a call to $\ttt{auth}$ that $\ttt{assert}$s
the BDOZ authentication property. Note that the
opening of $\sx{y}{2} + \macgv{\mesg{b}}$ reveals no
information to client 1 about $\sx{y}{2}$ since $\macgv{\mesg{b}}$ is
in a unform random distribution by assumption. It is
also malicious secure since cheating is detected
by the $\ttt{assert}$.

Following this, each party non-interactively calculates their share of
$\sx{x}{1} \ftimes \sx{y}{2}$, while client 2 calculates the MAC of
their share, and reveals both their share and MAC to client 1. These
are reveals since information about $\sx{y}{2}$ is exposed by
the multiplication.  Client 1 also calculates the appropriate key and
authenticates client 2's share following the BDOZ SHE MAC scheme
\cite{10.1007/978-3-030-68869-1_3} and the standard Beaver protocol
\cite{evans2018pragmatic}. Hence the protocol is correct and
malicious secure given trusted client 1. The authentication scheme
can also be run symmetrically on client 2.

\begin{fpfig}[t]{Authenticated 2-Party Multiplication.}{fig-beaver}
{\footnotesize
  \begin{verbatimtab}
    secopen("a","x","d",1,2);
    secopen("a","x","d",2,1);
    secopen("b","y","e",1,2);
    secopen("b","y","e",2,1);
    let xys =
      macsum(macctimes(macshare("b"), m["d"]),
             macsum(macctimes(macshare("a"), m["e"]),
                    macshare("c")))
    in
    let xyk = mack("b") * m["d"] + mack("d") * m["d"] + mack("c")               
    in
    secreveal(xys,xyk,"1",1,2);
    secreveal(maccsum(xys,m["d"] * m["e"]),
              xyk - m["d"] * m["e"],
              "2",2,1);
    out@1 := (p[1] + p[2])@1;
    out@2 := (p[1] + p[2])@2;
  \end{verbatimtab}
}
\end{fpfig}


\subsection{Cheating Detection and Integrity}

We can carry out similar proofs of passive security for the protocol
in Figure \ref{fig-beaver} as for GMW, even using automated tactics
for the protocol in $\mathbb{F}_2$. But in the case of BDOZ we are
also concerned with malicious security. To demonstrate this, we can
show that the protocol satisfies integrity in the sense of Definition
\ref{def-integrity}. To do so, we observe that it satisfies a stronger
property, that we call cheating detection. Cheating detection says
that the adversary can only execute the protocol honestly, or or else
gets caught (and abort). Note in particular that this is \emph{not}
a form of endorsement (the dual of declassification), which allows
the adversary some leeway. Rather, it is a confirmation of high integrity.

Focusing in, we identify the \emph{adversarial inputs} as the messages
sent from the adversary to honest parties on which honest responses
to the adversary may depend. We want to say that these are the messages
that must be legitimate.
\begin{definition}
  Given $\prog$ with $\iov(\prog) = S \cup V \cup O$,
  let $X_H \subseteq \{ x \mid x \in (\houtputs \cup O_H) \wedge x \in \dom(\store) \}$.
  Then the \emph{adversarial inputs to $X_H$} is the least set
  $X_C \subseteq \cinputs$ such that $\progtt(\prog) \not\vdash X_C * X_H$.
\end{definition}
Now, we can characterize protocols with cheating detection as those where
the adversary can only behave honestly/passively, or else cause abort.
\begin{definition}[Cheating Detection]
  \emph{Cheating is detected} in $\prog$ with $\iov(\prog) = S \cup V \cup O$ iff
  for all  $\store \in \aruns(\prog)$,
  letting $X_H = \{ x \mid x \in (\houtputs \cup O_H) \wedge x \in \dom(\store) \}$,
  and letting $X_C$ be the adversarial inputs to $X_H$,
  there exists $\sigma'\in \runs(\prog)$
  with $\store_{X_C} = \store'_{X_C}$.  
\end{definition}

It is straightforward to demonstrate that cheating detection has integrity,
since only the ``passive'' adversary can elicit a response from honest parties. 
\begin{lemma}
  \label{lemma-cheating}
  If cheating is detected in $\prog$, then $\prog$ has integrity.
\end{lemma}

In the case of BDOZ, cheating detection is accomplished by the information-theoretic
security of the encryption scheme \cite{evans2018pragmatic}. Furthermore, the symmetry of
the protocol in Figure \ref{fig-beaver} ensures that both parties will authenticate
shares, so it is robust to corruption of either party. 
