\subsection{2-Party BDOZ and Integrity Enforcement}
\label{section-example-bdoz}

\begin{fpfig}[t]{2-party BDOZ protocol library.}{fig-bdoz}
{\footnotesize{
\begin{verbatimtab}
auth(s,m,k,i) { assert(m == k + (m["delta"] * s))@i; }
  
sum_she(z,x,y,i) {
  m[z++"s"]@i := (m[x++"s"] + m[y++"s"])@i;
  m[z++"m"]@i := (m[x++"m"] + m[y++"m"])@i;
  m[z++"k"]@i := (m[x++"k"] + m[y++"k"])@i
}

open(x,i1,i2){
  m[x++"exts"]@i1 := m[x++"s"]@i2;
  m[x++"extm"]@i1 := m[x++"m"]@i2;
  auth(m[x++"exts"], m[x++"extm"], m[x++"k"], i1);
  m[x]@i1 := (m[x++"exts"] + m[x++"s"])@i1
}
\end{verbatimtab}
}}
\end{fpfig}


%%%%%%%%% OLD VERSION BELOW
\begin{comment}

\begin{fpfig}[t]{2-Party BDOZ Protocol Library.}{fig-bdoz}
{\footnotesize{
  \begin{verbatimtab}
    macsum(s1,s2)
    { { share = s1.share + s2.share; mac = s1.mac + s2.mac } }
    
    maccsum(s,c)
    { { share = s.share + c; mac = s.mac + c } }
    
    macctimes(s,c)
    { { share = s.share * c; mac = s.mac * c } }
    
    macshare(w) { {  share = m[w]; mac = m[w++"mac"] } }

    mack(w) { m[w++"k"] }
    
    auth(s,m,k,i) { assert(m = k + m["delta"] * s)@i }
    
    secopen(w1,w2,w3,i1,i2)
    {
      let locsum =  macsum(macshare(w1), macshare(w2)) in
      m[w3++"s"]@i1 := (locsum.share)@i2;
      m[w3++"smac"]@i1 := (locsum.mac)@i2;
      auth(m[w3++"s"],m[w3++"smac"],mack(w1) + mack(w2),i1);
      m[w3]@i1 := (m[w3++"s"] + (locsum.share))@i1
    }

    secreveal(s,k,w,i1,i2)
    {
      p[w] = s.share@i2;
      p[w++"mac"] = s.mac@i2;
      auth(p[w],p[w++"mac"],k,i1)    
    } \end{verbatimtab}
}}
\end{fpfig}

\end{comment}
    


In a malicious setting, ``detecting cheating'' by adding
information-theoretic secure MAC codes to shares is a fundamental
approach realized by protocols such as BDOZ \cite{XXX} and SPDZ
\cite{XXX}.  These protocols assume a pre-processing phase that
distributes shares with MAC codes to clients.  This integrates well
with pre-processed Beaver Triples to implement malicious secure, and
relatively efficient, multiplication \cite{XXX}. Recall
that Beaver triples are values $a,b,c$ with $\fcod{a * b} = c$,
unique per multiplication gate, that are secret shared with clients
during pre-processing. Here we consider the 2-party version.

A field value $v$ is secret shared among 2 clients in BDOZ in the same
manner as in GMW.  Each client $\cid$ gets a pair of the form
$(v_\cid,m_\cid)$, where $v_\cid$ is a share of $v$ reconstructed by
addition, i.e., $v = \fcod{v_1 \fplus v_2}$, and $m_\cid$ is a MAC of
$v_\cid$.  More precisely, $m_\cid = k + (k_\Delta * v_\cid)$ where
\emph{keys} $k$ and $k_\Delta$ are known only to $\cid' \ne \cid$ and
$k_\Delta$. The \emph{local key} $k$ is unique per MAC, while the
\emph{global key} $k_\Delta$ is common to all MACs authenticated by
$\cid'$. This is a semi-homomorphic encryption scheme that supports
addition of shares and multiplication of shares by a constant
\cite{XXX}. For more details the reader is referred to \cite{XXX}. We
note that while we restrict values $v$, $m$, and $k$ to the same field
in this presentation for simplicity, in general $m$ and $k$ can be in
extensions of $\mathbb{Z}_p$.

We can capture both the preliminary distribution of Beaver triples and BDOZ shares
as a pre-processing predicate that establishes conditions for initial
memories (see Definition \ref{def-progtt}).  Here we assume two input
secrets $\elab{\secret{x}}{1}$ and $\elab{\secret{y}}{2}$ and a single
Beaver Triple to compute $\elab{\secret{x}}{1} \ftimes
\elab{\secret{y}}{2}$, but we can extend this for additional gates.
As for GMW, we use $\macgv{\mesg{w}}$ to refer to secret-shared values
reconstructed with addition, where by convention shares are message
values $\elab{\mesg{w}}{\cid}$ for all $\cid$.
\begin{definition}[BDOZ preprocessing]
  Define:
  \begin{mathpar}
    \mathit{shares} \defeq
    \{ \elab{\mesg{w}}{\cid}\ |\ \cid \in \{ 1, 2 \} \wedge w \in \{ a,b,c,x,y \}  \}

    \mathit{macs} \defeq  \{ \elab{\mesg{w\ttt{mac}}}{\cid}\ |\ \cid \in \{ 1, 2 \} \wedge w \in \{ a,b,c,x,y \}  \}

    \mathit{keys} \defeq  \begin{array}{l}\{ \elab{\mesg{w\ttt{k}}}{\cid}\ |\ \cid \in \{ 1, 2 \} \wedge w \in
    \{ a,b,c,x,y \}  \} \cup \\ \{ \elab{\mesg{\ttt{delta}}}{\cid}\ |\ \cid \in \{ 1, 2 \} \} \end{array}
  \end{mathpar}
  Then a memory $\store$ satisfies BDOZ preprocessing iff:
  $$\dom(\store) = \{ \elab{\secret{x}}{1}, \elab{\secret{y}}{2} \} \cup \mathit{shares}
  \cup \mathit{macs} \cup \mathit{keys}$$
  and, writing $\store(\macgv{\mesg{w}})$ to denote
  $\fcod{\store(\elab{\mesg{w}}{1}) + \store(\elab{\mesg{w}}{2})}$,
  the following conditions hold:
  \begin{mathpar}
    \store(\macgv{\mesg{x}}) = \store(\elab{\secret{x}}{1})
    
    \store(\macgv{\mesg{y}}) = \store(\elab{\secret{y}}{2})
    
    \fcod{\store(\macgv{\mesg{a}}) * \store(\macgv{\mesg{b}})} = \store(\macgv{\mesg{c}})
  \end{mathpar}
  and for all $\cid,\cid' \in \{1,2\}$ with $\cid \ne \cid'$ and $w \in \{ a,b,c,x,y\}$:
  $$\lcod{\store, \mesg{w\ttt{mac}}}{\cid} =
  \lcod{\store, \mesg{wk} + \mesg{\ttt{delta}} * \mesg{w}}{\cid'}$$
\end{definition}

With these conventions, our BDOZ library is defined in Figure \ref{fig-bdoz}.
In $\metaprot$ we represent BDOZ share, MAC pairs as records:
$$
\ttt{\{ share = }v\ttt{;  mac =}\ m \ttt{\}}
$$
and we define $\ttt{macsum}$ for addition of shares,
$\ttt{maccsum}$ for addition of a share and a constant, and
$\ttt{macctimes}$ for multiplication of a share and a constant
in the BDOZ encryption scheme \cite{XXX}. The $\ttt{auth}$
function implements the MAC check as an $\assert$.

We also define a function $\ttt{secopen}$ to implement ``secure
opening''.  In this standard subprotocol, the value
$\macgv{\secret{w_1}} \fplus \macgv{\mesg{w_2}}$ is reconstructed as
$\mesg{w_3}$, by each client $\cid_2$ communicating
$\lcod{\mesg{w_1} + \mesg{w_2}}{\cid_2}$ to $\cid_1$.  Assuming that
$\macgv{\mesg{w_2}}$ is in an independent uniform distribution,
this perfectly hides $\secret{w_1}$. In a mutiplication gate
either $a$ or $b$ of a Beaver triple are used in secure opening,
so, e.g., given $
(\varnothing,\ttt{secopen}(a, x, d, 2, 1) \redxs (\prog,())$
we have both:
\begin{mathpar}
  \conddetx{\progtt(\prog)}{\{\macgv{\mesg{a}},\elab{\secret{x}}{1} \}}{\{ \elab{\mesg{d}}{2} \}}
  
  \sep{\progtt(\prog)}{\{ \elab{\mesg{d}}{2} \}}{\{ \elab{\secret{x}}{1}  \}}
\end{mathpar}
Furthermore, client 2's authentication of the sum of shares with the
sum of their keys detects any attempted cheating by $1$.
Finally, the $\ttt{secreveal}$ function
is very similar to $\decodegmw$, except with the addition
of authentication of revealed shares to ensure malicious security. 

\begin{fpfig}[t]{Authenticated 2-party multiplication with trusted Client 1.}{fig-beaver}
{\footnotesize
  \begin{verbatimtab}
sum("a","x","d",1,2);
open("d",1,2);
sum("b","y","e",1,2);
open("e",1,2);

p["xys2"] := (m["bs"] * m["d"] + m["as"] * m["e"] + m["cs"])@2
p["xym2"] := (m["bm"] * m["d"] + m["am"] * m["e"] + m["cm"])@2
m["xyk"]@1 := (m["bk"] * m["d"] + m["ak"] * m["e"] + m["ck"])@1;

m["xys"]@1 := (m["bs"] * m["d"] + m["as"] * m["e"] + m["cs"] +
               m["d"] * m["e"])

auth(p["xys2"], p["xym2"], m["xyk"], 1);
out@1 := m["xys"] + p["xys2"]@1;
\end{verbatimtab}
}
\end{fpfig}

\begin{comment}

\begin{fpfig}[t]{Authenticated 2-Party Multiplication.}{fig-beaver}
{\footnotesize
  \begin{verbatimtab}
    secopen("a","x","d",1,2);
    secopen("a","x","d",2,1);
    secopen("b","y","e",1,2);
    secopen("b","y","e",2,1);
    let xys =
      macsum(macctimes(macshare("b"), m["d"]),
             macsum(macctimes(macshare("a"), m["e"]),
                    macshare("c")))
    in
    let xyk = mack("b") * m["d"] + mack("a") * m["d"] + mack("c")               
    in
    secreveal(xys,xyk,"1",1,2);
    secreveal(maccsum(xys,m["d"] * m["e"]),
              xyk - m["d"] * m["e"],
              "2",2,1);
    out@1 := (p[1] + p[2])@1;
    out@2 := (p[1] + p[2])@2;
  \end{verbatimtab}
}
\end{fpfig}

\end{comment}


The full protocol for malicious secure product of secrets $x$ (that
is, $\elab{\secret{x}}{1}$) and $y$ (that is, $\elab{\secret{y}}{2}$)
using Beaver triple $a,b,c$ is defined in Figure \ref{fig-bdoz}. Both
parties interact in secure opening of $x \fplus a$ and $y + b$,
followed by the non-interactive computation of shares of $x * y$
for secure reveals as per standard protocol
\cite{XXX}. The non-interactive reconstruction of the local authentication
key for both the secure openings and the final reveal is enabled by the
semi-homorphic properties of the BDOZ scheme.

\subsection{Cheating Detection and Integrity}

We can carry out similar proofs of passive security for the protocol in
Figure \ref{fig-beaver} as for GMW, even using automated tactics for
the protocol in $\mathbb{F}_2$. But in the case of BDOZ we are also
concerned with malicious security. To demonstrate this, we can
demonstrate that the protocol satisfies integrity in the sense of
Definition \ref{def-integrity}. To do so, we observe that it satisfies
a stronger property, that we call cheating detection. Intuitively,
integrity says that the only thing that the adversary can do in the
malicious model is to elicit the same responses from honest parties
that an honest run of the protocol would elicit. Cheating detection
says that the adversary can only execute the protocol honestly, or
or else gets caught (and abort).

Focusing in, we identify the adversarial inputs as the messages
sent from the adversary to honest parties, on which honest responses
to the adversary may depend. We want to say that these are the messages
that must be legitimate.
\begin{definition}
  Given $\prog$ with $\iov(\prog) = S \cup V \cup O$,
  let $X_H \subseteq \{ x \mid x \in (\houtputs \cup O_H) \wedge x \in \dom(\store) \}$.
  Then the \emph{adversarial inputs to $X_H$} is the least set
  $X_C \subseteq \cinputs$ such that $\progtt(\prog) \not\vdash X_C * X_H$.
\end{definition}
Now, we can characterize protocols with cheating detection as those where
adversarial inputs to honest reponses must themselves be constructed honestly. 
\begin{definition}[Cheating Detection]
  \emph{Cheating is detected} in $\prog$ with $\iov(\prog) = S \cup V \cup O$ iff
  for all  $\store \in \aruns(\prog)$,
  letting $X_H = \{ x \mid x \in (\houtputs \cup O_H) \wedge x \in \dom(\store) \}$,
  and letting $X_C$ be the adversarial inputs to $X_H$,
  there exists $\sigma'\in \runs(\prog)$
  with $\store_{X_C} = \store'_{X_C}$.  
\end{definition}

It is straightforward to demonstrate that cheating detection has integrity,
since only the ``passive'' adversary can elicit a response from honest parties. 
\begin{lemma}
  If cheating is detected in $\prog$, then $\prog$ has integrity.
\end{lemma}

In the case of BDOZ, cheating detection is accomplished by the information-theoretic
security of the encryption scheme\cite{XXX}. Furthermore, the symmetry of
the protocol in Figure \ref{fig-beaver} ensures that both parties will authenticate
shares, so it is robust to corruption of either party. 
