\section{Security Model}
\label{section-pmf}

\emph{Simulator security} in the MPC setting means that protocols do
not reveal any secret information other than what is implicitly
declassified by the output of the ideal functionality that the
protocol implements. Characterization of both real world protocol
execution and simulation is defined probabilistically. Following
previous work \cite{barthe2019probabilistic} we use probability mass
functions to express joint dependencies between input and output
variables, as a metric of information leakage.

\subsection{Probability Mass Functions} 

We define joint probability mass functions (pmfs) in the standard
manner, though following \cite{barthe2019probabilistic} we use
memories to denote mappings of variables to values (i.e., outcomes),
so for example given a pmf $\pmf$ we will write $\pmf(\{ \sx{1}{a}
\mapsto 0, \vx{2}{b} \mapsto 1 \})$ to denote the probability that
$\sx{1}{a} = 0$ and $\vx{2}{b} = 1$ in the final memory of a
program containing these variables.
\begin{definition}
  A \emph{probability mass function} $\pmf$ is a function
  mapping memories in $\mems(X)$ for some $X$ to values in $\mathbb{F}_p$, such that:
  $$
  \sum_{\store \in \mems(X)} \pmf(\store) \  = \ 1
  $$
\end{definition}
%To recover succinct and familiar notation, we may omit the domain of a
%distribution when it is clear from an application context-
%i.e., we allow the following sugaring:
%$$
%\pdf{}(\store) \defeq \pdf{\dom(\store)}(\store)
%$$
Now, we can define a notion of marginal and conditional
distributions as follows, which are standard for discrete
probability mass functions. 
\begin{definition}
  Given $\pmf$ the \emph{marginal distribution} of variables $X$
  in $\pmf$, denoted $\margd{\pmf}{X}$, is defined as follows:
  $$
  \forall \store \in \mems(X) \quad . \quad \margd{\pmf}{X}(\store) =
  \sum_{\store' \in \mems(X-\dom(\dom(\pmf)))} \pmf(\store \cap \store')
  $$
\end{definition}

\begin{definition}
  Given $\margd{\pmf}{X}$, let $\stores$ be a set of memories with the
  same domain $Y \subseteq X$. Then the \emph{conditional distribution given
  $\stores$}  denoted
  $\condd{\pmf}{X}{\stores}$ is a distribution with domain $X$ where for all
  $\store \in \mems(X)$:
  $$
  \condd{\pmf}{X}{\stores}(\store) =
  (\sum_{\store' \in \stores} \margd{\pmf}{X}(\store \cap \store')) /
  (\sum_{\store' \in \stores} \margd{\pmf}{Y}(\store'))
  $$
  where $\margd{\pmf}{X}(\store \cap \store')) = 0$ if $\store \cap \store'$ is undefined.
\end{definition}
To recover familiar notation we allow the syntactic
sugarings $\condd{\pmf}{X}{\store}  \defeq \condd{\pmf}{X}{\{ \store\}}$, and
$\pmf(\store)  \defeq \margd{\pmf}{X}(\store)$ and $\pmf(\store|\stores) \defeq
\condd{\pmf}{X}{\stores}(\store)$ where $\dom(\store) = X$.
%\begin{eqnarray*}
%  \condd{\pmf}{X}{\store}  &\defeq& \condd{\pmf}{X}{\{ \store\}}\\
%  \pmf(\store)  &\defeq& \margd{\pmf}{X}(\store)  \qquad \dom(\store) = X\\
%  \pmf(\store|\stores)  &\defeq& \condd{\pmf}{X}{\stores}(\store) \qquad \dom(\store) = X
%\end{eqnarray*}

We also define the \emph{support} of a distribution in the usual manner-
it is the set of values a set of variables can take on with non-zero
probability.
\begin{definition}[Support]
  $\support(\pmf) \defeq \{ (v_1,\ldots,v_n) \mid
  \pmf(x_1 \mapsto v_1, \ldots, x_n \mapsto v_n) > 0 \} $
\end{definition}

\subsection{Basic Distribution of a Protocol}
Now we can define the probability distribution of a program $\prog$,
that we denote $\progtt(\prog)$. Since $\fedcat$ is deterministic the
results of any run are determined by the input secrets together with
the random tape. And since we constrain programs to not overwrite
views, we are assured that \emph{final} memories contain both a
complete record of all initial secrets as well as views resulting from
communicated information. 

We consider random tapes to be selected from a uniform distribution
modulo relevant pre-processing, e.g., Beaver triples and MACed share
distributions.  We will also assume that any possible initial secrets
are equally likely when formulating the program distribution.
We define $\runs(\prog)$ as the set of final memories resulting
from execution of $\prog$ given any initial memory, and
treat all elements of $\runs(\prog)$ as equally likely.
This establishes the basic program distribution that can be
marginalized and conditioned to quantify input/output
information dependencies. 
%In this
%setting, given a program $\prog$ with $\iov(\prog) = S \cup V$ and
%$\flips(\prog) = F$ we will consider all $\store \in \mems(S \cup V
%\cup F)$ such that $ \config{\store_{S \cup F}}{\prog} \redxs
%\config{\store_}{\varnothing} $ to be equally probable, establishing
%the basic distribution of the program. %From this, we can immediately
%derive the marginal distribution of $S \cup V$ to reason about
%dependencies between secrets and views.
\begin{definition}
  \label{def-progd}
  \label{definition-progd}
  Given program $\prog$ with $\secrets(\prog) = S$ and $\flips(\prog) = R$, and
  any assumed pre-processing predicate $\preproc$ on memories, define:
  $$
  \runs(\prog) \defeq \{ \store \mid \preproc(\store) \wedge
  (\dom(\store) = \iov(\prog) \cup R) \wedge (\store_{S \cup R},\prog) \redxs (\store,\varnothing) \}
  $$
  Then the \emph{basic distribution of $\prog$}, written $\progtt(\prog)$, is
  defined such that for all $\store \in \mems(\iov(\prog) \cup R)$:
  $$
  \progtt(\store) =  1 / |\runs(\prog)| \ \text{if}\ \store \in \runs(\prog), \text{otherwise}\ 0
  $$
  In some cases, we will also be concerned with the (joint)
  probabilities of expression interpretation given a preceding program
  execution, and we write $\progtt(\prog, \be)$ to denote the program
  distribution $\progtt(\prog;\itv := \be)$ where $\itv$ is a
  special variable that is never used in programs.
\end{definition}

\subsection{Passive Security}

Information about honest secrets can be revealed to corrupt clients
through messages sent from honest to corrupt clients, and through
publicly broadcast information from honest clients. Dually,
corrupt clients can impact protocol integrity through the messages
sent from corrupt to honest clients, and through publicly broadcast information
from corrupt clients. We call the former \emph{corrupt views}, and
the latter \emph{honest views}. Generally we let $V$ range over sets
of views.
\begin{definition}[Corrupt and Honest Views]
  Given a program $\prog$ with $\iov(\prog) = S \cup M \cup P \cup O$,
  define $\views(\prog) \defeq M \cup P$, and define $\houtputs$ as the messages in
  $V = \views(\prog)$ sent from honest
  to corrupt parties, called \emph{corrupt views}:
  $$
  \houtputs = \{\ \rvl{w} \mid\ \reveal{w}{\be}{\cid} \in \prog
  \wedge \cid \in H \ \} 
  \cup \{\ \elab{\mesg{w}}{\cid}\ \mid\  \eassign{\mesg{w}}{\cid}{\be}{\cid'} \in \prog
  \wedge \cid \in C \wedge \cid' \in H \ \} 
  $$
  and similarly define $\cinputs$ as the messages in $V$ sent from corrupt to honest
  parties, called \emph{honest views}:
  $$
  \cinputs = \{\ \rvl{w} \mid\ \reveal{w}{\be}{\cid} \in \prog
  \wedge \cid \in C \ \} 
  \cup \{\ \elab{\mesg{w}}{\cid}\ \mid\  \eassign{\mesg{w}}{\cid}{\be}{\cid'} \in \prog
  \wedge \cid \in H \wedge \cid' \in C \ \} 
  $$
\end{definition}

In the passive setting the simulator must construct a probabilistic
algorithm $\SIM$, aka a \emph{simulation}, that is parameterized by
corrupt inputs and the output of an ideal functionality, and that
returns a distribution of corrupt views that is indistinguishable
from the distribution of corrupt views in the real world
protocol execution.

Given inputs $\store$ and ideal
functionality output $v$, we write $ \prob(\SIM(\store,v) = \store') $
to denote the probability that $\SIM(\store,v)$ returns corrupt views
$\store'$ as a result. We can then define the probability distribution
of corrupt views reconstructed by the simulator as follows:
\begin{definition}
  Given $\store$, and $v$, we write $\dist(\SIM(\store,v))$ to
  denote the distribution of corrupt views reconstructed by the
  simulation, where for
  all $\store' \in \mems(V)$:
  $$
  \dist(\SIM(\store,v))(\store')\ \defeq\ \prob(\SIM(\store,v) = \store') 
  $$
\end{definition}
Then we can define passive security in the real/ideal
model as follows. 
\begin{definition}[Passive Security]
  Assume given a program $\prog$ that correctly implements an ideal
  functionality $\idealf$, with $\views(\prog) = V$.  Then $\prog$
  is \emph{passive secure in the simulator model} iff for all
  partitions of the federation into honest and corrupt sets $H$ and $C$
  and for all $\store \in \mems(S)$ there exists a
  simulation $\SIM$ such that:
  $$
  \dist(\SIM(\store_{C},\idealf(\store))) = \condd{\progtt(\prog)}{\houtputs}{\store}
  $$
\end{definition}
Observe that this definition relies on protocol correctness, which
we can define as follows.
\begin{definition}[Passive Correctness]
  Given $\prog$ with output variables $\out{1},\ldots,\out{n}$ and ideal
  functionality $\idealf$, we say that $\prog$ is \emph{passive correct}
  for $\idealf$ iff for all $\store \in \mems(\secrets(\prog))$
  with $\idealf(\store) = v_1,\ldots,v_n$ we have
  $\progtt(\prog)(\out{1} \mapsto v_1,\ldots,\out{n} \mapsto v_n \mid \store) = 1$.
\end{definition}

\subsection{Malicious Security}

In the malicious model we assume that corrupt clients are in the thrall of an
adversary $\adversary$ who does not necessarily follow the rules of the protocol.
We model this by positing a $\arewrite$ function which is given a corrupt memory
$\store_C$ and expression $\be$, and returns a rewritten expression that can
be interpreted to yield a corrupt input. Then we define an evaluation
relation that incorporates the adversary:
$$
\begin{array}{rclr}
  (\store, \xassign{x}{\be}{\cid};\prog) &\aredx&
  (\extend{\store}{x}{\lcod{\store,\be}{\cid}}, \prog) & \cid \in H\\
  (\store, \xassign{x}{\be}{\cid};\prog) &\aredx&
  (\extend{\store}{x}{\lcod{\arewrite(\store_C,\be)}{\cid}}, \prog) & \cid \in C
\end{array}
$$

$$
\begin{array}{rcl@{\qquad}r}
  (\store,\elab{\assert{\phi(\be)}}{\cid};\prog) &\redx& (\store,\prog) & \text{if\ } \phi(\lcod{\store,\be}{\cid}) \text{\ or\ } \cid \in C\\
  (\store,\elab{\assert{\phi(\be)}}{\cid};\prog) &\redx& (\store,\bot)  & \text{if\ } \neg\phi(\store,\lcod{\store,\be}{\cid})
\end{array}
$$

Given this we define the $\adversary$ distribution of $\prog$. 
\begin{definition}
  \label{def-progd}
  \label{definition-progd}
  Given program $\prog$ with $\iov(\prog) = S \cup V \cup O$ and $\flips(\prog) = R$, and
  any assumed pre-processing predicate $\preproc$ on memories, define:
  $$
  \aruns(\prog) \defeq \{ \store \mid \preproc(\store)  \wedge
  ((\store_{S\cup R},\prog) \aredx^* (\store,\varnothing) \vee (\store_{S \cup R},\prog) \aredx^* (\store,\bot)) \}
  $$
  and also define:
  $$
  \botruns(\prog) \defeq \{ \store\{ x_1 \mapsto \bot, \ldots, x_n \mapsto \bot \} \mid
  \store \in \aruns(\prog) \wedge \{ x_1,\ldots,x_n\} = (V \cup O) - \dom(\store) \}
  $$
  Then the \emph{$\adversary$ distribution of $\prog$}, written $\progtt(\prog,\adversary)$, is
  defined such that for all $\store \in \mems(\iov(\prog) \cup R)$:
  $$
  \progtt(\prog,\adversary) =  1 / |\botruns(\prog)| \ \text{if}\ \store \in \botruns(\prog), \text{otherwise}\ 0
  $$
\end{definition}

\begin{definition}[Malicious Simulation]
  Given a protocol $\prog$ with $\iov(\prog) = S \cup V \cup O$, honest and corrupt 
  clients $H$ and $C$, adversary $\adversary$, and honest inputs
  $\store \in \mems(S_H)$, the \emph{malicious simulation}  $\SIM(\store)$ has three phases:
  \begin{enumerate}
  \item In the first phase $\SIM_1$, $\adversary$ gives the
    simulator some $\store' \in \mems(S_C)$, and the simulator consults an
    oracle to compute $\idealf(\store \cup \store') \in \mems(O)$.
  \item In the second phase $\SIM_2$, the simulator is given the corrupt
    outputs $\idealf(\store \cup \store')_C$, which are again given to
    $\adversary$, who decides either to abort or not. If so, then the
    simulator is given $\sigma_{\mathit{out}} \defeq \{ \out{\cid} \mapsto \bot \mid \cid \in H \}$
    and arbitrary internal state $\Sigma$.
    Otherwise the simulator is given $\sigma_{\mathit{out}} \defeq \idealf(\store \cup \store')_H$
    and $\Sigma$.
  \item In the third phase $\SIM_3$, given $\store_{\mathit{out}}$ and $\Sigma$, the simulator
    finally outputs
    $\store_{\mathit{out}} \cup \store_{\mathit{views}}$ for some
    calculated $\store_{\mathit{views}} \in \mems(\houtputs)$.
  \end{enumerate}
\end{definition}

\begin{definition}[Malicious Security]
  We write $\dist(\SIM(\store))$ to
  denote the distribution of honest outputs and corrupt views reconstructed by the
  malicious simulation, where for
  all $\store'$:
  $$
  \dist(\SIM(\store))(\store')\ \defeq\ \prob(\SIM(\store) = \store') 
  $$
  Then a protocol $\prog$ with $\iov(\prog) = S \cup V \cup O$ is \emph{malicious
  secure} iff for all $H$, $C$, $\adversary$, and $\store \in \mems(S_H)$:
  $$
  \dist(\SIM(\store)) = \condd{\progtt(\prog,\adversary)}{\houtputs \cup O_H}{\store}
  $$  
\end{definition}
