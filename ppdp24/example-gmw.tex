\section{2-Party GMW and Passive Security Proof Tactics}
\label{section-metalang-gmw}
\label{section-example-gmw}

\begin{fpfig}[t]{2-Party GMW circuit library with And gate (T), and type annotations (B).}{fig-gmw}
{\footnotesize
  \begin{verbatimtab}
    encodegmw(in1,in2) {
      v[1, in1++"out"] := s[1,in1] xor flip[1,in1];
      v[2, in1++"out"] := flip[1,in1];
      v[1, in2++"out"] := s[2,in2] xor flip[2,in2];
      v[2, in2++"out"] := flip[2,in2]
      { shares1 = { c1 = v[1, in1++"out"]; c2 = v[2, in1++"out"] };
        shares2 = { c1 = v[1, in2++"out"]; c2 = v[2, in2++"out"]} } 
    }
    
    andtablegmw(b1, b2) {
      let r11 = (b1 xor true) and (b2 xor true) in
      let r10 = (b1 xor true) and (b2 xor false) in
      let r01 = (b1 xor false) and (b2 xor true) in
      let r00 = (bl xor false) and (b2 xor false) in
      { v1 = r11; v2 = r10; v3 = r01; v4 = r00 }
    }
    
    andgmw(g, shares1, shares2) {
      let r = flip[1,g++".r"] in
      let table = andtablegmw(shares1.c1, shares2.c1) in
      let r11 =  r xor table.v1 in
      let r10 =  r xor table.v2 in
      let r01 =  r xor table.v3 in
      let r00 =  r xor table.v4 in
        v[2,g++"out"] := OT4(shares1.c2, shares2.c2, r11, r10, r01, r00);
      v[1,g++"out"] := r;
      { c1 = v[1,g++"out"]; c2 = v[2,g++"out"]}
    }
    
    decodegmw(shares) { v[0,"output"] := shares.c1 xor shares.c2 }   \end{verbatimtab}
}
\end{fpfig}


As an extended example of our language and security model, and how the
automated techniques in Section \ref{section-bruteforce} can serve
as tactics integrated with PSL/Lilac-style proofs, we consider GMW
circuits.  The GMW protocol is a garbled binary circuit protocol.  We
will assume the 2-party version, though it generalizes to $n$
parties\cite{XXX}. GMW uses a common technique in MPC, which is to
represent values $v$ as distributed shares $v_1$ and $v_2$ with $v =
v_1 \fplus v_2$. This trick maintains secrecy of $v$ from both
parties, and in GMW it is used to maintain the intermediate values of
internal gate outputs in circuits. In related literature the notation
$\macgv{x}$ is used to represent the ``true'' value of $x$ and $[x]$
is often used to represent the share of given party.

To capture this convention, which is used in many other protocols, we
introduce a new naming convention for ``global view'' elements
$\macgv{\mesg{w}}$, which denote the summed value of
$\elab{\mesg{w}}{1}$ and $\elab{\mesg{w}}{2}$ in a protocol
run. This concept integrates program distributions in the
usual manner, as the probability of the outcome of summation
of two variables in the distribution.
\begin{definition}
  For all $\mesg{w}$ define:
  $$\pmf(\macgv{\mesg{w}} = v) \defeq \sum_{\sigma \in A} \pmf(\sigma)$$
  and define:
  $$\condd{\pmf}{X}{\macgv{\mesg{w}} = v}(\sigma) \defeq  \sum_{\sigma' \in A} \condd{\pmf}{X}{\sigma'}(\sigma)$$
  where $A$ is:
  $$\{ \store \in \mems(\{ \elab{\mesg{w}}{1},\elab{\mesg{w}}{2} \} ) \mid
      \fcod{\store(\elab{\mesg{w}}{1}) + \store(\elab{\mesg{w}}{2})} = v \}$$
\end{definition}

For full details of the GMW protocol the reader is referred to
\cite{evans2018pragmatic}. Our implementation libary is shown in
Figure \ref{fig-gmw}, and includes encoding functions, where
input secrets are split into shares, $\eand$ and $\exor$ gate
functions, and a decoding function. Note that $\exor$ requires
no interaction between parties, while $\eand$ necessitates and
1-of-4 oblivious transfer. The gate computation is
done entirely in secrect, and the decoding function
is where the declassification occurs-- both parties reveal
their shares of the final gate output $\macgv{z}$.

For example, the following program uses our GMW library to define
a circuit with a single \eand gate and input secrets $\ttt{s1}$ and
$\ttt{s2}$ from client's 1 and 2 respectively:
\begin{verbatimtab}
         let s1 = encodegmw("s1") in
         let s2 = encodegmw("s2") in
         decodegmw(andgmw("z",s1,s2))
\end{verbatimtab}
By convention we will assume that all gates are assigned unique output
identifiers $\ttt{"z"}$, and that all programs are in the form
of a sequence of let-bindings followed by a call to $\decodegmw$
wrapping a circuit.

\paragraph{Oblivious Transfer} A passive secure OT protocol
based on previous work \cite{XXX} can be defined in $\metaprot$,
however this protocol assumes some shared randomness. Alternatively,
a simple passive secure OT can be defined with the addition of
public key cryptography as a primitive. But given the diversity
of approaches to OT, we instead assume that OT is abstract with
respect to its implementation, where calls to OT in $\mathbb{F}_2$
are of the following form-- given a \emph{choice bit}
$\be_1$ provided by a receiver $\cid$, the sender
sends either $\be_2$ or $\be_3$.
$$
\OT{\elab{\be_1}{\cid}}{\be_2}{\be_3}
$$
Critically, the sender learns nothing about $\be_1$ and the
receiver learns nothing about the unselected value, so we interpret
these calls in our implementation in $\mathbb{F}_2$ as follows.
$$
\begin{array}{l}
\solve{\stores}{\OT{\elab{\be_1}{\cid_1}}{\be_2}{\be_3}}{\cid_2} = \\
\qquad ((\solve{\stores}{\be_1}{\cid_1}) \cap 
(\solve{\stores}{\be_2}{\cid_2})) \cup \\
\qquad ((\stores - (\solve{\stores}{\be_1}{\cid_1})) \cap
(\solve{\stores}{\be_3}{\cid_2})
\end{array}
$$

\subsection{Correctness Proof with Verification Tactics}

As discussed above and in related work \cite{XXX}, probabilistic
separation conditional on certain variables-- e.g., secret inputs or
public outputs-- is a key mechanism for reasoning about MPC protocol
security. Following \cite{XXX}, we define a conditional separation
relation $\condsep{\pmf}{X_1}{X_2}{X_3}$ to mean that
under the condition of all value assignments for
$X_2$ and $X_3$ are separable under $\pmf$ on the condition
of any value assignment of $X_1$-- i.e., conditionally
on any $\store \in \mems(X_1)$. 
\begin{definition}[Conditional Separation]
  We write $\condsep{\pmf}{X_1}{X_2}{X_3}$ iff for all
  $\store' \in \mems(X_1)$ and $\store \in \mems(X_2 \cup X_3)$
  and letting $X = X_1 \cup X_2 \cup X_3$ we have:
  $$\condd{\pmf}{X}{\store'}(\store) =
  \condd{\pmf}{X}{\store'}(\store_{X_2}) *
  \condd{\pmf}{X}{\store'}(\store_{X_3})$$
\end{definition}
Another key concept needed especially for reasoning about
circuits is conditional determinism. For example, if
$\macgv{z}$ is an output of an internal gate, it will
definitely be computed using random variables, however,
it \emph{should} be determistic under any set of input
secrets $S$, since we assume that $\idealf$ is
deterministic.
\begin{definition}[Conditional Determinism]  
  We write $\conddetx{\pmf}{X_1}{X_2}$ iff for all
  $\store' \in \mems(X_1)$ there exists 
  $\store \in \mems(X_2)$ such that
  $\condd{\pmf}{X_1 \cup X_2}{\store'}(\store) = 1$.
\end{definition}

Given these definitions, we can formulate an invariant
for circuit computation with respect to internal gates
as follows. It says that the output of any gate is
deterministic wrt inputs $S$, and conditionally
on $S$ the output $\macgv{\mesg{z}}$ remains
separable from corrupt views and both shares of
$\macgv{\mesg{z}}$. This last nuance is critical,
since those shares will in fact be revealed if
$\macgv{\mesg{z}}$ is decoded as the circuit output. 
\begin{lemma}[GMW Invariant]
  \label{lemma-gmwinvariant}
  Given:
  $$ (\varnothing,e) \redxs (\prog,\decodegmw(E[\mesg{z}])) $$
  Then both of the following conditions hold for all $H$ and $C$ where $\iov(\prog) = S \cup M$:
  \begin{enumerate}
    \item $\conddetx{\progtt(\prog)}{S}{\{\macgv{\mesg{z}}\}}$
    \item $\condsep{\progtt(\prog)}{S}{\{\macgv{\mesg{z}}\}}{(M_C \cup \{ \elab{\mesg{z}}{1}, \elab{\mesg{z}}{2} \})}$
  \end{enumerate}
\end{lemma}
To prove this, we can formulate and automatically prove gate-level
versions of the invariant. This serves as a proof tactic
that simplifies the proof of the GMW invariant. It establishes
that the $\eand$ gate output is deterministic conditional on
the inputs, and is separable from the output shares and
either parties' received messages. 
\begin{lemma}[And Gate Tactic]
  \label{lemma-gmwtactic}
  %Define:
  %$$
  %\begin{array}{c}
  %  \prog_{i} \defeq \\
  %  \eassign{\mesg{x}}{1}{\flip{x}}{1}; \eassign{\mesg{x}}{2}{\flip{x}}{2}; \\
  %  \eassign{\mesg{y}}{1}{\flip{y}}{1}; \eassign{\mesg{y}}{2}{\flip{y}}{2} 
  %\end{array}
  %$$
  Given:
  $$
  \begin{array}{c}
  (\varnothing,\andgmw(z,\mesg{x},\mesg{y}) \redxs %\\
  (\prog,\mesg{z})
  \end{array}
  $$
  Then both of the following conditions hold for both $\cid \in \{ 1,2 \}$ where $\iov(\prog) = M$:
  \begin{enumerate}
  \item
    $\conddetx{\progtt(\prog)}{\{ \macgv{\mesg{x}},\macgv{\mesg{y}} \}}{\{ \macgv{\mesg{z}} \}}$
   \item $\condsep
      {\progtt(\prog)}
      {\{ \macgv{\mesg{x}},\macgv{\mesg{y}} \}}
      {\{ \macgv{\mesg{z}} \}}
      {M_{\{\cid\}} \cup \{ \elab{\mesg{z}}{1},\elab{\mesg{z}}{2} \}}$
  \end{enumerate}
\end{lemma}
\begin{proof}
Verified automatically using techniques described in Section \ref{section-bruteforce}.  
\end{proof}

To properly integrate the local reasoning of Lemma \ref{lemma-gmwtactic} with
the global reasoning of Lemma \ref{lemma-gmwinvariant}, we can demonstrate
the following.
\begin{lemma}
  \label{lemma-conditioning}
  Each of the following hold:
  \begin{enumerate}
    \item Given $\condp{\pmf}{X_1}{\detx{X_2}}$ and
      $\condp{\pmf}{X_2}{\detx{X_3}}$, then $\condp{\pmf}{X_1}{\detx{X_3}}$.
    \item Given $\condp{\pmf}{X_1}{\detx{X_2}}$ and
      $\condsep{\pmf}{X_2}{X_3}{X_4}$, then $\condsep{\pmf}{X_1}{X_3}{X_4}$.
    \item Given $\condsep{\pmf}{X_1}{X_2}{X_3}$ and $\condp{\pmf}{X_1}{\detx{X_2}}$
      and $\condp{\pmf}{X_2}{\detx{X_4}}$, then $\condsep{\pmf}{X_1}{X_4}{X_3}$.
    \item Given $\condsep{\pmf}{X_1 \cup X_2}{X_3}{X_4}$ and $\condp{\pmf}{X_1 \cup X_2}{\detx{X_2}}$,
      then $\condsep{\pmf}{X_1}{X_2 \cup X_3}{X_4}$.
  \end{enumerate}
\end{lemma}

Then we can put the pieces together to prove the invariant, using automated tactics
for gate-level reasoning.  
\begin{proof}[Proof of Lemma \ref{lemma-gmwinvariant}]
  By induction on the length of $(\varnothing,e) \redxs (\prog,\decodegmw(E[\mesg{z}]))$.
  Encoding establishes the invariant in the basis. The most interesting inductive
  case is the $\andgmw$ gate. 
  \paragraph{Case $\andgmw$.} In this case we have:
  $$
  \begin{array}{c}
  (\varnothing,e) \redxs (\prog',\decodegmw(E[\andgmw(z,\mesg{x},\mesg{y})])) \redxs \\
    (\prog,\decodegmw(E[\mesg{z}]))
  \end{array}
  $$
  The induction hypothesis, together with the assumed uniquenenss of $z$, gives:
  \begin{mathpar}
  \condsep{\progtt(\prog)}{S^1}{\{ \macgv{\mesg{x}} \}}{(M^1_C \cup \{ \elab{\mesg{x}}{1}, \elab{\mesg{x}}{2} \})}
  
  \condsep{\progtt(\prog)}{S^2}{\{ \macgv{\mesg{y}} \}}{(M^2_C \cup \{ \elab{\mesg{y}}{1}, \elab{\mesg{y}}{2} \})}

  \conddetx{\progtt(\prog)}{S^1}{\{ \macgv{\mesg{x}} \}}
  
  \conddetx{\progtt(\prog)}{S^1}{\{ \macgv{\mesg{x}} \}}    
  \end{mathpar}
  This, together with Lemma \ref{lemma-gmwtactic} (1) and Lemma \ref{lemma-conditioning} (1)
  give:
  $$
  \conddetx{\progtt(\prog)}{S^1 \cup S^2}{\{ \macgv{\mesg{z}} \}}
  $$
  and Lemma  \ref{lemma-gmwtactic} (2) and Lemma \ref{lemma-conditioning} (2-3) gives:
  $$
  \begin{array}{c}
    \progtt(\prog)|S^1 \cup S^2) \vdash \\
    {\{ \macgv{\mesg{z}} \}}{(M^1_C \cup M^2_C \cup \{ \elab{\mesg{x}}{2}, \elab{\mesg{y}}{2}, \elab{\mesg{z}}{1}, \elab{\mesg{z}}{2} \})}
  \end{array}
  $$
\end{proof}

\begin{theorem}
  If $e$ is a GMW circuit protocol in $\metaprot$ with $(\varnothing,e) \redxs (\prog,())$
  then $\prog$ satisfies noninterference modulo output. 
\end{theorem}

\begin{proof}
  Given that $e$ is a GMW circuit protocol, then by definition we have:
  $$
  (\varnothing,e) \redxs (\prog',\decodegmw(\mesg{z})) \redxs (\prog,())
  $$
  where by Lemma \ref{lemma-gmwinvariant} and definition of $\decodegmw$,
  for any $H$ and $C$ letting $\iov(\prog) = S \cup M \cup P \cup O$ we
  have:
  \begin{mathpar}
    \conddetx{\progtt(\prog)}{S}{\{\macgv{\mesg{z}}\}}

    \condsep{\progtt(\prog)}{S}{\{\macgv{\mesg{z}}\}}{(M_C \cup \{ \elab{\mesg{z}}{1}, \elab{\mesg{z}}{2} \})}
  \end{mathpar}
  so also by definition of $\decodegmw$ we have:
  \begin{mathpar}
    \conddetx{\progtt(\prog)}{S_H \cup S_C}{O}
    
    \condsep{\progtt(\prog)}{S_H \cup S_C}{O}{(M_C \cup P)}
  \end{mathpar}
  and this by Lemma \ref{lemma-conditioning} (4) we have:
  $$
  \condsep{\progtt(\prog)}{S_C}{(O \cup S_H)}{(M_C \cup P)}
  $$
  which implies the result.
\end{proof}
