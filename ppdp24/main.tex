\documentclass[sigconf]{acmart}

\usepackage{amsmath}
\usepackage{amstext}
\usepackage{fp-frame}
\usepackage[latin1]{inputenc}
\usepackage{mathpartir}
\usepackage{fancyvrb}
\usepackage{moreverb}
\usepackage{stmaryrd}
\usepackage{enumerate}
\usepackage{thmtools,thm-restate}
\usepackage{comment}
\usepackage{booktabs,array}
\usepackage{rotating}

\newcommand{\note}[1]{\noindent\textit{(\textbf{$\star$note$\star$:}\ \ #1)}}

\newcommand{\evals}{\Downarrow}
\newcommand{\diverges}{\Uparrow}
\newcommand{\intt}{\mathrm{int}}
\newcommand{\unitt}{\mathrm{unit}}
\newcommand{\boolt}{\mathrm{bool}}
\newcommand{\floatt}{\mathrm{float}}
\newcommand{\stringt}{\mathrm{string}}
\newcommand{\chart}{\mathrm{char}}
\newcommand{\vb}[1]{\verb+#1+}
\newcommand{\evalexmp}[2]{\texttt{#1}\ \ensuremath{\evals}\ \texttt{#2}}
\newcommand{\texmp}[2]{\texttt{#1\ :\ #2}}
\newcommand{\skipper}{\bigskip\\}
\newcommand{\fyi}{\noindent\textbf{\textit{fyi:}}\ }
\newcommand{\NB}{\noindent\textbf{NB:\ }}
\newcommand{\const}{\ensuremath{\mathbf{c}}}
\newcommand{\defn}{\heading{definition}}
\newcommand{\defeq}{\triangleq}
\newcommand{\nat}{\mathbb{N}}
\newcommand{\atom}{\texttt{const}}
\def\squareforqed{\hbox{\rlap{$\sqcap$}$\sqcup$}}
\def\qed{\ifmmode\squareforqed\else{\unskip\nobreak\hfil
\penalty50\hskip1em\null\nobreak\hfil\squareforqed
\parfillskip=0pt\finalhyphendemerits=0\endgraf}\fi}
\newcommand{\exampletab}[1]{\skipper\begin{tabular}{lll}#1\end{tabular}\skipper}
\newcommand{\verbtab}[1]{\skipper\begin{verbatimtab}{#1}\end{verbatimtab}\skipper}
\newcommand{\eqntab}[1]{\skipper\begin{tabular}{rcl}#1\end{tabular}\skipper}
\newcommand{\recdefn}[1]{\{#1\}}
\newcommand{\ttt}[1]{\texttt{#1}}
\newcommand{\gdesc}[1]{\text{\textit{#1}}}
\newcommand{\true}{\mathrm{true}}
\newcommand{\false}{\mathrm{false}}
\newcommand{\etrue}{\texttt{true}}
\newcommand{\efalse}{\texttt{false}}
\newcommand{\reval}{\Rightarrow}
\newcommand{\Dand}{\ \mathrm{and}\ }
\newcommand{\Dor}{\ \mathrm{or}\ }
\newcommand{\Dxor}{\ \mathrm{xor}\ }
\newcommand{\Dnot}{\mathrm{not}\ }
\newcommand{\cod}[1]{\llbracket #1 \rrbracket}
\newcommand{\lcod}[2]{\llbracket #1 \rrbracket_{#2}}
\newcommand{\Dplus}{\mathrm{Plus}}
\newcommand{\Dminus}{\mathrm{Minus}}
\newcommand{\Dequal}{\mathrm{=}}
\newcommand{\Dabs}[2]{(\mathrm{Function}\ #1 \rightarrow #2)}
\newcommand{\Dfix}[3]{(\mathrm{Fix}\ #1 . #2 \rightarrow #3)}
\newcommand{\Dite}[3]{\mathrm{If}\ #1\ \mathrm{Then}\ #2\ \mathrm{Else}\ #3}
\newcommand{\dotminus}{\stackrel{.}{-}}
\newcommand{\Dlet}[3]{\mathrm{Let}\ #1 = #2\ \mathrm{In}\ #3}
\newcommand{\Dletrec}[4]{\mathrm{Let\ Rec}\ #1\ #2 = #3\ \mathrm{In}\ #4}
\newcommand{\Dfst}{\mathrm{left}}
\newcommand{\Dsnd}{\mathrm{right}}
\newcommand{\labset}{\mathit{Lab}}
\newcommand{\Drec}[1]{\{ #1 \}}
\newcommand{\linfer}[3]{\inferrule*[right=(\TirName{#1})]{#2}{#3}}
\newcommand{\lab}[1]{\mathrm{#1}}
\newcommand{\loc}{\ell}
\newcommand{\Dref}[1]{\mathrm{Ref}\,#1}
%\newcommand{\store}{\mathcal{M}}
\newcommand{\store}{m}
%\newcommand{\stores}{\overline{\store}}
\newcommand{\stores}{M}
\newcommand{\config}[2]{\langle #1,#2 \rangle}
\newcommand{\configf}[2]{\begin{array}[t]{l}\langle #1\\,\\ #2 \rangle \end{array}}
\newcommand{\extend}[3]{#1\{#2 \mapsto #3\}}
\newcommand{\emptystore}{\{\}}
\newcommand{\storedefn}[1]{\{#1\}}
\newcommand{\Dret}[1]{\mathrm{Return}\,#1}
\newcommand{\Draise}[1]{\mathrm{Raise}\,#1}
\newcommand{\Dexn}[2]{\#\!#1\,#2}
\newcommand{\Dtry}[3]{\mathrm{Try}\,#1\,\mathrm{With}\,#2 \rightarrow #3}
\newcommand{\xname}{\mathit{exn}}
\newcommand{\Dboolt}{\mathrm{Bool}}
\newcommand{\Dreft}[1]{#1\,\mathrm{ref}}
\newcommand{\reft}[1]{#1\,\mathrm{ref}}
\newcommand{\Dintt}{\mathrm{Int}}
%\newcommand{\tjudge}[3]{#1 \vdash #2 : #3}
\newcommand{\textend}[3]{#1;#2:#3}
\newcommand{\fnty}[2]{#1 \rightarrow #2}
\newcommand{\TDabs}[3]{(\mathrm{Function}\ (#1 : #2) \rightarrow #3)}
\newcommand{\TDfix}[5]{(\mathrm{Fix}\ #1 . (#2 : #3) : #4 \rightarrow #5)}
\newcommand{\TDletrec}[6]{\Dletrec{#1}{#2 : #3}{#4 : #5}{#6}}
\newcommand{\emptyenv}{\varnothing}
\newcommand{\tfail}{\mathbf{fail}}
\newcommand{\tcheck}{\mathrm{TC}}
\newcommand{\tcheckfail}{\mathbf{TypeMismatch}}
\newcommand{\algtab}[1]
{
\vspace*{-3mm}
\begin{tabbing}
\hspace*{12mm}\=\hspace{9mm}\=\hspace{9mm}\=\hspace{6mm}\=\hspace{6mm}\=
\hspace{6mm}\=
#1
\end{tabbing}
}
\newcommand{\eassign}[2]{#1 := #2}
\newcommand{\ederef}[1]{\,!#1}
\newcommand{\declass}[2]{\mathrm{declassify}_{#2}(#1)}
\newcommand{\eendorse}[2]{\mathrm{endorse}_{#2}(#1)}
\newcommand{\lt}{\left\{}
\newcommand{\rt}{\right\}}
\newcommand{\Lt}{\left\{\!\!\right.}
\newcommand{\Rt}{\left.\!\!\right\}}
\newcommand{\tinfer}{\mathit{PT}}
\newcommand{\unify}{\mathit{unify}}
\newcommand{\tsubn}{\varphi}
\newcommand{\scheme}[2]{\forall #1 . #2}
\newcommand{\Dself}{\mathrm{this}}
\newcommand{\Dsuper}{\mathrm{super}}
\newcommand{\Dsend}[3]{#1.#2(#3)}
\newcommand{\Dselect}[2]{#1.#2}
\newcommand{\Demptyclass}{\mathrm{EmptyClass}}
%\newcommand{\Dclass}[3]{\mathrm{Class}\ \mathrm{Extends}\ #1\ \mathrm{Inst}
%\ #2\ \mathrm{Meth}\ #3}
\newcommand{\Dclass}[2]{\mathrm{Class}\ \mathrm{Inst} \ #1\ \mathrm{Meth}\ #2}
\newcommand{\Dobj}[2]{\mathrm{Object}\ \mathrm{Inst}\ #1\ \mathrm{Meth}\ #2}
%\newcommand{\Dclassf}[3]{
%\begin{array}[t]{l}
%\mathrm{Class}\ \mathrm{Extends}\ #1 \\
%\quad \mathrm{Inst}\\
%\qquad #2 \\ 
%\quad \mathrm{Meth}\\
%\qquad #3
%\end{array}
%}
\newcommand{\Dclassf}[3]{
\begin{array}[t]{l}
\mathrm{Class}\\
\quad \mathrm{Inst}\\
\qquad #1 \\ 
\quad \mathrm{Meth}\\
\qquad #2
\end{array}
}
\newcommand{\Dobjf}[2]{
\begin{array}[t]{l}
\mathrm{Object}\\
\quad \mathrm{Inst}\\
\qquad #1 \\ 
\quad \mathrm{Meth}\\
\qquad #2
\end{array}
}
\newcommand{\Dnew}[1]{\mathrm{New}\ #1}
\newcommand{\vtab}[1]{\begin{verbatimtab}[4]#1\end{verbatimtab}}

\newcounter{topiccounter}
\setcounter{topiccounter}{1}
\newcommand{\topic}[1]
    {\noindent \textbf{Topic \arabic{topiccounter}.\ \textit{#1}. } \stepcounter{topiccounter}}


\newcommand{\lcalc}{$\lambda$-calculus}
\newcommand{\redx}{\rightarrow}
\newcommand{\redxs}{\redx^*}
\newcommand{\idfn}{\mathit{ID}}
\newcommand{\mlfn}[2]{\mathrm{fun}\, #1 \rightarrow #2}
\newcommand{\mlrecfn}[3]{\mathrm{fix}\,#1.#2 \rightarrow #3}
\newcommand{\mlfix}{\mathrm{fix}}
\newcommand{\eite}[3]{\mathrm{if}\ #1\ \mathrm{then} \ #2\ \mathrm{else} \ #3\ }
\newcommand{\esucc}[1]{\texttt{succ}\ #1}
\newcommand{\epred}[1]{\texttt{pred}\ #1}
\newcommand{\eiszero}[1]{\texttt{iszero}\ #1}
\newcommand{\ezero}{\texttt{0}}
\newcommand{\elet}[3]{\mathrm{let}\ #1 = #2\ \mathrm{in}\ #3}
\newcommand{\eletrec}[3]{\mathrm{letrec}\ #1 = #2\ \mathrm{in}\ #3}
\newcommand{\fv}{\mathrm{fv}}
\newcommand{\ourml}{\mathit{ML}_{\mathit{Cat}}}
\newcommand{\raisexn}{\mathrm{raise}}
\newcommand{\handler}[3]{\mathrm{try}\, #1\, \mathrm{with}\, \exn(#2) \Rightarrow #3}
\newcommand{\exn}{\mathit{exn}}
\newcommand{\dom}{\mathrm{dom}}
\newcommand{\efst}{\mathrm{fst}}
\newcommand{\esnd}{\mathrm{snd}}
\newcommand{\natt}{\textrm{Nat}}
\newcommand{\earray}{\mathrm{array}}
\newcommand{\varray}{\alpha}
\newcommand{\length}{\mathit{length}}
\newcommand{\arrayml}{\ourml^{\earray}}
\newcommand{\stackml}{\ourml^{\mathit{stack}}}
\newcommand{\flowml}{\ourml^{\mathit{flow}}}
\newcommand{\taintml}{\ourml^{\mathit{taint}}}
\newcommand{\secfail}{\mathbf{secfail}}
\newcommand{\tr}{\theta}
\newcommand{\rewrite}[1]{\mathcal{R}(#1)}
%\newcommand{\secprop}{\mathcal{P}}
%\newcommand{\trprop}{\hat{\secprop}}
\newcommand{\Prop}{\mathbf{P}}
\newcommand{\Hprop}{\mathbf{H}}
\newcommand{\secprop}{\phi}
\newcommand{\hyprop}{\eta}
\newcommand{\trprop}{\gamma}
\newcommand{\tracess}{\Sigma}
\newcommand{\trsprop}{\sigma}
\newcommand{\traces}{\Psi}
\newcommand{\fpkeyword}[1]{\mathrm{#1}}
\newcommand{\ebinop}[2]{#1\,\mathit{binop}\,#2}
\newcommand{\eenablepriv}[2]{\fpkeyword{enable}\ #1\ \fpkeyword{for}\ #2}
\newcommand{\echeckpriv}[2]{\fpkeyword{check}\ #1\ \fpkeyword{then}\ #2}
\newcommand{\esigned}[2]{#1.#2}
\newcommand{\enabled}{\mathit{enabledprivs}}
%\newcommand{\acl}{\mathcal{A}}
\newcommand{\priv}{\pi}
\newcommand{\privs}{\mathit{R}}
\newcommand{\prin}{p}
\newcommand{\nobody}{\mathit{nobody}}
\newcommand{\po}{\preceq}
\newcommand{\seclattice}{\mathcal{S}}
\newcommand{\binsl}{\mathcal{S}_{\mathrm{bin}}}
\newcommand{\seclevs}{\mathcal{L}}
\newcommand{\latel}{\varsigma}
\newcommand{\hilab}{H}
\newcommand{\lolab}{L}
\newcommand{\hiloc}{\mathit{hi}}
\newcommand{\loloc}{\mathit{low}}
\newcommand{\labty}[2]{#1 \cdot #2}
\newcommand{\labval}[2]{#1 \cdot #2}
\newcommand{\mi}[1]{\mathit{#1}}
\newcommand{\pc}{\latel_{\mathit{pc}}}
\newcommand{\cfnty}[3]{#1 \rightarrow_{#2} #3}
\newcommand{\pow}{\mathrm{pow}}

%\newcommand{\tr}{\theta}


\newcommand{\problemheading}[1]{\noindent\textbf{#1}\ }
\newcounter{problemcounter}
\setcounter{problemcounter}{1}

\newcommand{\EC}[1]
    {\problemheading{Extra Credit \textit{(#1 points)}.}}

\newcommand{\pproblem}[1]
    {\problemheading{Problem \arabic{problemcounter} \textit{(#1 points)}.} \stepcounter{problemcounter}}
           
\newcommand{\gproblem}[1]
    {\problemheading{Problem \arabic{problemcounter} \textit{(Graduate Students Only; #1 points)}.} \stepcounter{problemcounter}}

\newcommand{\problem}
    {\problemheading{Problem \arabic{problemcounter}.} \stepcounter{problemcounter}}

\newcounter{solncounter}
\setcounter{solncounter}{1}
\newcommand{\solution}
    {\problemheading{Solution to Problem \arabic{solncounter}.} \stepcounter{solncounter}}

\newcommand{\chash}{\mathcal{H}}
\newcommand{\acl}{\mathit{Auth}}
\newcommand{\opn}{\mathit{op}}
\newcommand{\egid}{\mathit{egid}}
\newcommand{\euid}{\mathit{euid}}
\newcommand{\suid}{\ttt{suid}}
\newcommand{\sgid}{\ttt{sgid}}
\newcommand{\uxroot}{\ttt{root}}
\newcommand{\fowner}[1]{\mathit{owner}_{#1}}
\newcommand{\fgroup}[1]{\mathit{group}_{#1}}
\newcommand{\gprivs}[1]{\mathit{Privs_{#1}}.\mathit{group}}
\newcommand{\uprivs}[1]{\mathit{Privs_{#1}}.\mathit{owner}}
\newcommand{\oprivs}[1]{\mathit{Privs_{#1}}.\mathit{other}}
\newcommand{\uxprivs}[1]{\mathit{Privs_{#1}}}

\newcommand{\seclab}{\mathcal{L}}
\newcommand{\sle}{\preceq}
\newcommand{\ile}{\preceq_I}
\newcommand{\ilab}{\seclab_I}

\newcommand{\minifed}{\mathit{Overture}}
\newcommand{\minicat}{\minifed}
\newcommand{\fedprot}{\minifed}
\newcommand{\metaprot}{\mathit{Prelude}}
\newcommand{\mlscat}{\mathit{mlscat}}
\newcommand{\flowcat}{\mathit{flowcat}}
\newcommand{\dflowcat}{\mathit{dflowcat}}
\newcommand{\minicatde}{\mathit{minicat}_{\mathit{de}}}
\newcommand{\minicatexp}{\mathit{minicat}_{\mathit{taint}}}
%\newcommand{\prog}{\mathcal{P}}
\newcommand{\prog}{\pi}
\newcommand{\main}{\mathit{main}}
\renewcommand{\reval}{\redx}
\renewcommand{\Dite}{\eite}


%\renewcommand{\labty}[2]{#2}
\newcommand{\fnsty}{\Sigma}
\newcommand{\secty}{\latel}

\newcommand{\tc}{\mathrm{TC}}
\newcommand{\validate}{\mathrm{validate}}


\newcommand{\mlsid}[1]{\mathrm{mls}(#1)}
\newcommand{\mlsredx}[1]{\redx_{\mlsid{#1}}}
\newcommand{\confid}{\mathit{flow}}
\newcommand{\taintid}{\mathit{dflow}}
\newcommand{\credx}{\redx_{\confid}}
\newcommand{\tredx}{\redx_{\taintid}}
\newcommand{\ccod}[1]{\lcod{\confid}{#1}}
\newcommand{\tcod}[1]{\lcod{\taintid}{#1}}
\renewcommand{\mod}{\ \textrm{mod}\ }


\newcommand{\mtrace}[1]{\mathit{trace}_{#1}}
\newcommand{\mtraces}[1]{\mathit{traces}_{#1}}
\newcommand{\head}{\mathit{hd}}
\newcommand{\memt}{\mathit{mems}}

\newcommand{\bop}{\ \mathit{binop}\ }
\newcommand{\ak}{K}
\newcommand{\ik}{K_i}
%\newcommand{\deassign}[2]{\eassign{#1}{\mathrm{declassify}(#2)}
\newcommand{\deassign}[2]{#1 :=  [#2]_\wedge }
%\newcommand{\deassign}[2]{#1\ \wedge\!\,=  #2}

\newcommand{\mems}{\mathit{mems}}
\newcommand{\mto}{\mapsto}
\newcommand{\pdf}[1]{D_{#1}}
\newcommand{\margd}[2]{{#1}_{#2}}
\newcommand{\condd}[3]{#1_{({#2}|{#3})}}
\newcommand{\progd}{\mathrm{PD}}
\newcommand{\progtt}{\mathrm{BD}}
\newcommand{\vars}{\mathit{vars}}
\newcommand{\iov}{\mathit{iovars}}
\newcommand{\flips}{\mathit{flips}}
\newcommand{\keys}{\mathit{keys}}
\newcommand{\fedcat}{\minifed}

\newcommand{\sx}[2]{\texttt{s[#1,"#2"]}}
\newcommand{\fx}[2]{\texttt{f[#1,"#2"]}}
\newcommand{\vx}[2]{\texttt{v[#1,"#2"]}}

\newcommand{\IF}[1]{#1_{\mathit{i}}}
\newcommand{\idealf}{\mathcal{F}}
\newcommand{\SIM}{\mathrm{Sim}}
\newcommand{\prob}{\mathrm{Pr}}
\newcommand{\dist}{\mathrm{D}}

\def\TirName#1{\text{\sc #1}}

\newcommand{\srct}{\tau}
\newcommand{\cidty}[1]{\ttt{cid(}#1\ttt{)}}
\newcommand{\stringty}[1]{\ttt{string(}#1\ttt{)}}
\newcommand{\unity}{\mathtt{unit}}
\newcommand{\jpdty}[2]{\mathtt{jpd}(#1,#2)}
\newcommand{\viewst}{\mathcal{V}}
\newcommand{\tjudge}[5]{#1, #2 \vdash #3 : #4,#5}
\newcommand{\bet}[1]{\ttt{bool[}#1\ttt{]}}
\newcommand{\tas}{\mathcal{A}}


\newcommand{\flip}[2]{\ttt{flip[}#1\ttt{,}#2\ttt{]}}
\newcommand{\secret}[2]{\ttt{s[}#1\ttt{,}#2\ttt{]}}
\newcommand{\view}[2]{\ttt{v[}#1\ttt{,}#2\ttt{]}}
\newcommand{\oracle}[1]{\ttt{H[}#1\ttt{]}}
\newcommand{\Oracle}{H}
\renewcommand{\etrue}{\ttt{true}}
\renewcommand{\efalse}{\ttt{false}}
\newcommand{\enot}{\ttt{not}}
\newcommand{\eand}{\ttt{and}}
\newcommand{\eor}{\ttt{or}}
\newcommand{\exor}{\ttt{xor}}
\renewcommand{\elet}[3]{\ttt{let}\ #1\ \ttt{=}\ #2\ \ttt{in}\ #3}
\newcommand{\vloc}[2]{#1@#2}
\renewcommand{\redx}{\xrightarrow{}}
\renewcommand{\redxs}{\xrightarrow{}^{*}}
\newcommand{\lredx}[1]{\xrightarrow{#1}}
\newcommand{\mem}{M}
\newcommand{\randos}{R}
\newcommand{\tape}{\randos}
\newcommand{\secrets}{S}
\newcommand{\clients}{C}
\newcommand{\views}{V}
\newcommand{\str}{\varsigma}
\newcommand{\cid}{\iota}
\newcommand{\send}[2]{#1\ \ttt{:=}\ #2}
\newcommand{\OT}[3]{\ttt{OT(} #1 \ttt{,}\ #2 \ttt{,}\ #3 \ttt{)}}
\newcommand{\select}[3]{\ttt{select(} #1 \ttt{,}\ #2 \ttt{,}\ #3 \ttt{)}}
\newcommand{\codebase}{\mathcal{C}}
\newcommand{\interp}[1]{\llbracket #1 \rrbracket}
\newcommand{\finterp}[2]{\llbracket #1 \rrbracket_{#2}}
\newcommand{\prot}{\rho}
\newcommand{\Tapes}{\mathcal{R}}
\newcommand{\outloc}{\mathit{output}}
\newcommand{\pdist}{\mathit{pd}}
\newcommand{\genpdf}{\mathrm{PD}}
\newcommand{\card}[1]{|#1|}
\newcommand{\setdefn}[2]{\{#1\ |\ #2 \}}
\newcommand{\tapes}{\mathit{tapes}}
\newcommand{\nimo}{\mathit{NIMO}}
\newcommand{\pni}{\mathit{PNI}}
\newcommand{\passec}{PS}
\newcommand{\parties}{\mathcal{P}}
\newcommand{\iout}{\mathit{output}}
\newcommand{\kideal}{k_i}
\newcommand{\jpdf}{\mathrm{pdf}}
\newcommand{\leakproof}{\mathit{LP}}
\newcommand{\flab}{\ell}
\newcommand{\be}{\varepsilon}
\newcommand{\instr}{\mathbf{c}}
\newcommand{\solve}[2]{\mathit{models}\ #1\ #2}
\newcommand{\itv}{\mathit{it}}
\newcommand{\outv}{\mathit{out}}
\newcommand{\NIMO}{\mathit{NIMO}}
\newcommand{\gNIMO}{\mathit{gNIMO}}
\newcommand{\gates}{\mathit{gates}}
\newcommand{\owl}{\mathit{owl}}
\newcommand{\logit}[1]{\lfloor #1 \rfloor}
\newcommand{\runs}{\mathit{runs}}
\newcommand{\cruns}{\hat{\mathit{runs}}}
\newcommand{\cprogd}{\hat{\progd}}
\newcommand{\cprogtt}{\hat{\progtt}}
\newcommand{\datalog}{\mathit{datalog}}
\newcommand{\concat}{\ttt{|\!|}}
\newcommand{\wired}{\mathit{wired}}
\newcommand{\gc}[3]{\mathit{goc}(#1,#2,#3)}
\newcommand{\vc}[3]{#1 \vdash #2 \sim #3}
\newcommand{\sep}[3]{#1 \vdash #2 * #3}
\newcommand{\gtab}{\mathit{table}}
\newcommand{\vdefs}{\mathit{vdefs}}
\newcommand{\funcVar}{\$}
%\newcommand{\pmf}{\mathrm{Pr}}
\newcommand{\pmf}{\mathit{P}}

%%%% REVISION DEFS

\renewcommand{\flip}[1]{\ttt{r[}#1\ttt{]}}
\newcommand{\locflip}{\ttt{r[}\mathit{local}\ttt{]}}
\renewcommand{\secret}[1]{\ttt{s[}#1\ttt{]}}
\newcommand{\key}[1]{\ttt{k[}#1\ttt{]}}
\newcommand{\mesg}[1]{\ttt{m[}#1\ttt{]}}
\newcommand{\out}[1]{\elab{\ttt{out}}{#1}}
\newcommand{\rvl}[1]{\ttt{p[}#1\ttt{]}}
\renewcommand{\oracle}[1]{\ttt{H[}#1\ttt{]}}
\newcommand{\elab}[2]{#1_{#2}}
\renewcommand{\eassign}[4]{\elab{#1}{#2} := \elab{#3}{#4}}
\newcommand{\pubout}[3]{\out{#1} := \elab{#2}{#3}}
\newcommand{\reveal}[3]{\rvl{#1} := \elab{#2}{#3}}
\newcommand{\sk}[1]{\mathrm{sk}[#1]}
\newcommand{\pk}[2]{\mathrm{pk}[#1,#2]}
\newcommand{\kgen}[1]{\mathit{kgen}(#1)}
\newcommand{\adversary}{\mathcal{A}}
\newcommand{\aredx}{\redx_{\adversary}}
\newcommand{\aredxs}{\redxs_{\adversary}}
\newcommand{\arewrite}{\mathit{rewrite}_{\adversary}}
\newcommand{\cinputs}{V_{C \rhd H}}
\newcommand{\houtputs}{V_{H \rhd C}}
\newcommand{\aruns}{\mathit{runs}_\adversary}
\newcommand{\att}{\mathrm{AD}}
\newcommand{\support}{\mathit{support}}
\renewcommand{\store}{\sigma}
\newcommand{\ctxt}[2]{\{ #1 \}_{#2}}
\newcommand{\cpub}{\mathit{pub}}
\renewcommand{\runs}{\mathit{runs}}
\newcommand{\pattern}[1]{\lfloor #1 \rfloor}
\newcommand{\fcod}[1]{\lcod{#1}{}}
\renewcommand{\flips}{\mathit{rands}}
\newcommand{\kmat}{\kappa}
\renewcommand{\Oracle}{\mathbb{O}}
\newcommand{\afilter}{\mathit{afilter}}
\renewcommand{\select}[3]{\mathtt{if}\ #1\ \mathtt{then}\ #2\ \mathtt{else}\ #3}
\newcommand{\fp}{\mathit{P}}
\newcommand{\ftimes}{*}
\newcommand{\fplus}{+}
\newcommand{\fminus}{-}
\newcommand{\mactimes}{\,\hat{\ftimes}\,}%{\otimes}
\newcommand{\macplus}{\,\hat{\fplus}\,}%\oplus}
\newcommand{\macminus}{\,\hat{\fminus}\,}%{\ominus}
\newcommand{\macv}[1]{\langle #1 \rangle}
\newcommand{\mack}[2]{\langle #1 \rangle.\ttt{k}_{#2}}
\newcommand{\macshare}[1]{\langle #1 \rangle.\ttt{share}}
\newcommand{\macauth}{\mathrm{auth}}
\newcommand{\fieldty}{\mathrm{F}}
\newcommand{\cipherty}{\mathit{c}}
\newcommand{\macty}{\hat{\fieldty}}%_{\mathit{mac}}}}
\renewcommand{\unity}[1]{\mathit{U}(#1)}
\renewcommand{\labty}[3]{#1^{#2}_{#3}}
\newcommand{\memenv}{\mathcal{M}}
\newcommand{\tensor}{\multimap}
\newcommand{\lib}{\mathcal{L}}
\newcommand{\okt}{\mathit{OK}}
\newcommand{\vty}{t}
\newcommand{\disty}{\dot{\vty}}
\newcommand{\tlev}[1]{\mathcal{T}(#1)}
\newcommand{\otp}{\mathrm{sum}}
\newcommand{\macotp}{\hat{\otp}}

\long\def\cnote#1{{\small\textbf{\textit{\color{violet}(*#1 -- Chris*)}}}}
\long\def\jnote#1{{\small\textbf{\textit{\color{brown}(*#1 -- Joe*)}}}}


\newcommand{\compwrapfig}
{
  \begin{figure}[t]
    \begin{tabular}{lccccccc}
      & 
      \begin{sideways} probabilistic language \end{sideways} &
      \begin{sideways} probabilistic conditioning \end{sideways} & 
      \begin{sideways} low-level protocols \end{sideways} & 
      \begin{sideways} passive security \end{sideways} & 
      \begin{sideways} malicious security \end{sideways}& 
      \begin{sideways} hyperproperties \end{sideways}& 
      \begin{sideways} automation \end{sideways}\\\hline\hline
      Haskell EDSL \cite{6266151} & \checkmark &  & \checkmark  & \checkmark & & \checkmark & \checkmark \\\hline
      MPC in SecreC \cite{almeida2018enforcing} & \checkmark & \checkmark &   & \checkmark & & \checkmark & \checkmark \\\hline
      $\lambda_{\text{obliv}}$ \cite{darais2019language} & \checkmark & & \checkmark & & & \checkmark & \checkmark \\\hline
      PSL \cite{barthe2019probabilistic} & \checkmark & & \checkmark & & & & \\\hline
      Lilac \cite{li2023lilac} & \checkmark & \checkmark & & & & & \\\hline
      Wys$^*$ \cite{wysstar} & & & & \checkmark & & \checkmark & \checkmark\\\hline
      Viaduct \cite{10.1145/3453483.3454074,viaduct-UC} & & & & \checkmark & \checkmark & \checkmark & \checkmark\\\hline
      MPC in EasyCrypt \cite{8429300} &  \checkmark &  \checkmark &  \checkmark & \checkmark & \checkmark & \checkmark & \\\hline
      $\metaprot$/$\minifed$ \cite{skalka-near-ppdp24} & \checkmark & \checkmark & \checkmark & * & \checkmark & \checkmark & * \\\hline
      This work & \checkmark & \checkmark & \checkmark & \checkmark & \checkmark & \checkmark & \checkmark\\
      \hline
    \end{tabular}
    \caption{Comparison of systems for verification of MPC security in PLs. * indicates relatively limited support.}
    \label{fig-comp-wrap}
  \end{figure}
}

\newcommand{\minicatsyntaxfig}{
\begin{fpfig}[t]{Syntax of $\minicat$}{fig-minicat-syntax}
$$
    \begin{array}{rcl@{\hspace{4mm}}r}
      \multicolumn{4}{l}{v \in \mathbb{F}_p,\ w \in \mathrm{String},\ \cid \in \mathrm{Clients} }\\[2mm] %, \bop \in \{ \eand, \eor, \exor \}} \\[2mm]
      \be &::=& v \mid \flip{w} \mid \secret{w} \mid \mesg{w} \mid \rvl{w} \mid \be \fminus \be \mid \be \fplus \be \mid \be \ftimes \be \mid \OT{\be}{\cid}{\be}{\be} & \textit{expressions}\\[1mm]
      x &::=& \elab{\flip{w}}{\cid} \mid \elab{\secret{w}}{\cid} \mid \elab{\mesg{w}}{\cid} \mid  \rvl{w} \mid \out{\cid} & \textit{variables} \\[1mm]
      \prog &::=& \eassign{\mesg{w}}{\cid}{\be}{\cid} \mid \reveal{w}{\be}{\cid} \mid \pubout{\cid}{\be}{\cid} \mid \prog;\prog \mid \pskip & \textit{protocols}
    \end{array}
$$
\end{fpfig}    
}

\newcommand{\minicatredxfig}{
\begin{fpfig}[t]{Semantics of $\minicat$ expressions (T) and programs (B).}{fig-minicat-redx}
 $$
  \begin{array}{c@{\hspace{5mm}}c}
  \begin{array}{rcl}
    \lcod{\store, v}{\cid} &=& v\\
    \lcod{\store, \be_1 \fplus \be_2}{\cid} &=& \fcod{\lcod{\store, \be_1}{\cid} \fplus \lcod{\store, \be_2}{\cid}}\\ 
    \lcod{\store, \be_1 \fminus \be_2}{\cid} &=& \fcod{\lcod{\store, \be_1}{\cid} \fminus \lcod{\store, \be_2}{\cid}}\\ 
    \lcod{\store, \be_1 \ftimes \be_2}{\cid} &=& \fcod{\lcod{\store, \be_1}{\cid} \ftimes \lcod{\store, \be_2}{\cid}}\\
  \end{array} & 
  \begin{array}{rcl}
    \lcod{\store, \flip{w}}{\cid} &=& \store(\elab{\flip{w}}{\cid})\\
    \lcod{\store, \secret{w}}{\cid} &=& \store(\elab{\secret{w}}{\cid})\\
    \lcod{\store, \mesg{w}}{\cid} &=& \store(\elab{\mesg{w}}{\cid})\\
    \lcod{\store, \rvl{w}}{\cid} &=& \store(\rvl{w})\\
    %\lcod{\store, \OT{\be_1}{\cid_1}{\be_2}{\be_3}}{\cid_2} &=&
    %\begin{cases}
    %  \lcod{\store,\be_2}{\cid_2} \text{\ if\ } \lcod{\store,\be_1}{\cid_1} = 0 \\
    %  \lcod{\store,\be_3}{\cid_2} \text{\ if\ } \lcod{\store,\be_1}{\cid_1} = 1 \\
    %\end{cases}
  \end{array}
  \end{array}
  $$

  %$$
  %\lcod{\store, \OT{\be_1}{\cid_1}{\be_2}{\be_3}}{\cid_2} =
  %  \begin{cases}
  %    \lcod{\store,\be_2}{\cid_2} \text{\ if\ } \lcod{\store,\be_1}{\cid_1} = 0 \\
  %    \lcod{\store,\be_3}{\cid_2} \text{\ if\ } \lcod{\store,\be_1}{\cid_1} = 1 \\
  %  \end{cases}
  %$$
  %
\begin{mathpar}
  (\store, \pskip) \redx \store
  
  (\store, \xassign{x}{\be}{\cid}) \redx \extend{\store}{x}{\lcod{\store,\be}{\cid}}
  
  \inferrule
      {(\store_1,\prog_1) \redx \store_2 \\ (\store_2,\prog_2) \redx \store_3 }
      {(\store_1,\prog_1;\prog_2) \redx \store_3}
      %(\store, \eassign{\mesg{w}}{\cid_1}{\be}{\cid_2};\prog) \redx (\extend{\store}{\mesg{w}_{\cid_1}}{\lcod{\store,\be}{\cid_2}}, \prog)    
      %(\store, \reveal{w}{\be}{\cid};\prog) \redx (\extend{\store}{\rvl{w}}{\lcod{\store,\be}{\cid}}, \prog)   
      %(\store, \pubout{\cid}{\be}{\cid};\prog) \redx (\extend{\store}{\out{\cid}}{\lcod{\store,\be}{\cid}}, \prog)
\end{mathpar}
\end{fpfig}
}

\newcommand{\minicataredxfig}{
\begin{fpfig}[t]{Adversarial semantics of $\minicat$.}{fig-minicat-aredx}
\begin{mathpar}
  \inferrule
      { \cid \in H }
      { (\store, \xassign{x}{\be}{\cid}) \aredx \extend{\store}{x}{\lcod{\store,\be}{\cid}} }
      
  \inferrule
      {\cid \in C }
      { (\store, \xassign{x}{\be}{\cid}) \aredx \extend{\store}{x}{\lcod{\arewrite(\store_C,\be)}{\cid}}}
      
  \inferrule
      {\lcod{\store,\be_1}{\cid} = \lcod{\store,\be_2}{\cid}  \text{\ or\ } \cid \in C}
      { (\store,\elab{\assert{\be_1 = \be_2}}{\cid}) \aredx \store }
      
  \inferrule
      {\lcod{\store,\be_1}{\cid} \ne \lcod{\store,\be_2}{\cid}}
      {(\store,\elab{\assert{\be_1 = \be_2}}{\cid}) \aredx \abort}
  
  \inferrule
      {(\store_1,\prog_1) \aredx \store_2 \\ (\store_2,\prog_2) \aredx \store_3 }
      {(\store_1,\prog_1;\prog_2) \aredx \store_3}

  \inferrule
      {(\store_1,\prog_1) \aredx \abort}
      {(\store_1,\prog_1;\prog_2) \aredx \abort}
      
  \inferrule
      {(\store_1,\prog_1) \aredx \store_2 \\ (\store_2,\prog_2) \aredx \abort }
      {(\store_1,\prog_1;\prog_2) \aredx \store_2}
\end{mathpar}
\end{fpfig}
}

\newcommand{\minicathtripfig}{
\begin{fpfig}[t]{$\minicat$ Hoare logic derivation rules}{fig-minicathtrip}
\begin{mathpar}
  \inferrule[Skip]
            {}
            {\htrip{\eqs}{\pskip}{\eqs}}
            
  \inferrule[Assign]
            {}
            {\htrip{\eqs[\toeq{\elab{\be}{\cid_2}}/x]}{\xassign{x}{\be}{\cid_2}}{\eqs}}
  
  \inferrule[Assert]
            {E \models \toeq{\elab{\be_1}{\cid}} \eop \toeq{\elab{\be_2}{\cid}}}
            {\htrip{E}{\elab{\assert{\be_1 = \be_2}}{\cid}}{E}}

  %\reveal{w}{\be}{\cid}
  %
  %\pubout{\cid}{\be}{\cid}
  \inferrule[Seq]
      {\htrip{\eqs_1}{\prog_1}{\eqs_2} \\ \htrip{\eqs_2}{\prog_2}{\eqs_3} }
      {\htrip{\eqs_1}{\prog_1;\prog_2}{\eqs_3}}

  \inferrule[Consequence]
      {\htrip{\eqs_1'}{\prog}{\eqs_2'} \\ \eqs_1 \models \eqs_1' \\  \eqs_2' \models \eqs_2 }
      {\htrip{\eqs_1}{\prog}{\eqs_2}}

  \inferrule[Frame]
      {\htrip{\eqs_1}{\prog}{\eqs_2} \\ \vars(\eqs) \cap \avars(\prog) = \varnothing}
      {\htrip{\eqs_1 \wedge \eqs}{\prog}{\eqs_2 \wedge \eqs}}
      
\inferrule[Hyp]
          {\htrip{\eqs_1}{\prog}{\eqs_2}}
          {\htrip{\eqs_1 \wedge (\eqs_2 \impl \eqs)}{\prog}{\eqs_2 \wedge \eqs}}
\end{mathpar}
\end{fpfig}
}

\newcommand{\stylocalfig}{
\begin{fpfig}[t]{Local (non-interactive) share type derivation rules.}{fig-stylocal}
\begin{mathpar}
  \inferrule[Shared]
      {(\stt : \sty) \in \Gamma}
      {\Gamma,\eqs \vdash \stt : \sty}
      
  \inferrule[Val]
      {}
      {\Gamma,\eqs \vdash \pubx{v} : \pubty}

  \inferrule[HEAdd]
      {\Gamma,\eqs \vdash [\phi^1_1,\phi^1_2] : \sharety\\
        \Gamma,\eqs \vdash [\phi^2_1,\phi^2_2] : \sharety}
      {\Gamma,\eqs \vdash [\phi^1_1\fplus\phi^2_1,\phi^1_2\fplus\phi^2_2] : \sharety}

  \inferrule[HEAddPub]
      {\Gamma,\eqs \vdash \pubx{\phi} : \pubty\\
        \Gamma,\eqs \vdash [\phi_1,\phi_2] : \sharety}
      {\Gamma,\eqs \vdash [\phi_1\fplus\phi,\phi_2] : \sharety}
            
  \inferrule[HEMultPub]
      {\Gamma,\eqs \vdash [\phi] : \pubty\\
        \Gamma,\eqs \vdash [\phi_1,\phi_2] : \sharety}
      {\Gamma,\eqs \vdash [\phi_1\ftimes\phi,\phi_2\ftimes\phi] : \sharety}

  \inferrule[PubOp]
      {\Gamma,\eqs \vdash \pubx{\phi_1} : \pubty\\
        \Gamma,\eqs \vdash \pubx{\phi_2} : \pubty}
      {\Gamma,\eqs \vdash \pubx{\phi_1\bop\phi_2}: \pubty}
      
  \inferrule[ShareEntails]
      {\Gamma,\eqs \vdash [\phi_1',\phi_2'] : \sharety \\
        \eqs \models \phi_1 \eop \phi_1' \wedge \phi_2 \eop \phi_2'}
      {\Gamma,\eqs \vdash [\phi_1,\phi_2] : \sharety} 
      
  \inferrule[PubEntails]
      {\Gamma,\eqs \vdash \pubx{\phi'} : \sty \\ \eqs \models \phi \eop \phi'}
      {\Gamma,\eqs \vdash \pubx{\phi} : \sty} 
\end{mathpar}
\end{fpfig}
}

\newcommand{\styinteractivefig}{
\begin{fpfig}[t]{Interactive share type derivation rules.}{fig-styinteractive}
\begin{mathpar}
  \inferrule[GenShares]
      {
        \genshares(\phi,\rx{w}{\cid}) = \phi_1,\phi_2 \\ 
        \eqs \models x_1 \eop \phi_1 \wedge x_2 \eop \phi_2  
      }
      {\eqs \vdash \Gamma,R \stredx \Gamma \cup \setit{[x_1,x_2] : \sharety}, R \uplus \setit{\rx{w}{\cid}}}

  \inferrule[SecOpen]
      {
        \Gamma, \eqs \vdash [\px{w_1},\px{w_2}] : \sharety\\
        \eqs \models  \px{w_1} + \px{w_2} \eop \phi' + \rx{w}{\oid}
      }
      {\eqs \vdash \Gamma,R \stredx \Gamma \cup \setit{\pubx{\recon(\px{w_1},\px{w_2})}  : \pubty}, R \uplus \setit{\rx{w}{\cid_{\oid}}}}

  \inferrule[Reveal]
      {
        \Gamma, \eqs \vdash [\px{w_1},\px{w_2}] : \sharety
      }
      {\eqs \vdash \Gamma,R \stredx \Gamma \cup \setit{\pubx{\recon(\px{w_1},\px{w_2})}  : \outty}, R}

  \inferrule[Output]
      {
        \Gamma, \eqs \vdash \pubx{\out{\cid}} : \outty
      }
      {\eqs \vdash \Gamma,R \stredx \Gamma \cup \setit{\pubx{\out{\cid}} : \outty}, R}
\end{mathpar}
\end{fpfig}
}

\newcommand{\cpjfig}{
\begin{fpfig}[t]{Syntax and Derivation Rules for $\minicat$ Confidentiality Types}{fig-cpj}
\small{
$$
\begin{array}{rcl@{\hspace{3mm}}l}
  t &::=& x \mid \cty{x}{T} \\
  \ty & \in & 2^{t} & \gdesc{confidentiality types}\\
  \Gamma &::=& \varnothing \mid \Gamma; x : \ty & \gdesc{confidentiality type environments}
\end{array} 
$$
\medskip
\begin{mathpar}
  \inferrule[DepTy]
  {}
  {\eqj{\varnothing}{\eqs}{\phi}{\vars(\phi)}}
  
  \inferrule[Encode]
  {\eqs \models \phi \eop \phi' \oplus \rx{w}{\cid} \\
   \oplus \in \{ \fplus,\fminus \}\\
   \eqj{R}{\eqs}{\phi'}{\ty}}
  {\eqj{R;\{ \rx{w}{\cid} \}}{\eqs}{\phi}{\setit{\cty{\rx{w}{\cid}}{\ty}}}}
\end{mathpar}

\begin{mathpar}
  \inferrule[Send]
            {\eqj{R}{\eqs}{\phi}{\ty}}
            {\cpj{R}{\eqs}{x \eop \phi}{(x : \ty)}}
            
  \inferrule[Seq]
            {\cpj{R_1}{\eqs}{\phi_1}{\Gamma_1}\\
             \cpj{R_2}{\eqs}{\phi_2}{\Gamma_2}}
            {\cpj{R_1;R_2}{\eqs}{\phi_1 \wedge \phi_2}{\Gamma_1;\Gamma_2}}
\end{mathpar}
}
\end{fpfig}
}

\newcommand{\leakjfig}{
\begin{fpfig}[t]{Dependencies in Views: Derivation Rules}{fig-leakj}
\small{
\begin{mathpar}
  \inferrule
      {\leakclose{\Gamma}{\ty_1}{\ty_2} \\ \leakclose{\Gamma}{\ty_2}{\ty_3}}
      {\leakclose{\Gamma}{\ty_1}{\ty_3}}

      \leakclose{\Gamma}{\ty \cup \setit{\mx{w}{\cid}}}{\ty \cup \Gamma(\mx{w}{\cid})}

      \leakclose{\Gamma}{\ty_1 \cup \setit{x, \cty{x}{\ty_2}}}{\ty_1\cup\ty_2}
\end{mathpar}

\begin{mathpar}
  \inferrule
      {}
      {\leakj{\Gamma}{\varnothing}{\varnothing}}

\inferrule
      {\leakj{\Gamma}{M}{\ty'} \\ \leakclose{\Gamma}{\ty'\cup\setit{x}}{\ty}}
      {\leakj{\Gamma}{M \cup \setit{x}}{\ty}}
\end{mathpar}
}
\end{fpfig}
}

\newcommand{\ipjfig}{
\begin{fpfig}[t]{Syntax and derivation rules of $\minicat$ integrity types}{fig-ipj}
\small{
$$
\begin{array}{rcl@{\hspace{4mm}}l}
  \latel &::=& \hilab \mid \lolab & \gdesc{integrity labels} \\
  \Delta &::=& \varnothing \mid \Delta; x : \ity{\cid}{V} & \gdesc{integrity type environments}
\end{array} 
$$

\begin{mathpar}
  \inferrule[Value]
  {}
  {\itj{\cid}{v}{\varnothing}}
  
  \inferrule[Secret]
  {}
  {\itj{\cid}{\secret{w}}{\varnothing}}
  
  \inferrule[Rando]
  {}
  {\itj{\cid}{\flip{w}}{\varnothing}}
  
  \inferrule[Mesg]
  {}
  {\itj{\cid}{\mesg{w}}{\setit{\mx{w}{\cid}}}}
    
  \inferrule[PubM]
  {}
  {\itj{\cid}{\rvl{w}}{\setit{\rvl{w}}}}

  \inferrule[Binop]
  {\itj{\cid}{\be_1}{V_1} \\
   \itj{\cid}{\be_2}{V_2} \\ \oplus \in \{ \fplus,\fminus,\ftimes \}}
  {\itj{\cid}{\be_1 \oplus \be_2}{V_1 \cup V_2}}
%
%  \inferrule[IntegrityWeaken]
%  {\itj{\Delta}{\eqs}{\cid}{\be}{\latel_1} \\ \latel_1 \sle \latel_2}
%  {\itj{\Delta}{\eqs}{\cid}{\be}{\latel_2}}
\end{mathpar}

\begin{mathpar}
  \inferrule[Send]
            {\itj{\cid}{\be}{V}}
            {\ipj{\eqs}{\xassign{x}{\be}{\cid}}{(x : \ity{\cid}{V})}}
             
%  \inferrule[Assert]
%            {\eqs \models \toeq{\elab{\be_1}{\cid}} = \toeq{\elab{\be_2}{\cid}}}
%            {\ej{\Delta}{R}{\eqs}{\elab{\assert{\be_1 = \be_2}}{\cid}}{\Delta}{\eqs}}
%            
  \inferrule[Seq]
            {\ipj{\eqs}{\prog_1}{\Delta_1}\\
             \ipj{\eqs}{\prog_2}{\Delta_2}}
            {\ipj{\eqs}{\prog_1;\prog_2}{\Delta_1;\Delta_2}}

  \inferrule[MAC]
            {\eqs \models \toeq{\elab{\assert{\macbdoz{w}}}{\cid}}}
            {\ipj{\eqs}{\elab{\assert{\macbdoz{w}}}{\cid}}{(\mx{w\ttt{s}}{\cid}: \ity{\cid}{\varnothing})}}
%
%  \inferrule[MAC]
%            {\eqs \models 
%              \mx{w\ttt{m}}{\cid} \eop \mx{w\ttt{k}}{\cid} \fplus \ttt{(}\mx{\ttt{delta}}{\cid} \ftimes
%                  \mx{w\ttt{s}}{\cid}\ttt{)}}
%            {\ipj{\Delta}{\eqs}{
%                \elab{\assert{\mesg{w\ttt{m}} \eop \mesg{w\ttt{k}} \fplus \ttt{(}\mesg{\ttt{delta}} \ftimes
%                  \mesg{w\ttt{s}}\ttt{)}}}{\cid}}{\Delta;\mx{w\ttt{s}}{\cid}: \hilab }}
\end{mathpar}
}
\end{fpfig}
}

\newcommand{\cheatjfig}{
\begin{fpfig}[t]{Assigning integrity labels to variables}{fig-cheatj}
\small{
\begin{mathpar}
  \inferrule
      {}
      {\cheatj{\varnothing}{H,C}{\seclev_{H,C}}}
      
  \inferrule
      {\cheatj{\Delta}{H,C}{\seclev} \\ \cid \in H}
      {\cheatj{\Delta; x : \ity{\cid}{V}}{H,C}{\extend{\seclev}{x}{\hilab \wedge (\bigwedge_{x \in V} \seclev_2(x))}}}
      
  \inferrule
      {\cheatj{\Delta}{H,C}{\seclev} \\ \cid \in C}
      {\cheatj{\Delta; x : \ity{\cid}{V}}{H,C}{\extend{\seclev}{x}{\lolab}}}
\end{mathpar}
}
\end{fpfig}
}

\newcommand{\metaprotsyntaxfig}{
  \begin{fpfig}[t]{$\metaprot$ syntax}{fig-metaprotsyntax}
$$
\begin{array}{rcl@{\hspace{7mm}}r}
  %\notg{x} &::=& \elab{\flip{e}}{e} \mid \elab{\secret{e}}{e} \mid \elab{\mesg{e}}{e} \mid \rvl{e} \mid \out{e}\\[2mm]
  \multicolumn{3}{l}{\flab \in \mathrm{Field},\   y \in \mathrm{EVar}, \  f \in \mathrm{FName}}\\[1mm]
  %x &\in& \mathrm{EVar}\\
  %f &\in& \mathrm{FName}\\[2mm]
  e &::=& \mv \mid \flip{e} \mid \secret{e} \mid \mesg{e} \mid \rvl{e} \mid \outkw \mid e \bop e 
   \mid y \mid & \gdesc{expressions}\\
  & & e.\flab \mid \elab{e}{e} \mid \elet{y}{e}{e} \mid  f(e,\ldots,e) \mid \{ \flab = e; \ldots; \flab = e \} \\[1mm]
  %  & \textit{expressions}\\
  \cmd &::=& %\msend{e}{e}{e}{e} \mid \reveal{e}{e}{e} \mid \pubout{e}{e}{e} \mid
  \assign{e}{e} \mid f(e,\ldots,e) \mid \elet{y}{e}{\instr} \mid  \cmd;\cmd 
    \mid \elab{\assert{e = e}}{e} & \gdesc{instructions}\\[1mm] %\pre{\eqs} \mid \post{\eqs} \\[1mm]
  \bop &::=& \fplus \mid \fminus \mid \ftimes \mid \concat  \\[1mm]% \textit{operators}\\[2mm]
  \mv &::=& w \mid \cid \mid \be \mid x \mid \{ \flab = \mv;\ldots;\flab = \mv \} 
  & \gdesc{values}\\[1mm] % \mid \ttt{()} \\[1mm] %& \textit{values}\\[2mm]
  \tau &::=& \stringty \mid \cidty \mid \{ \flab_1 : \tau_1; \ldots; \flab_n : \tau_n \} & \gdesc{basic types}\\[1mm]
\mathit{fn} &::=& f(y,\ldots,y) \{ e \} \mid  f(y : \tau,\ldots,y : \tau) \{ \cmd \} & \textit{functions} % \mathit{fn} &::=& f(y,\ldots,y) \{ e \} \mid  f(y,\ldots,y) \{ \cmd \} & \textit{functions}
  %\phi &::=& \elab{\flip{e}}{e} \mid \elab{\secret{e}}{e} \mid \elab{\mesg{e}}{e} \mid \rvl{e} \mid \out{e} \mid \phi \fplus \phi \mid \phi \fminus \phi \mid \phi \ftimes \phi \\
  %\eqs &::=& \phi \eop \phi \mid \eqs \wedge \eqs 
\end{array}
$$
\end{fpfig}
}

\newcommand{\metaprotexprsemanticsfig}{
  \begin{fpfig}[t]{Semantics of $\metaprot$ expressions.}{fig-metaprotexprsemantics}
  \begin{mathpar}
  \inferrule
      {e_1 \redx \mv \\ e_2[\mv/y] \redx \mv'}
      {\elet{y}{e_1}{e_2} \redx \mv'}
      
  \inferrule
      {e_1 \redx \be \\ e_2 \redx \cid}
      {\elab{e_1}{e_2} \redx \elab{\be}{\cid}}

  \inferrule
      {\codebase(f) = y_1,\ldots,y_n,\ e \\ e_1 \redx \mv_1 \cdots e_n \redx \mv_n \\
        e[\mv_1/y_1]\cdots[\mv_n/y_n] \redx \mv}
      {f(e_1,\ldots,e_n) \redx \mv}

  \inferrule
      {e_1 \redx \mv_1 \\ \cdots \\ e_n \redx \mv_n }
      {\{ \flab_1 = e_1; \ldots; \flab_n = e_n \} \redx \{ \flab_1 = \mv_1; \ldots; \flab_n = \mv_n \} }

  \inferrule
      {e \redx \{\ldots; \flab = \mv; \ldots\}}
      {e.\flab \redx \mv}

  \inferrule
      {e_1 \redx w_1 \\ e_2 \redx w_2}
      {e_1 \concat e_2 \redx w_1w_2}

  \inferrule
      {e \redx w}
      {\mesg{e} \redx \mesg{w}}
      
  \inferrule
      {e_1 \redx \be_1 \\ e_2 \redx \be_2}     
      {e_1 \fplus e_2 \redx \be_1 \fplus \be_2}
      
  \inferrule
      {}     
      {\mv \redx \mv}
\end{mathpar}
\end{fpfig}
}

\newcommand{\metaprotinstrsemanticsfig}{
  \begin{fpfig}[t]{Semantics of $\metaprot$ instructions.}{fig-metaprotinstrsemantics}
%\small{
\begin{mathpar}
  \inferrule
      {e_1 \redx x \\ e_2 \redx \elab{\be}{\cid}}
      {\assign{e_1}{e_2} \redx \xassign{x}{\be}{\cid}}

  \inferrule
      {e_1 \redx \be_1 \\ e_2 \redx \be_2 \\ e_3 \redx \cid}
      {\elab{\assert{e_1 = e_2}}{e_3} \redx \elab{\assert{\be_1 = \be_2}}{\cid}}

  \inferrule
      {\codebase(f) = y_1 : \tau_1, \ldots, y_n : \tau_n, \instr \\
        e_1 \redx \mv_1 \ \cdots \ e_n \redx \mv_n \\
        \instr[\mv_1/y_1]\cdots[\mv_n/y_n] \redx \prog
      }
      {f(e_1,\ldots,e_n) \redx \prog}
      
  \inferrule
      {e \redx \mv \\ \instr[\mv/y] \redx \prog}
      {\elet{y}{e}{\instr} \redx \prog}

  \inferrule
      {e_1 \redx \prog_1 \\ e_2 \redx \prog_2}
      {e_1;e_2 \redx \prog_1;\prog_2}
\end{mathpar}
%}
\end{fpfig}
}

\newcommand{\atjfig}{
  \begin{fpfig}[t]{Algorithmic type judgements for $\minicat$.}{fig-atj}
\small{
\begin{mathpar}
  \atj{x}{\varnothing}{\setit{x}}

  \inferrule
  {\atj{\phi}{R}{\ty} \\ \rx{w}{\cid}\not\in R \\ \oplus \in \setit{\fplus,\fminus}}
  {\atj{\phi \oplus \rx{w}{\cid}}{R \cup \setit{\rx{w}{\cid}}}{\setit{\cty{\rx{w}{\cid}}{\ty}}}}

  \inferrule
  {\atj{\phi_1}{R_1}{\ty_1} \\
   \atj{\phi_2}{R_2}{\ty_2} \\ \oplus \in \{ \fplus,\fminus,\ftimes \}}
  {\atj{\phi_1 \oplus \phi_2}{R_1;R_2}{\ty_1 \cup \ty_2}}
\end{mathpar}
}
\end{fpfig}
}

\newcommand{\peqfig}{
\begin{fpfig}[t]{Abstract constraints and their evaluation semantics.}{fig-peq}
$$
  \begin{array}{rclr}
    \peq &::=& e \eop e \mid \notg{\eqs} \wedge \notg{\eqs} \mid \notg{\eqs} \vee \notg{\eqs}
  \mid \notg{\eqs} \impl \notg{\eqs} \mid \neg\eqs & \qquad \textit{abstract constraints}
  \end{array}
$$
\begin{mathpar}
  \inferrule
      {e_1 \redx \elab{\be_1}{\cid_1} \\ e_2 \redx \elab{\be_2}{\cid_2} }
      {e_1 \bop e_2 \redx \toeq{\elab{\be_1}{\cid_1}} \bop \toeq{\elab{\be_2}{\cid_2}}}
      
  \inferrule
      {e_1 \redx \elab{\be_1}{\cid_1} \\ e_2 \redx \elab{\be_2}{\cid_2} }
      {e_1 \eop e_2 \redx \toeq{\elab{\be_1}{\cid_1}} \eop \toeq{\elab{\be_2}{\cid_2}}}
      
  \inferrule
      {\notg{\eqs_1} \redx \eqs_1 \\ \notg{\eqs_2} \redx \eqs_2 }
      {\notg{\eqs_1} \wedge \notg{\eqs_2} \redx \eqs_1 \wedge \eqs_2}

  \inferrule
      {\notg{\eqs_1} \redx \eqs_1 \\ \notg{\eqs_2} \redx \eqs_2 }
      {\notg{\eqs_1} \vee \notg{\eqs_2} \redx \eqs_1 \vee \eqs_2}

  \inferrule
      {\notg{\eqs_1} \redx \eqs_1 \\ \notg{\eqs_2} \redx \eqs_2 }
      {\notg{\eqs_1} \impl \notg{\eqs_2} \redx \eqs_1 \impl \eqs_2}

  \inferrule
      {\peq \redx \eqs}
      {\neg\peq\redx \neg\eqs}

  \inferrule
      {}
      {\eqs \redx \eqs}
\end{mathpar}
\end{fpfig}
}

\newcommand{\metahtripfig}{
\begin{fpfig}[t]{Algorithmic $\metaprot$ Abstract Entailment and Hoare Triple Deduction Rules}{fig-metahtrip}
\begin{mathpar}          
  \inferrule[Mesg]
            {}
            {\htrip{\eqtrue}{\xassign{e_1}{e_2}{e_3}}{e_1 \eop \elab{e_2}{e_3}}}

  %\inferrule[Encode]
  %          {\mx{e_1}{e_2} \redx x \\ \notg{\phi} \redx \phi \\
  %            \eqs \models x \eop \phi\\
  %            \atj{\phi}{R}{\ty}}
  %          {\mtj{\eqcast{\mx{e_1}{e_2}}{\notg{\phi}}}{\eqs}{(x : \ty)}{R}{\varnothing}{\eqs}}
  %
  \inferrule[Assert]
            {}
            {\htrip{\elab{e_1}{e_3} \eop \elab{e_2}{e_3}}{\elab{\assert{e_1 = e_2}}{e_3}}{\eqtrue}}

  \inferrule[Seq]          
            {\htrip{\peq_1^1}{\cmd_1}{\peq_2^1} \\ \htrip{\peq_1^2}{\cmd_2}{\peq_2^2}}
            {\htrip{\peq_1^1 \wedge (\peq_2^1 \impl \peq_1^2)}{\cmd_1;\cmd_2}{\peq_2^1 \wedge \peq_2^2}}

  \inferrule[Let]
            {\htrip{\peq_1}{\cmd[e/y]}{\peq_2}}
            {\htrip{\peq_1}{\elet{y}{e}{\cmd}}{\peq_2}}

  \inferrule[App]
            {\htrip{\peq_1}{f(y_1:\tau_1,\ldots,y_n:\tau_n)}{\peq_2}}
            {\htrip{\peq_1[e_1/y_1 \cdots e_n/y_n]}{f(e_1,\ldots,e_n)}{\peq_2[e_1/y_1 \cdots e_n/y_n]}}

  \inferrule[Fn]
            {\codebase(f) = y_1 : \tau_1, \ldots, y_n : \tau_n, \cmd \\ \htrip{\peq_1}{\cmd}{\peq_2}}
            {\htrip{\peq_1}{f(y_1:\tau_1,\ldots,y_n:\tau_n)}{\peq_2}}

  \inferrule[Hardpack]
            {\precond(f) = \peq_1 \\ \postcond(f) = \peq_2 \\
              \htrip{\peq_1'}{f(y_1:\tau_1,\ldots,y_n:\tau_n)}{\peq_2'} \\
              \forall y_1:\tau_1,\ldots,y_n:\tau_n . 
            \peq_1 \models \peq_1'\ \text{and}\ \peq_1 \wedge \peq_2' \models \peq_2}
            {\htrip{\peq_1}{f(y_1:\tau_1,\ldots,y_n:\tau_n)}{\peq_2}}
\end{mathpar}\\[4mm]
$$
  \inferrule[GenEntails]
          {\mv_1,\ldots,\mv_n = \fresh(\tau_1,\ldots,\tau_n) \\ \peq_1[\mv_1/y_n \cdots \mv_n/y_n] \redx \eqs_1 \\
           \peq_2[\mv_1/y_n \cdots \mv_n/y_n] \redx \eqs_2 \\ \eqs_1 \models \eqs_2}
          {\forall y_1:\tau_1,\ldots,y_n:\tau_n . \peq_1 \models \peq_2 }
$$
\end{fpfig}
}

\newcommand{\stafig}{
\begin{fpfig}[t]{$\metaprot$ share type annotation syntax (top), semantics (middle), and
    hoare triple derivation rules (bottom).}{fig-sta}
$$
\begin{array}{rclr}
  \notg{\stt} &::=& [e,e] \mid \pubx{e}   & \qquad \textit{abstract share terms} \\   
  \sta &::=& e_1 \kwas e_2 \mid \notg{\stt} : \sty & \qquad \textit{share type annotations} \\     
  \cmd &::=& \cdots \mid \sta
\end{array}
$$

\bigskip

\begin{mathpar}
  \inferrule
      {e_1 \redx \phi_1 \\ e_2 \redx \phi_2}
      {[e_1,e_2] \redx [\phi_1,\phi_2]}
      
  \inferrule
      {e \redx \phi}
      {\pubx{e} \redx \pubx{\phi}}
      
  \inferrule
      {e_1 \redx \phi_1 \\ e_2 \redx \phi_2}
      {e_1 \kwas e_2 \redx \pskip}

  \inferrule
      {\notg{\stt} \redx \stt}
      {\stt : \sty \redx \pskip}
\end{mathpar}

\bigskip

\begin{mathpar}
  \htrip{e_1 \eop e_2}{e_1 \kwas e_2}{\true}

  \htrip{\true}{\notg{\stt} : \sty}{\true}
\end{mathpar}
\end{fpfig}
}

\newcommand{\shtripstafig}{
\begin{fpfig}[t]{Typing rules for $\metaprot$ share type annotations.}{fig-shtripsta}
\begin{mathpar}
  \inferrule[Coerce]
      {e_1 \redx \phi_1 \\ e_2 \redx \phi_2}
      {\shtrip{\Gamma, R, \xdefs}{e_1 \kwas e_2}{\Gamma, R, \xdefs[\phi_1 \mapsto \phi_2]}}

  \inferrule[LocalShares]
      {e_1 \redx x_1 \\ e_2 \redx x_2  \\
        \Gamma \Vdash [\xdefs(x_1), \xdefs(x_2)] : \sharety}
      {\shtrip{\Gamma, R, \xdefs}{[e_1,e_2] : \sharety}
        {\Gamma \cup \setit{[x_1,x_2] : \sharety}, R, \xdefs}}

  \inferrule[GenShares]
      {e_1 \redx x_1 \\ e_2 \redx x_2  \\ \genshares(\phi, \rx{w}{\cid}) = \xdefs(x_1),\xdefs(x_2)}
      {\shtrip{\Gamma, R, \xdefs}{[e_1,e_2] : \sharety}
        {\Gamma \cup \setit{[x_1,x_2] : \sharety}, R \uplus \setit{\rx{w}{\cid}}, \xdefs}}

  \inferrule[LocalPublic]
      {e \redx \phi \\ \Gamma \Vdash \pubx{\xdefs(\phi)} : \pubty}
      {\shtrip{\Gamma, R, \xdefs}{\pubx{e} : \pubty}
        {\Gamma \cup \setit{\pubx{\phi} : \pubty}, R, \xdefs}}

  \inferrule[MaskedPublic]
      {e_1 \redx x_1 \\ e_2 \redx x_2  \\
        \Gamma \Vdash [x_1, x_2] : \sharety \\
        \xdefs(\recon(x_1,x_2))  = \phi + \rx{w}{\oid}}
      {\shtrip{\Gamma, R, \xdefs}{\pubx{\recon(e_1,e_2)} : \pubty}
        {\Gamma \cup \setit{\pubx{\recon(x_1,x_2)} : \pubty}, R \uplus \setit{\rx{w}{\oid}}, \xdefs}}

  \inferrule[Reveal]
      {e_1 \redx x_1 \\ e_2 \redx x_2  \\
        \Gamma \Vdash [x_1, x_2] : \sharety}
      {\shtrip{\Gamma, R, \xdefs}{\pubx{\recon(e_1,e_2)} : \outty}
        {\Gamma \cup \setit{\pubx{\recon(x_1,x_2)} : \outty}, R, \xdefs}}

  \inferrule[Output]
      {e \redx \out{\cid} \\
       %\xdefs(\out{\cid}) = \recon(x_1,x_2) \\
       \Gamma \Vdash \pubx{\xdefs(\out{\cid})} : \outty }
      {\shtrip{\Gamma, R, \xdefs}{\out{\cid} : \outty}
        {\Gamma \cup \setit{\pubx{\out{\cid}} : \outty}, R, \xdefs}}
\end{mathpar}
\end{fpfig}
}

\newcommand{\shtripcmdfig}{
\begin{fpfig}[t]{Typing rules for $\metaprot$ instructions and functions.}{fig-shtripcmd}
\begin{mathpar}
  \inferrule[Defn]
      {e_1 \redx x \\ e_2 \redx \elab{\be}{\cid}}
      {\shtrip{\Gamma, R, \xdefs}{e_1 := e_2}{\Gamma, R, \xdefs[x \mapsto \toeq{\elab{\be}{\cid}}]}}

  \inferrule[Seq]    
      {\shtrip{\Gamma_1,R_1,\xdefs_1}{\cmd_1}{\Gamma_2,R_2,\xdefs_2} \\
       \shtrip{\Gamma_2,R_2,\xdefs_2}{\cmd_2}{\Gamma_3,R_3,\xdefs_3}}
      {\shtrip{\Gamma_1,R_1,\xdefs_1}{\cmd_1;\cmd_2}{\Gamma_3,R_3,\xdefs_3}}

  \inferrule[Let]
      {\shtrip{\Gamma_1,R_1,\xdefs_1}{\cmd[e/y]}{\Gamma_2,R_2,\xdefs_2}}
      {\shtrip{\Gamma_1,R_1,\xdefs_1}{\elet{y}{e}{\cmd}}{\Gamma_2,R_2,\xdefs_2}}

  \inferrule[App]
      {f : \Pi y_1,\ldots,y_n. \notg{\Gamma_1} \rightarrow (\notg{\Gamma_2}, \notg{R'}) \\
        \notg{\Gamma_1}[e_1/y_1 \cdots e_n/y_n] \redx \Gamma_1 \\
        \notg{\Gamma_2}[e_1/y_1 \cdots e_n/y_n] \redx \Gamma_2 \\
        \notg{R'}[e_1/y_1 \cdots e_n/y_n] \redx R'        
      }
      {\shtrip{\Gamma \cup \Gamma_1, R, \xdefs}{f(e_1,\ldots,e_n)}{\Gamma \cup \Gamma_1 \cup \Gamma_2, R \uplus R', \xdefs}}
      
  \inferrule[Fn]
            {\codebase(f) = y_1 : \tau_1, \ldots, y_n : \tau_n, \cmd \\
              \mv_1,\ldots,\mv_n = \fresh(\tau_1,\ldots,\tau_n) \\
              \notg{\Gamma_1}[\mv_1/y_1 \cdots \mv_n/y_n] \redx \Gamma_1 \\
              \notg{\Gamma_2}[\mv_1/y_1 \cdots \mv_n/y_n] \redx \Gamma_2 \\
              \notg{R}[\mv_1/y_1 \cdots \mv_n/y_n] \redx R \\  
     \shtrip{\Gamma_1, \varnothing, \varnothing}{\cmd[\mv_1/y_1 \cdots \mv_n/y_n]}{\Gamma_2, R,\xdefs}}
    {f : \Pi y_1,\ldots,y_n. \notg{\Gamma_1} \rightarrow (\notg{\Gamma_2}, \notg{R})}
\end{mathpar}
\end{fpfig}
}

\newcommand{\oldnotgfig}{
\begin{fpfig}[t]{Evaluation of expressions within types and constraints.}{fig-notg}
$$
\begin{array}{cc}
  \begin{array}{rcl}
    \notg{x} &::=& \elab{\flip{e}}{e} \mid \elab{\secret{e}}{e} \mid \elab{\mesg{e}}{e} \mid \rvl{e} \mid \out{e}\\
  \notg{\phi} &::=& \notg{x} \mid \notg{\phi} \fplus \notg{\phi} \mid \notg{\phi} \fminus \notg{\phi} \mid \notg{\phi} \ftimes \notg{\phi} \\
  \notg{\eqs} &::=& \notg{\phi} \eop \notg{\phi} \mid \notg{\eqs} \wedge \notg{\eqs} \\
  \notg{X} &\in& 2^{\notg{x}}
\end{array}& \qquad
\begin{array}{rcl}
  \notg{t} &::=& e \mid \cty{e}{\notg{\ty}} \\
  \notg{\ty} & \in & 2^{\notg{t}}\\
  \notg{\Gamma} &::=& \varnothing \mid \notg\Gamma; e : \notg{\ty}\\
  \notg\Delta &::=& \varnothing \mid \notg\Delta; e : \ity{e}{\notg{V}}
\end{array}
\end{array}
$$

\begin{mathpar}
  \inferrule
      {\notg{\phi_1} \redx \phi_1 \\ \notg{\phi_2} \redx \phi_2}     
      {\notg{\phi_1} \ftimes \notg{\phi_2} \redx \phi_1 \ftimes \phi_2}

  \inferrule
      {\notg{\phi_1} \redx \phi_1 \\ \notg{\phi_2} \redx \phi_2}
      {\notg{\phi_1} \eop \notg{\phi_2} \redx \phi_1 \eop \phi_2}

  \inferrule
      {\notg{\eqs_1} \redx \eqs_1 \\ \notg{\eqs_2} \redx \eqs_2 }
      {\notg{\eqs_1} \wedge \notg{\eqs_2} \redx \eqs_1 \wedge \eqs_2}
\end{mathpar}

\begin{mathpar}
  \inferrule
      {e \redx x \\ \notg{\ty} \redx \ty}
      {\cty{e}{\notg{\ty}} \redx \cty{x}{\ty}}
      
  \inferrule
      {\notg{t_1} \redx t_1 \\ \cdots \\ \notg{t_n} \redx t_n}
      {\setit{\notg{t_1},\ldots,\notg{t_n}} \redx \setit{ t_1,\ldots,t_n }}

  \inferrule
      {\notg{\Gamma} \redx \Gamma \\ e \redx x \\ \notg{\ty} \redx \ty }
      {\notg{\Gamma}; e : \notg{\ty} \redx \Gamma; x : \ty }

  \inferrule
      {\notg{\Delta} \redx \Delta \\ e_1 \redx x  \\ e_2 \redx \cid \\ \notg{V} \redx V}
      {\notg{\Delta}; e_1 : \ity{e_2}{\notg{V}} \redx \Delta; x : \ity{\cid}{V} }

  \inferrule
      {\notg{\eqs_1} \redx \eqs_1 \\ \notg{\Gamma} \redx \\ \notg{R} \redx R
        \\ \notg{\Delta} \redx \Delta \\ \notg{\eqs_2} \redx \eqs_2}
      {\hty{\notg{\eqs_1}}{\notg{\Gamma}}{\notg{R}}{\notg{\Delta}}{\notg{\eqs_2}} \redx
        \hty{\eqs_1}{\Gamma}{R}{\Delta}{\eqs_2}}
\end{mathpar}
    
\end{fpfig}
}

\newcommand{\mtjfig}{
\begin{fpfig}[h]{$\metaprot$ type derivation rules for instructions.}{fig-mtj}
\begin{mathpar}
  \inferrule[Mesg]
            {\assign{e_1}{e_2} \redx \xassign{x}{\be}{\cid}  \\ \atj{\toeq{\elab{\be}{\cid}}}{R}{\ty} \\
              \itj{\cid}{\be}{V}}
            {\mtj{\assign{e_1}{e_2}}{\eqs}{(x:\ty)}{R}{(x : \ity{\cid}{V})}{\eqs \wedge x \eop \toeq{\elab{\be}{\cid}}}}

  \inferrule[Encode]
            {\mx{e_1}{e_2} \redx x \\ \notg{\phi} \redx \phi \\
              \eqs \models x \eop \phi\\
              \atj{\phi}{R}{\ty}}
            {\mtj{\eqcast{\mx{e_1}{e_2}}{\notg{\phi}}}{\eqs}{(x : \ty)}{R}{\varnothing}{\eqs}}

  \inferrule[Assert]
            {\elab{\assert{e_1 = e_2}}{e_3} \redx \elab{\assert{\be_1 = \be_2}}{\cid} \\
             \ipj{\eqs}{\elab{\assert{\be_1 = \be_2}}{\cid}}{\Delta}}
            {\mtj{\elab{\assert{e_1 = e_2}}{e_3}}{\eqs}{\varnothing}{\varnothing}{\Delta}{\eqs}}
            
  \inferrule[App]
            {f : \dht{y_1,\ldots,y_n}{\notg{\eqs_1}}{\notg{\Gamma}}{\notg{R}}{\notg{\Delta}}{\notg{\eqs_2}} \\
              e_1 \redx \mv_1\ \cdots\ e_n \redx \mv_n \\
              (\hty{\notg{\eqs_1}}{\notg{\Gamma}}{\notg{R}}{\notg{\Delta}}{\notg{\eqs_2}})[\mv_1/y_1]\cdots[\mv_n/y_n] \redx
                    \hty{\eqs_1}{\Gamma}{R}{\Delta}{\eqs_2} \\
              \eqs \models \eqs_1}
            {\mtj{f(e_1,\ldots,e_n)}{\eqs}{\Gamma}{R}{\Delta}{\eqs \wedge \eqs_2}}

  \inferrule[Seq]          
            {\mtj{\prog_1}{\eqs_1}{\Gamma_1}{R_1}{\Delta_1}{\eqs_2} \\
             \mtj{\prog_2}{\eqs_2}{\Gamma_2}{R_2}{\Delta_2}{\eqs_3}}
            {\mtj{\prog_1;\prog_2}{\eqs_1}{\Gamma_1;\Gamma_2}{R_1;R_2}{\Delta_1;\Delta_2}{\eqs_3}}
\end{mathpar}
\end{fpfig}
}

\newcommand{\mtjfnfig}{
\begin{fpfig}[h]{$\metaprot$ type derivation rules for function definitions.}{fig-mtjfn}
\begin{mathpar}
   \inferrule[Fn]
            {\codebase(f) = y_1,\ldots,y_n, \instr \\
              \mtj{\instr[\mv_1/y_1]\cdots[\mv_n/y_n]}{\eqs_1}{\Gamma}{R}{\Delta}{\eqs_2}\\
              \fresh(\mv_1,\ldots,\mv_n) \\
              %\subn = [\mv_1/y_1]\cdots[\mv_n/y_n] \\
              (\hty{\notg{\eqs_1}}{\notg{\Gamma}}{\notg{R}}{\notg{\Delta}}{\notg{\eqs_2}})[\mv_1/y_1]\cdots[\mv_n/y_n]  \redx
                    \hty{\eqs_1}{\Gamma}{R}{\Delta}{\eqs_2} }
            {f : \dht{y_1,\ldots,y_n}{\notg{\eqs_1}}{\notg{\Gamma}}{\notg{R}}{\notg{\Delta}}{\notg{\eqs_2}}}

  \inferrule[FnPre]
            {f : \dht{y_1,\ldots,y_n}{\notg{\eqs}}{\notg{\Gamma}}{\notg{R}}{\notg{\Delta}}{\notg{\eqs_2}} \\
              \precond(f) = \notg{\eqs_1} \\
              \fresh(\mv_1,\ldots,\mv_n) \\
              \notg{\eqs}[\mv_1/y_1]\cdots[\mv_n/y_n]  \redx \eqs \\
              \notg{\eqs_1}[\mv_1/y_1]\cdots[\mv_n/y_n]  \redx \eqs_1 \\
              \eqs_1 \models \eqs             
            }
            {f : \dht{y_1,\ldots,y_n}{\notg{\eqs_1}}{\notg{\Gamma}}{\notg{R}}{\notg{\Delta}}{\notg{\eqs_2}}}

  \inferrule[FnPost]
            {f : \dht{y_1,\ldots,y_n}{\notg{\eqs_1}}{\notg{\Gamma}}{\notg{R}}{\notg{\Delta}}{\notg{\eqs}} \\
              \postcond(f) = \notg{\eqs_2} \\
              \fresh(\mv_1,\ldots,\mv_n) \\
              \notg{\eqs}[\mv_1/y_1]\cdots[\mv_n/y_n]  \redx \eqs \\
              \notg{\eqs_2}[\mv_1/y_1]\cdots[\mv_n/y_n]  \redx \eqs_2 \\
              \eqs \models \eqs_2              
            }
            {f : \dht{y_1,\ldots,y_n}{\notg{\eqs_1}}{\notg{\Gamma}}{\notg{R}}{\notg{\Delta}}{\notg{\eqs_2}}}
\end{mathpar}
\end{fpfig}
}

\newcommand{\preludetestfig}{
\begin{fpfig}[b]{Verification time for $\metaprot$ binary and arithmetic circuit libraries and small circuits, in seconds.}{fig-preludetest}
\begin{center}
\begin{tabular}{r|c|c|c|c}
   \textit{field size:} & 2 & $2^{17} - 1$ &  $2^{31} - 1$ &  $2^{61} - 1$ \\
  \hline
  \text{GMW} & .377 & N/A & N/A & N/A \\
  \text{Passive Beaver} & .476 & .48 & .466 & .644 \\
  \text{Malicious Beaver (BDOZ)} & .464 & .528 & .641 & .653  
\end{tabular}
\end{center}
\end{fpfig}
}


\acmConference[PPDP]{Principles and Practice of Declarative Programming}{2024}{Milan}

\begin{document}

\title{Language-Based Security for Low-Level MPC}

\author{Christian Skalka}
\affiliation{
  \institution{University of Vermont}
  \city{}
  \country{}
%  \city{Burlington}
%  \country{USA}
}
\email{ceskalka@uvm.edu}

\author{Joseph Near}
\affiliation{
  \institution{University of Vermont}
  \city{}
  \country{}
%  \city{Burlington}
%  \country{USA}
}
\email{jnear@uvm.edu}

\begin{abstract}
  Secure Multi-Party Computation (MPC) is an important
  enabling technology for data privacy in modern distributed
  applications. Currently, proof methods for low-level MPC protocols
  are primarily manual and thus tedious and error-prone, and are also
  non-standardized and unfamiliar to most PL theorists. As a step
  towards better language support and language-based enforcement, we
  develop a new staged PL for defining a variety of low-level
  probabilistic MPC protocols. We also formulate a collection of
  confidentiality and integrity hyperproperties for our language model
  that are familiar from information flow, including conditional
  noninterference, delimited release, and robust declassification. We
  demonstrate their relation to standard MPC threat models of passive
  and malicious security, and how they can be leveraged in security
  verification of protocols. To prove these properties we develop
  automated tactics in $\mathbb{F}_2$ that can be integrated with
  separation-logic style reasoning.
\end{abstract}

%%
%% The code below is generated by the tool at http://dl.acm.org/ccs.cfm.
%% Please copy and paste the code instead of the example below.
%%
\begin{CCSXML}
<ccs2012>
   <concept>
       <concept_id>10002978.10002986.10002990</concept_id>
       <concept_desc>Security and privacy~Logic and verification</concept_desc>
       <concept_significance>500</concept_significance>
       </concept>
   <concept>
       <concept_id>10003752.10003753.10003757</concept_id>
       <concept_desc>Theory of computation~Probabilistic computation</concept_desc>
       <concept_significance>300</concept_significance>
       </concept>
   <concept>
       <concept_id>10003752.10003790.10003806</concept_id>
       <concept_desc>Theory of computation~Programming logic</concept_desc>
       <concept_significance>500</concept_significance>
       </concept>
 </ccs2012>
\end{CCSXML}

\ccsdesc[500]{Security and privacy~Logic and verification}
\ccsdesc[500]{Theory of computation~Probabilistic computation}
\ccsdesc[500]{Theory of computation~Programming logic}


%%
%% Keywords. The author(s) should pick words that accurately describe
%% the work being presented. Separate the keywords with commas.
\keywords{Secure multiparty computation, security verification, probabilistic programming, programming languages, information flow.}

\maketitle

\section{Introduction}

Secure Multi-Party Computation (MPC) protocols support data privacy in
important modern, distributed applications such as privacy-preserving
machine learning and Zero-Knowledge proofs in blockchains \cite{XXX}
\cnote{Need some banger cites here, maybe more MPC hype.}. MPC methods
have been developed by the cryptography community for many years,
while receiving the attention of the programming languages community
only relatively recently. In contrast, information flow security has
received significant attention in PL theory and practice especially
since the turn of the century \cite{1159651}, including a menagerie of
variants, enforcement mechanisms, and programming frameworks. Much of
this has been enabled by the unified metatheory of
\emph{hyperproperties} \cite{10.5555/1891823.1891830} that establishes
a common conceptual framework for reasoning about and implementing
systems with information flow security.

Our goal is to explore and establish connections between the security
model of MPC-- the \emph{real/ideal} aka \emph{simulator security}
model -- and trace-based hyperproperties, and to leverage these
connections to obtain automated enforcement mechanisms for MPC
protocol development. Currently, proof methods for MPC protocol
development are well-studied \cite{Lindell2017} but manual and 
thus tedious and error-prone, and are also non-standardized and
unfamiliar to most PL theorists. Therefore our exploration will
make both theoretical and practical contributions by bridging a gap between
information flow and simulator security methodologies.

\subsection{The Analysis Challenge of MPC} MPC protocols involve communication
between a group of distributed participants called a \emph{federation}
that collaboratively compute and publish the result of some known
\emph{ideal functionality} $\idealf$, while keeping each party's input
``secret'', without the use of a trusted third party. This last part
is critical. For example, if we take $\idealf$ to be the majority vote
function, a protocol for computing $\idealf$ is MPC-secure if, given
any set of input votes, it correctly computes and publishes the voting
result but reveals no other information to the public or to other
participants. However, by publishing the result, some information
about individual votes may be implicitly declassified.  For example,
in the case of a majority vote in a federation of size 3, if the
motion carries and party 1 has voted no, then party 1 knows exactly
the votes of parties 2 and 3. This cannot be avoided due to the nature
of $\idealf$.

Security in the MPC setting thus means that protocols cannot reveal
any secret information other than what is impicitly declassified by
the publicized output of the ideal functionality. The security model
also assumes that some subset of participants can be corrupted and
collude adversarially to possibly infer more secret information. The
accepted method of demonstrating protocol security in this setting is
to define a \emph{simulation} algorithm that runs in the ``ideal''
world which, given just the inputs of corrupted parties and the output
of the ideal functionality, is able to reconstruct information that
corrupted parties receive in their so-called \emph{views} of the
protocol running in the real world.  This implies that adversarial
views provide no information beyond what is provided by the ideal
output alone. Simulation is defined probabilistically since MPC
protocols typically rely on cryptographic and probabilistic methods.

In Section \ref{section-hyperprop-passive} we formalize real/ideal
security, and we define and discuss simple examples of MPC protocols
in Section \ref{section-minicat-examples} and more complex ones in
Section \ref{section-composition}. But an immediate and main point in
relation to security hyperproperties is that, due to the potential for
information release in MPC, simulator security is \emph{not} a strict
probabilistic noninterference or trace obliviousness property, as we
show in Section \ref{section-hyperprop-ni}- rather, the public output
allows and sets an upper bound on declassification.

\subsection{Related Work}

We distinguish between \emph{extensional} vs.~\emph{intensional}
properties and analysis of MPC protocols. By extensional, we mean
simulator security itself as well as analysis concerned with the use
and interaction of complete secure protocols. By intensional, we mean
analysis and inner workings of protocols themselves. While the
probabilistic formulations we develop could conceivably used
extensionally, our main focus is on intensional properties of
protocols, both through whole-program analysis and compositional
properties of program components that support security.

Previous work on analysis for the SecreC language
\cite{almeida2018enforcing,10.1145/2637113.2637119} is concerned with
extensional properties of MPC, in particular the specification and
enforcement of declassification bounds in programs that use MPC in
subprograms. This work is explicitly reminiscent of information flow
approaches such as delimited information release
\cite{10.1007/978-3-540-37621-7_9} and downgrading policies and
relaxed noninterference \cite{10.1145/1040305.1040319}. However, their
program logic assumes correctness of the underlying MPC protocols.
The Wys$^\star$ language \cite{wysstar}, based on Wysteria
\cite{rastogi2014wysteria}, has similar goals and includes a
trace-based semantics for reasoning about the extensional interactions
of MPC protocols. Their compiler also enforces some intensional
properties, in particular that underlying multi-threaded protocols
enforce the single-threaded source language semantics.



\subsection{Contributions}

In this paper, we make the following contributions.
\begin{itemize}
\item A new probabilistic programming language $\minifed$ for defining
  synchronous distributed protocols over the binary field. We consider example
  protocols including oblivious transfer, communications encryption with one-time
  pads, and additive secret sharing (Section \ref{section-minicat}).
  \cnote{I'm hoping $\minifed$ hasn't been used already.}
\item A formulation of $\minifed$ program distributions, including
  distributions of secrets and views, supporting expression of security
  properties (Section \ref{section-pmf}). This intensional characterization
  is useful for expressing internal characteristics of protocols, and
  is needed for MPC due to the fact that obliviousness alone is not sufficient
  to capture characteristics of the real/ideal model, in particular, the
  allowance of some information leakage through public outputs. 
\item A novel hyperproperty of program execution traces, called
  \emph{noninterference modulo output ($\NIMO$)}, based on our
  formulation of program distributions, that implies passive security
  (Section \ref{section-hyperprop}).
\item A brute force mechanism for verifying passive security in
  $\minifed$ protocols, that is also amenable to HPC optimizations via
  translation of $\minifed$ protocols into Datalog (Section \ref{section-bruteforce}). 
\item A new metaprogramming language $\metaprot$ that dynamically
  generates $\minifed$ protocols. It includes control and data structures
  and is able to express logical protocol components, and enjoys a
  type safety result that guarantees that generated protocols
  are semantically well-defined (Section \ref{section-metalang}).
\item A formulation of Yao's Garbled Circuits, and a formulation of a
  compositional property of an extensible library of garbled gate
  components that is verifiable automatically. This property
  guarantees that any well-formed circuits using the library are
  passive secure (Section \ref{section-composition}).
\end{itemize}



\section{The $\minicat$ Protocol Language}
\label{section-lang}

The $\minifed$ language establishes a basic model of synchronous
protocols between a federation of \emph{clients} exchanging values in
the binary field. A model of synchronous communication captures a wide
range of MPC protocols. Concurrency is out of scope in this work but
an avenue for future work. The lack of sophisticated control
structures in $\minifed$ is intentional, since minimizing features
eases analysis and control abstractions such as function definitions
can be integrated into a metalanguage that generates $\minifed$
programs (Section \ref{section-metalang}).

We identify clients by natural numbers and federations- finite sets of
clients- are always given statically.  Our threat model assumes a
partition of the federation into \emph{honest} $H$ and \emph{corrupt}
$C$ subsets. We model probabilistic programming via a \emph{random
tape} semantics. That is, we will assume that programs can make
reference to values chosen from a uniform random distributions defined
in the initial program memory.  Programs aka protocols execute
deterministically given the random tape.

\subsection{Syntax}

\minifedfig

The syntax of $\minifed$, defined in Figure \ref{fig-minifed},
includes values $v$ and standard operations of addition, subtraction,
and multiplication in a finite field $\mathbb{F}_p$ where $p$ is some
prime.  Protocols are given input secret values $\secret{w}$ as well
as random samples $\flip{w}$ on the input tape, implemented using a
\emph{memory} as described below (Section
\ref{section-lang-semantics}) where $w$ is a distinguishing 
identifier string. Protocols are sequences of assignment commands of three
different forms:
\begin{itemize}
\item $\eassign{\mesg{w}}{\cid_2}{\be}{\cid_1}$: This
  is a \emph{message send} where expression $\be$ is computed
  by client $\cid_1$ and sent to client $\cid_2$ as message
  $\mesg{w}$.
\item $\reveal{w}{\be}{\cid}$: This
  is a \emph{public reveal} where expression $\be$ is computed
  by client $\cid$ and broadcast to the federation, typically
  to communicate intermediate results for use in final output
  computations.
\item $\pubout{\cid}{\be}{\cid}$: This
  is an \emph{output} where expression $\be$ is computed
  by client $\cid$ and reported as its output. As a
  sanity condition we disallow commands
  $\pubout{\cid_1}{\be}{\cid_2}$ where $\cid_1\ne\cid_2$.
\end{itemize}
For example, in the following protocol, a client 1
subtracts a random sample $\flip{y}$ from $\mathbb{F}_p$ from their
secret value $\secret{x}$ and sends the result to client
2 as a message $\mesg{z}$:
$$
\eassign{\mesg{z}}{2}{(\secret{x} - \flip{y})}{1}
$$
Both messages $\mesg{w}$ and reveals $\rvl{w}$ can be
referenced in expressions once they've been defined.
This distinction between messages and broadcast public
reveal is consistent with previous formulations, e.g.,
\cite{6266151}.

We let $x$ range over \emph{variables} which are identifiers where
client ownership is specified- e.g.,
$\elab{\mesg{\mathit{foo}}}{\cid}$ is a message $\mathit{foo}$ that
was sent to $\cid$. We let $X$ range over sets of variables, and more
specifically, $S$ ranges over sets of secret variables
$\elab{\secret{w}}{\cid}$, $R$ ranges over sets of random variables
$\elab{\flip{w}}{\cid}$, $M$ ranges over sets of message variables
$\elab{\mesg{w}}{\cid}$, $P$ ranges over sets of public variables
$\rvl{w}$, and $O$ ranges over sets of output variables $\out{\cid}$.
Given a program $\prog$, we write $\iov(\prog)$ to denote the set $S
\cup M \cup P \cup O$ of variables in $\prog$ with ownership made
explicit and $\secrets(\prog)$ to denote $S$, and we write
$\flips(\prog)$ to denote the set $R$ of random samplings in $\prog$
with ownership made explicit. We write $\vars(\prog)$ to denote
$\iov(\prog) \cup \flips(\prog)$. For any set of variables $X$ and
clients $I$, we write $X_I$ to denote the subset of $X$ owned by any
client $\cid \in I$, in particular we write $X_H$ and $X_C$ to denote the
subsets belonging to honest and corrupt parties, respectively.

\subsection{Semantics}
\label{section-lang-semantics}

\emph{Memories} are fundamental to the semantics of $\fedcat$ and
provide random tape and secret inputs to protocols, and also record
message sends, public broadcast, and client outputs. Memories $\store$ are finite
(partial) mapping from variables $x$ to values $v \in \mathbb{Z}_p$. The \emph{domain} of a
memory is written $\dom(\store)$ and is the finite set of variables on
which the memory is defined. We write $\store\{ x \mapsto v\}$ for
$x\not\in\dom(\store)$ to denote the memory $\store'$ such that
$\store'(x) = v$ and otherwise $\store'(y) = \store(y)$ for all $y
\in \dom(\store)$. We write $\store \subseteq \store'$ iff
$\dom(\store) \subseteq \dom(\store')$ and $\store(x) =
\store'(x)$ for all $x \in \dom(\store)$. We write $\store \cap
\store'$ to denote the combination of $\store$ and $\store'$
assuming $\store(x) = \store'(x)$ for all $x \in \dom(\store)
\cap \dom(\store')$, otherwise $\store \cap \store'$ is undefined.
We write $\store \subseteq \store'$ iff $\store \cap \store'
= \store$.

Given a set of variables $X$, we write $\store_X$ to denote the
memory $\store$ restricted to the domain $X$, and we define
$\mems(X)$ as the set of all memories with domain $X$:
$$
\mems(X) \defeq \{ \store \mid \dom(\store) = X \}
$$
Thus, given a protocol $\prog$, the set of all random tapes for
$\prog$ is $\mems(\flips(\prog))$.
%We let $\stores$ range
%over sets of memories with the same domain, and abusing notation
%we write $\dom(\stores)$ to denote the common domain,
%and $\stores_X \defeq \{ \store_X | \store \in \stores \}$.

Given a variable-free expression $\be$, we write $\cod{\be}$ to denote
the standard interpretation of $\be$ in the arithmetic field
$\mathbb{Z}_{p}$. With the introduction of variables to expressions,
we need to interpret variables with respect to a specific memory, and
all variables used in an expression must belong to a specified client.
Thus, we denote interpretation of expressions $\be$ computed on a
client $\cid$ as $\lcod{\store,\be}{\cid}$. This interpretation is
defined in Figure \ref{fig-minifed}. The small-step reduction relation
$\redx$ is then defined in Figure \ref{fig-minifed} to evaluate
commands. Reduction is a relation on \emph{configurations} $(\store,
\prog)$ where all three command forms- message send, broadcast, and
output- are implemented as updates to the memory $\store$. We write
$\redxs$ to denote the reflexive, transitive closure of\ $\redx$.

\subsection{Example: Passive Secure Addition}
\label{section-lang-example}

Shamir addition leverages homomorphic properties of addition in
arithmetic fields to implement secret addition. If a field value $v_1$
is uniformly random, then $v_1 \fminus v_2$ is an encryption of $v_2$
where $v_1$ is an information theoretically secure one-time-pad, which
is exploited for secret sharing, noting that $v_2$ can be
reconstructed by summing $v_1$ and $v_3 \defeq v_1 \fminus v_2$. 

In $\minifed$, to privately sum secret values $\secret{\cid}$, each
client $\cid$ in the federation $\{ 1, 2, 3 \}$ samples a value
$\locflip$ that can be used as a one-time pad with another random
sample $\flip{x}$ and $\secret{\cid}$. This yields two secret shares
communicated as messages to the other clients, while each client keeps
$\locflip$ as its own share.
$$
\begin{array}{lll}
  \elab{\mesg{s1}}{2} &:=& \elab{(\secret{1} \fminus \locflip \fminus \flip{x})}{1} \\ 
  \elab{\mesg{s1}}{3} &:=& \elab{\flip{x}}{1} \\ 
  \elab{\mesg{s2}}{1} &:=& \elab{(\secret{2} \fminus \locflip \fminus \flip{x})}{2} \\ 
  \elab{\mesg{s2}}{3} &:=& \elab{\flip{x}}{2} \\ 
  \elab{\mesg{s3}}{1} &:=& \elab{(\secret{3} \fminus \locflip \fminus \flip{x})}{3} \\ 
  \elab{\mesg{s3}}{2} &:=& \elab{\flip{x}}{3}
\end{array}
$$
This scheme guarantees that messages
are viewed as random noise by any observer 
besides $\cid$ \cite{barthe2019probabilistic}. Next, each client
publicly reveals the sum of all of its shares, including its local
share. This step does reveal information about secrets-- note in
particular that $\locflip$ is reused and is no longer a one-time-pad:
$$
\begin{array}{lll}
  \rvl{1} &:=& \elab{(\locflip \fplus \mesg{s2} \fplus \mesg{s3})}{1} \\ 
  \rvl{2} &:=& \elab{(\mesg{s1} \fplus \locflip \fplus \mesg{s3})}{2} \\
  \rvl{3} &:=& \elab{(\mesg{s1} \fplus \mesg{s2} \fplus \locflip)}{3} 
\end{array}
$$
Finally, each client outputs the sum of each sum of shares, yielding
the sum of secrets. The protocol is correct because the outputs are all the
true sum of secrets, and it is secure because no more information about the
secrets other than that revealed by their sum is exposed.
$$
%\elab{\mesg{o1}}{2} &:=& \elab{(\locflip \fplus \mesg{s2} \fplus \mesg{s3})}{1} \\ 
  %\elab{\mesg{o1}}{3} &:=& \elab{(\locflip \fplus \mesg{s2} \fplus \mesg{s3})}{1} \\ 
  %\elab{\mesg{o2}}{1} &:=& \elab{(\mesg{s1} \fplus \locflip \fplus \mesg{s3})}{2} \\
  %\elab{\mesg{o2}}{3} &:=& \elab{(\mesg{s1} \fplus \locflip \fplus \mesg{s3})}{2} \\ 
  %\elab{\mesg{o3}}{1} &:=& \elab{(\mesg{s1} \fplus \mesg{s2} \fplus \locflip)}{3} \\ 
  %\elab{\mesg{o3}}{2} &:=& \elab{(\mesg{s1} \fplus \mesg{s2} \fplus \locflip)}{3}\\ 
  %\pubout{1} &:=& \elab{(\locflip \fplus \mesg{s2} \fplus \mesg{s3} + \mesg{o2} + \mesg{o3})}{1}
\begin{array}{lll}
  \out{1} &:=& \elab{(\rvl{1} \fplus \rvl{2} + \rvl{3})}{1}\\
  \out{2} &:=& \elab{(\rvl{1} \fplus \rvl{2} + \rvl{3})}{2}\\
  \out{3} &:=& \elab{(\rvl{1} \fplus \rvl{2} + \rvl{3})}{3}
\end{array}
$$


\section{Security Model}

\begin{definition}
  We write $\vc{\pmf}{x}{y}$ iff $\pmf(\{ x \mapsto 0\}\ |\ \{ y \mapsto 0 \}) =
  \pmf(\{ x \mapsto 1\}\ |\ \{ y \mapsto 1 \}) = 1$.
  We write $\sep{\pmf}{X}{Y}$ iff for all
    $\store \in \mems(X \cup Y)$ we have
  $\margd{\pmf}{X \cup Y}(\store) =
  \pmf(\store_X) * \pmf(\store_Y)$
\end{definition}

\subsection{Passive Security}

The simulator is represented by a probabilistic algorithm $\SIM_C$,
aka a \emph{simulation}, that is parameterized by corrupt inputs and
the output of an ideal functionality, and that returns a set of
adversarial views (as a memory) with some probability. Given
corrupt inputs $\store$ and ideal functionality output $v$,  
we write
$
\prob(\SIM(\store,v) = \store')
$
to denote the probability that $\SIM(\store,v)$
returns corrupt views $\store'$ as a result. We can then define the
probability distribution of corrupt views reconstructed
by the simulator as follows:
\begin{definition}
  Given $C$, $\store$, and $v$, we write $\dist(\SIM(\store,v))$ to
  denote the distribution of corrupt views reconstructed by the
  simulation, where for
  all $\store' \in \mems(V)$:
  $$
  \dist(\SIM(\store,v))(\store')\ \defeq\ \prob(\SIM(\store,v) = \store') 
  $$
\end{definition}

Then we can define passive security in the real/ideal
model as follows. 
\begin{definition}[Passive Security]
  Assume given a program $\prog$ that correctly implements an ideal
  functionality $\idealf$, with $\iov(\prog) = (S,M,O)$.  Then $\prog$
  is \emph{passive secure in the simulator model} iff for all
  partitions of the federation into honest and corrupt sets $H$ and $C$
  with $|C| < |H|$ and for all $\store \in \mems(S)$ there exists a
  simulation $\SIM$ such that:
  $$
  \dist(\SIM(\store_{S_C},\idealf(\store))) = \condd{\progtt(\prog)}{M_C}{\store}
  $$
\end{definition}

\subsection{Malicious Security}

$$
\begin{array}{rclr}
  (\store, \eassign{\mesg{w}}{\cid_1}{\be}{\cid_2};\prog) &\aredx&
  (\extend{\store}{\mesg{w}_{\cid_1}}{\lcod{\store,\be}{\cid_2}}, \prog) & \cid_2 \in H\\
  (\store, \eassign{\mesg{w}}{\cid_1}{\be}{\cid_2};\prog) &\aredx&
  (\extend{\store}{\mesg{w}_{\cid_1}}{\lcod{\arewrite(\store_C,\be)}{\cid_2}}, \prog) & \cid_2 \in C
\end{array}
$$

\begin{definition}[Corrupt Inputs, Honest Outputs]
  Given a program $\prog$ with $\iov(\prog) = (S,M,O)$ , define $\cinputs$ as the
  messages in $M$ sent from corrupt to honest parties:
  $$
  \cinputs = \{\ \elab{\mesg{w}}{\cid}\ \mid\  \elab{\mesg{w}}{\cid} \in M \wedge \eassign{\mesg{w}}{\cid}{\be}{\cid'} \in \prog
  \wedge \cid \in H \wedge \cid' \in C \ \} 
  $$
  and similarly define $\houtputs$ as the messages in $M$ sent from honest to corrupt parties.
  %Define also $(\afilter\ \prog)$ as $\prog$ with all instructions of the form $\eassign{\mesg{w}}{\cid}{\be}{\cid'}$ removed
  %where $\cid \in H \wedge \cid' \in C$.
\end{definition}

$$
\dist(\SIM(\store_{S_H})) = \condd{\progtt(\prog)}{\houtputs \cup O_H}{\store_{S_H}}
$$


\section{Security Hyperproperties}

In this Section we formulate probabilistic versions of well-studied
hyperproperties of confidentiality and integrity, including noninterference,
gradual release, declassification, and robust declassification.
We demonstrate a soundness relation between noninterference and
passive security, and between robust declassification and malicious
security. We subsequently leverage this relation to enforce
malicious security using ``traditional'' security type methods
in Section \ref{section-types}. Previous work has explored
a similar approach to security type enforcement
\cite{6266151,almeida2018enforcing} but mainly
for aspects of passive security.

\subsection{Passive Security and Noninterference}

Since MPC protocols release some information about secrets through
outputs of $\idealf$, they do not enjoy strict noninterference.  As
discussed in Section \ref{section-lang}, public reveals and protocol
outputs are fundamentally forms of declassification.  But consistent
with other work \cite{8429300}, we can formulate a version of
probabilistic noninterence conditioned on output that is sound
for passive security. It says that if two low-equivalent secret
inputs generate the same output, then the distributions of corrupt
views are the same. 
\begin{definition}[Noninterference modulo output]
  \label{definition-NIMO}
  We say that a program $\prog$ satisfies \emph{noninterference modulo output}
  iff for all $H$ and $C$ and 
  $\store_1,\store_2 \in \mems(S)$ we have:
  $$
  (\store_1 =_C \store_2 \ \wedge \ 
  (\condd{\progtt(\prog)}{O}{\store_1} = \condd{\progtt(\prog)}{O}{\store_2}))
  \implies 
  (\condd{\progtt(\prog)}{\houtputs}{\store_1} = \condd{\progtt(\prog)}{\houtputs}{\store_2})
  $$
  where $\iov(\prog) = S \cup V \cup O$.
\end{definition}
Intuitively, this conditional noninterference property implies that
the simulator can just run the protocol in simulation to
reconstruct real world corrupt view distributions. But it requires
that the simulator can tractably ``pre-image'' a given output of
a functionality $\idealf$, to determine the inputs that
could have produced it. This pre-image is called a
\emph{kernel} in recent work \cite{XXX}.
\begin{definition}
  Given a functionality $\idealf$ and output value $v$, its
  \emph{kernel}, denoted $\ik(\idealf,v)$ is
  $
  \{ \store\ |\ \idealf(\store) = v \}
  $.
  We say that $\idealf$ is \emph{pre-imageable} iff $\ik(\idealf, v)$ for all
  $v$ can be computed tractably.
\end{definition}
A soundness result for passive security can then be stated as follows.
We prove this in a separate manuscript \cite{XXX}, and it is also
essentially the same as ``perfect passive NI security'' which
has a similar soundness property \cite{8429300}.  
\begin{restatable}{theorem}{nimosecure}
  \label{theorem-nimo}
  Assume given pre-imageable $\idealf$ and a protocol $\prog$ that
  correctly implements $\idealf$.  If $\prog$ satisfies noninference modulo output
  then $\prog$ is passive secure.
\end{restatable}

\subsection{Gradual Release as a Design Pattern}

Previous work has discussed how MPC security is not noninterference,
but rather how ideal functional sets an upper bound on
declassification \cite{6266151,almeida2018enforcing}. Nevertheless,
probabilistic noninterference is preserved by components of
cryptographic protocols generally, and can be expressed using
\emph{probabilistic independence} \cite{darais2019language,barthe2019probabilistic}
We introduce important notation to express independence:
\begin{definition}
%  We write $\vc{\pmf}{x}{y}$ iff $\pmf(\{ x \mapsto 0\}\ |\ \{ y \mapsto 0 \}) =
%  \pmf(\{ x \mapsto 1\}\ |\ \{ y \mapsto 1 \}) = 1$.
  We write $\sep{\pmf}{X}{Y}$ iff for all
    $\store \in \mems(X \cup Y)$ we have
  $\margd{\pmf}{X \cup Y}(\store) =
  \pmf(\store_X) * \pmf(\store_Y)$
\end{definition}

In fact, MPC protocols typically satisfy a \emph{gradual release}
property\cite{XXX}, where messages exchanged remain probabilistically separable
from secrets, with only declassification events (reveals and outputs)
releasing information about honest secrets. 
\begin{definition}
  Given $H,C$, a protocol $\prog$ with $\iov(\prog) = S \cup M \cup P \cup O$
  satisfies \emph{gradual release} iff
  $\sep{\progtt(\prog)}{\houtputs}{S_H}$.
\end{definition}

\subsection{Malicious Security and Robust Declassification}

\begin{definition}
  We say that a protocol $\prog$ with $\iov(\prog) = (S,M)$ satisfies \emph{active confidentiality} iff the following conditions hold
  for all adversaries $\adversary$:
  \begin{enumerate}
  \item $\ \,\forall \store \in \mems(S_H) \ .\ \support(\progtt(\prog)(\{ \outv \}|\store)) =
    \support(\progtt(\prog,\adversary)(\{ \outv \}|\store))$
  \item $\begin{array}[t]{l}\forall \store_1, \store_2 \in \mems(S_H), \store \in \mems(\cinputs)\ . \\
    \quad
    \condd{\progtt(\prog,\adversary)}{\{ \outv \}}{\store_1 \cup \store} =
    \condd{\progtt(\prog,\adversary)}{\{ \outv \}}{\store_2 \cup \store} \\
    \qquad \implies\\
    \quad
    \condd{\progtt(\prog,\adversary)}{\houtputs}{\store_1 \cup \store} =
    \condd{\progtt(\prog,\adversary)}{\houtputs}{\store_2 \cup \store}\end{array}$
  \end{enumerate}
\end{definition}

\begin{theorem}
  If $\prog$ satisfies active confidentiality for all $H,C$ then it is malicious secure.
\end{theorem}

\begin{definition}
  A protocol $\prog$ with has \emph{integrity} iff 
  $\forall \adversary . \runs_\adversary(\prog) \subseteq \runs(\prog)$.
\end{definition}

\begin{definition}
  A protocol $\prog$ with $\iov(\prog) = (S,M,O)$ is \emph{malicious correct} iff:
  $$
  \forall \adversary, \store \in \mems(S_H) \ .\ \support(\progtt(\prog)(O_H|\store)) =
    \support(\progtt(\prog,\adversary)(O_H|\store))
  $$
\end{definition}

\begin{theorem}
  If a protocol has integrity it is malicious correct.
\end{theorem}

\begin{theorem}
  If a protocol is passive secure with integrity, then it satisfies active confidentiality.
\end{theorem}

\begin{theorem}
  If a protocol is passive secure with integrity, then it is malicious secure.
\end{theorem}

\begin{definition}[Robust Declassification]
  A protocol satisfies \emph{robust declassification} iff it has integrity and
  satisfies gradual release. 
\end{definition}

\begin{theorem}
  Passive security with robust declassification implies malicious security.
\end{theorem}


\section{Rewriting $\minicat$ to Stratified Datalog}
\label{section-bruteforce}

Here we show how to rewrite any $\prog$ to an equivalent Datalog
program, which supports application of recent work in linear algebraic
interpretation of Datalog and optimizations of model computation on
high performance computers \cite{sakama2017linear}. The method here also enumerates
$\runs(\prog)$ memory-by-memory, rather than in a ``batched'' manner
as in our first method, allowing parallelization of model
computation. The rewriting we describe here is to Datalog with
negation, with a negation-as-failure model, though we can also
use techniques in \cite{sakama2017linear} to eliminate negation from
resulting programs. \emph{Atoms} are $\minifed$ variables, \emph{literals}
are atoms or negated atoms, and clause bodies are conjunctions of literals.
A \emph{Datalog program} is a list of clauses.
$$
\begin{array}{rclr}
  \alpha &::=& \view{\cid}{w} \mid  \secret{\cid}{w} \mid \flip{\cid}{w} \mid \oracle{w} & \qquad \textit{(atoms)}\\
  \mathit{body} &::=&  \alpha \mid \neg \alpha \mid \alpha \wedge \mathit{body} \mid \neg\alpha \wedge \mathit{body} \mid \varnothing \\
  \mathit{clause} &::=& \alpha \gets \mathit{body}
\end{array}
$$

The first step in converting a protocol $\prot$ to a Datalog
program is to apply $\solve$ ``locally'' to each view definition
in $\prog$, obtaining constraints on memories that satisfy each
view in isolation.  
\begin{lemma} Let $\vars(\be)$ be the variables in $\be$, and define:
$$
{vtt}(\eassign{\view{\cid}{w}}{\be}) \defeq (\view{\cid}{w}, (\solve{(\mems(\vars\ \be))}{\be}))
$$
Then ${vtt}(\eassign{\view{\cid}{w}}{\be}) = (\view{\cid}{w},\stores)$ where $\stores
  = \{ \store \in \mems(\vars\ \be) \mid \lcod{\store,\be}{\cid} = 1 \}$ for some $\cid$.
\end{lemma}
Given this definition, the mapping of ${vtt}$ across a program
$\prog$- i.e., $(\mathit{map}\ {vtt}\ \prog)$- essentially defines the
logic program for ``view atoms'' modulo syntactic conversion. We can
accomplish the latter as follows, where $\datalog(\prog)$ defines the
full conversion.
\begin{definition} We define the conversion from memories to
  literals and clause bodies as follows:
\begin{mathpar}
  \logit{\alpha \mapsto 1} = \alpha

  \logit{\alpha \mapsto 0} = \neg \alpha

  \logit{\{ \alpha_1 \mapsto \beta_1, \ldots, \alpha_n \mapsto \beta_n\}} =
  \logit{\alpha_1 \mapsto \beta_1} \wedge \cdots \wedge \logit{\alpha_n \mapsto \beta_n}
\end{mathpar}
Given pairs $(\alpha,\stores)$ in the range of ${vtt}$, we define the conversion
to clauses as follows:
\begin{mathpar}
  \mathit{clauses}(\alpha,\{ \store_1,...,\store_n \}) = \alpha \gets \logit{\store_1} \vee \cdots \vee \alpha \gets \logit{\store_n}
\end{mathpar}
The $\minifed$-to-Datalog conversion is then defined as:
$$
\datalog(\prog) \defeq  \mathit{map}\ \mathit{clauses}\ (\mathit{map}\ {vtt}\ \prog)
$$
\end{definition}

In addition to converting view definitions to logic clauses, we also need to convert
secrets and random tapes. Since we assume given values for these in an arbitrary run of
the program, we can capture these as a ``fact base'' in our program, where
a fact is a clause of the form $\alpha \gets \varnothing$ and means that $\alpha$
is true in any model of the program. 
\begin{definition}
  Given $\store$, let $\{\alpha_1,\ldots,\alpha_n \} =
  \{ \alpha \in \dom(\store) \mid \store(\alpha) = 1 \}$.
  Then define:
  $$
  \mathit{facts}(\store) = \alpha_1 \gets \varnothing, \ldots, \alpha_n \gets \varnothing
  $$
\end{definition}

For any safe $\prog$ with $\iov(\prog) = S \cup V$ and $\flips(\prog) = F$, and
$\store \in \mems(S \cup F)$, it is the case
that $(\mathit{facts}(\store),\datalog(\prog))$ is a \emph{normal}, \emph{stratified}
(in fact, non-recursive) Datalog program, and so has a unique Least Herbrand Model
and is thus amenable to HPC optimization techniques \cite{aspis2018linear}. 
To compute $\runs(\prog)$, and thus $\progd(\prog)$, we compute
the Least Herbrand Model $\store$ of $(\mathit{facts}(\store'),\datalog(\prog))$
for all $\store' \in \mems(S \cup F)$, observing that model computation for
each $\store'$ can be done in parallel. The following establishes
correctness of this approach.
\begin{lemma}
  For all $\prog$ with $\iov(\prog) = S \cup V$ and $\flips(\prog) = F$,
  $\datalog(\prog)$ is a \emph{normal}, \emph{stratified}
  program \cite{aspis2018linear}, and $\store$ is the unique Least Herbrand
  Model of $(\mathit{facts}(\store_{S \cup F}),\datalog(\prog))$
  iff $\store \in \runs(\prog)$.
\end{lemma}
A full empirical exploration of the application of HPC optimizations
is beyond the scope of this paper but is a compelling topic for future
work. The reader is referred to \cite{nguyen2022enhancing} for
empirical results of HPC optimizations in other logic programming
contexts.


\section{A Protocol Metalanguage}
\label{section-metalang}

\begin{fpfig}[t]{Syntax of $\metaprot$.}{fig-metaprot-syntax}
$$
\begin{array}{rcl@{\hspace{8mm}}r}
\flab &\in& \mathrm{Field}\\
x &\in& \mathrm{EVar}\\
f &\in& \mathrm{FName}\\[2mm]
e &::=& b \mid \flip{e}{e} \mid \secret{e}{e} \mid \view{e}{e} \mid \oracle{e} \mid \enot\ e \mid e\ \eand\ e \mid e\ \exor\ e \mid & \textit{expressions}\\[0mm]
& & \select{e}{e}{e} \mid 
\send{\view{e}{e}}{e} \mid \send{\view{e}{e}}{\OT{e}{e}{e}} \mid e;e \mid \\[0mm]
& & x \mid \elet{x}{e}{e} \mid f(e,\ldots,e) \mid \{ \flab = e; \ldots; \flab = e \}
\mid e.\flab \mid e\concat e \mid (e) \\[2mm]
v &::=& w \mid \cid \mid \be \mid \{ \flab = v;\ldots;\flab = v \} 
\mid \ttt{()} & \textit{values}\\[2mm]
{fn} &::=& f(x,\ldots,x) \{ e \} & \textit{functions}
\end{array}
$$
\end{fpfig}

Large practical MPC computations are based on much larger protocols
than the examples we've considered so far. These larger protocols are
typically based on compositional units. An example of this is Yao's
Garbled Circuits (YGC), which are composed of so-called garbled gates.
Languages for defined garbled circuits, beginning with Fairplay \cite{269581},
treat gates as compositional units that are wired together by the programmer
to generate a complete circuit. The $\fedprot$ language is low-level
and does not include abstractions for defining composable elements. 

In this Section we introduce the $\metaprot$ language which includes
structured data and function definitions, which are sufficiently
expressive to define composable protocol elements such as garbled
gates. The $\metaprot$ language is a \emph{metalanguage}, in the sense
that it produces $\fedprot$ protocols as a result of computation. That
is, $\metaprot$ is a high-level language that generates low-level
protocol code.

\subsection{Syntax}

The syntax of $\metaprot$ is defined in Figure
\ref{fig-metaprot-syntax}.  It includes a syntax of function
definitions and records, and values include client ids, identifier
strings, and boolean expressions. Expression forms allow dynamic
construction of boolean expression forms and view assignments. When
$\metaprot$ programs construct a $\fedprot$ assignment, a side effect
occurs whereby the assignment is added to the end of the $\fedprot$
program accumulated during evaluation.

Formally, we consider a complete metaprogram to include both a
codebase and a ``main'' program that uses the codebase. 
\begin{definition}
A \emph{codebase} $\codebase$ is a list of function 
declarations. We write $ \codebase(f) = x_1,\ldots,x_n,\ e$
iff $f(x_1,\ldots,x_n) \{ e \} \in \codebase$.
A \emph{metaprogram}, aka \emph{mataprotocol} is a pair of a 
codebase and expression $\codebase, e$. We may omit
$\codebase$ if it is clear from context.  
\end{definition}

When we consider the example of YGC in detail below, our focus will be
on developing a codebase that can be used to define arbitrary
circuits, i.e., complete and concrete protocols. Since strings and
identifiers can be constructed manually, and expressions can occur
inside assignments and boolean expression forms, function definitions
can generalize over $\fedprot$-level patterns to obtain composable
program units. As a simple example, consider 3 party secret
sharing as illustrated in Example \ref{example-he}. We can
define a function $\ttt{share3}$ that abstracts the process
of splitting a given client's secret into 3 separate shares.
\begin{example} \label{example-share3} The function $\ttt{share3}$ 
  splits a client's secret into 3 shares returned as a record
  with fields $\ttt{s1-3}$:
  \begin{verbatimtab}
    share3(client, secretid)
    {
      let s1 = flip[client, share1] in
      let s2 = flip[client, share2] in
      let s3 = (s1 xor s2) xor s[client, s: || secretid] in
      {s1 = s1;s2 = s2;s3 = s3}
    } \end{verbatimtab}
  Here is $\metaprot$ program that uses this this function definition:
  \begin{verbatimtab}
    let shares = share3(1, mysecret) in
    v[2,s1] := shares.s2;
    v[3,s1] := shares.s3 \end{verbatimtab}
  which generates the following $\minifed$ program, as we formalize in Example \ref{example-share3-eval}
  below:
  \begin{verbatimtab}
    v[2,s1] := flip[1, share2];
    v[3,s1] := flip[1, share1] xor flip[1, share2] xor s[1, s:mysecret] \end{verbatimtab}
\end{example}

\subsection{Semantics}

\begin{fpfig}[t]{Evaluation contexts and operational semantics of $\metaprot$.}{fig-metaprot-semantics}
$$
\begin{array}{rcl@{\hspace{3mm}}r}
E &::=& [\,] \mid \enot\ E \mid E\ \bop\ e \mid v\ \bop\ E \mid  \flip{E}{e} \mid \secret{E}{e} \mid \view{E}{e} \mid \oracle{E} \mid  \\[1mm]
& & \flip{\cid}{E} \mid \secret{\cid}{E} \mid \view{\cid}{E} \mid \send{E}{e} \mid \send{\view{\cid}{w}}{E} \mid \OT{E}{e}{e} \\[1mm]
& & \mid \OT{v}{E}{e} \mid \OT{v}{v}{E} \mid \select{E}{e}{e} \mid \select{v}{E}{e} \mid \\[1mm]
& & \select{v}{v}{E} \mid \elet{x}{E}{e} \mid f(v,\ldots,v,E,e,\ldots,e) \mid \\[1mm]
& & \{ \flab = v;\ldots;\flab = v;\flab = E;\flab = e;\ldots;\flab = e \} \mid E.\flab \mid E\concat e \mid v \concat E
\end{array}
$$
\medskip
$$
\begin{array}{rcl@{\hspace{10mm}}r}
\config{\prog}{\elet{x}{v}{e}} &\redx& \config{\prog}{e[v/x]}\\
\config{\prog}{f(v_1,...,v_n)} &\redx&
\config{\prog}{e[v_1/x_1,\ldots,v_n/x_n]} & 
 \codebase(f) = x_1,\ldots,x_n,\ e\\
\config{\prog}{\{\ldots; \flab = v; \ldots\}.\flab} &\redx&
 \config{\prog}{v}\\
 \config{\prog}{w_1\concat w_2} &\redx& \config{\prog}{w_1w_2}\\
 \config{\prog}{v;e} &\redx& \config{\prog}{e}\\
\config{\prog}{\instr} &\redx& \config{\prog;\instr}{()}\\
\config{\prog}{E[e]} &\redx& \config{\prog'}{E[e']} & \text{if}\ \config{\prog}{e} \redx \config{\prog'}{e'} 
\end{array}
$$
\end{fpfig}

We define a small-step evaluation aka reduction relation $\redx$ in
Figure \ref{fig-metaprot-semantics}.  We write $\redxs$ to denote the
reflexive, transitive closure of $\redx$. Reduction is defined on
\emph{configurations} which are pairs of the form $\config{\prog}{e}$,
where $\prog$ is the $\minifed$ program accumulated during evaluation.
In this definition we write $e[v/x]$ to denote the substitution of $v$
for free occurences of $x$ in $e$. The rules are mostly standard,
except why a concrete $\minifed$ assignment is encountered it is added
to the end of $\prog$.

The rules rely on a definition of \emph{evaluation contexts} $E$
allowing computation within a larger program context, where $E[e]$
denotes an expression with $e$ in the hole $[]$ of $E$. Evaluation
contexts include boolean expression forms, allowing generalization
and instantiation of compositional program elements.
\begin{example}
  \label{example-share3-eval}
  Let $\codebase,e_{\ref{example-share3}}$ be the $\metaprot$ program and let 
  $\prog_{\ref{example-share3}}$ be the  $\minifed$ program defined
  in Example \ref{example-share3}. We refer to the latter as ``accumulated''
  by evaluation of the former in the sense that $\config{\varnothing}{e_{\ref{example-share3}}}
  \redxs \config{\prog_{\ref{example-share3}}}{\ttt{()}}$.
\end{example}

\subsection{Type Theory and Static Type Safety}

It is desirable to statically enforce safety of both $\metaprot$
programs and the safety of the $\fedprot$ programs they
generate. Although safety of the latter could be enforced
post-generation by a direct analysis, for large programs this can be
much more expensive and it is also better to not wast time on
resource-intensive compilation of programs with known errors
\cite{kreuter2012billion}. Some consequences of safety errors, for example accidental
reuse of one-time pads, can also undermine security.

The type syntax of $\metaprot$ is defined in Figure
\ref{fig-metaprot-tsyntax}. It includes a weak form of dependency:
string types $\stringty{e}$ and client types $\cidty{e}$
are parameterized by expressions $e$ that precisely reflect the
type of the value. Boolean expression forms have the type
$\bet{e}$ indexed by expressions $e$ indicating the client
id type of the expression. The dependency is weak in the
sense that expressions in types are a strict subset of
expression forms- for any $\stringty{e}$ the expression $e$
is either a variable, a string, or a concatenation form,
and for any $\cidty{e}$ or $\bet{e}$ the expression $e$
is either a variable or a client id $\cid$. 

Type judgements for expressions are of the form
$\tjudge{\viewst_1}{\gamma}{e}{\viewst_2}$ where the \emph{view
effect} $\viewst_1$ denotes the views that have been defined so far,
and $\viewst_2$ records new views defined as the effect of the
expression on the residual $\minifed$ program.  Type judgements are
syntax-directed- selected rules are shown in Figure
\ref{fig-metaprot-tjudge}. The $\TirName{AssignT}$ rule
captures the effect of new view definitions. The
$\TirName{And}$ rule illustrates how program safety is enforced,
by ensuring that subexpressions of boolean expressions have the
same owner.

The $\TirName{FnT}$ and $\TirName{Appt}$ rules apply to function
definition and application respectively, and rely on the
definition of function input type annotations $\tas$ and
type term substitutions $\sigma$. 
\begin{definition}
  A \emph{function input type annotation} $\tas$ is a mapping from
  function names $f$ to type products $\tau_1 * \cdots * \tau_n$.
  A \emph{type term substitution} $\sigma$ is a mapping from
  $\minifed$ variables $x$ to values, where $\sigma(\tau)$ denotes
  the replacement of occurences of $x$ in $\tau$ with $\sigma(x)$. 
\end{definition}
We assume that input type annotations $\tas$ are provided by the
programmer for all function definitions. This guarantees that
$\metaprot$ type checking is straightforward and efficient.
Function types are of the form:
$$
\tau_1 * \cdots * \tau_n \rightarrow \tau,\viewst
$$
where $\viewst$ denotes the effect of the function on the residual
program.  The function type can be understood as a dependent $\Pi$
type, with every term variable bound. When applied, these variables
are instantiated with a type term substitution $\sigma$. In our
implementation, we adapt \emph{synthesis} as defined in Dependent ML \cite{10.1145/292540.292560}
to obtain this $\sigma$- essentially this is a match on the syntactic
structure of types and expressions. 
\begin{example}
  Given $\ttt{share3}$ as defined in Example \ref{example-share3} and
  annotation:
  $$\tas(\ttt{share3}) =  \ttt{cid(client)}\ *\ \ttt{string(sid)}$$
  the type of $\ttt{share3}$ is $\tas(\ttt{share3}) \rightarrow \tau,\varnothing$
  where $\tau$ is:
  $$
  \ttt{\{ s1 : bool[client]; s2 : bool[client]; s3 : bool[client] \}}
  $$
\end{example}

\begin{fpfig}[t]{Type Syntax of $\metaprot$.}{fig-metaprot-tsyntax}
$$
\begin{array}{rcl@{\hspace{2mm}}r}
\srct &::=& \cidty{e} \mid \stringty{e} \mid \bet{e} \mid  & \gdesc{types}\\ 
 &&  \{ \flab : \srct;\ldots;\flab : \srct \} \mid \tau * \cdots * \tau \rightarrow \tau,\viewst \\[1mm]
\viewst  &::=& \view{e}{e};\viewst \mid \varnothing   & \gdesc{view effects}\\[1mm]
\Gamma &::=& \Gamma; x : \tau \mid \varnothing & \gdesc{type environments}    
\end{array}
$$
\end{fpfig}

\begin{fpfig}[t]{Selected $\metaprot$ type judgement rules.}{fig-metaprot-tjudge}
\begin{mathpar}
\inferrule[\TirName{VarT}]
{}
{\tjudge{\viewst}{\Gamma}{x}{\Gamma(x)}{\viewst}}

\inferrule[\TirName{CidT}]
{}
{\tjudge{\viewst}{\Gamma}{\cid}{\cidty{\cid}}{\viewst}}

\inferrule[\TirName{StringT}]
{}
{\tjudge{\viewst}{\Gamma}{w}{\stringty{w}}{\viewst}}

\inferrule[\TirName{ConcatT}]
{\tjudge{\viewst}{\Gamma}{e_1}{\stringty{e_1'}}{\viewst_1}\\
\tjudge{\viewst_1}{\Gamma}{e_2}{\stringty{e_2'}}{\viewst_2}
}
{\tjudge{\viewst}{\Gamma}{e_1||e_2}{\stringty{e_1' ||e_2'}}{\viewst_2}}

\inferrule[\TirName{BoolT}]
{}
{\tjudge{\viewst}{\Gamma}{\etrue}{\bet{\cid}}{\viewst}}

\inferrule[\TirName{OracleT}]
{\tjudge{\viewst}{\Gamma}{e}{\stringty{e'}}{\viewst'}}
{\tjudge{\viewst}{\Gamma}{\oracle{e}}{\bet{\cid}}{\viewst'}}

\inferrule[\TirName{SecretT}]
{\tjudge{\viewst}{\Gamma}{e_1}{\cidty{e_1'}}{\viewst_1}\\
\tjudge{\viewst_1}{\Gamma}{e_2}{\stringty{e_2'}}{\viewst_2}}
{\tjudge{\viewst}{\Gamma}{\secret{e_1}{e_2}}{\bet{e_1'}}{\views_2}}

\inferrule[\TirName{AndT}]
{
\tjudge{\viewst}{\Gamma}{e_1}{\bet{e}}{\viewst_1}\\
\tjudge{\viewst_1}{\Gamma}{e_2}{\bet{e}}{\viewst_2}
}
{\tjudge{\viewst}{\Gamma}{e_1\ \eand\ e_2}{\bet{e}}{\viewst_2}}

\inferrule[\TirName{AssignT}]
{
\tjudge{\viewst}{\Gamma}{e_1}{\cidty{e_1'}}{\viewst_1}\\
\tjudge{\viewst_1}{\Gamma}{e_2}{\stringty{e_2'}}{\viewst_2}\\
\tjudge{\viewst_2}{\Gamma}{e_3}{\bet{e_3'}}{\viewst_3}
}
{
\tjudge{\viewst}{\Gamma}{\eassign{\view{e_1}{e_2}}{e_3}}{\unity}{(\viewst_3 ; \view{e_1'}{e_2'} )}
}

\inferrule[AppT]
{\Gamma(f) =  
 \tau_1 * \cdots * \tau_n \rightarrow \tau, \viewst_f \\ 
 \tjudge{\viewst}{\Gamma}{e_1}{\sigma(\tau_1)}{\viewst_1}
 \ \cdots\  
 \tjudge{\viewst_{n-1}}{\Gamma}{e_n}{\sigma(\tau_n)}{\viewst_n}}
{\tjudge{\viewst}{\Gamma}{f(e_1,\ldots,e_n)}{\sigma(\tau)}{(\viewst_n ; \sigma(\viewst_f))}}

\inferrule[FnT]
{
  \codebase(f) = x_1,\ldots,x_n,\ e \\ \tas(f) = \tau_1 * \cdots * \tau_n \\
  \tjudge{\varnothing}{\Gamma; x_1 : \tau_1; \ldots; x_n : \tau_n}{e}{\tau}{\viewst_f}
}
{ \Gamma \vdash f : \tau_1 * \cdots * \tau_n \rightarrow \tau, \viewst_f }

\inferrule[ProgT]
{
\forall f \in \dom(\codebase)\ .\ \Gamma \vdash f : \Gamma(f) \\ \Gamma \vdash e : \tau, \view{\cid_1}{w_1};\ldots;\view{\cid_1}{w_1}
}
{
\Gamma \vdash \codebase,e : \tau,\{\view{\cid_1}{w_1}\} \sqcup \cdots \sqcup \{ \view{\cid_n}{w_n}\}
}
\end{mathpar}
\end{fpfig}

Top-level type judgements are of the form $\Gamma \vdash \codebase, e
: \tau, V$, where all the functions in $\codebase$ are well-typed in
$\Gamma$, the top level view effect $V$ is a set of concrete
$\fedprot$ views which are constructed by disjoint union of the views
in the effect of $e$- the disjointness requirement guarantees that
views are uniquely defined. Our $\metaprot$ type safety result is
formulated as follows. In addition to safe execution of the
metaprogram, it also guarantees safety of the residual $\fedprot$
program. Of course, it is important to note that ``safety'' here
means the usual type safety property that programs do not get stuck,
it does not imply any security hyperproperties. 
\begin{theorem}[$\metaprot$ Type Safety]
  \label{theorem-metalang-safety}
  Given $\codebase$, $e$, and $\Gamma$ with $\Gamma \vdash \codebase,e : \unity : V$,
  then $\config{\varnothing}{e} \redxs \config{\prog}{\ttt{()}}$ where
  $\prog$ is safe with $\iov(\prog) = S \cup V$ for some $S$.
\end{theorem}


\section{2-Party GMW and Passive Security Proof Tactics}
\label{section-metalang-gmw}
\label{section-example-gmw}

\begin{fpfig}[t]{2-Party GMW circuit library with And gate.}{fig-gmw}
{\footnotesize
  \begin{verbatimtab}
    encodegmw(in, i1, i2) {
      m[in]@i2 := (s[in] xor r[in])@i2;
      m[in]@i1 := r[in1]@2;
      m[in]
    }
    
    andtablegmw(b1, b2, r) {
      let r11 = r xor (b1 xor true) and (b2 xor true) in
      let r10 = r xor (b1 xor true) and (b2 xor false) in
      let r01 = r xor (b1 xor false) and (b2 xor true) in
      let r00 = r xor (bl xor false) and (b2 xor false) in
      { v1 = r11; v2 = r10; v3 = r01; v4 = r00 }
    }
    
    andgmw(g, v1, v2) {
      let r = r[g] in
      let table = andtablegmw(v1,v2,r) in
      m[g]@2 := OT4(v1,v2,table,2,1);
      m[g]@1 := r;
      m[g]
    }
    
    decodegmw(v) {
      p[1] := v@1; p[2] := v@2;
      out@1 := (p[1] + [2])@1;
      out@2 :=(p[1] + [2])@2
    }
  \end{verbatimtab}
}
\end{fpfig}


As an extended example of our language and security model, and how the
automated techniques in Section \ref{section-bruteforce} can serve
as tactics integrated with PSL/Lilac-style proofs, we consider GMW
circuits.  The GMW protocol is a garbled binary circuit protocol.  We
will assume the 2-party version, though it generalizes to $n$
parties\cite{XXX}. GMW uses a common technique in MPC, which is to
represent values $v$ as distributed shares $v_1$ and $v_2$ with $v =
v_1 \fplus v_2$. This trick maintains secrecy of $v$ from both
parties, and in GMW it is used to maintain the intermediate values of
internal gate outputs in circuits. In related literature the notation
$\macgv{x}$ is used to represent the ``true'' value of $x$ and $[x]$
is often used to represent the share of given party.

To capture this convention, which is used in many other protocols, we
introduce a new naming convention for ``global view'' elements
$\macgv{\mesg{w}}$, which denote the summed value of
$\elab{\mesg{w}}{1}$ and $\elab{\mesg{w}}{2}$ in a protocol
run. This concept integrates program distributions in the
usual manner, as the probability of the outcome of summation
of two variables in the distribution.
\begin{definition}
  For all $\mesg{w}$ define:
  $$\pmf(\macgv{\mesg{w}} = v) \defeq \sum_{\sigma \in A} \pmf(\sigma)$$
  and define:
  $$\condd{\pmf}{X}{\macgv{\mesg{w}} = v}(\sigma) \defeq  \sum_{\sigma' \in A} \condd{\pmf}{X}{\sigma'}(\sigma)$$
  where $A$ is:
  $$\{ \store \in \mems(\{ \elab{\mesg{w}}{1},\elab{\mesg{w}}{2} \} ) \mid
      \fcod{\store(\elab{\mesg{w}}{1}) + \store(\elab{\mesg{w}}{2})} = v \}$$
\end{definition}

For full details of the GMW protocol the reader is referred to
\cite{evans2018pragmatic}. Our implementation libary is shown in
Figure \ref{fig-gmw}, and includes encoding functions, where
input secrets are split into shares, $\eand$ and $\exor$ gate
functions, and a decoding function. Note that $\exor$ requires
no interaction between parties, while $\eand$ necessitates and
1-of-4 oblivious transfer. The gate computation is
done entirely in secrect, and the decoding function
is where the declassification occurs-- both parties reveal
their shares of the final gate output $\macgv{z}$.

For example, the following program uses our GMW library to define
a circuit with a single \eand gate and input secrets $\ttt{s1}$ and
$\ttt{s2}$ from client's 1 and 2 respectively:
\begin{verbatimtab}
         let s1 = encodegmw("s1") in
         let s2 = encodegmw("s2") in
         decodegmw(andgmw("z",s1,s2))
\end{verbatimtab}
By convention we will assume that all gates are assigned unique output
identifiers $\ttt{"z"}$, and that all programs are in the form
of a sequence of let-bindings followed by a call to $\decodegmw$
wrapping a circuit.

\paragraph{Oblivious Transfer} A passive secure oblivious transfer (OT) protocol
based on previous work \cite{XXX} can be defined in $\metaprot$,
however this protocol assumes some shared randomness. Alternatively,
a simple passive secure OT can be defined with the addition of
public key cryptography as a primitive. But given the diversity
of approaches to OT, we instead assume that OT is abstract with
respect to its implementation, where calls to OT in $\mathbb{F}_2$
are of the following form-- given a \emph{choice bit}
$\be_1$ provided by a receiver $\cid$, the sender
sends either $\be_2$ or $\be_3$.
$$
\OT{\elab{\be_1}{\cid}}{\be_2}{\be_3}
$$
Critically, the sender learns nothing about $\be_1$ and the
receiver learns nothing about the unselected value, so we interpret
these calls in our implementation in $\mathbb{F}_2$ as follows.
$$
\begin{array}{l}
\solve{\stores}{\OT{\elab{\be_1}{\cid_1}}{\be_2}{\be_3}}{\cid_2} = \\
\qquad ((\solve{\stores}{\be_1}{\cid_1}) \cap 
(\solve{\stores}{\be_2}{\cid_2})) \cup \\
\qquad ((\stores - (\solve{\stores}{\be_1}{\cid_1})) \cap
(\solve{\stores}{\be_3}{\cid_2})
\end{array}
$$

\subsection{Correctness Proof with Verification Tactics}

As discussed above and in related work \cite{XXX}, probabilistic
separation conditional on certain variables-- e.g., secret inputs or
public outputs-- is a key mechanism for reasoning about MPC protocol
security. Following \cite{XXX}, we define a conditional separation
relation $\condsep{\pmf}{X_1}{X_2}{X_3}$ to mean that
under the condition of all value assignments for
$X_2$ and $X_3$ are separable under $\pmf$ on the condition
of any value assignment of $X_1$-- i.e., conditionally
on any $\store \in \mems(X_1)$. 
\begin{definition}[Conditional Separation]
  We write $\condsep{\pmf}{X_1}{X_2}{X_3}$ iff for all
  $\store' \in \mems(X_1)$ and $\store \in \mems(X_2 \cup X_3)$
  and letting $X = X_1 \cup X_2 \cup X_3$ we have:
  $$\condd{\pmf}{X}{\store'}(\store) =
  \condd{\pmf}{X}{\store'}(\store_{X_2}) *
  \condd{\pmf}{X}{\store'}(\store_{X_3})$$
\end{definition}
Another key concept needed especially for reasoning about
circuits is conditional determinism. For example, if
$\macgv{z}$ is an output of an internal gate, it will
definitely be computed using random variables, however,
it \emph{should} be determistic under any set of input
secrets $S$, since we assume that $\idealf$ is
deterministic.
\begin{definition}[Conditional Determinism]  
  We write $\conddetx{\pmf}{X_1}{X_2}$ iff for all
  $\store' \in \mems(X_1)$ there exists 
  $\store \in \mems(X_2)$ such that
  $\condd{\pmf}{X_1 \cup X_2}{\store'}(\store) = 1$.
\end{definition}

Given these definitions, we can formulate an invariant
for circuit computation with respect to internal gates
as follows. It says that the output of any gate is
deterministic wrt inputs $S$, and conditionally
on $S$ the output $\macgv{\mesg{z}}$ remains
separable from corrupt views and both shares of
$\macgv{\mesg{z}}$. This last nuance is critical,
since those shares will in fact be revealed if
$\macgv{\mesg{z}}$ is decoded as the circuit output. 
\begin{lemma}[GMW Invariant]
  \label{lemma-gmwinvariant}
  Given:
  $$ (\varnothing,e) \redxs (\prog,\decodegmw(E[\mesg{z}])) $$
  Then both of the following conditions hold for all $H$ and $C$ where $\iov(\prog) = S \cup M$:
  \begin{enumerate}
    \item $\conddetx{\progtt(\prog)}{S}{\{\macgv{\mesg{z}}\}}$
    \item $\condsep{\progtt(\prog)}{S}{\{\macgv{\mesg{z}}\}}{(M_C \cup \{ \elab{\mesg{z}}{1}, \elab{\mesg{z}}{2} \})}$
  \end{enumerate}
\end{lemma}
To prove this, we can formulate and automatically prove gate-level
versions of the invariant. This serves as a proof tactic
that simplifies the proof of the GMW invariant. It establishes
that the $\eand$ gate output is deterministic conditional on
the inputs, and is separable from the output shares and
either parties' received messages. 
\begin{lemma}[And Gate Tactic]
  \label{lemma-gmwtactic}
  %Define:
  %$$
  %\begin{array}{c}
  %  \prog_{i} \defeq \\
  %  \eassign{\mesg{x}}{1}{\flip{x}}{1}; \eassign{\mesg{x}}{2}{\flip{x}}{2}; \\
  %  \eassign{\mesg{y}}{1}{\flip{y}}{1}; \eassign{\mesg{y}}{2}{\flip{y}}{2} 
  %\end{array}
  %$$
  Given:
  $$
  \begin{array}{c}
  (\varnothing,\andgmw(z,\mesg{x},\mesg{y}) \redxs %\\
  (\prog,\mesg{z})
  \end{array}
  $$
  Then both of the following conditions hold for both $\cid \in \{ 1,2 \}$ where $\iov(\prog) = M$:
  \begin{enumerate}
  \item
    $\conddetx{\progtt(\prog)}{\{ \macgv{\mesg{x}},\macgv{\mesg{y}} \}}{\{ \macgv{\mesg{z}} \}}$
   \item $\condsep
      {\progtt(\prog)}
      {\{ \macgv{\mesg{x}},\macgv{\mesg{y}} \}}
      {\{ \macgv{\mesg{z}} \}}
      {M_{\{\cid\}} \cup \{ \elab{\mesg{z}}{1},\elab{\mesg{z}}{2} \}}$
  \end{enumerate}
\end{lemma}
\begin{proof}
Verified automatically using techniques described in Section \ref{section-bruteforce}.  
\end{proof}

To properly integrate the local reasoning of Lemma \ref{lemma-gmwtactic} with
the global reasoning of Lemma \ref{lemma-gmwinvariant}, we can demonstrate
the following.
\begin{lemma}
  \label{lemma-conditioning}
  Each of the following hold:
  \begin{enumerate}
    \item Given $\condp{\pmf}{X_1}{\detx{X_2}}$ and
      $\condp{\pmf}{X_2}{\detx{X_3}}$, then $\condp{\pmf}{X_1}{\detx{X_3}}$.
    \item Given $\condp{\pmf}{X_1}{\detx{X_2}}$ and
      $\condsep{\pmf}{X_2}{X_3}{X_4}$, then $\condsep{\pmf}{X_1}{X_3}{X_4}$.
    \item Given $\condsep{\pmf}{X_1}{X_2}{X_3}$ and $\condp{\pmf}{X_1}{\detx{X_2}}$
      and $\condp{\pmf}{X_2}{\detx{X_4}}$, then $\condsep{\pmf}{X_1}{X_4}{X_3}$.
    \item Given $\condsep{\pmf}{X_1 \cup X_2}{X_3}{X_4}$ and $\condp{\pmf}{X_1 \cup X_2}{\detx{X_3}}$,
      then $\condsep{\pmf}{X_1}{X_2 \cup X_3}{X_4}$.
  \end{enumerate}
\end{lemma}

Then we can put the pieces together to prove the invariant, using automated tactics
for gate-level reasoning.  
\begin{proof}[Proof of Lemma \ref{lemma-gmwinvariant}]
  By induction on the length of $(\varnothing,e) \redxs (\prog,\decodegmw(E[\mesg{z}]))$.
  Encoding establishes the invariant in the basis. The most interesting inductive
  case is the $\andgmw$ gate. 
  \paragraph{Case $\andgmw$.} In this case we have:
  $$
  \begin{array}{c}
  (\varnothing,e) \redxs (\prog',\decodegmw(E[\andgmw(z,\mesg{x},\mesg{y})])) \redxs \\
    (\prog,\decodegmw(E[\mesg{z}]))
  \end{array}
  $$
  The induction hypothesis, together with the assumed uniquenenss of $z$, gives:
  \begin{mathpar}
  \condsep{\progtt(\prog)}{S^1}{\{ \macgv{\mesg{x}} \}}{(M^1_C \cup \{ \elab{\mesg{x}}{1}, \elab{\mesg{x}}{2} \})}
  
  \condsep{\progtt(\prog)}{S^2}{\{ \macgv{\mesg{y}} \}}{(M^2_C \cup \{ \elab{\mesg{y}}{1}, \elab{\mesg{y}}{2} \})}

  \conddetx{\progtt(\prog)}{S^1}{\{ \macgv{\mesg{x}} \}}
  
  \conddetx{\progtt(\prog)}{S^2}{\{ \macgv{\mesg{y}} \}}    
  \end{mathpar}
  This, together with Lemma \ref{lemma-gmwtactic} (1) and Lemma \ref{lemma-conditioning} (1)
  give:
  $$
  \conddetx{\progtt(\prog)}{S^1 \cup S^2}{\{ \macgv{\mesg{z}} \}}
  $$
  and Lemma  \ref{lemma-gmwtactic} (2) and Lemma \ref{lemma-conditioning} (2-3) gives:
  $$
  \begin{array}{c}
    \progtt(\prog)|S^1 \cup S^2) \vdash \\
    {\{ \macgv{\mesg{z}} \}} * {(M^1_C \cup M^2_C \cup \{ \elab{\mesg{x}}{2}, \elab{\mesg{y}}{2}, \elab{\mesg{z}}{1}, \elab{\mesg{z}}{2} \})}
  \end{array}
  $$
\end{proof}

\begin{theorem}
  \label{theorem-gmw}
  If $e$ is a GMW circuit protocol in $\metaprot$ with $(\varnothing,e) \redxs (\prog,())$
  then $\prog$ satisfies noninterference modulo output. 
\end{theorem}

\begin{proof}
  Given that $e$ is a GMW circuit protocol, then by definition we have:
  $$
  (\varnothing,e) \redxs (\prog',\decodegmw(\mesg{z})) \redxs (\prog,())
  $$
  where by Lemma \ref{lemma-gmwinvariant} and definition of $\decodegmw$,
  for any $H$ and $C$ letting $\iov(\prog) = S \cup M \cup P \cup O$ we
  have:
  \begin{mathpar}
    \conddetx{\progtt(\prog)}{S}{\{\macgv{\mesg{z}}\}}

    \condsep{\progtt(\prog)}{S}{\{\macgv{\mesg{z}}\}}{(M_C \cup \{ \elab{\mesg{z}}{1}, \elab{\mesg{z}}{2} \})}
  \end{mathpar}
  so also by definition of $\decodegmw$ we have:
  \begin{mathpar}
    \conddetx{\progtt(\prog)}{S_H \cup S_C}{O}
    
    \condsep{\progtt(\prog)}{S_H \cup S_C}{O}{(M_C \cup P)}
  \end{mathpar}
  and this by Lemma \ref{lemma-conditioning} (4) we have:
  $$
  \condsep{\progtt(\prog)}{S_C}{(O \cup S_H)}{(M_C \cup P)}
  $$
  which implies the result.
\end{proof}


\subsection{2-Party BDOZ and Integrity Enforcement}
\label{section-example-bdoz}

\begin{fpfig}[t]{2-party BDOZ circuit library: sum gates and secure opening.}{fig-bdozsum}
{\footnotesize
\begin{verbatimtab}
  _open(x,i1,i2){
    m[x++"exts"]@i1 := m[x++"s"]@i2;
    m[x++"extm"]@i1 := m[x++"m"]@i2;
    assert(m[x++"extm"] = m[x++"k"] + (m["delta"] * m[x++"exts"]))@i1;
    m[x]@i1 := (m[x++"exts"] + m[x++"s"])@i1
  }
  
  pre: { m[x++"m"]@i2 == m[x++"k"]@i1 + (m["delta"]@i1 * m[x++"s"])@i2 /\
         m[y++"m"]@i2 == m[y++"k"]@i1 + (m["delta"]@i1 * m[y++"s"])@i2 }
  _sum(z, x, y,i1,i2) {
      m[z++"s"]@i2 := (m[x++"s"] + m[y++"s"])@i2;
      m[z++"m"]@i2 := (m[x++"m"] + m[y++"m"])@i2;
      m[z++"k"]@i1 := (m[x++"k"] + m[y++"k"])@i1
  }
  post: { m[z++"m"]@i2 == m[z++"k"]@i1 + (m["delta"]@i1 * m[z++"s"]@i2 } 
  
  sum(z,x,y) { _sum(z,x,y,1,2); _sum(z,x,y,2,1) }

  open(x) { _open(x,1,2); _open(x,2,1) }
\end{verbatimtab}
}
\end{fpfig}

\begin{fpfig}[t]{2-party BDOZ circuit library: multiplication gates.}{fig-bdozmult}
{\footnotesize
\begin{verbatimtab}
  _mult1(z, x, y) {
      m[z++"s"]@1 :=
        (m[z++"bs"] * m[z++"d"] + m[z++"as"] * m[z++"e"] + m[z++"cs"])@1;
      m[z++"m"]@1 :=
        (m[z++"bm"] * m[z++"d"] + m[z++"am"] * m[z++"e"] + m[z++"cm"])@1;
      m[z++"k"]@2 :=
        (m[z++"bk"] * m[z++"d"] + m[z++"ak"] * m[z++"e"] + m[z++"ck"])@2    
  }
  post: { m[z++"m"]@2 == m[z++"k"]@1 + (m["delta"]@1 * m[z++"s"]@1 }

  mult(z,x,y) {
      sum(z++"a", x, z++"d");
      open(z++"d");
      sum(z++"b", y, z++"e");
      open(z++"e"); 
      _mult1(z,x,y); _mult2(z,x,y)
  }
  post: {  m[z++"s"]@1 + m[z++"s"]@2 ==
          (m[x++"s"]@1 + m[x++"s"]@2) * (m[y++"s"]@1 + m[y++"s"]@2)} 
  
\end{verbatimtab}
}
\end{fpfig}

% hints for confidentiality
\begin{comment}
      m[z++"ds"]@1 as m[x++"s"]@2 + r[z++"as"]@2;
      m[z++"ds"]@2 as m[x++"s"]@1 + r[z++"as"]@1;
      m[z++"ms"]@2 as m[z++"k"]@1 + (m["delta"]@1 * m[z++"ds"]@2);
      m[z++"ms"]@1 as m[z++"k"]@2 + (m["delta"]@2 * m[z++"ds"]@1);
\end{comment}


In a malicious setting, ``detecting cheating'' by adding
information-theoretic secure MAC codes to shares is a fundamental
approach realized by protocols such as BDOZ \cite{XXX} and SPDZ
\cite{XXX}.  These protocols assume a pre-processing phase that
distributes shares with MAC codes to clients.  This integrates well
with pre-processed Beaver Triples to implement malicious secure, and
relatively efficient, multiplication \cite{XXX}. Recall
that Beaver triples are values $a,b,c$ with $\fcod{a * b} = c$,
unique per multiplication gate, that are secret shared with clients
during pre-processing. Here we consider the 2-party version.

A field value $v$ is secret shared among 2 clients in BDOZ in the same
manner as in GMW.  Each client $\cid$ gets a pair of the form
$(v_\cid,m_\cid)$, where $v_\cid$ is a share of $v$ reconstructed by
addition, i.e., $v = \fcod{v_1 \fplus v_2}$, and $m_\cid$ is a MAC of
$v_\cid$.  More precisely, $m_\cid = k + (k_\Delta * v_\cid)$ where
\emph{keys} $k$ and $k_\Delta$ are known only to $\cid' \ne \cid$ and
$k_\Delta$. The \emph{local key} $k$ is unique per MAC, while the
\emph{global key} $k_\Delta$ is common to all MACs authenticated by
$\cid'$. This is a semi-homomorphic encryption scheme that supports
addition of shares and multiplication of shares by a constant
\cite{XXX}. For more details the reader is referred to \cite{XXX}. We
note that while we restrict values $v$, $m$, and $k$ to the same field
in this presentation for simplicity, in general $m$ and $k$ can be in
extensions of $\mathbb{Z}_p$.

We can capture both the preliminary distribution of Beaver triples and BDOZ shares
as a pre-processing predicate that establishes conditions for initial
memories (see Definition \ref{def-progtt}).  Here we assume two input
secrets $\elab{\secret{x}}{1}$ and $\elab{\secret{y}}{2}$ and a single
Beaver Triple to compute $\elab{\secret{x}}{1} \ftimes
\elab{\secret{y}}{2}$, but we can extend this for additional gates.
As for GMW, we use $\macgv{\mesg{w}}$ to refer to secret-shared values
reconstructed with addition, where by convention shares are message
values $\elab{\mesg{w}}{\cid}$ for all $\cid$.
\begin{definition}[BDOZ preprocessing]
  Define:
  \begin{mathpar}
    \mathit{shares} \defeq
    \{ \elab{\mesg{w}}{\cid}\ |\ \cid \in \{ 1, 2 \} \wedge w \in \{ a,b,c,x,y \}  \}

    \mathit{macs} \defeq  \{ \elab{\mesg{w\ttt{mac}}}{\cid}\ |\ \cid \in \{ 1, 2 \} \wedge w \in \{ a,b,c,x,y \}  \}

    \mathit{keys} \defeq  \begin{array}{l}\{ \elab{\mesg{w\ttt{k}}}{\cid}\ |\ \cid \in \{ 1, 2 \} \wedge w \in
    \{ a,b,c,x,y \}  \} \cup \\ \{ \elab{\mesg{\ttt{delta}}}{\cid}\ |\ \cid \in \{ 1, 2 \} \} \end{array}
  \end{mathpar}
  Then a memory $\store$ satisfies BDOZ preprocessing iff:
  $$\dom(\store) = \{ \elab{\secret{x}}{1}, \elab{\secret{y}}{2} \} \cup \mathit{shares}
  \cup \mathit{macs} \cup \mathit{keys}$$
  and, writing $\store(\macgv{\mesg{w}})$ to denote
  $\fcod{\store(\elab{\mesg{w}}{1}) + \store(\elab{\mesg{w}}{2})}$,
  the following conditions hold:
  \begin{mathpar}
    \store(\macgv{\mesg{x}}) = \store(\elab{\secret{x}}{1})
    
    \store(\macgv{\mesg{y}}) = \store(\elab{\secret{y}}{2})
    
    \fcod{\store(\macgv{\mesg{a}}) * \store(\macgv{\mesg{b}})} = \store(\macgv{\mesg{c}})
  \end{mathpar}
  and for all $\cid,\cid' \in \{1,2\}$ with $\cid \ne \cid'$ and $w \in \{ a,b,c,x,y\}$:
  $$\lcod{\store, \mesg{w\ttt{mac}}}{\cid} =
  \lcod{\store, \mesg{wk} + \mesg{\ttt{delta}} * \mesg{w}}{\cid'}$$
\end{definition}

With these conventions, our BDOZ library is defined in Figure \ref{fig-bdoz}.
In $\metaprot$ we represent BDOZ share, MAC pairs as records:
$$
\ttt{\{ share = }v\ttt{;  mac =}\ m \ttt{\}}
$$
and we define $\ttt{macsum}$ for addition of shares,
$\ttt{maccsum}$ for addition of a share and a constant, and
$\ttt{macctimes}$ for multiplication of a share and a constant
in the BDOZ encryption scheme \cite{XXX}. The $\ttt{auth}$
function implements the MAC check as an $\assert$.

We also define a function $\ttt{secopen}$ to implement ``secure
opening''.  In this standard subprotocol, the value
$\macgv{\secret{w_1}} \fplus \macgv{\mesg{w_2}}$ is reconstructed as
$\mesg{w_3}$, by each client $\cid_2$ communicating
$\lcod{\mesg{w_1} + \mesg{w_2}}{\cid_2}$ to $\cid_1$.  Assuming that
$\macgv{\mesg{w_2}}$ is in an independent uniform distribution,
this perfectly hides $\secret{w_1}$. In a mutiplication gate
either $a$ or $b$ of a Beaver triple are used in secure opening,
so, e.g., given $
(\varnothing,\ttt{secopen}(a, x, d, 2, 1) \redxs (\prog,())$
we have both:
\begin{mathpar}
  \conddetx{\progtt(\prog)}{\{\macgv{\mesg{a}},\elab{\secret{x}}{1} \}}{\{ \elab{\mesg{d}}{2} \}}
  
  \sep{\progtt(\prog)}{\{ \elab{\mesg{d}}{2} \}}{\{ \elab{\secret{x}}{1}  \}}
\end{mathpar}
Furthermore, client 2's authentication of the sum of shares with the
sum of their keys detects any attempted cheating by $1$.
Finally, the $\ttt{secreveal}$ function
is very similar to $\decodegmw$, except with the addition
of authentication of revealed shares to ensure malicious security. 

\begin{fpfig}[t]{Authenticated 2-Party Multiplication.}{fig-beaver}
{\footnotesize
  \begin{verbatimtab}
    secopen("a","x","d",1,2);
    secopen("a","x","d",2,1);
    secopen("b","y","e",1,2);
    secopen("b","y","e",2,1);
    let xys =
      macsum(macctimes(macshare("b"), m["d"]),
             macsum(macctimes(macshare("a"), m["e"]),
                    macshare("c")))
    in
    let xyk = mack("b") * m["d"] + mack("d") * m["d"] + mack("c")               
    in
    secreveal(xys,xyk,"1",1,2);
    secreveal(maccsum(xys,m["d"] * m["e"]),
              xyk - m["d"] * m["e"],
              "2",2,1);
    out@1 := (p[1] + p[2])@1;
    out@2 := (p[1] + p[2])@2;
  \end{verbatimtab}
}
\end{fpfig}


The full protocol for malicious secure product of secrets $x$ (that
is, $\elab{\secret{x}}{1}$) and $y$ (that is, $\elab{\secret{y}}{2}$)
using Beaver triple $a,b,c$ is defined in Figure \ref{fig-bdoz}. Both
parties interact in secure opening of $x \fplus a$ and $y + b$,
followed by the non-interactive computation of shares of $x * y$
for secure reveals as per standard protocol
\cite{XXX}. The non-interactive reconstruction of the local authentication
key for both the secure openings and the final reveal is enabled by the
semi-homorphic properties of the BDOZ scheme.

\subsection{Cheating Detection and Integrity}

We can carry out similar proofs of passive security for the protocol in
Figure \ref{fig-beaver} as for GMW, even using automated tactics for
the protocol in $\mathbb{F}_2$. But in the case of BDOZ we are also
concerned with malicious security. To demonstrate this, we can
demonstrate that the protocol satisfies integrity in the sense of
Definition \ref{def-integrity}. To do so, we observe that it satisfies
a stronger property, that we call cheating detection. Intuitively,
integrity says that the only thing that the adversary can do in the
malicious model is to elicit the same responses from honest parties
that an honest run of the protocol would elicit. Cheating detection
says that the adversary can only execute the protocol honestly, or
or else gets caught (and abort).

Focusing in, we identify the adversarial inputs as the messages
sent from the adversary to honest parties, on which honest responses
to the adversary may depend. We want to say that these are the messages
that must be legitimate.
\begin{definition}
  Given $\prog$ with $\iov(\prog) = S \cup V \cup O$,
  let $X_H \subseteq \{ x \mid x \in (\houtputs \cup O_H) \wedge x \in \dom(\store) \}$.
  Then the \emph{adversarial inputs to $X_H$} is the least set
  $X_C \subseteq \cinputs$ such that $\progtt(\prog) \not\vdash X_C * X_H$.
\end{definition}
Now, we can characterize protocols with cheating detection as those where
adversarial inputs to honest reponses must themselves be constructed honestly. 
\begin{definition}[Cheating Detection]
  \emph{Cheating is detected} in $\prog$ with $\iov(\prog) = S \cup V \cup O$ iff
  for all  $\store \in \aruns(\prog)$,
  letting $X_H = \{ x \mid x \in (\houtputs \cup O_H) \wedge x \in \dom(\store) \}$,
  and letting $X_C$ be the adversarial inputs to $X_H$,
  there exists $\sigma'\in \runs(\prog)$
  with $\store_{X_C} = \store'_{X_C}$.  
\end{definition}

It is straightforward to demonstrate that cheating detection has integrity,
since only the ``passive'' adversary can elicit a response from honest parties. 
\begin{lemma}
  \label{lemma-cheating}
  If cheating is detected in $\prog$, then $\prog$ has integrity.
\end{lemma}

In the case of BDOZ, cheating detection is accomplished by the information-theoretic
security of the encryption scheme\cite{XXX}. Furthermore, the symmetry of
the protocol in Figure \ref{fig-beaver} ensures that both parties will authenticate
shares, so it is robust to corruption of either party. 


\bibliographystyle{ACM-Reference-Format}
\bibliography{logic-bibliography,secure-computation-bibliography}

\end{document}
\endinput
