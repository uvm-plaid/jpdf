\documentclass[sigconf]{acmart}

\usepackage{amsmath}
\usepackage{amstext}
\usepackage{fp-frame}
\usepackage[latin1]{inputenc}
\usepackage{mathpartir}
\usepackage{fancyvrb}
\usepackage{moreverb}
\usepackage{stmaryrd}
\usepackage{enumerate}
\usepackage{thmtools,thm-restate}
\usepackage{comment}
\usepackage{booktabs,array}
\usepackage{rotating}

\newcommand{\evals}{\Downarrow}
\newcommand{\diverges}{\Uparrow}
\newcommand{\intt}{\mathrm{int}}
\newcommand{\unitt}{\mathrm{unit}}
\newcommand{\boolt}{\mathrm{bool}}
\newcommand{\floatt}{\mathrm{float}}
\newcommand{\stringt}{\mathrm{string}}
\newcommand{\chart}{\mathrm{char}}
\newcommand{\vb}[1]{\verb+#1+}
\newcommand{\evalexmp}[2]{\texttt{#1}\ \ensuremath{\evals}\ \texttt{#2}}
\newcommand{\texmp}[2]{\texttt{#1\ :\ #2}}
\newcommand{\skipper}{\bigskip\\}
\newcommand{\fyi}{\noindent\textbf{\textit{fyi:}}\ }
\newcommand{\NB}{\noindent\textbf{NB:\ }}
\newcommand{\const}{\ensuremath{\mathbf{c}}}
\newcommand{\defn}{\heading{definition}}
\newcommand{\defeq}{\triangleq}
\newcommand{\nat}{\mathbb{N}}
\newcommand{\atom}{\texttt{const}}
\def\squareforqed{\hbox{\rlap{$\sqcap$}$\sqcup$}}
\def\qed{\ifmmode\squareforqed\else{\unskip\nobreak\hfil
\penalty50\hskip1em\null\nobreak\hfil\squareforqed
\parfillskip=0pt\finalhyphendemerits=0\endgraf}\fi}
\newcommand{\exampletab}[1]{\skipper\begin{tabular}{lll}#1\end{tabular}\skipper}
\newcommand{\verbtab}[1]{\skipper\begin{verbatimtab}{#1}\end{verbatimtab}\skipper}
\newcommand{\eqntab}[1]{\skipper\begin{tabular}{rcl}#1\end{tabular}\skipper}
\newcommand{\recdefn}[1]{\{#1\}}
\newcommand{\ttt}[1]{\texttt{#1}}
\newcommand{\gdesc}[1]{\text{\textit{#1}}}
\newcommand{\true}{\mathrm{true}}
\newcommand{\false}{\mathrm{false}}
\newcommand{\etrue}{\texttt{true}}
\newcommand{\efalse}{\texttt{false}}
\newcommand{\reval}{\Rightarrow}
\newcommand{\Dand}{\ \mathrm{and}\ }
\newcommand{\Dor}{\ \mathrm{or}\ }
\newcommand{\Dxor}{\ \mathrm{xor}\ }
\newcommand{\Dnot}{\mathrm{not}\ }
\newcommand{\cod}[1]{\llbracket #1 \rrbracket}
\newcommand{\lcod}[2]{\llbracket #1 \rrbracket_{#2}}
\newcommand{\Dplus}{\mathrm{Plus}}
\newcommand{\Dminus}{\mathrm{Minus}}
\newcommand{\Dequal}{\mathrm{=}}
\newcommand{\Dabs}[2]{(\mathrm{Function}\ #1 \rightarrow #2)}
\newcommand{\Dfix}[3]{(\mathrm{Fix}\ #1 . #2 \rightarrow #3)}
\newcommand{\Dite}[3]{\mathrm{If}\ #1\ \mathrm{Then}\ #2\ \mathrm{Else}\ #3}
\newcommand{\dotminus}{\stackrel{.}{-}}
\newcommand{\Dlet}[3]{\mathrm{Let}\ #1 = #2\ \mathrm{In}\ #3}
\newcommand{\Dletrec}[4]{\mathrm{Let\ Rec}\ #1\ #2 = #3\ \mathrm{In}\ #4}
\newcommand{\Dfst}{\mathrm{left}}
\newcommand{\Dsnd}{\mathrm{right}}
\newcommand{\labset}{\mathit{Lab}}
\newcommand{\Drec}[1]{\{ #1 \}}
\newcommand{\linfer}[3]{\inferrule*[right=(\TirName{#1})]{#2}{#3}}
\newcommand{\lab}[1]{\mathrm{#1}}
\newcommand{\loc}{\ell}
\newcommand{\Dref}[1]{\mathrm{Ref}\,#1}
%\newcommand{\store}{\mathcal{M}}
\newcommand{\store}{m}
%\newcommand{\stores}{\overline{\store}}
\newcommand{\stores}{\Sigma}
\newcommand{\config}[2]{( #1,#2 )}
\newcommand{\configf}[2]{\begin{array}[t]{l}\langle #1\\,\\ #2 \rangle \end{array}}
\newcommand{\extend}[3]{#1\{#2 \mapsto #3\}}
\newcommand{\emptystore}{\{\}}
\newcommand{\storedefn}[1]{\{#1\}}
\newcommand{\Dret}[1]{\mathrm{Return}\,#1}
\newcommand{\Draise}[1]{\mathrm{Raise}\,#1}
\newcommand{\Dexn}[2]{\#\!#1\,#2}
\newcommand{\Dtry}[3]{\mathrm{Try}\,#1\,\mathrm{With}\,#2 \rightarrow #3}
\newcommand{\xname}{\mathit{exn}}
\newcommand{\Dboolt}{\mathrm{Bool}}
\newcommand{\Dreft}[1]{#1\,\mathrm{ref}}
\newcommand{\reft}[1]{#1\,\mathrm{ref}}
\newcommand{\Dintt}{\mathrm{Int}}
%\newcommand{\tjudge}[3]{#1 \vdash #2 : #3}
\newcommand{\textend}[3]{#1;#2:#3}
\newcommand{\fnty}[2]{#1 \rightarrow #2}
\newcommand{\TDabs}[3]{(\mathrm{Function}\ (#1 : #2) \rightarrow #3)}
\newcommand{\TDfix}[5]{(\mathrm{Fix}\ #1 . (#2 : #3) : #4 \rightarrow #5)}
\newcommand{\TDletrec}[6]{\Dletrec{#1}{#2 : #3}{#4 : #5}{#6}}
\newcommand{\emptyenv}{\varnothing}
\newcommand{\tfail}{\mathbf{fail}}
\newcommand{\tcheck}{\mathrm{TC}}
\newcommand{\tcheckfail}{\mathbf{TypeMismatch}}
\newcommand{\algtab}[1]
{
\vspace*{-3mm}
\begin{tabbing}
\hspace*{12mm}\=\hspace{9mm}\=\hspace{9mm}\=\hspace{6mm}\=\hspace{6mm}\=
\hspace{6mm}\=
#1
\end{tabbing}
}
\newcommand{\assign}[2]{#1 := #2}
\newcommand{\ederef}[1]{\,!#1}
\newcommand{\declass}[2]{\mathrm{declassify}_{#2}(#1)}
\newcommand{\eendorse}[2]{\mathrm{endorse}_{#2}(#1)}
\newcommand{\lt}{\left\{}
\newcommand{\rt}{\right\}}
\newcommand{\Lt}{\left\{\!\!\right.}
\newcommand{\Rt}{\left.\!\!\right\}}
\newcommand{\tinfer}{\mathit{PT}}
\newcommand{\unify}{\mathit{unify}}
\newcommand{\tsubn}{\varphi}
\newcommand{\scheme}[2]{\forall #1 . #2}
\newcommand{\Dself}{\mathrm{this}}
\newcommand{\Dsuper}{\mathrm{super}}
\newcommand{\Dsend}[3]{#1.#2(#3)}
\newcommand{\Dselect}[2]{#1.#2}
\newcommand{\Demptyclass}{\mathrm{EmptyClass}}
%\newcommand{\Dclass}[3]{\mathrm{Class}\ \mathrm{Extends}\ #1\ \mathrm{Inst}
%\ #2\ \mathrm{Meth}\ #3}
\newcommand{\Dclass}[2]{\mathrm{Class}\ \mathrm{Inst} \ #1\ \mathrm{Meth}\ #2}
\newcommand{\Dobj}[2]{\mathrm{Object}\ \mathrm{Inst}\ #1\ \mathrm{Meth}\ #2}
%\newcommand{\Dclassf}[3]{
%\begin{array}[t]{l}
%\mathrm{Class}\ \mathrm{Extends}\ #1 \\
%\quad \mathrm{Inst}\\
%\qquad #2 \\ 
%\quad \mathrm{Meth}\\
%\qquad #3
%\end{array}
%}
\newcommand{\Dclassf}[3]{
\begin{array}[t]{l}
\mathrm{Class}\\
\quad \mathrm{Inst}\\
\qquad #1 \\ 
\quad \mathrm{Meth}\\
\qquad #2
\end{array}
}
\newcommand{\Dobjf}[2]{
\begin{array}[t]{l}
\mathrm{Object}\\
\quad \mathrm{Inst}\\
\qquad #1 \\ 
\quad \mathrm{Meth}\\
\qquad #2
\end{array}
}
\newcommand{\Dnew}[1]{\mathrm{New}\ #1}
\newcommand{\vtab}[1]{\begin{verbatimtab}[4]#1\end{verbatimtab}}

\newcounter{topiccounter}
\setcounter{topiccounter}{1}
\newcommand{\topic}[1]
    {\noindent \textbf{Topic \arabic{topiccounter}.\ \textit{#1}. } \stepcounter{topiccounter}}


\newcommand{\lcalc}{$\lambda$-calculus}
\newcommand{\redx}{\rightarrow}
\newcommand{\redxs}{\redx^*}
\newcommand{\idfn}{\mathit{ID}}
\newcommand{\mlfn}[2]{\mathrm{fun}\, #1 \rightarrow #2}
\newcommand{\mlrecfn}[3]{\mathrm{fix}\,#1.#2 \rightarrow #3}
\newcommand{\mlfix}{\mathrm{fix}}
\newcommand{\eite}[3]{\mathrm{if}\ #1\ \mathrm{then} \ #2\ \mathrm{else} \ #3\ }
\newcommand{\esucc}[1]{\texttt{succ}\ #1}
\newcommand{\epred}[1]{\texttt{pred}\ #1}
\newcommand{\eiszero}[1]{\texttt{iszero}\ #1}
\newcommand{\ezero}{\texttt{0}}
\newcommand{\elet}[3]{\mathrm{let}\ #1 = #2\ \mathrm{in}\ #3}
\newcommand{\eletrec}[3]{\mathrm{letrec}\ #1 = #2\ \mathrm{in}\ #3}
\newcommand{\fv}{\mathrm{fv}}
\newcommand{\ourml}{\mathit{ML}_{\mathit{Cat}}}
\newcommand{\raisexn}{\mathrm{raise}}
\newcommand{\handler}[3]{\mathrm{try}\, #1\, \mathrm{with}\, \exn(#2) \Rightarrow #3}
\newcommand{\exn}{\mathit{exn}}
\newcommand{\dom}{\mathrm{dom}}
\newcommand{\efst}{\mathrm{fst}}
\newcommand{\esnd}{\mathrm{snd}}
\newcommand{\natt}{\textrm{Nat}}
\newcommand{\earray}{\mathrm{array}}
\newcommand{\varray}{\alpha}
\newcommand{\length}{\mathit{length}}
\newcommand{\arrayml}{\ourml^{\earray}}
\newcommand{\stackml}{\ourml^{\mathit{stack}}}
\newcommand{\flowml}{\ourml^{\mathit{flow}}}
\newcommand{\taintml}{\ourml^{\mathit{taint}}}
\newcommand{\secfail}{\mathbf{secfail}}
\newcommand{\tr}{\theta}
\newcommand{\rewrite}[1]{\mathcal{R}(#1)}
%\newcommand{\secprop}{\mathcal{P}}
%\newcommand{\trprop}{\hat{\secprop}}
\newcommand{\Prop}{\mathbf{P}}
\newcommand{\Hprop}{\mathbf{H}}
\newcommand{\secprop}{\phi}
\newcommand{\hyprop}{\eta}
\newcommand{\trprop}{\gamma}
\newcommand{\tracess}{\Sigma}
\newcommand{\trsprop}{\sigma}
\newcommand{\traces}{\Psi}
\newcommand{\fpkeyword}[1]{\mathrm{#1}}
\newcommand{\ebinop}[2]{#1\,\mathit{binop}\,#2}
\newcommand{\eenablepriv}[2]{\fpkeyword{enable}\ #1\ \fpkeyword{for}\ #2}
\newcommand{\echeckpriv}[2]{\fpkeyword{check}\ #1\ \fpkeyword{then}\ #2}
\newcommand{\esigned}[2]{#1.#2}
\newcommand{\enabled}{\mathit{enabledprivs}}
%\newcommand{\acl}{\mathcal{A}}
\newcommand{\priv}{\pi}
\newcommand{\privs}{\mathit{R}}
\newcommand{\prin}{p}
\newcommand{\nobody}{\mathit{nobody}}
\newcommand{\po}{\preceq}
\newcommand{\seclattice}{\mathcal{S}}
\newcommand{\binsl}{\mathcal{S}_{\mathrm{bin}}}
\newcommand{\seclevs}{\mathcal{L}}
\newcommand{\latel}{\varsigma}
\newcommand{\hilab}{\mathrm{High}}
\newcommand{\lolab}{\mathrm{Low}}
\newcommand{\hiloc}{\mathit{hi}}
\newcommand{\loloc}{\mathit{low}}
\newcommand{\labty}[2]{#1 \cdot #2}
\newcommand{\labval}[2]{#1 \cdot #2}
\newcommand{\mi}[1]{\mathit{#1}}
\newcommand{\pc}{\latel_{\mathit{pc}}}
\newcommand{\cfnty}[3]{#1 \rightarrow_{#2} #3}
\newcommand{\pow}{\mathrm{pow}}

%\newcommand{\tr}{\theta}


\newcommand{\chash}{\mathcal{H}}
\newcommand{\acl}{\mathit{Auth}}
\newcommand{\opn}{\mathit{op}}
\newcommand{\egid}{\mathit{egid}}
\newcommand{\euid}{\mathit{euid}}
\newcommand{\suid}{\ttt{suid}}
\newcommand{\sgid}{\ttt{sgid}}
\newcommand{\uxroot}{\ttt{root}}
\newcommand{\fowner}[1]{\mathit{owner}_{#1}}
\newcommand{\fgroup}[1]{\mathit{group}_{#1}}
\newcommand{\gprivs}[1]{\mathit{Privs_{#1}}.\mathit{group}}
\newcommand{\uprivs}[1]{\mathit{Privs_{#1}}.\mathit{owner}}
\newcommand{\oprivs}[1]{\mathit{Privs_{#1}}.\mathit{other}}
\newcommand{\uxprivs}[1]{\mathit{Privs_{#1}}}

\newcommand{\seclab}{\mathcal{L}}
\newcommand{\sle}{\preceq}
\newcommand{\ile}{\preceq_I}
\newcommand{\ilab}{\seclab_I}

\newcommand{\minifed}{\mathit{Overture}}
\newcommand{\minicat}{\minifed}
\newcommand{\fedprot}{\minifed}
\newcommand{\metaprot}{\mathit{Prelude}}
\newcommand{\mlscat}{\mathit{mlscat}}
\newcommand{\flowcat}{\mathit{flowcat}}
\newcommand{\dflowcat}{\mathit{dflowcat}}
\newcommand{\minicatde}{\mathit{minicat}_{\mathit{de}}}
\newcommand{\minicatexp}{\mathit{minicat}_{\mathit{taint}}}
%\newcommand{\prog}{\mathcal{P}}
\newcommand{\prog}{\pi}
\newcommand{\main}{\mathit{main}}
\renewcommand{\reval}{\redx}
\renewcommand{\Dite}{\eite}


%\renewcommand{\labty}[2]{#2}
\newcommand{\fnsty}{\Sigma}
\newcommand{\secty}{\latel}

\newcommand{\tc}{\mathrm{TC}}
\newcommand{\validate}{\mathrm{validate}}


\newcommand{\mlsid}[1]{\mathrm{mls}(#1)}
\newcommand{\mlsredx}[1]{\redx_{\mlsid{#1}}}
\newcommand{\confid}{\mathit{flow}}
\newcommand{\taintid}{\mathit{dflow}}
\newcommand{\credx}{\redx_{\confid}}
\newcommand{\tredx}{\redx_{\taintid}}
\newcommand{\ccod}[1]{\lcod{\confid}{#1}}
\newcommand{\tcod}[1]{\lcod{\taintid}{#1}}
\renewcommand{\mod}{\ \textrm{mod}\ }


\newcommand{\mtrace}[1]{\mathit{trace}_{#1}}
\newcommand{\mtraces}[1]{\mathit{traces}_{#1}}
\newcommand{\head}{\mathit{hd}}
\newcommand{\memt}{\mathit{mems}}

\newcommand{\bop}{\ \mathit{binop}\ }
\newcommand{\ak}{K}
\newcommand{\ik}{\mathit{kernel}}
%\newcommand{\deassign}[2]{\eassign{#1}{\mathrm{declassify}(#2)}
\newcommand{\deassign}[2]{#1 :=  [#2]_\wedge }
%\newcommand{\deassign}[2]{#1\ \wedge\!\,=  #2}

\newcommand{\mems}{\mathit{mems}}
\newcommand{\mto}{\mapsto}
\newcommand{\pdf}[1]{D_{#1}}
\newcommand{\margd}[2]{{#1}_{#2}}
\newcommand{\condd}[3]{#1_{({#2}|{#3})}}
\newcommand{\progd}{\mathrm{PD}}
\newcommand{\progtt}{\mathrm{BD}}
\newcommand{\vars}{\mathit{vars}}
\newcommand{\iov}{\mathit{iovars}}
\newcommand{\flips}{\mathit{flips}}
\newcommand{\keys}{\mathit{keys}}
\newcommand{\fedcat}{\minifed}

%\newcommand{\sx}[2]{\texttt{s[#1,"#2"]}}
%\newcommand{\fx}[2]{\texttt{f[#1,"#2"]}}
%\newcommand{\vx}[2]{\texttt{v[#1,"#2"]}}

\newcommand{\IF}[1]{#1_{\mathit{i}}}
\newcommand{\idealf}{\mathcal{F}}
\newcommand{\SIM}{\mathrm{Sim}}
\newcommand{\prob}{\mathrm{Pr}}
\newcommand{\dist}{\mathrm{D}}

\def\TirName#1{\text{\sc #1}}

\newcommand{\srct}{\tau}
\newcommand{\cidty}[1]{\ttt{cid(}#1\ttt{)}}
\newcommand{\stringty}[1]{\ttt{string(}#1\ttt{)}}
\newcommand{\unity}{\mathtt{unit}}
\newcommand{\jpdty}[2]{\mathtt{jpd}(#1,#2)}
\newcommand{\viewst}{\mathcal{V}}
\newcommand{\tjudge}[5]{#1, #2 \vdash #3 : #4,#5}
\newcommand{\bet}[1]{\ttt{bool[}#1\ttt{]}}
\newcommand{\tas}{\mathcal{A}}


\newcommand{\flip}[2]{\ttt{flip[}#1\ttt{,}#2\ttt{]}}
\newcommand{\secret}[2]{\ttt{s[}#1\ttt{,}#2\ttt{]}}
\newcommand{\view}[2]{\ttt{v[}#1\ttt{,}#2\ttt{]}}
\newcommand{\oracle}[1]{\ttt{H[}#1\ttt{]}}
\newcommand{\Oracle}{H}
\renewcommand{\etrue}{\ttt{true}}
\renewcommand{\efalse}{\ttt{false}}
\newcommand{\enot}{\ttt{not}}
\newcommand{\eand}{\ttt{and}}
\newcommand{\eor}{\ttt{or}}
\newcommand{\exor}{\ttt{xor}}
\renewcommand{\elet}[3]{\ttt{let}\ #1\ \ttt{=}\ #2\ \ttt{in}\ #3}
\newcommand{\vloc}[2]{#1@#2}
\renewcommand{\redx}{\xrightarrow{}}
\renewcommand{\redxs}{\xrightarrow{}^{*}}
\newcommand{\lredx}[1]{\xrightarrow{#1}}
\newcommand{\mem}{M}
\newcommand{\randos}{R}
\newcommand{\tape}{\randos}
\newcommand{\secrets}{\mathit{secrets}}
\newcommand{\outputs}{\mathit{outputs}}
\newcommand{\clients}{C}
\newcommand{\views}{\mathit{views}}
\newcommand{\str}{\varsigma}
\newcommand{\cid}{\iota}
\newcommand{\send}[2]{#1\ \ttt{:=}\ #2}
\newcommand{\msend}[4]{\elab{\mesg{#1}}{#2}\ \ttt{:=}\ \elab{#3}{#4}}
\newcommand{\OT}[3]{\ttt{OT(} #1 \ttt{,}\ #2 \ttt{,}\ #3 \ttt{)}}
\newcommand{\select}[3]{\ttt{select(} #1 \ttt{,}\ #2 \ttt{,}\ #3 \ttt{)}}
\newcommand{\mux}[3]{\ttt{mux(} #1 \ttt{,}\ #2 \ttt{,}\ #3 \ttt{)}}
\newcommand{\codebase}{\mathcal{C}}
\newcommand{\interp}[1]{\llbracket #1 \rrbracket}
\newcommand{\finterp}[2]{\llbracket #1 \rrbracket_{#2}}
\newcommand{\prot}{\rho}
\newcommand{\Tapes}{\mathcal{R}}
\newcommand{\outloc}{\mathit{output}}
\newcommand{\pdist}{\mathit{pd}}
\newcommand{\genpdf}{\mathrm{PD}}
\newcommand{\card}[1]{|#1|}
\newcommand{\setdefn}[2]{\{#1\ |\ #2 \}}
\newcommand{\tapes}{\mathit{tapes}}
\newcommand{\nimo}{\mathit{NIMO}}
\newcommand{\pni}{\mathit{PNI}}
\newcommand{\passec}{PS}
\newcommand{\parties}{\mathcal{P}}
\newcommand{\iout}{\mathit{output}}
\newcommand{\kideal}{k_i}
\newcommand{\jpdf}{\mathrm{pdf}}
\newcommand{\leakproof}{\mathit{LP}}
\newcommand{\flab}{\ell}
\newcommand{\be}{\varepsilon}
\newcommand{\instr}{\mathbf{c}}
\newcommand{\solvealg}{\mathit{models}}
\newcommand{\solve}[3]{\solvealg\ #1\ #2\ #3}
\newcommand{\itv}{\mathit{it}}
\newcommand{\outv}{\mathit{out}}
\newcommand{\NIMO}{\mathit{NIMO}}
\newcommand{\gNIMO}{\mathit{gNIMO}}
\newcommand{\gates}{\mathit{gates}}
\newcommand{\owl}{\mathit{owl}}
\newcommand{\logit}[1]{\lfloor #1 \rfloor}
\newcommand{\runs}{\mathit{runs}}
\newcommand{\cruns}{\hat{\mathit{runs}}}
\newcommand{\cprogd}{\hat{\progd}}
\newcommand{\cprogtt}{\hat{\progtt}}
\newcommand{\datalog}{\mathit{datalog}}
%\newcommand{\concat}{\ttt{|\!|}}
\newcommand{\concat}{\ttt{++}}
\newcommand{\wired}{\mathit{wired}}
\newcommand{\gc}[3]{\mathit{goc}(#1,#2,#3)}
\newcommand{\vc}[3]{#1 \vdash #2 \sim #3}
\newcommand{\detx}[1]{\mathbf{D}(#1)}
\newcommand{\unix}[1]{\mathbf{U}(#1)}
\newcommand{\sep}[3]{#1 \vdash #2 * #3}
\newcommand{\condp}[3]{#1|#2 \vdash #3}
\newcommand{\conddetx}[3]{\condp{#1}{#2}{\detx{#3}}}
\newcommand{\condsep}[4]{\condp{#1}{#2}{#3 * #4}}
\newcommand{\condunix}[3]{\condp{#1}{#2}{\unix{#3}}}
\newcommand{\gtab}{\mathit{table}}
\newcommand{\vdefs}{\mathit{vdefs}}
\newcommand{\funcVar}{\$}
%\newcommand{\pmf}{\mathrm{Pr}}
\newcommand{\pmf}{\mathit{P}}

%%%% REVISION DEFS

\renewcommand{\flip}[1]{\ttt{r[}#1\ttt{]}}
\newcommand{\locflip}{\ttt{r[}\mathtt{local}\ttt{]}}
\renewcommand{\secret}[1]{\ttt{s[}#1\ttt{]}}
\newcommand{\key}[1]{\ttt{k[}#1\ttt{]}}
\newcommand{\mesg}[1]{\ttt{m[}#1\ttt{]}}
\newcommand{\outkw}{\ttt{out}}
\newcommand{\out}[1]{\elab{\outkw}{#1}}
\newcommand{\rvl}[1]{\ttt{p[}#1\ttt{]}}
\renewcommand{\oracle}[1]{\ttt{H[}#1\ttt{]}}
%\newcommand{\elab}[2]{#1_{#2}}
\newcommand{\elab}[2]{#1\ttt{@}#2}
\newcommand{\eassign}[4]{\elab{#1}{#2} := \elab{#3}{#4}}
\newcommand{\xassign}[3]{#1 := \elab{#2}{#3}}
\newcommand{\pubout}[3]{\out{#1} := \elab{#2}{#3}}
\newcommand{\reveal}[3]{\rvl{#1} := \elab{#2}{#3}}
\newcommand{\sk}[1]{\mathrm{sk}[#1]}
\newcommand{\pk}[2]{\mathrm{pk}[#1,#2]}
\newcommand{\kgen}[1]{\mathit{kgen}(#1)}
\newcommand{\adversary}{\mathcal{A}}
\newcommand{\aredx}{\redx_{\adversary}}
\newcommand{\aredxs}{\redxs_{\adversary}}
\newcommand{\arewrite}{\mathit{rewrite}_{\adversary}}
\newcommand{\cinputs}{V_{C \rhd H}}
\newcommand{\houtputs}{V_{H \rhd C}}
\newcommand{\aruns}{\mathit{runs}_\adversary}
\newcommand{\botruns}{\mathit{runs}_{\adversary,\bot}}
\newcommand{\att}{\mathrm{AD}}
\newcommand{\support}{\mathit{support}}
\renewcommand{\store}{\sigma}
\newcommand{\ctxt}[2]{\{ #1 \}_{#2}}
\newcommand{\cpub}{\mathit{pub}}
\renewcommand{\runs}{\mathit{runs}}
\newcommand{\pattern}[1]{\lfloor #1 \rfloor}
\newcommand{\fcod}[1]{\lcod{#1}{}}
\renewcommand{\flips}{\mathit{rands}}
\newcommand{\kmat}{\kappa}
\renewcommand{\Oracle}{\mathbb{O}}
\newcommand{\afilter}{\mathit{afilter}}
%\renewcommand{\select}[3]{\mathtt{if}\ #1\ \mathtt{then}\ #2\ \mathtt{else}\ #3}
\newcommand{\fp}{\mathit{P}}
\newcommand{\ftimes}{*}
\newcommand{\fplus}{+}
\newcommand{\fminus}{-}
\newcommand{\mactimes}{\,\hat{\ftimes}\,}%{\otimes}
\newcommand{\macplus}{\,\hat{\fplus}\,}%\oplus}
\newcommand{\macminus}{\,\hat{\fminus}\,}%{\ominus}
\newcommand{\macgv}[1]{\langle #1 \rangle}
\newcommand{\macv}{\hat{v}}
\newcommand{\macx}[2]{\macgv{\elab{ #1 }{#2}}}
\newcommand{\mack}[2]{#1.\ttt{k}_{#2}}
\newcommand{\macshare}[1]{\langle #1 \rangle.\ttt{share}}
\newcommand{\macopen}{\mathrm{open}}
\newcommand{\macauth}{\mathrm{auth}}
\newcommand{\fieldty}{\mathrm{F}}
\newcommand{\cipherty}{\mathit{c}}
\newcommand{\macty}{\hat{\fieldty}}%_{\mathit{mac}}}}
\renewcommand{\unity}[1]{\mathit{U}(#1)}
\renewcommand{\labty}[3]{#1^{#2}_{#3}}
\newcommand{\memenv}{\mathcal{M}}
\newcommand{\tensor}{\multimap}
\newcommand{\lib}{\mathcal{L}}
\newcommand{\okt}{\mathit{OK}}
\newcommand{\vty}{t}
\newcommand{\disty}{\dot{\vty}}
\newcommand{\tlev}[1]{\mathcal{T}(#1)}
\newcommand{\otp}{\mathrm{sum}}
\newcommand{\macotp}{\hat{\mathrm{minus}}}
\newcommand{\preproc}{\mathit{preproc}}
\newcommand{\assert}[1]{\ttt{assert(}#1\ttt{)}}
\newcommand{\mv}{\nu}
\newcommand{\andgmw}{\ttt{andgmw}}
\newcommand{\decodegmw}{\ttt{decodegmw}}
\newcommand{\bodies}{\mathit{bodies}}

\newcommand{\sx}[2]{\elab{\secret{#1}}{#2}}
\newcommand{\mx}[2]{\elab{\mesg{#1}}{#2}} 
\newcommand{\px}[1]{\rvl{#1}} 
\newcommand{\rx}[2]{\elab{\flip{#1}}{#2}}
\newcommand{\ox}[2]{\elab{\out{#1}}{#2}}
\newcommand{\eqcast}[2]{#1\ \ttt{as}\ #2}
\newcommand{\signals}[1]{\stackrel{#1}{\leadsto}}

\newcommand{\tj}[6]{#1,#2,#3 \vdash_{#4} #5 : #6}
\newcommand{\itj}[3]{\vdash_{#1} #2 : #3}
\newcommand{\ipj}[3]{#1 \vdash #2 : #3}
\newcommand{\ej}[6]{#1,#2,#3 \vdash #4 : #5,#6}
\newcommand{\cty}[2]{c(#1,#2)}
\newcommand{\setit}[1]{\{ #1 \}}
\newcommand{\ty}{T}
\newcommand{\ity}[2]{#1 \cdot #2}
\newcommand{\lty}[2]{#1 \cdot #2}
\newcommand{\eqs}{\mathit{E}}
\newcommand{\toeq}[1]{\lfloor #1 \rfloor}
\newcommand{\eop}{\equiv}
\newcommand{\autheq}[1]{\phi_{\mathrm{auth}}(#1)}
\newcommand{\upgrade}[1]{\uparrow #1}
\newcommand{\seclev}{\mathcal{L}}

\newcommand{\tsig}{\mathrm{sig}}
\newcommand{\subn}{\rho}
\newcommand{\hty}[5]{\{ #1 \}\ #2,#3 \cdot #4\  \{ #5 \} }
\newcommand{\dht}[6]{\Pi #1 . \hty{#2}{#3}{#4}{#5}{#6}}
\newcommand{\mtj}[6]{\vdash #1 : \hty{#2}{#3}{#4}{#5}{#6}}
\newcommand{\atj}[3]{\Vdash #1 : (#2,#3)}
\newcommand{\eqj}[4]{#1,#2 \vdash #3 : #4}
\newcommand{\cpj}[4]{#1,#2 \vdash #3 : #4}
\newcommand{\leakj}[3]{#1,#2 \vdash_{\mathit{leak}} #3}
\newcommand{\cheatj}[3]{#1 \underset{#2}{\leadsto} #3}
\newcommand{\icod}[2]{\cod{#1}_{#2}}
\newcommand{\prejoin}{\sqcap}

\renewcommand{\redx}{\Rightarrow}
\renewcommand{\redxs}{\redx}
\newcommand{\abort}{\bot}
\newcommand{\pre}[1]{\ttt{pre}(#1)}
\newcommand{\post}[1]{\ttt{post}(#1)}
\newcommand{\eqflag}{\mathit{sw}}
\newcommand{\eqon}{\ttt{on}}
\newcommand{\eqoff}{\ttt{off}}
\newcommand{\eqtrans}[1]{\lfloor #1 \rfloor}
\newcommand{\mc}[4]{(#1,#2,#3,#4)}
\newcommand{\cmd}{\instr}
\renewcommand{\OT}[4]{\ttt{OT(}\elab{#1}{#2},#3,#4 \ttt{)}}
\newcommand{\eqspre}{\eqs_{\mathit{pre}}}
\newcommand{\macbdoz}[1]{\psi_{\mathit{BDOZ}}(#1)}
\newcommand{\initigamma}{\mathit{init}}
\newcommand{\notg}[1]{\breve{#1}}

\long\def\cnote#1{{\small\textbf{\textit{\color{red}(*#1 -- Chris*)}}}}
\long\def\jnote#1{{\small\textbf{\textit{\color{brown}(*#1 -- Joe*)}}}}

\newcommand{\mynote}[2]
    {{\color{red} \fbox{\bfseries\sffamily\scriptsize#1}
    {\small$\blacktriangleright$\textsf{\emph{#2}}$\blacktriangleleft$}}~}
%\newcommand{\mynote}[2]{}
\newcommand{\todo}[1]{\mynote{TODO}{#1}}
    
%\newcommand{\note}[1]{\noindent\textit{(\textbf{$\star$note$\star$:}\ \ #1)}}


\newcommand{\compwrapfig}
{
  \begin{wrapfigure}{r}{0cm}
    \begin{tabular}{lccccccc}
      & 
      \begin{sideways} probabilistic language \end{sideways} &
      \begin{sideways} probabilistic conditioning \end{sideways} & 
      \begin{sideways} low-level protocols \end{sideways} & 
      \begin{sideways} passive security \end{sideways} & 
      \begin{sideways} malicious security \end{sideways}& 
      \begin{sideways} hyperproperties \end{sideways}& 
      \begin{sideways} automation \end{sideways}\\\hline\hline
      Haskell EDSL \cite{6266151} & \checkmark &  & \checkmark  & \checkmark & & \checkmark & \checkmark \\\hline
      MPC in SecreC \cite{almeida2018enforcing} & \checkmark & \checkmark &   & \checkmark & & \checkmark & \checkmark \\\hline
      $\lambda_{\text{obliv}}$ \cite{darais2019language} & \checkmark & & \checkmark & & & \checkmark & \checkmark \\\hline
      PSL \cite{barthe2019probabilistic} & \checkmark & & \checkmark & & & & \\\hline
      Lilac \cite{li2023lilac} & \checkmark & \checkmark & & & & & \\\hline
      Wys$^*$ \cite{wysstar} & & & & \checkmark & & \checkmark & \checkmark\\\hline
      Viaduct \cite{10.1145/3453483.3454074,viaduct-UC} & & & & \checkmark & \checkmark & \checkmark & \checkmark\\\hline
      MPC in EasyCrypt \cite{8429300} &  \checkmark &  \checkmark &  \checkmark & \checkmark & \checkmark & \checkmark & \\\hline
      $\metaprot$/$\minifed$ \cite{skalka-near-ppdp24} & \checkmark & \checkmark & \checkmark & \checkmark & \checkmark & \checkmark & * \\\hline
      This work & \checkmark & \checkmark & \checkmark & \checkmark & \checkmark & \checkmark & \checkmark\\
      \hline
    \end{tabular}
    \caption{Comparison of systems for verification of MPC security in PLs. * indicates limited support for automation.}
    \label{fig-comp-wrap}
  \end{wrapfigure}
}

\newcommand{\minicatsyntaxfig}{
\begin{fpfig}[t]{Syntax of $\minicat$}{fig-minicat-syntax}
\small{
$$
    \begin{array}{rcl@{\hspace{4mm}}r}
      \multicolumn{4}{l}{v \in \mathbb{F}_p,\ w \in \mathrm{String},\ \cid \in \mathrm{Clients} \subset  \mathbb{N} }\\[2mm] %, \bop \in \{ \eand, \eor, \exor \}} \\[2mm]
      \be &::=& v \mid \flip{w} \mid \secret{w} \mid \mesg{w} \mid \rvl{w} \mid \be \fminus \be \mid \be \fplus \be \mid \be \ftimes \be \mid \OT{\be}{\cid}{\be}{\be} & \textit{expressions}\\[1mm]
      x &::=& \elab{\flip{w}}{\cid} \mid \elab{\secret{w}}{\cid} \mid \elab{\mesg{w}}{\cid} \mid  \rvl{w} \mid \out{\cid} & \textit{variables} \\[1mm]
      \prog &::=& \eassign{\mesg{w}}{\cid}{\be}{\cid} \mid \reveal{w}{\be}{\cid} \mid \pubout{\cid}{\be}{\cid} \mid \prog;\prog & \textit{protocols}
    \end{array}
$$}
\end{fpfig}    
}

\newcommand{\minicatredxfig}{
\begin{fpfig}[t]{Semantics of $\minicat$ expressions (T) and programs (B).}{fig-minicat-redx}
\small{
 $$
  \begin{array}{c@{\hspace{5mm}}c}
  \begin{array}{rcl}
    \lcod{\store, v}{\cid} &=& v\\
    \lcod{\store, \be_1 \fplus \be_2}{\cid} &=& \fcod{\lcod{\store, \be_1}{\cid} \fplus \lcod{\store, \be_2}{\cid}}\\ 
    \lcod{\store, \be_1 \fminus \be_2}{\cid} &=& \fcod{\lcod{\store, \be_1}{\cid} \fminus \lcod{\store, \be_2}{\cid}}\\ 
    \lcod{\store, \be_1 \ftimes \be_2}{\cid} &=& \fcod{\lcod{\store, \be_1}{\cid} \ftimes \lcod{\store, \be_2}{\cid}}\\
  \end{array} & 
  \begin{array}{rcl}
    \lcod{\store, \flip{w}}{\cid} &=& \store(\elab{\flip{w}}{\cid})\\
    \lcod{\store, \secret{w}}{\cid} &=& \store(\elab{\secret{w}}{\cid})\\
    \lcod{\store, \mesg{w}}{\cid} &=& \store(\elab{\mesg{w}}{\cid})\\
    \lcod{\store, \rvl{w}}{\cid} &=& \store(\rvl{w})\\
    %\lcod{\store, \OT{\be_1}{\cid_1}{\be_2}{\be_3}}{\cid_2} &=&
    %\begin{cases}
    %  \lcod{\store,\be_2}{\cid_2} \text{\ if\ } \lcod{\store,\be_1}{\cid_1} = 0 \\
    %  \lcod{\store,\be_3}{\cid_2} \text{\ if\ } \lcod{\store,\be_1}{\cid_1} = 1 \\
    %\end{cases}
  \end{array}
  \end{array}
  $$

  %$$
  %\lcod{\store, \OT{\be_1}{\cid_1}{\be_2}{\be_3}}{\cid_2} =
  %  \begin{cases}
  %    \lcod{\store,\be_2}{\cid_2} \text{\ if\ } \lcod{\store,\be_1}{\cid_1} = 0 \\
  %    \lcod{\store,\be_3}{\cid_2} \text{\ if\ } \lcod{\store,\be_1}{\cid_1} = 1 \\
  %  \end{cases}
  %$$
  %
\begin{mathpar}
  (\store, \xassign{x}{\be}{\cid}) \redx \extend{\store}{x}{\lcod{\store,\be}{\cid}}
  
  \inferrule
      {(\store_1,\prog_1) \redx \store_2 \\ (\store_2,\prog_2) \redx \store_3 }
      {(\store_1,\prog_1;\prog_2) \redx \store_3}
      %(\store, \eassign{\mesg{w}}{\cid_1}{\be}{\cid_2};\prog) \redx (\extend{\store}{\mesg{w}_{\cid_1}}{\lcod{\store,\be}{\cid_2}}, \prog)    
      %(\store, \reveal{w}{\be}{\cid};\prog) \redx (\extend{\store}{\rvl{w}}{\lcod{\store,\be}{\cid}}, \prog)   
      %(\store, \pubout{\cid}{\be}{\cid};\prog) \redx (\extend{\store}{\out{\cid}}{\lcod{\store,\be}{\cid}}, \prog)
\end{mathpar}
}
\end{fpfig}
}

\newcommand{\minicataredxfig}{
\begin{fpfig}[t]{Adversarial semantics of $\minicat$.}{fig-minicat-aredx}
\small{
\begin{mathpar}
  \inferrule
      { \cid \in H }
      { (\store, \xassign{x}{\be}{\cid}) \aredx \extend{\store}{x}{\lcod{\store,\be}{\cid}} }
      
  \inferrule
      {\cid \in C }
      { (\store, \xassign{x}{\be}{\cid}) \aredx \extend{\store}{x}{\lcod{\arewrite(\store_C,\be)}{\cid}}}
      
  \inferrule
      {\lcod{\store,\be_1}{\cid} = \lcod{\store,\be_2}{\cid}  \text{\ or\ } \cid \in C}
      { (\store,\elab{\assert{\be_1 = \be_2}}{\cid}) \aredx \store }
      
  \inferrule
      {\lcod{\store,\be_1}{\cid} \ne \lcod{\store,\be_2}{\cid}}
      {(\store,\elab{\assert{\be_1 = \be_2}}{\cid}) \aredx \abort}
  
  \inferrule
      {(\store_1,\prog_1) \aredx \store_2 \\ (\store_2,\prog_2) \aredx \store_3 }
      {(\store_1,\prog_1;\prog_2) \aredx \store_3}

  \inferrule
      {(\store_1,\prog_1) \aredx \abort}
      {(\store_1,\prog_1;\prog_2) \aredx \abort}
      
  \inferrule
      {(\store_1,\prog_1) \aredx \store_2 \\ (\store_2,\prog_2) \aredx \abort }
      {(\store_1,\prog_1;\prog_2) \aredx \store_2}
\end{mathpar}}
\end{fpfig}
}

\newcommand{\cpjfig}{
\begin{fpfig}[t]{Syntax and Derivation Rules for $\minicat$ Confidentiality Types}{fig-cpj}
\small{
$$
\begin{array}{rcl@{\hspace{3mm}}l}
  t &::=& x \mid \cty{x}{T} \\
  \ty & \in & 2^{t} & \gdesc{confidentiality types}\\
  \Gamma &::=& \varnothing \mid \Gamma; x : \ty & \gdesc{confidentiality type environments}
\end{array} 
$$
\medskip
\begin{mathpar}
  \inferrule[DepTy]
  {}
  {\eqj{\varnothing}{\eqs}{\phi}{\vars(\phi)}}
  
  \inferrule[Encode]
  {\eqs \models \phi \eop \phi' \oplus \rx{w}{\cid} \\
   \oplus \in \{ \fplus,\fminus \}\\
   \eqj{R}{\eqs}{\phi'}{\ty}}
  {\eqj{R;\{ \rx{w}{\cid} \}}{\eqs}{\phi}{\setit{\cty{\rx{w}{\cid}}{\ty}}}}
\end{mathpar}

\begin{mathpar}
  \inferrule[Send]
            {\eqj{R}{\eqs}{\phi}{\ty}}
            {\cpj{R}{\eqs}{x \eop \phi}{(x : \ty)}}
            
  \inferrule[Seq]
            {\cpj{R_1}{\eqs}{\phi_1}{\Gamma_1}\\
             \cpj{R_2}{\eqs}{\phi_2}{\Gamma_2}}
            {\cpj{R_1;R_2}{\eqs}{\phi_1 \wedge \phi_2}{\Gamma_1;\Gamma_2}}
\end{mathpar}
}
\end{fpfig}
}

\newcommand{\leakjfig}{
\begin{fpfig}[t]{Dependencies in Views: Derivation Rules}{fig-leakj}
\small{
\begin{mathpar}
  \inferrule
      {\leakclose{\Gamma}{\ty_1}{\ty_2} \\ \leakclose{\Gamma}{\ty_2}{\ty_3}}
      {\leakclose{\Gamma}{\ty_1}{\ty_3}}

      \leakclose{\Gamma}{\ty \cup \setit{\mx{w}{\cid}}}{\ty \cup \Gamma(\mx{w}{\cid})}

      \leakclose{\Gamma}{\ty_1 \cup \setit{x, \cty{x}{\ty_2}}}{\ty_1\cup\ty_2}
\end{mathpar}

\begin{mathpar}
  \inferrule
      {}
      {\leakj{\Gamma}{\varnothing}{\varnothing}}

\inferrule
      {\leakj{\Gamma}{M}{\ty'} \\ \leakclose{\Gamma}{\ty'\cup\setit{x}}{\ty}}
      {\leakj{\Gamma}{M \cup \setit{x}}{\ty}}
\end{mathpar}
}
\end{fpfig}
}

\newcommand{\ipjfig}{
\begin{fpfig}[t]{Syntax and derivation rule of $\minicat$ integrity types}{fig-ipj}
\small{
$$
\begin{array}{rcl@{\hspace{4mm}}l}
  \latel &::=& \hilab \mid \lolab & \gdesc{integrity labels} \\
  \Delta &::=& \varnothing \mid \Delta; x : \ity{\cid}{V} & \gdesc{integrity type environments}
\end{array} 
$$

\begin{mathpar}
  \inferrule[Value]
  {}
  {\itj{\cid}{v}{\varnothing}}
  
  \inferrule[Secret]
  {}
  {\itj{\cid}{\secret{w}}{\varnothing}}
  
  \inferrule[Rando]
  {}
  {\itj{\cid}{\flip{w}}{\varnothing}}
  
  \inferrule[Mesg]
  {}
  {\itj{\cid}{\mesg{w}}{\setit{\mx{w}{\cid}}}}
    
  \inferrule[PubM]
  {}
  {\itj{\cid}{\rvl{w}}{\setit{\rvl{w}}}}

  \inferrule[Binop]
  {\itj{\cid}{\be_1}{V_1} \\
   \itj{\cid}{\be_2}{V_2} \\ \oplus \in \{ \fplus,\fminus,\ftimes \}}
  {\itj{\cid}{\be_1 \oplus \be_2}{V_1 \cup V_2}}
%
%  \inferrule[IntegrityWeaken]
%  {\itj{\Delta}{\eqs}{\cid}{\be}{\latel_1} \\ \latel_1 \sle \latel_2}
%  {\itj{\Delta}{\eqs}{\cid}{\be}{\latel_2}}
\end{mathpar}

\begin{mathpar}
  \inferrule[Send]
            {\itj{\cid}{\be}{V}}
            {\ipj{\eqs}{\xassign{x}{\be}{\cid}}{(x : \ity{\cid}{V})}}
             
%  \inferrule[Assert]
%            {\eqs \models \toeq{\elab{\be_1}{\cid}} = \toeq{\elab{\be_2}{\cid}}}
%            {\ej{\Delta}{R}{\eqs}{\elab{\assert{\be_1 = \be_2}}{\cid}}{\Delta}{\eqs}}
%            
  \inferrule[Seq]
            {\ipj{\eqs}{\prog_1}{\Delta_1}\\
             \ipj{\eqs}{\prog_2}{\Delta_2}}
            {\ipj{\eqs}{\prog_1;\prog_2}{\Delta_1;\Delta_2}}

  \inferrule[MAC]
            {\eqs \models \toeq{\elab{\assert{\macbdoz{w}}}{\cid}}}
            {\ipj{\eqs}{\elab{\assert{\macbdoz{w}}}{\cid}}{(\mx{w\ttt{s}}{\cid}: \ity{\cid}{\varnothing})}}
%
%  \inferrule[MAC]
%            {\eqs \models 
%              \mx{w\ttt{m}}{\cid} \eop \mx{w\ttt{k}}{\cid} \fplus \ttt{(}\mx{\ttt{delta}}{\cid} \ftimes
%                  \mx{w\ttt{s}}{\cid}\ttt{)}}
%            {\ipj{\Delta}{\eqs}{
%                \elab{\assert{\mesg{w\ttt{m}} \eop \mesg{w\ttt{k}} \fplus \ttt{(}\mesg{\ttt{delta}} \ftimes
%                  \mesg{w\ttt{s}}\ttt{)}}}{\cid}}{\Delta;\mx{w\ttt{s}}{\cid}: \hilab }}
\end{mathpar}
}
\end{fpfig}
}

\newcommand{\cheatjfig}{
\begin{fpfig}[t]{Assigning integrity labels to variables}{fig-cheatj}
\small{
\begin{mathpar}
  \inferrule
      {}
      {\cheatj{\varnothing}{H,C}{\seclev_{H,C}}}
      
  \inferrule
      {\cheatj{\Delta}{H,C}{\seclev} \\ \cid \in H}
      {\cheatj{\Delta; x : \ity{\cid}{V}}{H,C}{\extend{\seclev}{x}{\hilab \wedge (\bigwedge_{x \in V} \seclev_2(x))}}}
      
  \inferrule
      {\cheatj{\Delta}{H,C}{\seclev} \\ \cid \in C}
      {\cheatj{\Delta; x : \ity{\cid}{V}}{H,C}{\extend{\seclev}{x}{\lolab}}}
\end{mathpar}
}
\end{fpfig}
}

\newcommand{\metaprotsyntaxfig}{
  \begin{fpfig}[t]{$\metaprot$ syntax.}{fig-metaprotsyntax}
\small{
$$
\begin{array}{rcl@{\hspace{4mm}}r}
  %\notg{x} &::=& \elab{\flip{e}}{e} \mid \elab{\secret{e}}{e} \mid \elab{\mesg{e}}{e} \mid \rvl{e} \mid \out{e}\\[2mm]
  \multicolumn{3}{l}{\flab \in \mathrm{Field},\   y \in \mathrm{EVar}, \  f \in \mathrm{FName}}\\[1mm]
  %x &\in& \mathrm{EVar}\\
  %f &\in& \mathrm{FName}\\[2mm]
  e &::=& \mv \mid \flip{e} \mid \secret{e} \mid \mesg{e} \mid \rvl{e} \mid \outkw \mid e \bop e 
   \mid y \mid & \gdesc{expressions}\\
  & & e.\flab \mid \elab{e}{e} \mid \elet{y}{e}{e} \mid  f(e,\ldots,e) \mid \{ \flab = e; \ldots; \flab = e \} \\[1mm]
  %  & \textit{expressions}\\
  \cmd &::=& %\msend{e}{e}{e}{e} \mid \reveal{e}{e}{e} \mid \pubout{e}{e}{e} \mid
  \assign{e}{e} \mid f(e,\ldots,e) \mid
  \elet{y}{e}{\instr} \mid  \cmd;\cmd & \gdesc{instructions}\\
         & & \elab{\assert{e = e}}{e} \mid \eqcast{\mx{e}{e}}{\notg{\phi}} \\[1mm] %\pre{\eqs} \mid \post{\eqs} \\[1mm]
  \bop &::=& \fplus \mid \fminus \mid \ftimes \mid \concat  \\[1mm]% \textit{operators}\\[2mm]
  \mv &::=& w \mid \cid \mid \be \mid x \mid \{ \flab = \mv;\ldots;\flab = \mv \} 
  & \gdesc{values}\\[1mm] % \mid \ttt{()} \\[1mm] %& \textit{values}\\[2mm]
  \mathit{fn} &::=& f(y,\ldots,y) \{ e \} \mid  f(y,\ldots,y) \{ \cmd \} & \textit{functions}
  %\phi &::=& \elab{\flip{e}}{e} \mid \elab{\secret{e}}{e} \mid \elab{\mesg{e}}{e} \mid \rvl{e} \mid \out{e} \mid \phi \fplus \phi \mid \phi \fminus \phi \mid \phi \ftimes \phi \\
  %\eqs &::=& \phi \eop \phi \mid \eqs \wedge \eqs 
\end{array}
$$
}
\end{fpfig}
}

\newcommand{\metaprotexprsemanticsfig}{
  \begin{fpfig}[t]{Semantics of $\metaprot$ expressions.}{fig-metaprotexprsemantics}
\small{
  \begin{mathpar}
  \inferrule
      {e_1 \redx \mv \\ e_2[\mv/y] \redx \mv'}
      {\elet{y}{e_1}{e_2} \redx \mv'}
      
  \inferrule
      {e_1 \redx \be \\ e_2 \redx \cid}
      {\elab{e_1}{e_2} \redx \elab{\be}{\cid}}

  \inferrule
      {\codebase(f) = y_1,\ldots,y_n,\ e \\ e_1 \redx \mv_1 \cdots e_n \redx \mv_n \\
        e[\mv_1/y_1]\cdots[\mv_n/y_n] \redx \mv}
      {f(e_1,\ldots,e_n) \redx \mv}

  \inferrule
      {e_1 \redx \mv_1 \\ \cdots \\ e_n \redx \mv_n }
      {\{ \flab_1 = e_1; \ldots; \flab_n = e_n \} \redx \{ \flab_1 = \mv_1; \ldots; \flab_n = \mv_n \} }

  \inferrule
      {e \redx \{\ldots; \flab = \mv; \ldots\}}
      {e.\flab \redx \mv}

  \inferrule
      {e_1 \redx w_1 \\ e_2 \redx w_2}
      {e_1 \concat e_2 \redx w_1w_2}

  \inferrule
      {e \redx w}
      {\mesg{e} \redx \mesg{w}}
      
  \inferrule
      {e_1 \redx \be_1 \\ e_2 \redx \be_2}     
      {e_1 \fplus e_2 \redx \be_1 \fplus \be_2}
\end{mathpar}
}
\end{fpfig}
}

\newcommand{\metaprotinstrsemanticsfig}{
  \begin{fpfig}[t]{Semantics of $\metaprot$ instructions.}{fig-metaprotinstrsemantics}
%\small{
\begin{mathpar}
  \inferrule
      {e_1 \redx x \\ e_2 \redx \elab{\be}{\cid}}
      {\assign{e_1}{e_2} \redx \xassign{x}{\be}{\cid}}

  \inferrule
      {e_1 \redx \be_1 \\ e_2 \redx \be_2 \\ e_3 \redx \cid}
      {\elab{\assert{e_1 = e_2}}{e_3} \redx \elab{\assert{\be_1 = \be_2}}{\cid}}

  \inferrule
      {\codebase(f) = y_1,\ldots,y_n, \instr \\e_1 \redx \mv_1 \ \cdots \ e_n \redx \mv_n \\
        \instr[\mv_1/y_1]\cdots[\mv_n/y_n] \redx \prog
      }
      {f(e_1,\ldots,e_n) \redx \prog}
      
  \inferrule
      {e \redx \mv \\ \instr[\mv/y] \redx \prog}
      {\elet{y}{e}{\instr} \redx \prog}

  \inferrule
      {e_1 \redx \prog_1 \\ e_2 \redx \prog_2}
      {e_1;e_2 \redx \prog_1;\prog_2}
\end{mathpar}
%}
\end{fpfig}
}

\newcommand{\atjfig}{
  \begin{fpfig}[t]{Algorithmic type judgements for $\minicat$.}{fig-atj}
\small{
\begin{mathpar}
  \atj{x}{\varnothing}{\setit{x}}

  \inferrule
  {\atj{\phi}{R}{\ty} \\ \rx{w}{\cid}\not\in R \\ \oplus \in \setit{\fplus,\fminus}}
  {\atj{\phi \oplus \rx{w}{\cid}}{R \cup \setit{\rx{w}{\cid}}}{\setit{\cty{\rx{w}{\cid}}{\ty}}}}

  \inferrule
  {\atj{\phi_1}{R_1}{\ty_1} \\
   \atj{\phi_2}{R_2}{\ty_2} \\ \oplus \in \{ \fplus,\fminus,\ftimes \}}
  {\atj{\phi_1 \oplus \phi_2}{R_1;R_2}{\ty_1 \cup \ty_2}}
\end{mathpar}
}
\end{fpfig}
}


\newcommand{\notgfig}{
\begin{fpfig}[t]{Evaluation of expressions within types and constraints.}{fig-notg}
$$
\begin{array}{cc}
  \begin{array}{rcl}
    \notg{x} &::=& \elab{\flip{e}}{e} \mid \elab{\secret{e}}{e} \mid \elab{\mesg{e}}{e} \mid \rvl{e} \mid \out{e}\\
  \notg{\phi} &::=& \notg{x} \mid \notg{\phi} \fplus \notg{\phi} \mid \notg{\phi} \fminus \notg{\phi} \mid \notg{\phi} \ftimes \notg{\phi} \\
  \notg{\eqs} &::=& \notg{\phi} \eop \notg{\phi} \mid \notg{\eqs} \wedge \notg{\eqs} \\
  \notg{X} &\in& 2^{\notg{x}}
\end{array}& \qquad
\begin{array}{rcl}
  \notg{t} &::=& e \mid \cty{e}{\notg{\ty}} \\
  \notg{\ty} & \in & 2^{\notg{t}}\\
  \notg{\Gamma} &::=& \varnothing \mid \notg\Gamma; e : \notg{\ty}\\
  \notg\Delta &::=& \varnothing \mid \notg\Delta; e : \ity{e}{\notg{V}}
\end{array}
\end{array}
$$

\begin{mathpar}
  \inferrule
      {\notg{\phi_1} \redx \phi_1 \\ \notg{\phi_2} \redx \phi_2}     
      {\notg{\phi_1} \ftimes \notg{\phi_2} \redx \phi_1 \ftimes \phi_2}

  \inferrule
      {\notg{\phi_1} \redx \phi_1 \\ \notg{\phi_2} \redx \phi_2}
      {\notg{\phi_1} \eop \notg{\phi_2} \redx \phi_1 \eop \phi_2}

  \inferrule
      {\notg{\eqs_1} \redx \eqs_1 \\ \notg{\eqs_2} \redx \eqs_2 }
      {\notg{\eqs_1} \wedge \notg{\eqs_2} \redx \eqs_1 \wedge \eqs_2}
\end{mathpar}

\begin{mathpar}
  \inferrule
      {e \redx x \\ \notg{\ty} \redx \ty}
      {\cty{e}{\notg{\ty}} \redx \cty{x}{\ty}}
      
  \inferrule
      {\notg{t_1} \redx t_1 \\ \cdots \\ \notg{t_n} \redx t_n}
      {\setit{\notg{t_1},\ldots,\notg{t_n}} \redx \setit{ t_1,\ldots,t_n }}

  \inferrule
      {\notg{\Gamma} \redx \Gamma \\ e \redx x \\ \notg{\ty} \redx \ty }
      {\notg{\Gamma}; e : \notg{\ty} \redx \Gamma; x : \ty }

  \inferrule
      {\notg{\Delta} \redx \Delta \\ e_1 \redx x  \\ e_2 \redx \cid \\ \notg{V} \redx V}
      {\notg{\Delta}; e_1 : \ity{e_2}{\notg{V}} \redx \Delta; x : \ity{\cid}{V} }

  \inferrule
      {\notg{\eqs_1} \redx \eqs_1 \\ \notg{\Gamma} \redx \\ \notg{R} \redx R
        \\ \notg{\Delta} \redx \Delta \\ \notg{\eqs_2} \redx \eqs_2}
      {\hty{\notg{\eqs_1}}{\notg{\Gamma}}{\notg{R}}{\notg{\Delta}}{\notg{\eqs_2}} \redx
        \hty{\eqs_1}{\Gamma}{R}{\Delta}{\eqs_2}}
\end{mathpar}
    
\end{fpfig}
}

\newcommand{\mtjfig}{
\begin{fpfig}[h]{$\metaprot$ type derivation rules for instructions.}{fig-mtj}
\begin{mathpar}
  \inferrule[Mesg]
            {\assign{e_1}{e_2} \redx \xassign{x}{\be}{\cid}  \\ \atj{\toeq{\elab{\be}{\cid}}}{R}{\ty} \\
              \itj{\cid}{\be}{V}}
            {\mtj{\assign{e_1}{e_2}}{\eqs}{(x:\ty)}{R}{(x : \ity{\cid}{V})}{\eqs \wedge x \eop \toeq{\elab{\be}{\cid}}}}

  \inferrule[Encode]
            {\mx{e_1}{e_2} \redx x \\ \notg{\phi} \redx \phi \\
              \eqs \models x \eop \phi\\
              \atj{\phi}{R}{\ty}}
            {\mtj{\eqcast{\mx{e_1}{e_2}}{\notg{\phi}}}{\eqs}{(x : \ty)}{R}{\varnothing}{\eqs}}

  \inferrule[Assert]
            {\elab{\assert{e_1 = e_2}}{e_3} \redx \elab{\assert{\be_1 = \be_2}}{\cid} \\
             \ipj{\eqs}{\elab{\assert{\be_1 = \be_2}}{\cid}}{\Delta}}
            {\mtj{\elab{\assert{e_1 = e_2}}{e_3}}{\eqs}{\varnothing}{\varnothing}{\Delta}{\eqs}}
            
  \inferrule[App]
            {f : \dht{y_1,\ldots,y_n}{\notg{\eqs_1}}{\notg{\Gamma}}{\notg{R}}{\notg{\Delta}}{\notg{\eqs_2}} \\
              e_1 \redx \mv_1\ \cdots\ e_n \redx \mv_n \\
              (\hty{\notg{\eqs_1}}{\notg{\Gamma}}{\notg{R}}{\notg{\Delta}}{\notg{\eqs_2}})[\mv_1/y_1]\cdots[\mv_n/y_n] \redx
                    \hty{\eqs_1}{\Gamma}{R}{\Delta}{\eqs_2} \\
              \eqs \models \eqs_1}
            {\mtj{f(e_1,\ldots,e_n)}{\eqs}{\Gamma}{R}{\Delta}{\eqs \wedge \eqs_2}}

  \inferrule[Seq]          
            {\mtj{\prog_1}{\eqs_1}{\Gamma_1}{R_1}{\Delta_1}{\eqs_2} \\
             \mtj{\prog_2}{\eqs_2}{\Gamma_2}{R_2}{\Delta_2}{\eqs_3}}
            {\mtj{\prog_1;\prog_2}{\eqs_1}{\Gamma_1;\Gamma_2}{R_1;R_2}{\Delta_1;\Delta_2}{\eqs_3}}
\end{mathpar}
\end{fpfig}
}

\newcommand{\mtjfnfig}{
\begin{fpfig}[h]{$\metaprot$ type derivation rules for function definitions.}{fig-mtjfn}
\begin{mathpar}
   \inferrule[Fn]
            {\codebase(f) = y_1,\ldots,y_n, \instr \\
              \mtj{\instr[\mv_1/y_1]\cdots[\mv_n/y_n]}{\eqs_1}{\Gamma}{R}{\Delta}{\eqs_2}\\
              \fresh(\mv_1,\ldots,\mv_n) \\
              %\subn = [\mv_1/y_1]\cdots[\mv_n/y_n] \\
              (\hty{\notg{\eqs_1}}{\notg{\Gamma}}{\notg{R}}{\notg{\Delta}}{\notg{\eqs_2}})[\mv_1/y_1]\cdots[\mv_n/y_n]  \redx
                    \hty{\eqs_1}{\Gamma}{R}{\Delta}{\eqs_2} }
            {f : \dht{y_1,\ldots,y_n}{\notg{\eqs_1}}{\notg{\Gamma}}{\notg{R}}{\notg{\Delta}}{\notg{\eqs_2}}}

  \inferrule[FnPre]
            {f : \dht{y_1,\ldots,y_n}{\notg{\eqs}}{\notg{\Gamma}}{\notg{R}}{\notg{\Delta}}{\notg{\eqs_2}} \\
              \precond(f) = \notg{\eqs_1} \\
              \fresh(\mv_1,\ldots,\mv_n) \\
              \notg{\eqs}[\mv_1/y_1]\cdots[\mv_n/y_n]  \redx \eqs \\
              \notg{\eqs_1}[\mv_1/y_1]\cdots[\mv_n/y_n]  \redx \eqs_1 \\
              \eqs_1 \models \eqs             
            }
            {f : \dht{y_1,\ldots,y_n}{\notg{\eqs_1}}{\notg{\Gamma}}{\notg{R}}{\notg{\Delta}}{\notg{\eqs_2}}}

  \inferrule[FnPost]
            {f : \dht{y_1,\ldots,y_n}{\notg{\eqs_1}}{\notg{\Gamma}}{\notg{R}}{\notg{\Delta}}{\notg{\eqs}} \\
              \postcond(f) = \notg{\eqs_2} \\
              \fresh(\mv_1,\ldots,\mv_n) \\
              \notg{\eqs}[\mv_1/y_1]\cdots[\mv_n/y_n]  \redx \eqs \\
              \notg{\eqs_2}[\mv_1/y_1]\cdots[\mv_n/y_n]  \redx \eqs_2 \\
              \eqs \models \eqs_2              
            }
            {f : \dht{y_1,\ldots,y_n}{\notg{\eqs_1}}{\notg{\Gamma}}{\notg{R}}{\notg{\Delta}}{\notg{\eqs_2}}}
\end{mathpar}
\end{fpfig}
}


\acmConference[PPDP]{Principles and Practice of Declarative Programming}{2024}{Milan}

\begin{document}

\title{Language-Based Security for Low-Level MPC}

\author{Christian Skalka}
\affiliation{
  \institution{University of Vermont}
  \city{}
  \country{}
%  \city{Burlington}
%  \country{USA}
}
\email{ceskalka@uvm.edu}

\author{Joseph Near}
\affiliation{
  \institution{University of Vermont}
  \city{}
  \country{}
%  \city{Burlington}
%  \country{USA}
}
\email{jnear@uvm.edu}

\begin{abstract}
  Secure Multi-Party Computation (MPC) protocols is an important
  enabling technology for data privacy in modern distributed
  applications. Currently, proof methods for low-level MPC protocols
  are primarily manual and thus tedious and error-prone, and are also
  non-standardized and unfamiliar to most PL theorists. As a step
  towards better language support and language-based enforcement, we
  develop a new staged PL for defining a variety of low-level
  probabilistic MPC protocols. We also formulate a collection of
  confidentiality and integrity hyperproperties for our language model
  that are familiar from information flow, including conditional
  noninterference, delimited release, and robust declassification. We
  demonstrate their relation to standard MPC threat models of passive
  and malicious security, and how they can be leveraged in security
  verification of protocols. To prove these properties we develop
  automated tactics in $\mathbb{F}_2$ that can be integrated with
  separation-logic style reasoning.
\end{abstract}

%%
%% The code below is generated by the tool at http://dl.acm.org/ccs.cfm.
%% Please copy and paste the code instead of the example below.
%%
\begin{CCSXML}
<ccs2012>
   <concept>
       <concept_id>10002978.10002986.10002990</concept_id>
       <concept_desc>Security and privacy~Logic and verification</concept_desc>
       <concept_significance>500</concept_significance>
       </concept>
   <concept>
       <concept_id>10003752.10003753.10003757</concept_id>
       <concept_desc>Theory of computation~Probabilistic computation</concept_desc>
       <concept_significance>300</concept_significance>
       </concept>
   <concept>
       <concept_id>10003752.10003790.10003806</concept_id>
       <concept_desc>Theory of computation~Programming logic</concept_desc>
       <concept_significance>500</concept_significance>
       </concept>
 </ccs2012>
\end{CCSXML}

\ccsdesc[500]{Security and privacy~Logic and verification}
\ccsdesc[500]{Theory of computation~Probabilistic computation}
\ccsdesc[500]{Theory of computation~Programming logic}


%%
%% Keywords. The author(s) should pick words that accurately describe
%% the work being presented. Separate the keywords with commas.
\keywords{Secure multiparty computation, security verification, probabilistic programming, programming languages, information flow.}

\maketitle

\section{Introduction}

Secure Multi-Party Computation (MPC) protocols support data privacy in
important modern, distributed applications such as privacy-preserving
machine learning \cite{li2021privacy, knott2021crypten,
  koch2020privacy, liu2020privacy} and Zero-Knowledge proofs in
blockchains \cite{ishai2009zero, lu2019honeybadgermpc,
  gao2022symmeproof, tomaz2020preserving}. The security semantics of
MPC include both confidentiality and integrity properties incorporated
into models such as real/ideal (aka simulator) security and universal
composability (UC), developed primarily by the cryptography community
\cite{evans2018pragmatic}.  Related proof methods are well-studied \cite{Lindell2017}
but mostly manual. Somewhat independently, a significant
body of work in programming languages has focused on definition and
enforcement of confidentiality and integrity \emph{hyperproperties}
\cite{10.5555/1891823.1891830} such as noninterference and delimited
release \cite{sabelfeld2009declassification}. Following a tradition of connecting
cryptographic and PL-based security models \cite{10.1007/3-540-44929-9_1}, recent work has
also recognized connections between MPC security models and
hyperpoperties of, e.g., noninterference \cite{8429300}, and even
leveraged these connections to enforce MPC security through mechanisms
such as security types \cite{10.1145/3453483.3454074}. Major benefits of this connection
in an MPC setting include better language abstractions for defining
protocols and for mechanisation and even automation of security proofs.
The goal of this paper is to develop a PL model for defining a variety of
low-level probabilistic MPC protocols, to formulate a collection
of confidentiality and integrity hyperproperties for our model
model with familiar information flow analogs, and to show how these
properties can be leveraged for improved proof automation.

The distinction between high- and low-level languages for MPC is
important. High-level languages such as Wysteria \cite{} and Viaduct
\cite{10.1145/3453483.3454074} are designed to provide effective
programming of full applications. These language designs incorporate
sophisticated verified compilation techniques such as orchestration
\cite{viaduct-UC} to guarantee high-level security properties, and
they rely on \emph{libraries} of low-level MPC protocols, such as
binary and arithmetic circuits. These low-level protocols encapsulate
abstractions such as secret sharing and semi-homomorphic encrypytion,
and rely on probabilistic progamming methods. So, low-level MPC
programming and protocol verification is a distinct challenge and both
critical to the general challenge of PL design for MPC and
complementary to high-level language design.

The connection between information flow hyperproperties and MPC
security is also complicated especially at a low level.  MPC protocols
involve communication between a group of distributed clients called a
\emph{federation} that collaboratively compute and publish the result
of some known \emph{ideal functionality} $\idealf$, maintaining
confidentiality of inputs to $\idealf$ without the use of a trusted
third party. However, since the outputs of $\idealf$ are public, some
information about inputs is inevitably leaked. Thus, the ideal
functionality establishes a declassification policy
\cite{sabelfeld2009declassification}, which is more difficult to
enforce than pure noninterference.  And subtleties of, e.g.,
semi-homomorphic encryption are central to both confidentiality and
integrity properties of protocols and similarly difficult to track
with coarse-grained security types alone.

Nevertheless, as previous authors have observed
\cite{5a51987acaa84c43bb4bf5bcc7d01683}, low-level protocol design
patterns such as secret sharing and circuit gate structure have
compositional properties that can be independently verified and then
leveraged in larger proof contexts. We contribute to this line of work
by developing an automated verification technique for subprotocols and
show how it can be integrated as a tactic in a larger security proof.

\compfig

\subsection{Related Work}
\label{section-related-work}

Our main focus is on PL design and automated and semi-automated
reasoning about security properties of low-level MPC protocols.  We
consider related systems in several dimensions, including
whether they are aimed at low-level design with probabilistic features,
whether they support reasoning about conditional probabilities which
are central to real/ideal security, whether they consider MPC
security through the lens of hyperproperties, and whether they consider
passive and/or malicious security models. We summarize this comparison
in Figure \ref{fig-comp}, with the caveat that works vary in the degree
of development in each dimension.

As mentioned above, several high-level languages have been developed
for writing MPC applications, and frequently exploit the connection
between hyperproperties and MPC security. Previous work on analysis
for the SecreC language
\cite{almeida2018enforcing,10.1145/2637113.2637119} is concerned with
properties of complex MPC circuits, in particular a user-friendly
specification and automated enforcement of declassification bounds in
programs that use MPC in subprograms. The Wys$^\star$ language
\cite{wysstar}, based on Wysteria \cite{rastogi2014wysteria}, has
similar goals and includes a trace-based semantics for reasoning about
the interactions of MPC protocols. Their compiler also guarantees that
underlying multi-threaded protocols enforce the single-threaded source
language semantics. These two lines of work were focused on passive
security. The Viaduct language
\cite{10.1145/3453483.3454074} has a well-developed
information flow type system that automatically enforces both
confidentiality and integrity through hyperproperties such as robust
declassification, in addition to rigorous compilation guarantees
through orchestration \cite{viaduct-UC}. However, these high level
languages lack probabilistic features and other abstractions of
low-level protocols, the implementation and security of which are
typically assumed as a selection of library compnonents.

Various related low-level languages with probabilistic features have
also been developed. The $\lambda_{\mathrm{obliv}}$ language
\cite{darais2019language} uses a type system is used to automatically
enforce so-called probabilistic trace obliviousness.  But similar to
previous work on oblivious data structures \cite{10.1145/3498713},
obliviousness is related to pure noninterference, not the relaxed form
related to passive MPC security. The Haskell-based security type system in
\cite{6266151} enforces a version of conditioned noninterence that is
sound for passive security in a probabilisitic low-level setting, but
they do not consider malicious security. And properties of real/ideal
passive and malicious security for a probabilistic language have been
formulated in EasyCrypt \cite{8429300}-- though their proof methods,
while mechanized, are fully manual, and their formulation of malicious
security is not as clearly related to robust declassification as is the
one we present in Section \ref{section-hyper}. 

Program logics for probabilistic languages and specifically reasoning
about properties such as joint probabilistic independence is also
important related work. Probabilistic Separation Logic (PSL)
\cite{barthe2019probabilistic} develops a logical framework for
reasoning about probabilistic independence (aka separation) in
programs, and they consider several (hyper)properties, such as perfect
secrecy of one-time-pads and indistinguishability in secret sharing,
that are critical to MPC. However, their methods are manual, and
don't include conditional independence (separation). This
latter issue has been addressed in Lilac \cite{li2023lilac}. The
application of Lilac-style reasoning to MPC protocols has not
previously been explored, as we do in Section
\ref{section-example-gmw}.

Our work also shares many ideas with probabilistic programming
languages designed to perform (exact or approximate) statistical
inference~\cite{holtzen2020scaling, carpenter2017stan, wood2014new,
  bingham2019pyro, albarghouthi2017fairsquare, de2007problog,
  pfeffer2009figaro, saad2021sppl}. Our setting, however, requires
verifying properties beyond inference, including conditional
statistical independence. Recent work by Li et al.~\cite{li2023lilac} proposes a
manual approach for proving such properties, but does not provide
automation.

\subsection{Overview and Contributions}

\paragraph{Language design.} In Section \ref{section-lang} we
develop a new probabilistic programming language $\minifed$ for
defining synchronous distributed protocols over an arbitrary
arithmetic field. The syntax and semantics provides a succinct account
of synchronous messaging between protocol \emph{clients}. In Section
\ref{section-metalang} we define a metalanguage $\metaprot$ that
dynamically generates $\minifed$ protocols. It is able to express
important low-level abstractions, as we illustrate via implementations
of protocols including Shamir addition (Section \ref{section-lang}),
GMW boolean circuits (Section \ref{section-example-gmw}), and Beaver
Triple multiplication gates with BDOZ authentication (Section
\ref{section-example-bdoz}).

\paragraph{Hyperproperty formulation.} In Section \ref{section-model} we
develop our formalism for expressing the joint probability mass function of
program variables, and give standard definitions of passive and
malicious real/ideal security in our model. In Section
\ref{section-hyper}, we formulate a variety of familiar information
flow properties in our probabilistic setting, including condition
noninterference, delimited release, and robust declassification, and
consider the relation between these and real/ideal security.  While it
has been previously shown that probabilistic conditional
noninterference is sound for passive security, we formulate new
properties of integrity which, paired with passive security, imply
malicious security (Theorem \ref{theorem-integrity}). We observe
in Section \ref{section-example-bdoz} that authentication mechanisms
such as BDOZ/SPDZ style MACs enforce a strictly weaker property
of ``cheating detection''. 


\paragraph{Fully and partially automated verification.} In Section
\ref{section-bruteforce} we develop a method for automatically
computing the pmf of $\minifed$ protocols in $\mathbb{F}_2$, that can
be automatically queried to enforce hyperperproperties of
security. This method is perfectly accurate but has high complexity;
we show this can be partially mitigated by conversion of protocols in
$\mathbb{F}_2$ to stratified Datalog which is amenable to HPC
acceleration. Furthermore, in Section \ref{section-examples-gmw} we
consider in detail how this automated technique can be used as a local
automated tactic for proving security in arbitrarily large GMW
circuits using conditional probabilistic independence as in
\cite{li2023lilac}.


\section{The $\minicat$ Protocol Language}

\begin{fpfig}[t]{Top-to-bottom: Basic $\minifed$ syntax, expression interpretation, and reduction rules.}{fig-minifed}
  {
    $$
    \begin{array}{rcl@{\hspace{8mm}}r}
      \multicolumn{4}{l}{v \in \mathbb{Z}_p,\ w \in \mathrm{String},\ \cid \in \mathrm{Clients} \subset  \mathbb{N} }\\[2mm] %, \bop \in \{ \eand, \eor, \exor \}} \\[2mm]
      \be &::=& v \mid \flip{w} \mid \secret{w} \mid \mesg{w} \mid \rvl{w} \mid \be \fminus \be \mid \be \fplus \be \mid \be \ftimes \be \mid f \mid \be\,\be & \textit{expressions}\\[2mm]
      x &::=& \elab{\flip{w}}{\cid} \mid \elab{\secret{w}}{\cid} \mid \elab{\mesg{w}}{\cid} \mid \rvl{w} \mid \out{\cid} & \textit{protocl variables} \\[2mm]
      %& &  \select{\be}{\be}{\be} \mid \ctxt{v}{k} \mid \key{w} \mid \sk{\be}(\be) \mid \pk{\be}{\be}(\be) \mid \pk{\be}{\be} \\[2mm]
      %& &  \select{\fp(\be)}{\be}{\be} \ctxt{v,\be}{k}  \mid \sk{\be}(\be) \mid \pk{\be}{\be}(\be) \mid \pk{\be}{\be} \\[2mm]
      \instr &::=& \eassign{\mesg{w}}{\cid}{\be}{\cid} \mid
      \reveal{w}{e}{\cid} \mid \pubout{\cid}{\be}{\cid} & \textit{commands} \\[2mm]
      \prog &::=& \varnothing \mid \instr; \prog & \textit{protocols}
    \end{array}
    $$
  
  \rule{130mm}{0.5pt}

  $$
  \begin{array}{c@{\hspace{5mm}}c}
  \begin{array}{rcl}
    \lcod{\store, v}{\cid} &=& v\\
    \lcod{\store, \be_1 \fplus \be_2}{\cid} &=& \fcod{\lcod{\store, \be_1}{\cid} \fplus \lcod{\store, \be_2}{\cid}}\\ 
    \lcod{\store, \be_1 \fminus \be_2}{\cid} &=& \fcod{\lcod{\store, \be_1}{\cid} \fminus \lcod{\store, \be_2}{\cid}}\\ 
    \lcod{\store, \be_1 \ftimes \be_2}{\cid} &=& \fcod{\lcod{\store, \be_1}{\cid} \ftimes \lcod{\store, \be_2}{\cid}}
  \end{array} & 
  \begin{array}{rcl}
    \lcod{\store, \flip{w}}{\cid} &=& \store(\elab{\flip{w}}{\cid})\\
    \lcod{\store, \secret{w}}{\cid} &=& \store(\elab{\secret{w}}{\cid})\\
    \lcod{\store, \mesg{w}}{\cid} &=& \store(\elab{\mesg{w}}{\cid})\\
    \lcod{\store, \rvl{w}}{\cid} &=& \store(\rvl{w})\\
    \lcod{\store, f\,e_1\,\cdots\, e_n}{\cid} &=& \delta(f,\lcod{\store, e_1}{\cid},\ldots,\lcod{\store,e_n}{\cid})
  \end{array}
  \end{array}
  $$

  \vspace{4mm}
  
  \rule{130mm}{0.5pt}

  \begin{mathpar}
    (\store, \eassign{\mesg{w}}{\cid_1}{\be}{\cid_2};\prog) \redx (\extend{\store}{\mesg{w}_{\cid_1}}{\lcod{\store,\be}{\cid_2}}, \prog)
    
    (\store, \reveal{w}{\be}{\cid};\prog) \redx (\extend{\store}{\rvl{w}}{\lcod{\store,\be}{\cid}}, \prog)
    
    (\store, \pubout{\cid}{\be}{\cid};\prog) \redx (\extend{\store}{\out{\cid}}{\lcod{\store,\be}{\cid}}, \prog)
  \end{mathpar}
  }
\end{fpfig}

The $\minifed$ language provides a simple model of synchronous
protocols between a federation of \emph{clients} exchanging values in
the binary field. We will identify clients by natural numbers, and
federations- finite sets of clients- are always given statically.
As we will see, our threat model assumes a partition of the federation
into \emph{honest} $H$ and \emph{corrupt} $C$ subsets.

We model probabilistic programming via a \emph{random tape}
semantics. That is, we will assume that programs can make reference to
values chosen from a uniform random distributions defined in the
initial program memory.  Programs aka protocols execute
deterministically given the random tape.

\subsection{Syntax} The syntax of $\minifed$, defined in
Figure \ref{fig-minifed}, includes values $v$ and standard
operations of addition, subtraction, and multiplication in
a finite field $\mathbb{Z}_p$ with $p$ prime. 
Protocols are given input secret values $\secret{w}$
as well as random samples $\flip{w}$ on the input
tape, both of which are distinguished by
strings $w$. Protocols are sequences of assignment
commands of three different forms:
\begin{itemize}
\item $\eassign{\mesg{w}}{\cid_2}{\be}{\cid_1}$: This
  is a \emph{message send} where expression $\be$ is computed
  by client $\cid_1$ and sent to client $\cid_2$ as message
  $\mesg{w}$.
\item $\reveal{w}{\be}{\cid}$: This
  is a \emph{public reveal} where expression $\be$ is computed
  by client $\cid$ and broadcast to the federation.
\item $\pubout{\cid}{\be}{\cid}$: This
  is an \emph{output} where expression $\be$ is computed
  by client $\cid$ and reported as its output.
\end{itemize}
Both messages $\mesg{w}$ and reveals $\rvl{w}$ can be
referenced in expressions, once they've been assigned.

We let $x$ range over \emph{variables}  which are identifiers
where client ownership is specified- e.g., $\elab{\mesg{\mathit{foo}}}{\cid}$
is a message $\mathit{foo}$ that was sent to $\cid$. We let $X$
range over sets of variables, and more specifically, $S$ ranges over sets of secret variables $\elab{\secret{w}}{\cid}$, $R$ ranges over sets of random variables $\elab{\flip{w}}{\cid}$, $M$ ranges over sets of message variables $\elab{\mesg{w}}{\cid}$, $P$ ranges over sets of public variables $\rvl{w}$, and $O$ ranges over sets of output variabels $\out{\cid}$.
Given a program $\prog$, we write $\iov(\prog)$ to
denote the set of $S \cup M \cup P \cup O$ of variables in $\prog$
with ownership made explicit, and we write $\flips(\prog)$ to
denote the set $R$ of random samplings in $\prog$ with ownership
made explicit. We write
$\vars(\prog)$ to denote $\iov(\prog) \cup \flips(\prog)$. For any set
of variables $X$ and parties $P$, we write $X_P$ to denote the subset
of $X$ owned by any party in $P$, in particular we write $X_H$ and $X_C$ to
denote the subsets belonging to honest and corrupt parties,
respectively.

\subsubsection{Library Functions} $\minifed$ expression syntax also supports
calls to library functions $f$ which can be applied to muliple arguments in a
curried style. This allows encapsulation and separate
definition of primitive operations such as one-time-pads and message
authentication, as we will illustrate with examples. This approach is
useful since it parameterizes these definitions, and 
supports verification of behavior specified with types, as we
discuss in Section \ref{section-types}.

\subsection{Semantics}

\emph{Memories} are fundamental to the semantics of $\fedcat$ and
provide random tape and secret inputs to protocols, and also record
message sends, public broadcast, and client outputs. Memories $\store$ are finite
(partial) mapping from variables $x$ to values $v \in \mathbb{Z}_p$. The \emph{domain} of a
memory is written $\dom(\store)$ and is the finite set of variables on
which the memory is defined. We write $\store\{ x \mapsto v\}$ for
$x\not\in\dom(\store)$ to denote the memory $\store'$ such that
$\store'(x) = v$ and otherwise $\store'(y) = \store(y)$ for all $y
\in \dom(\store)$. We write $\store \subseteq \store'$ iff
$\dom(\store) \subseteq \dom(\store')$ and $\store(x) =
\store'(x)$ for all $x \in \dom(\store)$. We write $\store \cap
\store'$ to denote the combination of $\store$ and $\store'$
assuming $\store(x) = \store'(x)$ for all $x \in \dom(\store)
\cap \dom(\store')$, otherwise $\store \cap \store'$ is undefined.
We write $\store \subseteq \store'$ iff $\store \cap \store'
= \store$.

Given a set of variables $X$, we write $\store_X$ to denote the
memory $\store$ restricted to the domain $X$, and we define
$\mems(X)$ as the set of all memories with domain $X$:
$$
\mems(X) \defeq \{ \store \mid \dom(\store) = X \}
$$
Thus, given a protocol $\prog$, the set of all random tapes for
$\prog$ is $\mems(\flips(\prog))$.
%We let $\stores$ range
%over sets of memories with the same domain, and abusing notation
%we write $\dom(\stores)$ to denote the common domain,
%and $\stores_X \defeq \{ \store_X | \store \in \stores \}$.

Given a variable-free expression $\be$, we write $\cod{\be}$ to denote
the standard interpretation of $\be$ in the arithmetic field
$\mathbb{Z}_{p}$. With the introduction of variables to expressions,
we need to interpret variables with respect to a specific memory, and
all variables used in an expression must belong to a specified client.
Thus, we denote interpretation of expressions $\be$ computed on a
client $\cid$ as $\lcod{\store,\be}{\cid}$. This interpretation is
defined in Figure \ref{fig-minifed}. It is also parameterized by
$\delta$ which defines the semantics of library functions $f$.

The small-step reduction relation $\redx$ is then defined in Figure
\ref{fig-minifed} to evaluate commands. Reduction is a relation on
\emph{configurations} $(\store, \prog)$ where all three command forms-
message send, broadcast, and output- are implemented as updates to the
memory $\store$. We write $\redxs$ to denote the reflexive, transitive
closure of\ $\redx$. 

\subsection{Example: Passive-Secure Addition}

Shamir addition leverages homomorphic properties of addition in
arithmetic fields to implement secret addition. If a field value $v_1$
is in a uniform random distribution with other variables in a program,
then $v_1 \fplus v_2$ is an encryption of $v_2$ where $v_1$ is an
information theoretically secure one-time-pad, which is exploited for
secret sharing. Of course, $\fplus$ is also a meaningful operation
over any two field values regardless of their distributions.

To capture this distinction we introduce a function $\otp$
with the following specification:
$$
\delta(\otp,v_1,v_2) \defeq v_1 \fplus v_2
$$
Although the semantics are the same as addition, the use of $\otp$
makes a declarative distinction, but more importantly we will be
able to assign a type to $\otp$ that enforces the one-time discipline
on its first argument via type linearity as will be discussed in Section
\ref{section-types}.

To sum their secret values $\secret{\cid}$, each client $\cid$ in
the federation $\{ 1, 2, 3 \}$  samples a value $\locflip$
that can be used as a one-time pad in summation with another
random sample $\flip{x}$ and $\secret{\cid}$. This yields
two secret shares communicated as messages to the other clients,
while each client keeps $\locflip$ as its own share.
$$
\begin{array}{lll}
  \elab{\mesg{s1}}{2} &:=& \elab{(\otp\ \locflip\ (\flip{x} \fplus \secret{1})}{1} \\ 
  \elab{\mesg{s1}}{3} &:=& \elab{\flip{x}}{1} \\ 
  \elab{\mesg{s2}}{1} &:=& \elab{(\otp\ \locflip\ (\flip{x} \fplus \secret{2})}{2} \\ 
  \elab{\mesg{s2}}{3} &:=& \elab{\flip{x}}{2} \\ 
  \elab{\mesg{s3}}{1} &:=& \elab{(\otp\ \locflip\ (\flip{x} \fplus \secret{3})}{3} \\ 
  \elab{\mesg{s3}}{2} &:=& \elab{\flip{x}}{3}
\end{array}
$$
Due to field properties of $\fplus$ this scheme guarantees that messages
are viewed as random noise by any observer 
besides $\cid$ \cite{barthe2019probabilistic}. Next, each client
publicly reveals the sum of all of its shares, including its local
share. This does reveal information about secrets. Further there
is no one-time-pad to use in this summation.
$$
\begin{array}{lll}
  \rvl{1} &:=& \elab{(\locflip \fplus \mesg{s2} \fplus \mesg{s3})}{1} \\ 
  \rvl{2} &:=& \elab{(\mesg{s1} \fplus \locflip \fplus \mesg{s3})}{2} \\
  \rvl{3} &:=& \elab{(\mesg{s1} \fplus \mesg{s2} \fplus \locflip)}{3} 
\end{array}
$$
Finally, each client outputs the sum of each sum of shares, yielding
the sum of secrets. Note that this stage exposes no more information
than the previous public reveals. 
$$
%\elab{\mesg{o1}}{2} &:=& \elab{(\locflip \fplus \mesg{s2} \fplus \mesg{s3})}{1} \\ 
  %\elab{\mesg{o1}}{3} &:=& \elab{(\locflip \fplus \mesg{s2} \fplus \mesg{s3})}{1} \\ 
  %\elab{\mesg{o2}}{1} &:=& \elab{(\mesg{s1} \fplus \locflip \fplus \mesg{s3})}{2} \\
  %\elab{\mesg{o2}}{3} &:=& \elab{(\mesg{s1} \fplus \locflip \fplus \mesg{s3})}{2} \\ 
  %\elab{\mesg{o3}}{1} &:=& \elab{(\mesg{s1} \fplus \mesg{s2} \fplus \locflip)}{3} \\ 
  %\elab{\mesg{o3}}{2} &:=& \elab{(\mesg{s1} \fplus \mesg{s2} \fplus \locflip)}{3}\\ 
  %\pubout{1} &:=& \elab{(\locflip \fplus \mesg{s2} \fplus \mesg{s3} + \mesg{o2} + \mesg{o3})}{1}
\begin{array}{lll}
  \out{1} &:=& \elab{(\rvl{1} \fplus \rvl{2} + \rvl{3})}{1}\\
  \out{2} &:=& \elab{(\rvl{1} \fplus \rvl{2} + \rvl{3})}{2}\\
  \out{3} &:=& \elab{(\rvl{1} \fplus \rvl{2} + \rvl{3})}{3}
\end{array}
$$
It is well-known that additive secret sharing is passive
secure. That is, any adversarial observer can gain no more information
from the messages exchanged in the protocol than what is exposed by
the output alone. However, malicious adversaries can corrupt this
protocol by injecting ``fake'' sums of shares in their public reveals.

\subsection{Example: Malicious Secure Product}

$$
\begin{array}{rcl@{\hspace{8mm}}r}
  \multicolumn{3}{l}{m,k \in \mathbb{Z}_p} \qquad \macv ::= (v,[m_1,\ldots,m_n]) &
  \textit{MACed\ values}\\[2mm]
  \be &::=& \cdots \macv{v} \mid \be \macplus \be \mid \be \mactimes \be \mid \be \macminus \be\\[2mm]
  x &::=& \macx{\secret{w}}{\cid} \mid \macx{\flip{w}}{\cid} \mid \mack{x}{\cid}
\end{array}
  

    
$$
(v, [m_1,\ldots,m_n])
$$
$$
k_1,\ldots,k_n
$$
$$
m_\cid = k_\cid + (k_\Delta * v)
$$

$$
\delta(\macplus,(v^1, [m_1^1,\ldots,m^1_n]),(v^2, [m_1^2,\ldots,m^2_n]))
\defeq
(v^1 \fplus v^2, [m_1^1 \fplus m_1^2 ,\ldots,m^1_n \fplus m_n^2])
$$

$$
\elab{\macgv{\elab{v}{\cid}}}{1} \macplus \cdots \macplus \elab{\macgv{\elab{v}{\cid}}}{n} =
(v,\ldots)
$$

$$
\delta(\macotp,v_1,v_2) \defeq v_1 \macplus v_2
$$

$$
\delta(\macauth, (v, [\ldots,m_\cid,\ldots]), k_\cid) \defeq
     (v, [\ldots,m_\cid,\ldots]) \text{\ if\ } m_i = k_i + (k_\Delta * v)
$$

$$
\begin{array}{lcl}
  \elab{\mesg{a}}{2} &:=&
  \elab{(\macotp\ \macgv{\elab{\secret{x}}{1}}\ \macgv{\elab{\flip{a}}{\Oracle}})}{1}\\
  \elab{\mesg{a}}{1} &:=&
  \elab{(\macotp\ \macgv{\elab{\secret{x}}{1}}\ \macgv{\elab{\flip{a}}{\Oracle}})}{2}\\
  \elab{\mesg{b}}{2} &:=&
  \elab{(\macotp\ \macgv{\elab{\secret{y}}{2}}\ \macgv{\elab{\flip{b}}{\Oracle}})}{1}\\
  \elab{\mesg{b}}{1} &:=&
  \elab{(\macotp\ \macgv{\elab{\secret{y}}{2}}\ \macgv{\elab{\flip{b}}{\Oracle}})}{2}\\
  \elab{\mesg{d}}{1} &:=&
  \elab{(\macauth(\mesg{a}, \mack{\elab{\secret{x}}{1}}{2} \fminus \mack{\elab{\flip{a}}{\Oracle}}{2}) \macplus (\macgv{\elab{\secret{x}}{1}}\macminus\macgv{\elab{\flip{a}}{\Oracle}}))}{1}\\
  \elab{\mesg{e}}{1}&:=&
  \elab{(\macauth(\mesg{b}, \mack{\elab{\secret{y}}{2}}{2} \fminus \mack{\elab{\flip{b}}{\Oracle}}{2}) \macplus (\macgv{\elab{\secret{y}}{2}}\macminus\macgv{\elab{\flip{b}}{\Oracle}}))}{1}\\
  \rvl{1} &:=&
  \elab{( (\mesg{d} \mactimes \mesg{e}) \macplus
          (\mesg{d} \mactimes \macgv{\elab{\flip{b}}{\Oracle}}) \macplus
          (\mesg{e} \mactimes \macgv{\elab{\flip{a}}{\Oracle}}) \macplus \macgv{\elab{\secret{c}}{\Oracle}}
    )}{1}\\
  \elab{\mesg{d}}{2} &:=&
  \elab{(\macauth(\mesg{a}, \mack{\elab{\secret{x}}{1}}{1} \fminus \mack{\elab{\flip{a}}{\Oracle}}{1}) \macplus (\macgv{\elab{\secret{x}}{1}}\macminus\macgv{\elab{\flip{a}}{\Oracle}}))}{2}\\
  \elab{\mesg{e}}{2}&:=&
  \elab{(\macauth(\mesg{b}, \mack{\elab{\secret{y}}{2}}{1} \fminus \mack{\elab{\flip{b}}{\Oracle}}{1}) \macplus (\macgv{\elab{\secret{y}}{2}}\macminus\macgv{\elab{\flip{b}}{\Oracle}}))}{2}\\
  \rvl{2} &:=&
  \elab{( (\mesg{d} \mactimes \mesg{e}) \macplus
          (\mesg{d} \mactimes \macgv{\elab{\flip{b}}{\Oracle}}) \macplus
          (\mesg{e} \mactimes \macgv{\elab{\flip{a}}{\Oracle}}) \macplus \macgv{\elab{\secret{c}}{\Oracle}}
    )}{2}\\
  \out{1} &:=& \elab{(\rvl{1} \macplus \macauth(\rvl{2},\ldots))}{1} \\
  \out{2} &:=& \elab{(\macauth(\rvl{1},\ldots) \macplus \rvl{2})}{2}
\end{array}
$$




\section{Security Model}
\label{section-pmf}
\label{section-model}

MPC protocols are intended to implement some \emph{ideal
functionality} $\idealf$ with per-client outputs. In the $\minifed$
setting, Given a protocol $\prog$ that implements $\idealf$, with
$\iov(\prog) = S \cup V \cup O$, the domain of $\idealf$ is $\mems(S)$
and its range is $\mems(O)$.  Real/ideal security in the MPC
setting means that, given $\store \in \mems(S)$, a secure protocol
$\prog$ does not reveal any more information about honest secrets
$\store_H$ to parties in $C$ beyond what is implicitly declassified by
$\idealf(\sigma)$. Security comes in \emph{passive} and
\emph{malicious} flavors, wherein the adversary either follows the
rules or not, respectively. Characterization of both real world
protocol execution and simulation is defined
probabilistically. Following previous work
\cite{barthe2019probabilistic} we use probability mass functions to
express joint dependencies between input and output variables, as a
metric of information leakage.

\subsection{Probability Mass Functions} 

We define joint probability mass functions (pmfs) in the standard
manner, though following \cite{barthe2019probabilistic} we use
memories to denote mappings of variables to values (i.e., outcomes),
so for example given a pmf $\pmf$ we will write $\pmf(\{ \elab{\secret{x}}{1}
\mapsto 0, \elab{\mesg{y}}{2} \mapsto 1 \})$ to denote the (joint) probability that
$\elab{\secret{x}}{1} = 0 \wedge \elab{\mesg{y}}{2} = 1$.
\begin{definition}
  A \emph{probability mass function} $\pmf$ is a function
  mapping memories in $\mems(X)$ for some $X$ to values in $\mathbb{F}_p$, such that:
  $$
  \sum_{\store \in \mems(X)} \pmf(\store) \  = \ 1
  $$
\end{definition}
%To recover succinct and familiar notation, we may omit the domain of a
%distribution when it is clear from an application context-
%i.e., we allow the following sugaring:
%$$
%\pdf{}(\store) \defeq \pdf{\dom(\store)}(\store)
%$$
Now, we can define a notion of marginal and conditional
distributions as follows, which are standard for discrete
probability mass functions. 
\begin{definition}
  Given $\pmf$ the \emph{marginal distribution} of variables $X$
  in $\pmf$, denoted $\margd{\pmf}{X}$, is defined as follows:
  $$
  \forall \store \in \mems(X) \quad . \quad \margd{\pmf}{X}(\store) =
  \sum_{\store' \in \mems(X-\dom(\dom(\pmf)))} \pmf(\store \cap \store')
  $$
\end{definition}

\begin{definition}
  Given $\margd{\pmf}{X}$, let $\stores$ be a set of memories with the
  same domain $Y \subseteq X$. Then the \emph{conditional distribution given
  $\stores$}  denoted
  $\condd{\pmf}{X}{\stores}$ is a distribution with domain $X$ where for all
  $\store \in \mems(X)$:
  $$
  \condd{\pmf}{X}{\stores}(\store) =
  (\sum_{\store' \in \stores} \margd{\pmf}{X}(\store \cap \store')) /
  (\sum_{\store' \in \stores} \margd{\pmf}{Y}(\store'))
  $$
  where $\margd{\pmf}{X}(\store \cap \store')) = 0$ if $\store \cap \store'$ is undefined.
\end{definition}
To recover familiar notation we allow the syntactic
sugarings $\condd{\pmf}{X}{\store}  \defeq \condd{\pmf}{X}{\{ \store\}}$, and
$\pmf(\store)  \defeq \margd{\pmf}{X}(\store)$ and $\pmf(\store|\stores) \defeq
\condd{\pmf}{X}{\stores}(\store)$ where $\dom(\store) = X$.
%\begin{eqnarray*}
%  \condd{\pmf}{X}{\store}  &\defeq& \condd{\pmf}{X}{\{ \store\}}\\
%  \pmf(\store)  &\defeq& \margd{\pmf}{X}(\store)  \qquad \dom(\store) = X\\
%  \pmf(\store|\stores)  &\defeq& \condd{\pmf}{X}{\stores}(\store) \qquad \dom(\store) = X
%\end{eqnarray*}

We also define the \emph{support} of a distribution in the usual manner-
it is the set of values a set of variables can take on with non-zero
probability.
\begin{definition}[Support]
  $\support(\pmf) \defeq \{ (v_1,\ldots,v_n) \mid
  \pmf(x_1 \mapsto v_1, \ldots, x_n \mapsto v_n) > 0 \} $
\end{definition}

\subsection{Basic Distribution of a Protocol}
Now we can define the probability distribution of a program $\prog$,
that we denote $\progtt(\prog)$. Since $\fedcat$ is deterministic the
results of any run are determined by the input values together with
the random tape. And since we constrain programs to not overwrite
views, we are assured that \emph{final} memories contain both a
complete record of all initial secrets as well as views resulting from
communicated information. 

Our semantics require that random tapes contain values for all program
values $\elab{\flip{w}}{\cid}$ sampled from a uniform distribution
over $\mathbb{F}_p$. Input memories also contain input secret values
and possibly other initial view elements as a result of
pre-processing, e.g., Beaver triples for efficient multiplication,
and/or MACed share distributions as in BDOZ/SPDZ
\cite{evans2018pragmatic,10.1007/978-3-030-68869-1_3}. We define
$\runs(\prog)$ as the set of final memories resulting from execution
of $\prog$ given any initial memory, and treat all elements of
$\runs(\prog)$ as equally likely.  This establishes the basic program
distribution that can be marginalized and conditioned to quantify
input/output information dependencies.
%In this
%setting, given a program $\prog$ with $\iov(\prog) = S \cup V$ and
%$\flips(\prog) = F$ we will consider all $\store \in \mems(S \cup V
%\cup F)$ such that $ \config{\store_{S \cup F}}{\prog} \redxs
%\config{\store_}{\varnothing} $ to be equally probable, establishing
%the basic distribution of the program. %From this, we can immediately
%derive the marginal distribution of $S \cup V$ to reason about
%dependencies between secrets and views.
\begin{definition}
  \label{def-progtt}
  \label{def-progd}
  \label{definition-progd}
  Given $\prog$ with $\secrets(\prog) = S$ and $\flips(\prog) = R$ and
  pre-processing predicate $\preproc$ on memories, define:
  $$
  \begin{array}{c}
    \runs(\prog) \defeq \\
    \{ \store \mid \exists \store_1 \in \mems(R) . 
    \exists \store_2 . \preproc(\store_2) \wedge
    %(\dom(\store) = \iov(\prog) \cup R) \wedge
    (\store_1 \cap \store_2,\prog) \redxs (\store,\varnothing) \}
  \end{array}
  $$
  By default, $\preproc(\store) \iff \dom(\store) = S$, i.e.,
  the initial memory contains all input secrets in a uniform
  marginal distribution. Then the \emph{basic distribution of $\prog$}, written $\progtt(\prog)$, is
  defined such that for all $\store \in \mems(\iov(\prog) \cup R)$:
  $$
  \progtt(\prog)(\store) =  1 / |\runs(\prog)| \ \text{if}\ \store \in \runs(\prog), \text{otherwise}\ 0
  $$
  
  %In some cases, we will also be concerned with the (joint)
  %probabilities of expression interpretation given a preceding program
  %execution, and we write $\progtt(\prog, \be)$ to denote the program
  %distribution $\progtt(\prog;\itv := \be)$ where $\itv$ is a
  %special variable that is never used in programs.
\end{definition}


\subsection{Honest and Corrupt Views}

Information about honest secrets can be revealed to corrupt clients
through messages sent from honest to corrupt clients, and through
publicly broadcast information from honest clients. Dually,
corrupt clients can impact protocol integrity through the messages
sent from corrupt to honest clients, and through publicly broadcast information
from corrupt clients. We call the former \emph{corrupt views}, and
the latter \emph{honest views}. Generally we let $V$ range over sets
of views.
\begin{definition}[Corrupt and Honest Views]
  Given a program $\prog$ with $\iov(\prog) = S \cup M \cup P \cup O$,
  define $\views(\prog) \defeq M \cup P$, and define $\houtputs$ as
  the messages and reveals in $V \defeq M \cup P$ sent from honest to corrupt
  parties, called \emph{corrupt views}:
  $$
  \begin{array}{lcl}
    \houtputs & \defeq
        & \{\ \rvl{w} \mid\ \reveal{w}{\be}{\cid} \in \prog \wedge \cid \in H \ \}\ \cup \\
      & & \{\ \elab{\mesg{w}}{\cid}\ \mid\  \eassign{\mesg{w}}{\cid}{\be}{\cid'} \in
           \prog \wedge \cid \in C \wedge \cid' \in H \ \} 
  \end{array}
  $$
  and similarly define $\cinputs$ as the subset of $V$ sent from corrupt to honest
  parties, called \emph{honest views}:
  $$
  \begin{array}{lcl}
    \cinputs &  \defeq
        & \{\ \rvl{w} \mid\ \reveal{w}{\be}{\cid} \in \prog \wedge \cid \in C \ \} \ \cup\\
      & & \{\ \elab{\mesg{w}}{\cid}\ \mid\  \eassign{\mesg{w}}{\cid}{\be}{\cid'} \in
              \prog \wedge \cid \in H \wedge \cid' \in C \ \}
  \end{array}
  $$
\end{definition}

\subsection{Passive Correctness and Security}

In the passive setting we assume that $H$ and $C$ follow the
rules of protocols and share messages as expected. A first
consideration is whether a given protocol is \emph{correct}
with respect to an ideal functionality. 
\begin{definition}[Passive Correctness]
  %Given $\prog$ with output variables $\out{1},\ldots,\out{n}$ and ideal
  We say that a protocol $\prog$ is \emph{passive correct} for a functionality
  $\idealf$ iff for all $\store \in \mems(\secrets(\prog))$
  we have $\progtt(\prog)(\idealf(\store) \mid \store) = 1$.
  %with $\idealf(\store) = v_1,\ldots,v_n$ we have
  %$\progtt(\prog)(\out{1} \mapsto v_1,\ldots,\out{n} \mapsto v_n \mid \store) = 1$.
\end{definition}

In the passive setting the simulator must construct a probabilistic
algorithm $\SIM$, aka a \emph{simulation}, that is parameterized by
corrupt inputs and the output of an ideal functionality, and that
returns a reconstruction of corrupt views that is probabilistically
indistinguishable from the corrupt views in the real world protocol
execution.
\begin{definition}
  Given $\store$, and $v$,we write $ \prob(\SIM(\store,v) = \store') $
  to denote the probability that $\SIM(\store,v)$ returns corrupt views
  $\store'$ as a result. We write $\dist(\SIM(\store,v))$ to
  denote the distribution of corrupt views reconstructed by the
  simulation, where for
  all $\store' \in \mems(V)$:
  $$
  \dist(\SIM(\store,v))(\store')\ \defeq\ \prob(\SIM(\store,v) = \store') 
  $$
\end{definition}
Then we can define passive security in the real/ideal
model as follows. 
\begin{definition}[Passive Security]
  Assume given a program $\prog$ that correctly implements an ideal
  functionality $\idealf$, with $\views(\prog) = V$.  Then $\prog$
  is \emph{passive secure in the simulator model} iff there exists
  ad simulation $\SIM$ such that for all
  partitions of the federation into honest and corrupt sets $H$ and $C$
  and for all $\store \in \mems(S)$:
  $$
  \dist(\SIM(\store_{C},\idealf(\store))) = \condd{\progtt(\prog)}{\houtputs}{\store}
  $$
\end{definition}

\subsection{Malicious Security}

In the malicious model we assume that corrupt clients are in the
thrall of an adversary $\adversary$ who does not necessarily follow
the rules of the protocol.  We model this by positing a $\arewrite$
function which is given a corrupt memory $\store_C$ and expression
$\be$, and returns a rewritten expression that can be interpreted to
yield a corrupt input. We define the evaluation relation that
incorporates the adversary in Figure \ref{fig-adversary}.

\adversaryfig

A key technical distinction of the malicious setting is that it
typically incorporates ``abort''. Honest parties implement strategies
to detect rule-breaking-- aka \emph{cheating}-- by using, e.g.,
message authentication codes with semi-homomorphic properties as in
BDOZ/SPDZ \cite{10.1007/978-3-030-68869-1_3}. If cheating is detected,
the protocol is aborted. To model this, we extend $\minifed$ with an
\ttt{assert} command and extend the range of memories with
$\bot$. Note that the adversary is free to ignore their own
assertions.
\begin{definition}
  We add assertions of the form $\elab{\assert{\phi(\be)}}{\cid}$ to the command
  syntax of $\minifed$, where $\phi$ is a decidable predicate on
  $\mathbb{F}_p$ and with operational semantics given in Figure
  \ref{fig-adversary}. We also extend the range of memories $\store$
  to $\mathbb{F}_p \cup \{ \bot \}$.
\end{definition}

It is necessary to add $\bot$ to the range of memories since
the possibility of abort needs to be reflected in adversarial
runs of a protocol. We can define $\aruns(\prog,\adversary)$
as the ``prefix'' memories that result from possibly-aborting
protocols, but we also need to ``pad out'' the memories
of partial runs with $\bot$, as we define in $\botruns(\prog,\adversary)$,
to properly reflect the contents of views and outputs even in case of abort. 
\begin{definition}
  \label{def-aprogd}
  \label{def-aprogtt}
  \label{definition-aprogd}
  Given program $\prog$ with $\iov(\prog) = S \cup V \cup O$ and $\flips(\prog) = R$, and
  any assumed pre-processing predicate $\preproc$ on memories, define:
  $$
  \begin{array}{c}
    \aruns(\prog) \defeq \\
    \{ \store \mid \exists \store_1 \in \mems(R) . 
    \exists \store_2 . \preproc(\store_2) \wedge
    %(\dom(\store) = \iov(\prog) \cup R) \wedge
    (\store_1 \cap \store_2,\prog) \aredxs (\store,\varnothing) \}
  \end{array}
  $$
  where by default, $\preproc(\store) \iff \dom(\store) = S$, and also define the following
  which pads out undefined views and outputs with $\bot$:
  $$
  \begin{array}{l}
    \botruns(\prog) \defeq \\
    \qquad \{ \store\{ x_1 \mapsto \bot, \ldots, x_n \mapsto \bot \} \mid \\
    \qquad \phantom{\{} \store \in \aruns(\prog) \wedge \{ x_1,\ldots,x_n\} = (V \cup O) - \dom(\store) \}
  \end{array}
  $$
  Then the \emph{$\adversary$ distribution of $\prog$}, written $\progtt(\prog,\adversary)$, is
  defined such that for all $\store \in \mems(\iov(\prog) \cup R)$:
  $$
  \progtt(\prog,\adversary)(\store) =  1 / |\botruns(\prog)| \ \text{if}\ \store \in \botruns(\prog), \text{otherwise}\ 0
  $$
\end{definition}

Given this preamble, we can define malicious simulation and malicious security
in a standard manner \cite{evans2018pragmatic}, as follows.
\begin{definition}[Malicious Simulation]
  Given a protocol $\prog$ with $\iov(\prog) = S \cup V \cup O$, honest and corrupt 
  clients $H$ and $C$, adversary $\adversary$, and honest inputs
  $\store \in \mems(S_H)$, the \emph{malicious simulation}  $\SIM(\store)$ has three phases:
  \begin{enumerate}
  \item In the first phase $\SIM_1$, $\adversary$ gives the
    simulator some $\store' \in \mems(S_C)$, and the simulator consults an
    oracle to compute $\idealf(\store \cup \store') \in \mems(O)$.
  \item In the second phase $\SIM_2$, the simulator is given the corrupt
    outputs $\idealf(\store \cup \store')_C$, which are again given to
    $\adversary$, who decides either to abort or not. If so, then the
    simulator is given $\sigma_{\mathit{out}} \defeq \{ \out{\cid} \mapsto \bot \mid \cid \in H \}$
    and arbitrary internal state $\Sigma$.
    Otherwise the simulator is given $\sigma_{\mathit{out}} \defeq \idealf(\store \cup \store')_H$
    and $\Sigma$.
  \item In the third phase $\SIM_3$, given $\store_{\mathit{out}}$ and $\Sigma$, the simulator
    finally outputs
    $\store_{\mathit{out}} \cup \store_{\mathit{views}}$ for some
    calculated $\store_{\mathit{views}} \in \mems(\houtputs)$.
  \end{enumerate}
\end{definition}

\begin{definition}[Malicious Security]
  We write $\dist(\SIM(\store))$ to
  denote the distribution of honest outputs and corrupt views reconstructed by the
  malicious simulation, where for
  all $\store'$:
  $$
  \dist(\SIM(\store))(\store')\ \defeq\ \prob(\SIM(\store) = \store') 
  $$
  Then a protocol $\prog$ with $\iov(\prog) = S \cup V \cup O$ is \emph{malicious
  secure} iff for all $H$, $C$, $\adversary$, and $\store \in \mems(S_H)$:
  $$
  \dist(\SIM(\store)) = \condd{\progtt(\prog,\adversary)}{\houtputs \cup O_H}{\store}
  $$  
\end{definition}


\section{Security Hyperproperties}
\label{section-hyper}

In this Section we formulate probabilistic versions of well-studied
hyperproperties of confidentiality and integrity, including
noninterference, gradual release, declassification, and robust
declassification.  We follow nomenclature developed in previous work
on characterizing declassification policies in deterministic settings
\cite{sabelfeld2009declassification}, but adapt them to our
probabilistic one.


%We demonstrate a soundness relation between noninterference and
%passive security, and between robust declassification and malicious
%security.
%Some previous work on security type enforcement
%\cite{6266151,almeida2018enforcing} and has considered similar relationships
%but mainly for aspects of passive security.

\subsection{Conditional Noninterference}

Since MPC protocols release some information about secrets through
outputs of $\idealf$, they do not enjoy strict noninterference.  As
discussed in Section \ref{section-lang}, public reveals and protocol
outputs are fundamentally forms of declassification.  But consistent
with other work \cite{8429300}, we can formulate a version of
probabilistic noninterference conditioned on output that is sound
for passive security. 
\begin{definition}[Noninterference modulo output]
  \label{definition-NIMO}
  We say that a program $\prog$ with $\iov(\prog) = S \cup V \cup O$
  satisfies \emph{noninterference modulo output}
  iff for all $H$ and $C$ and $\store_1 \in \mems(S_C \cup O)$ and $\store_2 \in \mems(\houtputs)$
  we have:
  $$
  \condd{\progtt(\prog)}{S_H}{\store_1} = \condd{\progtt(\prog)}{S_H}{\store_1 \uplus \store_2}
 $$
\end{definition}
%\begin{definition}[Noninterference modulo output]
%  \label{definition-NIMO}
%  We say that a program $\prog$ satisfies \emph{noninterference modulo output}
%  iff for all $H$ and $C$ and 
%  $\store_1,\store_2 \in \mems(S)$ we have:
%  $$
%  (\store_1 =_C \store_2 \ \wedge \ 
%  (\condd{\progtt(\prog)}{O}{\store_1} = \condd{\progtt(\prog)}{O}{\store_2}))
%  \implies 
%  (\condd{\progtt(\prog)}{\houtputs}{\store_1} = \condd{\progtt(\prog)}{\houtputs}{\store_2})
%  $$
%  where $\iov(\prog) = S \cup V \cup O$.
%\end{definition}
This conditional noninterference property implies that
corrupt views give the adversary no better chance of guessing honest
secrets than just the output and corrupt inputs do. So the simulator
can just arbitrarily pick any honest secrets that could have produced
the given outputs and run the protocol in simulation to reconstruct
real world corrupt views. This requires that the simulator can
tractably ``pre-image'' a given output of a functionality $\idealf$,
to determine the inputs that could have produced it. This equivalence
class is called a \emph{kernel} in recent work \cite{10.1145/3571740}.
\begin{definition}
  Given a functionality $\idealf$ and outputs $\store_{\mathit{out}}$, their 
  \emph{kernel}, denoted $\ik(\idealf,\store_{\mathit{out}})$ is
  $
  \{ \store\ |\ \idealf(\store) = \store_{\mathit{out}} \}
  $.
  We say that $\idealf$ is \emph{pre-imageable} iff $\ik(\idealf, \store_{\mathit{out}})$ for all
  $\store_{\mathit{out}}$ can be computed tractably.
\end{definition}
A soundness result for passive security can then be given as follows.
It is essentially the same as ``perfect passive NI security'' explored
in previous work \cite{8429300}.  
\begin{restatable}{theorem}{nimosecure}
  \label{theorem-nimo}
  Assume given pre-imageable $\idealf$ and a protocol $\prog$ that
  correctly implements $\idealf$.  If $\prog$ satisfies noninterference modulo output
  then $\prog$ is passive secure.
\end{restatable}


\subsection{Gradual Release}

Probabilistic noninterference is related to perfect secrecy and is
preserved by components of cryptographic protocols generally. It can
be expressed using probabilistic independence, aka separation,
\cite{darais2019language,barthe2019probabilistic}, and we adopt the
following notation to express independence:
\begin{definition}
%  We write $\vc{\pmf}{x}{y}$ iff $\pmf(\{ x \mapsto 0\}\ |\ \{ y \mapsto 0 \}) =
%  \pmf(\{ x \mapsto 1\}\ |\ \{ y \mapsto 1 \}) = 1$.
  We write $\sep{\pmf}{X_1}{X_2}$ iff for all
    $\store \in \mems(X_1 \cup X_2)$ we have
  $\margd{\pmf}{X_1 \cup X_2}(\store) =
  \pmf(\store_{X_1}) * \pmf(\store_{X_2})$
\end{definition}

In practice, MPC protocols typically satisfy a \emph{gradual
release} property \cite{sabelfeld2009declassification}, where messages
exchanged remain probabilistically separable from secrets, with only
declassification events (reveals and outputs) releasing information
about honest secrets.  A key difference is that while these
declassification events essentially define the policy in gradual
release, the ideal functionality sets the release policy for MPC
passive security, so its necessary to show that declassification
events respect these bounds.
\begin{definition}
  Given $H,C$, a protocol $\prog$ with $\iov(\prog) = S \cup M \cup P \cup O$
  satisfies \emph{gradual release} iff
  $\sep{\progtt(\prog)}{M_C}{S_H}$.
\end{definition}

\subsection{Integrity and Robust Declassification}

%\begin{definition}[Malicious Correctness]
%  A protocol $\prog$ with $\iov(\prog) = S \cup V \cup O$ is \emph{malicious correct} iff
%  $
%  \forall \adversary, \store \in \mems(S_H) \ .\ \support(\progtt(\prog)(O_H|\store)) \supseteq
%    \support(\progtt(\prog,\adversary)(O_H|\store))
%  $.
%\end{definition}

\emph{Integrity} is an important hyperproperty in security models that admit
malicious adversaries. Consistent with formulations in deterministic settings,
we have already defined protocol confidentiality as the preservation of low equivalence
(of secrets and views), and now we define protocol integrity as the preservation
of high equivalence (of secrets and views). Intuitively, this property says
that any adversarial strategy either ``mimics'' a passive strategy with some
choice of inputs or causes an abort.
\begin{definition}[Integrity]
  \label{def-integrity}
  We say that a protocol $\prog$ with $\iov(\prog) = S \cup V \cup O$ has
  \emph{integrity} iff for all $H$, $C$, and $\adversary$,
  if $\store \in \aruns(\prog)$ 
  then there exists $\store' \in \mems(S)$ with $\store_{S_H} = \store'_{S_H} $ and:
    $$
    \condd{\progtt(\prog,\adversary)}{X}{\store_{S_H \cup \cinputs}} =
    \condd{\progtt(\prog)}{X}{\store'}
    $$ 
  where $X \defeq (\houtputs \cup O_H) \cap \dom(\store)$. 
\end{definition}
A first important observation is that integrity preserves protocol correctness
for honest outputs, except for the possibility of abort. 
\begin{lemma}
  \label{lemma-malicious-correct}
  If a protocol $\prog$ with $\iov(\prog) = S \cup V \cup O$ is passive correct for
  $\idealf$ and
  has integrity, then for all $H$, $C$, $\adversary$, $\store_1 \in \mems(S_H)$,
  $\ox{\cid} \in (O_H)$, and $\mv \in \mathbb{F}_p$, if:
  $$
  \progtt(\prog,\adversary)(\{ \ox{\cid} \mapsto \mv \} \mid \store_1) > 0
  $$
  then exists $\store_2 \in \mems(S_C)$ such that:
  $$
  \idealf(\store_1 \uplus \store_2)(\ox{\cid}) = \mv
  $$
\end{lemma}
The following result establishes that integrity implies malicious
security for protocols that are passive secure (which also subsumes
correctness). 
\begin{theorem}
  \label{theorem-integrity}
  If a protocol is passive secure and has integrity, then it
  is malicious secure.
\end{theorem}

\begin{proof}
  Let $\prog$ be some protocol with passive security and integrity
  where $\iov(\prog) = S \cup V \cup O$, and let $\adversary$ be some
  adversary. Suppose $\store \in \aruns(\prog,\adversary)$.
  As integrity requires, and as Lemma \ref{lemma-malicious-correct}
  demonstrates with respect to outputs, the most the adversary can do
  in the presence of integrity is to elicit the same responses from
  the honest parties-- via the strategy $\store_{\cinputs}$-- as
  are elicited from some passive run of the protocol using
  some $\store' \in \mems(S)$ where $\store'_H = \store_{S_H}$,
  and perhaps to force an abort after some number of message
  exchanges.

  Therefore, in simulation, $\adversary$ can provide the simulator
  with some $\store'_C$ in $\SIM_1$ which its strategy is ``impersonating'',
  allowing $\idealf(\store')$ to
  be communicated to $\adversary$ in $\SIM_2$ who can then
  decide whether or not to abort. In the case of abort, the
  subset of $\houtputs$ to be defined can be communicated to
  $\SIM_3$, along with $\idealf(\store')$, via $\Sigma$. 
  In $\SIM_3$, the simulator can then run $\prog$ in simulation
  with inputs $\store'_C$ and arbitrary $\store'' \in \mems(S_H)$
  such that $\store'_C \uplus \store'' \in \ik(\idealf,\idealf(\store'))$.
  The assumption of passive security of $\prog$ implies the result.
\end{proof}

The hyperproperty of robust declassification \cite{930133} similarly
combines a confidentiality property with integrity to establish that
malicious actors cannot declassify more information than is intended
by policy. But in this prior work, this policy is established
by the declassifications themselves, as in gradual release.
Thus, we can define a robust declassification property as follows. 
\begin{definition}[Robust Declassification]
  A protocol satisfies \emph{robust declassification} iff it has integrity and
  satisfies gradual release. 
\end{definition}
However, it is important to note that gradual release, and
hence robust declassification, are not sufficient to establish
passive or malicious simulator security, where the declassification
policy is established by the ideal functionality $\idealf$. 
\begin{theorem}
  Robust declassification does not imply malicious security, but
  passive security with robust declassification implies malicious security.
\end{theorem}



\section{Brute Force Verification}
\label{section-bruteforce}

We have shown how our formulation of program distributions, a precise
definition of intensional protocol behavior, can be used to enforce
the extensional property of MPC security. This intensional description
also has the benefit of being flexible- we can use the same mechanisms
to desribe $\NIMO$, noninterference, perfect secrecy, violations of
obliviousness, etc. These mechanisms can also be exploited to support
automated program analysis.

We initially explore a brute force approach to automatically verifying
program behaviour, the first step of which is to compute the basic
distribution of a program $\prog$. The program distribution
$\progd(\prog)$ can obviously be automatically derived from that, as
can be any of its marginal or conditional distributions. In this
Section we consider two automated mechanisms for deriving
$\progd(\prog)$ in this way. The first is through a straightforward
computation of all final program memories.  The second approach
rewrites $\minifed$ programs to stratified Datalog programs, that are
amenable HPC optimizations such as parallelization and GPU matrix
computations as shown in recent work \cite{XXX}.

The basic distribution of a program $\prog$ is based on the
final memories resulting from any run of $\prog$, denoted
$\runs(\prog)$.
\begin{definition}
  Given $\prog$ with $\iov(\prog) = S \cup V$ and $\flips(\prog) = F$:
  $$
  \runs(\prog) \defeq \{ \store \in \mems(S\cup F \cup V) \mid \config{\store_{S \cup F}}{\prog} \redxs \config{\store}{\varnothing} \}
  $$
\end{definition}
As we've observed, $|\runs(\prog)| = 2^{|S \cup F|}$, so the size of
$\runs(\prog)$ is exponential in the amount of randomness used in the
protocol which generally can be used as a proxy for the protocol
size. In Section \ref{section-composition} we will show how the
methods described here can be used to individually verify
compositional properties of small program components, that guarantee
security properties in larger protocols that combine them. Thus, while
scalability to program size is a clear challenge for our brute force
techniques alone, our ultimate goal is to use them for verification of
compositional properties in larger programs.

\begin{fpfig}[t]{Filtering memories that satisfy a boolean expression.}{fig-solve}
\begin{eqnarray*}
\solve{\stores}{\etrue} &=& \stores\\
\solve{\stores}{\efalse} &=& \varnothing\\
\solve{\stores}{\flip{\cid}{w}} &=& \{ \store \in \stores \mid \store(\flip{\cid}{w}) \} \\
\solve{\stores}{\secret{\cid}{w}} &=& \{ \store \in \stores \mid \store(\secret{\cid}{w}) \} \\
\solve{\stores}{\view{\cid}{w}} &=& \{ \store \in \stores \mid \store(\view{\cid}{w}) \} \\
\solve{\stores}{\oracle{w}} &=& \{ \store \in \stores \mid \store(\oracle{w}) \} \\
\solve{\stores}{(\enot\ \be)} &=& \stores - (\solve{\stores}{\be})\\
\solve{\stores}{(\be_1\ \eand\ \be_2)} &=& (\solve{\stores}{\be_1}) \cap (\solve{\stores}{\be_2}) \\
\solve{\stores}{(\be_1\ \eor\ \be_2)} &=& (\solve{\stores}{\be_1}) \cup (\solve{\stores}{\be_2}) \\
\solve{\stores}{(\be_1\ \exor\ \be_2)} &=&
 ((\solve{\stores}{\be_1}) \cap (\stores - \solve{\stores}{\be_2})) \cup\\
 && ((\stores - \solve{\stores}{\be_1}) \cap (\solve{\stores}{\be_2})) \\
\solve{\stores}{\select{\be_1}{\be_2}{\be_3}} &=&
 ((\solve{\stores}{\be_1}) \cap (\solve{\stores}{\be_2})) \cup \\
 && ((\stores - \solve{\stores}{\be_1}) \cap (\solve{\stores}{\be_3})) \\
\solve{\stores}{\OT{\be_1}{\be_2}{\be_3}} &=&
 ((\solve{\stores}{\be_1}) \cap (\solve{\stores}{\be_2})) \cup\\
 && ((\stores - \solve{\stores}{\be_1}) \cap (\solve{\stores}{\be_3}))
\end{eqnarray*}
\end{fpfig}

\subsection{Method 1: Enumerating $\runs(\prog)$}

Here we describe a straightforward technique for obtaining program
distributions $\progd(\prog)$ through direct computation of
$\runs(\prog)$. An implementation of these techniques along with
Examples from Sections \ref{section-minicat-examples} and
\ref{section-metalang-ygc} is available online \cite{XXX}.

In Figure \ref{fig-solve} we define the algorithm $\solve{\stores}{\be}$
which filters $\stores$ based on satisfaction of $\be$. This algorithm
is fundamental to our brute force methods. It's correctness is characterized
as follows. 
\begin{lemma}
  For all $\stores$ and $\be$ with $\vars(\be) \subseteq \dom(\stores)$,
  $(\solve{\stores}{\be}) = \{ \store \in \stores \ \mid\ \lcod{\store,\be}{\cid} = 1 \}$
  for some $\cid$.
\end{lemma}
\begin{proof}
  Since we assume safety of programs we can assume that all variables in $\be$ have the
  same owner $\cid$. The result otherwise follows in a straightforward manner by induction
  on $\be$. 
\end{proof}

\begin{lemma}
  Given $\prog$ where $\iov(\prog) = S \cup V$ and $\flips(\prog) = F$. Define:
  \begin{eqnarray*}
    {tt}\ \ \stores\ (\eassign{\view{\cid}{w}}{\be}) &\defeq& \begin{array}{l}
      \mathrm{let}\ \stores' = \solve{\stores}{\be} \ \mathrm{in}\\
      \ \ \{\extend{\store}{\view{\cid}{w}}{\etrue} \mid \store \in \stores' \}\ \cup\\
      \ \ \{\extend{\store}{\view{\cid}{w}}{\efalse} \mid \store \in \stores - \stores' \}\end{array}\\[2mm]
    \cruns(\prog) &\defeq& \mathit{foldl}\ {tt}\ \mems(S \cup F)\ \prog
  \end{eqnarray*}
  Then $\cruns(\prog) = \runs(\prog)$.
\end{lemma}

\begin{lemma}
  Given $\prog$ where $\iov(\prog) = S \cup V$ and $\flips(\prog) = F$. Define:
  $$
  \pdf{S \cup F \cup V}(\store) = \begin{cases}1/2^{|S \cup F|} & \text{if\ } \store \in
    \cruns(\prog) \\ 0 & \text{otherwise} \end{cases} 
  $$
  Then:
  $$
  \margd{\pdf{S \cup F \cup V}}{S \cup V} = \progd(\prog)
  $$
\end{lemma}

\subsection{Method 2: Rewriting to Datalog}

$$
f(\eassign{\view{\cid}{w}}{\be}) = (\view{\cid}{w}, (\solve{(\mems(\vars\ \be))}{\be}))
$$

$$
\mathit{map}\ f\ \prog
$$

\begin{mathpar}
  \logit{x \mapsto 1} = x

  \logit{x \mapsto 0} = \neg x

  \logit{\{ x_1 \mapsto \beta_1, \ldots, x_n \mapsto \beta_n\}} =
  \logit{x_1 \mapsto \beta_1} \wedge \cdots \wedge \logit{x_n \mapsto \beta_n}

  \mathit{clauses}(x,\{ \store_1,...,\store_n \}) = x \gets \logit{\store_1} \vee \cdots \vee x \gets \logit{\store_n}
\end{mathpar}

$$
\mathit{map}\ \mathit{clauses}\ (\mathit{map}\ f\ \prog)
$$


\section{The $\metaprot$ Metalanguage}
\label{section-metalang}

\begin{fpfig}[t]{Syntax of $\metaprot$.}{fig-metaprot-syntax}
$$
\begin{array}{rcl@{\hspace{8mm}}r}
\flab &\in& \mathrm{Field}\\
x &\in& \mathrm{EVar}\\
f &\in& \mathrm{FName}\\[2mm]
e &::=& b \mid \flip{e}{e} \mid \secret{e}{e} \mid \view{e}{e} \mid \oracle{e} \mid \enot\ e \mid e\ \eand\ e \mid e\ \exor\ e \mid & \textit{expressions}\\[0mm]
& & \select{e}{e}{e} \mid 
\send{\view{e}{e}}{e} \mid \send{\view{e}{e}}{\OT{e}{e}{e}} \mid e;e \mid \\[0mm]
& & x \mid \elet{x}{e}{e} \mid f(e,\ldots,e) \mid \{ \flab = e; \ldots; \flab = e \}
\mid e.\flab \mid e\concat e \mid (e) \\[2mm]
v &::=& w \mid \cid \mid \be \mid \{ \flab = v;\ldots;\flab = v \} 
\mid \ttt{()} & \textit{values}\\[2mm]
{fn} &::=& f(x,\ldots,x) \{ e \} & \textit{functions}
\end{array}
$$
\end{fpfig}

Large practical MPC computations are based on much larger protocols
than the examples we've considered so far. These larger protocols are
typically based on compositional units. An example of this is Yao's
Garbled Circuits (YGC), which are composed of so-called garbled gates.
Languages for defined garbled circuits, beginning with Fairplay \cite{269581},
treat gates as compositional units that are wired together by the programmer
to generate a complete circuit. The $\fedprot$ language is low-level
and does not include abstractions for defining composable elements. 

In this Section, we introduce the $\metaprot$ language which includes
structured data and function definitions, which are sufficiently
expressive to define composable protocol elements such as garbled
gates. The $\metaprot$ language is a \emph{metalanguage}, in the sense
that it produces $\fedprot$ protocols as a result of computation. That
is, $\metaprot$ is a high-level language that generates low-level
protocol code.

\subsection{Syntax}

The syntax of $\metaprot$ is defined in Figure
\ref{fig-metaprot-syntax}.  It includes a syntax of function
definitions and records, and values include client ids, identifier
strings, and boolean expressions. Expression forms allow dynamic
construction of boolean expression forms and view assignments. When
$\metaprot$ programs construct an $\fedprot$ assignment, a side effect
occurs whereby the assignment is added to the end of the $\fedprot$
program accumulated during evaluation.

Formally, we consider a complete metaprogram to include both a
codebase and a ``main'' program that uses the codebase. 
\begin{definition}
A \emph{codebase} $\codebase$ is a list of function 
declarations. We write $ \codebase(f) = x_1,\ldots,x_n,\ e$
iff $f(x_1,\ldots,x_n) \{ e \} \in \codebase$.
A \emph{metaprogram}, aka \emph{metaprotocol}  is a pair of a 
codebase and expression $\codebase, e$. We may omit
$\codebase$ if it is clear from context.  
\end{definition}

When we consider the example of YGC in detail below, our focus will be
on developing a codebase that can be used to define arbitrary
circuits, i.e., complete and concrete protocols. Since strings and
identifiers can be constructed manually, and expressions can occur
inside assignments and boolean expression forms, function definitions
can generalize over $\fedprot$-level patterns to obtain composable
program units. As a simple example, consider 3 party secret
sharing as illustrated in Example \ref{example-he}. We can
define a function $\ttt{share3}$ that abstracts the process
of splitting a given client's secret into 3 separate shares.
\begin{example} \label{example-share3} The function $\ttt{share3}$ 
  splits a client's secret into 3 shares returned as a record
  with fields $\ttt{s1-3}$:
  \begin{verbatimtab}
    share3(client, secretid)
    {
      let s1 = flip[client, share1] in
      let s2 = flip[client, share2] in
      let s3 = (s1 xor s2) xor s[client, s: || secretid] in
      {s1 = s1;s2 = s2;s3 = s3}
    } \end{verbatimtab}
  Here is a $\metaprot$ program that uses this function definition:
  \begin{verbatimtab}
    let shares = share3(1, mysecret) in
    v[2,s1] := shares.s2;
    v[3,s1] := shares.s3 \end{verbatimtab}
  which generates the following $\minifed$ program, as we formalize in Example \ref{example-share3-eval}
  below:
  \begin{verbatimtab}
    v[2,s1] := flip[1, share2];
    v[3,s1] := flip[1, share1] xor flip[1, share2] xor s[1, s:mysecret] \end{verbatimtab}
\end{example}

\subsection{Semantics}

\begin{fpfig}[t]{Evaluation contexts and operational semantics of $\metaprot$.}{fig-metaprot-semantics}
$$
\begin{array}{rcl@{\hspace{3mm}}r}
E &::=& [\,] \mid \enot\ E \mid E\ \bop\ e \mid v\ \bop\ E \mid  \flip{E}{e} \mid \secret{E}{e} \mid \view{E}{e} \mid \oracle{E} \mid  \\[1mm]
& & \flip{\cid}{E} \mid \secret{\cid}{E} \mid \view{\cid}{E} \mid \send{E}{e} \mid \send{\view{\cid}{w}}{E} \mid \OT{E}{e}{e} \\[1mm]
& & \mid \OT{v}{E}{e} \mid \OT{v}{v}{E} \mid \select{E}{e}{e} \mid \select{v}{E}{e} \mid \\[1mm]
& & \select{v}{v}{E} \mid \elet{x}{E}{e} \mid f(v,\ldots,v,E,e,\ldots,e) \mid \\[1mm]
& & \{ \flab = v;\ldots;\flab = v;\flab = E;\flab = e;\ldots;\flab = e \} \mid E.\flab \mid E\concat e \mid v \concat E
\end{array}
$$
\medskip
$$
\begin{array}{rcl@{\hspace{10mm}}r}
\config{\prog}{\elet{x}{v}{e}} &\redx& \config{\prog}{e[v/x]}\\
\config{\prog}{f(v_1,...,v_n)} &\redx&
\config{\prog}{e[v_1/x_1,\ldots,v_n/x_n]} & 
 \codebase(f) = x_1,\ldots,x_n,\ e\\
\config{\prog}{\{\ldots; \flab = v; \ldots\}.\flab} &\redx&
 \config{\prog}{v}\\
 \config{\prog}{w_1\concat w_2} &\redx& \config{\prog}{w_1w_2}\\
 \config{\prog}{v;e} &\redx& \config{\prog}{e}\\
\config{\prog}{\instr} &\redx& \config{\prog;\instr}{()}\\
\config{\prog}{E[e]} &\redx& \config{\prog'}{E[e']} & \text{if}\ \config{\prog}{e} \redx \config{\prog'}{e'} 
\end{array}
$$
\end{fpfig}

We define a small-step evaluation aka reduction relation $\redx$ in
Figure \ref{fig-metaprot-semantics}.  We write $\redxs$ to denote the
reflexive, transitive closure of $\redx$. Reduction is defined on
\emph{configurations} which are pairs of the form $\config{\prog}{e}$,
where $\prog$ is the $\minifed$ program accumulated during evaluation.
In this definition we write $e[v/x]$ to denote the substitution of $v$
for free occurences of $x$ in $e$. The rules are mostly standard,
except when a concrete $\minifed$ assignment is encountered it is added
to the end of $\prog$.

The rules rely on a definition of \emph{evaluation contexts} $E$
allowing computation within a larger program context, where $E[e]$
denotes an expression with $e$ in the hole $[]$ of $E$. Evaluation
contexts include boolean expression forms, allowing generalization
and instantiation of compositional program elements.
\begin{example}
  \label{example-share3-eval}
  Let $\codebase,e_{\ref{example-share3}}$ be the $\metaprot$ program and let 
  $\prog_{\ref{example-share3}}$ be the  $\minifed$ program defined
  in Example \ref{example-share3}. We refer to the latter as ``accumulated''
  by evaluation of the former in the sense that $\config{\varnothing}{e_{\ref{example-share3}}}
  \redxs \config{\prog_{\ref{example-share3}}}{\ttt{()}}$.
\end{example}

\subsection{Type Theory and Static Type Safety}

It is desirable to statically enforce safety of both $\metaprot$
programs and the safety of the $\fedprot$ programs they
generate. Although safety of the latter could be enforced
post-generation by a direct analysis, for large programs this can be
much more expensive and it is also better to not waste time on
resource-intensive compilation of programs with known errors
\cite{kreuter2012billion}. Some consequences of safety errors, for example, accidental
reuse of one-time pads can also undermine security.

The type syntax of $\metaprot$ is defined in Figure
\ref{fig-metaprot-tsyntax}. It includes a weak form of dependency:
string types $\stringty{e}$ and client types $\cidty{e}$
are parameterized by expressions $e$ that precisely reflect the
type of the value. Boolean expression forms have the type
$\bet{e}$ indexed by expressions $e$ indicating the client
id type of the expression. The dependency is weak in the
sense that expressions in types are a strict subset of
expression forms- for any $\stringty{e}$ the expression $e$
is either a variable, a string, or a concatenation form,
and for any $\cidty{e}$ or $\bet{e}$ the expression $e$
is either a variable or a client id $\cid$. 

Type judgements for expressions are of the form
$\tjudge{\viewst_1}{\gamma}{e}{\viewst_2}$ where the \emph{view
effect} $\viewst_1$ denotes the views that have been defined so far,
and $\viewst_2$ records new views defined as the effect of the
expression on the residual $\minifed$ program.  Type judgements are
syntax-directed- selected rules are shown in Figure
\ref{fig-metaprot-tjudge}. The $\TirName{AssignT}$ rule
captures the effect of new view definitions. The
$\TirName{And}$ rule illustrates how program safety is enforced,
by ensuring that subexpressions of boolean expressions have the
same owner.

The $\TirName{FnT}$ and $\TirName{Appt}$ rules apply to function
definition and application respectively, and rely on the
definition of function input type annotations $\tas$ and
type term substitutions $\sigma$. 
\begin{definition}
  A \emph{function input type annotation} $\tas$ is a mapping from
  function names $f$ to type products $\tau_1 * \cdots * \tau_n$.
  A \emph{type term substitution} $\sigma$ is a mapping from
  $\minifed$ variables $x$ to values, where $\sigma(\tau)$ denotes
  the replacement of occurences of $x$ in $\tau$ with $\sigma(x)$. 
\end{definition}
We assume that input type annotations $\tas$ are provided by the
programmer for all function definitions. This guarantees that
$\metaprot$ type checking is straightforward and efficient.
Function types are of the form:
$$
\tau_1 * \cdots * \tau_n \rightarrow \tau,\viewst
$$
where $\viewst$ denotes the effect of the function on the residual
program.  The function type can be understood as a dependent $\Pi$
type, with every term variable bound. When applied, these variables
are instantiated with a type term substitution $\sigma$. In our
implementation, we adapt \emph{synthesis} as defined in Dependent ML \cite{10.1145/292540.292560}
to obtain this $\sigma$- essentially this is a match on the syntactic
structure of types and expressions. 
\begin{example}
  Given $\ttt{share3}$ as defined in Example \ref{example-share3} and
  annotation:
  $$\tas(\ttt{share3}) =  \ttt{cid(client)}\ *\ \ttt{string(sid)}$$
  the type of $\ttt{share3}$ is $\tas(\ttt{share3}) \rightarrow \tau,\varnothing$
  where $\tau$ is:
  $$
  \ttt{\{ s1 : bool[client]; s2 : bool[client]; s3 : bool[client] \}}
  $$
\end{example}

\begin{fpfig}[t]{Type Syntax of $\metaprot$.}{fig-metaprot-tsyntax}
$$
\begin{array}{rcl@{\hspace{2mm}}r}
\srct &::=& \cidty{e} \mid \stringty{e} \mid \bet{e} \mid  & \gdesc{types}\\ 
 &&  \{ \flab : \srct;\ldots;\flab : \srct \} \mid \tau * \cdots * \tau \rightarrow \tau,\viewst \\[1mm]
\viewst  &::=& \view{e}{e};\viewst \mid \varnothing   & \gdesc{view effects}\\[1mm]
\Gamma &::=& \Gamma; x : \tau \mid \varnothing & \gdesc{type environments}    
\end{array}
$$
\end{fpfig}

\begin{fpfig}[t]{Selected $\metaprot$ type judgement rules.}{fig-metaprot-tjudge}
\begin{mathpar}
\inferrule[\TirName{VarT}]
{}
{\tjudge{\viewst}{\Gamma}{x}{\Gamma(x)}{\viewst}}

\inferrule[\TirName{CidT}]
{}
{\tjudge{\viewst}{\Gamma}{\cid}{\cidty{\cid}}{\viewst}}

\inferrule[\TirName{StringT}]
{}
{\tjudge{\viewst}{\Gamma}{w}{\stringty{w}}{\viewst}}

\inferrule[\TirName{ConcatT}]
{\tjudge{\viewst}{\Gamma}{e_1}{\stringty{e_1'}}{\viewst_1}\\
\tjudge{\viewst_1}{\Gamma}{e_2}{\stringty{e_2'}}{\viewst_2}
}
{\tjudge{\viewst}{\Gamma}{e_1||e_2}{\stringty{e_1' ||e_2'}}{\viewst_2}}

\inferrule[\TirName{BoolT}]
{}
{\tjudge{\viewst}{\Gamma}{\etrue}{\bet{\cid}}{\viewst}}

\inferrule[\TirName{OracleT}]
{\tjudge{\viewst}{\Gamma}{e}{\stringty{e'}}{\viewst'}}
{\tjudge{\viewst}{\Gamma}{\oracle{e}}{\bet{\cid}}{\viewst'}}

\inferrule[\TirName{SecretT}]
{\tjudge{\viewst}{\Gamma}{e_1}{\cidty{e_1'}}{\viewst_1}\\
\tjudge{\viewst_1}{\Gamma}{e_2}{\stringty{e_2'}}{\viewst_2}}
{\tjudge{\viewst}{\Gamma}{\secret{e_1}{e_2}}{\bet{e_1'}}{\views_2}}

\inferrule[\TirName{AndT}]
{
\tjudge{\viewst}{\Gamma}{e_1}{\bet{e}}{\viewst_1}\\
\tjudge{\viewst_1}{\Gamma}{e_2}{\bet{e}}{\viewst_2}
}
{\tjudge{\viewst}{\Gamma}{e_1\ \eand\ e_2}{\bet{e}}{\viewst_2}}

\inferrule[\TirName{AssignT}]
{
\tjudge{\viewst}{\Gamma}{e_1}{\cidty{e_1'}}{\viewst_1}\\
\tjudge{\viewst_1}{\Gamma}{e_2}{\stringty{e_2'}}{\viewst_2}\\
\tjudge{\viewst_2}{\Gamma}{e_3}{\bet{e_3'}}{\viewst_3}
}
{
\tjudge{\viewst}{\Gamma}{\eassign{\view{e_1}{e_2}}{e_3}}{\unity}{(\viewst_3 ; \view{e_1'}{e_2'} )}
}

\inferrule[AppT]
{\Gamma(f) =  
 \tau_1 * \cdots * \tau_n \rightarrow \tau, \viewst_f \\ 
 \tjudge{\viewst}{\Gamma}{e_1}{\sigma(\tau_1)}{\viewst_1}
 \ \cdots\  
 \tjudge{\viewst_{n-1}}{\Gamma}{e_n}{\sigma(\tau_n)}{\viewst_n}}
{\tjudge{\viewst}{\Gamma}{f(e_1,\ldots,e_n)}{\sigma(\tau)}{(\viewst_n ; \sigma(\viewst_f))}}

\inferrule[FnT]
{
  \codebase(f) = x_1,\ldots,x_n,\ e \\ \tas(f) = \tau_1 * \cdots * \tau_n \\
  \tjudge{\varnothing}{\Gamma; x_1 : \tau_1; \ldots; x_n : \tau_n}{e}{\tau}{\viewst_f}
}
{ \Gamma \vdash f : \tau_1 * \cdots * \tau_n \rightarrow \tau, \viewst_f }

\inferrule[ProgT]
{
\forall f \in \dom(\codebase)\ .\ \Gamma \vdash f : \Gamma(f) \\ \Gamma \vdash e : \tau, \view{\cid_1}{w_1};\ldots;\view{\cid_1}{w_1}
}
{
\Gamma \vdash \codebase,e : \tau,\{\view{\cid_1}{w_1}\} \sqcup \cdots \sqcup \{ \view{\cid_n}{w_n}\}
}
\end{mathpar}
\end{fpfig}

Top-level type judgements are of the form $\Gamma \vdash \codebase, e
: \tau, V$, where all the functions in $\codebase$ are well-typed in
$\Gamma$, the top level view effect $V$ is a set of concrete
$\fedprot$ views which are constructed by a disjoint union of the views
in the effect of $e$- the disjointness requirement guarantees that
views are uniquely defined. Our $\metaprot$ type safety result is
formulated as follows. In addition to safe execution of the
metaprogram, it also guarantees safety of the residual $\fedprot$
program. Of course, it is important to note that ``safety'' here
means the usual type safety property that programs do not get stuck,
it does not imply any security hyperproperties. 
\begin{theorem}[$\metaprot$ Type Safety]
  \label{theorem-metalang-safety}
  Given $\codebase$, $e$, and $\Gamma$ with $\Gamma \vdash \codebase,e : \unity : V$,
  then $\config{\varnothing}{e} \redxs \config{\prog}{\ttt{()}}$ where
  $\prog$ is safe with $\iov(\prog) = S \cup V$ for some $S$.
\end{theorem}

\subsection{Example: GMW}
\label{section-metalang-gmw}

\begin{fpfig}[t]{2-Party GMW circuit library with And gate (T), and type annotations (B).}{fig-gmw}
{\footnotesize
  \begin{verbatimtab}
    encodegmw(in1,in2) {
      v[1, in1++"out"] := s[1,in1] xor flip[1,in1];
      v[2, in1++"out"] := flip[1,in1];
      v[1, in2++"out"] := s[2,in2] xor flip[2,in2];
      v[2, in2++"out"] := flip[2,in2]
      { shares1 = { c1 = v[1, in1++"out"]; c2 = v[2, in1++"out"] };
        shares2 = { c1 = v[1, in2++"out"]; c2 = v[2, in2++"out"]} } 
    }
    
    andtablegmw(b1, b2) {
      let r11 = (b1 xor true) and (b2 xor true) in
      let r10 = (b1 xor true) and (b2 xor false) in
      let r01 = (b1 xor false) and (b2 xor true) in
      let r00 = (bl xor false) and (b2 xor false) in
      { v1 = r11; v2 = r10; v3 = r01; v4 = r00 }
    }
    
    andgmw(g, shares1, shares2) {
      let r = flip[1,g++".r"] in
      let table = andtablegmw(shares1.c1, shares2.c1) in
      let r11 =  r xor table.v1 in
      let r10 =  r xor table.v2 in
      let r01 =  r xor table.v3 in
      let r00 =  r xor table.v4 in
        v[2,g++"out"] := OT4(shares1.c2, shares2.c2, r11, r10, r01, r00);
      v[1,g++"out"] := r;
      { c1 = v[1,g++"out"]; c2 = v[2,g++"out"]}
    }
    
    decodegmw(shares) { v[0,"output"] := shares.c1 xor shares.c2 }   \end{verbatimtab}
}
\end{fpfig}


\begin{fpfig}[t]{GMW library type annotations.}{fig-gmw-types}
{\footnotesize
  \begin{verbatimtab}
   encodegmw   : string(gid) * string(gid)
    
   andtablegmw : { k = bool[i]; p = bool[i] }
    
   andgmw      : string(gid) *  { c1 = bool[1]; c2 = bool[2] } * { c1 = bool[1]; c2 = bool[2] }
    
   decodegmw   : { c1 = bool[1]; c2 = bool[2] }  \end{verbatimtab}
}
\end{fpfig}


The GMW protocol uses secret sharing to represent data flowing through
circuits. In the 2-party case, clients 1 and 2 each share their input
secrets, and use those shares to represent inputs to gates. Outputs
are also represented as shares. We refer to the pair of shares
representing any particular value as a \emph{wire value}, and
we represent them via records of the form
$
\ttt{\{c1 = } v_1\ttt{;c2 = } v_2\ttt{\}} 
$
where $v_1$ and $v_2$ are client 1 and 2's shares respectively.

For full details of the GMW protocol the reader is referred to
\cite{evans2018pragmatic}. Our implementation libary is shown in
Figure \ref{fig-gmw}, with type signatures for the library functions
shown in Figure \ref{fig-gmw-types}. We show the And gate since it is
an interesting component. The Figure includes the
following top-level functions:
\begin{itemize}
\item \ttt{encodegmw}: This function encodes client 1's and client 2's
  secret bits called $\ttt{s[1,s1]}$ and $\ttt{s[1,s2]}$ into two
  distinct wire values (pairs of shares).
\item \ttt{andgmw}: This function defines the gate $\ttt{g}$, the
  identifier $\ttt{g}$ being used to distinguish randomness used
  within.  In our version client 1 builds the output table (using
  \ttt{andtablegmw})and transfers the correct output share to client 2
  using 1-out-of-4 OT as per standard GMW protocol.
\item \ttt{decode}: This function decodes and publishes a wire value
  by $\ttt{xor}$ing the shares. Note that this requires both client 1
  and 2 to publicize their shares.
\end{itemize}
\begin{example}
  \label{example-gmw-andcircuit}
The following program uses our GMW library to define
a circuit with a single And gate and input secrets $\ttt{s1}$ and
$\ttt{s2}$ from client's 1 and 2 respectively. 
\begin{verbatimtab}
  let ss = encodegmw(s1,s2) in
  v[0,output] := decode(andgmw(0,ss.shares1,ss.shares2))
\end{verbatimtab}
\end{example}

\subsection{Example: YGC}
\label{section-metalang-ygc}

\begin{fpfig}[t]{Yao's Garbled Circuits, auxiliary functions.}{fig-ygc-aux}
{\footnotesize
\begin{verbatimtab}
  keygen(gid, b1, b2) { select4(b1,b2,H[gid || "1"],H[gid || "2"],H[gid || "3"],H[gid || "4"]) }
  
  keysgen(gid, b1, b2)
  {
    let k11 = keygen(gid,b1,b2) in
    let k10 = keygen(gid,b1,not b2) in
    let k01 = keygen(gid,not b1,b2) in
    let k00 = keygen(gid,not b1,not b2) in
    {k11 = k11; k10 = k10; k01 = k01; k00 = k00}
  }
  
  andtable(keys, bt, ap, bp)
  {
    let r11 = (keys.k11 xor bt) in 
    let r10 = (keys.k10 xor (not bt)) in
    let r01 = (keys.k01 xor (not bt)) in
    let r00 = (keys.k00 xor (not bt)) in
    permute4(ap,bp,r11,r10,r01,r00)
  }
  
  sharetable(gid, vid, table)
  {   
    v[1, "gate:" || gid || vid || "1"] := table.v1;
    v[1, "gate:" || gid || vid || "2"] := table.v2;
    v[1, "gate:" || gid || vid || "3"] := table.v3;
    v[1, "gate:" || gid || vid || "4"] := table.v4
  }
\end{verbatimtab}
}
\end{fpfig}

\begin{fpfig}[t]{Yao's Garbled Circuits, garbled gates and evaluation code.}{fig-ygc-gates}
{\footnotesize
\begin{verbatimtab}
  garbledecode(wl)
  {
    let r1 = wl.k xor true in
    let r0 = (not wl.k) xor false in
    v[1,"OUTtt1"] := select[wl.p,r1,r0];
    v[1,"OUTtt2"] := select[not wl.p,r1,r0]
  }
  
  evaldecode(wl, p) { wl.k xor select[wl.p,v[1,"OUTtt1"],v[1,"OUTtt2"]] }
  
  evalgate(gid, wla, wlb)
  {
    let k = keygen(gid,wla.k,wlb.k) in
    let ct = select4(wla.p,wlb.p,
               v[1,gid || "tt1"],v[1,gid || "tt2"],v[1,gid || "tt3"],v[1,gid || "tt4"]) in
    let cp = select4(wla.p,wlb.p,
               v[1,gid || "pt1"],v[1,gid || "pt2"],v[1,gid || "pt3"],v[1,gid || "pt4"]) in
    { k = k xor ct; p = k xor cp }
  }
  
  andgate(gid, wla, wlb, wlc) 
  {
    let keys = keysgen(gid,wla.k,wlb.k) in
    sharetable(gid,"tt",andtable(keys,wlc.k,wla.p,wlb.p));
    sharetable(gid,"pt",andtable(keys,wlc.p,wla.p,wlb.p))
  }
  
  encode(gid, wla,wlb)
  {
    let wla = { k = flip[2,"fwl1"]; p = flip[2,"pwl1"] } in
    let wlb = { k = flip[2,"fwl2"]; p = flip[2,"pwl2"] } in
    { wv1 = { k = OT[s[1,"0"],wla.k,(not wla.k)]; p = OT[s[1,"0"],wla.p,(not wla.p)]}; 
      wv2 = { k = select[s[2,"0"],wlb.k,(not wlb.k)]; p = select[s[2,"0"],wlb.p,(not wlb.p)] } }
  }
\end{verbatimtab}
}
\end{fpfig}

%  andtable(keys, bt, ap, bp)
%  {
%    let r11 = (keys.k11 xor bt) in 
%    let r10 = (keys.k10 xor (not bt)) in
%    let r01 = (keys.k01 xor (not bt)) in
%    let r00 = (keys.k00 xor (not bt)) in
%    permute4(ap,bp,r11,r10,r01,r00)
%  }


In Figures \ref{fig-ygc-aux} and \ref{fig-ygc-gates} we define a
codebase for Yao's garbled circuits (YGC). This definition follows the
\emph{point-and-permute} method described in \cite{evans2018pragmatic}
and elsewhere, to which the reader is referred for more in-depth discussion.
In this implementation client 2 is the \emph{garbler} and
client 1 is the \emph{evaluator}. The garbler builds the garbled
tables and shares them with the evaluator, who then evaluates
the gate in an oblivious fashion until the final public output is
generated through decryption. This definition is well-typed,
with input type annotations for top-level functions listed in
Figure \ref{fig-ygc-types}. Well-typed programs using these
libraries are therefore guaranteed to yield safe $\minifed$
programs. 

\emph{Wire labels} are fundamental to YGC, and essentially represent
gate output values in an encrypted form. In our definition, wire
labels are represented by records $\ttt{\{ k = }\beta_1\ttt{; p =
}\beta_2\ttt{ \}}$, where $\ttt{k}$ is the \emph{key bit} and
$\ttt{p}$ is the \emph{pointer bit}, and $\beta_1$ and $\beta_2$ are
flips. Flips in each output wire label are owned by the garbler and
are unique per gate by definition of their identifying string, and the
representation of $0$ is the negation of $1$. For example, here is the
representation of 1 and 0 respectively in the output wire label for a
hypothetical gate 6:
\begin{mathpar}
  \ttt{\{ k = flip[2,gate:6.k]; p =  flip[2,gate:6.p]] \}}
    
  \ttt{\{ k = not flip[2,gate:6.k]; p =  not flip[2,gate:6.p]] \}}
\end{mathpar}
The pointer bits in wire labels are used to select permuted rows in
table garblings. The key bits are used to identify a unique key for
table row in each garbled gate. Intuitively, if $\beta_1$ and
$\beta_2$ are either key or pointer bits encoding 1 on two input wire
labels to a binary gate, rows and keys in the gate are enumerated in
the order:
$$
\neg\beta_1\neg\beta_2,\ \neg\beta_1\beta_2,\ \beta_1\neg\beta_2,\ \beta_1\beta_2
$$

In our implementation, gates are wired together using gate
identifiers, which are strings $w$. Top-level functionality in Figures
\ref{fig-ygc-aux} and \ref{fig-ygc-gates} includes the following:
\begin{itemize}
\item \ttt{andgate}: This defines a subprotocol for the garbler
  to define a garbled gate $\ttt{gid}$ with input wires from gates
  $\ttt{ga}$ and $\ttt{gb}$. The garbler generates keys and garbles
  the rows in YGC fashion, them with client $1$ in
  views in a standard form. For example, the view for
  a hypothetical gate 6, row 2 garbled truth table is $\ttt{v[1,gate:6tt2]}$.
  We note that garbled gates of other binary operators can be obtained with
  replacement of $\ttt{andtable}$ with appropriate garbled table definitions. 
\item \ttt{evalgate}: This defines a subprotocol for the evaluator to
  evaluate gate $\ttt{gid}$ given input wire values $\ttt{wva}$ and
  $\ttt{wvb}$.
\item \ttt{garbledecode} and \ttt{evaldecode}: The former function
  defines the garbler's protocol for encrypting the circuit
  output from final gate $\ttt{gid}$, and the latter defines
  the evaluator's output decryption protocol.
\item \ttt{encode}: This defines the initial phase of the protocol,
  where the evaluator receives the wire value from their own
  secret $\sx{1}{sa}$ via $\ttt{OT}$, and the garbler communicates
  the wire value for their own secret $\sx{2}{sb}$ directly.
\end{itemize}
\begin{example}
  \label{example-ygc-andcircuit}
The following program uses our YGC library to define
a circuit with a single and gate and input secrets $\ttt{s1}$ and
$\ttt{s2}$ from client's 1 and 2 respectively. 
\begin{verbatimtab}
  andgate(0,s1,s2);
  garbledecode(0);
  let secrets = encode(s1,s2) in
  v[0,output] := decode(evalgate(0, secrets.wv1, secrets.wv2))
\end{verbatimtab}
\end{example}


\section{2-Party GMW and Passive Security Proof Tactics}
\label{section-metalang-gmw}
\label{section-example-gmw}

\begin{fpfig}[t]{2-Party GMW circuit library with And gate (T), and type annotations (B).}{fig-gmw}
{\footnotesize
  \begin{verbatimtab}
    encodegmw(in1,in2) {
      v[1, in1++"out"] := s[1,in1] xor flip[1,in1];
      v[2, in1++"out"] := flip[1,in1];
      v[1, in2++"out"] := s[2,in2] xor flip[2,in2];
      v[2, in2++"out"] := flip[2,in2]
      { shares1 = { c1 = v[1, in1++"out"]; c2 = v[2, in1++"out"] };
        shares2 = { c1 = v[1, in2++"out"]; c2 = v[2, in2++"out"]} } 
    }
    
    andtablegmw(b1, b2) {
      let r11 = (b1 xor true) and (b2 xor true) in
      let r10 = (b1 xor true) and (b2 xor false) in
      let r01 = (b1 xor false) and (b2 xor true) in
      let r00 = (bl xor false) and (b2 xor false) in
      { v1 = r11; v2 = r10; v3 = r01; v4 = r00 }
    }
    
    andgmw(g, shares1, shares2) {
      let r = flip[1,g++".r"] in
      let table = andtablegmw(shares1.c1, shares2.c1) in
      let r11 =  r xor table.v1 in
      let r10 =  r xor table.v2 in
      let r01 =  r xor table.v3 in
      let r00 =  r xor table.v4 in
        v[2,g++"out"] := OT4(shares1.c2, shares2.c2, r11, r10, r01, r00);
      v[1,g++"out"] := r;
      { c1 = v[1,g++"out"]; c2 = v[2,g++"out"]}
    }
    
    decodegmw(shares) { v[0,"output"] := shares.c1 xor shares.c2 }   \end{verbatimtab}
}
\end{fpfig}


As an extended example of our language and security model, and how the
automated techniques in Section \ref{section-bruteforce} can serve
as tactics integrated with PSL/Lilac-style proofs, we consider GMW
circuits.  The GMW protocol is a garbled binary circuit protocol.  We
will assume the 2-party version, though it generalizes to $n$
parties \cite{goldreich2019play}. GMW uses a common technique in MPC, which is to
represent values $v$ as distributed shares $v_1$ and $v_2$ with $v =
v_1 \fplus v_2$. This trick maintains secrecy of $v$ from both
parties, and in GMW it is used to maintain the intermediate values of
internal gate outputs in circuits. In related literature the notation
$\macgv{x}$ is used to represent the ``true'' value of $x$ and $[x]$
is often used to represent the share of given party.

To capture this convention, which is used in many other protocols, we
introduce a new naming convention for ``global view'' elements
$\macgv{\mesg{w}}$, which denote the summed value of
$\elab{\mesg{w}}{1}$ and $\elab{\mesg{w}}{2}$ in a protocol
run. This concept integrates program distributions in the
usual manner, as the probability of the outcome of summation
of two variables in the distribution.
\begin{definition}
  For all $\mesg{w}$ define:
  $$\pmf(\macgv{\mesg{w}} = v) \defeq \sum_{\sigma \in A} \pmf(\sigma)$$
  and also for all $\store' \in \mems(X)$ define:
  $$\condd{\pmf}{X}{\macgv{\mesg{w}} = v}(\sigma) \defeq  \sum_{\sigma \in A} \condd{\pmf}{X}{\sigma}(\sigma')$$
  where $A$ is:
  $$\{ \store \in \mems(\{ \elab{\mesg{w}}{1},\elab{\mesg{w}}{2} \} ) \mid
      \fcod{\store(\elab{\mesg{w}}{1}) + \store(\elab{\mesg{w}}{2})} = v \}$$
\end{definition}

For full details of the GMW protocol the reader is referred to
\cite{evans2018pragmatic}. Our implementation library is shown in
Figure \ref{fig-gmw}, and includes encoding functions, where
input secrets are split into shares, $\eand$ and $\exor$ gate
functions, and a decoding function. Note that $\exor$ requires
no interaction between parties, while conjunction necessitates
1-of-4 oblivious transfer. The gate computation is
done entirely in secret, and the decoding function
is where the declassification occurs-- both parties reveal
their shares of the final gate output $\macgv{z}$.

For example, the following program uses our GMW library to define
a circuit with a single \eand gate and input secrets $\ttt{s1}$ and
$\ttt{s2}$ from client's 1 and 2 respectively:
\begin{verbatimtab}
         let s1 = encodegmw("s1",1,2) in
         let s2 = encodegmw("s2",2,1) in
         decodegmw(andgmw("z",s1,s2))
\end{verbatimtab}
By convention we will assume that all gates are assigned unique output
identifiers $\ttt{"z"}$, and that all programs are in the form
of a sequence of let-bindings followed by a call to $\decodegmw$
wrapping a circuit.

\paragraph{Oblivious Transfer} A passive secure oblivious transfer (OT) protocol
based on previous work \cite{barthe2019probabilistic} can be defined in $\metaprot$,
however this protocol assumes some shared randomness. Alternatively,
a simple passive secure OT can be defined with the addition of
public key cryptography as a primitive. But given the diversity
of approaches to OT, we instead assume that OT is abstract with
respect to its implementation, where calls to OT in $\mathbb{F}_2$
are of the following form-- given a \emph{choice bit}
$\be_1$ provided by a receiver $\cid$, the sender
sends either $\be_2$ or $\be_3$.
$$
\OT{\elab{\be_1}{\cid}}{\be_2}{\be_3}
$$
Critically, the sender learns nothing about $\be_1$ and the
receiver learns nothing about the unselected value, so we interpret
these calls in our implementation in $\mathbb{F}_2$ as follows.
$$
\begin{array}{l}
\solve{\stores}{\OT{\elab{\be_1}{\cid_1}}{\be_2}{\be_3}}{\cid_2} = \\
\qquad ((\solve{\stores}{\be_1}{\cid_1}) \cap 
(\solve{\stores}{\be_2}{\cid_2})) \cup \\
\qquad ((\stores - (\solve{\stores}{\be_1}{\cid_1})) \cap
(\solve{\stores}{\be_3}{\cid_2})
\end{array}
$$

\subsection{Correctness Proof with Verification Tactics}

As discussed above and in related work \cite{8429300}, probabilistic
separation conditional on certain variables-- e.g., secret inputs or
public outputs-- is a key mechanism for reasoning about MPC protocol
security. Following \cite{barthe2019probabilistic}, we define a
conditional separation relation $\condsep{\pmf}{X_1}{X_2}{X_3}$ to
mean that $X_2$ and $X_3$ are independent in $\pmf$ conditionally on
any assignment of values to $X_1$-- i.e., conditionally on any $\store
\in \mems(X_1)$. Another key concept needed especially for reasoning
about circuits is conditional determinism. For example, if $\macgv{z}$
is an output of an internal gate, it will definitely be computed using
random variables, however, it \emph{should} be deterministic under any
set of input secrets $S$, since we assume that $\idealf$ is
deterministic. Conditional uniformity is also important, since the
gradual release property of many protocols means that messages appear
in a uniform distribution to the adversary.
\begin{definition}[Conditioning Properties] 
  Given $\pmf$ and $X_1,X_2,X_3 \subseteq \dom(\pmf)$, 
  we write:
  \begin{itemize}
  \item $\condsep{\pmf}{X_1}{X_2}{X_3}$ iff for all
    $\store' \in \mems(X_1)$ and $\store \in \mems(X_2 \cup X_3)$ we have
    $\pmf(\store|\store') = \pmf(\store_{X_2}|\store') *  \pmf(\store_{X_3}|\store')$.
  \item $\conddetx{\pmf}{X_1}{X_2}$ iff for all
    $\store' \in \mems(X_1)$ there exists 
    $\store \in \mems(X_2)$ such that $\pmf(\store|\store') = 1$.
  \item $\condunix{\pmf}{X_1}{X_2}$ iff for all
    $\store' \in \mems(X_1)$ and
    $\store \in \mems(X_2)$ we have
    $\pmf(\store|\store') = 1/p^{|X_2|}$.
  \end{itemize}
\end{definition}

Given these definitions, we can formulate an invariant for circuit
computation with respect to internal gates as follows. It says that
the output of any gate is deterministic given inputs $S$, and
conditionally on $S$ corrupt views are always in a uniform random
distribution (pure noise), while output $\macgv{\mesg{z}}$ remains
separable from corrupt views and both shares of
$\macgv{\mesg{z}}$. This last condition (incidentally missing from PSL
due to the reliance on more recent innovations in conditioning logic
\cite{li2023lilac}) is critical since those shares will in fact be
revealed if $\macgv{\mesg{z}}$ is decoded as the circuit output.
\begin{lemma}[GMW Invariant]
  \label{lemma-gmwinvariant}
  Given:
  $$ (\varnothing,e) \redxs (\prog,\decodegmw(E[\mesg{z}])) $$
  Then all of the following conditions hold for all $H$ and $C$ where $\iov(\prog) = S \cup M$:
  \begin{enumerate}
  \item $\conddetx{\progtt(\prog)}{S}{\{\macgv{\mesg{z}}\}}$
  \item $\condunix{\progtt(\prog)}{S}{M_C}$
  \item $\condsep{\progtt(\prog)}{S}{\{\macgv{\mesg{z}}\}}{\{ \elab{\mesg{z}}{1}, \elab{\mesg{z}}{2} \})}$
  \end{enumerate}
\end{lemma}
To prove this, we can formulate and automatically prove local,
gate-level versions of the invariant. This serves as a proof tactic
that simplifies the proof of the GMW invariant. 
\begin{lemma}[And Gate Tactic]
  \label{lemma-gmwtactic}
  %Define:
  %$$
  %\begin{array}{c}
  %  \prog_{i} \defeq \\
  %  \eassign{\mesg{x}}{1}{\flip{x}}{1}; \eassign{\mesg{x}}{2}{\flip{x}}{2}; \\
  %  \eassign{\mesg{y}}{1}{\flip{y}}{1}; \eassign{\mesg{y}}{2}{\flip{y}}{2} 
  %\end{array}
  %$$
  Given:
  $$
  \begin{array}{c}
  (\varnothing,\andgmw(z,\mesg{x},\mesg{y}) \redxs %\\
  (\prog,\mesg{z})
  \end{array}
  $$
  Then all of the following conditions hold for both $\cid \in \{ 1,2 \}$ where $\iov(\prog) = M$:
  \begin{enumerate}
  \item
    $\conddetx{\progtt(\prog)}{\{ \macgv{\mesg{x}},\macgv{\mesg{y}} \}}{\{ \macgv{\mesg{z}} \}}$
  \item $\condunix{\progtt(\prog)}{\{ \macgv{\mesg{x}},\macgv{\mesg{y}} \}}{\{ \elab{\mesg{z}}{\cid} \}}$
  \item $\condsep
    {\progtt(\prog)}
    {\{ \macgv{\mesg{x}},\macgv{\mesg{y}} \}}
    {\{ \macgv{\mesg{z}} \}}
    {\{ \elab{\mesg{z}}{1},\elab{\mesg{z}}{2} \}}$
  \end{enumerate}
\end{lemma}
\begin{proof}
Verified automatically using techniques described in Section \ref{section-bruteforce}.  
\end{proof}

To properly integrate the local reasoning of Lemma \ref{lemma-gmwtactic} with
the global reasoning of Lemma \ref{lemma-gmwinvariant}, we can demonstrate
the following properties of conditioning-based reasoning. While a frame rule
was developed in \cite{li2023lilac}, these properties are distinct and
useful for reasoning about MPC. We omit proofs for brevity. 
\begin{lemma}
  \label{lemma-conditioning}
  Given $\pmf$, each of the following hold for $S,V_1,V_2,V_3 \in \dom(\pmf)$:
  \begin{enumerate}
    \item Given $\condp{\pmf}{S}{\detx{V_1}}$ and
      $\condp{\pmf}{V_1}{\detx{V_2}}$, then $\condp{\pmf}{S}{\detx{V_2}}$.
    \item Given $\condp{\pmf}{S}{\detx{V_1}}$ and
      $\condp{\pmf}{V_1}{\unix{V_2}}$, then $\condp{\pmf}{S}{\unix{V_2}}$.
    \item Given $\condp{\pmf}{S}{\detx{V_1}}$ and
      $\condsep{\pmf}{V_1}{V_2}{V_3}$, then $\condsep{\pmf}{S}{V_2}{V_3}$.
    %\item Given $\condsep{\pmf}{X_1}{X_2}{X_3}$ and $\condp{\pmf}{X_1}{\detx{X_2}}$
    %  and $\condp{\pmf}{X_2}{\detx{X_4}}$, then $\condsep{\pmf}{X_1}{X_4}{X_3}$.
    %\item Given $\condsep{\pmf}{S_1 \cup S_2}{V_3}{X_4}$ and $\condp{\pmf}{X_1 \cup X_2}{\detx{X_3}}$,
    %  then $\condsep{\pmf}{X_1}{X_2 \cup X_3}{X_4}$.
  \end{enumerate}
\end{lemma}

Then we can put the pieces together to prove the invariant, using automated tactics
for gate-level reasoning. We sketch some elements of this proof to focus on
novel aspects of our technique. 
\begin{proof}[Proof of Lemma \ref{lemma-gmwinvariant}]
  By induction on the length of $(\varnothing,e) \redxs (\prog,\decodegmw(E[\mesg{z}]))$.
  Encoding establishes the invariant in the basis. The most interesting inductive
  case is the $\andgmw$ gate. 
  \paragraph{Case $\andgmw$.} In this case we have:
  $$
  \begin{array}{c}
  (\varnothing,e) \redxs (\prog',\decodegmw(E[\andgmw(z,\mesg{x},\mesg{y})])) \redxs \\
    (\prog,\decodegmw(E[\mesg{z}]))
  \end{array}
  $$
  Letting $\iov(\prog) = S \cup M$, by definition and the induction hypothesis there
  exist $S^1,S^2 \subseteq S$ and $M^1,M^2 \subseteq M$
  such that:
  \begin{enumerate}[\hspace{5mm}(x.1)]
  \item $\conddetx{\progtt(\prog)}{S^1}{\{\macgv{\mesg{x}}\}}$
  \item $\condunix{\progtt(\prog)}{S^1}{M^1_C}$
  %\item $\condsep{\progtt(\prog)}{S^1}{\{\macgv{\mesg{x}}\}}{\{ \elab{\mesg{x}}{1}, \elab{\mesg{x}}{2} \})}$
  \end{enumerate}
  and also:
  \begin{enumerate}[\hspace{5mm}(y.1)]
  \item $\conddetx{\progtt(\prog)}{S^2}{\{\macgv{\mesg{y}}\}}$
  \item $\condunix{\progtt(\prog)}{S^2}{M^2_C}$
  %\item $\condsep{\progtt(\prog)}{S^2}{\{\macgv{\mesg{y}}\}}{\{ \elab{\mesg{y}}{1}, \elab{\mesg{y}}{2} \})}$
  \end{enumerate}
  and by Lemma \ref{lemma-gmwtactic} we have:
  \begin{enumerate}[\hspace{5mm}(z.1)]
  \item
    $\conddetx{\progtt(\prog)}{\{ \macgv{\mesg{x}},\macgv{\mesg{y}} \}}{\{ \macgv{\mesg{z}} \}}$
  \item $\condunix{\progtt(\prog)}{\{ \macgv{\mesg{x}},\macgv{\mesg{y}} \}}{\{ \elab{\mesg{z}}{\cid} \}}$
  \item $\condsep
    {\progtt(\prog)}
    {\{ \macgv{\mesg{x}},\macgv{\mesg{y}} \}}
    {\{ \macgv{\mesg{z}} \}}
    {\{ \elab{\mesg{z}}{1},\elab{\mesg{z}}{2} \}}$
  \end{enumerate}
  Given $x.1$ and $y.1$ and assumed uniqueness of random variables
  used in each gate we have $\conddetx{\progtt(\prog)}{S}{\{
    \macgv{\mesg{x}},\macgv{\mesg{y}}\}}$, so it follows from $z.1$
  and Lemma \ref{lemma-conditioning} (1) that
  $\conddetx{\progtt(\prog)}{S}{\{\macgv{\mesg{z}}\}}$, and also by
  $z.3$ and Lemma \ref{lemma-conditioning} (3):
  $$\condsep
  {\progtt(\prog)}
  {S}
  {\{ \macgv{\mesg{z}} \}}
  {\{ \elab{\mesg{z}}{1},\elab{\mesg{z}}{2} \}}$$
  Additionally Lemma \ref{lemma-conditioning} (2) gives $\condunix{\progtt(\prog)}{S}{\{ \elab{\mesg{z}}{\cid} \}}$,
  and $x.2$ and $y.2$ give $\condunix{\progtt(\prog)}{S}{(M^1 \cup M^2)_C}$, $z.3$ together with
  uniqueness of gate identifiers implies:
  $$
  \condunix{\progtt(\prog)}{S}{(M^1 \cup M^2)_C \cup \{  \elab{\mesg{z}}{\cid} \}}
  $$
  This implies the result.
\end{proof}
The preceding Lemma, together with some additional observations about decoding, establish correctness
of arbitrary circuits. 
\begin{theorem}
  \label{theorem-gmw}
  If $e$ is a GMW circuit protocol in $\metaprot$ with $(\varnothing,e) \redxs (\prog,())$
  then $\prog$ satisfies noninterference modulo output. 
\end{theorem}

\begin{proof}
  Given that $e$ is a GMW circuit protocol, then by definition we have:
  $$
  (\varnothing,e) \redxs (\prog',\decodegmw(\mesg{z})) \redxs (\prog,())
  $$
  where by Lemma \ref{lemma-gmwinvariant} and definition of $\decodegmw$,
  for any $H$ and $C$ letting $\iov(\prog) = S \cup M \cup P \cup O$ we
  have:
  \begin{mathpar}
    \conddetx{\progtt(\prog)}{S}{\{\macgv{\mesg{z}}\}}
    \condunix{\progtt(\prog)}{S}{M_C}
    \condsep{\progtt(\prog)}{S}{\{\macgv{\mesg{z}}\}}{\{ \elab{\mesg{z}}{1}, \elab{\mesg{z}}{2} \})}
  \end{mathpar}
  and by definition of $\decodegmw$ the value $\{\macgv{\mesg{z}}\}$ is output
  after the parties publicly reveal $\{ \elab{\mesg{z}}{1}, \elab{\mesg{z}}{2} \}$, so we have:
  \begin{mathpar} 
    \condunix{\progtt(\prog)}{S}{M_C}
    
    \condsep{\progtt(\prog)}{S}{O}{P}
  \end{mathpar}
  which implies the result.
\end{proof}


\subsection{2-Party BDOZ and Integrity Enforcement}
\label{section-example-bdoz}

\begin{fpfig}[t]{2-party BDOZ protocol library.}{fig-bdoz}
{\footnotesize{
\begin{verbatimtab}
auth(s,m,k,i) { assert(m == k + (m["delta"] * s))@i; }
  
sum_she(z,x,y,i) {
  m[z++"s"]@i := (m[x++"s"] + m[y++"s"])@i;
  m[z++"m"]@i := (m[x++"m"] + m[y++"m"])@i;
  m[z++"k"]@i := (m[x++"k"] + m[y++"k"])@i
}

open(x,i1,i2){
  m[x++"exts"]@i1 := m[x++"s"]@i2;
  m[x++"extm"]@i1 := m[x++"m"]@i2;
  auth(m[x++"exts"], m[x++"extm"], m[x++"k"], i1);
  m[x]@i1 := (m[x++"exts"] + m[x++"s"])@i1
}
\end{verbatimtab}
}}
\end{fpfig}


%%%%%%%%% OLD VERSION BELOW
\begin{comment}

\begin{fpfig}[t]{2-Party BDOZ Protocol Library.}{fig-bdoz}
{\footnotesize{
  \begin{verbatimtab}
    macsum(s1,s2)
    { { share = s1.share + s2.share; mac = s1.mac + s2.mac } }
    
    maccsum(s,c)
    { { share = s.share + c; mac = s.mac + c } }
    
    macctimes(s,c)
    { { share = s.share * c; mac = s.mac * c } }
    
    macshare(w) { {  share = m[w]; mac = m[w++"mac"] } }

    mack(w) { m[w++"k"] }
    
    auth(s,m,k,i) { assert(m = k + m["delta"] * s)@i }
    
    secopen(w1,w2,w3,i1,i2)
    {
      let locsum =  macsum(macshare(w1), macshare(w2)) in
      m[w3++"s"]@i1 := (locsum.share)@i2;
      m[w3++"smac"]@i1 := (locsum.mac)@i2;
      auth(m[w3++"s"],m[w3++"smac"],mack(w1) + mack(w2),i1);
      m[w3]@i1 := (m[w3++"s"] + (locsum.share))@i1
    }

    secreveal(s,k,w,i1,i2)
    {
      p[w] = s.share@i2;
      p[w++"mac"] = s.mac@i2;
      auth(p[w],p[w++"mac"],k,i1)    
    } \end{verbatimtab}
}}
\end{fpfig}

\end{comment}
    


In a malicious setting, ``detecting cheating'' by adding
information-theoretic secure MAC codes to shares is a fundamental
approach realized by protocols such as BDOZ and SPDZ
\cite{SPDZ1,SPDZ2,BDOZ,10.1007/978-3-030-68869-1_3}.  These protocols
assume a pre-processing phase that distributes shares with MAC codes
to clients.  This integrates well with pre-processed Beaver Triples to
implement malicious secure, and relatively efficient, multiplication
\cite{evans2018pragmatic}. Recall that Beaver triples are values $a,b,c$ with
$\fcod{a * b} = c$, unique per multiplication gate, that are secret
shared with clients during pre-processing. Here we consider the
2-party version.

A field value $v$ is secret shared among 2 clients in BDOZ in the same
manner as in GMW.  Each client $\cid$ gets a pair of the form
$(v_\cid,m_\cid)$, where $v_\cid$ is a share of $v$ reconstructed by
addition, i.e., $v = \fcod{v_1 \fplus v_2}$, and $m_\cid$ is a MAC of
$v_\cid$.  More precisely, $m_\cid = k + (k_\Delta * v_\cid)$ where
\emph{keys} $k$ and $k_\Delta$ are known only to $\cid' \ne \cid$ and
$k_\Delta$. The \emph{local key} $k$ is unique per MAC, while the
\emph{global key} $k_\Delta$ is common to all MACs authenticated by
$\cid'$. This is a semi-homomorphic encryption scheme that supports
addition of shares and multiplication of shares by a constant-- for
more details the reader is referred to Orsini
\cite{10.1007/978-3-030-68869-1_3}. We note that while we restrict
values $v$, $m$, and $k$ to the same field in this presentation for
simplicity, in general $m$ and $k$ can be in extensions of
$\mathbb{Z}_p$.

We can capture both the preliminary distribution of Beaver triples and BDOZ shares
as a pre-processing predicate that establishes conditions for initial
memories (see Definition \ref{def-progtt}).  Here we assume two input
secrets $\elab{\secret{x}}{1}$ and $\elab{\secret{y}}{2}$ and a single
Beaver Triple to compute $\elab{\secret{x}}{1} \ftimes
\elab{\secret{y}}{2}$, but we can extend this for additional gates.
As for GMW, we use $\macgv{\mesg{w}}$ to refer to secret-shared values
reconstructed with addition, where by convention shares are message
values $\elab{\mesg{w}}{\cid}$ for all $\cid$.
\begin{definition}[BDOZ preprocessing]
  Define:
  \begin{mathpar}
    \mathit{shares} \defeq
    \{ \elab{\mesg{w}}{\cid}\ |\ \cid \in \{ 1, 2 \} \wedge w \in \{ a,b,c,x,y \}  \}

    \mathit{macs} \defeq  \{ \elab{\mesg{w\ttt{mac}}}{\cid}\ |\ \cid \in \{ 1, 2 \} \wedge w \in \{ a,b,c,x,y \}  \}

    \mathit{keys} \defeq  \begin{array}{l}\{ \elab{\mesg{w\ttt{k}}}{\cid}\ |\ \cid \in \{ 1, 2 \} \wedge w \in
    \{ a,b,c,x,y \}  \} \cup \\ \{ \elab{\mesg{\ttt{delta}}}{\cid}\ |\ \cid \in \{ 1, 2 \} \} \end{array}
  \end{mathpar}
  Then a memory $\store$ satisfies BDOZ preprocessing iff:
  $$\dom(\store) = \{ \elab{\secret{x}}{1}, \elab{\secret{y}}{2} \} \cup \mathit{shares}
  \cup \mathit{macs} \cup \mathit{keys}$$
  and, writing $\store(\macgv{\mesg{w}})$ to denote
  $\fcod{\store(\elab{\mesg{w}}{1}) + \store(\elab{\mesg{w}}{2})}$,
  the following conditions hold:
  \begin{mathpar}
    \store(\macgv{\mesg{x}}) = \store(\elab{\secret{x}}{1})
    
    \store(\macgv{\mesg{y}}) = \store(\elab{\secret{y}}{2})
    
    \fcod{\store(\macgv{\mesg{a}}) * \store(\macgv{\mesg{b}})} = \store(\macgv{\mesg{c}})
  \end{mathpar}
  and for all $\cid,\cid' \in \{1,2\}$ with $\cid \ne \cid'$ and $w \in \{ a,b,c,x,y\}$:
  $$\lcod{\store, \mesg{w\ttt{mac}}}{\cid} =
  \lcod{\store, \mesg{wk} + \mesg{\ttt{delta}} * \mesg{w}}{\cid'}$$
\end{definition}

With these conventions, our BDOZ library is defined in Figure \ref{fig-bdoz}.
In $\metaprot$ we represent BDOZ share, MAC pairs as records:
$$
\ttt{\{ share = }v\ttt{;  mac =}\ m \ttt{\}}
$$
and we define $\ttt{macsum}$ for addition of shares,
$\ttt{maccsum}$ for addition of a share and a constant, and
$\ttt{macctimes}$ for multiplication of a share and a constant
in the BDOZ encryption scheme \cite{10.1007/978-3-030-68869-1_3}. The $\ttt{auth}$
function implements the MAC check as an $\assert$.

We also define a function $\ttt{secopen}$ to implement ``secure
opening''.  In this standard subprotocol, the value
$\macgv{\secret{w_1}} \fplus \macgv{\mesg{w_2}}$ is reconstructed as
$\mesg{w_3}$, by each client $\cid_2$ communicating
$\lcod{\mesg{w_1} + \mesg{w_2}}{\cid_2}$ to $\cid_1$.  Assuming that
$\macgv{\mesg{w_2}}$ is in an independent uniform distribution,
this perfectly hides $\secret{w_1}$. In a mutiplication gate
either $a$ or $b$ of a Beaver triple are used in secure opening,
so, e.g., given $
(\varnothing,\ttt{secopen}(a, x, d, 2, 1) \redxs (\prog,())$
we have both:
\begin{mathpar}
  \conddetx{\progtt(\prog)}{\{\macgv{\mesg{a}},\elab{\secret{x}}{1} \}}{\{ \elab{\mesg{d}}{2} \}}
  
  \sep{\progtt(\prog)}{\{ \elab{\mesg{d}}{2} \}}{\{ \elab{\secret{x}}{1}  \}}
\end{mathpar}
Furthermore, client 2's authentication of the sum of shares with the
sum of their keys detects any attempted cheating by $1$.
Finally, the $\ttt{secreveal}$ function
is very similar to $\decodegmw$, except with the addition
of authentication of revealed shares to ensure malicious security. 

\begin{fpfig}[t]{Authenticated 2-party multiplication with trusted Client 1.}{fig-beaver}
{\footnotesize
  \begin{verbatimtab}
sum("a","x","d",1,2);
open("d",1,2);
sum("b","y","e",1,2);
open("e",1,2);

p["xys2"] := (m["bs"] * m["d"] + m["as"] * m["e"] + m["cs"])@2
p["xym2"] := (m["bm"] * m["d"] + m["am"] * m["e"] + m["cm"])@2
m["xyk"]@1 := (m["bk"] * m["d"] + m["ak"] * m["e"] + m["ck"])@1;

m["xys"]@1 := (m["bs"] * m["d"] + m["as"] * m["e"] + m["cs"] +
               m["d"] * m["e"])

auth(p["xys2"], p["xym2"], m["xyk"], 1);
out@1 := m["xys"] + p["xys2"]@1;
\end{verbatimtab}
}
\end{fpfig}

\begin{comment}

\begin{fpfig}[t]{Authenticated 2-Party Multiplication.}{fig-beaver}
{\footnotesize
  \begin{verbatimtab}
    secopen("a","x","d",1,2);
    secopen("a","x","d",2,1);
    secopen("b","y","e",1,2);
    secopen("b","y","e",2,1);
    let xys =
      macsum(macctimes(macshare("b"), m["d"]),
             macsum(macctimes(macshare("a"), m["e"]),
                    macshare("c")))
    in
    let xyk = mack("b") * m["d"] + mack("a") * m["d"] + mack("c")               
    in
    secreveal(xys,xyk,"1",1,2);
    secreveal(maccsum(xys,m["d"] * m["e"]),
              xyk - m["d"] * m["e"],
              "2",2,1);
    out@1 := (p[1] + p[2])@1;
    out@2 := (p[1] + p[2])@2;
  \end{verbatimtab}
}
\end{fpfig}

\end{comment}


The full protocol for malicious secure product of secrets $x$ (that
is, $\elab{\secret{x}}{1}$) and $y$ (that is, $\elab{\secret{y}}{2}$)
using Beaver triple $a,b,c$ is defined in Figure \ref{fig-bdoz}. Both
parties interact in secure opening of $x \fplus a$ and $y + b$,
followed by the non-interactive computation of shares of $x * y$ for
secure reveals as per standard protocol
\cite{10.1007/978-3-030-68869-1_3}. The non-interactive reconstruction
of the local authentication key for both the secure openings and the
final reveal is enabled by the semi-homorphic properties of the BDOZ
scheme.

\subsection{Cheating Detection and Integrity}

We can carry out similar proofs of passive security for the protocol in
Figure \ref{fig-beaver} as for GMW, even using automated tactics for
the protocol in $\mathbb{F}_2$. But in the case of BDOZ we are also
concerned with malicious security. To demonstrate this, we can
demonstrate that the protocol satisfies integrity in the sense of
Definition \ref{def-integrity}. To do so, we observe that it satisfies
a stronger property, that we call cheating detection. Intuitively,
integrity says that the only thing that the adversary can do in the
malicious model is to elicit the same responses from honest parties
that an honest run of the protocol would elicit. Cheating detection
says that the adversary can only execute the protocol honestly, or
or else gets caught (and abort).

Focusing in, we identify the adversarial inputs as the messages
sent from the adversary to honest parties, on which honest responses
to the adversary may depend. We want to say that these are the messages
that must be legitimate.
\begin{definition}
  Given $\prog$ with $\iov(\prog) = S \cup V \cup O$,
  let $X_H \subseteq \{ x \mid x \in (\houtputs \cup O_H) \wedge x \in \dom(\store) \}$.
  Then the \emph{adversarial inputs to $X_H$} is the least set
  $X_C \subseteq \cinputs$ such that $\progtt(\prog) \not\vdash X_C * X_H$.
\end{definition}
Now, we can characterize protocols with cheating detection as those where
adversarial inputs to honest reponses must themselves be constructed honestly. 
\begin{definition}[Cheating Detection]
  \emph{Cheating is detected} in $\prog$ with $\iov(\prog) = S \cup V \cup O$ iff
  for all  $\store \in \aruns(\prog)$,
  letting $X_H = \{ x \mid x \in (\houtputs \cup O_H) \wedge x \in \dom(\store) \}$,
  and letting $X_C$ be the adversarial inputs to $X_H$,
  there exists $\sigma'\in \runs(\prog)$
  with $\store_{X_C} = \store'_{X_C}$.  
\end{definition}

It is straightforward to demonstrate that cheating detection has integrity,
since only the ``passive'' adversary can elicit a response from honest parties. 
\begin{lemma}
  \label{lemma-cheating}
  If cheating is detected in $\prog$, then $\prog$ has integrity.
\end{lemma}

In the case of BDOZ, cheating detection is accomplished by the information-theoretic
security of the encryption scheme \cite{evans2018pragmatic}. Furthermore, the symmetry of
the protocol in Figure \ref{fig-beaver} ensures that both parties will authenticate
shares, so it is robust to corruption of either party. 


\bibliographystyle{ACM-Reference-Format}
\bibliography{logic-bibliography,secure-computation-bibliography}

\end{document}
\endinput
