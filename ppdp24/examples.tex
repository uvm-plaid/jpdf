\section{Examples and Proof Methods}
\label{section-examples}

\subsection{2-Party GMW}
\label{section-metalang-gmw}

\begin{fpfig}[t]{2-Party GMW circuit library with And gate (T), and type annotations (B).}{fig-gmw}
{\footnotesize
  \begin{verbatimtab}
    encodegmw(in1,in2) {
      v[1, in1++"out"] := s[1,in1] xor flip[1,in1];
      v[2, in1++"out"] := flip[1,in1];
      v[1, in2++"out"] := s[2,in2] xor flip[2,in2];
      v[2, in2++"out"] := flip[2,in2]
      { shares1 = { c1 = v[1, in1++"out"]; c2 = v[2, in1++"out"] };
        shares2 = { c1 = v[1, in2++"out"]; c2 = v[2, in2++"out"]} } 
    }
    
    andtablegmw(b1, b2) {
      let r11 = (b1 xor true) and (b2 xor true) in
      let r10 = (b1 xor true) and (b2 xor false) in
      let r01 = (b1 xor false) and (b2 xor true) in
      let r00 = (bl xor false) and (b2 xor false) in
      { v1 = r11; v2 = r10; v3 = r01; v4 = r00 }
    }
    
    andgmw(g, shares1, shares2) {
      let r = flip[1,g++".r"] in
      let table = andtablegmw(shares1.c1, shares2.c1) in
      let r11 =  r xor table.v1 in
      let r10 =  r xor table.v2 in
      let r01 =  r xor table.v3 in
      let r00 =  r xor table.v4 in
        v[2,g++"out"] := OT4(shares1.c2, shares2.c2, r11, r10, r01, r00);
      v[1,g++"out"] := r;
      { c1 = v[1,g++"out"]; c2 = v[2,g++"out"]}
    }
    
    decodegmw(shares) { v[0,"output"] := shares.c1 xor shares.c2 }   \end{verbatimtab}
}
\end{fpfig}


The GMW protocol is a garbled binary circuit protocol. In this section
we will assume the 2-party version, though it generalizes to $n$
parties\cite{XXX}. GMW uses a common technique in MPC, which is to
represent values $v$ as distributed shares $v_1$ and $v_2$ with
$v = v_1 \fplus v_2$. This trick maintains secrecy of $v$
from both parties, and in GMW it is used to maintain the intermediate
values of internal gate outputs in circuits. In related literature
the notation $\macgv{x}$ is used to represent the ``true''
value of $x$ and $[x]$ is often used to represent the share of
given party.

To capture this convention, which is used in many other protocols, we
introduce a new naming convention for ``global view'' elements
$\macgv{\mesg{w}}$, which denote the summed value of
$\elab{\mesg{w}}{1}$ and $\elab{\mesg{w}}{2}$ in a protocol
run. This concept integrates program distributions in the
usual manner, as the probability of the outcome of summation
of two variables in the distribution.
\begin{definition}
  For all $\mesg{w}$ define:
  $$\pmf(\macgv{\mesg{w}} = v) \defeq \sum_{\sigma \in A} \pmf(\sigma)$$
  and define:
  $$\condd{\pmf}{X}{\macgv{\mesg{w}} = v}(\sigma) \defeq  \sum_{\sigma' \in A} \condd{\pmf}{X}{\sigma'}(\sigma)$$
  where $A$ is:
  $$\{ \store \in \mems(\{ \elab{\mesg{w}}{1},\elab{\mesg{w}}{2} \} ) \mid
      \fcod{\store(\elab{\mesg{w}}{1}) + \store(\elab{\mesg{w}}{2})} = v \}$$
\end{definition}

For full details of the GMW protocol the reader is referred to
\cite{evans2018pragmatic}. Our implementation libary is shown in
Figure \ref{fig-gmw}, and includes encoding functions, where
input secrets are split into shares, $\eand$ and $\exor$ gate
functions, and a decoding function. Note that $\exor$ requires
no interaction between parties, while $\eand$ necessitates and
1-of-4 oblivious transfer. The gate computation is
done entirely in secrect, and the decoding function
is where the declassification occurs-- both parties reveal
their shares of the final gate output $\macgv{z}$.

For example, the following program uses our GMW library to define
a circuit with a single \eand gate and input secrets $\ttt{s1}$ and
$\ttt{s2}$ from client's 1 and 2 respectively:
\begin{verbatimtab}
         let s1 = encodegmw("s1") in
         let s2 = encodegmw("s2") in
         decodegmw(andgmw("z",s1,s2))
\end{verbatimtab}
By convention we will assume that all gates are assigned unique output
identifiers $\ttt{"z"}$, and that all programs are in the form
of a sequence of let-bindings followed by a call to $\decodegmw$
wrapping a circuit.

\paragraph{Oblivious Transfer} A passive secure OT protocol
based on previous work \cite{XXX} can be defined in $\metaprot$,
however this protocol assumes some shared randomness. Alternatively,
a simple passive secure OT can be defined with the addition of
public key cryptography as a primitive. But given the diversity
of approaches to OT, we instead assume that OT is abstract with
respect to its implementation, where calls to OT in $\mathbb{F}_2$
are of the following form-- given a \emph{choice bit}
$\be_1$ provided by a receiver $\cid$, the sender
sends either $\be_2$ or $\be_3$.
$$
\OT{\elab{\be_1}{\cid}}{\be_2}{\be_3}
$$
Critically, the sender learns nothing about $\be_1$ and the
receiver learns nothing about the unselected value, so we interpret
these calls in our implementation in $\mathbb{F}_2$ as follows.
$$
\begin{array}{l}
\solve{\stores}{\OT{\elab{\be_1}{\cid_1}}{\be_2}{\be_3}}{\cid_2} = \\
\qquad ((\solve{\stores}{\be_1}{\cid_1}) \cap 
(\solve{\stores}{\be_2}{\cid_2})) \cup \\
\qquad ((\stores - (\solve{\stores}{\be_1}{\cid_1})) \cap
(\solve{\stores}{\be_3}{\cid_2})
\end{array}
$$

\subsubsection{Correctness Proof with Verification Tactics}

As discussed above and in related work \cite{XXX}, probabilistic
separation conditional on certain variables-- e.g., secret inputs or
public outputs-- is a key mechanism for reasoning about MPC protocol
security. Following \cite{XXX}, we define a conditional separation
relation $\condsep{\pmf}{X_1}{X_2}{X_3}$ to mean that
under the condition of all value assignments for
$X_2$ and $X_3$ are separable under $\pmf$ on the condition
of any value assignment of $X_1$-- i.e., conditionally
on any $\store \in \mems(X_1)$. 
\begin{definition}[Conditional Separation]
  We write $\condsep{\pmf}{X_1}{X_2}{X_3}$ iff for all
  $\store' \in \mems(X_1)$ and $\store \in \mems(X_2 \cup X_3)$
  and letting $X = X_1 \cup X_2 \cup X_3$ we have:
  $$\condd{\pmf}{X}{\store'}(\store) =
  \condd{\pmf}{X}{\store'}(\store_{X_2}) *
  \condd{\pmf}{X}{\store'}(\store_{X_3})$$
\end{definition}
Another key concept needed especially for reasoning about
circuits is conditional determinism. For example, if
$\macgv{z}$ is an output of an internal gate, it will
definitely be computed using random variables, however,
it \emph{should} be determistic under any set of input
secrets $S$, since we assume that $\idealf$ is
deterministic.
\begin{definition}[Conditional Determinism]  
  We write $\conddetx{\pmf}{X_1}{X_2}$ iff for all
  $\store' \in \mems(X_1)$ there exists 
  $\store \in \mems(X_2)$ such that
  $\condd{\pmf}{X_1 \cup X_2}{\store'}(\store) = 1$.
\end{definition}

Given these definitions, we can formulate an invariant
for circuit computation with respect to internal gates
as follows. It says that the output of any gate is
deterministic wrt inputs $S$, and conditionally
on $S$ the output $\macgv{\mesg{z}}$ remains
separable from corrupt views and both shares of
$\macgv{\mesg{z}}$. This last nuance is critical,
since those shares will in fact be revealed if
$\macgv{\mesg{z}}$ is decoded as the circuit output. 
\begin{lemma}[GMW Invariant]
  \label{lemma-gmwinvariant}
  Given:
  $$ (\varnothing,e) \redxs (\prog,\decodegmw(E[\mesg{z}])) $$
  Then both of the following conditions hold for all $H$ and $C$ where $\iov(\prog) = S \cup M$:
  \begin{enumerate}
    \item $\conddetx{\progtt(\prog)}{S}{\{\macgv{\mesg{z}}\}}$
    \item $\condsep{\progtt(\prog)}{S}{\{\macgv{\mesg{z}}\}}{(M_C \cup \{ \elab{\mesg{z}}{1}, \elab{\mesg{z}}{2} \})}$
  \end{enumerate}
\end{lemma}
To prove this, we can formulate and automatically prove gate-level
versions of the invariant. This serves as a proof tactic
that simplifies the proof of the GMW invariant. It establishes
that the $\eand$ gate output is deterministic conditional on
the inputs, and is separable from the output shares and
either parties' received messages. 
\begin{lemma}[And Gate Tactic]
  \label{lemma-gmwtactic}
  %Define:
  %$$
  %\begin{array}{c}
  %  \prog_{i} \defeq \\
  %  \eassign{\mesg{x}}{1}{\flip{x}}{1}; \eassign{\mesg{x}}{2}{\flip{x}}{2}; \\
  %  \eassign{\mesg{y}}{1}{\flip{y}}{1}; \eassign{\mesg{y}}{2}{\flip{y}}{2} 
  %\end{array}
  %$$
  Given:
  $$
  \begin{array}{c}
  (\varnothing,\andgmw(z,\mesg{x},\mesg{y}) \redxs %\\
  (\prog,\mesg{z})
  \end{array}
  $$
  Then both of the following conditions hold for both $\cid \in \{ 1,2 \}$ where $\iov(\prog) = M$:
  \begin{enumerate}
  \item
    $\conddetx{\progtt(\prog)}{\{ \macgv{\mesg{x}},\macgv{\mesg{y}} \}}{\{ \macgv{\mesg{z}} \}}$
   \item $\condsep
      {\progtt(\prog)}
      {\{ \macgv{\mesg{x}},\macgv{\mesg{y}} \}}
      {\{ \macgv{\mesg{z}} \}}
      {M_{\{\cid\}} \cup \{ \elab{\mesg{z}}{1},\elab{\mesg{z}}{2} \}}$
  \end{enumerate}
\end{lemma}
\begin{proof}
Verified automatically using techniques described in Section \ref{section-bruteforce}.  
\end{proof}

To properly integrate the local reasoning of Lemma \ref{lemma-gmwtactic} with
the global reasoning of Lemma \ref{lemma-gmwinvariant}, we can demonstrate
the following.
\begin{lemma}
  \label{lemma-conditioning}
  Each of the following hold:
  \begin{enumerate}
    \item Given $\condp{\pmf}{X_1}{\detx{X_2}}$ and
      $\condp{\pmf}{X_2}{\detx{X_3}}$, then $\condp{\pmf}{X_1}{\detx{X_3}}$.
    \item Given $\condp{\pmf}{X_1}{\detx{X_2}}$ and
      $\condsep{\pmf}{X_2}{X_3}{X_4}$, then $\condsep{\pmf}{X_1}{X_3}{X_4}$.
    \item Given $\condsep{\pmf}{X_1}{X_2}{X_3}$ and $\condp{\pmf}{X_1}{\detx{X_2}}$
      and $\condp{\pmf}{X_2}{\detx{X_4}}$, then $\condsep{\pmf}{X_1}{X_4}{X_3}$.
  \end{enumerate}
\end{lemma}

Then we can put the pieces together to prove the invariant, using automated tactics
for gate-level reasoning.  
\begin{proof}[Proof of Lemma \ref{lemma-gmwinvariant}]
  By induction on the length of $(\varnothing,e) \redxs (\prog,\decodegmw(E[\mesg{z}]))$.
  Encoding establishes the invarian in the basis. The most interesting inductive
  case is the $\andgmw$ gate. 
  \paragraph{Case $\andgmw$.} In this case we have:
  $$
  \begin{array}{c}
  (\varnothing,e) \redxs (\prog',\decodegmw(E[\andgmw(z,\mesg{x},\mesg{y})])) \redxs \\
    (\prog,\decodegmw(E[\mesg{z}]))
  \end{array}
  $$
  The induction hypothesis, together with the assumed uniquenenss of $z$, gives:
  \begin{mathpar}
  \condsep{\progtt(\prog)}{S^1}{\{ \macgv{\mesg{x}} \}}{(M^1_C \cup \{ \elab{\mesg{x}}{1}, \elab{\mesg{x}}{2} \})}
  
  \condsep{\progtt(\prog)}{S^2}{\{ \macgv{\mesg{y}} \}}{(M^2_C \cup \{ \elab{\mesg{y}}{1}, \elab{\mesg{y}}{2} \})}

  \conddetx{\progtt(\prog)}{S^1}{\{ \macgv{\mesg{x}} \}}
  
  \conddetx{\progtt(\prog)}{S^1}{\{ \macgv{\mesg{x}} \}}    
  \end{mathpar}
  This, together with Lemma \ref{lemma-gmwtactic} (1) and Lemma \ref{lemma-conditioning} (1)
  give:
  $$
  \conddetx{\progtt(\prog)}{S^1 \cup S^2}{\{ \macgv{\mesg{z}} \}}
  $$
  and Lemma  \ref{lemma-gmwtactic} (2) and Lemma \ref{lemma-conditioning} (2-3) gives:
  $$
  \begin{array}{c}
    \progtt(\prog)|S^1 \cup S^2) \vdash \\
    {\{ \macgv{\mesg{z}} \}}{(M^1_C \cup M^2_C \cup \{ \elab{\mesg{x}}{2}, \elab{\mesg{y}}{2}, \elab{\mesg{z}}{1}, \elab{\mesg{z}}{2} \})}
  \end{array}
  $$
\end{proof}

\subsection{2-Party BDOZ}

\begin{fpfig}[t]{2-party BDOZ protocol library.}{fig-bdoz}
{\footnotesize{
\begin{verbatimtab}
auth(s,m,k,i) { assert(m == k + (m["delta"] * s))@i; }
  
sum_she(z,x,y,i) {
  m[z++"s"]@i := (m[x++"s"] + m[y++"s"])@i;
  m[z++"m"]@i := (m[x++"m"] + m[y++"m"])@i;
  m[z++"k"]@i := (m[x++"k"] + m[y++"k"])@i
}

open(x,i1,i2){
  m[x++"exts"]@i1 := m[x++"s"]@i2;
  m[x++"extm"]@i1 := m[x++"m"]@i2;
  auth(m[x++"exts"], m[x++"extm"], m[x++"k"], i1);
  m[x]@i1 := (m[x++"exts"] + m[x++"s"])@i1
}
\end{verbatimtab}
}}
\end{fpfig}


%%%%%%%%% OLD VERSION BELOW
\begin{comment}

\begin{fpfig}[t]{2-Party BDOZ Protocol Library.}{fig-bdoz}
{\footnotesize{
  \begin{verbatimtab}
    macsum(s1,s2)
    { { share = s1.share + s2.share; mac = s1.mac + s2.mac } }
    
    maccsum(s,c)
    { { share = s.share + c; mac = s.mac + c } }
    
    macctimes(s,c)
    { { share = s.share * c; mac = s.mac * c } }
    
    macshare(w) { {  share = m[w]; mac = m[w++"mac"] } }

    mack(w) { m[w++"k"] }
    
    auth(s,m,k,i) { assert(m = k + m["delta"] * s)@i }
    
    secopen(w1,w2,w3,i1,i2)
    {
      let locsum =  macsum(macshare(w1), macshare(w2)) in
      m[w3++"s"]@i1 := (locsum.share)@i2;
      m[w3++"smac"]@i1 := (locsum.mac)@i2;
      auth(m[w3++"s"],m[w3++"smac"],mack(w1) + mack(w2),i1);
      m[w3]@i1 := (m[w3++"s"] + (locsum.share))@i1
    }

    secreveal(s,k,w,i1,i2)
    {
      p[w] = s.share@i2;
      p[w++"mac"] = s.mac@i2;
      auth(p[w],p[w++"mac"],k,i1)    
    } \end{verbatimtab}
}}
\end{fpfig}

\end{comment}
    


In a malicious setting, detecting cheating by adding
information-theoretic secure MAC codes to shares is a fundamental approach
realized by protocols such as BDOZ \cite{XXX} and SPDZ \cite{XXX}.
These protocols assume a pre-processing phase that distributes shares
with MAC codes to clients.  This integrates well with pre-processed
Beaver Triples to implement malicious secure, and relatively
efficient, multiplication \cite{XXX}.  Here we show how to implement
malicious secure two party multiplication in the BDOZ style in
$\metaprot$. 

A field value $v$ is secret shared among 2 client in BDOZ in
the same manner as in GMW.
Each client $\cid$ gets a pair of the form $(v_\cid,m_\cid)$, where
$v_\cid$ is a secret share of $v$, i.e., $v = \fcod{v_1 \fplus v_2}$.
Further, field values $m_\cid$ are MACs of $v_\cid$ that are authenticated
by the other client's key. In particular, $m_\cid = k + (k_\Delta *
v_\cid)$ where \emph{keys} $k$ and $k_\Delta$ are known only to $\cid'
\ne \cid$ and $k_\Delta$. Further, there is a unique \emph{local key}
$k$ for each shared value, while the \emph{global key} $k_\Delta$ is
common to all MACs authenticated by $\cid'$. By extending certain
field operations to these MACed values, a semi-homorphic encryption
scheme is obtained that is adequate to compute Beaver Triples. For
example, if local keys $k_1$ and $k_2$ authenticate $\macv_1$ and
$\macv_2$, then $k_1 \fplus k_2$ authenticates $\macv_1 \macplus
\macv_2$, where $\macplus$ is addition extended to MACed
values. Semantics for summation of MACed shares, and multipication of
shares and plain field values, are given in Figure \ref{fig-bdoz}.
For more details the reader is referred to \cite{XXX}. We note
that while we restrict values $v$, $m$, and $k$ to the same
field in this presentation for simplicity, in general $m$ and
$k$ can be in extensions of $\mathbb{Z}_p$. 

We define variable forms $\macx{\secret{w}}{\cid}$ and
$\macx{\flip{w}}{\cid}$ to range over local MACed shares of a secret
and random sample owned by client $\cid$, respectively. We let
$\mack{x}{}$ denote the local key for authenticating shares of $x$.
We also assume a global key $\flip{\ttt{delta}}$ on each client.
Pre-processing is modeled by assuming that shares are provided on
input random tapes (memories) with the required properties, as defined
on the bottom of Figure \ref{fig-bdoz}.

Similarly to field values, we can use MACed shares in
uniform independent distributions as one-time-pads for
hiding other values, using subtraction over MACed shares.
So we introduce a function $\macotp$ to do this
declaratively.
$$
\delta(\macotp,v_1,v_2) \defeq \fcod{v_1 \macminus v_2}
$$

Another key idea in BDOZ is \emph{secure opening}, where distributed
shares of a value $v$ are communicated and combined to ``open'' the
same $v$ on each client. This relies on authentication to detect
cheating in the malicious setting. To implement secure opening we
define a $\macauth$ function that authenticates shares, and a
$\macopen$ function that extracts a field value from its shares.
\begin{mathpar}
\delta(\macauth, (v, m), k) \defeq
     (v, m) \text{\ if\ } m = k \fplus (\flip{\ttt{delta}} \ftimes v)
 
\delta(\macopen, (v_1, m_1), (v_2,m_2)) \defeq
\fcod{v_1 \fplus v_2}
\end{mathpar}
This authentication eliminates cheating and guarantees protocol
integrity. As we will show in Section \ref{section-types}, this
can be reflected in a security type for $\macauth$.

Putting these pieces together, we can implement 2-party multiplication with
Beaver Triples as follows. Recall that Beaver Triples are values
$(a,b,c)$ such that $c = \fcod{b \ftimes a}$- we implement them here
as a secret $\secret{c}$ and samples $\flip{a}$ and $\flip{b}$ owned
by an \emph{oracle} $\Oracle$ that we always assume is honest but is
otherwise not involved in the protocol. We define pre-processing
properties of Beaver Triples in Figure \ref{fig-bdoz}. Only one
Beaver triple can be used per multiplication to ensure secure
openings, but our scheme can be extended with more Beaver triples
and reuse of them will be rule out by type linearity as we will
discuss in Section \ref{section-types}.

First, clients 1 and 2 compute (aka \emph{open}) $\secret{x} \fplus
\flip{a}$ and $\secret{y} \fplus \flip{b}$ where $\secret{x}$ and
$\secret{y}$ are 1 and 2's secrets respectively.  Then they both
perform a secure opening of $\secret{x} \fplus \flip{a}$ and
$\secret{y} \fplus \flip{b}$, by authenticating each others' shares,
and record them as $\mesg{d}$ and $\mesg{e}$ respectively. Note
that the secrets remain hidden since $\flip{a}$ and $\flip{b}$
are random values.

\begin{definition}[BDOZ preprocessing]
  Define:
  \begin{mathpar}
    \mathit{shares} \defeq
    \{ \elab{\mesg{w}}{\cid} | \cid \in \{ 1, 2 \} \wedge w \in \{ a,b,c,x,y \}  \}

    \mathit{macs} \defeq  \{ \elab{\mesg{w\ttt{mac}}}{\cid} | \cid \in \{ 1, 2 \} \wedge w \in \{ a,b,c,x,y \}  \}

    \mathit{keys} \defeq  \begin{array}{l}\{ \elab{\mesg{w\ttt{k}}}{\cid} | \cid \in \{ 1, 2 \} \wedge w \in
    \{ a,b,c,x,y \}  \} \cup \\ \{ \elab{\mesg{\ttt{delta}}}{\cid} | \cid \in \{ 1, 2 \} \} \end{array}
  \end{mathpar}
  Then a memory $\store$ satisfies BDOZ preprocessing iff:
  $$\dom(\store) = \mathit{shares}  \cup \mathit{macs} \cup \mathit{keys}$$
  and, writing $\store(\macgv{\mesg{w}})$ to denote
  $\fcod{\store(\elab{\mesg{w}}{1}) + \store(\elab{\mesg{w}}{2})}$,
  the following conditions hold for all $\cid \in \{1,2\}$ and $w \in \{ a,b,c,x,y\}$:
\begin{mathpar}
  \store(\macgv{\mesg{w@\cid}}) = \store(\elab{\secret{w}}{\cid})

  \fcod{\store(\macgv{\mesg{a}}) * \store(\macgv{\mesg{b}})} = \store(\macgv{\mesg{c}})

  \lcod{\store, \mesg{w\ttt{mac}}}{\cid} =
  \lcod{\store, \mesg{wk} + \mesg{\ttt{delta}} * \mesg{w}}{\cid'}  \qquad \cid \ne \cid'  
\end{mathpar}
\end{definition}


\begin{definition}
  Given $\prog$ with $\iov(\prog) = S \cup V \cup O$,
  let $X_H \subseteq \{ x \mid x \in (\houtputs \cup O_H) \wedge x \in \dom(\store) \}$.
  Then the \emph{adversarial inputs to $X_H$} is the least set
  $X_C \subseteq \cinputs$ such that $\progtt(\prog) \not\vdash X_C * X_H$.
\end{definition}
\begin{definition}[Cheating Detection]
  \emph{Cheating is detected} in $\prog$ with $\iov(\prog) = S \cup V \cup O$ iff
  for all  $\store \in \aruns(\prog)$,
  letting $X_H = \{ x \mid x \in (\houtputs \cup O_H) \wedge x \in \dom(\store) \}$,
  and letting $X_C$ be the adversarial inputs to $X_H$,
  there exists $\sigma'\in \runs(\prog)$
  with $\store_{X_C} = \store'_{X_C}$.  
\end{definition}

\begin{lemma}
  If cheating is detected in $\prog$, then $\prog$ has integrity.
\end{lemma}
