\section{Security Hyperproperties}
\label{section-hyper}

In this Section we formulate probabilistic versions of well-studied
hyperproperties of confidentiality and integrity, including
noninterference, gradual release, declassification, and robust
declassification.  We follow nomenclature developed in previous work
on characterizing declassification policies in deterministic settings
\cite{sabelfeld2009declassification}, but adapt them to our
probabilistic one.


%We demonstrate a soundness relation between noninterference and
%passive security, and between robust declassification and malicious
%security.
%Some previous work on security type enforcement
%\cite{6266151,almeida2018enforcing} and has considered similar relationships
%but mainly for aspects of passive security.

\subsection{Conditional Noninterference}

Since MPC protocols release some information about secrets through
outputs of $\idealf$, they do not enjoy strict noninterference.  As
discussed in Section \ref{section-lang}, public reveals and protocol
outputs are fundamentally forms of declassification.  But consistent
with other work \cite{8429300}, we can formulate a version of
probabilistic noninterference conditioned on output that is sound
for passive security. 
\begin{definition}[Noninterference modulo output]
  \label{definition-NIMO}
  We say that a program $\prog$ with $\iov(\prog) = S \cup V \cup O$
  satisfies \emph{noninterference modulo output}
  iff for all $H$ and $C$ and $\store_1 \in \mems(S_C \cup O)$ and $\store_2 \in \mems(\houtputs)$
  we have:
  $$
  \condd{\progtt(\prog)}{S_H}{\store_1} = \condd{\progtt(\prog)}{S_H}{\store_1 \uplus \store_2}
 $$
\end{definition}
%\begin{definition}[Noninterference modulo output]
%  \label{definition-NIMO}
%  We say that a program $\prog$ satisfies \emph{noninterference modulo output}
%  iff for all $H$ and $C$ and 
%  $\store_1,\store_2 \in \mems(S)$ we have:
%  $$
%  (\store_1 =_C \store_2 \ \wedge \ 
%  (\condd{\progtt(\prog)}{O}{\store_1} = \condd{\progtt(\prog)}{O}{\store_2}))
%  \implies 
%  (\condd{\progtt(\prog)}{\houtputs}{\store_1} = \condd{\progtt(\prog)}{\houtputs}{\store_2})
%  $$
%  where $\iov(\prog) = S \cup V \cup O$.
%\end{definition}
This conditional noninterference property implies that
corrupt views give the adversary no better chance of guessing honest
secrets than just the output and corrupt inputs do. So the simulator
can just arbitrarily pick any honest secrets that could have produced
the given outputs and run the protocol in simulation to reconstruct
real world corrupt views. This requires that the simulator can
tractably ``pre-image'' a given output of a functionality $\idealf$,
to determine the inputs that could have produced it. This equivalence
class is called a \emph{kernel} in recent work \cite{10.1145/3571740}.
\begin{definition}
  Given a functionality $\idealf$ and outputs $\store_{\mathit{out}}$, their 
  \emph{kernel}, denoted $\ik(\idealf,\store_{\mathit{out}})$ is
  $
  \{ \store\ |\ \idealf(\store) = \store_{\mathit{out}} \}
  $.
  We say that $\idealf$ is \emph{pre-imageable} iff $\ik(\idealf, \store_{\mathit{out}})$ for all
  $\store_{\mathit{out}}$ can be computed tractably.
\end{definition}
A soundness result for passive security can then be given as follows.
It is essentially the same as ``perfect passive NI security'' explored
in previous work \cite{8429300}.  
\begin{restatable}{theorem}{nimosecure}
  \label{theorem-nimo}
  Assume given pre-imageable $\idealf$ and a protocol $\prog$ that
  correctly implements $\idealf$.  If $\prog$ satisfies noninterference modulo output
  then $\prog$ is passive secure.
\end{restatable}


\subsection{Gradual Release}

Probabilistic noninterference is related to perfect secrecy and is
preserved by components of cryptographic protocols generally. It can
be expressed using probabilistic independence, aka separation,
\cite{darais2019language,barthe2019probabilistic}, and we adopt the
following notation to express independence:
\begin{definition}
%  We write $\vc{\pmf}{x}{y}$ iff $\pmf(\{ x \mapsto 0\}\ |\ \{ y \mapsto 0 \}) =
%  \pmf(\{ x \mapsto 1\}\ |\ \{ y \mapsto 1 \}) = 1$.
  We write $\sep{\pmf}{X_1}{X_2}$ iff for all
    $\store \in \mems(X_1 \cup X_2)$ we have
  $\margd{\pmf}{X_1 \cup X_2}(\store) =
  \pmf(\store_{X_1}) * \pmf(\store_{X_2})$
\end{definition}

In practice, MPC protocols typically satisfy a \emph{gradual
release} property \cite{sabelfeld2009declassification}, where messages
exchanged remain probabilistically separable from secrets, with only
declassification events (reveals and outputs) releasing information
about honest secrets.  A key difference is that while these
declassification events essentially define the policy in gradual
release, the ideal functionality sets the release policy for MPC
passive security, so its necessary to show that declassification
events respect these bounds.
\begin{definition}
  Given $H,C$, a protocol $\prog$ with $\iov(\prog) = S \cup M \cup P \cup O$
  satisfies \emph{gradual release} iff
  $\sep{\progtt(\prog)}{M_C}{S_H}$.
\end{definition}

\subsection{Integrity and Robust Declassification}

%\begin{definition}[Malicious Correctness]
%  A protocol $\prog$ with $\iov(\prog) = S \cup V \cup O$ is \emph{malicious correct} iff
%  $
%  \forall \adversary, \store \in \mems(S_H) \ .\ \support(\progtt(\prog)(O_H|\store)) \supseteq
%    \support(\progtt(\prog,\adversary)(O_H|\store))
%  $.
%\end{definition}

\emph{Integrity} is an important hyperproperty in security models that admit
malicious adversaries. Consistent with formulations in deterministic settings,
we have already defined protocol confidentiality as the preservation of low equivalence
(of secrets and views), and now we define protocol integrity as the preservation
of high equivalence (of secrets and views). Intuitively, this property says
that any adversarial strategy either ``mimics'' a passive strategy with some
choice of inputs or causes an abort.
\begin{definition}[Integrity]
  \label{def-integrity}
  We say that a protocol $\prog$ with $\iov(\prog) = S \cup V \cup O$ has
  \emph{integrity} iff for all $H$, $C$, and $\adversary$,
  if $\store \in \aruns(\prog)$ 
  then there exists $\store' \in \mems(S)$ with $\store_{S_H} = \store'_{S_H} $ and:
    $$
    \condd{\progtt(\prog,\adversary)}{X}{\store_{S_H \cup \cinputs}} =
    \condd{\progtt(\prog)}{X}{\store'}
    $$ 
  where $X \defeq (\houtputs \cup O_H) \cap \dom(\store)$. 
\end{definition}
A first important observation is that integrity preserves protocol correctness
for honest outputs, except for the possibility of abort. 
\begin{lemma}
  \label{lemma-malicious-correct}
  If a protocol $\prog$ with $\iov(\prog) = S \cup V \cup O$ is passive correct for
  $\idealf$ and
  has integrity, then for all $H$, $C$, $\adversary$, $\store_1 \in \mems(S_H)$,
  $\ox{\cid} \in (O_H)$, and $\mv \in \mathbb{F}_p$, if:
  $$
  \progtt(\prog,\adversary)(\{ \ox{\cid} \mapsto \mv \} \mid \store_1) > 0
  $$
  then exists $\store_2 \in \mems(S_C)$ such that:
  $$
  \idealf(\store_1 \uplus \store_2)(\ox{\cid}) = \mv
  $$
\end{lemma}
The following result establishes that integrity implies malicious
security for protocols that are passive secure (which also subsumes
correctness). 
\begin{theorem}
  \label{theorem-integrity}
  If a protocol is passive secure and has integrity, then it
  is malicious secure.
\end{theorem}

\begin{proof}
  Let $\prog$ be some protocol with passive security and integrity
  where $\iov(\prog) = S \cup V \cup O$, and let $\adversary$ be some
  adversary. Suppose $\store \in \aruns(\prog)$.
  As integrity requires, and as Lemma \ref{lemma-malicious-correct}
  demonstrates with respect to outputs, the most the adversary can do
  in the presence of integrity is to elicit the same responses from
  the honest parties-- via the strategy $\store_{\cinputs}$-- as
  are elicited from some passive run of the protocol using
  some $\store' \in \mems(S)$ where $\store'_H = \store_{S_H}$,
  and perhaps to force an abort after some number of message
  exchanges.

  Therefore, in simulation, $\adversary$ can provide the simulator
  with some $\store'_C$ in $\SIM_1$ which its strategy is ``impersonating'',
  allowing $\idealf(\store')$ to
  be communicated to $\adversary$ in $\SIM_2$ who can then
  decide whether or not to abort. In the case of abort, the
  subset of $\houtputs$ to be defined can be communicated to
  $\SIM_3$, along with $\idealf(\store')$, via $\Sigma$. 
  In $\SIM_3$, the simulator can then run $\prog$ in simulation
  with inputs $\store'_C$ and arbitrary $\store'' \in \mems(S_H)$
  such that $\store'_C \uplus \store'' \in \ik(\idealf,\idealf(\store'))$.
  The assumption of passive security of $\prog$ implies the result.
\end{proof}

The hyperproperty of robust declassification \cite{930133} similarly
combines a confidentiality property with integrity to establish that
malicious actors cannot declassify more information than is intended
by policy. But in this prior work, this policy is established
by the declassifications themselves, as in gradual release.
Thus, we can define a robust declassification property as follows. 
\begin{definition}[Robust Declassification]
  A protocol satisfies \emph{robust declassification} iff it has integrity and
  satisfies gradual release. 
\end{definition}
However, it is important to note that gradual release, and
hence robust declassification, are not sufficient to establish
passive or malicious simulator security, where the declassification
policy is established by the ideal functionality $\idealf$. 
\begin{theorem}
  Robust declassification does not imply malicious security, but
  passive security with robust declassification implies malicious security.
\end{theorem}

