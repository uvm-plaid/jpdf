\section{Introduction}

Secure Multi-Party Computation (MPC) protocols support data privacy in
important modern, distributed applications such as privacy-preserving
machine learning \cite{li2021privacy, knott2021crypten,
  koch2020privacy, liu2020privacy} and Zero-Knowledge proofs in
blockchains \cite{ishai2009zero, lu2019honeybadgermpc,
  gao2022symmeproof, tomaz2020preserving}. MPC methods have been
developed by the cryptography community for many years, while
receiving the attention of the programming languages community more
recently. In contrast, information flow security has received
significant attention in PL theory and practice especially since the
turn of the century \cite{1159651}, including a menagerie of variants,
enforcement mechanisms, and programming frameworks. Much of this has
been enabled by the unified metatheory of \emph{hyperproperties}
\cite{10.5555/1891823.1891830}, including noninterference, that
establishes a common conceptual framework for reasoning about and
implementing systems with information flow security.

On its surface, security for MPC protocols seems similar to
probabilistic noninterference. However, MPC security differs from
noninterference in a subtle but fundamental way, and approaches for
verifying noninterference cannot naturally extend to MPC security. Our
goal is to explore connections between the security model of MPC-- the
\emph{real/ideal} aka \emph{simulator security} model -- and
trace-based hyperproperties, and to leverage these connections to
obtain automated enforcement mechanisms for MPC protocol
development. Currently, proof methods for MPC protocol development are
well-studied \cite{Lindell2017} with some previous work leveraging
connections with hyperproperties \cite{8429300}. But these methods
require fully manual proof construction and are thus tedious and
error-prone. Therefore our exploration makes both theoretical and
practical contributions.

\subsection{The Analysis Challenge of MPC}
MPC protocols involve communication
between a group of distributed participants called a \emph{federation}
that collaboratively compute and publish the result of some known
\emph{ideal functionality} $\idealf$, while keeping each party's input
``secret'', without the use of a trusted third party. This last part
is critical---the ideal functionality is a specification of what
information the parties agree to \emph{intentionally} reveal. For
example, if we take $\idealf$ to be the majority vote
function, a protocol for computing $\idealf$ is MPC-secure if, given
any set of input votes, it correctly computes and publishes the voting
result but reveals no other information to the public or to other
participants. However, by publishing the result, some information
about individual votes may be implicitly declassified.  For example,
in the case of a majority vote in a federation of size 3, if the
motion carries and party 1 has voted no, then party 1 knows exactly
the votes of parties 2 and 3. This is an \emph{intentional} release
of information and is not considered a security failure.

Security in the MPC setting thus means that protocols cannot reveal
any secret information other than what is implicitly declassified by
the publicized output of the ideal functionality. The security model
also assumes that some subset of participants can be corrupted and
collude adversarially to possibly infer more secret information. The
accepted method of demonstrating protocol security in this setting is
to define a \emph{simulation} algorithm that runs in the ``ideal''
world which, given just the inputs of corrupted parties and the output
of the ideal functionality, is able to reconstruct information that
corrupted parties receive in their so-called \emph{views} of the
protocol running in the real world.  This implies that adversarial
views provide no information beyond what is provided by the ideal
output alone. Simulation is defined probabilistically since MPC
protocols typically rely on cryptographic and probabilistic methods.

In Section \ref{section-hyperprop-passive} we formalize real/ideal
security, and we define and discuss simple examples of MPC protocols
in Section \ref{section-minicat-examples} and more complex ones in
Section \ref{section-metalang}. But an immediate and main point in
relation to security hyperproperties is that, due to the potential for
information release in MPC, simulator security is \emph{not} a strict
probabilistic noninterference or trace obliviousness property, as we
show in Section \ref{section-hyperprop-ni}- rather, the public output
allows and sets an upper bound on declassification.

\subsection{Overview and Contributions}

\paragraph{Language design.} In Section \ref{section-minicat} we
develop a new probabilistic programming language $\minifed$ for
defining synchronous distributed protocols over the binary field. The
syntax and semantics provides a succinct account of \emph{views} and
(synchronous) messaging between protocol \emph{clients}. In Section
\ref{section-metalang} we define a new metaprogramming language
$\metaprot$ that dynamically generates $\minifed$ protocols. It
includes control and data structures and is able to express logical
protocol components, and enjoys a type safety result that guarantees
that generated protocols are semantically well-defined.

\paragraph{Hyperproperty formulation.} In Section \ref{section-pmf} we
formulate \emph{program distributions}, a formalism for expressing
dependencies between secrets and views, useful for expressing
characteristics of protocols and precisely quantifying allowable
information leakage in terms of protocol outputs. On this basis we can
formulate security hyperproperties, including \emph{noninterference
modulo output ($\NIMO$)} that implies passive simulator security. But
program distributions are flexible and can be used to certify other
important properties of protocol components, such as probabilistic
independence, correlation, and distributions under conditioning.

\paragraph{Fully and partially automated verification.} In Section
\ref{section-automation} we develop an automated technique, aka
certification, for automatically verifying security properties such as
$\NIMO$ and other supporting properties of protocol components.  While
certification has high complexity, a compositional approach allows
certification of program components in isolation.  This enables a
semi-automated method for large circuits where we (1) define
certification for components, (2) certify components, (3) prove
$\NIMO$ based on the certifiedness of components. We develop novel
methods partially based on separation logic
\cite{barthe2019probabilistic} to enable (3).  In Section
\ref{section-composition} we illustrate our method to obtain
semi-automated verification for the GMW and Yao's Garbled
Circuit protocols, where certification can be re-used to validate new
gate implementations.

\compfig

\subsection{Related Work}
\label{section-related-work}

Our main focus is on automated and semi-automated reasoning about
security properties of specific protocols, both through whole-program
analysis and compositional properties of program components that
support security.

Previous work on analysis for the SecreC language
\cite{almeida2018enforcing,10.1145/2637113.2637119} is concerned with
properties of complex MPC circuits, in particular a user-friendly
specification and automated enforcement of declassification bounds in
programs that use MPC in subprograms. This work is explicitly
reminiscent of information flow approaches such as delimited
information release \cite{10.1007/978-3-540-37621-7_9}, downgrading
policies \cite{li2005downgrading}, and relaxed noninterference
\cite{10.1145/1040305.1040319}. However, their program logic assumes
correctness of the underlying MPC protocols.  The Wys$^\star$ language
\cite{wysstar}, based on Wysteria \cite{rastogi2014wysteria}, has
similar goals and includes a trace-based semantics for reasoning about
the interactions of MPC protocols. Their compiler also guarantees that
underlying multi-threaded protocols enforce the single-threaded source
language semantics. However their language ultimately compiles to
protocols assumed to be secure.  All of these systems leverage type
systems to obtain automated security analyses.

In the realm of protocol verification, previous work has shown how to
reduce proofs of simulator security to program equivalence assertions,
which supports computer-aided verification of MPC protocols in
EasyCrypt \cite{8429300}. This work also leveraged hyperproperty
characterizations of simulator security, both passive and
active. However, their EasyCrpyt security proof constructions are
entirely manual. Probabilistic Separation Logic
\cite{barthe2019probabilistic} develops a logical framework for
reasoning about probabilistic independence in programs.  They consider
several MPC protocols and show how to systematically deduce, e.g.,
probabilistic independence of input secrets and adversarial views that
are critical security properties.  However, their methods are manual. 
Also, their focus on independence does not address the subtleties of
dependence that arise during MPC protocol execution, and the authors
do not formulate or prove simulator security properties in
\cite{barthe2019probabilistic}. The $\lambda_{\mathrm{obliv}}$
language \cite{darais2019language} is perhaps most closely related to
our work. Here, a type system is used to automatically enforce
so-called probabilistic trace obliviousness and they show how to use
their system to implement several variants of oblivious RAM. However,
obliviousness is related to noninterference and does not address
allowable information leakage as in MPC, and their use of type
linearity disallows the reuse of sampled probabilistic values which is
common in the MPC idiom, as our examples will illustrate. Similarly,
previous work on oblivious data structures \cite{10.1145/3498713},
while intended for sound integration into MPC protocols, focus on
expression and enforcement of obliviousness and do not account for
allowable information leakage.

Our work also shares many ideas with probabilistic programming
languages designed to perform (exact or approximate)
inference~\cite{holtzen2020scaling, carpenter2017stan, wood2014new,
  bingham2019pyro, albarghouthi2017fairsquare, de2007problog,
  pfeffer2009figaro, saad2021sppl}. Our setting, however, requires
verifying properties beyond inference, including conditional
independence. Recent work by Li et al.~\cite{li2023lilac} proposes a
manual approach for proving such properties, but does not provide
automation.

