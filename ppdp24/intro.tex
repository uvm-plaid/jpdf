\section{Introduction}

Secure Multi-Party Computation (MPC) protocols support data privacy in
important modern, distributed applications such as privacy-preserving
machine learning \cite{li2021privacy, knott2021crypten,
  koch2020privacy, liu2020privacy} and Zero-Knowledge proofs in
blockchains \cite{ishai2009zero, lu2019honeybadgermpc,
  gao2022symmeproof, tomaz2020preserving}. The security semantics of
MPC include both confidentiality and integrity properties incorporated
into models such as real/ideal (aka simulator) security and universal
composability (UC), developed primarily by the cryptography community
\cite{evans2018pragmatic}.  Related proof methods are well-studied
\cite{Lindell2017} but mostly manual. Somewhat independently, a
significant body of work in programming languages has focused on
definition and enforcement of confidentiality and integrity
\emph{hyperproperties} \cite{10.5555/1891823.1891830} such as
noninterference and delimited release
\cite{sabelfeld2009declassification}. Following a tradition of
connecting cryptographic and PL-based security models
\cite{10.1007/3-540-44929-9_1,10.1145/3571740}, recent work has also
recognized connections between MPC security models and hyperpoperties
of, e.g., noninterference \cite{8429300}, and even leveraged these
connections to enforce MPC security through mechanisms such as
security types \cite{10.1145/3453483.3454074}. Major benefits of this
connection in an MPC setting include better language abstractions for
defining protocols and for mechanisation and even automation of
security proofs.  The goal of this paper is to develop a PL model for
defining a variety of low-level probabilistic MPC protocols, to
formulate a collection of confidentiality and integrity
hyperproperties for our model model with familiar information flow
analogs, and to show how these properties can be leveraged for
improved proof automation.

The distinction between high- and low-level languages for MPC is
important. High-level languages such as Wysteria
\cite{rastogi2014wysteria} and Viaduct \cite{10.1145/3453483.3454074}
are designed to provide effective programming of full
applications. These language designs incorporate sophisticated
verified compilation techniques such as orchestration
\cite{viaduct-UC} to guarantee high-level security properties, and
they rely on \emph{libraries} of low-level MPC protocols, such as
binary and arithmetic circuits. These low-level protocols encapsulate
abstractions such as secret sharing and semi-homomorphic encrypytion,
and rely on probabilistic progamming methods. So, low-level MPC
programming and protocol verification is a distinct challenge and both
critical to the general challenge of PL design for MPC and
complementary to high-level language design.

The connection between information flow hyperproperties and MPC
security is also complicated especially at a low level.  MPC protocols
involve communication between a group of distributed clients called a
\emph{federation} that collaboratively compute and publish the result
of some known \emph{ideal functionality} $\idealf$, maintaining
confidentiality of inputs to $\idealf$ without the use of a trusted
third party. However, since the outputs of $\idealf$ are public, some
information about inputs is inevitably leaked. Thus, the ideal
functionality establishes a declassification policy
\cite{sabelfeld2009declassification}, which is more difficult to
enforce than pure noninterference.  And subtleties of, e.g.,
semi-homomorphic encryption are central to both confidentiality and
integrity properties of protocols and similarly difficult to track
with coarse-grained security types alone.

Nevertheless, as previous authors have observed
\cite{5a51987acaa84c43bb4bf5bcc7d01683}, low-level protocol design
patterns such as secret sharing and circuit gate structure have
compositional properties that can be independently verified and then
leveraged in larger proof contexts. We contribute to this line of work
by developing an automated verification technique for subprotocols and
show how it can be integrated as a tactic in a larger security proof.

\compfig

\subsection{Related Work}
\label{section-related-work}

Our main focus is on PL design and automated and semi-automated
reasoning about security properties of low-level MPC protocols.  We
consider related systems in several dimensions, including
whether they are aimed at low-level design with probabilistic features,
whether they support reasoning about conditional probabilities which
are central to real/ideal security, whether they consider MPC
security through the lens of hyperproperties, and whether they consider
passive and/or malicious security models. We summarize this comparison
in Figure \ref{fig-comp}, with the caveat that works vary in the degree
of development in each dimension.

As mentioned above, several high-level languages have been developed
for writing MPC applications, and frequently exploit the connection
between hyperproperties and MPC security. Previous work on analysis
for the SecreC language
\cite{almeida2018enforcing,10.1145/2637113.2637119} is concerned with
properties of complex MPC circuits, in particular a user-friendly
specification and automated enforcement of declassification bounds in
programs that use MPC in subprograms. The Wys$^\star$ language
\cite{wysstar}, based on Wysteria \cite{rastogi2014wysteria}, has
similar goals and includes a trace-based semantics for reasoning about
the interactions of MPC protocols. Their compiler also guarantees that
underlying multi-threaded protocols enforce the single-threaded source
language semantics. These two lines of work were focused on passive
security. The Viaduct language
\cite{10.1145/3453483.3454074} has a well-developed
information flow type system that automatically enforces both
confidentiality and integrity through hyperproperties such as robust
declassification, in addition to rigorous compilation guarantees
through orchestration \cite{viaduct-UC}. However, these high level
languages lack probabilistic features and other abstractions of
low-level protocols, the implementation and security of which are
typically assumed as a selection of library compnonents.

Various related low-level languages with probabilistic features have
also been developed. The $\lambda_{\mathrm{obliv}}$ language
\cite{darais2019language} uses a type system is used to automatically
enforce so-called probabilistic trace obliviousness.  But similar to
previous work on oblivious data structures \cite{10.1145/3498713},
obliviousness is related to pure noninterference, not the relaxed form
related to passive MPC security. The Haskell-based security type system in
\cite{6266151} enforces a version of conditioned noninterence that is
sound for passive security in a probabilisitic low-level setting, but
they do not consider malicious security. And properties of real/ideal
passive and malicious security for a probabilistic language have been
formulated in EasyCrypt \cite{8429300}-- though their proof methods,
while mechanized, are fully manual, and their formulation of malicious
security is not as clearly related to robust declassification as is the
one we present in Section \ref{section-hyper}. 

Program logics for probabilistic languages and specifically reasoning
about properties such as joint probabilistic independence is also
important related work. Probabilistic Separation Logic (PSL)
\cite{barthe2019probabilistic} develops a logical framework for
reasoning about probabilistic independence (aka separation) in
programs, and they consider several (hyper)properties, such as perfect
secrecy of one-time-pads and indistinguishability in secret sharing,
that are critical to MPC. However, their methods are manual, and
don't include conditional independence (separation). This
latter issue has been addressed in Lilac \cite{li2023lilac}. The
application of Lilac-style reasoning to MPC protocols has not
previously been explored, as we do in Section
\ref{section-example-gmw}.

Our work also shares many ideas with probabilistic programming
languages designed to perform (exact or approximate) statistical
inference~\cite{holtzen2020scaling, carpenter2017stan, wood2014new,
  bingham2019pyro, albarghouthi2017fairsquare, de2007problog,
  pfeffer2009figaro, saad2021sppl}. Our setting, however, requires
verifying properties beyond inference, including conditional
statistical independence. Recent work by Li et al.~\cite{li2023lilac} proposes a
manual approach for proving such properties, but does not provide
automation.

\subsection{Overview and Contributions}

\paragraph{Language design.} In Section \ref{section-lang} we
develop a new probabilistic programming language $\minifed$ for
defining synchronous distributed protocols over an arbitrary
arithmetic field. The syntax and semantics provides a succinct account
of synchronous messaging between protocol \emph{clients}. In Section
\ref{section-metalang} we define a metalanguage $\metaprot$ that
dynamically generates $\minifed$ protocols. It is able to express
important low-level abstractions, as we illustrate via implementations
of protocols including Shamir addition (Section \ref{section-lang}),
GMW boolean circuits (Section \ref{section-example-gmw}), and Beaver
Triple multiplication gates with BDOZ authentication (Section
\ref{section-example-bdoz}).

\paragraph{Hyperproperty formulation.} In Section \ref{section-model} we
develop our formalism for expressing the joint probability mass function of
program variables, and give standard definitions of passive and
malicious real/ideal security in our model. In Section
\ref{section-hyper}, we formulate a variety of familiar information
flow properties in our probabilistic setting, including condition
noninterference, delimited release, and robust declassification, and
consider the relation between these and real/ideal security.  While it
has been previously shown that probabilistic conditional
noninterference is sound for passive security, we formulate new
properties of integrity which, paired with passive security, imply
malicious security (Theorem \ref{theorem-integrity}). We observe
in Section \ref{section-example-bdoz} that authentication mechanisms
such as BDOZ/SPDZ style MACs enforce a strictly weaker property
of ``cheating detection'' (Lemma \ref{lemma-cheating}). 


\paragraph{Fully and partially automated verification.} In Section
\ref{section-bruteforce} we develop a method for automatically
computing the pmf of $\minifed$ protocols in $\mathbb{F}_2$, that can
be automatically queried to enforce hyperperproperties of
security. This method is perfectly accurate but has high complexity;
we show this can be partially mitigated by conversion of protocols in
$\mathbb{F}_2$ to stratified Datalog which is amenable to HPC
acceleration. Furthermore, in Section \ref{section-example-gmw} we
consider in detail how this automated technique can be used as a local
automated tactic for proving security in arbitrarily large GMW
circuits using conditional probabilistic independence as in
\cite{li2023lilac} (Lemmas \ref{lemma-gmwtactic} and \ref{lemma-gmwinvariant}
and Theorem \ref{theorem-gmw}).
