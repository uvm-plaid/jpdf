\section{Automated Verification in $\mathbb{F}_2$}
\label{section-bruteforce}

In $\mathbb{F}_2$ a brute force strategy for computing
$\progtt(\prog)$ for any $\prog$ is to directly compute
$\runs(\prog)$. By querying $\progtt(\prog)$ we can verify any of the
hyperproperties discussed previously, or other properties such as
perfect secrecy \cite{XXX}.  Since $\runs(\prog)$ is exponential in
the size of $\secrets(\prog) \cup \flips(\prog)$, this strategy is
feasible only for smaller protocols. However, we can efficiently
convert protocols in the passive setting (without $\ttt{assert}$) to
stratified Datalog programs, and then extract $\runs(\prog)$ by
calculating Least Herbrand models by parallelization and/or other HPC
acceleration techniques for logic programs \cite{aspis2018linear}.
The rewriting we describe here is to Datalog with negation, with a
negation-as-failure model, though we can also use techniques in
\cite{sakama2017linear} to eliminate negation from resulting programs

In Figure \ref{fig-solve} we define the algorithm
$\solve{\stores}{\be}{\cid}$ which filters a given $\stores$ to obtain the
subset whose elements satisfy $\be$. In this figure and elsewhere for
$\mathbb{F}_2$ we use standard names of logical connectives
for field operations ($\eand$ and $\exor$ for $*$ and $+$ respectively) and
add $\enot$ and $\eor$ as trivial but convenient extensions. The correctness
of this operation is characterized as follows.
\begin{lemma}
  \label{lemma-solves}
  %For all $\stores$ and $\be$ with $\vars(\be) \subseteq \dom(\stores)$,
  $(\solve{\stores}{\be}{\cid}) = \{ \store \in \stores \ \mid\ \lcod{\store,\be}{\cid} = 1 \}$.
\end{lemma}

\solvefig

\begin{comment}
We immediately note that $\runs(\prog)$ can be obtained by a left-folding
of $\solvealg$ across $\prog$. 
\begin{lemma}
  \label{lemma-cruns}
  Given $\prog$ where $\iov(\prog) = S \cup V \cup O$ and $\flips(\prog) = R$. Define:
  \begin{eqnarray*}
    {tt}\ \ \stores\ (\xassign{x}{\be}{\cid}) &\defeq& \begin{array}{l}
      \mathrm{let}\ \stores' = \solve{\stores}{\be} \ \mathrm{in}\\
      \ \ \{\extend{\store}{\view{\cid}{w}}{1} \mid \store \in \stores' \}\ \cup\\
      \ \ \{\extend{\store}{\view{\cid}{w}}{0} \mid \store \in \stores - \stores' \}\end{array}\\[2mm]
    \cruns(\prog) &\defeq& \mathit{foldl}\ {tt}\ \mems(S \cup F)\ \prog
  \end{eqnarray*}
  Then $\cruns(\prog) = \runs(\prog)$.
\end{lemma}
\end{comment}

We define the syntax of Datalog as follows. As per standard
nomenclature, \emph{atoms} are $\minifed$ variables $x$,
\emph{literals} are atoms or negated atoms, and clause bodies are
conjunctions of literals.  A \emph{fact} is a clause with no body. A
\emph{Datalog program} is a list of clauses.
$$
\begin{array}{rclr}
  \mathit{body} &::=&  x \mid \neg x \mid x \wedge \mathit{body} \mid \neg x \wedge \mathit{body} \\
  \mathit{clause} &::=& x \gets \mathit{body} \mid x \gets
\end{array}
$$
When translating protocols, we need to extract the variables
that occur in expressions $\be$ computed by a client $\cid$,
written $\vars\ \be\ \cid$ where:
\begin{mathpar}
  \vars\ \secret{w} \cid \defeq \{ \elab{\secret{w}}{\cid} \}
  
  \vars\ \mesg{w} \cid \defeq \{ \elab{\mesg{w}}{\cid} \}

  \vars\ \rvl{w} \cid \defeq \{ \elab{\rvl{w}}{\cid} \}

  \vars\ \flip{w} \cid \defeq \{ \elab{\flip{w}}{\cid} \}

  \vars\ (\be_1 \exor \be_2)\ \cid \defeq (\vars\ \be_1\ \cid) \cup (\vars\ \be_2\ \cid)
\end{mathpar}
... and so on. Then, to begin converting a protocol $\prot$ to a Datalog
we define the function ${tt}$ that applies $\solve{\!}{\!}{\!\!}$ locally to each command
in $\prog$, obtaining the subset of memories that result
in a variable assignment of $1$.  
\begin{definition} Define:
$$
{tt}(\xassign{x}{\be}{\cid}) \defeq (x, (\solve{(\mems(\vars\ \be\ \cid))}{\be}{\cid}))
$$
\end{definition}
The mapping of ${tt}$ across a program
$\prog$-- i.e., $(\mathit{map}\ {tt}\ \prog)$--  essentially defines the
logic program, modulo some syntactic conversion. We can
accomplish the latter as follows, where $\datalog(\prog)$ defines the
full conversion.
\begin{definition} We define the conversion from memories to
  literals and clause bodies as follows:
\begin{mathpar}
  \logit{x \mapsto 1} \defeq x

  \logit{x \mapsto 0} \defeq \neg x

  \logit{\{ x_1 \mapsto \beta_1, \ldots, x_n \mapsto \beta_n\}} \defeq
  \logit{x_1 \mapsto \beta_1} \wedge \cdots \wedge \logit{x_n \mapsto \beta_n}
\end{mathpar}
Given pairs $(x,\stores)$ in the range of ${tt}$, we define the conversion
to clauses as  $\mathit{clauses}(x,\{ \store_1,...,\store_n \}) \defeq x \gets \logit{\store_1} \vee \cdots \vee x \gets \logit{\store_n}$.
The $\minifed$-to-Datalog conversion is then defined as:
$$
\datalog(\prog) \defeq  \mathit{map}\ \mathit{clauses}\ (\mathit{map}\ {tt}\ \prog)
$$
\end{definition}

In addition to converting view definitions to logic clauses, we also need to convert
secrets and random tapes. Since we assume given values for these in an arbitrary run of
the program, we can capture these as a particular fact base.
\begin{definition}
  Given $\store$, let $\{x_1,\ldots,x_n \} =
  \{ x \in \dom(\store) \mid \store(x) = 1 \}$.
  Then define $\mathit{facts}(\store) \defeq x_1 \gets, \ldots, x_n \gets$.
\end{definition}
The following result ties these pieces together and establishes
correctness of this approach.
\begin{lemma}
  For all $\prog$, 
  $\datalog(\prog)$ is a \emph{normal}, \emph{stratified}
  program \cite{aspis2018linear}, and $\store$ is the unique Least Herbrand
  Model of: $$(\mathit{facts}(\store_{\secrets(\prog) \cup \flips(\prog)}),\datalog(\prog))$$
  iff $\store \in \runs(\prog)$.
\end{lemma}
Finally, to compute $\runs(\prog)$, and thus $\progtt(\prog)$, we compute
the Least Herbrand Model $\store$ of $(\mathit{facts}(\store'),\datalog(\prog))$
for all $\store' \in \mems(\secrets(\prog) \cup \flips(\prog))$, observing
that model computation for distinct fact bases can be done in parallel. 

We can then query representations of $\runs(\prog)$ with
implementations of conditioning and marginalization to automatically
verify the passive model hyperproperties described in Section
\ref{section-hyper}-- in particular correctness and noninterference
modulo output, which imply passive security.  We have used this method
to verify security in $\mathbb{F}_2$ of Shamir addition as defined in Section
\ref{section-lang-example}, and single-gate $\eand$ and $\exor$ circuits with GMW
Beaver triples. We have also
verified properties compositional properties of gates themselves that establish
circuit invariants that imply passive security in larger circuits.
We discuss these examples and proof methods in Section \ref{section-examples}.


