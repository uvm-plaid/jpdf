\section{The $\minicat$ Protocol Language}

\begin{fpfig}[t]{Top-to-bottom: Basic $\minifed$ syntax, expression interpretation, and reduction rules.}{fig-minifed}
  {
    $$
    \begin{array}{rcl@{\hspace{8mm}}r}
      \multicolumn{4}{l}{v \in \mathbb{Z}_p,\ w \in \mathrm{String},\ \cid \in \mathrm{Clients} \subset  \mathbb{N} }\\[2mm] %, \bop \in \{ \eand, \eor, \exor \}} \\[2mm]
      \be &::=& v \mid \flip{w} \mid \secret{w} \mid \mesg{w} \mid \rvl{w} \mid \be \fminus \be \mid \be \fplus \be \mid \be \ftimes \be \mid f \mid \be\,\be & \textit{expressions}\\[2mm]
      x &::=& \elab{\flip{w}}{\cid} \mid \elab{\secret{w}}{\cid} \mid \elab{\mesg{w}}{\cid} \mid \rvl{w} \mid \out{\cid} & \textit{protocl variables} \\[2mm]
      %& &  \select{\be}{\be}{\be} \mid \ctxt{v}{k} \mid \key{w} \mid \sk{\be}(\be) \mid \pk{\be}{\be}(\be) \mid \pk{\be}{\be} \\[2mm]
      %& &  \select{\fp(\be)}{\be}{\be} \ctxt{v,\be}{k}  \mid \sk{\be}(\be) \mid \pk{\be}{\be}(\be) \mid \pk{\be}{\be} \\[2mm]
      \instr &::=& \eassign{\mesg{w}}{\cid}{\be}{\cid} \mid
      \reveal{w}{e}{\cid} \mid \pubout{\cid}{\be}{\cid} & \textit{commands} \\[2mm]
      \prog &::=& \varnothing \mid \instr; \prog & \textit{protocols}
    \end{array}
    $$
  
  \rule{130mm}{0.5pt}

  $$
  \begin{array}{c@{\hspace{5mm}}c}
  \begin{array}{rcl}
    \lcod{\store, v}{\cid} &=& v\\
    \lcod{\store, \be_1 \fplus \be_2}{\cid} &=& \fcod{\lcod{\store, \be_1}{\cid} \fplus \lcod{\store, \be_2}{\cid}}\\ 
    \lcod{\store, \be_1 \fminus \be_2}{\cid} &=& \fcod{\lcod{\store, \be_1}{\cid} \fminus \lcod{\store, \be_2}{\cid}}\\ 
    \lcod{\store, \be_1 \ftimes \be_2}{\cid} &=& \fcod{\lcod{\store, \be_1}{\cid} \ftimes \lcod{\store, \be_2}{\cid}}
  \end{array} & 
  \begin{array}{rcl}
    \lcod{\store, \flip{w}}{\cid} &=& \store(\elab{\flip{w}}{\cid})\\
    \lcod{\store, \secret{w}}{\cid} &=& \store(\elab{\secret{w}}{\cid})\\
    \lcod{\store, \mesg{w}}{\cid} &=& \store(\elab{\mesg{w}}{\cid})\\
    \lcod{\store, \rvl{w}}{\cid} &=& \store(\rvl{w})\\
    \lcod{\store, f\,e_1\,\cdots\, e_n}{\cid} &=& \delta(f,\lcod{\store, e_1}{\cid},\ldots,\lcod{\store,e_n}{\cid})
  \end{array}
  \end{array}
  $$

  \vspace{4mm}
  
  \rule{130mm}{0.5pt}

  \begin{mathpar}
    (\store, \eassign{\mesg{w}}{\cid_1}{\be}{\cid_2};\prog) \redx (\extend{\store}{\mesg{w}_{\cid_1}}{\lcod{\store,\be}{\cid_2}}, \prog)
    
    (\store, \reveal{w}{\be}{\cid};\prog) \redx (\extend{\store}{\rvl{w}}{\lcod{\store,\be}{\cid}}, \prog)
    
    (\store, \pubout{\cid}{\be}{\cid};\prog) \redx (\extend{\store}{\out{\cid}}{\lcod{\store,\be}{\cid}}, \prog)
  \end{mathpar}
  }
\end{fpfig}

The $\minifed$ language provides a simple model of synchronous
protocols between a federation of \emph{clients} exchanging values in
the binary field. We will identify clients by natural numbers, and
federations- finite sets of clients- are always given statically.
As we will see, our threat model assumes a partition of the federation
into \emph{honest} $H$ and \emph{corrupt} $C$ subsets.

We model probabilistic programming via a \emph{random tape}
semantics. That is, we will assume that programs can make reference to
values chosen from a uniform random distributions defined in the
initial program memory.  Programs aka protocols execute
deterministically given the random tape.

\subsection{Syntax} The syntax of $\minifed$, defined in
Figure \ref{fig-minifed}, includes values $v$ and standard
operations of addition, subtraction, and multiplication in
a finite field $\mathbb{Z}_p$ with $p$ prime. 
Protocols are given input secret values $\secret{w}$
as well as random samples $\flip{w}$ on the input
tape, both of which are distinguished by
strings $w$. Protocols are sequences of assignment
commands of three different forms:
\begin{itemize}
\item $\eassign{\mesg{w}}{\cid_2}{\be}{\cid_1}$: This
  is a \emph{message send} where expression $\be$ is computed
  by client $\cid_1$ and sent to client $\cid_2$ as message
  $\mesg{w}$.
\item $\reveal{w}{\be}{\cid}$: This
  is a \emph{public reveal} where expression $\be$ is computed
  by client $\cid$ and broadcast to the federation.
\item $\pubout{\cid}{\be}{\cid}$: This
  is an \emph{output} where expression $\be$ is computed
  by client $\cid$ and reported as its output.
\end{itemize}
Both messages $\mesg{w}$ and reveals $\rvl{w}$ can be
referenced in expressions, once they've been assigned.

We let $x$ range over \emph{variables}  which are identifiers
where client ownership is specified- e.g., $\elab{\mesg{\mathit{foo}}}{\cid}$
is a message $\mathit{foo}$ that was sent to $\cid$. We let $X$
range over sets of variables, and more specifically, $S$ ranges over sets of secret variables $\elab{\secret{w}}{\cid}$, $R$ ranges over sets of random variables $\elab{\flip{w}}{\cid}$, $M$ ranges over sets of message variables $\elab{\mesg{w}}{\cid}$, $P$ ranges over sets of public variables $\rvl{w}$, and $O$ ranges over sets of output variabels $\out{\cid}$.
Given a program $\prog$, we write $\iov(\prog)$ to
denote the set of $S \cup M \cup P \cup O$ of variables in $\prog$
with ownership made explicit, and we write $\flips(\prog)$ to
denote the set $R$ of random samplings in $\prog$ with ownership
made explicit. We write
$\vars(\prog)$ to denote $\iov(\prog) \cup \flips(\prog)$. For any set
of variables $X$ and parties $P$, we write $X_P$ to denote the subset
of $X$ owned by any party in $P$, in particular we write $X_H$ and $X_C$ to
denote the subsets belonging to honest and corrupt parties,
respectively.

\subsection{Semantics}

\emph{Memories} are fundamental to the semantics of $\fedcat$ and
provide the random tape and secret inputs to protocols, and record
view assignments. Memories $\store$ are finite (partial) mapping from
variables to values $v \in \mathbb{Z}_p$. The \emph{domain} of a
memory is written $\dom(\store)$ and is the finite set of variables on
which the memory is defined. We write $\store\{ x \mapsto v\}$ for
$x\not\in\dom(\store)$ to denote the memory $\store'$ such that
$\store'(x) = v$ and otherwise $\store'(y) = \store(y)$ for all $y
\in \dom(\store)$. We write $\store_1 \subseteq \store_2$ iff
$\dom(\store_1) \subseteq \dom(\store_2)$ and $\store_1(x) =
\store_2(x)$ for all $x \in \dom(\store_1)$. We write $\store_1 \cap
\store_2$ to denote the combination of $\store_1$ and $\store_2$
assuming $\store_1(x) = \store_2(x)$ for all $x \in \dom(\store_1)
\cap \dom(\store_2)$, otherwise $\store_1 \cap \store_2$ is undefined.
We write $\store_1 \subseteq \store_2$ iff $\store_1 \cap \store_2
= \store_1$.

Given a set of variables $X$, we write $\store_X$ to denote the
memory $\store$ restricted to the domain $X$, and we define
$\mems(X)$ as the set of all memories with domain $X$:
$$
\mems(X) \defeq \{ \store \mid \dom(\store) = X \}
$$
Thus, given a protocol $\prog$, the set of all random tapes for
$\prog$ is $\mems(\flips(\prog))$.
%We let $\stores$ range
%over sets of memories with the same domain, and abusing notation
%we write $\dom(\stores)$ to denote the common domain,
%and $\stores_X \defeq \{ \store_X | \store \in \stores \}$.

Given a variable-free boolean expression $\be$, we write $\cod{\be}$
to denote the standard interpretation of $\be$ in the arithmetic field
$\mathbb{Z}{p}$.
With the introduction of variables to expressions, we need to interpret
variables with respect to a specific memory, and all
variables used in an expression must belong to a specified client.  Thus,
we denote interpretation of expressions $\be$ possibly containing
variables as $\lcod{\store,\be}{\cid}$, where $\store$ associates
variables with values, and all variables must be owned by client
$\cid$. This is interpretation is defined in Figure \ref{fig-minifed}.


$$
\delta(\otp,v_1,v_2) \defeq v_1 \fplus v_2
$$

$$
\begin{array}{lll}
  \elab{\mesg{s1}}{2} &:=& \elab{(\otp\ \locflip\ (\flip{x} \fplus \secret{1})}{1} \\ 
  \elab{\mesg{s1}}{3} &:=& \elab{\flip{x}}{1} \\ 
  \elab{\mesg{s2}}{1} &:=& \elab{(\otp\ \locflip\ (\flip{x} \fplus \secret{2})}{2} \\ 
  \elab{\mesg{s2}}{3} &:=& \elab{\flip{x}}{2} \\ 
  \elab{\mesg{s3}}{1} &:=& \elab{(\otp\ \locflip\ (\flip{x} \fplus \secret{3})}{3} \\ 
  \elab{\mesg{s3}}{2} &:=& \elab{\flip{x}}{3} \\ 
  \rvl{1} &:=& \elab{(\locflip \fplus \mesg{s2} \fplus \mesg{s3})}{1} \\ 
  \rvl{2} &:=& \elab{(\mesg{s1} \fplus \locflip \fplus \mesg{s3})}{2} \\
  \rvl{3} &:=& \elab{(\mesg{s1} \fplus \mesg{s2} \fplus \locflip)}{3} \\ 
  \out{1} &:=& \elab{(\rvl{1} \fplus \rvl{2} + \rvl{3})}{1}\\
  \out{2} &:=& \elab{(\rvl{1} \fplus \rvl{2} + \rvl{3})}{2}\\
  \out{3} &:=& \elab{(\rvl{1} \fplus \rvl{2} + \rvl{3})}{3}
  %\elab{\mesg{o1}}{2} &:=& \elab{(\locflip \fplus \mesg{s2} \fplus \mesg{s3})}{1} \\ 
  %\elab{\mesg{o1}}{3} &:=& \elab{(\locflip \fplus \mesg{s2} \fplus \mesg{s3})}{1} \\ 
  %\elab{\mesg{o2}}{1} &:=& \elab{(\mesg{s1} \fplus \locflip \fplus \mesg{s3})}{2} \\
  %\elab{\mesg{o2}}{3} &:=& \elab{(\mesg{s1} \fplus \locflip \fplus \mesg{s3})}{2} \\ 
  %\elab{\mesg{o3}}{1} &:=& \elab{(\mesg{s1} \fplus \mesg{s2} \fplus \locflip)}{3} \\ 
  %\elab{\mesg{o3}}{2} &:=& \elab{(\mesg{s1} \fplus \mesg{s2} \fplus \locflip)}{3}\\ 
  %\pubout{1} &:=& \elab{(\locflip \fplus \mesg{s2} \fplus \mesg{s3} + \mesg{o2} + \mesg{o3})}{1}
\end{array}
$$

$$
(v, [m_1,\ldots,m_n])
$$
$$
k_1,\ldots,k_n
$$
$$
m_\cid = k_\cid + (k_\Delta * v)
$$

$$
\delta(\macplus,(v^1, [m_1^1,\ldots,m^1_n]),(v^2, [m_1^2,\ldots,m^2_n]))
\defeq
(v^1 \fplus v^2, [m_1^1 \fplus m_1^2 ,\ldots,m^1_n \fplus m_n^2])
$$

$$
\elab{\macv{\elab{v}{\cid}}}{1} \macplus \cdots \macplus \elab{\macv{\elab{v}{\cid}}}{n} =
(v,\ldots)
$$

$$
\delta(\macotp,v_1,v_2) \defeq v_1 \macplus v_2
$$

$$
\delta(\macauth, (v, [\ldots,m_\cid,\ldots]), k_\cid) \defeq
     (v, [\ldots,m_\cid,\ldots]) \text{\ if\ } m_i = k_i + (k_\Delta * v)
$$

$$
\begin{array}{lcl}
  \elab{\mesg{a}}{2} &:=&
  \elab{(\macotp\ \macv{\elab{\secret{x}}{1}}\ \macv{\elab{\flip{a}}{\Oracle}})}{1}\\
  \elab{\mesg{a}}{1} &:=&
  \elab{(\macotp\ \macv{\elab{\secret{x}}{1}}\ \macv{\elab{\flip{a}}{\Oracle}})}{2}\\
  \elab{\mesg{b}}{2} &:=&
  \elab{(\macotp\ \macv{\elab{\secret{y}}{2}}\ \macv{\elab{\flip{b}}{\Oracle}})}{1}\\
  \elab{\mesg{b}}{1} &:=&
  \elab{(\macotp\ \macv{\elab{\secret{y}}{2}}\ \macv{\elab{\flip{b}}{\Oracle}})}{2}\\
  \elab{\mesg{d}}{1} &:=&
  \elab{(\macauth(\mesg{a}, \mack{\elab{\secret{x}}{1}}{2} \fminus \mack{\elab{\flip{a}}{\Oracle}}{2}) \macplus (\macv{\elab{\secret{x}}{1}}\macminus\macv{\elab{\flip{a}}{\Oracle}}))}{1}\\
  \elab{\mesg{e}}{1}&:=&
  \elab{(\macauth(\mesg{b}, \mack{\elab{\secret{y}}{2}}{2} \fminus \mack{\elab{\flip{b}}{\Oracle}}{2}) \macplus (\macv{\elab{\secret{y}}{2}}\macminus\macv{\elab{\flip{b}}{\Oracle}}))}{1}\\
  \rvl{1} &:=&
  \elab{( (\mesg{d} \mactimes \mesg{e}) \macplus
          (\mesg{d} \mactimes \macv{\elab{\flip{b}}{\Oracle}}) \macplus
          (\mesg{e} \mactimes \macv{\elab{\flip{a}}{\Oracle}}) \macplus \macv{\elab{\secret{c}}{\Oracle}}
    )}{1}\\
  \elab{\mesg{d}}{2} &:=&
  \elab{(\macauth(\mesg{a}, \mack{\elab{\secret{x}}{1}}{1} \fminus \mack{\elab{\flip{a}}{\Oracle}}{1}) \macplus (\macv{\elab{\secret{x}}{1}}\macminus\macv{\elab{\flip{a}}{\Oracle}}))}{2}\\
  \elab{\mesg{e}}{2}&:=&
  \elab{(\macauth(\mesg{b}, \mack{\elab{\secret{y}}{2}}{1} \fminus \mack{\elab{\flip{b}}{\Oracle}}{1}) \macplus (\macv{\elab{\secret{y}}{2}}\macminus\macv{\elab{\flip{b}}{\Oracle}}))}{2}\\
  \rvl{2} &:=&
  \elab{( (\mesg{d} \mactimes \mesg{e}) \macplus
          (\mesg{d} \mactimes \macv{\elab{\flip{b}}{\Oracle}}) \macplus
          (\mesg{e} \mactimes \macv{\elab{\flip{a}}{\Oracle}}) \macplus \macv{\elab{\secret{c}}{\Oracle}}
    )}{2}\\
  \out{1} &:=& \elab{(\rvl{1} \macplus \macauth(\rvl{2},\ldots))}{1} \\
  \out{2} &:=& \elab{(\macauth(\rvl{1},\ldots) \macplus \rvl{2})}{2}
\end{array}
$$


