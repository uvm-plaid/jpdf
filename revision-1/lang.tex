\section{The $\minicat$ Protocol Language}

\begin{fpfig}[t]{Top-to-bottom: Basic $\minifed$ syntax, expression interpretation, and reduction rules.}{fig-minifed}
  {
    $$
    \begin{array}{rcl@{\hspace{8mm}}r}
      \multicolumn{4}{l}{v \in \mathbb{Z}_p,\ w \in \mathrm{String},\ \cid \in \mathrm{Clients} \subset  \mathbb{N} }\\[2mm] %, \bop \in \{ \eand, \eor, \exor \}} \\[2mm]
      \be &::=& v \mid \flip{w} \mid \secret{w} \mid \mesg{w} \mid \rvl{w} \mid \be \fminus \be \mid \be \fplus \be \mid \be \ftimes \be \mid f \mid \be\,\be & \textit{expressions}\\[2mm]
      x &::=& \elab{\flip{w}}{\cid} \mid \elab{\secret{w}}{\cid} \mid \elab{\mesg{w}}{\cid} \mid \rvl{w} \mid \out{\cid} & \textit{protocl variables} \\[2mm]
      %& &  \select{\be}{\be}{\be} \mid \ctxt{v}{k} \mid \key{w} \mid \sk{\be}(\be) \mid \pk{\be}{\be}(\be) \mid \pk{\be}{\be} \\[2mm]
      %& &  \select{\fp(\be)}{\be}{\be} \ctxt{v,\be}{k}  \mid \sk{\be}(\be) \mid \pk{\be}{\be}(\be) \mid \pk{\be}{\be} \\[2mm]
      \instr &::=& \eassign{\mesg{w}}{\cid}{\be}{\cid} \mid
      \reveal{w}{e}{\cid} \mid \pubout{\cid}{\be}{\cid} & \textit{commands} \\[2mm]
      \prog &::=& \varnothing \mid \instr; \prog & \textit{protocols}
    \end{array}
    $$
  
  \rule{130mm}{0.5pt}

  $$
  \begin{array}{c@{\hspace{5mm}}c}
  \begin{array}{rcl}
    \lcod{\store, v}{\cid} &=& v\\
    \lcod{\store, \be_1 \fplus \be_2}{\cid} &=& \fcod{\lcod{\store, \be_1}{\cid} \fplus \lcod{\store, \be_2}{\cid}}\\ 
    \lcod{\store, \be_1 \fminus \be_2}{\cid} &=& \fcod{\lcod{\store, \be_1}{\cid} \fminus \lcod{\store, \be_2}{\cid}}\\ 
    \lcod{\store, \be_1 \ftimes \be_2}{\cid} &=& \fcod{\lcod{\store, \be_1}{\cid} \ftimes \lcod{\store, \be_2}{\cid}}
  \end{array} & 
  \begin{array}{rcl}
    \lcod{\store, \flip{w}}{\cid} &=& \store(\elab{\flip{w}}{\cid})\\
    \lcod{\store, \secret{w}}{\cid} &=& \store(\elab{\secret{w}}{\cid})\\
    \lcod{\store, \mesg{w}}{\cid} &=& \store(\elab{\mesg{w}}{\cid})\\
    \lcod{\store, \rvl{w}}{\cid} &=& \store(\rvl{w})\\
    \lcod{\store, f\,\be_1\,\cdots\, \be_n}{\cid} &=& \delta(f,\lcod{\store, \be_1}{\cid},\ldots,\lcod{\store,\be_n}{\cid})
  \end{array}
  \end{array}
  $$

  \vspace{4mm}
  
  \rule{130mm}{0.5pt}

  \begin{mathpar}
    (\store, \eassign{\mesg{w}}{\cid_1}{\be}{\cid_2};\prog) \redx (\extend{\store}{\mesg{w}_{\cid_1}}{\lcod{\store,\be}{\cid_2}}, \prog)
    
    (\store, \reveal{w}{\be}{\cid};\prog) \redx (\extend{\store}{\rvl{w}}{\lcod{\store,\be}{\cid}}, \prog)
    
    (\store, \pubout{\cid}{\be}{\cid};\prog) \redx (\extend{\store}{\out{\cid}}{\lcod{\store,\be}{\cid}}, \prog)
  \end{mathpar}
  }
\end{fpfig}

The $\minifed$ language provides a simple model of synchronous
protocols between a federation of \emph{clients} exchanging values in
the binary field. We will identify clients by natural numbers, and
federations- finite sets of clients- are always given statically.
As we will see, our threat model assumes a partition of the federation
into \emph{honest} $H$ and \emph{corrupt} $C$ subsets.

We model probabilistic programming via a \emph{random tape}
semantics. That is, we will assume that programs can make reference to
values chosen from a uniform random distributions defined in the
initial program memory.  Programs aka protocols execute
deterministically given the random tape.

\subsection{Syntax} The syntax of $\minifed$, defined in
Figure \ref{fig-minifed}, includes values $v$ and standard
operations of addition, subtraction, and multiplication in
a finite field $\mathbb{Z}_p$ with $p$ prime. 
Protocols are given input secret values $\secret{w}$
as well as random samples $\flip{w}$ on the input
tape, both of which are distinguished by
strings $w$. Protocols are sequences of assignment
commands of three different forms:
\begin{itemize}
\item $\eassign{\mesg{w}}{\cid_2}{\be}{\cid_1}$: This
  is a \emph{message send} where expression $\be$ is computed
  by client $\cid_1$ and sent to client $\cid_2$ as message
  $\mesg{w}$.
\item $\reveal{w}{\be}{\cid}$: This
  is a \emph{public reveal} where expression $\be$ is computed
  by client $\cid$ and broadcast to the federation.
\item $\pubout{\cid}{\be}{\cid}$: This
  is an \emph{output} where expression $\be$ is computed
  by client $\cid$ and reported as its output.
\end{itemize}
Both messages $\mesg{w}$ and reveals $\rvl{w}$ can be
referenced in expressions, once they've been assigned.

We let $x$ range over \emph{variables}  which are identifiers
where client ownership is specified- e.g., $\elab{\mesg{\mathit{foo}}}{\cid}$
is a message $\mathit{foo}$ that was sent to $\cid$. We let $X$
range over sets of variables, and more specifically, $S$ ranges over sets of secret variables $\elab{\secret{w}}{\cid}$, $R$ ranges over sets of random variables $\elab{\flip{w}}{\cid}$, $M$ ranges over sets of message variables $\elab{\mesg{w}}{\cid}$, $P$ ranges over sets of public variables $\rvl{w}$, and $O$ ranges over sets of output variabels $\out{\cid}$.
Given a program $\prog$, we write $\iov(\prog)$ to
denote the set of $S \cup M \cup P \cup O$ of variables in $\prog$
with ownership made explicit, and we write $\flips(\prog)$ to
denote the set $R$ of random samplings in $\prog$ with ownership
made explicit. We write
$\vars(\prog)$ to denote $\iov(\prog) \cup \flips(\prog)$. For any set
of variables $X$ and parties $P$, we write $X_P$ to denote the subset
of $X$ owned by any party in $P$, in particular we write $X_H$ and $X_C$ to
denote the subsets belonging to honest and corrupt parties,
respectively.

\subsubsection{Library Functions} $\minifed$ expression syntax also supports
calls to library functions $f$ which can be applied to muliple arguments in a
curried style. This allows encapsulation and separate
definition of primitive operations such as one-time-pads and message
authentication, as we will illustrate with examples. This approach is
useful since it parameterizes these definitions, and 
supports verification of behavior specified with types, as we
discuss in Section \ref{section-types}.

\subsection{Semantics}

\emph{Memories} are fundamental to the semantics of $\fedcat$ and
provide random tape and secret inputs to protocols, and also record
message sends, public broadcast, and client outputs. Memories $\store$ are finite
(partial) mapping from variables $x$ to values $v \in \mathbb{Z}_p$. The \emph{domain} of a
memory is written $\dom(\store)$ and is the finite set of variables on
which the memory is defined. We write $\store\{ x \mapsto v\}$ for
$x\not\in\dom(\store)$ to denote the memory $\store'$ such that
$\store'(x) = v$ and otherwise $\store'(y) = \store(y)$ for all $y
\in \dom(\store)$. We write $\store \subseteq \store'$ iff
$\dom(\store) \subseteq \dom(\store')$ and $\store(x) =
\store'(x)$ for all $x \in \dom(\store)$. We write $\store \cap
\store'$ to denote the combination of $\store$ and $\store'$
assuming $\store(x) = \store'(x)$ for all $x \in \dom(\store)
\cap \dom(\store')$, otherwise $\store \cap \store'$ is undefined.
We write $\store \subseteq \store'$ iff $\store \cap \store'
= \store$.

Given a set of variables $X$, we write $\store_X$ to denote the
memory $\store$ restricted to the domain $X$, and we define
$\mems(X)$ as the set of all memories with domain $X$:
$$
\mems(X) \defeq \{ \store \mid \dom(\store) = X \}
$$
Thus, given a protocol $\prog$, the set of all random tapes for
$\prog$ is $\mems(\flips(\prog))$.
%We let $\stores$ range
%over sets of memories with the same domain, and abusing notation
%we write $\dom(\stores)$ to denote the common domain,
%and $\stores_X \defeq \{ \store_X | \store \in \stores \}$.

Given a variable-free expression $\be$, we write $\cod{\be}$ to denote
the standard interpretation of $\be$ in the arithmetic field
$\mathbb{Z}_{p}$. With the introduction of variables to expressions,
we need to interpret variables with respect to a specific memory, and
all variables used in an expression must belong to a specified client.
Thus, we denote interpretation of expressions $\be$ computed on a
client $\cid$ as $\lcod{\store,\be}{\cid}$. This interpretation is
defined in Figure \ref{fig-minifed}. It is also parameterized by
$\delta$ which defines the semantics of library functions $f$.

The small-step reduction relation $\redx$ is then defined in Figure
\ref{fig-minifed} to evaluate commands. Reduction is a relation on
\emph{configurations} $(\store, \prog)$ where all three command forms-
message send, broadcast, and output- are implemented as updates to the
memory $\store$. We write $\redxs$ to denote the reflexive, transitive
closure of\ $\redx$. 

\subsection{Example: Passive Secure Addition}

Shamir addition leverages homomorphic properties of addition in
arithmetic fields to implement secret addition. If a field value $v_1$
is in a uniform random distribution with other variables in a program,
then $v_1 \fplus v_2$ is an encryption of $v_2$ where $v_1$ is an
information theoretically secure one-time-pad, which is exploited for
secret sharing. Of course, $\fplus$ is also a meaningful operation
over any two field values regardless of their distributions.

To capture this distinction we introduce a function $\otp$
with the following specification:
$$
\delta(\otp,v_1,v_2) \defeq \fcod{v_1 \fplus v_2}
$$
Although the semantics are the same as addition, the use of $\otp$
makes a declarative distinction. But more importantly we will be
able to assign a type to $\otp$ that enforces the one-time discipline
on its first argument via type linearity, and reflects hiding of
its second argument via security type labels, as will be discussed in Section
\ref{section-types}.

To sum their secret values $\secret{\cid}$, each client $\cid$ in
the federation $\{ 1, 2, 3 \}$  samples a value $\locflip$
that can be used as a one-time pad in summation with another
random sample $\flip{x}$ and $\secret{\cid}$. This yields
two secret shares communicated as messages to the other clients,
while each client keeps $\locflip$ as its own share.
$$
\begin{array}{lll}
  \elab{\mesg{s1}}{2} &:=& \elab{(\otp\ \locflip\ (\flip{x} \fplus \secret{1})}{1} \\ 
  \elab{\mesg{s1}}{3} &:=& \elab{\flip{x}}{1} \\ 
  \elab{\mesg{s2}}{1} &:=& \elab{(\otp\ \locflip\ (\flip{x} \fplus \secret{2})}{2} \\ 
  \elab{\mesg{s2}}{3} &:=& \elab{\flip{x}}{2} \\ 
  \elab{\mesg{s3}}{1} &:=& \elab{(\otp\ \locflip\ (\flip{x} \fplus \secret{3})}{3} \\ 
  \elab{\mesg{s3}}{2} &:=& \elab{\flip{x}}{3}
\end{array}
$$
Due to field properties of $\fplus$ this scheme guarantees that messages
are viewed as random noise by any observer 
besides $\cid$ \cite{barthe2019probabilistic}. Next, each client
publicly reveals the sum of all of its shares, including its local
share. This does reveal information about secrets. Further there
is no one-time-pad to use in this summation.
$$
\begin{array}{lll}
  \rvl{1} &:=& \elab{(\locflip \fplus \mesg{s2} \fplus \mesg{s3})}{1} \\ 
  \rvl{2} &:=& \elab{(\mesg{s1} \fplus \locflip \fplus \mesg{s3})}{2} \\
  \rvl{3} &:=& \elab{(\mesg{s1} \fplus \mesg{s2} \fplus \locflip)}{3} 
\end{array}
$$
Finally, each client outputs the sum of each sum of shares, yielding
the sum of secrets. Note that this stage exposes no more information
than the previous public reveals. 
$$
%\elab{\mesg{o1}}{2} &:=& \elab{(\locflip \fplus \mesg{s2} \fplus \mesg{s3})}{1} \\ 
  %\elab{\mesg{o1}}{3} &:=& \elab{(\locflip \fplus \mesg{s2} \fplus \mesg{s3})}{1} \\ 
  %\elab{\mesg{o2}}{1} &:=& \elab{(\mesg{s1} \fplus \locflip \fplus \mesg{s3})}{2} \\
  %\elab{\mesg{o2}}{3} &:=& \elab{(\mesg{s1} \fplus \locflip \fplus \mesg{s3})}{2} \\ 
  %\elab{\mesg{o3}}{1} &:=& \elab{(\mesg{s1} \fplus \mesg{s2} \fplus \locflip)}{3} \\ 
  %\elab{\mesg{o3}}{2} &:=& \elab{(\mesg{s1} \fplus \mesg{s2} \fplus \locflip)}{3}\\ 
  %\pubout{1} &:=& \elab{(\locflip \fplus \mesg{s2} \fplus \mesg{s3} + \mesg{o2} + \mesg{o3})}{1}
\begin{array}{lll}
  \out{1} &:=& \elab{(\rvl{1} \fplus \rvl{2} + \rvl{3})}{1}\\
  \out{2} &:=& \elab{(\rvl{1} \fplus \rvl{2} + \rvl{3})}{2}\\
  \out{3} &:=& \elab{(\rvl{1} \fplus \rvl{2} + \rvl{3})}{3}
\end{array}
$$
It is well-known that additive secret sharing is passive
secure. That is, any adversarial observer can gain no more information
from the messages exchanged in the protocol than what is exposed by
the output alone. However, malicious adversaries can corrupt this
protocol by injecting ``fake'' sums of shares in their public reveals.

\subsection{Example: Malicious Secure Product}

\begin{fpfig}[t]{Top-to-bottom: Extended $\minifed$ syntax with BDOZ-style shares,
    interpretation of operations, and pre-processing properties of memories.}{fig-bdoz}
  {
$$
\begin{array}{rcl@{\hspace{8mm}}r}
  \multicolumn{3}{l}{m,k \in \mathbb{Z}_p \qquad \macv ::= (v,m)} &\\[2mm]
  %\textit{MACed\ values}\\[2mm]
  \be &::=& \cdots \mid \macv \mid \be \macplus \be \mid \be \mactimes \be \mid \be \macminus \be\\[2mm]
  x &::=& \cdots \mid \macx{\secret{w}}{\cid} \mid \macx{\flip{w}}{\cid} \mid \mack{x}{}
\end{array}
$$

\rule{130mm}{0.5pt}
\smallskip
$$
\begin{array}{rcl}
  \fcod{(v_1, m_1) \macplus (v_2, m_2)} &=& (v_1 \fplus v_2, m_1 \fplus m_2)\\[2mm]
  \fcod{(v, m) \mactimes v'} &=& (v \ftimes v', m \ftimes v')
\end{array}
$$
$$
\begin{array}{rcl}
  \lcod{\store, \be_1 \macplus \be_2}{\cid} &=&
  \fcod{\lcod{\store,\be_1}{\cid} \macplus \lcod{\store,\be_2}{\cid}}\\[2mm]  
  \lcod{\store, \be_1 \mactimes \be_2}{\cid} &=&
  \fcod{\lcod{\store,\be_1}{\cid} \mactimes \lcod{\store,\be_2}{\cid}}
\end{array}
$$

\medskip

\rule{130mm}{0.5pt}

\begin{mathpar}
  \inferrule
      {\store(\macx{x}{\cid_1}) = (v_1,m_1) \\
        \store(\macx{x}{\cid_2}) = (v_2,m_2)}
      {\store(x) = \fcod{v_1 \fplus v_2}}
      
  \inferrule
      {\store(\elab{x}{\cid'}) = (v,m) \\ \cid \ne \cid'}
      {m = \lcod{\store, \mack{x}{} \fplus \flip{\ttt{delta}} \ftimes v}{\cid}}

  \store(\elab{\secret{c}}{\Oracle}) = \lcod{\store,\flip{a} \ftimes \flip{b}}{\Oracle}
\end{mathpar}
}
\end{fpfig}

Eliminating cheating by adding information theoretic secure MAC codes
to shares is a fundamental approach realized by protocols such as BDOZ
\cite{XXX} and SPDZ \cite{XXX}.  These protocols assume a
pre-processing phase that distributes shares with MAC codes to
clients.  This integrates well with pre-processed Beaver Triples to
implement malicious secure, and relatively efficient, multiplication
\cite{XXX}.  Here we show how to implement malicious secure two party
multiplication in the BDOZ style via some extensions of $\minifed$.

A field value $v$ is secret shared among 2 clientz in BDOZ as follows.
Each client $\cid$ gets a pair of the form $(v_\cid,m_\cid)$, where
$v_\cid$ is a secret share of $v$, i.e., $v = \fcod{v_1 \fplus v_2}$.
Further, field values $m_\cid$ are MACs of $v_\cid$ that are authenticated
by the other client's key. In particular, $m_\cid = k + (k_\Delta *
v_\cid)$ where \emph{keys} $k$ and $k_\Delta$ are known only to $\cid'
\ne \cid$ and $k_\Delta$. Further, there is a unique \emph{local key}
$k$ for each shared value, while the \emph{global key} $k_\Delta$ is
common to all MACs authenticated by $\cid'$. By extending certain
field operations to these MACed values, a semi-homorphic encryption
scheme is obtained that is adequate to compute Beaver Triples. For
example, if local keys $k_1$ and $k_2$ authenticate $\macv_1$ and
$\macv_2$, then $k_1 \fplus k_2$ authenticates $\macv_1 \macplus
\macv_2$, where $\macplus$ is addition extended to MACed
values. Semantics for summation of MACed shares, and multipication of
shares and plain field values, are given in Figure \ref{fig-bdoz}.
For more details the reader is referred to \cite{XXX}. We note
that while we restrict values $v$, $m$, and $k$ to the same
field in this presentation for simplicity, in general $m$ and
$k$ can be in extensions of $\mathbb{Z}_p$. 

We define variable forms $\macx{\secret{w}}{\cid}$ and
$\macx{\flip{w}}{\cid}$ to range over local MACed shares of a secret
and random sample owned by client $\cid$, respectively. We let
$\mack{x}{}$ denote the local key for authenticating shares of $x$.
We also assume a global key $\flip{\ttt{delta}}$ on each client.
Pre-processing is modeled by assuming that shares are provided on
input random tapes (memories) with the required properties, as defined
on the bottom of Figure \ref{fig-bdoz}.

Similarly to field values, we can use MACed shares in
uniform independent distributions as one-time-pads for
hiding other values, using subtraction over MACed shares.
So we introduce a function $\macotp$ to do this
declaratively.
$$
\delta(\macotp,v_1,v_2) \defeq \fcod{v_1 \macminus v_2}
$$

Another key idea in BDOZ is \emph{secure opening}, where distributed
shares of a value $v$ are communicated and combined to ``open'' the
same $v$ on each client. This relies on authentication to detect
cheating in the malicious setting. To implement secure opening we
define a $\macauth$ function that authenticates shares, and a
$\macopen$ function that extracts a field value from its shares.
\begin{mathpar}
\delta(\macauth, (v, m), k) \defeq
     (v, m) \text{\ if\ } m = k \fplus (\flip{\ttt{delta}} \ftimes v)
 
\delta(\macopen, (v_1, m_1), (v_2,m_2)) \defeq
\fcod{v_1 \fplus v_2}
\end{mathpar}
This authentication eliminates cheating and guarantees protocol
integrity. As we will show in Section \ref{section-types}, this
can be reflected in a security type for $\macauth$.

Putting these pieces together, we can implement 2-party multiplication with
Beaver Triples as follows. Recall that Beaver Triples are values
$(a,b,c)$ such that $c = \fcod{b \ftimes a}$- we implement them here
as a secret $\secret{c}$ and samples $\flip{a}$ and $\flip{b}$ owned
by an \emph{oracle} $\Oracle$ that we always assume is honest but is
otherwise not involved in the protocol. We define pre-processing
properties of Beaver Triples in Figure \ref{fig-bdoz}. Only one
Beaver triple can be used per multiplication to ensure secure
openings, but our scheme can be extended with more Beaver triples
and reuse of them will be rule out by type linearity as we will
discuss in Section \ref{section-types}.

First, clients 1 and 2 compute (aka \emph{open}) $\secret{x} \fplus
\flip{a}$ and $\secret{y} \fplus \flip{b}$ where $\secret{x}$ and
$\secret{y}$ are 1 and 2's secrets respectively.  Then they both
perform a secure opening of $\secret{x} \fplus \flip{a}$ and
$\secret{y} \fplus \flip{b}$, by authenticating each others' shares,
and record them as $\mesg{d}$ and $\mesg{e}$ respectively. Note
that the secrets remain hidden since $\flip{a}$ and $\flip{b}$
are random values.
$$
\begin{array}{lcl}
  \elab{\mesg{a}}{2} &:=&
  \elab{(\macotp\ \macgv{\elab{\flip{a}}{\Oracle}}\ \macgv{\elab{\secret{x}}{1}})}{1}\\
  \elab{\mesg{a}}{1} &:=&
  \elab{(\macotp\ \macgv{\elab{\flip{a}}{\Oracle}}\ \macgv{\elab{\secret{x}}{1}})}{2}\\
  \elab{\mesg{b}}{2} &:=&
  \elab{(\macotp\ \macgv{\elab{\flip{b}}{\Oracle}}\ \macgv{\elab{\secret{y}}{2}})}{1}\\
  \elab{\mesg{b}}{1} &:=&
  \elab{(\macotp\ \macgv{\elab{\flip{b}}{\Oracle}}\ \macgv{\elab{\secret{y}}{2}})}{2}\\
  \elab{\mesg{d}}{1} &:=&
  \elab{(\macopen\ \macauth(\mesg{a}, \mack{\macx{\secret{x}}{1}}{2} \fminus \mack{\macx{\flip{a}}{\Oracle}}{2}) \ (\macgv{\elab{\secret{x}}{1}}\macminus\macgv{\elab{\flip{a}}{\Oracle}}))}{1}\\
  \elab{\mesg{e}}{1}&:=&
  \elab{(\macopen\ \macauth(\mesg{b}, \mack{\macx{\secret{y}}{2}}{2} \fminus \mack{\macx{\flip{b}}{\Oracle}}{2}) \ (\macgv{\elab{\secret{y}}{2}}\macminus\macgv{\elab{\flip{b}}{\Oracle}}))}{1}\\
  \elab{\mesg{d}}{2} &:=&
  \elab{(\macopen\ \macauth(\mesg{a}, \mack{\macx{\secret{x}}{1}}{1} \fminus \mack{\macx{\flip{a}}{\Oracle}}{1}) \ (\macgv{\elab{\secret{x}}{1}}\macminus\macgv{\elab{\flip{a}}{\Oracle}}))}{2}\\
  \elab{\mesg{e}}{2}&:=&
  \elab{(\macopen\ \macauth(\mesg{b}, \mack{\macx{\secret{y}}{2}}{1} \fminus \mack{\macx{\flip{b}}{\Oracle}}{1}) \ (\macgv{\elab{\secret{y}}{2}}\macminus\macgv{\elab{\flip{b}}{\Oracle}}))}{2}
\end{array}
$$
Now, each client uses their share of $\secret{c}$ to generate their
share of the final output, which is again reported via a secure opening.
Homomorphic key calculations are elided for brevity.
$$
\begin{array}{lcl}
  \rvl{1} &:=&
  \elab{( (\mesg{d} \ftimes \mesg{e}) \macplus
          (\mesg{d} \mactimes \macgv{\elab{\flip{b}}{\Oracle}}) \macplus
          (\mesg{e} \mactimes \macgv{\elab{\flip{a}}{\Oracle}}) \macplus \macgv{\elab{\secret{c}}{\Oracle}}
    )}{1}\\
  \rvl{2} &:=&
  \elab{( (\mesg{d} \ftimes \mesg{e}) \macplus
          (\mesg{d} \mactimes \macgv{\elab{\flip{b}}{\Oracle}}) \macplus
          (\mesg{e} \mactimes \macgv{\elab{\flip{a}}{\Oracle}}) \macplus \macgv{\elab{\secret{c}}{\Oracle}}
    )}{2}\\
  \out{1} &:=& \elab{(\macopen\ \rvl{1} \ \macauth(\rvl{2},\ldots))}{1} \\
  \out{2} &:=& \elab{(\macopen\ \macauth(\rvl{1},\ldots) \ \rvl{2})}{2} 
\end{array}
$$
Some information about input secrets is revealed by the product, but
secure opening ensures high integrity and hence robustness to
corruption, aka robust declassification.


