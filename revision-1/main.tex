\documentclass[acmsmall,screen,review]{acmart}

\usepackage{amsmath}
\usepackage{amstext}
\usepackage{fp-frame}
\usepackage[latin1]{inputenc}
\usepackage{mathpartir}
\usepackage{fancyvrb}
\usepackage{moreverb}
\usepackage{stmaryrd}
\usepackage{enumerate}
\usepackage{thmtools,thm-restate}
\usepackage{comment}
\usepackage{booktabs,array}

\newcommand{\note}[1]{\noindent\textit{(\textbf{$\star$note$\star$:}\ \ #1)}}

\newcommand{\evals}{\Downarrow}
\newcommand{\diverges}{\Uparrow}
\newcommand{\intt}{\mathrm{int}}
\newcommand{\unitt}{\mathrm{unit}}
\newcommand{\boolt}{\mathrm{bool}}
\newcommand{\floatt}{\mathrm{float}}
\newcommand{\stringt}{\mathrm{string}}
\newcommand{\chart}{\mathrm{char}}
\newcommand{\vb}[1]{\verb+#1+}
\newcommand{\evalexmp}[2]{\texttt{#1}\ \ensuremath{\evals}\ \texttt{#2}}
\newcommand{\texmp}[2]{\texttt{#1\ :\ #2}}
\newcommand{\skipper}{\bigskip\\}
\newcommand{\fyi}{\noindent\textbf{\textit{fyi:}}\ }
\newcommand{\NB}{\noindent\textbf{NB:\ }}
\newcommand{\const}{\ensuremath{\mathbf{c}}}
\newcommand{\defn}{\heading{definition}}
\newcommand{\defeq}{\triangleq}
\newcommand{\nat}{\mathbb{N}}
\newcommand{\atom}{\texttt{const}}
\def\squareforqed{\hbox{\rlap{$\sqcap$}$\sqcup$}}
\def\qed{\ifmmode\squareforqed\else{\unskip\nobreak\hfil
\penalty50\hskip1em\null\nobreak\hfil\squareforqed
\parfillskip=0pt\finalhyphendemerits=0\endgraf}\fi}
\newcommand{\exampletab}[1]{\skipper\begin{tabular}{lll}#1\end{tabular}\skipper}
\newcommand{\verbtab}[1]{\skipper\begin{verbatimtab}{#1}\end{verbatimtab}\skipper}
\newcommand{\eqntab}[1]{\skipper\begin{tabular}{rcl}#1\end{tabular}\skipper}
\newcommand{\recdefn}[1]{\{#1\}}
\newcommand{\ttt}[1]{\texttt{#1}}
\newcommand{\gdesc}[1]{\text{\textit{#1}}}
\newcommand{\true}{\mathrm{true}}
\newcommand{\false}{\mathrm{false}}
\newcommand{\etrue}{\texttt{true}}
\newcommand{\efalse}{\texttt{false}}
\newcommand{\reval}{\Rightarrow}
\newcommand{\Dand}{\ \mathrm{and}\ }
\newcommand{\Dor}{\ \mathrm{or}\ }
\newcommand{\Dxor}{\ \mathrm{xor}\ }
\newcommand{\Dnot}{\mathrm{not}\ }
\newcommand{\cod}[1]{\llbracket #1 \rrbracket}
\newcommand{\lcod}[2]{\llbracket #1 \rrbracket_{#2}}
\newcommand{\Dplus}{\mathrm{Plus}}
\newcommand{\Dminus}{\mathrm{Minus}}
\newcommand{\Dequal}{\mathrm{=}}
\newcommand{\Dabs}[2]{(\mathrm{Function}\ #1 \rightarrow #2)}
\newcommand{\Dfix}[3]{(\mathrm{Fix}\ #1 . #2 \rightarrow #3)}
\newcommand{\Dite}[3]{\mathrm{If}\ #1\ \mathrm{Then}\ #2\ \mathrm{Else}\ #3}
\newcommand{\dotminus}{\stackrel{.}{-}}
\newcommand{\Dlet}[3]{\mathrm{Let}\ #1 = #2\ \mathrm{In}\ #3}
\newcommand{\Dletrec}[4]{\mathrm{Let\ Rec}\ #1\ #2 = #3\ \mathrm{In}\ #4}
\newcommand{\Dfst}{\mathrm{left}}
\newcommand{\Dsnd}{\mathrm{right}}
\newcommand{\labset}{\mathit{Lab}}
\newcommand{\Drec}[1]{\{ #1 \}}
\newcommand{\linfer}[3]{\inferrule*[right=(\TirName{#1})]{#2}{#3}}
\newcommand{\lab}[1]{\mathrm{#1}}
\newcommand{\loc}{\ell}
\newcommand{\Dref}[1]{\mathrm{Ref}\,#1}
%\newcommand{\store}{\mathcal{M}}
\newcommand{\store}{m}
%\newcommand{\stores}{\overline{\store}}
\newcommand{\stores}{M}
\newcommand{\config}[2]{\langle #1,#2 \rangle}
\newcommand{\configf}[2]{\begin{array}[t]{l}\langle #1\\,\\ #2 \rangle \end{array}}
\newcommand{\extend}[3]{#1\{#2 \mapsto #3\}}
\newcommand{\emptystore}{\{\}}
\newcommand{\storedefn}[1]{\{#1\}}
\newcommand{\Dret}[1]{\mathrm{Return}\,#1}
\newcommand{\Draise}[1]{\mathrm{Raise}\,#1}
\newcommand{\Dexn}[2]{\#\!#1\,#2}
\newcommand{\Dtry}[3]{\mathrm{Try}\,#1\,\mathrm{With}\,#2 \rightarrow #3}
\newcommand{\xname}{\mathit{exn}}
\newcommand{\Dboolt}{\mathrm{Bool}}
\newcommand{\Dreft}[1]{#1\,\mathrm{ref}}
\newcommand{\reft}[1]{#1\,\mathrm{ref}}
\newcommand{\Dintt}{\mathrm{Int}}
%\newcommand{\tjudge}[3]{#1 \vdash #2 : #3}
\newcommand{\textend}[3]{#1;#2:#3}
\newcommand{\fnty}[2]{#1 \rightarrow #2}
\newcommand{\TDabs}[3]{(\mathrm{Function}\ (#1 : #2) \rightarrow #3)}
\newcommand{\TDfix}[5]{(\mathrm{Fix}\ #1 . (#2 : #3) : #4 \rightarrow #5)}
\newcommand{\TDletrec}[6]{\Dletrec{#1}{#2 : #3}{#4 : #5}{#6}}
\newcommand{\emptyenv}{\varnothing}
\newcommand{\tfail}{\mathbf{fail}}
\newcommand{\tcheck}{\mathrm{TC}}
\newcommand{\tcheckfail}{\mathbf{TypeMismatch}}
\newcommand{\algtab}[1]
{
\vspace*{-3mm}
\begin{tabbing}
\hspace*{12mm}\=\hspace{9mm}\=\hspace{9mm}\=\hspace{6mm}\=\hspace{6mm}\=
\hspace{6mm}\=
#1
\end{tabbing}
}
\newcommand{\eassign}[2]{#1 := #2}
\newcommand{\ederef}[1]{\,!#1}
\newcommand{\declass}[2]{\mathrm{declassify}_{#2}(#1)}
\newcommand{\eendorse}[2]{\mathrm{endorse}_{#2}(#1)}
\newcommand{\lt}{\left\{}
\newcommand{\rt}{\right\}}
\newcommand{\Lt}{\left\{\!\!\right.}
\newcommand{\Rt}{\left.\!\!\right\}}
\newcommand{\tinfer}{\mathit{PT}}
\newcommand{\unify}{\mathit{unify}}
\newcommand{\tsubn}{\varphi}
\newcommand{\scheme}[2]{\forall #1 . #2}
\newcommand{\Dself}{\mathrm{this}}
\newcommand{\Dsuper}{\mathrm{super}}
\newcommand{\Dsend}[3]{#1.#2(#3)}
\newcommand{\Dselect}[2]{#1.#2}
\newcommand{\Demptyclass}{\mathrm{EmptyClass}}
%\newcommand{\Dclass}[3]{\mathrm{Class}\ \mathrm{Extends}\ #1\ \mathrm{Inst}
%\ #2\ \mathrm{Meth}\ #3}
\newcommand{\Dclass}[2]{\mathrm{Class}\ \mathrm{Inst} \ #1\ \mathrm{Meth}\ #2}
\newcommand{\Dobj}[2]{\mathrm{Object}\ \mathrm{Inst}\ #1\ \mathrm{Meth}\ #2}
%\newcommand{\Dclassf}[3]{
%\begin{array}[t]{l}
%\mathrm{Class}\ \mathrm{Extends}\ #1 \\
%\quad \mathrm{Inst}\\
%\qquad #2 \\ 
%\quad \mathrm{Meth}\\
%\qquad #3
%\end{array}
%}
\newcommand{\Dclassf}[3]{
\begin{array}[t]{l}
\mathrm{Class}\\
\quad \mathrm{Inst}\\
\qquad #1 \\ 
\quad \mathrm{Meth}\\
\qquad #2
\end{array}
}
\newcommand{\Dobjf}[2]{
\begin{array}[t]{l}
\mathrm{Object}\\
\quad \mathrm{Inst}\\
\qquad #1 \\ 
\quad \mathrm{Meth}\\
\qquad #2
\end{array}
}
\newcommand{\Dnew}[1]{\mathrm{New}\ #1}
\newcommand{\vtab}[1]{\begin{verbatimtab}[4]#1\end{verbatimtab}}

\newcounter{topiccounter}
\setcounter{topiccounter}{1}
\newcommand{\topic}[1]
    {\noindent \textbf{Topic \arabic{topiccounter}.\ \textit{#1}. } \stepcounter{topiccounter}}


\newcommand{\lcalc}{$\lambda$-calculus}
\newcommand{\redx}{\rightarrow}
\newcommand{\redxs}{\redx^*}
\newcommand{\idfn}{\mathit{ID}}
\newcommand{\mlfn}[2]{\mathrm{fun}\, #1 \rightarrow #2}
\newcommand{\mlrecfn}[3]{\mathrm{fix}\,#1.#2 \rightarrow #3}
\newcommand{\mlfix}{\mathrm{fix}}
\newcommand{\eite}[3]{\mathrm{if}\ #1\ \mathrm{then} \ #2\ \mathrm{else} \ #3\ }
\newcommand{\esucc}[1]{\texttt{succ}\ #1}
\newcommand{\epred}[1]{\texttt{pred}\ #1}
\newcommand{\eiszero}[1]{\texttt{iszero}\ #1}
\newcommand{\ezero}{\texttt{0}}
\newcommand{\elet}[3]{\mathrm{let}\ #1 = #2\ \mathrm{in}\ #3}
\newcommand{\eletrec}[3]{\mathrm{letrec}\ #1 = #2\ \mathrm{in}\ #3}
\newcommand{\fv}{\mathrm{fv}}
\newcommand{\ourml}{\mathit{ML}_{\mathit{Cat}}}
\newcommand{\raisexn}{\mathrm{raise}}
\newcommand{\handler}[3]{\mathrm{try}\, #1\, \mathrm{with}\, \exn(#2) \Rightarrow #3}
\newcommand{\exn}{\mathit{exn}}
\newcommand{\dom}{\mathrm{dom}}
\newcommand{\efst}{\mathrm{fst}}
\newcommand{\esnd}{\mathrm{snd}}
\newcommand{\natt}{\textrm{Nat}}
\newcommand{\earray}{\mathrm{array}}
\newcommand{\varray}{\alpha}
\newcommand{\length}{\mathit{length}}
\newcommand{\arrayml}{\ourml^{\earray}}
\newcommand{\stackml}{\ourml^{\mathit{stack}}}
\newcommand{\flowml}{\ourml^{\mathit{flow}}}
\newcommand{\taintml}{\ourml^{\mathit{taint}}}
\newcommand{\secfail}{\mathbf{secfail}}
\newcommand{\tr}{\theta}
\newcommand{\rewrite}[1]{\mathcal{R}(#1)}
%\newcommand{\secprop}{\mathcal{P}}
%\newcommand{\trprop}{\hat{\secprop}}
\newcommand{\Prop}{\mathbf{P}}
\newcommand{\Hprop}{\mathbf{H}}
\newcommand{\secprop}{\phi}
\newcommand{\hyprop}{\eta}
\newcommand{\trprop}{\gamma}
\newcommand{\tracess}{\Sigma}
\newcommand{\trsprop}{\sigma}
\newcommand{\traces}{\Psi}
\newcommand{\fpkeyword}[1]{\mathrm{#1}}
\newcommand{\ebinop}[2]{#1\,\mathit{binop}\,#2}
\newcommand{\eenablepriv}[2]{\fpkeyword{enable}\ #1\ \fpkeyword{for}\ #2}
\newcommand{\echeckpriv}[2]{\fpkeyword{check}\ #1\ \fpkeyword{then}\ #2}
\newcommand{\esigned}[2]{#1.#2}
\newcommand{\enabled}{\mathit{enabledprivs}}
%\newcommand{\acl}{\mathcal{A}}
\newcommand{\priv}{\pi}
\newcommand{\privs}{\mathit{R}}
\newcommand{\prin}{p}
\newcommand{\nobody}{\mathit{nobody}}
\newcommand{\po}{\preceq}
\newcommand{\seclattice}{\mathcal{S}}
\newcommand{\binsl}{\mathcal{S}_{\mathrm{bin}}}
\newcommand{\seclevs}{\mathcal{L}}
\newcommand{\latel}{\varsigma}
\newcommand{\hilab}{H}
\newcommand{\lolab}{L}
\newcommand{\hiloc}{\mathit{hi}}
\newcommand{\loloc}{\mathit{low}}
\newcommand{\labty}[2]{#1 \cdot #2}
\newcommand{\labval}[2]{#1 \cdot #2}
\newcommand{\mi}[1]{\mathit{#1}}
\newcommand{\pc}{\latel_{\mathit{pc}}}
\newcommand{\cfnty}[3]{#1 \rightarrow_{#2} #3}
\newcommand{\pow}{\mathrm{pow}}

%\newcommand{\tr}{\theta}


\newcommand{\problemheading}[1]{\noindent\textbf{#1}\ }
\newcounter{problemcounter}
\setcounter{problemcounter}{1}

\newcommand{\EC}[1]
    {\problemheading{Extra Credit \textit{(#1 points)}.}}

\newcommand{\pproblem}[1]
    {\problemheading{Problem \arabic{problemcounter} \textit{(#1 points)}.} \stepcounter{problemcounter}}
           
\newcommand{\gproblem}[1]
    {\problemheading{Problem \arabic{problemcounter} \textit{(Graduate Students Only; #1 points)}.} \stepcounter{problemcounter}}

\newcommand{\problem}
    {\problemheading{Problem \arabic{problemcounter}.} \stepcounter{problemcounter}}

\newcounter{solncounter}
\setcounter{solncounter}{1}
\newcommand{\solution}
    {\problemheading{Solution to Problem \arabic{solncounter}.} \stepcounter{solncounter}}

\newcommand{\chash}{\mathcal{H}}
\newcommand{\acl}{\mathit{Auth}}
\newcommand{\opn}{\mathit{op}}
\newcommand{\egid}{\mathit{egid}}
\newcommand{\euid}{\mathit{euid}}
\newcommand{\suid}{\ttt{suid}}
\newcommand{\sgid}{\ttt{sgid}}
\newcommand{\uxroot}{\ttt{root}}
\newcommand{\fowner}[1]{\mathit{owner}_{#1}}
\newcommand{\fgroup}[1]{\mathit{group}_{#1}}
\newcommand{\gprivs}[1]{\mathit{Privs_{#1}}.\mathit{group}}
\newcommand{\uprivs}[1]{\mathit{Privs_{#1}}.\mathit{owner}}
\newcommand{\oprivs}[1]{\mathit{Privs_{#1}}.\mathit{other}}
\newcommand{\uxprivs}[1]{\mathit{Privs_{#1}}}

\newcommand{\seclab}{\mathcal{L}}
\newcommand{\sle}{\preceq}
\newcommand{\ile}{\preceq_I}
\newcommand{\ilab}{\seclab_I}

\newcommand{\minifed}{\mathit{Overture}}
\newcommand{\minicat}{\minifed}
\newcommand{\fedprot}{\minifed}
\newcommand{\metaprot}{\mathit{Prelude}}
\newcommand{\mlscat}{\mathit{mlscat}}
\newcommand{\flowcat}{\mathit{flowcat}}
\newcommand{\dflowcat}{\mathit{dflowcat}}
\newcommand{\minicatde}{\mathit{minicat}_{\mathit{de}}}
\newcommand{\minicatexp}{\mathit{minicat}_{\mathit{taint}}}
%\newcommand{\prog}{\mathcal{P}}
\newcommand{\prog}{\pi}
\newcommand{\main}{\mathit{main}}
\renewcommand{\reval}{\redx}
\renewcommand{\Dite}{\eite}


%\renewcommand{\labty}[2]{#2}
\newcommand{\fnsty}{\Sigma}
\newcommand{\secty}{\latel}

\newcommand{\tc}{\mathrm{TC}}
\newcommand{\validate}{\mathrm{validate}}


\newcommand{\mlsid}[1]{\mathrm{mls}(#1)}
\newcommand{\mlsredx}[1]{\redx_{\mlsid{#1}}}
\newcommand{\confid}{\mathit{flow}}
\newcommand{\taintid}{\mathit{dflow}}
\newcommand{\credx}{\redx_{\confid}}
\newcommand{\tredx}{\redx_{\taintid}}
\newcommand{\ccod}[1]{\lcod{\confid}{#1}}
\newcommand{\tcod}[1]{\lcod{\taintid}{#1}}
\renewcommand{\mod}{\ \textrm{mod}\ }


\newcommand{\mtrace}[1]{\mathit{trace}_{#1}}
\newcommand{\mtraces}[1]{\mathit{traces}_{#1}}
\newcommand{\head}{\mathit{hd}}
\newcommand{\memt}{\mathit{mems}}

\newcommand{\bop}{\ \mathit{binop}\ }
\newcommand{\ak}{K}
\newcommand{\ik}{K_i}
%\newcommand{\deassign}[2]{\eassign{#1}{\mathrm{declassify}(#2)}
\newcommand{\deassign}[2]{#1 :=  [#2]_\wedge }
%\newcommand{\deassign}[2]{#1\ \wedge\!\,=  #2}

\newcommand{\mems}{\mathit{mems}}
\newcommand{\mto}{\mapsto}
\newcommand{\pdf}[1]{D_{#1}}
\newcommand{\margd}[2]{{#1}_{#2}}
\newcommand{\condd}[3]{#1_{({#2}|{#3})}}
\newcommand{\progd}{\mathrm{PD}}
\newcommand{\progtt}{\mathrm{BD}}
\newcommand{\vars}{\mathit{vars}}
\newcommand{\iov}{\mathit{iovars}}
\newcommand{\flips}{\mathit{flips}}
\newcommand{\keys}{\mathit{keys}}
\newcommand{\fedcat}{\minifed}

\newcommand{\sx}[2]{\texttt{s[#1,"#2"]}}
\newcommand{\fx}[2]{\texttt{f[#1,"#2"]}}
\newcommand{\vx}[2]{\texttt{v[#1,"#2"]}}

\newcommand{\IF}[1]{#1_{\mathit{i}}}
\newcommand{\idealf}{\mathcal{F}}
\newcommand{\SIM}{\mathrm{Sim}}
\newcommand{\prob}{\mathrm{Pr}}
\newcommand{\dist}{\mathrm{D}}

\def\TirName#1{\text{\sc #1}}

\newcommand{\srct}{\tau}
\newcommand{\cidty}[1]{\ttt{cid(}#1\ttt{)}}
\newcommand{\stringty}[1]{\ttt{string(}#1\ttt{)}}
\newcommand{\unity}{\mathtt{unit}}
\newcommand{\jpdty}[2]{\mathtt{jpd}(#1,#2)}
\newcommand{\viewst}{\mathcal{V}}
\newcommand{\tjudge}[5]{#1, #2 \vdash #3 : #4,#5}
\newcommand{\bet}[1]{\ttt{bool[}#1\ttt{]}}
\newcommand{\tas}{\mathcal{A}}


\newcommand{\flip}[2]{\ttt{flip[}#1\ttt{,}#2\ttt{]}}
\newcommand{\secret}[2]{\ttt{s[}#1\ttt{,}#2\ttt{]}}
\newcommand{\view}[2]{\ttt{v[}#1\ttt{,}#2\ttt{]}}
\newcommand{\oracle}[1]{\ttt{H[}#1\ttt{]}}
\newcommand{\Oracle}{H}
\renewcommand{\etrue}{\ttt{true}}
\renewcommand{\efalse}{\ttt{false}}
\newcommand{\enot}{\ttt{not}}
\newcommand{\eand}{\ttt{and}}
\newcommand{\eor}{\ttt{or}}
\newcommand{\exor}{\ttt{xor}}
\renewcommand{\elet}[3]{\ttt{let}\ #1\ \ttt{=}\ #2\ \ttt{in}\ #3}
\newcommand{\vloc}[2]{#1@#2}
\renewcommand{\redx}{\xrightarrow{}}
\renewcommand{\redxs}{\xrightarrow{}^{*}}
\newcommand{\lredx}[1]{\xrightarrow{#1}}
\newcommand{\mem}{M}
\newcommand{\randos}{R}
\newcommand{\tape}{\randos}
\newcommand{\secrets}{S}
\newcommand{\clients}{C}
\newcommand{\views}{V}
\newcommand{\str}{\varsigma}
\newcommand{\cid}{\iota}
\newcommand{\send}[2]{#1\ \ttt{:=}\ #2}
\newcommand{\OT}[3]{\ttt{OT(} #1 \ttt{,}\ #2 \ttt{,}\ #3 \ttt{)}}
\newcommand{\select}[3]{\ttt{select(} #1 \ttt{,}\ #2 \ttt{,}\ #3 \ttt{)}}
\newcommand{\codebase}{\mathcal{C}}
\newcommand{\interp}[1]{\llbracket #1 \rrbracket}
\newcommand{\finterp}[2]{\llbracket #1 \rrbracket_{#2}}
\newcommand{\prot}{\rho}
\newcommand{\Tapes}{\mathcal{R}}
\newcommand{\outloc}{\mathit{output}}
\newcommand{\pdist}{\mathit{pd}}
\newcommand{\genpdf}{\mathrm{PD}}
\newcommand{\card}[1]{|#1|}
\newcommand{\setdefn}[2]{\{#1\ |\ #2 \}}
\newcommand{\tapes}{\mathit{tapes}}
\newcommand{\nimo}{\mathit{NIMO}}
\newcommand{\pni}{\mathit{PNI}}
\newcommand{\passec}{PS}
\newcommand{\parties}{\mathcal{P}}
\newcommand{\iout}{\mathit{output}}
\newcommand{\kideal}{k_i}
\newcommand{\jpdf}{\mathrm{pdf}}
\newcommand{\leakproof}{\mathit{LP}}
\newcommand{\flab}{\ell}
\newcommand{\be}{\varepsilon}
\newcommand{\instr}{\mathbf{c}}
\newcommand{\solve}[2]{\mathit{models}\ #1\ #2}
\newcommand{\itv}{\mathit{it}}
\newcommand{\outv}{\mathit{out}}
\newcommand{\NIMO}{\mathit{NIMO}}
\newcommand{\gNIMO}{\mathit{gNIMO}}
\newcommand{\gates}{\mathit{gates}}
\newcommand{\owl}{\mathit{owl}}
\newcommand{\logit}[1]{\lfloor #1 \rfloor}
\newcommand{\runs}{\mathit{runs}}
\newcommand{\cruns}{\hat{\mathit{runs}}}
\newcommand{\cprogd}{\hat{\progd}}
\newcommand{\cprogtt}{\hat{\progtt}}
\newcommand{\datalog}{\mathit{datalog}}
\newcommand{\concat}{\ttt{|\!|}}
\newcommand{\wired}{\mathit{wired}}
\newcommand{\gc}[3]{\mathit{goc}(#1,#2,#3)}
\newcommand{\vc}[3]{#1 \vdash #2 \sim #3}
\newcommand{\sep}[3]{#1 \vdash #2 * #3}
\newcommand{\gtab}{\mathit{table}}
\newcommand{\vdefs}{\mathit{vdefs}}
\newcommand{\funcVar}{\$}
%\newcommand{\pmf}{\mathrm{Pr}}
\newcommand{\pmf}{\mathit{P}}

%%%% REVISION DEFS

\renewcommand{\flip}[1]{\ttt{r[}#1\ttt{]}}
\newcommand{\locflip}{\ttt{r[}\mathit{local}\ttt{]}}
\renewcommand{\secret}[1]{\ttt{s[}#1\ttt{]}}
\newcommand{\key}[1]{\ttt{k[}#1\ttt{]}}
\newcommand{\mesg}[1]{\ttt{m[}#1\ttt{]}}
\newcommand{\out}[1]{\elab{\ttt{out}}{#1}}
\newcommand{\rvl}[1]{\ttt{p[}#1\ttt{]}}
\renewcommand{\oracle}[1]{\ttt{H[}#1\ttt{]}}
\newcommand{\elab}[2]{#1_{#2}}
\renewcommand{\eassign}[4]{\elab{#1}{#2} := \elab{#3}{#4}}
\newcommand{\pubout}[3]{\out{#1} := \elab{#2}{#3}}
\newcommand{\reveal}[3]{\rvl{#1} := \elab{#2}{#3}}
\newcommand{\sk}[1]{\mathrm{sk}[#1]}
\newcommand{\pk}[2]{\mathrm{pk}[#1,#2]}
\newcommand{\kgen}[1]{\mathit{kgen}(#1)}
\newcommand{\adversary}{\mathcal{A}}
\newcommand{\aredx}{\redx_{\adversary}}
\newcommand{\aredxs}{\redxs_{\adversary}}
\newcommand{\arewrite}{\mathit{rewrite}_{\adversary}}
\newcommand{\cinputs}{V_{C \rhd H}}
\newcommand{\houtputs}{V_{H \rhd C}}
\newcommand{\aruns}{\mathit{runs}_\adversary}
\newcommand{\att}{\mathrm{AD}}
\newcommand{\support}{\mathit{support}}
\renewcommand{\store}{\sigma}
\newcommand{\ctxt}[2]{\{ #1 \}_{#2}}
\newcommand{\cpub}{\mathit{pub}}
\renewcommand{\runs}{\mathit{runs}}
\newcommand{\pattern}[1]{\lfloor #1 \rfloor}
\newcommand{\fcod}[1]{\lcod{#1}{}}
\renewcommand{\flips}{\mathit{rands}}
\newcommand{\kmat}{\kappa}
\renewcommand{\Oracle}{\mathbb{O}}
\newcommand{\afilter}{\mathit{afilter}}
\renewcommand{\select}[3]{\mathtt{if}\ #1\ \mathtt{then}\ #2\ \mathtt{else}\ #3}
\newcommand{\fp}{\mathit{P}}
\newcommand{\ftimes}{*}
\newcommand{\fplus}{+}
\newcommand{\fminus}{-}
\newcommand{\mactimes}{\,\hat{\ftimes}\,}%{\otimes}
\newcommand{\macplus}{\,\hat{\fplus}\,}%\oplus}
\newcommand{\macminus}{\,\hat{\fminus}\,}%{\ominus}
\newcommand{\macv}[1]{\langle #1 \rangle}
\newcommand{\mack}[2]{\langle #1 \rangle.\ttt{k}_{#2}}
\newcommand{\macshare}[1]{\langle #1 \rangle.\ttt{share}}
\newcommand{\macauth}{\mathrm{auth}}
\newcommand{\fieldty}{\mathrm{F}}
\newcommand{\cipherty}{\mathit{c}}
\newcommand{\macty}{\hat{\fieldty}}%_{\mathit{mac}}}}
\renewcommand{\unity}[1]{\mathit{U}(#1)}
\renewcommand{\labty}[3]{#1^{#2}_{#3}}
\newcommand{\memenv}{\mathcal{M}}
\newcommand{\tensor}{\multimap}
\newcommand{\lib}{\mathcal{L}}
\newcommand{\okt}{\mathit{OK}}
\newcommand{\vty}{t}
\newcommand{\disty}{\dot{\vty}}
\newcommand{\tlev}[1]{\mathcal{T}(#1)}
\newcommand{\otp}{\mathrm{sum}}
\newcommand{\macotp}{\hat{\otp}}

\long\def\cnote#1{{\small\textbf{\textit{\color{violet}(*#1 -- Chris*)}}}}
\long\def\jnote#1{{\small\textbf{\textit{\color{brown}(*#1 -- Joe*)}}}}



\begin{document}

\title{Automating Verification of MPC Security: $\metaprot$ to $\minifed$}

\author{Author names withheld for double-blind reviewing}

\begin{abstract}
Secure Multi-Party Computation (MPC) protocols support data privacy in
important modern distributed applications. Security for MPC protocols
is superficially similar to probabilistic noninterference, but differs
in a subtle but fundamental way, and approaches for verifying
noninterference cannot naturally extend to MPC security.  Currently,
proof methods for MPC protocols are well-studied but manual and thus
tedious and error-prone, and are also non-standardized and unfamiliar
to most PL theorists.  Our goal is to leverage connections between the
security model of MPC and trace-based hyperproperties to obtain
automated proof methods for MPC protocol development.  We develop a
language model with a tightly coupled notion of probabilistic program
distributions, as a foundation for fully and partially automated
verification of passive MPC security in protocols including
arbitrarily large YGC and GMW circuits.
\end{abstract}

%%
%% The code below is generated by the tool at http://dl.acm.org/ccs.cfm.
%% Please copy and paste the code instead of the example below.
%%
\begin{CCSXML}
<ccs2012>
   <concept>
       <concept_id>10002978.10002986.10002990</concept_id>
       <concept_desc>Security and privacy~Logic and verification</concept_desc>
       <concept_significance>500</concept_significance>
       </concept>
   <concept>
       <concept_id>10003752.10003753.10003757</concept_id>
       <concept_desc>Theory of computation~Probabilistic computation</concept_desc>
       <concept_significance>300</concept_significance>
       </concept>
   <concept>
       <concept_id>10003752.10003790.10003806</concept_id>
       <concept_desc>Theory of computation~Programming logic</concept_desc>
       <concept_significance>500</concept_significance>
       </concept>
 </ccs2012>
\end{CCSXML}

\ccsdesc[500]{Security and privacy~Logic and verification}
\ccsdesc[500]{Theory of computation~Probabilistic computation}
\ccsdesc[500]{Theory of computation~Programming logic}


%%
%% Keywords. The author(s) should pick words that accurately describe
%% the work being presented. Separate the keywords with commas.
\keywords{Secure multiparty computation, security verification, probabilistic programming, static analysis.}

%\maketitle

\section{The $\minicat$ Protocol Language}
\label{section-lang}

The $\minifed$ language establishes a basic model of synchronous
protocols between a federation of \emph{clients} exchanging values in
the binary field. A model of synchronous communication captures a wide
range of MPC protocols. Concurrency is out of scope in this work but
an avenue for future work. The lack of sophisticated control
structures in $\minifed$ is intentional, since minimizing features
eases analysis and control abstractions such as function definitions
can be integrated into a metalanguage that generates $\minifed$
programs (Section \ref{section-metalang}).

We identify clients by natural numbers and federations- finite sets of
clients- are always given statically.  Our threat model assumes a
partition of the federation into \emph{honest} $H$ and \emph{corrupt}
$C$ subsets. We model probabilistic programming via a \emph{random
tape} semantics. That is, we will assume that programs can make
reference to values chosen from a uniform random distributions defined
in the initial program memory.  Programs aka protocols execute
deterministically given the random tape.

\subsection{Syntax}

\minifedfig

The syntax of $\minifed$, defined in Figure \ref{fig-minifed},
includes values $v$ and standard operations of addition, subtraction,
and multiplication in a finite field $\mathbb{F}_p$ where $p$ is some
prime.  Protocols are given input secret values $\secret{w}$ as well
as random samples $\flip{w}$ on the input tape, implemented using a
\emph{memory} as described below (Section
\ref{section-lang-semantics}) where $w$ is a distinguishing 
identifier string. Protocols are sequences of assignment commands of three
different forms:
\begin{itemize}
\item $\eassign{\mesg{w}}{\cid_2}{\be}{\cid_1}$: This
  is a \emph{message send} where expression $\be$ is computed
  by client $\cid_1$ and sent to client $\cid_2$ as message
  $\mesg{w}$.
\item $\reveal{w}{\be}{\cid}$: This
  is a \emph{public reveal} where expression $\be$ is computed
  by client $\cid$ and broadcast to the federation, typically
  to communicate intermediate results for use in final output
  computations.
\item $\pubout{\cid}{\be}{\cid}$: This
  is an \emph{output} where expression $\be$ is computed
  by client $\cid$ and reported as its output. As a
  sanity condition we disallow commands
  $\pubout{\cid_1}{\be}{\cid_2}$ where $\cid_1\ne\cid_2$.
\end{itemize}
For example, in the following protocol, a client 1
subtracts a random sample $\flip{y}$ from $\mathbb{F}_p$ from their
secret value $\secret{x}$ and sends the result to client
2 as a message $\mesg{z}$:
$$
\eassign{\mesg{z}}{2}{(\secret{x} - \flip{y})}{1}
$$
Both messages $\mesg{w}$ and reveals $\rvl{w}$ can be
referenced in expressions once they've been defined.
This distinction between messages and broadcast public
reveal is consistent with previous formulations, e.g.,
\cite{6266151}.

We let $x$ range over \emph{variables} which are identifiers where
client ownership is specified- e.g.,
$\elab{\mesg{\mathit{foo}}}{\cid}$ is a message $\mathit{foo}$ that
was sent to $\cid$. We let $X$ range over sets of variables, and more
specifically, $S$ ranges over sets of secret variables
$\elab{\secret{w}}{\cid}$, $R$ ranges over sets of random variables
$\elab{\flip{w}}{\cid}$, $M$ ranges over sets of message variables
$\elab{\mesg{w}}{\cid}$, $P$ ranges over sets of public variables
$\rvl{w}$, and $O$ ranges over sets of output variables $\out{\cid}$.
Given a program $\prog$, we write $\iov(\prog)$ to denote the set $S
\cup M \cup P \cup O$ of variables in $\prog$ with ownership made
explicit and $\secrets(\prog)$ to denote $S$, and we write
$\flips(\prog)$ to denote the set $R$ of random samplings in $\prog$
with ownership made explicit. We write $\vars(\prog)$ to denote
$\iov(\prog) \cup \flips(\prog)$. For any set of variables $X$ and
clients $I$, we write $X_I$ to denote the subset of $X$ owned by any
client $\cid \in I$, in particular we write $X_H$ and $X_C$ to denote the
subsets belonging to honest and corrupt parties, respectively.

\subsection{Semantics}
\label{section-lang-semantics}

\emph{Memories} are fundamental to the semantics of $\fedcat$ and
provide random tape and secret inputs to protocols, and also record
message sends, public broadcast, and client outputs. Memories $\store$ are finite
(partial) mapping from variables $x$ to values $v \in \mathbb{Z}_p$. The \emph{domain} of a
memory is written $\dom(\store)$ and is the finite set of variables on
which the memory is defined. We write $\store\{ x \mapsto v\}$ for
$x\not\in\dom(\store)$ to denote the memory $\store'$ such that
$\store'(x) = v$ and otherwise $\store'(y) = \store(y)$ for all $y
\in \dom(\store)$. We write $\store \subseteq \store'$ iff
$\dom(\store) \subseteq \dom(\store')$ and $\store(x) =
\store'(x)$ for all $x \in \dom(\store)$. We write $\store \cap
\store'$ to denote the combination of $\store$ and $\store'$
assuming $\store(x) = \store'(x)$ for all $x \in \dom(\store)
\cap \dom(\store')$, otherwise $\store \cap \store'$ is undefined.
We write $\store \subseteq \store'$ iff $\store \cap \store'
= \store$.

Given a set of variables $X$, we write $\store_X$ to denote the
memory $\store$ restricted to the domain $X$, and we define
$\mems(X)$ as the set of all memories with domain $X$:
$$
\mems(X) \defeq \{ \store \mid \dom(\store) = X \}
$$
Thus, given a protocol $\prog$, the set of all random tapes for
$\prog$ is $\mems(\flips(\prog))$.
%We let $\stores$ range
%over sets of memories with the same domain, and abusing notation
%we write $\dom(\stores)$ to denote the common domain,
%and $\stores_X \defeq \{ \store_X | \store \in \stores \}$.

Given a variable-free expression $\be$, we write $\cod{\be}$ to denote
the standard interpretation of $\be$ in the arithmetic field
$\mathbb{Z}_{p}$. With the introduction of variables to expressions,
we need to interpret variables with respect to a specific memory, and
all variables used in an expression must belong to a specified client.
Thus, we denote interpretation of expressions $\be$ computed on a
client $\cid$ as $\lcod{\store,\be}{\cid}$. This interpretation is
defined in Figure \ref{fig-minifed}. The small-step reduction relation
$\redx$ is then defined in Figure \ref{fig-minifed} to evaluate
commands. Reduction is a relation on \emph{configurations} $(\store,
\prog)$ where all three command forms- message send, broadcast, and
output- are implemented as updates to the memory $\store$. We write
$\redxs$ to denote the reflexive, transitive closure of\ $\redx$.

\subsection{Example: Passive Secure Addition}
\label{section-lang-example}

Shamir addition leverages homomorphic properties of addition in
arithmetic fields to implement secret addition. If a field value $v_1$
is uniformly random, then $v_1 \fminus v_2$ is an encryption of $v_2$
where $v_1$ is an information theoretically secure one-time-pad, which
is exploited for secret sharing, noting that $v_2$ can be
reconstructed by summing $v_1$ and $v_3 \defeq v_1 \fminus v_2$. 

In $\minifed$, to privately sum secret values $\secret{\cid}$, each
client $\cid$ in the federation $\{ 1, 2, 3 \}$ samples a value
$\locflip$ that can be used as a one-time pad with another random
sample $\flip{x}$ and $\secret{\cid}$. This yields two secret shares
communicated as messages to the other clients, while each client keeps
$\locflip$ as its own share.
$$
\begin{array}{lll}
  \elab{\mesg{s1}}{2} &:=& \elab{(\secret{1} \fminus \locflip \fminus \flip{x})}{1} \\ 
  \elab{\mesg{s1}}{3} &:=& \elab{\flip{x}}{1} \\ 
  \elab{\mesg{s2}}{1} &:=& \elab{(\secret{2} \fminus \locflip \fminus \flip{x})}{2} \\ 
  \elab{\mesg{s2}}{3} &:=& \elab{\flip{x}}{2} \\ 
  \elab{\mesg{s3}}{1} &:=& \elab{(\secret{3} \fminus \locflip \fminus \flip{x})}{3} \\ 
  \elab{\mesg{s3}}{2} &:=& \elab{\flip{x}}{3}
\end{array}
$$
This scheme guarantees that messages
are viewed as random noise by any observer 
besides $\cid$ \cite{barthe2019probabilistic}. Next, each client
publicly reveals the sum of all of its shares, including its local
share. This step does reveal information about secrets-- note in
particular that $\locflip$ is reused and is no longer a one-time-pad:
$$
\begin{array}{lll}
  \rvl{1} &:=& \elab{(\locflip \fplus \mesg{s2} \fplus \mesg{s3})}{1} \\ 
  \rvl{2} &:=& \elab{(\mesg{s1} \fplus \locflip \fplus \mesg{s3})}{2} \\
  \rvl{3} &:=& \elab{(\mesg{s1} \fplus \mesg{s2} \fplus \locflip)}{3} 
\end{array}
$$
Finally, each client outputs the sum of each sum of shares, yielding
the sum of secrets. The protocol is correct because the outputs are all the
true sum of secrets, and it is secure because no more information about the
secrets other than that revealed by their sum is exposed.
$$
%\elab{\mesg{o1}}{2} &:=& \elab{(\locflip \fplus \mesg{s2} \fplus \mesg{s3})}{1} \\ 
  %\elab{\mesg{o1}}{3} &:=& \elab{(\locflip \fplus \mesg{s2} \fplus \mesg{s3})}{1} \\ 
  %\elab{\mesg{o2}}{1} &:=& \elab{(\mesg{s1} \fplus \locflip \fplus \mesg{s3})}{2} \\
  %\elab{\mesg{o2}}{3} &:=& \elab{(\mesg{s1} \fplus \locflip \fplus \mesg{s3})}{2} \\ 
  %\elab{\mesg{o3}}{1} &:=& \elab{(\mesg{s1} \fplus \mesg{s2} \fplus \locflip)}{3} \\ 
  %\elab{\mesg{o3}}{2} &:=& \elab{(\mesg{s1} \fplus \mesg{s2} \fplus \locflip)}{3}\\ 
  %\pubout{1} &:=& \elab{(\locflip \fplus \mesg{s2} \fplus \mesg{s3} + \mesg{o2} + \mesg{o3})}{1}
\begin{array}{lll}
  \out{1} &:=& \elab{(\rvl{1} \fplus \rvl{2} + \rvl{3})}{1}\\
  \out{2} &:=& \elab{(\rvl{1} \fplus \rvl{2} + \rvl{3})}{2}\\
  \out{3} &:=& \elab{(\rvl{1} \fplus \rvl{2} + \rvl{3})}{3}
\end{array}
$$


\section{Security Model}

\begin{definition}
  We write $\vc{\pmf}{x}{y}$ iff $\pmf(\{ x \mapsto 0\}\ |\ \{ y \mapsto 0 \}) =
  \pmf(\{ x \mapsto 1\}\ |\ \{ y \mapsto 1 \}) = 1$.
  We write $\sep{\pmf}{X}{Y}$ iff for all
    $\store \in \mems(X \cup Y)$ we have
  $\margd{\pmf}{X \cup Y}(\store) =
  \pmf(\store_X) * \pmf(\store_Y)$
\end{definition}

\subsection{Passive Security}

The simulator is represented by a probabilistic algorithm $\SIM_C$,
aka a \emph{simulation}, that is parameterized by corrupt inputs and
the output of an ideal functionality, and that returns a set of
adversarial views (as a memory) with some probability. Given
corrupt inputs $\store$ and ideal functionality output $v$,  
we write
$
\prob(\SIM(\store,v) = \store')
$
to denote the probability that $\SIM(\store,v)$
returns corrupt views $\store'$ as a result. We can then define the
probability distribution of corrupt views reconstructed
by the simulator as follows:
\begin{definition}
  Given $C$, $\store$, and $v$, we write $\dist(\SIM(\store,v))$ to
  denote the distribution of corrupt views reconstructed by the
  simulation, where for
  all $\store' \in \mems(V)$:
  $$
  \dist(\SIM(\store,v))(\store')\ \defeq\ \prob(\SIM(\store,v) = \store') 
  $$
\end{definition}

Then we can define passive security in the real/ideal
model as follows. 
\begin{definition}[Passive Security]
  Assume given a program $\prog$ that correctly implements an ideal
  functionality $\idealf$, with $\iov(\prog) = (S,M,O)$.  Then $\prog$
  is \emph{passive secure in the simulator model} iff for all
  partitions of the federation into honest and corrupt sets $H$ and $C$
  with $|C| < |H|$ and for all $\store \in \mems(S)$ there exists a
  simulation $\SIM$ such that:
  $$
  \dist(\SIM(\store_{S_C},\idealf(\store))) = \condd{\progtt(\prog)}{M_C}{\store}
  $$
\end{definition}

\subsection{Malicious Security}

$$
\begin{array}{rclr}
  (\store, \eassign{\mesg{w}}{\cid_1}{\be}{\cid_2};\prog) &\aredx&
  (\extend{\store}{\mesg{w}_{\cid_1}}{\lcod{\store,\be}{\cid_2}}, \prog) & \cid_2 \in H\\
  (\store, \eassign{\mesg{w}}{\cid_1}{\be}{\cid_2};\prog) &\aredx&
  (\extend{\store}{\mesg{w}_{\cid_1}}{\lcod{\arewrite(\store_C,\be)}{\cid_2}}, \prog) & \cid_2 \in C
\end{array}
$$

\begin{definition}[Corrupt Inputs, Honest Outputs]
  Given a program $\prog$ with $\iov(\prog) = (S,M,O)$ , define $\cinputs$ as the
  messages in $M$ sent from corrupt to honest parties:
  $$
  \cinputs = \{\ \elab{\mesg{w}}{\cid}\ \mid\  \elab{\mesg{w}}{\cid} \in M \wedge \eassign{\mesg{w}}{\cid}{\be}{\cid'} \in \prog
  \wedge \cid \in H \wedge \cid' \in C \ \} 
  $$
  and similarly define $\houtputs$ as the messages in $M$ sent from honest to corrupt parties.
  %Define also $(\afilter\ \prog)$ as $\prog$ with all instructions of the form $\eassign{\mesg{w}}{\cid}{\be}{\cid'}$ removed
  %where $\cid \in H \wedge \cid' \in C$.
\end{definition}

$$
\dist(\SIM(\store_{S_H})) = \condd{\progtt(\prog)}{\houtputs \cup O_H}{\store_{S_H}}
$$


\section{Security Hyperproperties}

In this Section we formulate probabilistic versions of well-studied
hyperproperties of confidentiality and integrity, including noninterference,
gradual release, declassification, and robust declassification.
We demonstrate a soundness relation between noninterference and
passive security, and between robust declassification and malicious
security. We subsequently leverage this relation to enforce
malicious security using ``traditional'' security type methods
in Section \ref{section-types}. Previous work has explored
a similar approach to security type enforcement
\cite{6266151,almeida2018enforcing} but mainly
for aspects of passive security.

\subsection{Passive Security and Noninterference}

Since MPC protocols release some information about secrets through
outputs of $\idealf$, they do not enjoy strict noninterference.  As
discussed in Section \ref{section-lang}, public reveals and protocol
outputs are fundamentally forms of declassification.  But consistent
with other work \cite{8429300}, we can formulate a version of
probabilistic noninterence conditioned on output that is sound
for passive security. It says that if two low-equivalent secret
inputs generate the same output, then the distributions of corrupt
views are the same. 
\begin{definition}[Noninterference modulo output]
  \label{definition-NIMO}
  We say that a program $\prog$ satisfies \emph{noninterference modulo output}
  iff for all $H$ and $C$ and 
  $\store_1,\store_2 \in \mems(S)$ we have:
  $$
  (\store_1 =_C \store_2 \ \wedge \ 
  (\condd{\progtt(\prog)}{O}{\store_1} = \condd{\progtt(\prog)}{O}{\store_2}))
  \implies 
  (\condd{\progtt(\prog)}{\houtputs}{\store_1} = \condd{\progtt(\prog)}{\houtputs}{\store_2})
  $$
  where $\iov(\prog) = S \cup V \cup O$.
\end{definition}
Intuitively, this conditional noninterference property implies that
the simulator can just run the protocol in simulation to
reconstruct real world corrupt view distributions. But it requires
that the simulator can tractably ``pre-image'' a given output of
a functionality $\idealf$, to determine the inputs that
could have produced it. This pre-image is called a
\emph{kernel} in recent work \cite{XXX}.
\begin{definition}
  Given a functionality $\idealf$ and output value $v$, its
  \emph{kernel}, denoted $\ik(\idealf,v)$ is
  $
  \{ \store\ |\ \idealf(\store) = v \}
  $.
  We say that $\idealf$ is \emph{pre-imageable} iff $\ik(\idealf, v)$ for all
  $v$ can be computed tractably.
\end{definition}
A soundness result for passive security can then be stated as follows.
We prove this in a separate manuscript \cite{XXX}, and it is also
essentially the same as ``perfect passive NI security'' which
has a similar soundness property \cite{8429300}.  
\begin{restatable}{theorem}{nimosecure}
  \label{theorem-nimo}
  Assume given pre-imageable $\idealf$ and a protocol $\prog$ that
  correctly implements $\idealf$.  If $\prog$ satisfies noninference modulo output
  then $\prog$ is passive secure.
\end{restatable}

\subsection{Gradual Release as a Design Pattern}

Previous work has discussed how MPC security is not noninterference,
but rather how ideal functional sets an upper bound on
declassification \cite{6266151,almeida2018enforcing}. Nevertheless,
probabilistic noninterference is preserved by components of
cryptographic protocols generally, and can be expressed using
\emph{probabilistic independence} \cite{darais2019language,barthe2019probabilistic}
We introduce important notation to express independence:
\begin{definition}
%  We write $\vc{\pmf}{x}{y}$ iff $\pmf(\{ x \mapsto 0\}\ |\ \{ y \mapsto 0 \}) =
%  \pmf(\{ x \mapsto 1\}\ |\ \{ y \mapsto 1 \}) = 1$.
  We write $\sep{\pmf}{X}{Y}$ iff for all
    $\store \in \mems(X \cup Y)$ we have
  $\margd{\pmf}{X \cup Y}(\store) =
  \pmf(\store_X) * \pmf(\store_Y)$
\end{definition}

In fact, MPC protocols typically satisfy a \emph{gradual release}
property\cite{XXX}, where messages exchanged remain probabilistically separable
from secrets, with only declassification events (reveals and outputs)
releasing information about honest secrets. 
\begin{definition}
  Given $H,C$, a protocol $\prog$ with $\iov(\prog) = S \cup M \cup P \cup O$
  satisfies \emph{gradual release} iff
  $\sep{\progtt(\prog)}{\houtputs}{S_H}$.
\end{definition}

\subsection{Malicious Security and Robust Declassification}

\begin{definition}
  We say that a protocol $\prog$ with $\iov(\prog) = (S,M)$ satisfies \emph{active confidentiality} iff the following conditions hold
  for all adversaries $\adversary$:
  \begin{enumerate}
  \item $\ \,\forall \store \in \mems(S_H) \ .\ \support(\progtt(\prog)(\{ \outv \}|\store)) =
    \support(\progtt(\prog,\adversary)(\{ \outv \}|\store))$
  \item $\begin{array}[t]{l}\forall \store_1, \store_2 \in \mems(S_H), \store \in \mems(\cinputs)\ . \\
    \quad
    \condd{\progtt(\prog,\adversary)}{\{ \outv \}}{\store_1 \cup \store} =
    \condd{\progtt(\prog,\adversary)}{\{ \outv \}}{\store_2 \cup \store} \\
    \qquad \implies\\
    \quad
    \condd{\progtt(\prog,\adversary)}{\houtputs}{\store_1 \cup \store} =
    \condd{\progtt(\prog,\adversary)}{\houtputs}{\store_2 \cup \store}\end{array}$
  \end{enumerate}
\end{definition}

\begin{theorem}
  If $\prog$ satisfies active confidentiality for all $H,C$ then it is malicious secure.
\end{theorem}

\begin{definition}
  A protocol $\prog$ with has \emph{integrity} iff 
  $\forall \adversary . \runs_\adversary(\prog) \subseteq \runs(\prog)$.
\end{definition}

\begin{definition}
  A protocol $\prog$ with $\iov(\prog) = (S,M,O)$ is \emph{malicious correct} iff:
  $$
  \forall \adversary, \store \in \mems(S_H) \ .\ \support(\progtt(\prog)(O_H|\store)) =
    \support(\progtt(\prog,\adversary)(O_H|\store))
  $$
\end{definition}

\begin{theorem}
  If a protocol has integrity it is malicious correct.
\end{theorem}

\begin{theorem}
  If a protocol is passive secure with integrity, then it satisfies active confidentiality.
\end{theorem}

\begin{theorem}
  If a protocol is passive secure with integrity, then it is malicious secure.
\end{theorem}

\begin{definition}[Robust Declassification]
  A protocol satisfies \emph{robust declassification} iff it has integrity and
  satisfies gradual release. 
\end{definition}

\begin{theorem}
  Passive security with robust declassification implies malicious security.
\end{theorem}


\begin{fpfig}[t]{Selected $\metaprot$ type judgement rules.}{fig-metaprot-tjudge}
{\small
\begin{mathpar}
\inferrule[\TirName{VarT}]
{}
{\tjudge{\viewst}{\Gamma}{x}{\Gamma(x)}{\viewst}}

\inferrule[\TirName{CidT}]
{}
{\tjudge{\viewst}{\Gamma}{\cid}{\cidty{\cid}}{\viewst}}

\inferrule[\TirName{StringT}]
{}
{\tjudge{\viewst}{\Gamma}{w}{\stringty{w}}{\viewst}}

\inferrule[\TirName{ConcatT}]
{\tjudge{\viewst}{\Gamma}{e_1}{\stringty{e_1'}}{\viewst_1}\\
\tjudge{\viewst_1}{\Gamma}{e_2}{\stringty{e_2'}}{\viewst_2}
}
{\tjudge{\viewst}{\Gamma}{e_1||e_2}{\stringty{e_1' ||e_2'}}{\viewst_2}}

\inferrule[\TirName{BoolT}]
{}
{\tjudge{\viewst}{\Gamma}{\etrue}{\bet{\cid}}{\viewst}}

\inferrule[\TirName{OracleT}]
{\tjudge{\viewst}{\Gamma}{e}{\stringty{e'}}{\viewst'}}
{\tjudge{\viewst}{\Gamma}{\oracle{e}}{\bet{\cid}}{\viewst'}}

\inferrule[\TirName{SecretT}]
{\tjudge{\viewst}{\Gamma}{e_1}{\cidty{e_1'}}{\viewst_1}\\
\tjudge{\viewst_1}{\Gamma}{e_2}{\stringty{e_2'}}{\viewst_2}}
{\tjudge{\viewst}{\Gamma}{\secret{e_1}{e_2}}{\bet{e_1'}}{\views_2}}

\inferrule[\TirName{AndT}]
{
\tjudge{\viewst}{\Gamma}{e_1}{\bet{e}}{\viewst_1}\\
\tjudge{\viewst_1}{\Gamma}{e_2}{\bet{e}}{\viewst_2}
}
{\tjudge{\viewst}{\Gamma}{e_1\ \eand\ e_2}{\bet{e}}{\viewst_2}}

\inferrule[\TirName{AssignT}]
{
\tjudge{\viewst}{\Gamma}{e_1}{\cidty{e_1'}}{\viewst_1}\\
\tjudge{\viewst_1}{\Gamma}{e_2}{\stringty{e_2'}}{\viewst_2}\\
\tjudge{\viewst_2}{\Gamma}{e_3}{\bet{e_3'}}{\viewst_3}
}
{
\tjudge{\viewst}{\Gamma}{\eassign{\view{e_1}{e_2}}{e_3}}{\unity}{(\viewst_3 ; \view{e_1'}{e_2'} )}
}

\inferrule[AppT]
{\Gamma(f) =  
 \tau_1 * \cdots * \tau_n \rightarrow \tau, \viewst_f \\ 
 \tjudge{\viewst}{\Gamma}{e_1}{\sigma(\tau_1)}{\viewst_1}
 \ \cdots\  
 \tjudge{\viewst_{n-1}}{\Gamma}{e_n}{\sigma(\tau_n)}{\viewst_n}}
{\tjudge{\viewst}{\Gamma}{f(e_1,\ldots,e_n)}{\sigma(\tau)}{(\viewst_n ; \sigma(\viewst_f))}}

\inferrule[FnT]
{
  \codebase(f) = x_1,\ldots,x_n,\ e \\ \tas(f) = \tau_1 * \cdots * \tau_n \\
  \tjudge{\varnothing}{\Gamma; x_1 : \tau_1; \ldots; x_n : \tau_n}{e}{\tau}{\viewst_f}
}
{ \Gamma \vdash f : \tau_1 * \cdots * \tau_n \rightarrow \tau, \viewst_f }

\inferrule[ProgT]
{
\forall f \in \dom(\codebase)\ .\ \Gamma \vdash f : \Gamma(f) \\ \Gamma \vdash e : \tau, \view{\cid_1}{w_1};\ldots;\view{\cid_1}{w_1}
}
{
\Gamma \vdash \codebase,e : \tau,\{\view{\cid_1}{w_1}\} \sqcup \cdots \sqcup \{ \view{\cid_n}{w_n}\}
}
\end{mathpar}
}
\end{fpfig}


%\newcommand{\sx}[2]{\elab{\secret{#1}}{#2}}
\newcommand{\mx}[2]{\elab{\mesg{#1}}{#2}} 
%\newcommand{\px}[2]{\elab{\rvl{#1}}{#2}} 
\newcommand{\rx}[2]{\elab{\flip{#1}}{#2}}
\newcommand{\ox}[2]{\elab{\out{#1}}{#2}}
\newcommand{\signals}{\leadsto}

\newcommand{\tj}[5]{#1,#2 \vdash_{#3} #4 : #5}
\newcommand{\cty}[2]{c(#1,#2)}
\newcommand{\setit}[1]{\{ #1 \}}
\newcommand{\ty}{T}
\newcommand{\eqs}{\mathit{E}}
\newcommand{\toeq}[1]{\lfloor #1 \rfloor}
\newcommand{\autheq}[1]{\phi_{\mathrm{auth}}(#1)}
\newcommand{\upgrade}[1]{\uparrow #1}

\renewcommand{\redx}{\Rightarrow}
\renewcommand{\redxs}{\redx}
\newcommand{\abort}{\bot}
\newcommand{\pre}[1]{\ttt{pre}(#1)}
\newcommand{\post}[1]{\ttt{post}(#1)}
\newcommand{\eqflag}{\mathit{sw}}
\newcommand{\eqon}{\ttt{on}}
\newcommand{\eqoff}{\ttt{off}}
\newcommand{\eqtrans}[1]{\lfloor #1 \rfloor}
\newcommand{\mc}[4]{(#1,#2,#3,#4)}
\newcommand{\cmd}{\instr}

$$
    \begin{array}{rcl@{\hspace{2mm}}r}
      \multicolumn{4}{l}{v \in \mathbb{F}_p,\ w \in \mathrm{String},\ \cid \in \mathrm{Clients} \subset  \mathbb{N} }\\[2mm] %, \bop \in \{ \eand, \eor, \exor \}} \\[2mm]
      \be &::=& \flip{w} \mid \secret{w} \mid \mesg{w} \mid \rvl{w} \mid & \textit{expressions}\\
      & & v \mid \be \fminus \be \mid \be \fplus \be \mid \be \ftimes \be \\[2mm]
      x &::=& \elab{\flip{w}}{\cid} \mid \elab{\secret{w}}{\cid} \mid \elab{\mesg{w}}{\cid} \mid \rvl{w} \mid \out{\cid} & \textit{variables} \\[2mm]
      %& &  \select{\be}{\be}{\be} \mid \ctxt{v}{k} \mid \key{w} \mid \sk{\be}(\be) \mid \pk{\be}{\be}(\be) \mid \pk{\be}{\be} \\[2mm]
      %& &  \select{\fp(\be)}{\be}{\be} \ctxt{v,\be}{k}  \mid \sk{\be}(\be) \mid \pk{\be}{\be}(\be) \mid \pk{\be}{\be} \\[2mm]
      \prog &::=& \eassign{\mesg{w}}{\cid}{\be}{\cid} \mid
      \reveal{w}{e}{\cid} \mid \pubout{\cid}{\be}{\cid} \mid \prog;\prog & \textit{protocols} 
    \end{array}
$$

\bigskip
    
 $$
  %\begin{array}{c@{\hspace{5mm}}c}
  \begin{array}{rcl}
    \lcod{\store, v}{\cid} &=& v\\
    \lcod{\store, \be_1 \fplus \be_2}{\cid} &=& \fcod{\lcod{\store, \be_1}{\cid} \fplus \lcod{\store, \be_2}{\cid}}\\ 
    \lcod{\store, \be_1 \fminus \be_2}{\cid} &=& \fcod{\lcod{\store, \be_1}{\cid} \fminus \lcod{\store, \be_2}{\cid}}\\ 
    \lcod{\store, \be_1 \ftimes \be_2}{\cid} &=& \fcod{\lcod{\store, \be_1}{\cid} \ftimes \lcod{\store, \be_2}{\cid}}\\
  %\end{array} 
  %\begin{array}{rcl}
    \lcod{\store, \flip{w}}{\cid} &=& \store(\elab{\flip{w}}{\cid})\\
    \lcod{\store, \secret{w}}{\cid} &=& \store(\elab{\secret{w}}{\cid})\\
    \lcod{\store, \mesg{w}}{\cid} &=& \store(\elab{\mesg{w}}{\cid})\\
    \lcod{\store, \rvl{w}}{\cid} &=& \store(\rvl{w})\\
    %\lcod{\store, f\,\be_1\,\cdots\, \be_n}{\cid} &=& \delta(f,\lcod{\store, \be_1}{\cid},\ldots,\lcod{\store,\be_n}{\cid})
  \end{array}
  %\end{array}
  $$

\bigskip

  \begin{mathpar}
    (\store, \xassign{x}{\be}{\cid}) \redx \extend{\store}{x}{\lcod{\store,\be}{\cid}}

    \inferrule
    {(\store_1,\be_1) \redx \store_2 \\ (\store_2,\be_2) \redx \store_3 }
    {(\store_1,\be_1;\be_2) \redx \store_3}
    %(\store, \eassign{\mesg{w}}{\cid_1}{\be}{\cid_2};\prog) \redx (\extend{\store}{\mesg{w}_{\cid_1}}{\lcod{\store,\be}{\cid_2}}, \prog)    
    %(\store, \reveal{w}{\be}{\cid};\prog) \redx (\extend{\store}{\rvl{w}}{\lcod{\store,\be}{\cid}}, \prog)   
    %(\store, \pubout{\cid}{\be}{\cid};\prog) \redx (\extend{\store}{\out{\cid}}{\lcod{\store,\be}{\cid}}, \prog)
  \end{mathpar}


$$
\begin{array}{rclr}
  (\store, \xassign{x}{\be}{\cid}) &\aredx&
  \extend{\store}{x}{\lcod{\store,\be}{\cid}} & \cid \in H\\
  (\store, \xassign{x}{\be}{\cid}) &\aredx&
  \extend{\store}{x}{\lcod{\arewrite(\store_C,\be)}{\cid}} & \cid \in C
\end{array}
$$

$$
\begin{array}{rcl@{\qquad}r}
  (\store,\elab{\assert{\be_1 = \be_2}}{\cid}) &\aredx& \store & \text{if\ }
  \lcod{\store,\be_1}{\cid} = \lcod{\store,\be_2}{\cid}  \text{\ or\ } \cid \in C\\
  (\store,\elab{\assert{\phi(\be)}}{\cid}) &\aredx& \abort & \text{if\ } \neg\phi(\store,\lcod{\store,\be}{\cid})
\end{array}
$$

\begin{mathpar}
  (\store, \xassign{x}{\be}{\cid}) \redx \extend{\store}{x}{\lcod{\store,\be}{\cid}}
  
  \inferrule
      {(\store_1,\be_1) \redx \abort}
      {(\store_1,\be_1;\be_2) \redx \abort}
\end{mathpar}

$$
\begin{array}{rcl}
  \multicolumn{3}{l}{\flab \in \mathrm{Field},\   y \in \mathrm{EVar}, \  f \in \mathrm{FName}}\\[1mm]
  %x &\in& \mathrm{EVar}\\
  %f &\in& \mathrm{FName}\\[2mm]
  e &::=& \mv \mid \flip{e} \mid \secret{e} \mid \mesg{e} \mid \rvl{e} \mid e \bop e \mid
  \elet{y}{e}{e} \mid \\
  & & f(e,\ldots,e) \mid \{ \flab = e; \ldots; \flab = e \} \mid e.\flab \\
  %  & \textit{expressions}\\
  \cmd &::=& \msend{e}{e}{e}{e} \mid \reveal{e}{e}{e} \mid \pubout{e}{e}{e} \mid
      \elab{\assert{e = e}}{e} \mid \\
  & & f(e,\ldots,e) \mid  \cmd;\cmd \mid \pre{\eqs} \mid \post{\eqs} \\[1mm]
  \bop &::=& \fplus \mid \fminus \mid \ftimes \mid \concat  \\[1mm]% \textit{operators}\\[2mm]
  \mv &::=& w \mid \cid \mid \be \mid \{ \flab = \mv;\ldots;\flab = \mv \} 
  \\ % \mid \ttt{()} \\[1mm] %& \textit{values}\\[2mm]
  \mathit{fn} &::=& f(y,\ldots,y) \{ e \} \mid  f(y,\ldots,y) \{ \cmd \} \\[1mm]%& \textit{functions}
  \phi &::=& \elab{\flip{e}}{e} \mid \elab{\secret{e}}{e} \mid \elab{\mesg{e}}{e} \mid \rvl{e} \mid \out{e} \mid \phi \fplus \phi \mid \phi \fminus \phi \mid \phi \ftimes \phi \\
  \eqs &::=& \phi = \phi \mid \eqs \wedge \eqs 
\end{array}
$$

\begin{mathpar}
  \inferrule
      {e[\mv/y] \redx \mv'}
      {\elet{y}{\mv}{e} \redx \mv'}

  \inferrule
      {\codebase(f) = y_1,\ldots,y_n,\ e \\ e_1 \redx \mv_1 \cdots e_n \redx \mv_n \\
        e[\mv_1/y_1]\cdots[\mv_n/y_n] \redx \mv}
      {f(e_1,\ldots,e_n) \redx \mv}

  \inferrule
      {e_1 \redx \mv_1 \cdots e_n \redx \mv_n }
      {\{ \flab_1 = e_1; \ldots; \flab_n = e_n \} \redx \{ \flab_1 = \mv_1; \ldots; \flab_n = \mv_n \} }

  \inferrule
      {e \redx \{\ldots; \flab = \mv; \ldots\}}
      {e.\flab \redx \mv}

  \inferrule
      {e_1 \redx w_1 \\ e_2 \redx w_2}
      {e_1 \concat e_2 \redx w_1w_2}
\end{mathpar}

\begin{mathpar}
  \inferrule
      {e_1 \redx \be_1 \\ e_2 \redx \be_2 \\ e \redx \cid}
      {\mc{\prog}{(\eqs_1,\eqs_2)}{\eqon}{\elab{\assert{e_1 = e_2}}{e}} \redx
        (\prog,(\eqs_1, \eqs_2 \wedge \eqtrans{\elab{\be_1}{\cid}} = \eqtrans{\elab{\be_2}{\cid}},\eqon)}
\end{mathpar}


%\bibliographystyle{ACM-Reference-Format}
%\bibliography{logic-bibliography,secure-computation-bibliography}

\end{document}
\endinput
