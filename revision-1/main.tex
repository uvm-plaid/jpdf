\documentclass[acmsmall,screen,review]{acmart}

\usepackage{amsmath}
\usepackage{amstext}
\usepackage{fp-frame}
\usepackage[latin1]{inputenc}
\usepackage{mathpartir}
\usepackage{fancyvrb}
\usepackage{moreverb}
\usepackage{stmaryrd}
\usepackage{enumerate}
\usepackage{thmtools,thm-restate}
\usepackage{comment}
\usepackage{booktabs,array}

\newcommand{\evals}{\Downarrow}
\newcommand{\diverges}{\Uparrow}
\newcommand{\intt}{\mathrm{int}}
\newcommand{\unitt}{\mathrm{unit}}
\newcommand{\boolt}{\mathrm{bool}}
\newcommand{\floatt}{\mathrm{float}}
\newcommand{\stringt}{\mathrm{string}}
\newcommand{\chart}{\mathrm{char}}
\newcommand{\vb}[1]{\verb+#1+}
\newcommand{\evalexmp}[2]{\texttt{#1}\ \ensuremath{\evals}\ \texttt{#2}}
\newcommand{\texmp}[2]{\texttt{#1\ :\ #2}}
\newcommand{\skipper}{\bigskip\\}
\newcommand{\fyi}{\noindent\textbf{\textit{fyi:}}\ }
\newcommand{\NB}{\noindent\textbf{NB:\ }}
\newcommand{\const}{\ensuremath{\mathbf{c}}}
\newcommand{\defn}{\heading{definition}}
\newcommand{\defeq}{\triangleq}
\newcommand{\nat}{\mathbb{N}}
\newcommand{\atom}{\texttt{const}}
\def\squareforqed{\hbox{\rlap{$\sqcap$}$\sqcup$}}
\def\qed{\ifmmode\squareforqed\else{\unskip\nobreak\hfil
\penalty50\hskip1em\null\nobreak\hfil\squareforqed
\parfillskip=0pt\finalhyphendemerits=0\endgraf}\fi}
\newcommand{\exampletab}[1]{\skipper\begin{tabular}{lll}#1\end{tabular}\skipper}
\newcommand{\verbtab}[1]{\skipper\begin{verbatimtab}{#1}\end{verbatimtab}\skipper}
\newcommand{\eqntab}[1]{\skipper\begin{tabular}{rcl}#1\end{tabular}\skipper}
\newcommand{\recdefn}[1]{\{#1\}}
\newcommand{\ttt}[1]{\texttt{#1}}
\newcommand{\gdesc}[1]{\text{\textit{#1}}}
\newcommand{\true}{\mathrm{true}}
\newcommand{\false}{\mathrm{false}}
\newcommand{\etrue}{\texttt{true}}
\newcommand{\efalse}{\texttt{false}}
\newcommand{\reval}{\Rightarrow}
\newcommand{\Dand}{\ \mathrm{and}\ }
\newcommand{\Dor}{\ \mathrm{or}\ }
\newcommand{\Dxor}{\ \mathrm{xor}\ }
\newcommand{\Dnot}{\mathrm{not}\ }
\newcommand{\cod}[1]{\llbracket #1 \rrbracket}
\newcommand{\lcod}[2]{\llbracket #1 \rrbracket_{#2}}
\newcommand{\Dplus}{\mathrm{Plus}}
\newcommand{\Dminus}{\mathrm{Minus}}
\newcommand{\Dequal}{\mathrm{=}}
\newcommand{\Dabs}[2]{(\mathrm{Function}\ #1 \rightarrow #2)}
\newcommand{\Dfix}[3]{(\mathrm{Fix}\ #1 . #2 \rightarrow #3)}
\newcommand{\Dite}[3]{\mathrm{If}\ #1\ \mathrm{Then}\ #2\ \mathrm{Else}\ #3}
\newcommand{\dotminus}{\stackrel{.}{-}}
\newcommand{\Dlet}[3]{\mathrm{Let}\ #1 = #2\ \mathrm{In}\ #3}
\newcommand{\Dletrec}[4]{\mathrm{Let\ Rec}\ #1\ #2 = #3\ \mathrm{In}\ #4}
\newcommand{\Dfst}{\mathrm{left}}
\newcommand{\Dsnd}{\mathrm{right}}
\newcommand{\labset}{\mathit{Lab}}
\newcommand{\Drec}[1]{\{ #1 \}}
\newcommand{\linfer}[3]{\inferrule*[right=(\TirName{#1})]{#2}{#3}}
\newcommand{\lab}[1]{\mathrm{#1}}
\newcommand{\loc}{\ell}
\newcommand{\Dref}[1]{\mathrm{Ref}\,#1}
%\newcommand{\store}{\mathcal{M}}
\newcommand{\store}{m}
%\newcommand{\stores}{\overline{\store}}
\newcommand{\stores}{\Sigma}
\newcommand{\config}[2]{( #1,#2 )}
\newcommand{\configf}[2]{\begin{array}[t]{l}\langle #1\\,\\ #2 \rangle \end{array}}
\newcommand{\extend}[3]{#1\{#2 \mapsto #3\}}
\newcommand{\emptystore}{\{\}}
\newcommand{\storedefn}[1]{\{#1\}}
\newcommand{\Dret}[1]{\mathrm{Return}\,#1}
\newcommand{\Draise}[1]{\mathrm{Raise}\,#1}
\newcommand{\Dexn}[2]{\#\!#1\,#2}
\newcommand{\Dtry}[3]{\mathrm{Try}\,#1\,\mathrm{With}\,#2 \rightarrow #3}
\newcommand{\xname}{\mathit{exn}}
\newcommand{\Dboolt}{\mathrm{Bool}}
\newcommand{\Dreft}[1]{#1\,\mathrm{ref}}
\newcommand{\reft}[1]{#1\,\mathrm{ref}}
\newcommand{\Dintt}{\mathrm{Int}}
%\newcommand{\tjudge}[3]{#1 \vdash #2 : #3}
\newcommand{\textend}[3]{#1;#2:#3}
\newcommand{\fnty}[2]{#1 \rightarrow #2}
\newcommand{\TDabs}[3]{(\mathrm{Function}\ (#1 : #2) \rightarrow #3)}
\newcommand{\TDfix}[5]{(\mathrm{Fix}\ #1 . (#2 : #3) : #4 \rightarrow #5)}
\newcommand{\TDletrec}[6]{\Dletrec{#1}{#2 : #3}{#4 : #5}{#6}}
\newcommand{\emptyenv}{\varnothing}
\newcommand{\tfail}{\mathbf{fail}}
\newcommand{\tcheck}{\mathrm{TC}}
\newcommand{\tcheckfail}{\mathbf{TypeMismatch}}
\newcommand{\algtab}[1]
{
\vspace*{-3mm}
\begin{tabbing}
\hspace*{12mm}\=\hspace{9mm}\=\hspace{9mm}\=\hspace{6mm}\=\hspace{6mm}\=
\hspace{6mm}\=
#1
\end{tabbing}
}
\newcommand{\assign}[2]{#1 := #2}
\newcommand{\ederef}[1]{\,!#1}
\newcommand{\declass}[2]{\mathrm{declassify}_{#2}(#1)}
\newcommand{\eendorse}[2]{\mathrm{endorse}_{#2}(#1)}
\newcommand{\lt}{\left\{}
\newcommand{\rt}{\right\}}
\newcommand{\Lt}{\left\{\!\!\right.}
\newcommand{\Rt}{\left.\!\!\right\}}
\newcommand{\tinfer}{\mathit{PT}}
\newcommand{\unify}{\mathit{unify}}
\newcommand{\tsubn}{\varphi}
\newcommand{\scheme}[2]{\forall #1 . #2}
\newcommand{\Dself}{\mathrm{this}}
\newcommand{\Dsuper}{\mathrm{super}}
\newcommand{\Dsend}[3]{#1.#2(#3)}
\newcommand{\Dselect}[2]{#1.#2}
\newcommand{\Demptyclass}{\mathrm{EmptyClass}}
%\newcommand{\Dclass}[3]{\mathrm{Class}\ \mathrm{Extends}\ #1\ \mathrm{Inst}
%\ #2\ \mathrm{Meth}\ #3}
\newcommand{\Dclass}[2]{\mathrm{Class}\ \mathrm{Inst} \ #1\ \mathrm{Meth}\ #2}
\newcommand{\Dobj}[2]{\mathrm{Object}\ \mathrm{Inst}\ #1\ \mathrm{Meth}\ #2}
%\newcommand{\Dclassf}[3]{
%\begin{array}[t]{l}
%\mathrm{Class}\ \mathrm{Extends}\ #1 \\
%\quad \mathrm{Inst}\\
%\qquad #2 \\ 
%\quad \mathrm{Meth}\\
%\qquad #3
%\end{array}
%}
\newcommand{\Dclassf}[3]{
\begin{array}[t]{l}
\mathrm{Class}\\
\quad \mathrm{Inst}\\
\qquad #1 \\ 
\quad \mathrm{Meth}\\
\qquad #2
\end{array}
}
\newcommand{\Dobjf}[2]{
\begin{array}[t]{l}
\mathrm{Object}\\
\quad \mathrm{Inst}\\
\qquad #1 \\ 
\quad \mathrm{Meth}\\
\qquad #2
\end{array}
}
\newcommand{\Dnew}[1]{\mathrm{New}\ #1}
\newcommand{\vtab}[1]{\begin{verbatimtab}[4]#1\end{verbatimtab}}

\newcounter{topiccounter}
\setcounter{topiccounter}{1}
\newcommand{\topic}[1]
    {\noindent \textbf{Topic \arabic{topiccounter}.\ \textit{#1}. } \stepcounter{topiccounter}}


\newcommand{\lcalc}{$\lambda$-calculus}
\newcommand{\redx}{\rightarrow}
\newcommand{\redxs}{\redx^*}
\newcommand{\idfn}{\mathit{ID}}
\newcommand{\mlfn}[2]{\mathrm{fun}\, #1 \rightarrow #2}
\newcommand{\mlrecfn}[3]{\mathrm{fix}\,#1.#2 \rightarrow #3}
\newcommand{\mlfix}{\mathrm{fix}}
\newcommand{\eite}[3]{\mathrm{if}\ #1\ \mathrm{then} \ #2\ \mathrm{else} \ #3\ }
\newcommand{\esucc}[1]{\texttt{succ}\ #1}
\newcommand{\epred}[1]{\texttt{pred}\ #1}
\newcommand{\eiszero}[1]{\texttt{iszero}\ #1}
\newcommand{\ezero}{\texttt{0}}
\newcommand{\elet}[3]{\mathrm{let}\ #1 = #2\ \mathrm{in}\ #3}
\newcommand{\eletrec}[3]{\mathrm{letrec}\ #1 = #2\ \mathrm{in}\ #3}
\newcommand{\fv}{\mathrm{fv}}
\newcommand{\ourml}{\mathit{ML}_{\mathit{Cat}}}
\newcommand{\raisexn}{\mathrm{raise}}
\newcommand{\handler}[3]{\mathrm{try}\, #1\, \mathrm{with}\, \exn(#2) \Rightarrow #3}
\newcommand{\exn}{\mathit{exn}}
\newcommand{\dom}{\mathrm{dom}}
\newcommand{\efst}{\mathrm{fst}}
\newcommand{\esnd}{\mathrm{snd}}
\newcommand{\natt}{\textrm{Nat}}
\newcommand{\earray}{\mathrm{array}}
\newcommand{\varray}{\alpha}
\newcommand{\length}{\mathit{length}}
\newcommand{\arrayml}{\ourml^{\earray}}
\newcommand{\stackml}{\ourml^{\mathit{stack}}}
\newcommand{\flowml}{\ourml^{\mathit{flow}}}
\newcommand{\taintml}{\ourml^{\mathit{taint}}}
\newcommand{\secfail}{\mathbf{secfail}}
\newcommand{\tr}{\theta}
\newcommand{\rewrite}[1]{\mathcal{R}(#1)}
%\newcommand{\secprop}{\mathcal{P}}
%\newcommand{\trprop}{\hat{\secprop}}
\newcommand{\Prop}{\mathbf{P}}
\newcommand{\Hprop}{\mathbf{H}}
\newcommand{\secprop}{\phi}
\newcommand{\hyprop}{\eta}
\newcommand{\trprop}{\gamma}
\newcommand{\tracess}{\Sigma}
\newcommand{\trsprop}{\sigma}
\newcommand{\traces}{\Psi}
\newcommand{\fpkeyword}[1]{\mathrm{#1}}
\newcommand{\ebinop}[2]{#1\,\mathit{binop}\,#2}
\newcommand{\eenablepriv}[2]{\fpkeyword{enable}\ #1\ \fpkeyword{for}\ #2}
\newcommand{\echeckpriv}[2]{\fpkeyword{check}\ #1\ \fpkeyword{then}\ #2}
\newcommand{\esigned}[2]{#1.#2}
\newcommand{\enabled}{\mathit{enabledprivs}}
%\newcommand{\acl}{\mathcal{A}}
\newcommand{\priv}{\pi}
\newcommand{\privs}{\mathit{R}}
\newcommand{\prin}{p}
\newcommand{\nobody}{\mathit{nobody}}
\newcommand{\po}{\preceq}
\newcommand{\seclattice}{\mathcal{S}}
\newcommand{\binsl}{\mathcal{S}_{\mathrm{bin}}}
\newcommand{\seclevs}{\mathcal{L}}
\newcommand{\latel}{\varsigma}
\newcommand{\hilab}{\mathrm{High}}
\newcommand{\lolab}{\mathrm{Low}}
\newcommand{\hiloc}{\mathit{hi}}
\newcommand{\loloc}{\mathit{low}}
\newcommand{\labty}[2]{#1 \cdot #2}
\newcommand{\labval}[2]{#1 \cdot #2}
\newcommand{\mi}[1]{\mathit{#1}}
\newcommand{\pc}{\latel_{\mathit{pc}}}
\newcommand{\cfnty}[3]{#1 \rightarrow_{#2} #3}
\newcommand{\pow}{\mathrm{pow}}

%\newcommand{\tr}{\theta}


\newcommand{\chash}{\mathcal{H}}
\newcommand{\acl}{\mathit{Auth}}
\newcommand{\opn}{\mathit{op}}
\newcommand{\egid}{\mathit{egid}}
\newcommand{\euid}{\mathit{euid}}
\newcommand{\suid}{\ttt{suid}}
\newcommand{\sgid}{\ttt{sgid}}
\newcommand{\uxroot}{\ttt{root}}
\newcommand{\fowner}[1]{\mathit{owner}_{#1}}
\newcommand{\fgroup}[1]{\mathit{group}_{#1}}
\newcommand{\gprivs}[1]{\mathit{Privs_{#1}}.\mathit{group}}
\newcommand{\uprivs}[1]{\mathit{Privs_{#1}}.\mathit{owner}}
\newcommand{\oprivs}[1]{\mathit{Privs_{#1}}.\mathit{other}}
\newcommand{\uxprivs}[1]{\mathit{Privs_{#1}}}

\newcommand{\seclab}{\mathcal{L}}
\newcommand{\sle}{\preceq}
\newcommand{\ile}{\preceq_I}
\newcommand{\ilab}{\seclab_I}

\newcommand{\minifed}{\mathit{Overture}}
\newcommand{\minicat}{\minifed}
\newcommand{\fedprot}{\minifed}
\newcommand{\metaprot}{\mathit{Prelude}}
\newcommand{\mlscat}{\mathit{mlscat}}
\newcommand{\flowcat}{\mathit{flowcat}}
\newcommand{\dflowcat}{\mathit{dflowcat}}
\newcommand{\minicatde}{\mathit{minicat}_{\mathit{de}}}
\newcommand{\minicatexp}{\mathit{minicat}_{\mathit{taint}}}
%\newcommand{\prog}{\mathcal{P}}
\newcommand{\prog}{\pi}
\newcommand{\main}{\mathit{main}}
\renewcommand{\reval}{\redx}
\renewcommand{\Dite}{\eite}


%\renewcommand{\labty}[2]{#2}
\newcommand{\fnsty}{\Sigma}
\newcommand{\secty}{\latel}

\newcommand{\tc}{\mathrm{TC}}
\newcommand{\validate}{\mathrm{validate}}


\newcommand{\mlsid}[1]{\mathrm{mls}(#1)}
\newcommand{\mlsredx}[1]{\redx_{\mlsid{#1}}}
\newcommand{\confid}{\mathit{flow}}
\newcommand{\taintid}{\mathit{dflow}}
\newcommand{\credx}{\redx_{\confid}}
\newcommand{\tredx}{\redx_{\taintid}}
\newcommand{\ccod}[1]{\lcod{\confid}{#1}}
\newcommand{\tcod}[1]{\lcod{\taintid}{#1}}
\renewcommand{\mod}{\ \textrm{mod}\ }


\newcommand{\mtrace}[1]{\mathit{trace}_{#1}}
\newcommand{\mtraces}[1]{\mathit{traces}_{#1}}
\newcommand{\head}{\mathit{hd}}
\newcommand{\memt}{\mathit{mems}}

\newcommand{\bop}{\ \mathit{binop}\ }
\newcommand{\ak}{K}
\newcommand{\ik}{\mathit{kernel}}
%\newcommand{\deassign}[2]{\eassign{#1}{\mathrm{declassify}(#2)}
\newcommand{\deassign}[2]{#1 :=  [#2]_\wedge }
%\newcommand{\deassign}[2]{#1\ \wedge\!\,=  #2}

\newcommand{\mems}{\mathit{mems}}
\newcommand{\mto}{\mapsto}
\newcommand{\pdf}[1]{D_{#1}}
\newcommand{\margd}[2]{{#1}_{#2}}
\newcommand{\condd}[3]{#1_{({#2}|{#3})}}
\newcommand{\progd}{\mathrm{PD}}
\newcommand{\progtt}{\mathrm{BD}}
\newcommand{\vars}{\mathit{vars}}
\newcommand{\iov}{\mathit{iovars}}
\newcommand{\flips}{\mathit{flips}}
\newcommand{\keys}{\mathit{keys}}
\newcommand{\fedcat}{\minifed}

%\newcommand{\sx}[2]{\texttt{s[#1,"#2"]}}
%\newcommand{\fx}[2]{\texttt{f[#1,"#2"]}}
%\newcommand{\vx}[2]{\texttt{v[#1,"#2"]}}

\newcommand{\IF}[1]{#1_{\mathit{i}}}
\newcommand{\idealf}{\mathcal{F}}
\newcommand{\SIM}{\mathrm{Sim}}
\newcommand{\prob}{\mathrm{Pr}}
\newcommand{\dist}{\mathrm{D}}

\def\TirName#1{\text{\sc #1}}

\newcommand{\srct}{\tau}
\newcommand{\cidty}[1]{\ttt{cid(}#1\ttt{)}}
\newcommand{\stringty}[1]{\ttt{string(}#1\ttt{)}}
\newcommand{\unity}{\mathtt{unit}}
\newcommand{\jpdty}[2]{\mathtt{jpd}(#1,#2)}
\newcommand{\viewst}{\mathcal{V}}
\newcommand{\tjudge}[5]{#1, #2 \vdash #3 : #4,#5}
\newcommand{\bet}[1]{\ttt{bool[}#1\ttt{]}}
\newcommand{\tas}{\mathcal{A}}


\newcommand{\flip}[2]{\ttt{flip[}#1\ttt{,}#2\ttt{]}}
\newcommand{\secret}[2]{\ttt{s[}#1\ttt{,}#2\ttt{]}}
\newcommand{\view}[2]{\ttt{v[}#1\ttt{,}#2\ttt{]}}
\newcommand{\oracle}[1]{\ttt{H[}#1\ttt{]}}
\newcommand{\Oracle}{H}
\renewcommand{\etrue}{\ttt{true}}
\renewcommand{\efalse}{\ttt{false}}
\newcommand{\enot}{\ttt{not}}
\newcommand{\eand}{\ttt{and}}
\newcommand{\eor}{\ttt{or}}
\newcommand{\exor}{\ttt{xor}}
\renewcommand{\elet}[3]{\ttt{let}\ #1\ \ttt{=}\ #2\ \ttt{in}\ #3}
\newcommand{\vloc}[2]{#1@#2}
\renewcommand{\redx}{\xrightarrow{}}
\renewcommand{\redxs}{\xrightarrow{}^{*}}
\newcommand{\lredx}[1]{\xrightarrow{#1}}
\newcommand{\mem}{M}
\newcommand{\randos}{R}
\newcommand{\tape}{\randos}
\newcommand{\secrets}{\mathit{secrets}}
\newcommand{\outputs}{\mathit{outputs}}
\newcommand{\clients}{C}
\newcommand{\views}{\mathit{views}}
\newcommand{\str}{\varsigma}
\newcommand{\cid}{\iota}
\newcommand{\send}[2]{#1\ \ttt{:=}\ #2}
\newcommand{\msend}[4]{\elab{\mesg{#1}}{#2}\ \ttt{:=}\ \elab{#3}{#4}}
\newcommand{\OT}[3]{\ttt{OT(} #1 \ttt{,}\ #2 \ttt{,}\ #3 \ttt{)}}
\newcommand{\select}[3]{\ttt{select(} #1 \ttt{,}\ #2 \ttt{,}\ #3 \ttt{)}}
\newcommand{\mux}[3]{\ttt{mux(} #1 \ttt{,}\ #2 \ttt{,}\ #3 \ttt{)}}
\newcommand{\codebase}{\mathcal{C}}
\newcommand{\interp}[1]{\llbracket #1 \rrbracket}
\newcommand{\finterp}[2]{\llbracket #1 \rrbracket_{#2}}
\newcommand{\prot}{\rho}
\newcommand{\Tapes}{\mathcal{R}}
\newcommand{\outloc}{\mathit{output}}
\newcommand{\pdist}{\mathit{pd}}
\newcommand{\genpdf}{\mathrm{PD}}
\newcommand{\card}[1]{|#1|}
\newcommand{\setdefn}[2]{\{#1\ |\ #2 \}}
\newcommand{\tapes}{\mathit{tapes}}
\newcommand{\nimo}{\mathit{NIMO}}
\newcommand{\pni}{\mathit{PNI}}
\newcommand{\passec}{PS}
\newcommand{\parties}{\mathcal{P}}
\newcommand{\iout}{\mathit{output}}
\newcommand{\kideal}{k_i}
\newcommand{\jpdf}{\mathrm{pdf}}
\newcommand{\leakproof}{\mathit{LP}}
\newcommand{\flab}{\ell}
\newcommand{\be}{\varepsilon}
\newcommand{\instr}{\mathbf{c}}
\newcommand{\solvealg}{\mathit{models}}
\newcommand{\solve}[3]{\solvealg\ #1\ #2\ #3}
\newcommand{\itv}{\mathit{it}}
\newcommand{\outv}{\mathit{out}}
\newcommand{\NIMO}{\mathit{NIMO}}
\newcommand{\gNIMO}{\mathit{gNIMO}}
\newcommand{\gates}{\mathit{gates}}
\newcommand{\owl}{\mathit{owl}}
\newcommand{\logit}[1]{\lfloor #1 \rfloor}
\newcommand{\runs}{\mathit{runs}}
\newcommand{\cruns}{\hat{\mathit{runs}}}
\newcommand{\cprogd}{\hat{\progd}}
\newcommand{\cprogtt}{\hat{\progtt}}
\newcommand{\datalog}{\mathit{datalog}}
%\newcommand{\concat}{\ttt{|\!|}}
\newcommand{\concat}{\ttt{++}}
\newcommand{\wired}{\mathit{wired}}
\newcommand{\gc}[3]{\mathit{goc}(#1,#2,#3)}
\newcommand{\vc}[3]{#1 \vdash #2 \sim #3}
\newcommand{\detx}[1]{\mathbf{D}(#1)}
\newcommand{\unix}[1]{\mathbf{U}(#1)}
\newcommand{\sep}[3]{#1 \vdash #2 * #3}
\newcommand{\condp}[3]{#1|#2 \vdash #3}
\newcommand{\conddetx}[3]{\condp{#1}{#2}{\detx{#3}}}
\newcommand{\condsep}[4]{\condp{#1}{#2}{#3 * #4}}
\newcommand{\condunix}[3]{\condp{#1}{#2}{\unix{#3}}}
\newcommand{\gtab}{\mathit{table}}
\newcommand{\vdefs}{\mathit{vdefs}}
\newcommand{\funcVar}{\$}
%\newcommand{\pmf}{\mathrm{Pr}}
\newcommand{\pmf}{\mathit{P}}

%%%% REVISION DEFS

\renewcommand{\flip}[1]{\ttt{r[}#1\ttt{]}}
\newcommand{\locflip}{\ttt{r[}\mathtt{local}\ttt{]}}
\renewcommand{\secret}[1]{\ttt{s[}#1\ttt{]}}
\newcommand{\key}[1]{\ttt{k[}#1\ttt{]}}
\newcommand{\mesg}[1]{\ttt{m[}#1\ttt{]}}
\newcommand{\outkw}{\ttt{out}}
\newcommand{\out}[1]{\elab{\outkw}{#1}}
\newcommand{\rvl}[1]{\ttt{p[}#1\ttt{]}}
\renewcommand{\oracle}[1]{\ttt{H[}#1\ttt{]}}
%\newcommand{\elab}[2]{#1_{#2}}
\newcommand{\elab}[2]{#1\ttt{@}#2}
\newcommand{\eassign}[4]{\elab{#1}{#2} := \elab{#3}{#4}}
\newcommand{\xassign}[3]{#1 := \elab{#2}{#3}}
\newcommand{\pubout}[3]{\out{#1} := \elab{#2}{#3}}
\newcommand{\reveal}[3]{\rvl{#1} := \elab{#2}{#3}}
\newcommand{\sk}[1]{\mathrm{sk}[#1]}
\newcommand{\pk}[2]{\mathrm{pk}[#1,#2]}
\newcommand{\kgen}[1]{\mathit{kgen}(#1)}
\newcommand{\adversary}{\mathcal{A}}
\newcommand{\aredx}{\redx_{\adversary}}
\newcommand{\aredxs}{\redxs_{\adversary}}
\newcommand{\arewrite}{\mathit{rewrite}_{\adversary}}
\newcommand{\cinputs}{V_{C \rhd H}}
\newcommand{\houtputs}{V_{H \rhd C}}
\newcommand{\aruns}{\mathit{runs}_\adversary}
\newcommand{\botruns}{\mathit{runs}_{\adversary,\bot}}
\newcommand{\att}{\mathrm{AD}}
\newcommand{\support}{\mathit{support}}
\renewcommand{\store}{\sigma}
\newcommand{\ctxt}[2]{\{ #1 \}_{#2}}
\newcommand{\cpub}{\mathit{pub}}
\renewcommand{\runs}{\mathit{runs}}
\newcommand{\pattern}[1]{\lfloor #1 \rfloor}
\newcommand{\fcod}[1]{\lcod{#1}{}}
\renewcommand{\flips}{\mathit{rands}}
\newcommand{\kmat}{\kappa}
\renewcommand{\Oracle}{\mathbb{O}}
\newcommand{\afilter}{\mathit{afilter}}
%\renewcommand{\select}[3]{\mathtt{if}\ #1\ \mathtt{then}\ #2\ \mathtt{else}\ #3}
\newcommand{\fp}{\mathit{P}}
\newcommand{\ftimes}{*}
\newcommand{\fplus}{+}
\newcommand{\fminus}{-}
\newcommand{\mactimes}{\,\hat{\ftimes}\,}%{\otimes}
\newcommand{\macplus}{\,\hat{\fplus}\,}%\oplus}
\newcommand{\macminus}{\,\hat{\fminus}\,}%{\ominus}
\newcommand{\macgv}[1]{\langle #1 \rangle}
\newcommand{\macv}{\hat{v}}
\newcommand{\macx}[2]{\macgv{\elab{ #1 }{#2}}}
\newcommand{\mack}[2]{#1.\ttt{k}_{#2}}
\newcommand{\macshare}[1]{\langle #1 \rangle.\ttt{share}}
\newcommand{\macopen}{\mathrm{open}}
\newcommand{\macauth}{\mathrm{auth}}
\newcommand{\fieldty}{\mathrm{F}}
\newcommand{\cipherty}{\mathit{c}}
\newcommand{\macty}{\hat{\fieldty}}%_{\mathit{mac}}}}
\renewcommand{\unity}[1]{\mathit{U}(#1)}
\renewcommand{\labty}[3]{#1^{#2}_{#3}}
\newcommand{\memenv}{\mathcal{M}}
\newcommand{\tensor}{\multimap}
\newcommand{\lib}{\mathcal{L}}
\newcommand{\okt}{\mathit{OK}}
\newcommand{\vty}{t}
\newcommand{\disty}{\dot{\vty}}
\newcommand{\tlev}[1]{\mathcal{T}(#1)}
\newcommand{\otp}{\mathrm{sum}}
\newcommand{\macotp}{\hat{\mathrm{minus}}}
\newcommand{\preproc}{\mathit{preproc}}
\newcommand{\assert}[1]{\ttt{assert(}#1\ttt{)}}
\newcommand{\mv}{\nu}
\newcommand{\andgmw}{\ttt{andgmw}}
\newcommand{\decodegmw}{\ttt{decodegmw}}
\newcommand{\bodies}{\mathit{bodies}}

\newcommand{\sx}[2]{\elab{\secret{#1}}{#2}}
\newcommand{\mx}[2]{\elab{\mesg{#1}}{#2}} 
\newcommand{\px}[1]{\rvl{#1}} 
\newcommand{\rx}[2]{\elab{\flip{#1}}{#2}}
\newcommand{\ox}[2]{\elab{\out{#1}}{#2}}
\newcommand{\eqcast}[2]{#1\ \ttt{as}\ #2}
\newcommand{\signals}[1]{\stackrel{#1}{\leadsto}}

\newcommand{\tj}[6]{#1,#2,#3 \vdash_{#4} #5 : #6}
\newcommand{\itj}[3]{\vdash_{#1} #2 : #3}
\newcommand{\ipj}[3]{#1 \vdash #2 : #3}
\newcommand{\ej}[6]{#1,#2,#3 \vdash #4 : #5,#6}
\newcommand{\cty}[2]{c(#1,#2)}
\newcommand{\setit}[1]{\{ #1 \}}
\newcommand{\ty}{T}
\newcommand{\ity}[2]{#1 \cdot #2}
\newcommand{\lty}[2]{#1 \cdot #2}
\newcommand{\eqs}{\mathit{E}}
\newcommand{\toeq}[1]{\lfloor #1 \rfloor}
\newcommand{\eop}{\equiv}
\newcommand{\autheq}[1]{\phi_{\mathrm{auth}}(#1)}
\newcommand{\upgrade}[1]{\uparrow #1}
\newcommand{\seclev}{\mathcal{L}}

\newcommand{\tsig}{\mathrm{sig}}
\newcommand{\subn}{\rho}
\newcommand{\hty}[5]{\{ #1 \}\ #2,#3 \cdot #4\  \{ #5 \} }
\newcommand{\dht}[6]{\Pi #1 . \hty{#2}{#3}{#4}{#5}{#6}}
\newcommand{\mtj}[6]{\vdash #1 : \hty{#2}{#3}{#4}{#5}{#6}}
\newcommand{\atj}[3]{\Vdash #1 : (#2,#3)}
\newcommand{\eqj}[4]{#1,#2 \vdash #3 : #4}
\newcommand{\cpj}[4]{#1,#2 \vdash #3 : #4}
\newcommand{\leakj}[3]{#1,#2 \vdash_{\mathit{leak}} #3}
\newcommand{\cheatj}[3]{#1 \underset{#2}{\leadsto} #3}
\newcommand{\icod}[2]{\cod{#1}_{#2}}
\newcommand{\prejoin}{\sqcap}

\renewcommand{\redx}{\Rightarrow}
\renewcommand{\redxs}{\redx}
\newcommand{\abort}{\bot}
\newcommand{\pre}[1]{\ttt{pre}(#1)}
\newcommand{\post}[1]{\ttt{post}(#1)}
\newcommand{\eqflag}{\mathit{sw}}
\newcommand{\eqon}{\ttt{on}}
\newcommand{\eqoff}{\ttt{off}}
\newcommand{\eqtrans}[1]{\lfloor #1 \rfloor}
\newcommand{\mc}[4]{(#1,#2,#3,#4)}
\newcommand{\cmd}{\instr}
\renewcommand{\OT}[4]{\ttt{OT(}\elab{#1}{#2},#3,#4 \ttt{)}}
\newcommand{\eqspre}{\eqs_{\mathit{pre}}}
\newcommand{\macbdoz}[1]{\psi_{\mathit{BDOZ}}(#1)}
\newcommand{\initigamma}{\mathit{init}}
\newcommand{\notg}[1]{\breve{#1}}

\long\def\cnote#1{{\small\textbf{\textit{\color{red}(*#1 -- Chris*)}}}}
\long\def\jnote#1{{\small\textbf{\textit{\color{brown}(*#1 -- Joe*)}}}}

\newcommand{\mynote}[2]
    {{\color{red} \fbox{\bfseries\sffamily\scriptsize#1}
    {\small$\blacktriangleright$\textsf{\emph{#2}}$\blacktriangleleft$}}~}
%\newcommand{\mynote}[2]{}
\newcommand{\todo}[1]{\mynote{TODO}{#1}}
    
%\newcommand{\note}[1]{\noindent\textit{(\textbf{$\star$note$\star$:}\ \ #1)}}



\begin{document}

\title{Automating Verification of MPC Security: $\metaprot$ to $\minifed$}

\author{Author names withheld for double-blind reviewing}

\begin{abstract}
Secure Multi-Party Computation (MPC) protocols support data privacy in
important modern distributed applications. Security for MPC protocols
is superficially similar to probabilistic noninterference, but differs
in a subtle but fundamental way, and approaches for verifying
noninterference cannot naturally extend to MPC security.  Currently,
proof methods for MPC protocols are well-studied but manual and thus
tedious and error-prone, and are also non-standardized and unfamiliar
to most PL theorists.  Our goal is to leverage connections between the
security model of MPC and trace-based hyperproperties to obtain
automated proof methods for MPC protocol development.  We develop a
language model with a tightly coupled notion of probabilistic program
distributions, as a foundation for fully and partially automated
verification of passive MPC security in protocols including
arbitrarily large YGC and GMW circuits.
\end{abstract}

%%
%% The code below is generated by the tool at http://dl.acm.org/ccs.cfm.
%% Please copy and paste the code instead of the example below.
%%
\begin{CCSXML}
<ccs2012>
   <concept>
       <concept_id>10002978.10002986.10002990</concept_id>
       <concept_desc>Security and privacy~Logic and verification</concept_desc>
       <concept_significance>500</concept_significance>
       </concept>
   <concept>
       <concept_id>10003752.10003753.10003757</concept_id>
       <concept_desc>Theory of computation~Probabilistic computation</concept_desc>
       <concept_significance>300</concept_significance>
       </concept>
   <concept>
       <concept_id>10003752.10003790.10003806</concept_id>
       <concept_desc>Theory of computation~Programming logic</concept_desc>
       <concept_significance>500</concept_significance>
       </concept>
 </ccs2012>
\end{CCSXML}

\ccsdesc[500]{Security and privacy~Logic and verification}
\ccsdesc[500]{Theory of computation~Probabilistic computation}
\ccsdesc[500]{Theory of computation~Programming logic}


%%
%% Keywords. The author(s) should pick words that accurately describe
%% the work being presented. Separate the keywords with commas.
\keywords{Secure multiparty computation, security verification, probabilistic programming, static analysis.}

%\maketitle

\section{The $\minicat$ Protocol Language}

\begin{fpfig}[t]{Top-to-bottom: Basic $\minifed$ syntax, expression interpretation, and reduction rules.}{fig-minifed}
  {
    $$
    \begin{array}{rcl@{\hspace{8mm}}r}
      \multicolumn{4}{l}{v \in \mathbb{Z}_p,\ w \in \mathrm{String},\ \cid \in \mathrm{Clients} \subset  \mathbb{N} }\\[2mm] %, \bop \in \{ \eand, \eor, \exor \}} \\[2mm]
      \be &::=& v \mid \flip{w} \mid \secret{w} \mid \mesg{w} \mid \rvl{w} \mid \be \fminus \be \mid \be \fplus \be \mid \be \ftimes \be \mid f \mid \be\,\be & \textit{expressions}\\[2mm]
      x &::=& \elab{\flip{w}}{\cid} \mid \elab{\secret{w}}{\cid} \mid \elab{\mesg{w}}{\cid} \mid \rvl{w} \mid \out{\cid} & \textit{protocl variables} \\[2mm]
      %& &  \select{\be}{\be}{\be} \mid \ctxt{v}{k} \mid \key{w} \mid \sk{\be}(\be) \mid \pk{\be}{\be}(\be) \mid \pk{\be}{\be} \\[2mm]
      %& &  \select{\fp(\be)}{\be}{\be} \ctxt{v,\be}{k}  \mid \sk{\be}(\be) \mid \pk{\be}{\be}(\be) \mid \pk{\be}{\be} \\[2mm]
      \instr &::=& \eassign{\mesg{w}}{\cid}{\be}{\cid} \mid
      \reveal{w}{e}{\cid} \mid \pubout{\cid}{\be}{\cid} & \textit{commands} \\[2mm]
      \prog &::=& \varnothing \mid \instr; \prog & \textit{protocols}
    \end{array}
    $$
  
  \rule{130mm}{0.5pt}

  $$
  \begin{array}{c@{\hspace{5mm}}c}
  \begin{array}{rcl}
    \lcod{\store, v}{\cid} &=& v\\
    \lcod{\store, \be_1 \fplus \be_2}{\cid} &=& \fcod{\lcod{\store, \be_1}{\cid} \fplus \lcod{\store, \be_2}{\cid}}\\ 
    \lcod{\store, \be_1 \fminus \be_2}{\cid} &=& \fcod{\lcod{\store, \be_1}{\cid} \fminus \lcod{\store, \be_2}{\cid}}\\ 
    \lcod{\store, \be_1 \ftimes \be_2}{\cid} &=& \fcod{\lcod{\store, \be_1}{\cid} \ftimes \lcod{\store, \be_2}{\cid}}
  \end{array} & 
  \begin{array}{rcl}
    \lcod{\store, \flip{w}}{\cid} &=& \store(\elab{\flip{w}}{\cid})\\
    \lcod{\store, \secret{w}}{\cid} &=& \store(\elab{\secret{w}}{\cid})\\
    \lcod{\store, \mesg{w}}{\cid} &=& \store(\elab{\mesg{w}}{\cid})\\
    \lcod{\store, \rvl{w}}{\cid} &=& \store(\rvl{w})\\
    \lcod{\store, f\,e_1\,\cdots\, e_n}{\cid} &=& \delta(f,\lcod{\store, e_1}{\cid},\ldots,\lcod{\store,e_n}{\cid})
  \end{array}
  \end{array}
  $$

  \vspace{4mm}
  
  \rule{130mm}{0.5pt}

  \begin{mathpar}
    (\store, \eassign{\mesg{w}}{\cid_1}{\be}{\cid_2};\prog) \redx (\extend{\store}{\mesg{w}_{\cid_1}}{\lcod{\store,\be}{\cid_2}}, \prog)
    
    (\store, \reveal{w}{\be}{\cid};\prog) \redx (\extend{\store}{\rvl{w}}{\lcod{\store,\be}{\cid}}, \prog)
    
    (\store, \pubout{\cid}{\be}{\cid};\prog) \redx (\extend{\store}{\out{\cid}}{\lcod{\store,\be}{\cid}}, \prog)
  \end{mathpar}
  }
\end{fpfig}

The $\minifed$ language provides a simple model of synchronous
protocols between a federation of \emph{clients} exchanging values in
the binary field. We will identify clients by natural numbers, and
federations- finite sets of clients- are always given statically.
As we will see, our threat model assumes a partition of the federation
into \emph{honest} $H$ and \emph{corrupt} $C$ subsets.

We model probabilistic programming via a \emph{random tape}
semantics. That is, we will assume that programs can make reference to
values chosen from a uniform random distributions defined in the
initial program memory.  Programs aka protocols execute
deterministically given the random tape.

\subsection{Syntax} The syntax of $\minifed$, defined in
Figure \ref{fig-minifed}, includes values $v$ and standard
operations of addition, subtraction, and multiplication in
a finite field $\mathbb{Z}_p$ with $p$ prime. 
Protocols are given input secret values $\secret{w}$
as well as random samples $\flip{w}$ on the input
tape, both of which are distinguished by
strings $w$. Protocols are sequences of assignment
commands of three different forms:
\begin{itemize}
\item $\eassign{\mesg{w}}{\cid_2}{\be}{\cid_1}$: This
  is a \emph{message send} where expression $\be$ is computed
  by client $\cid_1$ and sent to client $\cid_2$ as message
  $\mesg{w}$.
\item $\reveal{w}{\be}{\cid}$: This
  is a \emph{public reveal} where expression $\be$ is computed
  by client $\cid$ and broadcast to the federation.
\item $\pubout{\cid}{\be}{\cid}$: This
  is an \emph{output} where expression $\be$ is computed
  by client $\cid$ and reported as its output.
\end{itemize}
Both messages $\mesg{w}$ and reveals $\rvl{w}$ can be
referenced in expressions, once they've been assigned.

We let $x$ range over \emph{variables}  which are identifiers
where client ownership is specified- e.g., $\elab{\mesg{\mathit{foo}}}{\cid}$
is a message $\mathit{foo}$ that was sent to $\cid$. We let $X$
range over sets of variables, and more specifically, $S$ ranges over sets of secret variables $\elab{\secret{w}}{\cid}$, $R$ ranges over sets of random variables $\elab{\flip{w}}{\cid}$, $M$ ranges over sets of message variables $\elab{\mesg{w}}{\cid}$, $P$ ranges over sets of public variables $\rvl{w}$, and $O$ ranges over sets of output variabels $\out{\cid}$.
Given a program $\prog$, we write $\iov(\prog)$ to
denote the set of $S \cup M \cup P \cup O$ of variables in $\prog$
with ownership made explicit, and we write $\flips(\prog)$ to
denote the set $R$ of random samplings in $\prog$ with ownership
made explicit. We write
$\vars(\prog)$ to denote $\iov(\prog) \cup \flips(\prog)$. For any set
of variables $X$ and parties $P$, we write $X_P$ to denote the subset
of $X$ owned by any party in $P$, in particular we write $X_H$ and $X_C$ to
denote the subsets belonging to honest and corrupt parties,
respectively.

\subsubsection{Library Functions} $\minifed$ expression syntax also supports
calls to library functions $f$ which can be applied to muliple arguments in a
curried style. This allows encapsulation and separate
definition of primitive operations such as one-time-pads and message
authentication, as we will illustrate with examples. This approach is
useful since it parameterizes these definitions, and 
supports verification of behavior specified with types, as we
discuss in Section \ref{section-types}.

\subsection{Semantics}

\emph{Memories} are fundamental to the semantics of $\fedcat$ and
provide random tape and secret inputs to protocols, and also record
message sends, public broadcast, and client outputs. Memories $\store$ are finite
(partial) mapping from variables $x$ to values $v \in \mathbb{Z}_p$. The \emph{domain} of a
memory is written $\dom(\store)$ and is the finite set of variables on
which the memory is defined. We write $\store\{ x \mapsto v\}$ for
$x\not\in\dom(\store)$ to denote the memory $\store'$ such that
$\store'(x) = v$ and otherwise $\store'(y) = \store(y)$ for all $y
\in \dom(\store)$. We write $\store \subseteq \store'$ iff
$\dom(\store) \subseteq \dom(\store')$ and $\store(x) =
\store'(x)$ for all $x \in \dom(\store)$. We write $\store \cap
\store'$ to denote the combination of $\store$ and $\store'$
assuming $\store(x) = \store'(x)$ for all $x \in \dom(\store)
\cap \dom(\store')$, otherwise $\store \cap \store'$ is undefined.
We write $\store \subseteq \store'$ iff $\store \cap \store'
= \store$.

Given a set of variables $X$, we write $\store_X$ to denote the
memory $\store$ restricted to the domain $X$, and we define
$\mems(X)$ as the set of all memories with domain $X$:
$$
\mems(X) \defeq \{ \store \mid \dom(\store) = X \}
$$
Thus, given a protocol $\prog$, the set of all random tapes for
$\prog$ is $\mems(\flips(\prog))$.
%We let $\stores$ range
%over sets of memories with the same domain, and abusing notation
%we write $\dom(\stores)$ to denote the common domain,
%and $\stores_X \defeq \{ \store_X | \store \in \stores \}$.

Given a variable-free expression $\be$, we write $\cod{\be}$ to denote
the standard interpretation of $\be$ in the arithmetic field
$\mathbb{Z}_{p}$. With the introduction of variables to expressions,
we need to interpret variables with respect to a specific memory, and
all variables used in an expression must belong to a specified client.
Thus, we denote interpretation of expressions $\be$ computed on a
client $\cid$ as $\lcod{\store,\be}{\cid}$. This interpretation is
defined in Figure \ref{fig-minifed}. It is also parameterized by
$\delta$ which defines the semantics of library functions $f$.

The small-step reduction relation $\redx$ is then defined in Figure
\ref{fig-minifed} to evaluate commands. Reduction is a relation on
\emph{configurations} $(\store, \prog)$ where all three command forms-
message send, broadcast, and output- are implemented as updates to the
memory $\store$. We write $\redxs$ to denote the reflexive, transitive
closure of\ $\redx$. 

\subsection{Example: Passive-Secure Addition}

Shamir addition leverages homomorphic properties of addition in
arithmetic fields to implement secret addition. If a field value $v_1$
is in a uniform random distribution with other variables in a program,
then $v_1 \fplus v_2$ is an encryption of $v_2$ where $v_1$ is an
information theoretically secure one-time-pad, which is exploited for
secret sharing. Of course, $\fplus$ is also a meaningful operation
over any two field values regardless of their distributions.

To capture this distinction we introduce a function $\otp$
with the following specification:
$$
\delta(\otp,v_1,v_2) \defeq v_1 \fplus v_2
$$
Although the semantics are the same as addition, the use of $\otp$
makes a declarative distinction, but more importantly we will be
able to assign a type to $\otp$ that enforces the one-time discipline
on its first argument via type linearity as will be discussed in Section
\ref{section-types}.

To sum their secret values $\secret{\cid}$, each client $\cid$ in
the federation $\{ 1, 2, 3 \}$  samples a value $\locflip$
that can be used as a one-time pad in summation with another
random sample $\flip{x}$ and $\secret{\cid}$. This yields
two secret shares communicated as messages to the other clients,
while each client keeps $\locflip$ as its own share.
$$
\begin{array}{lll}
  \elab{\mesg{s1}}{2} &:=& \elab{(\otp\ \locflip\ (\flip{x} \fplus \secret{1})}{1} \\ 
  \elab{\mesg{s1}}{3} &:=& \elab{\flip{x}}{1} \\ 
  \elab{\mesg{s2}}{1} &:=& \elab{(\otp\ \locflip\ (\flip{x} \fplus \secret{2})}{2} \\ 
  \elab{\mesg{s2}}{3} &:=& \elab{\flip{x}}{2} \\ 
  \elab{\mesg{s3}}{1} &:=& \elab{(\otp\ \locflip\ (\flip{x} \fplus \secret{3})}{3} \\ 
  \elab{\mesg{s3}}{2} &:=& \elab{\flip{x}}{3}
\end{array}
$$
Due to field properties of $\fplus$ this scheme guarantees that messages
are viewed as random noise by any observer 
besides $\cid$ \cite{barthe2019probabilistic}. Next, each client
publicly reveals the sum of all of its shares, including its local
share. This does reveal information about secrets. Further there
is no one-time-pad to use in this summation.
$$
\begin{array}{lll}
  \rvl{1} &:=& \elab{(\locflip \fplus \mesg{s2} \fplus \mesg{s3})}{1} \\ 
  \rvl{2} &:=& \elab{(\mesg{s1} \fplus \locflip \fplus \mesg{s3})}{2} \\
  \rvl{3} &:=& \elab{(\mesg{s1} \fplus \mesg{s2} \fplus \locflip)}{3} 
\end{array}
$$
Finally, each client outputs the sum of each sum of shares, yielding
the sum of secrets. Note that this stage exposes no more information
than the previous public reveals. 
$$
%\elab{\mesg{o1}}{2} &:=& \elab{(\locflip \fplus \mesg{s2} \fplus \mesg{s3})}{1} \\ 
  %\elab{\mesg{o1}}{3} &:=& \elab{(\locflip \fplus \mesg{s2} \fplus \mesg{s3})}{1} \\ 
  %\elab{\mesg{o2}}{1} &:=& \elab{(\mesg{s1} \fplus \locflip \fplus \mesg{s3})}{2} \\
  %\elab{\mesg{o2}}{3} &:=& \elab{(\mesg{s1} \fplus \locflip \fplus \mesg{s3})}{2} \\ 
  %\elab{\mesg{o3}}{1} &:=& \elab{(\mesg{s1} \fplus \mesg{s2} \fplus \locflip)}{3} \\ 
  %\elab{\mesg{o3}}{2} &:=& \elab{(\mesg{s1} \fplus \mesg{s2} \fplus \locflip)}{3}\\ 
  %\pubout{1} &:=& \elab{(\locflip \fplus \mesg{s2} \fplus \mesg{s3} + \mesg{o2} + \mesg{o3})}{1}
\begin{array}{lll}
  \out{1} &:=& \elab{(\rvl{1} \fplus \rvl{2} + \rvl{3})}{1}\\
  \out{2} &:=& \elab{(\rvl{1} \fplus \rvl{2} + \rvl{3})}{2}\\
  \out{3} &:=& \elab{(\rvl{1} \fplus \rvl{2} + \rvl{3})}{3}
\end{array}
$$
It is well-known that additive secret sharing is passive
secure. That is, any adversarial observer can gain no more information
from the messages exchanged in the protocol than what is exposed by
the output alone. However, malicious adversaries can corrupt this
protocol by injecting ``fake'' sums of shares in their public reveals.

\subsection{Example: Malicious Secure Product}

$$
\begin{array}{rcl@{\hspace{8mm}}r}
  \multicolumn{3}{l}{m,k \in \mathbb{Z}_p} \qquad \macv ::= (v,[m_1,\ldots,m_n]) &
  \textit{MACed\ values}\\[2mm]
  \be &::=& \cdots \macv{v} \mid \be \macplus \be \mid \be \mactimes \be \mid \be \macminus \be\\[2mm]
  x &::=& \macx{\secret{w}}{\cid} \mid \macx{\flip{w}}{\cid} \mid \mack{x}{\cid}
\end{array}
  

    
$$
(v, [m_1,\ldots,m_n])
$$
$$
k_1,\ldots,k_n
$$
$$
m_\cid = k_\cid + (k_\Delta * v)
$$

$$
\delta(\macplus,(v^1, [m_1^1,\ldots,m^1_n]),(v^2, [m_1^2,\ldots,m^2_n]))
\defeq
(v^1 \fplus v^2, [m_1^1 \fplus m_1^2 ,\ldots,m^1_n \fplus m_n^2])
$$

$$
\elab{\macgv{\elab{v}{\cid}}}{1} \macplus \cdots \macplus \elab{\macgv{\elab{v}{\cid}}}{n} =
(v,\ldots)
$$

$$
\delta(\macotp,v_1,v_2) \defeq v_1 \macplus v_2
$$

$$
\delta(\macauth, (v, [\ldots,m_\cid,\ldots]), k_\cid) \defeq
     (v, [\ldots,m_\cid,\ldots]) \text{\ if\ } m_i = k_i + (k_\Delta * v)
$$

$$
\begin{array}{lcl}
  \elab{\mesg{a}}{2} &:=&
  \elab{(\macotp\ \macgv{\elab{\secret{x}}{1}}\ \macgv{\elab{\flip{a}}{\Oracle}})}{1}\\
  \elab{\mesg{a}}{1} &:=&
  \elab{(\macotp\ \macgv{\elab{\secret{x}}{1}}\ \macgv{\elab{\flip{a}}{\Oracle}})}{2}\\
  \elab{\mesg{b}}{2} &:=&
  \elab{(\macotp\ \macgv{\elab{\secret{y}}{2}}\ \macgv{\elab{\flip{b}}{\Oracle}})}{1}\\
  \elab{\mesg{b}}{1} &:=&
  \elab{(\macotp\ \macgv{\elab{\secret{y}}{2}}\ \macgv{\elab{\flip{b}}{\Oracle}})}{2}\\
  \elab{\mesg{d}}{1} &:=&
  \elab{(\macauth(\mesg{a}, \mack{\elab{\secret{x}}{1}}{2} \fminus \mack{\elab{\flip{a}}{\Oracle}}{2}) \macplus (\macgv{\elab{\secret{x}}{1}}\macminus\macgv{\elab{\flip{a}}{\Oracle}}))}{1}\\
  \elab{\mesg{e}}{1}&:=&
  \elab{(\macauth(\mesg{b}, \mack{\elab{\secret{y}}{2}}{2} \fminus \mack{\elab{\flip{b}}{\Oracle}}{2}) \macplus (\macgv{\elab{\secret{y}}{2}}\macminus\macgv{\elab{\flip{b}}{\Oracle}}))}{1}\\
  \rvl{1} &:=&
  \elab{( (\mesg{d} \mactimes \mesg{e}) \macplus
          (\mesg{d} \mactimes \macgv{\elab{\flip{b}}{\Oracle}}) \macplus
          (\mesg{e} \mactimes \macgv{\elab{\flip{a}}{\Oracle}}) \macplus \macgv{\elab{\secret{c}}{\Oracle}}
    )}{1}\\
  \elab{\mesg{d}}{2} &:=&
  \elab{(\macauth(\mesg{a}, \mack{\elab{\secret{x}}{1}}{1} \fminus \mack{\elab{\flip{a}}{\Oracle}}{1}) \macplus (\macgv{\elab{\secret{x}}{1}}\macminus\macgv{\elab{\flip{a}}{\Oracle}}))}{2}\\
  \elab{\mesg{e}}{2}&:=&
  \elab{(\macauth(\mesg{b}, \mack{\elab{\secret{y}}{2}}{1} \fminus \mack{\elab{\flip{b}}{\Oracle}}{1}) \macplus (\macgv{\elab{\secret{y}}{2}}\macminus\macgv{\elab{\flip{b}}{\Oracle}}))}{2}\\
  \rvl{2} &:=&
  \elab{( (\mesg{d} \mactimes \mesg{e}) \macplus
          (\mesg{d} \mactimes \macgv{\elab{\flip{b}}{\Oracle}}) \macplus
          (\mesg{e} \mactimes \macgv{\elab{\flip{a}}{\Oracle}}) \macplus \macgv{\elab{\secret{c}}{\Oracle}}
    )}{2}\\
  \out{1} &:=& \elab{(\rvl{1} \macplus \macauth(\rvl{2},\ldots))}{1} \\
  \out{2} &:=& \elab{(\macauth(\rvl{1},\ldots) \macplus \rvl{2})}{2}
\end{array}
$$




\section{Security Model}
\label{section-pmf}
\label{section-model}

MPC protocols are intended to implement some \emph{ideal
functionality} $\idealf$ with per-client outputs. In the $\minifed$
setting, Given a protocol $\prog$ that implements $\idealf$, with
$\iov(\prog) = S \cup V \cup O$, the domain of $\idealf$ is $\mems(S)$
and its range is $\mems(O)$.  Real/ideal security in the MPC
setting means that, given $\store \in \mems(S)$, a secure protocol
$\prog$ does not reveal any more information about honest secrets
$\store_H$ to parties in $C$ beyond what is implicitly declassified by
$\idealf(\sigma)$. Security comes in \emph{passive} and
\emph{malicious} flavors, wherein the adversary either follows the
rules or not, respectively. Characterization of both real world
protocol execution and simulation is defined
probabilistically. Following previous work
\cite{barthe2019probabilistic} we use probability mass functions to
express joint dependencies between input and output variables, as a
metric of information leakage.

\subsection{Probability Mass Functions} 

We define joint probability mass functions (pmfs) in the standard
manner, though following \cite{barthe2019probabilistic} we use
memories to denote mappings of variables to values (i.e., outcomes),
so for example given a pmf $\pmf$ we will write $\pmf(\{ \elab{\secret{x}}{1}
\mapsto 0, \elab{\mesg{y}}{2} \mapsto 1 \})$ to denote the (joint) probability that
$\elab{\secret{x}}{1} = 0 \wedge \elab{\mesg{y}}{2} = 1$.
\begin{definition}
  A \emph{probability mass function} $\pmf$ is a function
  mapping memories in $\mems(X)$ for some $X$ to values in $\mathbb{F}_p$, such that:
  $$
  \sum_{\store \in \mems(X)} \pmf(\store) \  = \ 1
  $$
\end{definition}
%To recover succinct and familiar notation, we may omit the domain of a
%distribution when it is clear from an application context-
%i.e., we allow the following sugaring:
%$$
%\pdf{}(\store) \defeq \pdf{\dom(\store)}(\store)
%$$
Now, we can define a notion of marginal and conditional
distributions as follows, which are standard for discrete
probability mass functions. 
\begin{definition}
  Given $\pmf$ the \emph{marginal distribution} of variables $X$
  in $\pmf$, denoted $\margd{\pmf}{X}$, is defined as follows:
  $$
  \forall \store \in \mems(X) \quad . \quad \margd{\pmf}{X}(\store) =
  \sum_{\store' \in \mems(X-\dom(\dom(\pmf)))} \pmf(\store \cap \store')
  $$
\end{definition}

\begin{definition}
  Given $\margd{\pmf}{X}$, let $\stores$ be a set of memories with the
  same domain $Y \subseteq X$. Then the \emph{conditional distribution given
  $\stores$}  denoted
  $\condd{\pmf}{X}{\stores}$ is a distribution with domain $X$ where for all
  $\store \in \mems(X)$:
  $$
  \condd{\pmf}{X}{\stores}(\store) =
  (\sum_{\store' \in \stores} \margd{\pmf}{X}(\store \cap \store')) /
  (\sum_{\store' \in \stores} \margd{\pmf}{Y}(\store'))
  $$
  where $\margd{\pmf}{X}(\store \cap \store')) = 0$ if $\store \cap \store'$ is undefined.
\end{definition}
To recover familiar notation we allow the syntactic
sugarings $\condd{\pmf}{X}{\store}  \defeq \condd{\pmf}{X}{\{ \store\}}$, and
$\pmf(\store)  \defeq \margd{\pmf}{X}(\store)$ and $\pmf(\store|\stores) \defeq
\condd{\pmf}{X}{\stores}(\store)$ where $\dom(\store) = X$.
%\begin{eqnarray*}
%  \condd{\pmf}{X}{\store}  &\defeq& \condd{\pmf}{X}{\{ \store\}}\\
%  \pmf(\store)  &\defeq& \margd{\pmf}{X}(\store)  \qquad \dom(\store) = X\\
%  \pmf(\store|\stores)  &\defeq& \condd{\pmf}{X}{\stores}(\store) \qquad \dom(\store) = X
%\end{eqnarray*}

We also define the \emph{support} of a distribution in the usual manner-
it is the set of values a set of variables can take on with non-zero
probability.
\begin{definition}[Support]
  $\support(\pmf) \defeq \{ (v_1,\ldots,v_n) \mid
  \pmf(x_1 \mapsto v_1, \ldots, x_n \mapsto v_n) > 0 \} $
\end{definition}

\subsection{Basic Distribution of a Protocol}
Now we can define the probability distribution of a program $\prog$,
that we denote $\progtt(\prog)$. Since $\fedcat$ is deterministic the
results of any run are determined by the input values together with
the random tape. And since we constrain programs to not overwrite
views, we are assured that \emph{final} memories contain both a
complete record of all initial secrets as well as views resulting from
communicated information. 

Our semantics require that random tapes contain values for all program
values $\elab{\flip{w}}{\cid}$ sampled from a uniform distribution
over $\mathbb{F}_p$. Input memories also contain input secret values
and possibly other initial view elements as a result of
pre-processing, e.g., Beaver triples for efficient multiplication,
and/or MACed share distributions as in BDOZ/SPDZ
\cite{evans2018pragmatic,10.1007/978-3-030-68869-1_3}. We define
$\runs(\prog)$ as the set of final memories resulting from execution
of $\prog$ given any initial memory, and treat all elements of
$\runs(\prog)$ as equally likely.  This establishes the basic program
distribution that can be marginalized and conditioned to quantify
input/output information dependencies.
%In this
%setting, given a program $\prog$ with $\iov(\prog) = S \cup V$ and
%$\flips(\prog) = F$ we will consider all $\store \in \mems(S \cup V
%\cup F)$ such that $ \config{\store_{S \cup F}}{\prog} \redxs
%\config{\store_}{\varnothing} $ to be equally probable, establishing
%the basic distribution of the program. %From this, we can immediately
%derive the marginal distribution of $S \cup V$ to reason about
%dependencies between secrets and views.
\begin{definition}
  \label{def-progtt}
  \label{def-progd}
  \label{definition-progd}
  Given $\prog$ with $\secrets(\prog) = S$ and $\flips(\prog) = R$ and
  pre-processing predicate $\preproc$ on memories, define:
  $$
  \begin{array}{c}
    \runs(\prog) \defeq \\
    \{ \store \mid \exists \store_1 \in \mems(R) . 
    \exists \store_2 . \preproc(\store_2) \wedge
    %(\dom(\store) = \iov(\prog) \cup R) \wedge
    (\store_1 \cap \store_2,\prog) \redxs (\store,\varnothing) \}
  \end{array}
  $$
  By default, $\preproc(\store) \iff \dom(\store) = S$, i.e.,
  the initial memory contains all input secrets in a uniform
  marginal distribution. Then the \emph{basic distribution of $\prog$}, written $\progtt(\prog)$, is
  defined such that for all $\store \in \mems(\iov(\prog) \cup R)$:
  $$
  \progtt(\prog)(\store) =  1 / |\runs(\prog)| \ \text{if}\ \store \in \runs(\prog), \text{otherwise}\ 0
  $$
  
  %In some cases, we will also be concerned with the (joint)
  %probabilities of expression interpretation given a preceding program
  %execution, and we write $\progtt(\prog, \be)$ to denote the program
  %distribution $\progtt(\prog;\itv := \be)$ where $\itv$ is a
  %special variable that is never used in programs.
\end{definition}


\subsection{Honest and Corrupt Views}

Information about honest secrets can be revealed to corrupt clients
through messages sent from honest to corrupt clients, and through
publicly broadcast information from honest clients. Dually,
corrupt clients can impact protocol integrity through the messages
sent from corrupt to honest clients, and through publicly broadcast information
from corrupt clients. We call the former \emph{corrupt views}, and
the latter \emph{honest views}. Generally we let $V$ range over sets
of views.
\begin{definition}[Corrupt and Honest Views]
  Given a program $\prog$ with $\iov(\prog) = S \cup M \cup P \cup O$,
  define $\views(\prog) \defeq M \cup P$, and define $\houtputs$ as
  the messages and reveals in $V \defeq M \cup P$ sent from honest to corrupt
  parties, called \emph{corrupt views}:
  $$
  \begin{array}{lcl}
    \houtputs & \defeq
        & \{\ \rvl{w} \mid\ \reveal{w}{\be}{\cid} \in \prog \wedge \cid \in H \ \}\ \cup \\
      & & \{\ \elab{\mesg{w}}{\cid}\ \mid\  \eassign{\mesg{w}}{\cid}{\be}{\cid'} \in
           \prog \wedge \cid \in C \wedge \cid' \in H \ \} 
  \end{array}
  $$
  and similarly define $\cinputs$ as the subset of $V$ sent from corrupt to honest
  parties, called \emph{honest views}:
  $$
  \begin{array}{lcl}
    \cinputs &  \defeq
        & \{\ \rvl{w} \mid\ \reveal{w}{\be}{\cid} \in \prog \wedge \cid \in C \ \} \ \cup\\
      & & \{\ \elab{\mesg{w}}{\cid}\ \mid\  \eassign{\mesg{w}}{\cid}{\be}{\cid'} \in
              \prog \wedge \cid \in H \wedge \cid' \in C \ \}
  \end{array}
  $$
\end{definition}

\subsection{Passive Correctness and Security}

In the passive setting we assume that $H$ and $C$ follow the
rules of protocols and share messages as expected. A first
consideration is whether a given protocol is \emph{correct}
with respect to an ideal functionality. 
\begin{definition}[Passive Correctness]
  %Given $\prog$ with output variables $\out{1},\ldots,\out{n}$ and ideal
  We say that a protocol $\prog$ is \emph{passive correct} for a functionality
  $\idealf$ iff for all $\store \in \mems(\secrets(\prog))$
  we have $\progtt(\prog)(\idealf(\store) \mid \store) = 1$.
  %with $\idealf(\store) = v_1,\ldots,v_n$ we have
  %$\progtt(\prog)(\out{1} \mapsto v_1,\ldots,\out{n} \mapsto v_n \mid \store) = 1$.
\end{definition}

In the passive setting the simulator must construct a probabilistic
algorithm $\SIM$, aka a \emph{simulation}, that is parameterized by
corrupt inputs and the output of an ideal functionality, and that
returns a reconstruction of corrupt views that is probabilistically
indistinguishable from the corrupt views in the real world protocol
execution.
\begin{definition}
  Given $\store$, and $v$,we write $ \prob(\SIM(\store,v) = \store') $
  to denote the probability that $\SIM(\store,v)$ returns corrupt views
  $\store'$ as a result. We write $\dist(\SIM(\store,v))$ to
  denote the distribution of corrupt views reconstructed by the
  simulation, where for
  all $\store' \in \mems(V)$:
  $$
  \dist(\SIM(\store,v))(\store')\ \defeq\ \prob(\SIM(\store,v) = \store') 
  $$
\end{definition}
Then we can define passive security in the real/ideal
model as follows. 
\begin{definition}[Passive Security]
  Assume given a program $\prog$ that correctly implements an ideal
  functionality $\idealf$, with $\views(\prog) = V$.  Then $\prog$
  is \emph{passive secure in the simulator model} iff there exists
  ad simulation $\SIM$ such that for all
  partitions of the federation into honest and corrupt sets $H$ and $C$
  and for all $\store \in \mems(S)$:
  $$
  \dist(\SIM(\store_{C},\idealf(\store))) = \condd{\progtt(\prog)}{\houtputs}{\store}
  $$
\end{definition}

\subsection{Malicious Security}

In the malicious model we assume that corrupt clients are in the
thrall of an adversary $\adversary$ who does not necessarily follow
the rules of the protocol.  We model this by positing a $\arewrite$
function which is given a corrupt memory $\store_C$ and expression
$\be$, and returns a rewritten expression that can be interpreted to
yield a corrupt input. We define the evaluation relation that
incorporates the adversary in Figure \ref{fig-adversary}.

\adversaryfig

A key technical distinction of the malicious setting is that it
typically incorporates ``abort''. Honest parties implement strategies
to detect rule-breaking-- aka \emph{cheating}-- by using, e.g.,
message authentication codes with semi-homomorphic properties as in
BDOZ/SPDZ \cite{10.1007/978-3-030-68869-1_3}. If cheating is detected,
the protocol is aborted. To model this, we extend $\minifed$ with an
\ttt{assert} command and extend the range of memories with
$\bot$. Note that the adversary is free to ignore their own
assertions.
\begin{definition}
  We add assertions of the form $\elab{\assert{\phi(\be)}}{\cid}$ to the command
  syntax of $\minifed$, where $\phi$ is a decidable predicate on
  $\mathbb{F}_p$ and with operational semantics given in Figure
  \ref{fig-adversary}. We also extend the range of memories $\store$
  to $\mathbb{F}_p \cup \{ \bot \}$.
\end{definition}

It is necessary to add $\bot$ to the range of memories since
the possibility of abort needs to be reflected in adversarial
runs of a protocol. We can define $\aruns(\prog,\adversary)$
as the ``prefix'' memories that result from possibly-aborting
protocols, but we also need to ``pad out'' the memories
of partial runs with $\bot$, as we define in $\botruns(\prog,\adversary)$,
to properly reflect the contents of views and outputs even in case of abort. 
\begin{definition}
  \label{def-aprogd}
  \label{def-aprogtt}
  \label{definition-aprogd}
  Given program $\prog$ with $\iov(\prog) = S \cup V \cup O$ and $\flips(\prog) = R$, and
  any assumed pre-processing predicate $\preproc$ on memories, define:
  $$
  \begin{array}{c}
    \aruns(\prog) \defeq \\
    \{ \store \mid \exists \store_1 \in \mems(R) . 
    \exists \store_2 . \preproc(\store_2) \wedge
    %(\dom(\store) = \iov(\prog) \cup R) \wedge
    (\store_1 \cap \store_2,\prog) \aredxs (\store,\varnothing) \}
  \end{array}
  $$
  where by default, $\preproc(\store) \iff \dom(\store) = S$, and also define the following
  which pads out undefined views and outputs with $\bot$:
  $$
  \begin{array}{l}
    \botruns(\prog) \defeq \\
    \qquad \{ \store\{ x_1 \mapsto \bot, \ldots, x_n \mapsto \bot \} \mid \\
    \qquad \phantom{\{} \store \in \aruns(\prog) \wedge \{ x_1,\ldots,x_n\} = (V \cup O) - \dom(\store) \}
  \end{array}
  $$
  Then the \emph{$\adversary$ distribution of $\prog$}, written $\progtt(\prog,\adversary)$, is
  defined such that for all $\store \in \mems(\iov(\prog) \cup R)$:
  $$
  \progtt(\prog,\adversary)(\store) =  1 / |\botruns(\prog)| \ \text{if}\ \store \in \botruns(\prog), \text{otherwise}\ 0
  $$
\end{definition}

Given this preamble, we can define malicious simulation and malicious security
in a standard manner \cite{evans2018pragmatic}, as follows.
\begin{definition}[Malicious Simulation]
  Given a protocol $\prog$ with $\iov(\prog) = S \cup V \cup O$, honest and corrupt 
  clients $H$ and $C$, adversary $\adversary$, and honest inputs
  $\store \in \mems(S_H)$, the \emph{malicious simulation}  $\SIM(\store)$ has three phases:
  \begin{enumerate}
  \item In the first phase $\SIM_1$, $\adversary$ gives the
    simulator some $\store' \in \mems(S_C)$, and the simulator consults an
    oracle to compute $\idealf(\store \cup \store') \in \mems(O)$.
  \item In the second phase $\SIM_2$, the simulator is given the corrupt
    outputs $\idealf(\store \cup \store')_C$, which are again given to
    $\adversary$, who decides either to abort or not. If so, then the
    simulator is given $\sigma_{\mathit{out}} \defeq \{ \out{\cid} \mapsto \bot \mid \cid \in H \}$
    and arbitrary internal state $\Sigma$.
    Otherwise the simulator is given $\sigma_{\mathit{out}} \defeq \idealf(\store \cup \store')_H$
    and $\Sigma$.
  \item In the third phase $\SIM_3$, given $\store_{\mathit{out}}$ and $\Sigma$, the simulator
    finally outputs
    $\store_{\mathit{out}} \cup \store_{\mathit{views}}$ for some
    calculated $\store_{\mathit{views}} \in \mems(\houtputs)$.
  \end{enumerate}
\end{definition}

\begin{definition}[Malicious Security]
  We write $\dist(\SIM(\store))$ to
  denote the distribution of honest outputs and corrupt views reconstructed by the
  malicious simulation, where for
  all $\store'$:
  $$
  \dist(\SIM(\store))(\store')\ \defeq\ \prob(\SIM(\store) = \store') 
  $$
  Then a protocol $\prog$ with $\iov(\prog) = S \cup V \cup O$ is \emph{malicious
  secure} iff for all $H$, $C$, $\adversary$, and $\store \in \mems(S_H)$:
  $$
  \dist(\SIM(\store)) = \condd{\progtt(\prog,\adversary)}{\houtputs \cup O_H}{\store}
  $$  
\end{definition}


%\newcommand{\sx}[2]{\elab{\secret{#1}}{#2}}
\newcommand{\mx}[2]{\elab{\mesg{#1}}{#2}} 
%\newcommand{\px}[2]{\elab{\rvl{#1}}{#2}} 
\newcommand{\rx}[2]{\elab{\flip{#1}}{#2}}
\newcommand{\ox}[2]{\elab{\out{#1}}{#2}}
\newcommand{\eqcast}[2]{#1\ \ttt{as}\ #2}
\newcommand{\signals}[1]{\stackrel{#1}{\leadsto}}

\newcommand{\tj}[6]{#1,#2,#3 \vdash_{#4} #5 : #6}
\newcommand{\ej}[6]{#1,#2,#3 \vdash #4 : #5,#6}
\newcommand{\cty}[2]{c(#1,#2)}
\newcommand{\setit}[1]{\{ #1 \}}
\newcommand{\ty}{T}
\newcommand{\lty}[2]{#1 \cdot #2}
\newcommand{\eqs}{\mathit{E}}
\newcommand{\toeq}[1]{\lfloor #1 \rfloor}
\newcommand{\eop}{\equiv}
\newcommand{\autheq}[1]{\phi_{\mathrm{auth}}(#1)}
\newcommand{\upgrade}[1]{\uparrow #1}
\newcommand{\seclev}{\mathcal{L}}

\newcommand{\mtj}[6]{#1,#2 \vdash #3 : (#4,#5,#6)}
\newcommand{\atj}[5]{#1,#2 \Vdash #3 : (#4,#5)}
\newcommand{\eqj}[4]{#1,#2 \vdash #3 : #4}
\newcommand{\cpj}[4]{#1,#2 \vdash #3 : #4}
\newcommand{\leakj}[3]{#1,#2 \vdash #3}
\newcommand{\tsig}{\mathrm{sig}}
\newcommand{\subn}{\rho}
\newcommand{\fty}[5]{\{ #1 \}\ #2\  \{ #3,#4,#5 \} }

\renewcommand{\redx}{\Rightarrow}
\renewcommand{\redxs}{\redx}
\newcommand{\abort}{\bot}
\newcommand{\pre}[1]{\ttt{pre}(#1)}
\newcommand{\post}[1]{\ttt{post}(#1)}
\newcommand{\eqflag}{\mathit{sw}}
\newcommand{\eqon}{\ttt{on}}
\newcommand{\eqoff}{\ttt{off}}
\newcommand{\eqtrans}[1]{\lfloor #1 \rfloor}
\newcommand{\mc}[4]{(#1,#2,#3,#4)}
\newcommand{\cmd}{\instr}
\renewcommand{\OT}[5]{\elab{\ttt{OT(}\elab{#1}{#2},#3,#4 \ttt{)}}{#5}}

\section{$\minicat$ Syntax and Semantics}

$$
    \begin{array}{rcl@{\hspace{2mm}}r}
      \multicolumn{4}{l}{v \in \mathbb{F}_p,\ w \in \mathrm{String},\ \cid \in \mathrm{Clients} \subset  \mathbb{N} }\\[2mm] %, \bop \in \{ \eand, \eor, \exor \}} \\[2mm]
      \be &::=& \flip{w} \mid \secret{w} \mid \mesg{w} \mid \rvl{w} \mid & \textit{expressions}\\
      & & v \mid \be \fminus \be \mid \be \fplus \be \mid \be \ftimes \be \\[2mm]
      x &::=& \elab{\flip{w}}{\cid} \mid \elab{\secret{w}}{\cid} \mid \elab{\mesg{w}}{\cid} \mid \rvl{w} \mid \out{\cid} & \textit{variables} \\[2mm]
      %& &  \select{\be}{\be}{\be} \mid \ctxt{v}{k} \mid \key{w} \mid \sk{\be}(\be) \mid \pk{\be}{\be}(\be) \mid \pk{\be}{\be} \\[2mm]
      %& &  \select{\fp(\be)}{\be}{\be} \ctxt{v,\be}{k}  \mid \sk{\be}(\be) \mid \pk{\be}{\be}(\be) \mid \pk{\be}{\be} \\[2mm]
      \prog &::=& \eassign{\mesg{w}}{\cid}{\be}{\cid} \mid
      \reveal{w}{e}{\cid} \mid \pubout{\cid}{\be}{\cid} \mid \prog;\prog & \textit{protocols} 
    \end{array}
$$

\bigskip
    
 $$
  %\begin{array}{c@{\hspace{5mm}}c}
  \begin{array}{rcl}
    \lcod{\store, v}{\cid} &=& v\\
    \lcod{\store, \be_1 \fplus \be_2}{\cid} &=& \fcod{\lcod{\store, \be_1}{\cid} \fplus \lcod{\store, \be_2}{\cid}}\\ 
    \lcod{\store, \be_1 \fminus \be_2}{\cid} &=& \fcod{\lcod{\store, \be_1}{\cid} \fminus \lcod{\store, \be_2}{\cid}}\\ 
    \lcod{\store, \be_1 \ftimes \be_2}{\cid} &=& \fcod{\lcod{\store, \be_1}{\cid} \ftimes \lcod{\store, \be_2}{\cid}}\\
  %\end{array} 
  %\begin{array}{rcl}
    \lcod{\store, \flip{w}}{\cid} &=& \store(\elab{\flip{w}}{\cid})\\
    \lcod{\store, \secret{w}}{\cid} &=& \store(\elab{\secret{w}}{\cid})\\
    \lcod{\store, \mesg{w}}{\cid} &=& \store(\elab{\mesg{w}}{\cid})\\
    \lcod{\store, \rvl{w}}{\cid} &=& \store(\rvl{w})\\
    %\lcod{\store, f\,\be_1\,\cdots\, \be_n}{\cid} &=& \delta(f,\lcod{\store, \be_1}{\cid},\ldots,\lcod{\store,\be_n}{\cid})
  \end{array}
  %\end{array}
  $$

\bigskip

  \begin{mathpar}
    (\store, \xassign{x}{\be}{\cid}) \redx \extend{\store}{x}{\lcod{\store,\be}{\cid}}

    \inferrule
    {(\store_1,\prog_1) \redx \store_2 \\ (\store_2,\prog_2) \redx \store_3 }
    {(\store_1,\prog_1;\prog_2) \redx \store_3}
    %(\store, \eassign{\mesg{w}}{\cid_1}{\be}{\cid_2};\prog) \redx (\extend{\store}{\mesg{w}_{\cid_1}}{\lcod{\store,\be}{\cid_2}}, \prog)    
    %(\store, \reveal{w}{\be}{\cid};\prog) \redx (\extend{\store}{\rvl{w}}{\lcod{\store,\be}{\cid}}, \prog)   
    %(\store, \pubout{\cid}{\be}{\cid};\prog) \redx (\extend{\store}{\out{\cid}}{\lcod{\store,\be}{\cid}}, \prog)
  \end{mathpar}

\section{$\minicat$ Adversarial Semantics}

$$
\begin{array}{rclr}
  (\store, \xassign{x}{\be}{\cid}) &\aredx&
  \extend{\store}{x}{\lcod{\store,\be}{\cid}} & \cid \in H\\
  (\store, \xassign{x}{\be}{\cid}) &\aredx&
  \extend{\store}{x}{\lcod{\arewrite(\store_C,\be)}{\cid}} & \cid \in C
\end{array}
$$

$$
\begin{array}{rcl@{\qquad}r}
  (\store,\elab{\assert{\be_1 = \be_2}}{\cid}) &\aredx& \store & \text{if\ }
  \lcod{\store,\be_1}{\cid} = \lcod{\store,\be_2}{\cid}  \text{\ or\ } \cid \in C\\
  (\store,\elab{\assert{\phi(\be)}}{\cid}) &\aredx& \abort & \text{if\ } \neg\phi(\store,\lcod{\store,\be}{\cid})
\end{array}
$$

\begin{mathpar}
  \inferrule
      {(\store_1,\prog_1) \aredx \store_2 \\ (\store_2,\prog_2) \aredx \store_3 }
      {(\store_1,\prog_1;\prog_2) \aredx \store_3}

  \inferrule
      {(\store_1,\prog_1) \aredx \abort}
      {(\store_1,\prog_1;\prog_2) \aredx \abort}
      
  \inferrule
      {(\store_1,\prog_1) \aredx \store_2 \\ (\store_2,\prog_2) \aredx \abort }
      {(\store_1,\prog_1;\prog_2) \aredx \abort}
\end{mathpar}

\section{$\minicat$ Constraint Typing}

$$
\begin{array}{rcl}
  \phi &::=& x \mid \phi \fplus \phi \mid \phi \fminus \phi \mid \phi \ftimes \phi \\
  \eqs &::=& \phi \eop \phi \mid \eqs \wedge \eqs 
\end{array}
$$

We write $\eqs_1 \models \eqs_2$ iff every model of $E_1$ is a model of $E_2$. Note that
this relation is reflexive and transitive.

\begin{mathpar}
  \toeq{x} = x

  \toeq{\elab{\be_1 \fplus \be_2}{\cid}} = \toeq{\elab{\be_2}{\cid}} \fplus \toeq{\elab{\be_1}{\cid}}

  \toeq{\elab{\be_1 \fminus \be_2}{\cid}} = \toeq{\elab{\be_2}{\cid}} \fminus \toeq{\elab{\be_1}{\cid}}

  \toeq{\elab{\be_1 \ftimes \be_2}{\cid}} = \toeq{\elab{\be_2}{\cid}} \ftimes \toeq{\elab{\be_1}{\cid}}
\end{mathpar}

\begin{mathpar}
  \toeq{\OT{\be_1}{\cid_1}{\be_2}{\be_3}{\cid_2}} =
  (\toeq{\elab{\be_1}{\cid_1}} \wedge \toeq{\elab{\be_3}{\cid_2}}) \vee
  (\neg\toeq{\elab{\be_1}{\cid_1}} \wedge \toeq{\elab{\be_2}{\cid_2}}) 
\end{mathpar}

\begin{mathpar}
  \toeq{\xassign{x}{\be}{\cid}} = x \eop \toeq{\elab{\be}{\cid}}

  \toeq{\prog_1;\prog_2} = \toeq{\prog_1} \wedge \toeq{\prog_2} 
\end{mathpar}

The motivating idea is that we can interpret any protocol $\prog$ as a set
of equality constraints $\toeq{\prog}$ and use an SMT solver to verify
properties relevant to correctness, confidentiality, and integrity.
Further, we can leverage entailment relation is critical for efficiency--
we can use annotations to obtain a weakened precondition for relevant properties.
That is, given $\prog$, program annotations or other cues can be used
to find a minimal $\eqs$ with $\toeq{\prog} \models \eqs$ for verifying
correctness and security.

\subsection{Confidentiality Types}

\begin{mathpar}
  \inferrule[DepTy]
  {}
  {\eqj{\varnothing}{\eqs}{\phi}{\vars(\phi)}}
  
  \inferrule[Encode]
  {\eqs \models \phi \eop \phi' \oplus \rx{w}{\cid} \\
   \oplus \in \{ \fplus,\fminus \}\\
   \eqj{R}{\eqs}{\phi'}{\ty}}
  {\eqj{R;\{ \rx{w}{\cid} \}}{\eqs}{\phi}{\setit{\cty{\rx{w}{\cid}}{\ty}}}}
\end{mathpar}

\begin{mathpar}
  \inferrule[Send]
            {\eqj{R}{\eqs}{\toeq{\elab{\be}{\cid}}}{\ty}}
            {\cpj{R}{\eqs}{\xassign{x}{\be}{\cid}}{x : \ty}}
            
  \inferrule[Seq]
            {\cpj{R_1}{\eqs}{\prog_1}{\Gamma_1}\\
             \cpj{R_2}{\eqs}{\prog_2}{\Gamma_2}}
            {\cpj{R_1;R_2}{\eqs}{\prog_1;\prog_2}{\Gamma_1;\Gamma_2}}
\end{mathpar}

\begin{definition}
  $\cpj{R}{\eqs}{\prog}{\Gamma}$ is \emph{valid} iff it is derivable and $\toeq{\prog} \models \eqs$.
\end{definition}

\begin{mathpar}
  \inferrule
      {\cid \in C}
      {\leakj{\Gamma}{C}{\Gamma(\mx{w}{\cid})}}

  \inferrule
      {\leakj{\Gamma}{C}{T_1 \cup T_2}}
      {\leakj{\Gamma}{C}{T_1}}

  \inferrule
      {\leakj{\Gamma}{C}{\setit{ \mx{w}{\cid} }}}
      {\leakj{\Gamma}{C}{\Gamma(\mx{w}{\cid})}}

  \inferrule
      {\leakj{\Gamma}{C}{\setit{ \rx{w}{\cid} }} \\ \leakj{\Gamma}{C}{\setit{ \cty{\rx{w}{\cid}}{\ty} }} }
      {\leakj{\Gamma}{C}{\ty}}
\end{mathpar}

\begin{theorem}
  If $\cpj{R}{\eqs}{\prog}{\Gamma}$ is valid and for all $H,C$
  it is not the case that $\leakj{\Gamma}{C}{\setit{\sx{w}{\cid}}}$ for $\cid \in H$,
  then $\prog$ satisfies gradual release.
\end{theorem}

\subsubsection{Example}

\begin{verbatimtab}
m[x]@1 := s2(s[x],-r[x],r[x])@2

// m[x]@1 == s[x]@2 + -r[x]@2 
// m[x]@1 : { c(r[x]@2, { s[x]@2 }) } 

m[y]@1 := OT(s[y]@1,-r[y],r[y])@2

// m[y]@1 == s[y]@1 + -r[y]@2
// m[y]@1 : { c(r[y]@2, { s[y]@1 }) } 
\end{verbatimtab}

\subsection{Compositional Type Verification in $\metaprot$}

\begin{mathpar}
  \inferrule[Mesg]
            {e_1 \redx \be \\ e_2 \redx \cid \\ \atj{R_1}{\eqs}{\toeq{\elab{\be}{\cid}}}{R_2}{\ty}}
            {\mtj{R_1}{\eqs}{\xassign{x}{e_1}{e_2}}{x:\ty}{R_1;R_2}{\eqs \wedge x \eop \toeq{\elab{\be}{\cid}}}}

  \inferrule[Encode]
            {e_1 \redx \be \\ e_2 \redx \cid \\ e_3 \redx \phi \\
              \eqs \models \toeq{\elab{\be}{\cid}} \eop \phi\\
              \atj{R_1}{\eqs}{\phi}{R_2}{\ty}}
            {\mtj{R_1}{\eqs}{\eqcast{\xassign{x}{e_1}{e_2}}{e_3}}{x:\ty}{R_1;R_2}{E\wedge x \eop \phi}}

  \inferrule[App]
            {\tsig(f) = \fty{\eqs_1}{x_1,\ldots,x_n}{\Gamma}{R}{\eqs_2} \\
              e_1 \redx \mv_1\ \cdots\ e_n \redx \mv_n \\
              \subn = [\mv_1/x_1]\cdots[\mv_n/x_n] \\
              \eqs \models \subn(\eqs_1)}
            {\mtj{R_1}{\eqs}{f(e_1,\ldots,e_n)}{\subn(\Gamma)}{R_1;\subn(R)}{\eqs \wedge \subn(\eqs_2)}}

  \inferrule[Seq]          
            {\mtj{R_1}{\eqs_1}{\prog_1}{\Gamma_2}{R_2}{\eqs_2}\\ \mtj{R_2}{\eqs_2}{\prog_2}{\Gamma_3}{R_3}{\eqs_3}}
            {\mtj{R_1}{\eqs_1}{\prog_1;\prog_2}{\Gamma_2;\Gamma_3}{R_3}{\eqs_3}}
\end{mathpar}

\subsection{Integrity Types}

\begin{mathpar}
  \inferrule[Value]
  {}
  {\tj{\Gamma}{\varnothing}{\eqs}{\cid}{v}{\lty{\varnothing}}{\hilab}}
  
  \inferrule[Secret]
  {}
  {\tj{\Gamma}{\varnothing}{\eqs}{\cid}{\secret{w}}{\lty{\setit{\sx{w}{\cid}}}{\seclev(\cid)}}}
  
  \inferrule[Rando]
  {}
  {\tj{\Gamma}{\varnothing}{\eqs}{\cid}{\flip{w}}{\lty{\setit{\rx{w}{\cid}}}{\seclev(\cid)}}}  
  
  \inferrule[Mesg]
  {}
  {\tj{\Gamma}{\varnothing}{\eqs}{\cid}{\mesg{w}}{\Gamma(\mx{w}{\cid})}}
    
  \inferrule[PubM]
  {}
  {\tj{\Gamma}{\varnothing}{\eqs}{\cid}{\rvl{w}}{\Gamma(\rvl{w})}}

  \inferrule[IntegrityWeaken]
  {\tj{\Gamma}{R}{\eqs}{\cid}{\be}{\lty{\ty}{\latel_1}} \\ \latel_1 \sle \latel_2}
  {\tj{\Gamma}{R}{\eqs}{\cid}{\be}{\lty{\ty}{\latel_2}}}

  \inferrule[Encode]
  {\tj{\Gamma}{\varnothing}{\eqs}{\cid}{\be}{\lty{\ty}{\latel}} \\
    \eqs \models \toeq{\elab{\be}{\cid}} = \phi \oplus \rx{w}{\cid'} \\ \oplus \in \{ \fplus,\fminus \}}
  {\tj{\Gamma}{\rx{w}{\cid}}{\eqs}{\cid}{\be}{\lty{\setit{\cty{\rx{w}{\cid'}}{\Gamma(\phi)}}}}{\latel}}
  
  %\inferrule[RandoDeduce]
  %{\tj{\Gamma}{\varnothing}{\eqs}{\cid}{\be}{\lty{\ty}{\latel}} \\ \eqs \models
  %            \toeq{\elab{\be}{\cid}} = \rx{w}{\cid}'}
  %{\tj{\Gamma}{\varnothing}{\eqs}{\cid}{\be}{\lty{\setit{\rx{w}{\cid}}}{\latel}}}
  %
  %\inferrule[Encode]
  %{\tj{\Gamma}{R_1}{\eqs}{\cid}{\be_1}{\lty{\ty}{\latel}} \\
  % \tj{\Gamma}{R_2}{\eqs}{\cid}{\be_2}{\lty{\setit{\rx{w}{\cid}}}{\latel}} \\ \oplus \in \{ \fplus,\fminus \}}
  %{\tj{\Gamma}{R1;R_2;\rx{w}{\cid}}{\eqs}{\cid}{\be_1 \oplus \be_2}{\lty{\setit{\cty{\rx{w}{\cid}}{\ty}}}}{\latel}}
  %
  \inferrule[Binop]
  {\tj{\Gamma}{R_1}{\eqs}{\cid}{\be_1}{\lty{\ty_1}{\latel}} \\
   \tj{\Gamma}{R_2}{\eqs}{\cid}{\be_2}{\lty{\ty_2}{\latel}} \\ \oplus \in \{ \fplus,\fminus,\ftimes \}}
  {\tj{\Gamma}{R_1;R_2}{\eqs}{\cid}{\be_1 \oplus \be_2}{\lty{\ty_1 \cup \ty_2}}{\latel}}
\end{mathpar}

\begin{mathpar}
  \inferrule[Send]
            {\tj{\Gamma}{R}{\eqs}{\cid}{\be}{\lty{\ty}{\seclev(\cid)}} \\
             \eqs' \models \eqs \wedge x = \toeq{\elab{\be}{\cid}}}
            {\ej{\Gamma}{R}{\eqs}{\xassign{x}{\be}{\cid}}{\Gamma; x : \lty{\ty}{\seclev(\cid)}}
              {\eqs'}}
             
  \inferrule[Assert]
            {\eqs \models \toeq{\elab{\be_1}{\cid}} = \toeq{\elab{\be_2}{\cid}}}
            {\ej{\Gamma}{R}{\eqs}{\elab{\assert{\be_1 = \be_2}}{\cid}}{\Gamma}{\eqs}}
            
  \inferrule[Seq]
            {\ej{\Gamma_1}{R_1}{\eqs_1}{\prog_1}{\Gamma_2}{\eqs_2}\\
             \ej{\Gamma_2}{R_2}{\eqs_2}{\prog_2}{\Gamma_3}{\eqs_3}}
            {\ej{\Gamma_1}{R_1;R_2}{\eqs_1}{\prog_1;\prog_2}{\Gamma_3}{\eqs_3}}

  \inferrule[Constraint]
      {\ej{\Gamma_1}{R}{\eqs_1}{\prog}{\Gamma_2}{\eqs_2} \\ \eqs_1' \models \eqs_1' \\ \eqs_2 \models \eqs_2'}
      {\ej{\Gamma_1}{R}{\eqs_1'}{\prog}{\Gamma_2}{\eqs_2'}}
             
  \inferrule[MAC]
            {\eqs \models 
              \mx{w\ttt{m}}{\cid} = \mx{w\ttt{k}}{\cid} \fplus \ttt{(}\mx{\ttt{delta}}{\cid} \ftimes
                  \mx{w\ttt{s}}{\cid}\ttt{)} \\
              \Gamma(\mx{w\ttt{s}}{\cid}) = \lty{\ty}{\latel}}
            {\ej{\Gamma}{R}{\eqs}{
                \elab{\assert{\mesg{w\ttt{m}} = \mesg{w\ttt{k}} \fplus \ttt{(}\mesg{\ttt{delta}} \ftimes
                  \mesg{w\ttt{s}}\ttt{)}}}{\cid}}{\Gamma;\mx{w\ttt{s}}{\cid}:\lty{\ty}{\hilab}}{\eqs}}
\end{mathpar}


\section{$\metaprot$ Syntax and Semantics}

$$
\begin{array}{rcl}
  \multicolumn{3}{l}{\flab \in \mathrm{Field},\   y \in \mathrm{EVar}, \  f \in \mathrm{FName}}\\[1mm]
  %x &\in& \mathrm{EVar}\\
  %f &\in& \mathrm{FName}\\[2mm]
  e &::=& \mv \mid \flip{e} \mid \secret{e} \mid \mesg{e} \mid \rvl{e} \mid e \bop e \mid
  \elet{y}{e}{e} \mid \\
  & & f(e,\ldots,e) \mid \{ \flab = e; \ldots; \flab = e \} \mid e.\flab \\
  %  & \textit{expressions}\\
  \cmd &::=& \msend{e}{e}{e}{e} \mid \reveal{e}{e}{e} \mid \pubout{e}{e}{e} \mid
      \elab{\assert{e = e}}{e} \mid \\
  & & f(e,\ldots,e) \mid  \cmd;\cmd \mid \pre{\eqs} \mid \post{\eqs} \\[1mm]
  \bop &::=& \fplus \mid \fminus \mid \ftimes \mid \concat  \\[1mm]% \textit{operators}\\[2mm]
  \mv &::=& w \mid \cid \mid \be \mid \{ \flab = \mv;\ldots;\flab = \mv \} 
  \\ % \mid \ttt{()} \\[1mm] %& \textit{values}\\[2mm]
  \mathit{fn} &::=& f(y,\ldots,y) \{ e \} \mid  f(y,\ldots,y) \{ \cmd \} \\[1mm]%& \textit{functions}
  \phi &::=& \elab{\flip{e}}{e} \mid \elab{\secret{e}}{e} \mid \elab{\mesg{e}}{e} \mid \rvl{e} \mid \out{e} \mid \phi \fplus \phi \mid \phi \fminus \phi \mid \phi \ftimes \phi \\
  \eqs &::=& \phi \eop \phi \mid \eqs \wedge \eqs 
\end{array}
$$

\begin{mathpar}
  \inferrule
      {e[\mv/y] \redx \mv'}
      {\elet{y}{\mv}{e} \redx \mv'}

  \inferrule
      {\codebase(f) = y_1,\ldots,y_n,\ e \\ e_1 \redx \mv_1 \cdots e_n \redx \mv_n \\
        e[\mv_1/y_1]\cdots[\mv_n/y_n] \redx \mv}
      {f(e_1,\ldots,e_n) \redx \mv}

  \inferrule
      {e_1 \redx \mv_1 \cdots e_n \redx \mv_n }
      {\{ \flab_1 = e_1; \ldots; \flab_n = e_n \} \redx \{ \flab_1 = \mv_1; \ldots; \flab_n = \mv_n \} }

  \inferrule
      {e \redx \{\ldots; \flab = \mv; \ldots\}}
      {e.\flab \redx \mv}

  \inferrule
      {e_1 \redx w_1 \\ e_2 \redx w_2}
      {e_1 \concat e_2 \redx w_1w_2}
\end{mathpar}

\begin{mathpar}
  \inferrule
      {e_1 \redx \be_1 \\ e_2 \redx \be_2 \\ e \redx \cid}
      {\mc{\prog}{(\eqs_1,\eqs_2)}{\eqon}{\elab{\assert{e_1 = e_2}}{e}} \redx
        (\prog;\elab{\assert{\be_1 = \be_2}}{\cid},(\eqs_1, \eqs_2 \wedge \eqtrans{\elab{\be_1}{\cid}} = \eqtrans{\elab{\be_2}{\cid}}),\eqon)}
      
  \inferrule
      {e_1 \redx \be_1 \\ e_2 \redx \be_2 \\ e \redx \cid}
      {\mc{\prog}{(\eqs_1,\eqs_2)}{\eqoff}{\elab{\assert{e_1 = e_2}}{e}} \redx
        (\prog;\elab{\assert{\be_1 = \be_2}}{\cid},(\eqs_1, \eqs_2,\eqoff)}
      
  \inferrule
      {e_1 \redx w \\ e_2 \redx \cid_1 \\ e_3 \redx \be \\ e_4 \redx \cid_2}
      {\mc{\prog}{(\eqs_1,\eqs_2)}{\eqon}{\msend{e_1}{e_2}{e_3}{e_4}} \redx
        (\prog;\msend{w}{\cid_1}{\be}{\cid_2},
        (\eqs_1 \wedge \mx{w}{\cid_1} = \eqtrans{\elab{\be}{\cid_2}}, \eqs_2),\eqon)}
      
  \inferrule
      {e_1 \redx w \\ e_2 \redx \cid_1 \\ e_3 \redx \be \\ e_4 \redx \cid_2}
      {\mc{\prog}{(\eqs_1,\eqs_2)}{\eqoff}{\msend{e_1}{e_2}{e_3}{e_4}} \redx
        (\prog;\msend{w}{\cid_1}{\be}{\cid_2},
        (\eqs_1, \eqs_1),\eqoff)}

  \inferrule
      {}
      {\mc{\prog}{(\eqs_1,\eqs_2)}{\eqon}{\pre{\eqs}} \redx (\prog,\eqs_1,\eqs_2 \wedge \eqs,\eqoff)}

  \inferrule
      {}
      {\mc{\prog}{(\eqs_1,\eqs_2)}{\eqoff}{\post{\eqs}} \redx (\prog,(\eqs_1 \wedge \eqs,\eqs_2),\eqon)}
            
  \inferrule
      {\mc{\prog_1}{(\eqs_{11},\eqs_{12})}{\eqflag_1}{\cmd_1} \redx
        (\prog_2,(\eqs_{21},\eqs_{22}),\eqflag_2) \\
       \mc{\prog_2}{(\eqs_{21},\eqs_{22})}{\eqflag_2}{\cmd_2} \redx
        (\prog_3,(\eqs_{31},\eqs_{32}),\eqflag_3)}
      {\mc{\prog_1}{(\eqs_{11},\eqs_{12})}{\eqflag_1}{\cmd_1;\cmd_2} \redx
        (\prog_3,(\eqs_{31},\eqs_{32}),\eqflag_3)}

  \inferrule
      {\codebase(f) = y_1,\ldots,y_n,\ \cmd \\ e_1 \redx \mv_1 \cdots e_n \redx \mv_n \\
        \mc{\prog_1}{(\eqs_{11},\eqs_{12})}{\eqflag_1}{\cmd[\mv_1/y_1,]\cdots[\mv_n/y_n]} \redx
        (\prog_2,(\eqs_{21},\eqs_{22}),\eqflag_2)}
      {\mc{\prog_1}{(\eqs_{11},\eqs_{12})}{\eqflag_1}{f(e_1,\ldots,e_n)} \redx
        (\prog_2,(\eqs_{21},\eqs_{22}),\eqflag_2)}
\end{mathpar}

\section{Examples}

\begin{verbatimtab}
    encodegmw(in, i1, i2) {
      m[in]@i2 := (s[in] xor r[in])@i2;
      m[in]@i1 := r[in]@i2
    }
    
    andtablegmw(b1, b2, r) {
      let r11 = r xor (b1 xor true) and (b2 xor true) in
      let r10 = r xor (b1 xor true) and (b2 xor false) in
      let r01 = r xor (b1 xor false) and (b2 xor true) in
      let r00 = r xor (bl xor false) and (b2 xor false) in
      { row1 = r11; row2 = r10; row3 = r01; row4 = r00 }
    }
    
    andgmw(z, x, y) {
      pre();
      let r = r[z] in
      let table = andtablegmw(m[x],m[y],r) in
      m[z]@2 := OT4(m[x],m[y],table,2,1);
      m[z]@1 := r@1;
      post(m[z]@1 xor m[z]@2 == (m[x]@1 xor m[x]@2) and (m[y]@1 xor m[y]@2))
    }
    
    xorgmw(z, x, y) {
      m[z]@1 := (m[x] xor m[y])@1; m[z]@2 := (m[x] xor m[y])@2;
    }
    
    decodegmw(z) {
      p["1"] := m[z]@1; p["2"] := m[z]@2;
      out@1 := (p["1"] xor p["2"])@1;
      out@2 := (p["1"] xor p["2"])@2
    }

    encodegmw("x",2,1);
    encodegmw("y",2,1);
    encodegmw("z",1,2);
    andgmw("g1","x","z");
    xorgmw("g2","g1","y");
    decodegmw("g2")
    pre();
    post(out@1 == (s["x"]@1 and s["z"]@2) xor s["y"]@1)
    
\end{verbatimtab}


\begin{verbatimtab}
  secopen(w1,w2,w3,i1,i2) {
      pre(m[w1++"m"]@i2 == m[w1++"k"]@i1 + (m["delta"]@i1 * m[w1++"s"]@i2 /\
          m[w1++"m"]@i2 == m[w1++"k"]@i1 + (m["delta"]@i1 * m[w1++"s"]@i2));
      let locsum =  macsum(macshare(w1), macshare(w2)) in
      m[w3++"s"]@i1 := (locsum.share)@i2;
      m[w3++"m"]@i1 := (locsum.mac)@i2;
      auth(m[w3++"s"],m[w3++"m"],mack(w1) + mack(w2),i1);
      m[w3]@i1 := (m[w3++"s"] + (locsum.share))@i1
  }

  
  _open(x,i1,i2){
    m[x++"exts"]@i1 := m[x++"s"]@i2;
    m[x++"extm"]@i1 := m[x++"m"]@i2;
    assert(m[x++"extm"] == m[x++"k"] + (m["delta"] * m[x++"exts"]));
    m[x]@i1 := (m[x++"exts"] + m[x++"s"])@i2
  }`
  
  _sum(z, x, y,i1,i2) {
      pre(m[x++"m"]@i2 == m[x++"k"]@i1 + (m["delta"]@i1 * m[x++"s"]@i2 /\
          m[y++"m"]@i2 == m[y++"k"]@i1 + (m["delta"]@i1 * m[y++"s"]@i2));
      m[z++"s"]@i2 := (m[x++"s"] + m[y++"s"])@i2;
      m[z++"m"]@i2 := (m[x++"m"] + m[y++"m"])@i2;
      m[z++"k"]@i1 := (m[x++"k"] + m[y++"k"])@i1;
      post(m[z++"m"]@i2 == m[z++"k"]@i1 + (m["delta"]@i1 * m[z++"s"]@i2)
  }

  sum(z,x,y) { _sum(z,x,y,1,2);_sum(z,x,y,2,1) }

  open(x) { _open(x,1,2); _open(x,2,1) } 


  sum("a","x","d");
  open("d");
  sum("b","y","e");
  open("e");
  let xys =
      macsum(macctimes(macshare("b"), m["d"]),
             macsum(macctimes(macshare("a"), m["e"]),
                    macshare("c")))
  let xyk = mack("b") * m["d"] + mack("a") * m["e"] + mack("c")
                    
  secopen("a","x","d",1,2);
    secopen("a","x","d",2,1);
    secopen("b","y","e",1,2);
    secopen("b","y","e",2,1);
    let xys =
      macsum(macctimes(macshare("b"), m["d"]),
             macsum(macctimes(macshare("a"), m["e"]),
                    macshare("c")))
    in
    let xyk = mack("b") * m["d"] + mack("d") * m["d"] + mack("c")               
    in
    secreveal(xys,xyk,"1",1,2);
    secreveal(maccsum(xys,m["d"] * m["e"]),
              xyk - m["d"] * m["e"],
              "2",2,1);
    out@1 := (p[1] + p[2])@1;
    out@2 := (p[1] + p[2])@2;
\end{verbatimtab}




%\bibliographystyle{ACM-Reference-Format}
%\bibliography{logic-bibliography,secure-computation-bibliography}

\end{document}
\endinput
