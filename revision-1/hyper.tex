\section{Security Hyperproperties}

\begin{definition}
  We write $\vc{\pmf}{x}{y}$ iff $\pmf(\{ x \mapsto 0\}\ |\ \{ y \mapsto 0 \}) =
  \pmf(\{ x \mapsto 1\}\ |\ \{ y \mapsto 1 \}) = 1$.
  We write $\sep{\pmf}{X}{Y}$ iff for all
    $\store \in \mems(X \cup Y)$ we have
  $\margd{\pmf}{X \cup Y}(\store) =
  \pmf(\store_X) * \pmf(\store_Y)$
\end{definition}


\subsection{Passive Security and Noninterference}

\cnote{Fix the $O$ output variable things, and scoping of $H,C$.}

\begin{definition}[Passive Confidentiality]
  \label{definition-NIMO}
  We say that a program $\prog$ satisfies \emph{passive confidentiality}
  iff for all $H$ and $C$ with $|C|\le|H|$ and 
  $\store_1,\store_2 \in \mems(S)$ we have:
  $$
  (\store_1 =_C \store_2 \ \wedge \ 
  (\condd{\progtt(\prog)}{\{ \outv \}}{\store_1} = \condd{\progtt(\prog)}{\{ \outv \}}{\store_2}))
  \implies 
  (\condd{\progtt(\prog)}{\houtputs}{\store_1} = \condd{\progtt(\prog)}{\houtputs}{\store_2})
  $$
  where $\iov(\prog) = (S,M)$.
\end{definition}

\subsection{Gradual Release as a Design Pattern}

\begin{definition}
  Given $H,C$, a protocol $\prog$ with $\iov(\prog) = (S,M,O)$ satisfies \emph{gradual release} iff
  $\sep{\progtt(\prog)}{M_C}{S_H}$.
\end{definition}

\subsection{Malicious Security and Robust Declassification}

\begin{definition}
  We say that a protocol $\prog$ with $\iov(\prog) = (S,M)$ satisfies \emph{active confidentiality} iff the following conditions hold
  for all adversaries $\adversary$:
  \begin{enumerate}
  \item $\ \,\forall \store \in \mems(S_H) \ .\ \support(\progtt(\prog)(\{ \outv \}|\store)) =
    \support(\progtt(\prog,\adversary)(\{ \outv \}|\store))$
  \item $\begin{array}[t]{l}\forall \store_1, \store_2 \in \mems(S_H), \store \in \mems(\cinputs)\ . \\
    \quad
    \condd{\progtt(\prog,\adversary)}{\{ \outv \}}{\store_1 \cup \store} =
    \condd{\progtt(\prog,\adversary)}{\{ \outv \}}{\store_2 \cup \store} \\
    \qquad \implies\\
    \quad
    \condd{\progtt(\prog,\adversary)}{\houtputs}{\store_1 \cup \store} =
    \condd{\progtt(\prog,\adversary)}{\houtputs}{\store_2 \cup \store}\end{array}$
  \end{enumerate}
\end{definition}

\begin{theorem}
  If $\prog$ satisfies active confidentiality for all $H,C$ then it is malicious secure.
\end{theorem}

\begin{definition}
  A protocol $\prog$ with has \emph{integrity} iff 
  $\forall \adversary . \runs_\adversary(\prog) \subseteq \runs(\prog)$.
\end{definition}

\begin{definition}
  A protocol $\prog$ with $\iov(\prog) = (S,M,O)$ is \emph{malicious correct} iff:
  $$
  \forall \adversary, \store \in \mems(S_H) \ .\ \support(\progtt(\prog)(O_H|\store)) =
    \support(\progtt(\prog,\adversary)(O_H|\store))
  $$
\end{definition}

\begin{theorem}
  If a protocol has integrity it is malicious correct.
\end{theorem}

\begin{theorem}
  If a protocol is passive secure with integrity, then it satisfies active confidentiality.
\end{theorem}

\begin{theorem}
  If a protocol is passive secure with integrity, then it is malicious secure.
\end{theorem}

\begin{definition}[Robust Declassification]
  A protocol satisfies \emph{robust declassification} iff it has integrity and
  satisfies gradual release. 
\end{definition}

\begin{theorem}
  Passive security with robust declassification implies malicious security.
\end{theorem}
