\section{Security Hyperproperties}

In this Section we formulate probabilistic versions of well-studied
hyperproperties of confidentiality and integrity, including noninterference,
gradual release, declassification, and robust declassification.
We demonstrate a soundness relation between noninterference and
passive security, and between robust declassification and malicious
security. We subsequently leverage this relation to enforce
malicious security using ``traditional'' security type methods
in Section \ref{section-types}. Previous work has explored
a similar approach to security type enforcement
\cite{6266151,almeida2018enforcing} but mainly
for aspects of passive security.

\subsection{Passive Security and Noninterference}

Since MPC protocols release some information about secrets through
outputs of $\idealf$, they do not enjoy strict noninterference.  As
discussed in Section \ref{section-lang}, public reveals and protocol
outputs are fundamentally forms of declassification.  But consistent
with other work \cite{8429300}, we can formulate a version of
probabilistic noninterence conditioned on output that is sound
for passive security. It says that if two low-equivalent secret
inputs generate the same output, then the distributions of corrupt
views are the same. 
\begin{definition}[Noninterference modulo output]
  \label{definition-NIMO}
  We say that a program $\prog$ satisfies \emph{noninterference modulo output}
  iff for all $H$ and $C$ and 
  $\store_1,\store_2 \in \mems(S)$ we have:
  $$
  (\store_1 =_C \store_2 \ \wedge \ 
  (\condd{\progtt(\prog)}{O}{\store_1} = \condd{\progtt(\prog)}{O}{\store_2}))
  \implies 
  (\condd{\progtt(\prog)}{\houtputs}{\store_1} = \condd{\progtt(\prog)}{\houtputs}{\store_2})
  $$
  where $\iov(\prog) = S \cup V \cup O$.
\end{definition}
Intuitively, this conditional noninterference property implies that
the simulator can just run the protocol in simulation to
reconstruct real world corrupt view distributions. But it requires
that the simulator can tractably ``pre-image'' a given output of
a functionality $\idealf$, to determine the inputs that
could have produced it. This pre-image is called a
\emph{kernel} in recent work \cite{XXX}.
\begin{definition}
  Given a functionality $\idealf$ and output value $v$, its
  \emph{kernel}, denoted $\ik(\idealf,v)$ is
  $
  \{ \store\ |\ \idealf(\store) = v \}
  $.
  We say that $\idealf$ is \emph{pre-imageable} iff $\ik(\idealf, v)$ for all
  $v$ can be computed tractably.
\end{definition}
A soundness result for passive security can then be stated as follows.
We prove this in a separate manuscript \cite{XXX}, and it is also
essentially the same as ``perfect passive NI security'' which
has a similar soundness property \cite{8429300}.  
\begin{restatable}{theorem}{nimosecure}
  \label{theorem-nimo}
  Assume given pre-imageable $\idealf$ and a protocol $\prog$ that
  correctly implements $\idealf$.  If $\prog$ satisfies noninference modulo output
  then $\prog$ is passive secure.
\end{restatable}

\subsection{Delimited Release}

Previous work has discussed how MPC security is not noninterference,
but rather how ideal functional sets an upper bound on
declassification \cite{6266151,almeida2018enforcing}. Nevertheless,
probabilistic noninterference is preserved by components of
cryptographic protocols generally, and can be expressed using
\emph{probabilistic independence} \cite{darais2019language,barthe2019probabilistic}
We introduce important notation to express independence:
\begin{definition}
%  We write $\vc{\pmf}{x}{y}$ iff $\pmf(\{ x \mapsto 0\}\ |\ \{ y \mapsto 0 \}) =
%  \pmf(\{ x \mapsto 1\}\ |\ \{ y \mapsto 1 \}) = 1$.
  We write $\sep{\pmf}{X}{Y}$ iff for all
    $\store \in \mems(X \cup Y)$ we have
  $\margd{\pmf}{X \cup Y}(\store) =
  \pmf(\store_X) * \pmf(\store_Y)$
\end{definition}

In fact, MPC protocols typically satisfy a \emph{delimited release}
property\cite{XXX}, where messages exchanged remain probabilistically separable
from secrets, with only declassification events (reveals and outputs)
releasing information about honest secrets. A key difference is that while
these declassification events essentially define the policy in delimited
release, the ideal functionality sets the release policy for MPC
passive security, so its necessary to show that declassification events respect
these bounds.
\begin{definition}
  Given $H,C$, a protocol $\prog$ with $\iov(\prog) = S \cup M \cup P \cup O$
  satisfies \emph{delimited release} iff
  $\sep{\progtt(\prog)}{M_C}{S_H}$.
\end{definition}

\subsection{Malicious Security and Robust Declassification}

\begin{definition}[Malicious Correctness]
  A protocol $\prog$ with $\iov(\prog) = S \cup V \cup O$ is \emph{malicious correct} iff
  $
  \forall \adversary, \store \in \mems(S_H) \ .\ \support(\progtt(\prog)(O_H|\store)) \supseteq
    \support(\progtt(\prog,\adversary)(O_H|\store))
  $.
\end{definition}

\begin{definition}[Integrity]
  We say that a protocol $\prog$ with $\iov(\prog) = S \cup V \cup O$ has
  \emph{integrity} iff for all $H$ and $C$ 
  it is malicious correct and for all adversaries $\adversary$,
  if $\store \in \aruns(\prog)$ 
  then there exists $\store' \in \mems(S)$ with $\store_{S_H} = \store'_{S_H} $ and:
    $$
    \condd{\progtt(\prog,\adversary)}{X}{\store_{S_H \cup \cinputs}} =
    \condd{\progtt(\prog)}{X}{\store'}
    $$ 
  where $X \defeq (\houtputs \cup O_H) \cap \dom(\store)$. 
\end{definition}

\begin{definition}[Integrity]
  We say that a protocol $\prog$ with $\iov(\prog) = S \cup V \cup O$ has
  \emph{integrity} iff for all $H$ and $C$ 
  it is malicious correct and for all adversaries $\adversary_1,\adversary_2$
  and memories $\store \in \mems((S \cup O)_H)$:
    $$
    \condd{\progtt(\prog,\adversary_1)}{\houtputs}{\store} =
    \condd{\progtt(\prog,\adversary_2)}{\houtputs}{\store}
    $$
\end{definition}

\begin{lemma}
  \label{lemma-integrity}
  If a protocol $\prog$ with $\iov(\prog) = S \cup V \cup O$ has
  integrity, then for all adversaries $\adversary$
  and memories $\store \in \mems((S \cup O)_H)$:
  $$
  \condd{\progtt(\prog,\adversary)}{\houtputs}{\store} =
  \condd{\progtt(\prog)}{\houtputs}{\store}
  $$
\end{lemma}

\begin{theorem}
  If a protocol satisfies noninterference modulo output with
  integrity, then it is malicious secure.
\end{theorem}

\begin{proof}
  Let $\prog$ be some protocol with passive security and integrity
  where $\iov(\prog) = S \cup V \cup O$, and let $\adversary$ be some
  adversary.  Given some $\store\in S_H$, by definition of malicious
  correctness the simulator can extract inputs $\store' \in (S_C)$
  from $\adversary$ where $\idealf(\store \cup \store')$ is equivalent
  to the real world outputs. These can then be used in a simulated run
  of $\prog$ along with $\store \in S_H$ to generate a distribution
  of corrupt views of $\margd{\progtt(\prog)}{\houtputs}$.
  By assumption of integrity and Lemma \ref{lemma-integrity},
  this is the same as the real-world distribution of corrupt views
  in the presence of $\adversary$. 
\end{proof}

\begin{definition}
  An adversary $\adversary$ is \emph{passive} if it runs $\prog$ honestly,
  except for possible substitution of corrupt input secrets.
\end{definition}

\begin{lemma}
  Protocols with only passive adversaries have integrity.
\end{lemma}

\begin{definition}[Robust Declassification]
  A protocol satisfies \emph{robust declassification} iff it has integrity and
  satisfies gradual release. 
\end{definition}

\begin{theorem}
  Passive security with robust declassification implies malicious security.
\end{theorem}
