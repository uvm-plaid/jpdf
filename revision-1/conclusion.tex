\section{Conclusion and Future Work}
\label{section-conclusion}

We have developed a new probabilistic programming language called
$\minifed$ for defining synchronous distributed MPC protocols, and a new
metaprogramming language $\metaprot$ that is able to express logical
protocol components such as circuit gates and that dynamically
generates $\minifed$ protocols. We have developed \emph{program
distributions}, a formalism for expressing dependencies between
secrets and views in protocols and precisely quantifying allowable
information leakage in terms of protocol outputs. 
A hyperproperty called $\NIMO$, predicated on program distributions,
was shown to guarantee MPC passive security. 

In the $\metaprot$-to-$\minicat$ model we've developed an automated
technique, aka \emph{certification} method, for automatically
verifying MPC security properties such as $\NIMO$ and other supporting
properties of protocol components.  While certification has high
complexity, a compositional approach allows certification of program
components in isolation.  This enables a semi-automated method for
large circuits where we (1) define certification for components, (2)
certify components, (3) prove $\NIMO$ based on the certifiedness of
components. We developed novel methods partially based on separation
to enable (3), and we've illustrated our approach by certifying GMW
and Yao's Garbled Circuit libraries in a manner that guarantees
security in arbitrarily large circuits. Certification tests can also
be re-used to validate circuit extensions.

We believe this line of research is promising for future work,
including extensions to malicious security, and to larger fields and
arithmetic circuits. HPC optimization of program distribution
calculations, to support analysis of larger circuit components, is
another interesting direction for future work.
