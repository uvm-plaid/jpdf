\section{Share Types}
\label{section-sty}

Our approach to static confidentiality analysis is to enforce the use
of sound techniques in semi-homomorphic encryption in protocols.  As
discussed in the example in Section \ref{section-smt-example},
protocols integrate interactive elements, such as the sharing of
one-time-pad (OTP) encrypted secrets, and noninteractive elements,
such as addition (of shares). To comport with this we introduce
distinct judgement forms for interactive and noninteractive protocol
elements. Our type system as presented will assume 2-client protocols,
but this is for clarity of presentation and the analysis can be
generalized to more parties.  So in this Section we define the
federation as $\setit{1,2}$ (client 1 and client 2).

Our confidentiality analysis mainly considers protocols in terms of
their SMT interpretations as described in Section \ref{section-smt}.
We introduce a \emph{share type} language $\sty$, classifying
\emph{share terms}. We represent share terms using standard notation
\cite{10.1007/978-3-030-68869-1_3}. A share term $[\phi_1,\phi_2]$
represents a secret-shared value where client 1 has $\phi_1$ and
client 2 has $\phi_2$, and this shared value can be reconstructed
as a public value $\pubx{\phi}$.
$$
\begin{array}{rcl@{\hspace{10mm}}l}
  \sty &::=& \sharety \mid \pubty \mid \outty & \textit{share types} \\
  \stt &::=& [\phi,\phi] \mid \pubx{\phi} & \textit{share terms} 
\end{array}
$$
\paragraph{Share generation and reconstruction}
A \emph{secret sharing scheme} includes functions
$(\genshares,\recon)$ that generate and reconstruct shares,
respectively. The $\genshares$ function takes some
input $\phi$ and randomness $\rx{x}{w}$ and produces
two shares (or as many is needed for the number of clients
in the federation): 
$$
\genshares(\phi,\rx{w}{\cid}) = \phi_1,\phi_2
$$
and:
$$
\recon(\genshares(\phi,\rx{w}{\cid})) = \phi
$$
In order for $\genshares$ to be \emph{sound} with regard to
confidentiality, we require
that both of $\phi_1$ and $\phi_2$ are marginally independent
of $\phi$, and also of any other values in the protocol.
This entails the requirement that $\rx{w}{\cid}$ is
consumed by $\genshares$, e.g., as an OTP.
We formally characterize soundness of $\genshares$
in our protocol model in Definition \ref{def-genshares}.
We emphasize that our type system is agnostic with respect
to the definition of $\genshares$ and $\recon$ modulo
soundness-- we use an additive shares encoding in this work,
but the analysis could apply instead to, e.g., Shamir
sharing.

More generally, when we ascribe the type $[\phi_1,\phi_2] : \sharety$,
this means that $[\phi_1,\phi_2]$ are shares of their reconstruction
$\recon(\phi_1,\phi_2)$, noting that shares can be obtained either by
$\genshares$ or by noninteractive homomorphic protocols.  Likewise,
$\pubx{\phi} : \outty$ means that $\phi$ has been publicly
\emph{revealed} via broadcast shares, and our type system requires
that $\phi \eop \recon(\phi_1, \phi_2)$ with $[\phi_1,\phi_2] :
\sharety$. This implies that the process of revealing $\phi$ reveals
nothing besides $\phi$.  The ascription $\pubx{\phi} : \pubty$ is
similar, with the additional meaning that $\phi$ is a publicly-known
value independent of any client inputs. This includes known constant
values used in protocols, and more interestingly \emph{secure
openings}. Secure openings are commonly used to interactively compute
public values that are masked with OTPs secret-shared with clients in
initialization protocols, as for example in Beaver Triple
multiplication which we discuss in Section \ref{section-examples}.

\stylocalfig

We proceed by formally describing the type analysis, then
discussing an illustrative example, and finally by demonstrating share
type soundness in Theorem \ref{theorem-sty}. This result shows that
our analysis enforces noninterference modulo output, in contrast to
the previous system for $\minicat$ that only enforce a gradual release
property \cite{skalka-near-esop25}. By connecting this analysis with
protocol correctness properties enforced by the Hoare logic discussed
in Section \ref{section-smt}, the output can also be connected to an
intended ideal functionality.

\subsection{Non-Interactive Typing Rules}

Our non-interactive typing rules are presented in Figure
\ref{fig-stylocal}. Judgements are of the form $\Gamma, \eqs \vdash
\stt : \sty$, where $\Gamma$ is a set of \emph{share type bindings} of
the form $\stt : \sty$. In the context of typing a protocol $\prog$
soundness requires that the constraint $\eqs$ must be entailed by
$\toeq{\prog}$. The syntax-directed rules are all known semi-homomorphic operations,
including addition of shares (\TirName{HEAdd}), and both addition and
multiplication of shares with a public value (\TirName{HEAddPub} and
\TirName{HEMultPub}, respectively) \cite{10.1007/978-3-030-68869-1_3}.
These rules also leverage SMT entailment in the \TirName{ShareEntails}
and \TirName{PubEntails} rules, meaning these operations are
allowable up to algebraic equivalence in the context of a protocol.

\subsection{Interactive Typing Rules} 

The interactive share typing rules are defined in Figure
\ref{fig-styinteractive}, with judgements of the form $\eqs \vdash
\Gamma_1,R_1, \stredx \Gamma_2,R_2$.  Each such judgement establishes
a sound type binding in $\Gamma_2$ for interactively-defined
variables-- that is, unicast and broadcast messages exchanged between
clients-- assuming the bindings in $\Gamma_1$. In addition to type
bindings, judgments track randomness used by $\genshares$, allowing us
to use type linearity to enforce one-time usage similar to previous
related work \cite{darais2019language}, where any randomness used in
the typing is not in $R_1$ and added to $R_2$. To this end, in these
rules $\uplus$ is disjoint union.

The \TirName{GenShares} rule generates the binding $[x_1,
  x_2] : \sharety$ if the term is a secret sharing as defined
by $\genshares$, and adds $\rx{w}{\cid}$ to the set of variables
consumed by $\genshares$, ensuring one-time usage. The
\TirName{Reveal} rule generates the binding $\pubx{\recon(\px{w_1},\px{w_2})} :
\outty$ with the requirement that $[\px{w_1},\px{w_2}]$ are typable as
shares. The \TirName{SecOpen} rule is similar except it allows the
assertion $\pubx{\recon(\px{w_1},\px{w_2})} : \pubty$ if we can additionally
show that the shared value is masked by the use of random value
$\rx{w}{\oid}$ as an OTP. Here we are positing a third party $\oid$
that secret-shares $\rx{w}{\oid}$ as part of an initialization
protocol. We illustrate secure openings in practice at more length
in Section \ref{section-examples}, though we note that appropriate
properties of initialization protocols can be encoded in $\eqspre$.

We write $\stredx^*$ to denote the transitive
closure of $\stredx$. For any protocol $\prog$, the goal of the
type analysis is to show that all interactive messages can be classified--
that is, to show that $\eqspre \wedge \toeq{\prog} \vdash
\varnothing,\varnothing \stredx^* \Gamma,R$ where all
interactive messages are classified in $\Gamma$. 

\styinteractivefig

\subsection{Instantiating $\genshares$}

It is well-known that adding or subtracting a sample from a uniform
distribution in a finite field yields a value in a uniform
distribution, meaning that samples can be used as OTPs with
perfect secrecy \cite{barthe2019probabilistic,darais2019language}.
So we can define:
\begin{eqnarray*}
  \genshares(\phi,\rx{w}{\cid}) &\defeq& \phi - \rx{w}{\cid}, \rx{w}{\cid}\\
  \recon(\phi_1,\phi_2) &\defeq& \phi_1 \fplus \phi_2
\end{eqnarray*}
where we require that $\genshares$ consumes the random value
$\rx{w}{\cid}$ as an OTP. 

This approach highlights an important interaction between our type
system and SMT entailment as defined by the $\TirName{ShareEntails}$
rule. In particular there are other methods for encrypting values in MPC
protocols that can be shown to be algebraically equivalent
to this fundamental schema. For example, in Yao's Garbled Circuits
(YGC) in $\mathbb{F}_{2}$, the ``garbler'' represents encrypted wire
values as a random sample (for 0) or its negation (for 1), and shares
its secret input encoding with the ``evaluator'' by using
it as a selection bit. That is, assuming that client 2 is the garbler and
client 1 is the evaluator, we can define:
$$
\xassign{\mx{w}{1}}{\mux{\secret{w}}{\flip{w}}{\neg\flip{w}}}{2}
$$
where for all $\be_1,\be_2,\be_3$ (with $\neg$ denoting negation in $\mathbb{F}_2$):
$$
\mux{\be_1}{\be_2}{\be_3} \defeq (\neg\be_1 \ftimes \be_2) \fplus (\be_1 \ftimes \be_3)
$$
and letting this protocol be $\prog$ the following is valid:
\begin{mathpar}
%\toeq{\xassign{\mx{w}{1}}{\OT{\secret{w}}{1}{\neg\flip{w}}{\flip{w}}}{2}}
   \toeq{\prog} \models \mx{w}{1} \eop \sx{w}{2} \fminus \rx{w}{2}
\end{mathpar}
For this reason SMT entailment is also an important enabling technology
for the interactive rules. 

\subsection{Share Typing Example} To illustrate details of the
type system we return to the example of additive sharing in the context
of a 2-client protocol. While the ideal functionality of this simple protocol
precisely reveals input secrets, it serves as an understandable example
of the type system mechanics.
We first focus on initial secret sharing steps--
let $\prog$ be the following protocol in $\fieldp{2}$ (noting
that $-$ is equivalent to xor in $\fieldp{2}$):
$$
\begin{array}{lll}
  \elab{\mesg{s1}}{2} := \elab{(\secret{1} \fminus \locflip)}{1}; \quad
  \elab{\mesg{s2}}{1} := \elab{(\secret{2} \fminus \locflip)}{2} 
\end{array}
$$
By $\TirName{GenShares}$ we can then assert:
$$
\toeq{\prog} \vdash \varnothing,\varnothing \stredx
\setit{[\mx{s2}{1}, \elab{\locflip}{2}] : \sharety,
  [\elab{\locflip}{1}, \mx{s1}{2}] : \sharety}, \setit{\elab{\locflip}{1},\elab{\locflip}{2}}
$$
Suppose the protocol goes on to reveal the negation of the
sum (xor-ing) of the inputs:
$$
\begin{array}{l}
  \px{1} := \elab{\mesg{s2} + \locflip + 1}{1};\quad \px{2} := \elab{\mesg{s1} + \locflip}{2};\\
  \out{1} := \elab{\px{1} + \px{2}}{1};\quad \out{2} := \elab{\px{1} + \px{2}}{2}
\end{array}
$$
Letting this extended protocol be $\prog'$, and letting:
$$
\begin{array}{rcl}
\Gamma &\defeq& \setit{[\mx{s2}{1}, \elab{\locflip}{2}] : \sharety,
  [\elab{\locflip}{1}, \mx{s1}{2}] : \sharety}\\
R &\defeq&  \setit{\elab{\locflip}{1},\elab{\locflip}{2}}
\end{array}
$$
observe we can then obtain share typings for noninteractive components for
addition of shares by \TirName{Shared} and \TirName{HeAdd}:
$$
\Gamma, \toeq{\prog'} \vdash
[\mx{s2}{1} + \elab{\locflip}{1}, \mx{s1}{2} + \elab{\locflip}{2}] : \sharety
$$
and hence for negation by \TirName{HeAddPub}:
$$
\Gamma, \toeq{\prog'} \vdash
[\mx{s2}{1} + \elab{\locflip}{1} + 1, \mx{s1}{2} + \elab{\locflip}{2}] : \sharety
$$
This implies:
$$
\toeq{\prog'} \vdash \varnothing,\varnothing \stredx
\Gamma \cup \setit{\pubx{\px{1} + \px{2}} : \outty}, R 
$$
Hence by \TirName{PubEntails} we have:
$$
\begin{array}{l}
\Gamma \cup \setit{\pubx{\px{1} + \px{2}} : \outty}, \toeq{\prog'} \vdash
\pubx{\out{1}} : \outty \\
\Gamma \cup \setit{\pubx{\px{1} + \px{2}} : \outty}, \toeq{\prog'} \vdash
\pubx{\out{2}} : \outty
\end{array}
$$

\subsection{Share Type Soundness}
\label{section-sty-sound}

Share type soundness (Theorem \ref{theorem-sty}) means that share
typing enforces noninterference modulo output. To prove this we need
to assert probabilistic properties in share typings, including for
functions of protocol variables such as reconstruction of shares. So
we make standard extensions of notation for probability distributions
to express probabilities, and conditional probabilities, of functions
of program variables $\phi$. For this extension we posit that
memories can incorporate variables $x_\phi$ that assign valuations
to functions $\phi$.
\begin{definition}
  We let $\Phi$ range over sets of SMT formulas $\phi$, and define
  $
  \mems(\Phi) \defeq \{ \store \mid \dom(\store) = \setit{\ x_{\phi}\ \mid\ \phi \in \Phi\ } \}
  $.
  Given pmf $\pmf$ and $\Phi$ with $\dom(\pmf) = \bigcup_{\phi \in \Phi} \dom(\phi)$,
  for all $\store \in \mems(\Phi)$ define:
%$$
%\pmf(\phi = v) \defeq \sum_{\store \in \mems(X) \ \text{with}\ \store(\phi) = v} \pmf(\store)
%$$
  $$
  \pmf(\store) \defeq \sum_{\store' \in \mems(X)}
  \begin{cases}
    \pmf(\store') \ \text{if}\ \forall \phi \in \Phi . \store'(\phi) = \store(x_\phi)  \\
    0 \ \text{otherwise}
  \end{cases}
  $$
\end{definition}

\begin{definition}
  Given $\pmf$, the \emph{conditional distribution}
  of $X$ given $\Phi$ where $\bigcup_{\phi \in \Phi} \dom(\phi) \cup X = \dom(\pmf)$ 
  is denoted $\condd{\pmf}{X}{\Phi}$ and defined as follows: 
  $$
  \forall \store_1 \in \mems(X)\ .\ \forall \store_2 \in \mems(\Phi) \ .\ 
  \condd{\pmf}{X}{\Phi}(\store_1 \uplus \store_2) \defeq
  \begin{cases}
    0 \text{\ if\ } \pmf(\store_2) = 0\\
    \pmf(\store_1 \uplus \store_2) / \pmf(\store_2) \text{\ otherwise}
  \end{cases}
  $$
\end{definition}

Now we define what we mean by soundness of $\genshares$. In essence
we mean perfect secrecy \cite{barthe2019probabilistic} in the context
of protocols where multiple encodings can occur. These
conditions are obviously met by, e.g., ``vanilla'' additive or Shamir
shares which generate shares in uniform marginal distributions.
\begin{definition}
  \label{def-genshares}
  We say that $\genshares$ is \emph{sound} iff given any $\prog$ with
  $\vars(\prog) = S \cup V \cup O \cup R$
  and $\phi$ and $\rx{w}{\cid} \not \in R$ and $\genshares(\phi,\rx{w}{\cid}) =
  \phi_1,\phi_2$ we have for all $a \in \setit{1,2}$:
  $$
  \sep{\progd(\prog)}{(S \cup V \cup O \cup \setit{\phi})}{\setit{\phi_a}}
  $$
\end{definition} 

Next we define a well-formedness property of tuples $\prog,\Gamma,R$
intuitively meaning that all the share types in $\Gamma$ are sound with respect
to $\prog$, and consumed randomness in $R$ as OTPs for initial sharing.
This is a representation invariant that we will show is preserved by
typing, and under closure conditions (Definition \ref{def-sty-closed})
the invariant guarantees confidentiality (Theorem \ref{theorem-OK}).
Property (i) ensures that views do not reveal information about secrets
consistent with noninterference modulo output.
Property (ii) ensures a key property of sharing-- that individual shares
don't reveal more about secrets than their reconstruction does. Property
(iii) ensures that public values are independent of outputs or secrets. Properties
(iv) and (v) ensure that OTPs used for generating shares or for
secure openings, respectively, are included in $R$.
\begin{definition}
  \label{def-OK}
  Given $\prog,\Gamma,R$ where $\iov(\prog) = S \cup M \cup P \cup O$. For all $a,b\in \setit{1,2}$ with $a \ne b$ let:
  \begin{eqnarray*}
    %\Phi_a^m & \defeq & \setit{ \phi_a \mid ([\phi_1,\phi_2] : \sharety) \in \Gamma} \\
    %\Phi_a^p & \defeq & \setit{ \phi_b \mid (\pubx{\recon(\phi_1,\phi_2)} : \sty) \in \Gamma}\\
    \Phi_a & \defeq & \setit{ \phi_a \mid ([\phi_1,\phi_2] : \sharety) \in \Gamma} \cup
                        \setit{ \phi_b \mid (\pubx{\recon(\phi_1,\phi_2)} : \sty) \in \Gamma} \\% \Phi_a^m \cup \Phi_a^p \\
    \Phi^p & \defeq & \setit{ \recon(\phi_1,\phi_2) \mid (\pubx{\recon(\phi_1,\phi_2)} : \pubty) \in \Gamma}\\
    \Phi^o & \defeq & \setit{ \recon(\phi_1,\phi_2) \mid (\pubx{\recon(\phi_1,\phi_2)} : \outty) \in \Gamma}
  \end{eqnarray*}
  Then we write $\prog,\Gamma,R : {OK}$ iff all of the following conditions
  hold:
  \begin{enumerate}[\hspace{5mm}i.]
  \item For all $a,b \in \setit{1,2}$ we have
    $
    \condd{\progtt(\prog)}{S_{\setit{\cid_a}}}{S_{\setit{\cid_b}} \cup \Phi^o}
    = 
    \condd{\progtt(\prog)}{S_{\setit{\cid_a}}}{S_{\setit{\cid_b}} \cup  \Phi_a \cup \Phi^o}
    $.
  \item For all $a,b \in \setit{1,2}$ and $([\phi_1,\phi_2] : \sharety) \in \Gamma$ we have:
    %$\condsep{\progd(\prog)}{\setit{\recon(\phi_i^1,\phi_i^2)}}{\setit{\phi_i^b}}{\recon(\phi_i^1,\phi_i^2)}$
    $$\condd{\progtt(\prog)}{S_{\setit{\cid_a}}}{S_{\setit{\cid_b}} \cup  \Phi_a \cup \Phi^o \cup \setit{\recon(\phi_i^1,\phi_i^2)}} =
    \condd{\progtt(\prog)}{S_{\setit{\cid_a}}}{S_{\setit{\cid_b}}  \cup  \Phi_a \cup \Phi^o \cup \setit{\recon(\phi_i^1,\phi_i^2), \phi_i^b}}$$
  \item $\sep{\progd(\prog)}{(S \cup O)}{\Phi^p}$
  \item  For all $a,b \in \setit{1,2}$ and if
    $([x_1,x_2] : \sharety) \in \Gamma$ with $x_a \in \imesgs{M}{\setit{b}}{\setit{a}}$ then
    $\genshares(\phi,\rx{w}{\cid}) = \phi_1,\phi_2$
    and $\toeq{\prog} \models x_1 \eop \phi_1 \wedge x_2 \eop \phi_2$ 
    for some $\phi$ and $\rx{w}{\cid} \in R$.
  \item For all $\phi \in \Phi^p$ if $\toeq{\prog} \models \phi \eop \phi' + \rx{w}{\cid_{\oid}}$ for some $\phi'$
    and $\rx{w}{\cid_{\oid}}$ then $\rx{w}{\cid_{\oid}} \in R$.
  \end{enumerate} 
\end{definition}
%\begin{definition}
%  \label{def-OK}
%  Given $\prog,\Gamma,R$ where $\secrets(\prog) = S$ and:
%  $$
%  \Gamma = \setit{[\phi_1^1,\phi_1^2] : \sharety, \ldots, [\phi_n^1,\phi_n^2] : \sharety}
%  \cup \setit{\pubx{\phi_1} : \pubty,\ldots,\pubx{\phi_j} : \pubty} \cup
%  \setit{\pubx{\phi^o_1} : \outty,\ldots,\pubx{\phi^o_k} : \outty}
%  $$
%  Then we write $\prog,\Gamma,R : {OK}$ iff all of the following conditions
%  hold:
%  \begin{enumerate}[\hspace{5mm}i.]
%  \item For all $a,b \in \setit{1,2}$ we have
%    $
%    \condd{\progtt(\prog)}{S_{\setit{\cid_a}}}{S_{\setit{\cid_b}} \cup \setit{\phi^o_1,\ldots,\phi^o_k}}
%    = 
%    \condd{\progtt(\prog)}{S_{\setit{\cid_a}}}{S_{\setit{\cid_b}} \cup \setit{\phi_1^b,\ldots,\phi_n^b} \cup
%      %\setit{\phi_1,\ldots,\phi_j}
%      \setit{\phi^o_1,\ldots,\phi^o_k}}
%    $.
%  \item For all $a,b \in \setit{1,2}$ and $0 < i \le n$ we have
%    %$\condsep{\progd(\prog)}{\setit{\recon(\phi_i^1,\phi_i^2)}}{\setit{\phi_i^b}}{\recon(\phi_i^1,\phi_i^2)}$
%    $\condd{\progtt(\prog)}{S_{\setit{\cid_a}}}{\setit{\recon(\phi_i^1,\phi_i^2)}} =
%    \condd{\progtt(\prog)}{S_{\setit{\cid_a}}}{\setit{\recon(\phi_i^1,\phi_i^2), \phi_i^b}}$
%  \item $\sep{\progd(\prog)}{(S \cup O)}{\setit{\phi_1,\ldots,\phi_j}}$
%  \item  For all $a,b \in \setit{1,2}$ and $0 < i \le n$, if
%    $([x_1,x_2] : \sharety) \in \Gamma$ with $\genshares(\phi,\rx{w}{\cid}) = \phi_1,\phi_2$
%    and $\toeq{\prog} \models x_1 \eop \phi_1 \wedge x_2 \eop \phi_2$ 
%    for some $\phi$ and $\rx{w}{\cid}$ then $\rx{w}{\cid} \in R$.
%  \item For all $0 < i \le j$, if $\toeq{\prog} \models \phi_i \eop \phi + \rx{w}{\cid_{\oid}}$ for
%    some $\phi$  and $\rx{w}{\cid_{\oid}}$ then $\rx{w}{\cid_{\oid}} \in R$.
%  \end{enumerate} 
%\end{definition}

%\begin{definition}
%  \label{def-genshares}
%  We say that $\genshares$ is \emph{sound} iff given any $\prog,\Gamma,R : {OK}$
%  and $\phi$ and $\rx{w}{\cid} \not \in R$ with $\genshares(\phi,\rx{w}{\cid}) =
%  \phi_1,\phi_2$ and $\toeq{\prog} \models x_1 \eop \phi_1 \wedge x_2 \eop \phi_2$
%  with $\secrets(\prog) = S$ and:
%  $$
%  \Gamma = \setit{[\phi_1^1,\phi_1^2] : \sharety, \ldots, [\phi_n^1,\phi_n^2] : \sharety}
%  \cup \setit{\pubx{\phi_1} : \pubty,\ldots,\pubx{\phi_j} : \pubty} \cup
%  \setit{\pubx{\phi^o_1} : \outty,\ldots,\pubx{\phi^o_k} : \outty}
%  $$
%  we have for all $a,b \in \setit{1,2}$:
%  \begin{enumerate}[\hspace{5mm}i.]
%    \item $\condd{\progtt(\prog)}{S_{\setit{\cid_a}}}{S_{\setit{\cid_b}} \cup \setit{\phi_1^b,\ldots,\phi_n^b} \cup
%      %\setit{\phi_1,\ldots,\phi_j}
%      \setit{\phi^o_1,\ldots,\phi^o_k}}
%    =
%    \condd{\progtt(\prog)}{S_{\setit{\cid_a}}}{S_{\setit{\cid_b}} \cup \setit{\phi_1^b,\ldots,\phi_n^b,x_b} \cup
%      %\setit{\phi_1,\ldots,\phi_j}
%      \setit{\phi^o_1,\ldots,\phi^o_k}}$
%    \item $\condd{\progtt(\prog)}{S_{\setit{\cid_a}}}{\setit{\recon(x_1,x_2)}} =
%      \condd{\progtt(\prog)}{S_{\setit{\cid_a}}}{\setit{\recon(x_1,x_2), x_b}}$
%    \end{enumerate}
%\end{definition} 

Next we show that, given well-formed typing environments, share type judgements preserve
the important properties (i-ii) in Definitions \ref{def-OK}. This result follows from
known semi-homomorphic properties of non-interactive operations that are syntactically
verified in typing judgements.
\begin{lemma}
  \label{lemma-sty-noninteractive-sound}
  Given $\prog,\Gamma,R : {OK}$ where $\secrets(\prog) = S$. For all $a,b\in \setit{1,2}$ with $a \ne b$ let:
  \begin{eqnarray*}
    %\Phi_a^m & \defeq & \setit{ \phi_a \mid ([\phi_1,\phi_2] : \sharety) \in \Gamma} \\
    %\Phi_a^p & \defeq & \setit{ \phi_b \mid (\pubx{\recon(\phi_1,\phi_2)} : \sty) \in \Gamma}\\
    \Phi_a & \defeq & \setit{ \phi_a \mid ([\phi_1,\phi_2] : \sharety) \in \Gamma} \cup
                        \setit{ \phi_b \mid (\pubx{\recon(\phi_1,\phi_2)} : \sty) \in \Gamma} \\% \Phi_a^m \cup \Phi_a^p \\
    \Phi^o & \defeq & \setit{ \recon(\phi_1,\phi_2) \mid (\pubx{\recon(\phi_1,\phi_2)} : \outty) \in \Gamma}
  \end{eqnarray*} 
  and also suppose $\Gamma, \toeq{\prog} \vdash [\phi_1,\phi_2] : \sharety$.
  Then for all $a,b \in \setit{1,2}$ with $a \ne b$ the following conditions hold:
    \begin{enumerate}[\hspace{5mm}i.]
    \item
    $
    \condd{\progtt(\prog)}{S_{\setit{\cid_a}}}{S_{\setit{\cid_b}} \cup \Phi^o}
    = 
    \condd{\progtt(\prog)}{S_{\setit{\cid_a}}}{S_{\setit{\cid_b}} \cup  \Phi_a \cup \setit{\phi_a} \cup \Phi^o}
    $
  \item 
    $\condd{\progtt(\prog)}{S_{\setit{\cid_a}}}{S_{\setit{\cid_b}} \cup  \Phi_a \cup \Phi^o \cup \setit{\recon(\phi_1,\phi_2)}} =
    \condd{\progtt(\prog)}{S_{\setit{\cid_a}}}{S_{\setit{\cid_b}}  \cup  \Phi_a \cup \Phi^o \cup \setit{\recon(\phi_1,\phi_2), \phi_b}}$
    \end{enumerate}
\end{lemma}
\begin{proof}
  The proof is by induction on the derivation of $\Gamma, \toeq{\prog} \vdash [\phi^1,\phi^2] : \sharety$.
  In the base case where the judgement follows by the \TirName{Shared} rule, the result
  follows trivially by Definition \ref{def-OK}. In cases \TirName{HEAdd}, \TirName{HEAddPub}, and
  \TirName{HEMultPub} the result follows by Definition \ref{def-OK} and known
  semi-homomorphic properties of secret sharing \cite{10.1007/978-3-030-68869-1_3}.
  Case \TirName{ShareEntails} follows by correctness of the SMT interpretation of
  protocols as in Theorem \ref{theorem-toeq}.
\end{proof}

Next, we show that interactive typing rules preserve well-formedness of type
environments. This Lemma follows by Lemma \ref{lemma-sty-noninteractive-sound},
and the security of share generation using OTPs as articulated in
Definition \ref{def-genshares}-- that is, shares
generated by $\genshares$ are in uniform marginal distributions. 
\begin{lemma}
  \label{lemma-OK}
  Given $\prog,\Gamma_1,R_1  : {OK}$
  and $\toeq{\prog}  \vdash \Gamma_1, R_1 \stredx \Gamma_2, R_2$.
  Then $\toeq{\prog}, \Gamma_2, R_2 : {OK}$.
\end{lemma}
\begin{proof}
  The proof follows by case analysis on  $\toeq{\prog}  \vdash \Gamma_1, R_1 \stredx \Gamma_2, R_2$.
  In case \TirName{GenShares} the result follows by Definition \ref{def-genshares},
  which satisfies relevant conditions (i-ii,iv) of Definition \ref{def-OK}. In case
  \TirName{SecOpen} the public value must be reconstructed shares, so Lemma
  \ref{lemma-sty-noninteractive-sound} enforces relevant conditions (i-ii) of Definition
  \ref{def-OK}, and the requirement that the public value be masked with an OTP satisfies conditions
  (iii) and (iv) of Definition \ref{def-OK}. In case \TirName{Reveal} since
  the output value must be reconstructed shares, Lemma \ref{lemma-sty-noninteractive-sound}
  satisfies relevant case (ii) of  Definition \ref{def-OK}. 
\end{proof}

In addition to typing judgements, in order to show that a protocol is
secure, we need to classify all corrupt views as shares, all
public values as secure openings, and all outputs as reconstructed
shares by adding them to $\Gamma$. We capture this formally with the following notion of
type environment closure.
\begin{definition}
  \label{def-sty-closed}
  $\Gamma$ is \emph{closed for $\prog$} iff for all $H,C$,
  for all $x \in \houtputs$ there exists $\phi$ such that either
  $([x,\phi] : \sharety) \in \Gamma$ if $C = \setit{1}$
  or  $([\phi,x] : \sharety) \in \Gamma$ if $C = \setit{2}$,
  and for all $\out{\cid} \in O$ we have $(\pubx{\out{\cid}} : \outty) \in \Gamma$.
\end{definition}
Now, the following is a consequence of Definitions \ref{def-OK} and \ref{def-sty-closed}, and then
our main Theorem \ref{theorem-sty} follows by Lemma \ref{lemma-OK} and Theorem \ref{theorem-OK}.
\begin{theorem}
  \label{theorem-OK}
  If $\prog,\Gamma,R : {OK}$ and $\Gamma$ is closed
  for $\prog$, then $\prog$ satisfies noninterference modulo output. 
\end{theorem}
\begin{proof}
  Let $\iov(\prog) = S \cup V \cup O$ and assume some partitioning $H,C$
  with $H = \setit{b}$ and $C = \setit{a}$ for $a,b \in \setit{1,2}$.
  %For any $\out{\cid} \in O$ with $\Gamma,\toeq{\prog} \vdash \pubx{\out{\cid}} : \outty$
  %we can assume wlog that $(\out{\cid} : \outty) \in \Gamma$, since we can
  %add the binding to $\Gamma'$ with  $\prog,\Gamma',R  : {OK}$ by Lemma
  %\ref{lemma-OK}.
  Let $\Phi_a$ and $\Phi^o$ be as defined in Definition \ref{def-OK}. 
  By Definition \ref{def-sty-closed} we have that $x \in \houtputs$
  implies $x \in \Phi_a$ and $(\out{\cid} : \outty) \in \Gamma$ for all
  $\out{\cid} \in O$. 
  Therefore by Definition \ref{def-OK} (and marginalizing away
  extraneous $(\stt : \sty) \in \Gamma$) we have:
  $$
  \condd{\progd(\prog)}{S_H}{S_C \cup O} = \condd{\progd(\prog)}{S_H}{S_C \cup \houtputs \cup \Phi^o \cup O}
  $$
  By definition of $\TirName{Output}$ it is also the case that for all
  $\out{\cid} \in O$ we have $\toeq{\prog} \models o \eop \phi$ for some
  $\phi \in \Phi^o$. Therefore we have:
  $$
  \condd{\progd(\prog)}{S_H}{S_C \cup O} = \condd{\progd(\prog)}{S_H}{S_C \cup \houtputs \cup O}
  $$
  which is noninterference modulo output by Definition \ref{definition-NIMO}.
\end{proof}

\begin{theorem}
  \label{theorem-sty}
  If $\toeq{\prog} \vdash \varnothing,\varnothing \stredx \Gamma,R$ and $\Gamma$ is closed
  for $\prog$, then $\prog$ satisfies noninterference modulo output. 
\end{theorem}

\begin{proof}
  By Lemma \ref{lemma-OK} we have that $\prog,\Gamma,R : {OK}$, so the result follows
  by Theorem \ref{theorem-OK}.
\end{proof}

% DEPRECATED
\begin{comment}
\begin{definition}
  We write $\eqs,\Gamma,R : {OK}$ iff both of the following conditions hold:
  \begin{enumerate}[\hspace{5mm}i.]
  \item For all $([\phi_1,\phi_2] : \sharety) \in \Gamma$,
    we have $\eqs \models \phi_1 \eop \phi - \rx{x}{\cid'} \wedge
    \phi_2 \eop \rx{y}{\cid'}$ for
    some $\phi$ and unique $\rx{x}{\cid'} \in R$, with $\cid \ne \cid'$.
  \item For all $(\pubx{\phi} : \pubty) \in \Gamma$,
    we have $\eqs \models \phi \eop \phi' - \rx{x}{\oid}$ for
    some $\phi'$ and unique $\rx{x}{\oid} \in R$.
  \end{enumerate}
\end{definition}

\begin{lemma}
  Given $\eqs,\Gamma,R  : {OK}$.
  and $\eqs  \vdash \Gamma,R \stredx R',\Gamma'$.
  Then $\eqs, \Gamma',R' : {OK}$.
\end{lemma}

Next, we make a simple definition to formalize what we mean by the interactive
variables in the share type environment $\Gamma$.
\begin{definition} Define $\ov(\Gamma)$ as follows:
\begin{eqnarray*}
\ov(\varnothing) &=& \varnothing \\[-1.5mm]
\ov(\setit{[\mx{z}{\cid}, \rx{y}{\cid'}] : \sharety} \cup \Gamma) &=& \setit{\mx{z}{\cid}} \cup \ov(\Gamma)\\[-1.5mm]
\ov(\setit{[\px{x} + \px{y}] : \pubty} \cup \Gamma) &=& \setit{\px{x},\px{y}} \cup \ov(\Gamma)\\[-1.5mm]
\ov(\setit{[\out{\cid}] : \outty} \cup \Gamma) &=& \setit{\out{\cid}} \cup \ov(\Gamma)
\end{eqnarray*}
\end{definition}
Given the above, we can formulate our key Lemma about noninteractive typing
judgements, which tells us that individual shares in isolation don't reveal anything
about their reconstruction. This result follows by Theorem \ref{theorem-toeq}
and known homomorphic properties of secret sharing schemes. 
\begin{lemma}
  \label{lemma-sty-noninteractive-sound}
  Assume given $\eqspre \wedge \toeq{\prog}, \Gamma,R: {OK}$  and
  $\iov(\prog) = S \cup V \cup O$
  and $\Gamma, \eqspre \wedge \toeq{\prog} \vdash [\phi_1,\phi_2] : \sharety$.
  Then, letting $X \defeq \ov(\Gamma) \cap \realviews{\setit{\cid_1}}{\setit{\cid_2}}$ for
  $\cid_1 \ne \cid_2$:
  $$
  \condd{\progtt(\prog)}{S_{\setit{\cid_1}}}{S_{\setit{\cid_2}}  \cup X \cup \setit{\phi_1 + \phi_2}}
  = 
  \condd{\progtt(\prog)}{S_{\setit{\cid_1}}}{S_{\setit{\cid_2}} \cup X \cup \setit{\phi_1 + \phi_2, \phi_1}}
  $$
\end{lemma}

Based on the preceding Lemma, we can show that noninteference with respect
to classified interactive variables is maintained by typing judgements. 
\begin{lemma}
  \label{lemma-styinteractive}
  Assume given $\eqspre \wedge \toeq{\prog}, \Gamma_1,R_1: {OK}$  and
  $\iov(\prog) = S \cup V \cup O$
  and $\eqspre \wedge \toeq{\prog} \vdash \Gamma_1,R_1 \stredx \Gamma_2,R_2$.
  Letting $X_1 \defeq \ov(\Gamma_1) \cap \realviews{\setit{\cid_1}}{\setit{\cid_2}}$,
  for $\cid_1 \ne \cid_2$ assume also:
  $$
  \condd{\progtt(\prog)}{S_{\setit{\cid_1}}}{S_{\setit{\cid_2}}  \cup X_1 \cup O}
  = 
  \condd{\progtt(\prog)}{S_{\setit{\cid_1}}}{S_{\setit{\cid_2}} \cup O}
  $$
  Then, letting $X_2 \defeq \ov(\Gamma_2) \cap \realviews{\setit{\cid_1}}{\setit{\cid_2}}$,
  we have:
  $$
  \condd{\progtt(\prog)}{S_{\setit{\cid_1}}}{S_{\setit{\cid_2}}  \cup X_2 \cup O}
  = 
  \condd{\progtt(\prog)}{S_{\setit{\cid_1}}}{S_{\setit{\cid_2}} \cup O}
  $$
\end{lemma}
Now we can prove our main Theorem, which follows immediately 
by Lemma \ref{lemma-styinteractive} and induction on $|\Gamma|$
(i.e., the number of bindings in the environment $\Gamma$ obtained
by a complete typing). 
\begin{theorem}[Share Type Soundness]
  \label{theorem-sty}
  Assume given $\prog$ with 
  $\iov(\prog) = S \cup V \cup O$
  and $\eqspre \wedge \toeq{\prog} \vdash \varnothing,\varnothing \stredx^*
  \Gamma,R$ where $\realviews{\setit{\cid_1}}{\setit{\cid_2}} \cup O
  \subseteq \ov(\Gamma)$ for all $\cid_1$, $\cid_2$ with $\cid_1 \ne \cid_2$.
  Then $\prog$ satisfies noninterference modulo output.
\end{theorem}
\end{comment}
