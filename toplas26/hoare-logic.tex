% Prelude Hoare logic counterexample
%
% {x = 0} f(){assert x = 0} {x = 0}
%
% {}g(){f()}{x = 0}

\section{Automated Hoare Logic for $\metaprot$}

\subsection{Corrections to $\metaprot$ Generalized Constraint Forms and Evaluation}

The syntax of $\notg{\eqs}$ in the ESOP paper was too complicated. It should simply be:
$$
\begin{array}{rcl@{\hspace{4mm}}r}
  \notg{\eqs} &::=& e \eop e \mid \notg{\eqs} \wedge \notg{\eqs} \mid \notg{\eqs} \vee \notg{\eqs}
  \mid \notg{\eqs} \impl \notg{\eqs}
\end{array}
$$
The semantics of generalized constraint evaluation can also be greatly simplified, as
follows:
\begin{mathpar}
  \inferrule
      {e_1 \redx \be_1 \\ e_2 \redx \be_2}
      {e_1 \eop e_2 \redx \toeq{\be_1} \eop \toeq{\be_2}}

  \inferrule
      {\notg{\eqs_1} \redx \eqs_1 \\ \notg{\eqs_2} \redx \eqs_2 }
      {\notg{\eqs_1} \wedge \notg{\eqs_2} \redx \eqs_1 \wedge \eqs_2}
\end{mathpar}
Given these changes, we also have some corrections to the $\TirName{Assert}$ and $\TirName{Mesg}$
rules below in the Hoare Logic, which referred to a syntactic form ``$\toeq{e}$'' which now seems meaningless.

The upshot is that evaluation of generalized constraints, mainly in the $\TirName{GenEntails}$
rule, can be defined very directly in terms of evaluation of $\metaprot$ expressions.

\subsection{$\minicat$ Hoare Logic}

\begin{mathpar}
  \inferrule[Assign]
            {}
            {\htrip{\eqs[\toeq{\elab{\be}{\cid_2}}/x]}{\xassign{x}{\be}{\cid_2}}{\eqs}}
  
  \inferrule[Assert]
            {E \models \toeq{\elab{\be_1}{\cid}} \eop \toeq{\elab{\be_2}{\cid}}}
            {\htrip{E}{\elab{\assert{\be_1 = \be_2}}{\cid}}{E}}

  %\reveal{w}{\be}{\cid}
  %
  %\pubout{\cid}{\be}{\cid}
  \inferrule[Seq]
      {\htrip{\eqs_1}{\prog_1}{\eqs_2} \\ \htrip{\eqs_2}{\prog_2}{\eqs_3} }
      {\htrip{\eqs_1}{\prog_1;\prog_2}{\eqs_3}}

  \inferrule[Consequence]
      {\htrip{\eqs_1'}{\prog}{\eqs_2'} \\ \eqs_1 \models \eqs_1' \\  \eqs_2' \models \eqs_2 }
      {\htrip{\eqs_1}{\prog}{\eqs_2}}

  \inferrule[Frame]
      {\htrip{\eqs_1}{\prog}{\eqs_2} \\ \vars(\eqs) \cap \avars(\prog) = \varnothing}
      {\htrip{\eqs_1 \wedge \eqs}{\prog}{\eqs_2 \wedge \eqs}}
\end{mathpar}


Derived rule:
\begin{mathpar}
\inferrule[Assigns]
          {\htrip{\eqs_1}{\prog}{\eqs_2}}% \\ \vars(\eqs_1) \cap \dom(\subn) = \varnothing}
          {\htrip{\eqs_1 \wedge \subn_{\prog}(\eqs)}{\prog}{\eqs_2 \wedge \eqs}}
\end{mathpar}


\subsection{$\metaprot$ Automated Abstract Hoare Logic}

Adding basic types to command function declarations to support generalize pre/post conditions. 

$$
\begin{array}{rcl@{\hspace{4mm}}r}
\tau &::=& \fieldp{p} \mid \mathit{string} \mid \mathit{cid} \mid \{ \flab_1 : \tau_1; \ldots; \flab_n : \tau_n \} & \gdesc{basic types}\\[2mm]
\mathit{fn} &::=& f(y,\ldots,y) \{ e \} \mid  f(y : \tau,\ldots,y : \tau) \{ \cmd \} & \textit{functions}
\end{array}
$$
\medskip

\noindent This is the generalized rule for entailment:

\begin{mathpar}
\inferrule[GenEntails]
          {\mv_1,\ldots,\mv_n = \fresh(\tau_1,\ldots,\tau_n) \\ \peq_1[\mv_1/y_n \cdots \mv_n/y_n] \redx \eqs_1 \\
           \peq_2[\mv_1/y_n \cdots \mv_n/y_n] \redx \eqs_2 \\ \eqs_1 \models \eqs_2}
          {\forall y_1:\tau_1,\ldots,y_n:\tau_n . \peq_1 \models \peq_2 }
\end{mathpar}

\begin{lemma}
  \label{lemma-genentails}
  If $\forall y_1:\tau_1,\ldots,y_n:\tau_n . \peq_1 \models \peq_2$ and
  $\peq_1[\mv_1/y_1]\cdots[\mv_n/y_n] \redx \eqs_1$ and
  $\peq_2[\mv_1/y_1]\cdots[\mv_n/y_n] \redx \eqs_2$, then
  $\eqs_1 \models \eqs_2$.  
\end{lemma}

\begin{mathpar}
  \inferrule[Mesg]
            {}
            {\htrip{\eqtrue}{\xassign{e_1}{e_2}{e_3}}{\eqtrue}, e_1 \eop \toeq{\elab{e_2}{e_3}}}

  %\inferrule[Encode]
  %          {\mx{e_1}{e_2} \redx x \\ \notg{\phi} \redx \phi \\
  %            \eqs \models x \eop \phi\\
  %            \atj{\phi}{R}{\ty}}
  %          {\mtj{\eqcast{\mx{e_1}{e_2}}{\notg{\phi}}}{\eqs}{(x : \ty)}{R}{\varnothing}{\eqs}}
  %
  \inferrule[Assert]
            {}
            {\htrip{\elab{e_1}{e_3} \eop \elab{e_2}{e_3}}{\elab{\assert{e_1 = e_2}}{e_3}}{\eqtrue}, \eqtrue}

  \inferrule[Seq]          
            {\htrip{\peq_1^1}{\cmd_1}{\peq_2^1},\peq_1 \\ \htrip{\peq_1^2}{\cmd_2}{\peq_2^2},\peq_2}
            {\htrip{\peq_1^1 \wedge ((\peq_2^1 \impl \peq_1^2) \vee (\peq_1 \impl \peq_1^2))}{\cmd_1;\cmd_2}{\peq_2^1 \wedge \peq_2^2},\peq_1 \wedge \peq_2}

  \inferrule[Let]
            {\htrip{\peq_1}{\cmd[e/y]}{\peq_2}, \peq}
            {\htrip{\peq_1}{\elet{y}{e}{\cmd}}{\peq_2}, \peq }

  \inferrule[App]
            {\htrip{\peq_1}{f(y_1:\tau_1,\ldots,y_n:\tau_n)}{\peq_2},\peq}
            {\htrip{\peq_1[e_1/y_1 \cdots e_n/y_n]}{f(e_1,\ldots,e_n)}{\peq_2[e_1/y_1 \cdots e_n/y_n]},\peq[e_1/y_1 \cdots e_n/y_n]}

  \inferrule[Fn]
            {\codebase(f) = y_1 : \tau_1, \ldots, y_n : \tau_n, \cmd \\ \htrip{\peq_1}{\cmd}{\peq_2},\peq}
            {\htrip{\peq_1}{f(y_1:\tau_1,\ldots,y_n:\tau_n)}{\peq_2},\peq}
            
  \inferrule[Hardpack]
            {\precond(f) = \peq_1 \\ \postcond(f) = \peq_2 \\
              \htrip{\peq_1'}{f(y_1:\tau_1,\ldots,y_n:\tau_n)}{\peq_2'},\peq \\
              \forall y_1:\tau_1,\ldots,y_n:\tau_n . \peq_1 \models \peq_1'\ \text{and}\ \peq_1 \models \peq \impl \peq_2}
            {\htrip{\peq_1}{f(y_1:\tau_1,\ldots,y_n:\tau_n)}{\peq_2},\true}
\end{mathpar}


\begin{lemma}
  \label{lemma-subn}
  Given $\eqs_1$ and $\subn$ with $\vars(\eqs_1) \cap \dom(\subn) = \varnothing$ we have
  $\eqs_1 \models \subn(\eqs_2)$ iff $\eqs_1 \models \toeq{\subn} \impl \eqs_2$ for
  any $\eqs_2$.
\end{lemma}

\begin{lemma}
  \label{lemma-hardpacksubn}
  Given $\eqs_1$, $\subn_1$, and $\subn_2$ with $\vars(\subn_1) \cap \dom(\subn_2)= \varnothing$
  and $\vars(\eqs) \cap (\dom(\subn_1) \cup \dom(\subn_1)) = \varnothing$ we have
  $\eqs_1 \models \subn_1(\eqs_2)$ implies $\eqs_1 \models \subn_1 \circ \subn_2(\eqs_2)$ for
  any $\eqs_2$.
\end{lemma}

\begin{lemma}[$\metaprot$ Hoare Triple Correctness]
  \label{lemma-htrip}
  Given $\htrip{\peq_1}{\cmd}{\peq_2},\peq$ with $\peq_1 \redx
  \eqs_1$, $\peq_2 \redx \eqs_2$, $\peq \redx \eqs$, and $\cmd \redx
  \prog$. Then given any $\eqs_0$ with $\vars(\eqs_0) \cap
  \avars(\prog) = \varnothing$ and $\eqs_0 \models \eqs_1$ we have
  $\htrip{\eqs_0}{\prog}{\eqs_2}$ is valid and there exists $\subn
  \subseteq \subn_\prog$ such that $\toeq{\subn}$ iff $\eqs$.
\end{lemma}

\begin{proof}
  By structural induction on the derivation of $\cmd \redx \prog$ and
  case analysis on $\cmd$.

  \textit{Case} $\cmd = (\xassign{e_1}{e_2}{e_3})$. In this case by definition
  and inversion of $\redx$ and $\TirName{Assign}$ we have:
  \begin{mathpar}
    e_1 \redx x

    e_2 \redx \be

    e_3 \redx \cid

    \prog = (\xassign{x}{\be}{\cid})

    \peq_1 \redx \true

    \peq_2 \redx \true

    \peq \redx x \eop \toeq{\elab{\be}{\cid}}
  \end{mathpar}
  Now by \TirName{Assign} we have $\htrip{\true}{\xassign{x}{\be}{\cid}}{\true}$ so
  for any $\eqs_0$ we have $\htrip{\eqs}{\xassign{x}{\be}{\cid}}{\true}$ by
  $\TirName{Consequence}$. And $\toeq{\subn_{\xassign{x}{\be}{\cid}}} \defeq
  x \eop \toeq{\elab{\be}{\cid}}$, so this case holds.

  \textit{Case} $\cmd = \elab{\assert{e_1 = e_2}}{e_3}$. In this case by definition we have:
  \begin{mathpar}
    e_1 \redx \be_1

    e_2 \redx \be_2

    e_3 \redx \cid

    \prog = \elab{\assert{\be_1 = \be_2}}{\cid}

    \peq_1 \redx \toeq{\elab{\be_1}{\cid}} \eop \toeq{\elab{\be_2}{\cid}}
    
    \peq_2 \redx \true

    \peq \redx \true
  \end{mathpar}
  But we have $\htrip{\toeq{\elab{\be_1}{\cid}} \eop \toeq{\elab{\be_2}{\cid}}}{\elab{\assert{\be_1 = \be_2}}{\cid}}{\toeq{\elab{\be_1}{\cid}} \eop \toeq{\elab{\be_2}{\cid}}}$ by $\TirName{Assert}$, so
  $\htrip{\toeq{\elab{\be_1}{\cid}} \eop \toeq{\elab{\be_2}{\cid}}}{\elab{\assert{\be_1 = \be_2}}{\cid}}{\true}$ by $\TirName{Consequence}$, and $\toeq{\subn_{\elab{\assert{\be_1 = \be_2}}{\cid}}} \defeq
  \true$, so this case holds.

  \textit{Case} $\cmd = \cmd_1;\cmd_2$. In this case by assumption and
  definition and inversion of $\redx$ and $\TirName{Seq}$ we have:
  \begin{mathpar}
    \cmd_1 \redx \prog_1

    \cmd_2 \redx \prog_2

    \prog = \prog_1;\prog_2

    \peq_1 \redx \eqs_1^1 \wedge ((\eqs_2^1 \impl \eqs_1^2) \vee (\eqs_1 \impl \eqs_1^2))
      
    \peq_2 \redx \eqs_2^1 \wedge \eqs_2^2

    \peq \redx \eqs_1 \wedge \eqs_2
  \end{mathpar}
  and by inversion of the $\TirName{Seq}$ rule we have:
  \begin{mathpar}
    \htrip{\peq_1^1}{\cmd_1}{\peq_2^1},\peq_1

    \htrip{\peq_1^2}{\cmd_2}{\peq_2^2},\peq_2
  \end{mathpar}
  Assume given $\eqs_0$ with $\vars(\eqs_0) \cap \avars(\prog) = \varnothing$ and
  $\eqs_0 \models  \eqs_1^1 \wedge ((\eqs_2^1 \impl \eqs_1^2) \vee (\eqs_1 \impl \eqs_1^2))$.
  We proceed by subcases to show $\htrip{\eqs_0}{\prog_1;\prog_2}{\eqs_2^1 \wedge \eqs_2^2}$.

  \textit{Subcase} $\eqs_0 \models  \eqs_1^1 \wedge (\eqs_2^1 \impl \eqs_1^2)$. By the
  induction hypothesis we have $\htrip{\eqs_0}{\prog_1}{\eqs_2^1}$, and since
  $\vars(\eqs_0) \cap \avars(\prog_1)$ by assumption, by
  \TirName{Frame} we have $\htrip{\eqs_0}{\prog_1}{\eqs_0 \wedge \eqs_2^1}$.
  But then $\eqs_0 \wedge \eqs_2^1 \models \peq_1^2$ by standard reasoning,
  so  $\htrip{\eqs_0}{\prog_1}{\eqs_1^2}$ by \TirName{Consequence}.
  Since also by the induction hypothesis we have $\htrip{\eqs_1^2}{\prog_2}{\eqs_2^2}$,
  and thus $\htrip{\eqs_1^2}{\prog_2}{\eqs_2^1 \wedge \eqs_2^2}$ by \TirName{Frame} \cnote{need to
    strengthen vars issue (also in second subcase)},
  by \TirName{Seq} we have $\htrip{\eqs_0}{\prog_1;\prog_2}{\eqs_2^1 \wedge \eqs_2^2}$.
  So this subcase holds.

  \textit{Subcase} $\eqs_0 \models  \eqs_1^1 \wedge (\eqs_1 \impl \eqs_1^2)$.
  By the induction hypothesis we have $\htrip{\eqs_0}{\prog_1}{\eqs_2^1}$,
  and also by the induction hypothesis and Lemmas \ref{lemma-subn} and
  \ref{lemma-hardpacksubn} we have $\eqs_0 \models \eqs_1^1 \wedge \subn_{\prog_1}(\eqs_1^2)$,
  thus $\eqs_0$ iff $\eqs_0 \wedge \subn_{\prog_1}(\eqs_1^2)$. Thus
  $\htrip{\eqs_0 \wedge \subn_{\prog_1}(\eqs_1^2)}{\prog_1}{\eqs_2^1}$
  by \TirName{Consequence}, so also
  $\htrip{\eqs_0 \wedge \subn_{\prog_1}(\eqs_1^2)}{\prog_1}{\eqs_2^1 \wedge \eqs_1^2}$
  by \TirName{Assigns} and
  $\htrip{\eqs_0}{\prog_1}{\eqs_2^1 \wedge \eqs_1^2}$ by \TirName{Consequence}.
  And by the induction hypothesis we
  have $\htrip{\eqs_1^2}{\prog_2}{\eqs_2^2}$, and thus
  $\htrip{\eqs_2^1 \wedge \eqs_1^2}{\prog_2}{\eqs_2^1 \wedge \eqs_2^2}$ by \TirName{Frame}.
  So by \TirName{Seq} we have $\htrip{\eqs_0}{\prog_1;\prog_2}{\eqs_2^1 \wedge \eqs_2^2}$,
  therefore this subcase holds.

  Now, the induction hypothesis implies the existence of $\subn_1, \subn_2$ with
  $\subn_i \subseteq \subn_{\prog_i}$ 
  and $\toeq{\subn_i}$ iff $\eqs_i$ for $i \in \setit{1,2}$. Hence
  $\subn_1 \circ \subn_2 \subseteq \subn_{\prog}$ and $\subn_1 \circ \subn_2$
  iff $\eqs_1 \wedge \eqs_2$. The case follows.

  \textit{Case} $\cmd = f(e_1,\ldots,e_n)$. In this case by assumption and
  definition and inversion of $\TirName{App}$ and $\redx$ we have:
  \begin{mathpar}
    \codebase(f) = y_1,\ldots,y_n, \instr

    e_1 \redx \mv_1 \cdots e_n \redx \mv_n

    \instr[\mv_1/y_1]\cdots[\mv_n/y_n] \redx \prog

    \peq_1 = \peq^0_1[\mv_1/y_1]\cdots[\mv_n/y_n]

    \peq_2 = \peq^0_2[\mv_1/y_1]\cdots[\mv_n/y_n]

    \peq = \true
    
    \htrip{\peq^0_1}{f(y_1:\tau_1,\ldots,y_n:\tau_n)}{\peq^0_2},\peq^0
  \end{mathpar}
  Now, the judgement $\htrip{\peq^0_1}{f(y_1:\tau_1,\ldots,y_n:\tau_n)}{\peq^0_2},\peq^0$
  follows by one or two subcases. Where it follows by \TirName{Fn} the result
  is immediate by definition and the induction hypothesis. Otherwise it follows by
  \TirName{Hardpack} and we proceed as follows.

  By inversion of \TirName{Hardpack} we have
  $\htrip{\peq_1'}{f(y_1:\tau_1,\ldots,y_n:\tau_n)}{\peq_2'},\peq$ by
  \TirName{Fn} where $\forall y_1:\tau_1,\ldots,y_n:\tau_n . \peq^0_1
  \models \peq_1'$. Assume $\peq_1'[\mv_1/y_1]\cdots[\mv_n/y_n] \redx
  \eqs_1'$ and $\peq_2'[\mv_1/y_1]\cdots[\mv_n/y_n] \redx
  \eqs_2'$\cnote{sanity conditions should guarantee existence of
    $\eqs_{1,2}'$}.  By Lemma \ref{lemma-genentails} we have $\eqs_1
  \models \eqs_1'$, thus sanity conditions on $\peq_1$ and the
  induction hypothesis entail $\htrip{\eqs_1}{\prog}{\eqs_2'}$. Also
  by inversion of \TirName{Hardpack} we have $\forall
  y_1:\tau_1,\ldots,y_n:\tau_n . \peq_1 \models \peq_0 \impl \peq_2$,
  implying $\eqs_1 \models \eqs \impl \eqs_2$ by Lemma
  \ref{lemma-genentails}. But then by Lemmas \ref{lemma-subn},
  \ref{lemma-hardpacksubn}, and the induction hypothesis we have
  $\eqs_1 \models \subn_{\prog}(\eqs_2)$, thus $\eqs_1$ iff $\eqs_1
  \wedge \subn_{\prog}(\eqs_2)$, and we can deduce $\htrip{\eqs_1
    \wedge \subn_{\prog}(\eqs_2)}{\prog}{\eqs_2'}$ by
  \TirName{Consequence}, thus $\htrip{\eqs_1 \wedge
    \subn_{\prog}(\eqs_2)}{\prog}{\eqs_2' \wedge \eqs_2}$ by
  \TirName{Assigns}, and thus $\htrip{\eqs_1}{\prog}{\eqs_2}$ by
  \TirName{Consequence}, so this case follows since
  $\htrip{\eqs_0}{\prog}{\eqs_2}$ for any $\eqs_0$
  with $\eqs_0 \models \eqs_1$ by \TirName{Consequence}.

  Finally, \textit{case} $\prog = \elet{y}{e}{\cmd}$ follows trivially
  by the induction hypothesis, which covers all cases to prove
  the result.
\end{proof}

\begin{theorem}
  If $\htrip{\eqs_1}{\ttt{main()}}{\eqs_2},\true$ as a consequence of
  \TirName{HardPack} and $\ttt{main}() \redx \prog$ then
  $\htrip{\eqs_1}{\prog}{\eqs_2}$.
\end{theorem}

\subsection{Yet Another Refinement}

\medskip

\begin{mathpar}
  \inferrule[Mesg]
            {}
            {\htrip{\eqtrue}{\xassign{e_1}{e_2}{e_3}}{e_1 \eop \elab{e_2}{e_3}}}

  %\inferrule[Encode]
  %          {\mx{e_1}{e_2} \redx x \\ \notg{\phi} \redx \phi \\
  %            \eqs \models x \eop \phi\\
  %            \atj{\phi}{R}{\ty}}
  %          {\mtj{\eqcast{\mx{e_1}{e_2}}{\notg{\phi}}}{\eqs}{(x : \ty)}{R}{\varnothing}{\eqs}}
  %
  \inferrule[Assert]
            {}
            {\htrip{\elab{e_1}{e_3} \eop \elab{e_2}{e_3}}{\elab{\assert{e_1 = e_2}}{e_3}}{\eqtrue}}

  \inferrule[Seq]          
            {\htrip{\peq_1^1}{\cmd_1}{\peq_2^1} \\ \htrip{\peq_1^2}{\cmd_2}{\peq_2^2}}
            {\htrip{\peq_1^1 \wedge (\peq_2^1 \impl \peq_1^2)}{\cmd_1;\cmd_2}{\peq_2^1 \wedge \peq_2^2}}

  \inferrule[Let]
            {\htrip{\peq_1}{\cmd[e/y]}{\peq_2}}
            {\htrip{\peq_1}{\elet{y}{e}{\cmd}}{\peq_2}}

  \inferrule[App]
            {\htrip{\peq_1}{f(y_1:\tau_1,\ldots,y_n:\tau_n)}{\peq_2}}
            {\htrip{\peq_1[e_1/y_1 \cdots e_n/y_n]}{f(e_1,\ldots,e_n)}{\peq_2[e_1/y_1 \cdots e_n/y_n]}}

  \inferrule[Fn]
            {\codebase(f) = y_1 : \tau_1, \ldots, y_n : \tau_n, \cmd \\ \htrip{\peq_1}{\cmd}{\peq_2}}
            {\htrip{\peq_1}{f(y_1:\tau_1,\ldots,y_n:\tau_n)}{\peq_2}}

  \inferrule[Hardpack]
            {\precond(f) = \peq_1 \\ \postcond(f) = \peq_2 \\
              \htrip{\peq_1'}{f(y_1:\tau_1,\ldots,y_n:\tau_n)}{\peq_2'} \\
              \forall y_1:\tau_1,\ldots,y_n:\tau_n . 
            \peq_1 \models \peq_1'\ \text{and}\ \peq_1 \wedge \peq_2' \models \peq_2}
            {\htrip{\peq_1}{f(y_1:\tau_1,\ldots,y_n:\tau_n)}{\peq_2}}
\end{mathpar}

\medskip

The above relies on the soundness of the \TirName{Hyp} rule below, which may be
derivable but at least is provably sound for the assumed correctness
property of the logic which is:
$$
\htrip{\eqs_1}{\prog}{\eqs_2} \ \textit{implies}\ \eqs_1 \wedge \toeq{\prog} \models \eqs_2
$$

\begin{mathpar}
\inferrule[Hyp]
          {\htrip{\eqs_1}{\prog}{\eqs_2}}
          {\htrip{\eqs_1 \wedge (\eqs_2 \impl \eqs)}{\prog}{\eqs_2 \wedge \eqs}}
\end{mathpar}

