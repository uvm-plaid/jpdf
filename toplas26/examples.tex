\section{Implementation and Extended Examples}
\label{section-examples}

In Section \ref{section-smt} we gave the example of 3-party additive
sharing for any field $\fieldp{p}$, in Section \ref{section-sty} we
discussed the example of YGC input encoding, and in Sections
\ref{section-metalang} and \ref{section-metaty} we presented and
discussed the example of gates in the GMW protocol for binary
circuits. In this section we present an extended example of 2-party
arithmetic circuits with multiplication gates defined using Beaver
triples \cite{evans2018pragmatic}. We first define a compositional
gate library in $\metaprot$ with correctness conditions defined using
the Hoare logic developed in Section \ref{section-metalang}, along
with a small example circuit (Figure \ref{fig-beaver}).  Then we
extend the library with type annotations illustrating the use of share
types developed in Sections \ref{section-sty} and \ref{section-metaty}
to enforce confidentiality properties (Figure
\ref{fig-beaver-types}). Lastly, we extend the base library with
additional pre and postcondition annotations to enforce integrity
properties supported by BDOZ/SPDZ style MACs
\cite{SPDZ1,SPDZ2,BDOZ,10.1007/978-3-030-68869-1_3} (Figure
\ref{fig-beaver-bdoz}).

\subsection{Evaluation}

We have implemented the $\metaprot$ language and automated Hoare logic
using an Antlr parser and Scala backend with cvc5 Java bindings for
interaction with proof engines.  To evaluate performance, we have
written and verified gate libraries using the GMW protocol for binary
circuits as excerpted in Figure \ref{fig-gmwprepost} in Section
\ref{section-metalang}, and for passive and malicious secure
arithmetic circuits using Beaver Triple multiplication as presented in
Figures \ref{fig-beaver} and \ref{fig-beaver-bdoz} in this Section.
We have considered field sizes typically used to encode binary, 16
bit, 32 bit, and 64 bit integers. Results are presented in Figure
\ref{fig-preludetest}. We found that verification of library
components was efficient even up to large field sizes. Since typing is
compositional (see, e.g., Figure \ref{fig-beaver-types}), verification
of gate properties relevant to \emph{security} will scale to larger
circuits.

\preludetestfig

Circuit \emph{correctness} also benefits from compositional
verification-- for example, the correctness of a GMW ``and'' gate (as
in Figure \ref{fig-gmwprepost}) only needs to be proven once, though
instantiated and integrated into the whole-program analysis needed to
verify a large circuit specification. We have verified 2, 8, and
64-bit vector zero-test GMW circuits containing 3, 15, and 127 gates
respectively, based on ``Bristol Fashion'' circuits \cite{bristol}. By
verified, we mean automated verification of the zero-test
specification expressed as a \ttt{postcondition}. We considered both
the cvc5 Bit-Vector and Satisfiability Modulo Finite Fields (SMFF)
theories as proof engines for these binary circuits.

Verification of the zero-test circuits was efficient in the Bit-Vector
theory, which verified the 64-bit zero test in GMW in less than 7
seconds. However, it took several minutes to verify even the small
2-bit zero-test circuit in SMFF, and larger circuits took too long to
consider for a practical evaluation. This could be attributed to the
relative maturity of the Bit-Vector theory as compared to SMFF.  In
any case, the SMT-based approach achieves categorically better
performance than previous brute-force methods developed for binary
circuits \cite{skalka-near-ppdp24}.

\subsection{Arithmetic Circuits with Beaver Triples}
\label{section-beaver}

\begin{fpfig}[t]{Authenticated 2-Party Multiplication.}{fig-beaver}
{\footnotesize
  \begin{verbatimtab}
    secopen("a","x","d",1,2);
    secopen("a","x","d",2,1);
    secopen("b","y","e",1,2);
    secopen("b","y","e",2,1);
    let xys =
      macsum(macctimes(macshare("b"), m["d"]),
             macsum(macctimes(macshare("a"), m["e"]),
                    macshare("c")))
    in
    let xyk = mack("b") * m["d"] + mack("d") * m["d"] + mack("c")               
    in
    secreveal(xys,xyk,"1",1,2);
    secreveal(maccsum(xys,m["d"] * m["e"]),
              xyk - m["d"] * m["e"],
              "2",2,1);
    out@1 := (p[1] + p[2])@1;
    out@2 := (p[1] + p[2])@2;
  \end{verbatimtab}
}
\end{fpfig}


In Figure \ref{fig-beaver} we define a 2-party arithmetic circuit with
4 input gates, 3 internal gates, and 1 output gate. The ideal functionality
is defined as $\ttt{IDEALF}$. As in GMW or BDOZ, circuit wire values
are represented as additive secret shares between two clients we call
$\ttt{1}$ and $\ttt{2}$. As in the GMW example discussed in Section
\ref{section-metaty}, we use string identifiers $g$ to distinguish
gates, and gate output wire shares are defined as identifiers
$\mx{g}{1}$ and $\mx{g}{2}$. The function $\ttt{RECON}$
represents share reconstruction \emph{in constraints} (not the
protocols themselves). Input gates are
implemented using the $\ttt{genshares}$ function as described in
Section \ref{section-sty}. The $\ttt{open}$ function
implements a federation reveal of a shared value through
share broadcast and reconstruction.

A \ttt{sum} gate can be defined non-interactively using 
homomorphic addition. The correctness \ttt{postcondition} is simple
and perhaps unnecessary to accelerate circuit verification, but is
appealing declaratively.  The multiplication gate \ttt{mult} is more
complex, where we use the common approach of \emph{Beaver Triples}
\cite{evans2018pragmatic}. Recall that Beaver triples are values
$a,b,c$ with $a$ and $b$ chosen randomly and $c = a * b$, unique per
multiplication gate, that are secret-shared with clients during a
pre-processing phase by an assumed oracular client $\ttt{3}$.
The uniqueness requirement is necessary because
$a$ and $b$ are random values that serve as OTP masks for secure
openings, implemented as \ttt{secopen}. In our encoding we follow the
convention that each gate output identifier distinguishes the Beaver
triple, so for example the share of the $a$ value for a gate
$\ttt{"g"}$ is identified by $\ttt{"ga"}$, etc. The correctness
property of the multiplication gate, expressed as a
\ttt{postcondition}, relies on the $\ttt{BEAVERTRIP}$
precondition. This compositional correctness property can then be used
in circuit construction, without the need for ``re-proof'' in our
Hoare logic inference.


The circuit itself is constructed in \ttt{main} using these
gates. This is a small example but arbitrarily large circuits can be
constructed with this library. The precondition of \ttt{main} requires
distinct Beaver triples for each multiplication gate in the circuit
(just one in this example). The postcondition allows verification of
circuit correctness-- that the protocol output is equal to the ideal
functionality $\ttt{IDEALF}$.

\subsection{Share Type Annotations.}

In Figure \ref{fig-beaver-types} we extend our previous gate definitions with
share type annotations to enforce confidentiality. The core idea is that
wire values must be \ttt{share}s. The \ttt{genshares}
function includes an annotation specifying that all input gates establish
the $\ttt{shares}$ invariant for wire values. In the \ttt{sum} function,
the \ttt{share} type preconditions for input wires are specified. The
output wire shares can be declared as \ttt{share}s without any coercion
needed, since given preconditions our typing algorithm ($\Vdash$) is
able to check the use of homomorphic addition. 

We add coercions and typing assertions to the $\ttt{secopen}$ function
to specify properties of secure openings. After noninteractively
summing the secret-shared wire value $\ttt{RECON(x)}$ and Beaver triple value
$\ttt{r[x++a]@3}$, the federation can publicly open the result since it is
masked by the random value. This is expressed using the coercion
$\ttt{as RECON(x) +  r[x++a]@3}$. The coercion then allows the
securely opened value to be declared $\ttt{public}$.

In the \ttt{mult} function, each client $\ttt{i}$ symmetrically
computes:
$$
\ttt{((m[z++"d"] * m[y]) + (-m[z++"e"] * m[z++"a"]) + m[z++"c"])@i}
$$
Since $\ttt{m[z++"d"]@i\,:\,public}\ $ and $\ \ttt{m[z++"e"]@i\,:\,public}$
are both established by $\ttt{secopen}$, and
$\ttt{m[y]@i}$, $\ttt{m[z++"a"]@i}$, and $\ttt{m[z++"c"]@i}$ are
all declared to be shares in preconditions, we have that each
of the shared computations $\ttt{(m[z++"d"] * m[y])@i}$ and $\ttt{(-m[z++"e"] * m[z++"a"])@i}$
are shares, so the gate outputs, which are then sums of shares, are shares, and this will
be successfully checked by syntax-directed $\Vdash$ type checking that
results from the declaration $\ttt{[m[z]@i1,m[z]@i2]\,:\,shares}$. 

\subsubsection{Malicious-Secure Arithmetic Circuits}

Our Definition \ref{def-integrity} establishes a general semantics of
integrity, while the $\minicat$ integrity type system presented in
Section \ref{section-ipj} focuses on the use of BDOZ/SPDZ-style
MACs. A main concern here is the enforcement of MAC checking which is
an algebraic property. Here we show how pre and postconditions can be
used to enforce these properties in our arithmetic circuit library, in
Figure \ref{fig-beaver-bdoz}. Note that this code is an extension of the
library presented in Figure \ref{fig-beaver} and does not incorporate
the type annotations in Figure \ref{fig-beaver-types}, but this is
just for clarity of the presentation and these features could be combined. 

In this code we use a convention where shared values $\ttt{x}$ are
represented as a 3-tuple of values, MAC, and keys, where the value and
MAC of client \ttt{1} are $\ttt{m[x++"s"]@1}$ and $\ttt{m[x++"m"]@1}$
respectively, and the corresponding key $\ttt{m[x++"k"]@2}$ is held by
client \ttt{2}. The correctness condition of this 3-tuple is
$\ttt{MACOK}(x,2,1)$. The same is true symmetrically, so the
$\ttt{MACed}$ predicate establishes both sides of the correctness
condition.  In general, this predicate is used in gate preconditions
to assume that the MAC conditions hold for input shares, and in
postconditions to ensure that output shares are $\ttt{MACed}$. The
homomorphic properties that preserve $\ttt{MACed}$ness, especially in
\ttt{mult}, are perhaps even more complex than the algebraic
correctness of gates, so it is beneficial to verify them compositional
especially for large circuits.

Our code is written so that share MACs are checked for when they are
\ttt{open}ed by both clients, specified by the appropriate
$\ttt{assert}$ included in the protocol. This ensures malicious
security of secure openings and outputs. As is standard in BDOZ/SPDZ,
we assume that input values, MACs, and keys are distributed to clients
by a pre-processing protocol with relevant postconditions
declared in the preconditions of $\ttt{main}$ (as in Figure
\ref{fig-beaver} with respect to Beaver triples).

