\section{Component Certification and Composition}
\label{section-composition}

Using YGC as a case study, in this Section we will discuss
compositional properties that can be automatically verified on small
program components using techniques developed in Section
\ref{section-bruteforce}. These properties compose to yield $\NIMO$
and scale to arbitrary programs. In our metatheory we will express
these properties as pre-~and post-conditions on components, so our
general approach is in a proof assistant style. 

\subsection{A Library for Yao's Garbled Circuits}
\label{section-composition-ygc}

\begin{fpfig}[t]{Yao's Garbled Circuits, auxiliary functions.}{fig-ygc-aux}
{\footnotesize
\begin{verbatimtab}
  keygen(gid, b1, b2) { select4(b1,b2,H[gid || "1"],H[gid || "2"],H[gid || "3"],H[gid || "4"]) }
  
  keysgen(gid, b1, b2)
  {
    let k11 = keygen(gid,b1,b2) in
    let k10 = keygen(gid,b1,not b2) in
    let k01 = keygen(gid,not b1,b2) in
    let k00 = keygen(gid,not b1,not b2) in
    {k11 = k11; k10 = k10; k01 = k01; k00 = k00}
  }
  
  andtable(keys, bt, ap, bp)
  {
    let r11 = (keys.k11 xor bt) in 
    let r10 = (keys.k10 xor (not bt)) in
    let r01 = (keys.k01 xor (not bt)) in
    let r00 = (keys.k00 xor (not bt)) in
    permute4(ap,bp,r11,r10,r01,r00)
  }
  
  sharetable(gid, vid, table)
  {   
    v[1, "gate:" || gid || vid || "1"] := table.v1;
    v[1, "gate:" || gid || vid || "2"] := table.v2;
    v[1, "gate:" || gid || vid || "3"] := table.v3;
    v[1, "gate:" || gid || vid || "4"] := table.v4
  }
\end{verbatimtab}
}
\end{fpfig}

\begin{fpfig}[t]{Yao's Garbled Circuits, garbled gates and evaluation code.}{fig-ygc-gates}
{\footnotesize
\begin{verbatimtab}
  garbledecode(wl)
  {
    let r1 = wl.k xor true in
    let r0 = (not wl.k) xor false in
    v[1,"OUTtt1"] := select[wl.p,r1,r0];
    v[1,"OUTtt2"] := select[not wl.p,r1,r0]
  }
  
  evaldecode(wl, p) { wl.k xor select[wl.p,v[1,"OUTtt1"],v[1,"OUTtt2"]] }
  
  evalgate(gid, wla, wlb)
  {
    let k = keygen(gid,wla.k,wlb.k) in
    let ct = select4(wla.p,wlb.p,
               v[1,gid || "tt1"],v[1,gid || "tt2"],v[1,gid || "tt3"],v[1,gid || "tt4"]) in
    let cp = select4(wla.p,wlb.p,
               v[1,gid || "pt1"],v[1,gid || "pt2"],v[1,gid || "pt3"],v[1,gid || "pt4"]) in
    { k = k xor ct; p = k xor cp }
  }
  
  andgate(gid, wla, wlb, wlc) 
  {
    let keys = keysgen(gid,wla.k,wlb.k) in
    sharetable(gid,"tt",andtable(keys,wlc.k,wla.p,wlb.p));
    sharetable(gid,"pt",andtable(keys,wlc.p,wla.p,wlb.p))
  }
  
  encode(gid, wla,wlb)
  {
    let wla = { k = flip[2,"fwl1"]; p = flip[2,"pwl1"] } in
    let wlb = { k = flip[2,"fwl2"]; p = flip[2,"pwl2"] } in
    { wv1 = { k = OT[s[1,"0"],wla.k,(not wla.k)]; p = OT[s[1,"0"],wla.p,(not wla.p)]}; 
      wv2 = { k = select[s[2,"0"],wlb.k,(not wlb.k)]; p = select[s[2,"0"],wlb.p,(not wlb.p)] } }
  }
\end{verbatimtab}
}
\end{fpfig}

%  andtable(keys, bt, ap, bp)
%  {
%    let r11 = (keys.k11 xor bt) in 
%    let r10 = (keys.k10 xor (not bt)) in
%    let r01 = (keys.k01 xor (not bt)) in
%    let r00 = (keys.k00 xor (not bt)) in
%    permute4(ap,bp,r11,r10,r01,r00)
%  }


In Figures \ref{fig-ygc-aux} and \ref{fig-ygc-gates} we define a
codebase for garbled circuits. This definition follows the
\emph{point-and-permute} method described in \cite{evans2018pragmatic} and elsewhere,
to which the reader is referred for more in-depth discussion.
In this implementation client 2 is the \emph{garbler} and
client 1 is the \emph{evaluator}. The garbler builds the garbled
tables and shares them with the evaluator, who then evaluates
the gate in an oblivious fashion until the final public output is
generated through decryption. This definition is well-typed,
with input type annotations for top-level functions listed in
Figure \ref{fig-ygc-types}. Well-typed programs using these
libraries are therefore guaranteed to yield safe $\minifed$
programs. 

\emph{Wire labels} are fundamental to YGC, and essentially represent
gate output values in an encrypted form. In our definition, wire
labels are represented by records $\ttt{\{ k = }\beta_1\ttt{; p =
}\beta_2\ttt{ \}}$, where $\ttt{k}$ is the \emph{key bit} and
$\ttt{p}$ is the \emph{pointer bit}, and $\beta_1$ and $\beta_2$ are
flips. Flips in each output wire label are owned by the garbler and
are unique per gate by definition of their identifying string, and the
representation of $0$ is the negation of $1$. For example, here is the
representation of 1 and 0 respectively in the output wire label for a
hypothetical gate 6:
\begin{mathpar}
  \ttt{\{ k = flip[2,gate:6.k]; p =  flip[2,gate:6.p]] \}}
    
  \ttt{\{ k = not flip[2,gate:6.k]; p =  not flip[2,gate:6.p]] \}}
\end{mathpar}
The pointer bits in wire labels are used to select permuted rows in
table garblings. The key bits are used to identify a unique key for
table row in each garbled gate. Intuitively, if $\beta_1$ and
$\beta_2$ are either key or pointer bits encoding 1 on two input wire
labels to a binary gate, rows and keys in the gate are enumerated in
the order:
$$
\neg\beta_1\neg\beta_2,\ \neg\beta_1\beta_2,\ \beta_1\neg\beta_2,\ \beta_1\beta_2
$$

In our implementation, gates are wired together using gate
identifiers, which are strings $w$. Top-level functionality in Figures
\ref{fig-ygc-aux} and \ref{fig-ygc-gates} includes the following:
\begin{itemize}
\item \ttt{andgate}: This defines a subprotocol for the garbler
  to define a garbled gate $\ttt{gid}$ with input wires from gates
  $\ttt{ga}$ and $\ttt{gb}$. The garbler generates keys and garbles
  the rows in YGC fashion, them with client $1$ in
  views in a standard form. For example, the view for
  a hypothetical gate 6, row 2 garbled truth table is $\ttt{v[1,gate:6tt2]}$.
  We note that garbled gates of other binary operators can be obtained with
  replacement of $\ttt{andtable}$ with appropriate garbled table definitions. 
\item \ttt{evalgate}: This defines a subprotocol for the evaluator to
  evaluate gate $\ttt{gid}$ given input wire values $\ttt{wva}$ and
  $\ttt{wvb}$.
\item \ttt{garbledecode} and \ttt{evaldecode}: The former function
  defines the garbler's protocol for encrypting the circuit
  output from final gate $\ttt{gid}$, and the latter defines
  the evaluator's output decryption protocol.
\item \ttt{encode}: This defines the initial phase of the protocol,
  where the evaluator receives the wire value from their own
  secret $\sx{1}{sa}$ via $\ttt{OT}$, and the garbler communicates
  the wire value for their own secret $\sx{2}{sb}$ directly.
\end{itemize}
\begin{example}
  \label{example-andcircuit}
The following program uses our YGC library to define
a circuit with a single and gate and input secrets $\ttt{s1}$ and
$\ttt{s2}$ from client's 1 and 2 respectively. 
\begin{verbatimtab}
  andgate(0,s1,s2);
  garbledecode(0);
  let secrets = encode(s1,s2) in
  v[0,output] := decode(evalgate(0, secrets.wv1, secrets.wv2))
\end{verbatimtab}
\end{example}
We have verified passive security of the $\fedprot$ protocol
generated by this and other small circuits using the
technique described in Lemma \ref{lemma-bruteforce-nimo}.
But large circuits with thousands of gates would be
intractable to verify with this method. In the next Section
we discuss compositional methods to address this issue.

\subsection{Component Certification for $\NIMO$}
\label{section-composition-certification}

We can use the automated verification methods developed in Section
\ref{section-brute-force} to \emph{certify} YGC
components. Certification obtains properties that we can then use
higher-level reasoning-- inspired in part by techniques in
\cite{barthe2019probabilistic}-- to prove that circuits built with
certified components satisfy $\NIMO$ (Theorem \ref{theorem-ygc-NIMO})
and thus are passive secure.

Our method relies on a definition of compositional units comprising
complete related subprotocols-- i.e., both the garbler's
construction of the gates and the evaluator's evaluation of them. We
therefore make the following definitions. However, we note that our
formal results are applicable to the YGC style where the garbler
shares the entire circuit prior to evaluation.
\begin{definition}
  \label{ygc-modules}
  We define a composable internal \ttt{and} gate sharing and evaluation function
  called $\ttt{andgg}$, and an output decoder gate share and evaluation function called
  $\ttt{decode}$, as follows. 
\begin{verbatimtab}
  andgg(g, ga, gb, wva, wvb) { andgate(g, ga, gb); evalgate(g, wva, wvb) }
  decode(g, v) { garbledecode(g); evaldecode(v) }
\end{verbatimtab}
\end{definition}

Our certification techniques are defined generally wrt implementations
of input encoding, decoding, and internal gates. In the following we
will refer to arbitrary decode, encode, and gate functions with the
restriction that any function in each of these categories has the same
valid type signature as $\ttt{decode}$, $\ttt{encode}$, and
$\ttt{andgg}$ respectively.

Certifications will demonstrate probabilistic independence of input
wire lables and garbled tables in a program's basic distribution, so
we need to consider dependencies of flips and gate views. Thus we make
the following definitions:
\begin{definition}
  We write $\progtt(\prog)$ to denote the \emph{basic distribution} of
  $\prog$ as defined in Definition \ref{def-progd}. For any program $\prog =
  (\eassign{v_1}{\be_1};\ldots;\eassign{v_n}{\be_n})$, define
  $\vdefs(\prog) \defeq \{ v_1,\ldots,v_n \}$.
\end{definition}
We also need to formalize notions of probabilistic independence, aka
\emph{separation}, and correlation. We borrow the symbols $*$ and $\sim$
from \cite{barthe2019probabilistic} with the same denotations.
\begin{definition}
  We write $\vc{\pdf{X}}{x}{y}$ iff $\pdf{X}(\{ x \mapsto 0\}\ |\ \{ y \mapsto 0 \}) =
  \pdf{X}(\{ x \mapsto 1\}\ |\ \{ y \mapsto 1 \}) = 1$.
  We write $\sep{\pdf{X}}{Y}{Z}$ iff for all
    $\store \in \mems(X \cup Y)$:
  $$(\margd{\pdf{X}}{Y \cup Z})(\store) =
  (\margd{\pdf{X}}{Y})(\store_Y) * (\margd{\pdf{X}}{Z})(\store_Z)$$ 
\end{definition}
It is also necessary to define notions of correlation with wire
labels, since wire values flow into and out of gates in either
positive of negative correlation with them.
\begin{definition}
  We write $\gc{\prog}{\{ \ttt{k = }\be_{k}; \ttt{p = }\be_{p} \}}{g}$ iff one of the
  following hold:
  \begin{enumerate}[\hspace{5mm}i.]
  \item $\vc{\progtt(\prog,\be_k)}{\itv}{\ttt{flip[2,gate:}g\ttt{.k]}}\quad \text{and} \quad
    \vc{\progtt(\prog,\be_p)}{\itv}{\ttt{flip[2,gate:}g\ttt{.p]}}$
  \item $\vc{\progtt(\prog,\enot\ \be_k)}{\itv}{\ttt{flip[2,gate:}g\ttt{.k]}}\quad \text{and}\quad
    \vc{\progtt(\prog,\enot\ \be_p)}{\itv}{\ttt{flip[2,gate:}g\ttt{.p]}}$
  \end{enumerate}
\end{definition}
Thus, if a value is in correlation with a wire label, it is either
correlated with the wire label or with the inverse of both the key and
pointer bits. For subsequent discussion we make the following addition
to the codebase:
$$
\ttt{invert(\{ k = bk; p = bp \}) \{ \{ k = not bk; p = not bp \} \}}
$$
Gate certification establishes separation of input labels from the
gate table and output value, and correlation of the output with
the output wire label under any input wire value combination.
\begin{definition}[Gate Certification]
  \label{definition-gate-certification}
Let $\mathit{gate}$ be a gate function. For $1 \le i \le 4$  let:
  $$
  \config{\varnothing}{\ttt{v[0,out] := } e_i}\redxs\config{\prog_i}{\ttt{()}}
  $$
  where
  \begin{enumerate}[\hspace{5mm}$e_1 \defeq$]
    \item $\mathit{gate}\ttt{(c,a,b,owl(a),owl(b));}$
    \item $\mathit{gate}\ttt{(c,a,b,owl(a),invert(owl(b)));}$
    \item $\mathit{gate}\ttt{(c,a,b,invert(owl(a)),owl(b));}$
    \item $\mathit{gate}\ttt{(c,a,b,invert(owl(a)),invert(owl(b)))}$
  \end{enumerate}
  Define:
  $${\small X \defeq \ttt{\{ flip[2,gate:a.k], flip[2,gate:a.p], flip[2,gate:b.k], flip[2,gate:b.p] \}}}$$
  Then $\mathit{gate}$ is \emph{certified} iff for all $1 \le i \le 4$:
  $$
  \sep{\progtt(\prog_i)}{X}{\vdefs(\prog_i)} 
  \quad \text{and} \quad \gc{\progd(\prog_i)}{\vx{0}{out}}{\ttt{owl(c)}}
  $$
\end{definition}
Encode certification establishes correlation of the encoded input values with
the secret wire labels, and separation of client 2's secret from client 1's views. 
\begin{definition}[Encode Certification]
  Let $\mathit{encode}$ be an encoding function, and let:
  $$
  \config{\varnothing}{\ttt{encode(s1,s2)}} \redxs
  \config{\prog}{\ttt{\{wv1=}v_1\ttt{;wv2=}v_2\ttt{\}}}
  $$
  Then $\mathit{encode}$ is \emph{certified} iff each of the following conditions hold:
  \begin{enumerate}[\hspace{5mm}i.]
  \item $\gc{\progd(\prog,v_1)}{\itv}{\ttt{owl(s1)}} \quad \text{and} \quad
    \gc{\progd(\prog,v_2)}{\itv}{\ttt{owl(s2)}}$
  \item $\sep{\progd(\prog,v_2)}{\{\ttt{s[2,s2]\}}}{(\vdefs(\prog) \cup \itv)}$
  \end{enumerate}
\end{definition}
Decode certification establishes separation of the input wire
label from the decoding tables under any input value condition. 
\begin{definition}[Decode Certification]
  Let $\mathit{decode}$ be a decoding function, and let:
  $$
  \config{\varnothing}{\mathit{decode}(\ttt{c}, \ttt{owl(c)})}
  \redxs \config{\prog_1}{\be_1}
  \quad \text{and} \quad
  \config{\varnothing}{\mathit{decode}(\ttt{c}, \ttt{invert(owl(c))})}
  \redxs \config{\prog_0}{\be_0}
  $$
  Then $\mathit{decode}$ is certified iff:
  $$
  \sep{\progtt(\prog_i)}{\ttt{\{flip[2,gate:c.k],flip[2,gate:c.p]\}}}{\vdefs(\prog_i)}
  $$
  %Then $\mathit{decode}$ is certified iff both of the following conditions hold:
  %\begin{enumerate}[\hspace{5mm}i.]
  %\item $\progd(\prog_1,\be_1)(\{ \itv \mapsto 1 \}) = 1$ and
  %  $\progd(\prog_0,\be_0)(\{ \itv \mapsto 0 \}) = 1$
  %\item $\sep{\progtt(\prog_i)}{\ttt{\{flip[2,gate:c.k],flip[2,gate:c.p]\}}}{\vdefs(\prog_i)}$
  %\end{enumerate}
\end{definition}
We have certified each of the YGC components detailed above. 
\begin{lemma}
  \label{lemma-certification}
  Each of $\ttt{decode}$, $\ttt{encode}$, and $\ttt{andgg}$ are certified.
\end{lemma}

\subsection{Correctness of Certified Components}
\label{section-composition-metatheory}

To demonstrate correctness of certification, we need to show that
isolated component certificates preserve relevant properties
when integrated into larger programs. This resonates with
\emph{framing} as developed in \cite{barthe2019probabilistic}.
In the following we will use $f$ to range over flips.
First, we show that if a flip $f$ is independent from the views
in a protocol (as is the case with input labels used to
encode a garbled table), then separation from views that
use $f$ (e.g., in an output label) is preserved. 
\begin{lemma}[Wire Framing]
  \label{lemma-wire-framing}
  If $\sep{\progtt(\prog)}{\{f\}}{\vdefs(\prog)}$ for flip $f$ and
  $\vars(\prog) \cap \vars(\prog') = \{ f \}$ then
  $\sep{\progtt(\prog';\prog)}{\vdefs(\prog')}{\vdefs(\prog)}$.
\end{lemma}
Next, we show that output correlations (e.g., between output wire values and
labels) are preserved by substitution of correlated values for flips (e.g.,
between input wire values and labels).
\begin{lemma}[Substitution]
  \label{lemma-substitution}
  If $\vc{\progtt(\prog,\be)}{\itv}{f}$ and $\vc{\progtt(\prog',\be')}{\itv}{f'}$
  and $\sep{\progtt(\prog';\prog)}{\vdefs(\prog')}{\vdefs(\prog)}$
  then $\vc{\progtt(\prog';\prog,\be[\be'/f'])}{\itv}{f}$.
\end{lemma}
Our wire framing and substitution Lemmas then allow us to generalize
certification of gates and encoding to arbitrary circuit contexts,
assuming that outputs are not wired to multiple inputs \cite{tate2003garbled,nieminen2023breaking}.
\begin{lemma}[Gate Composition]
  \label{lemma-compose}
Given well-typed $\prog_1;E[\mathit{gate}(g,g_1,g_2,v_1,v_2)]$ and certified $\mathit{gate}$ where:
$$
\config{\prog_1}{E[\mathit{gate}(g,g_1,g_2,v_1,v_2)]} \redxs \config{\prog_1;\prog_2}{E[v]}
$$
If the following preconditions hold:
\begin{enumerate}
\item $\{ g_1 \} \cap \{ g_2 \} \cap \wired(\prog_1) = \varnothing$ and $g \not\in \prog_1$
\item $\gc{\progd(\prog_1)}{v_1}{\owl(g_1)}$ and $\gc{\progd(\prog_1)}{v_2}{\owl(g_2)}$
\end{enumerate}
then we have as postconditions:
\begin{enumerate}
  \item $\gc{\progd(\prog_1;\prog_2)}{v}{\owl(g_2)}$
  \item $\sep{\progd(\prog_1;\prog_2,v)}{\vdefs(\prog_1)}{\vdefs(\prog_2) \cup \{\itv\}}$
\end{enumerate}
\end{lemma}
\begin{proof}
  Given that $g_1$ and $g_2$ are distinct and not wired in $\prog_1$
  we are assured that $\ttt{owl}(g_1)$ and $\ttt{owl}(g_1)$ are in
  independent uniform distributions, and given that $g \not\in
  \prog_1$ we are assured that $\ttt{owl}(g)$ contains entirely fresh
  flips. Thus, precondition 2, Lemma \ref{lemma-wire-framing}, and
  Definition \ref{definition-gate-certification} imply postcondition
  2.  Furthermore, since gate certification guarantees that
  correlation with $\ttt{owl}(g)$ is guaranteed under all possible gate
  input wire values, precondition 2 and Lemma \ref{lemma-substitution}
  guarantee postcondition 2.
\end{proof}

\begin{lemma}[Decode Composition]
  \label{lemma-decode}
Given well-typed $\prog_1;E[\mathit{decode}(g,v)]$ and certified $\mathit{decode}$ where 
$$
\config{\prog_1}{E[\ttt{decode}(g,v)]} \redxs \config{\prog_1;\prog_2}{E[\be]}
$$
If the following preconditions hold:
\begin{enumerate}
\item $g \not\in \cap \wired(\prog_1)$
\item $\gc{\progd(\prog_1)}{v}{\owl(g)}$
\end{enumerate}
then we have as postcondition:
\begin{enumerate}
  \item $\sep{\progd(\prog_1;\prog_2)}{\vdefs(\prog_1)}{\vdefs(\prog_2)}$
\end{enumerate}
\end{lemma}
Given the above, we can establish that any circuit built with certified
components is passive secure. 
\begin{theorem}
  \label{theorem-ygc-NIMO}
  If $e$ is a well-typed YGC circuit definition using certified components
  where no gate output is wired more than once, and $\config{\varnothing}{e}
  \redxs \config{\prog}{\varnothing}$, then $\NIMO(\prog)$. 
\end{theorem}

\begin{proof}
  Let $\iov(\prog) = S \cup (V \cup \outv)$. Noting that any
  encoding necessarily uses entirely fresh variables by Definition and
  type safety, by applying Lemmas \ref{lemma-compose} and
  \ref{lemma-decode} and the frame rule of
  \cite{barthe2019probabilistic} we obtain:
  $$
  \sep{\progd(\prog)}{(S_{\{2\}}\cup\{\outv\})}{V}
  $$
  Noting that the garbler (client 2) has no views, this means that
  the garbler's secrets are independent of the evaluator's views. The
  result follows by Definition \ref{definition-NIMO}.
\end{proof}


\begin{fpfig}[t]{YGC copy gate definitions.}{fig-ygc-copy}
  {\footnotesize
    \begin{verbatimtab}
      sharetab2(gid, tid, k, p, b)
      {
        let r1 = k xor b in
        let r0 = (not k) xor (not b) in
        v[1,gate: || gid || tid || 1] := select(p,r1,r0);
        v[1,gate: || gid || tid || 2] := select(not p,r1,r0);
      }
      
      copygate(ca,cb,g)
      {
        let wl = owl(g) in
        let owl1 = owl(ca) in
        let owl2 = owl(cb) in
        sharetab2(ca,tt,wl.k,wl.p,owl1.k); sharetab2(ca,pt,wl.k,wl.p,owl1.p);
        sharetab2(cb,tt,wl.k,wl.p,owl2.k); sharetab2(cb,pt,wl.k,wl.p,owl2.p)
      }
      
      evalcopy(ca,cb,wv)
      {
        let wv1k = wv.k xor select(wv.p,v[1,gate: || ca || tt1], v[1,gate: || ca || tt2]) in
        let wv1p = wv.k xor select(wv.p,v[1,gate: || ca || pt1], v[1,gate: || ca || pt1]) in
        let wv2k = wv.k xor select(wv.p,v[1,gate: || cb || tt1], v[1,gate: || cb || tt2]) in
        let wv2p = wv.k xor select(wv.p,v[1,gate: || cb || pt1], v[1,gate: || cb || pt1]) in
        { wv1 = { k = wv1k; p = wv1p }; wv2 = { k = wv2k; p = wv2p } }  
      }

      copy(ca,cb,g,wv) { copygate(ca,cb,g); evalcopy(ca,cb,wv) } 
    \end{verbatimtab}
  }
\end{fpfig}

\subsection{Building and Extending Automatically Secure Circuits}
\label{section-composition-copy}

Theorem \ref{theorem-nimo} means that any well-typed program using
$\ttt{encode}$, $\ttt{andgg}$, and $\ttt{decode}$ will generate a
circuit that is passive secure, provided that gate outputs are not
used more than once. To add an $\ttt{or}$ gate to our library, we can
use $\ttt{andgg}$ as a template, in fact simple modification of
$\ttt{andtable}$ is all that would be needed. Thus, this library
embodies a Fairplay-like language where well-typedness of metaprograms
guarantees safety and security of generated protocols.

The requirements that output wire labels are not used more than once
reflects a known result, that naive copy gate output is unsound in
YGC- i.e., passive security fails. In fact, we are unable to
automatically verify $\gNIMO$ for $\ttt{andgg}$ without the assumption
that input wire lables are in independent uniform distributions. We
can capture the necessary fix as in Figure \ref{fig-ygc-copy}. The
garbler creates a copy gate with two new output wire labels in
independent uniform distributions (in \ttt{copygate}), and permutes
them using the input wire label pointer after encrypting with the
input wire label keys. The evaluator then recovers the encrypted
output wire values during evaluation (in \ttt{evalcopy}). Given an
$\ttt{orgg}$ gate defined as outlined abotve, we can then use
secure $\ttt{copy}$ as follows: 
\begin{verbatimtab}
  let s = encode(s1,s2) in
  let s1c = copy(c1s1,c2s1,s1,s.wv1) in
  let s2c = copy(c1s2,c2s2,s2,s.wv2) in
  let wv1 = andgg(1,c1s1,c1s2,s1c.wv1,s2c.wv1) in
  let wv2 = orgg(2,c2s1,c2s2,s1c.wv2,s2c.wv2) in
  decode(3,andgg(3,1,2,wv1,wv2))
\end{verbatimtab}
These definitions of $\ttt{copy}$ and $\ttt{orgg}$ have been
automatically verified as in Section \ref{section-pre-post} for
$\ttt{andgg}$. In general, any new gate form- such as optimized gates
\cite{XXX}- can be verified automatically in a similar way.

While the copy gate supports secure programming patterns, security
still relies on programmer discipline, i.e., not wiring a gate more
than once. However, we can keep track of wired gates in type effects,
similar to the manner in which we keep track of defined views- type
dependence in our system is able to accurately precisely track
gate identifiers. This is also reminiscent of the use of linearity
in other related systems for enforcing obliviousness \cite{darais2019language}. 

