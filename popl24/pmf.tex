\section{Probability Mass Functions and Program Distributions}

In order to precisely characterize distributions of secrets and views
in programs we define probability mass functions in a standard manner,
following the formulation in \ref{XXX}. In this formulation, memories
denote conjunctions of variable assignments. Intuitively, given a
program $\prog$ with $\iov(\prog) = S \cup V$ and $\store \in
\mems(\iov(\prog))$, when considering the execution of $\prog$
initialized with secrets $\store_S$, we are fundamentally concerned
with the joint probability of the assignments in $\store_V$
conditioned by the assignments $\store_S$. This is because
probabilistic dependence between honest secrets and corrupt views
\emph{is} information leakage. 

\subsection{Basic Definitions} We begin by defining memory probability distributions as follows.
\begin{definition}
  A \emph{memory probability distribution} $\pdf{X}$ is a function
  mapping memories in $\mems(X)$ to values in $[0..1]$, such that:
  $$
  \sum_{\store \in \mems(X)} \pdf{X}(\store) \  = \ 1
  $$
  We call $X$ the \emph{domain} of $\pdf{X}$.
\end{definition}
To recover succinct and familiar notation, we may omit the domain of a
distribution when it is clear from an application context-
i.e., we allow the following sugaring:
$$
\pdf{}(\store) \defeq \pdf{\dom(\store)}(\store)
$$
Now, we can define a notion of marginal and conditional
distributions as follows, which are standard for discrete
probability mass functions. 
\begin{definition}
  Given $X$ the \emph{marginal distribution} of $Y \subseteq X$
  in a distribution $\pdf{X}$, denoted $\margd{\pdf{X}}{Y}$,
  is a distribution with domain $Y$ where for all
  $\store \in \mems(Y)$:
  $$
  (\margd{\pdf{X}}{Y})(\store) =
  \sum_{\store' \in \mems(X-Y)} \pdf{X}(\store \cup \store')
  $$
\end{definition}

\begin{definition}
  Given $\pdf{X}$, its \emph{conditional distribution given
  $\store$} with $\dom(\store) = Y$ and $Y \subseteq X$, denoted
  $\condd{\pdf{X}}{\store}$, is a distribution with domain $X - Y$ where for all
  $\store' \in \mems(X - Y)$:
  $$
  (\condd{\pdf{X}}{\store})(\store') =
      \pdf{X}(\store \cup \store') / ((\margd{\pdf{X}}{Y})(\store))
  $$
\end{definition}
To recover familiar notation we allow the following syntactic
sugarings:
\begin{eqnarray*}
\pdf{X}(\store|\store') &\defeq& (\condd{\pdf{X}}{\store})(\store')\\
\pdf{X}(\store) &\defeq& (\margd{\pdf{X}}{\dom(\store)})(\store) \qquad \dom(\store) \subset X
\end{eqnarray*}

\subsection{Program Distributions}
Now we can define the probability distribution of a program $\prog$,
that we denote $\progd(\prog)$. Since $\fedcat$ is deterministic the
results (view assignments) of any run are determined by the input
secrets together with the random tape. And since we constrain programs
to not overwrite views we are assured that \emph{final} memories
contain both a complete record of all initial secrets as well as views
resulting from communicated information.

We consider random tapes to be selected from a uniform distribution.
In our threat model we will also assume that any possible initial
secrets are equally likely when formulating the program distribution.
In this setting, given a program $\prog$ with $\iov(\prog) = S \cup
V$ and $\flips(\prog) = F$ we will consider all $\store
\in \mems(S \cup V \cup F)$ such that
$
\config{\store_{S \cup F}}{\prog} \redxs \config{\store_}{\varnothing}
$
to be equally probable, establishing the basic distribution of the
program. From this we can immediately derive the marginal distribution
of $S \cup V$ to reason about dependencies between secrets and views. 
\begin{definition}
  \label{def-progd}
  Given progam $\prog$ with $\iov(\prog) = (S,V)$ and $\flips(\prog) = F$, define $\stores$ as:
  $$
  \stores \defeq \{ \store \mid (\dom(\store) = S \cup V \cup F) \wedge (\store_{S \cup F},\prog) \redxs (\store,\varnothing) \}
  $$
  Define also $\pdf{S \cup V \cup F}$ as the program's \emph{basic distribution} such that for all
  $\store \in \mems(S \cup V \cup F)$:
  $$
  \pdf{S \cup V \cup F}(\store) =
  \begin{cases}
    1 / |\stores| & \text{if}\ \store \in \stores\\
    0 & \text{otherwise}
  \end{cases}
  $$
  Then the \emph{program distribution of $\prog$}, denoted $\progd(\prog)$, is the
  marginal distribution of $S \cup V$ in $\prog$'s basic distribution:
  $$
  \progd(\prog) =  \margd{\pdf{S \cup V \cup F}}{S\cup V}
  $$
  In some cases we will also be concerned with the (joint)
  probabilities of expression interpretation given a preceding program
  execution, and we write $\progd(\prog, \be)$ to denote the program
  distribution $\progd(\prog;\eassign{\itv}{\be})$ where $\itv$ is a
  special variable that is never used in programs.
\end{definition}

\subsection{Examples}

{\footnotesize
  $$
  \begin{array}{cc}
    \begin{array}{cccccc}
      \verb+s[1,0]+ & \verb+f[1,0]+ & \verb+v[2,0]+  & \verb+v[2,1]+ & \verb+v[2,2]+ & \verb+v[0,0]+\\
      \hline
      0 & 0 & 0 & 0 & 0 & 0 \\ 
      0 & 1 & 1 & 1 & 0 & 1 \\ 
      1 & 0 & 0 & 1 & 1 & 1 \\ 
      1 & 1 & 1 & 0 & 1 & 0
    \end{array}
    & 
    \begin{array}{ccccc}
      \verb+s[1,s]+ & \verb+f[1,sx]+ & \verb+f[1,sy]+ & \verb+v[0,0]+ & \verb+v[0,1]+ \\
      \hline
      0 & 0 & 0 & 0 & 0 \\ 
      0 & 0 & 1 & 1 & 0 \\ 
      0 & 1 & 0 & 0 & 1 \\ 
      0 & 1 & 1 & 1 & 1 \\
      1 & 0 & 0 & 0 & 0 \\ 
      1 & 0 & 1 & 0 & 0 \\ 
      1 & 1 & 0 & 1 & 1 \\ 
      1 & 1 & 1 & 1 & 1  
    \end{array}
  \end{array}
  $$
}
