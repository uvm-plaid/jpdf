\section{Component Certification and Composition}
\label{section-composition}

\begin{fpfig}[t]{2-Party GMW circuit library with And gate (T), and type annotations (B).}{fig-gmw}
{\footnotesize
  \begin{verbatimtab}
    encodegmw(in1,in2) {
      v[1, in1++"out"] := s[1,in1] xor flip[1,in1];
      v[2, in1++"out"] := flip[1,in1];
      v[1, in2++"out"] := s[2,in2] xor flip[2,in2];
      v[2, in2++"out"] := flip[2,in2]
      { shares1 = { c1 = v[1, in1++"out"]; c2 = v[2, in1++"out"] };
        shares2 = { c1 = v[1, in2++"out"]; c2 = v[2, in2++"out"]} } 
    }
    
    andtablegmw(b1, b2) {
      let r11 = (b1 xor true) and (b2 xor true) in
      let r10 = (b1 xor true) and (b2 xor false) in
      let r01 = (b1 xor false) and (b2 xor true) in
      let r00 = (bl xor false) and (b2 xor false) in
      { v1 = r11; v2 = r10; v3 = r01; v4 = r00 }
    }
    
    andgmw(g, shares1, shares2) {
      let r = flip[1,g++".r"] in
      let table = andtablegmw(shares1.c1, shares2.c1) in
      let r11 =  r xor table.v1 in
      let r10 =  r xor table.v2 in
      let r01 =  r xor table.v3 in
      let r00 =  r xor table.v4 in
        v[2,g++"out"] := OT4(shares1.c2, shares2.c2, r11, r10, r01, r00);
      v[1,g++"out"] := r;
      { c1 = v[1,g++"out"]; c2 = v[2,g++"out"]}
    }
    
    decodegmw(shares) { v[0,"output"] := shares.c1 xor shares.c2 }   \end{verbatimtab}
}
\end{fpfig}


The Yao's Garbled Circuits (YGC) and GMW protocols are examples of a
common idiom in MPC protocol design- arbitrarily large \emph{circuits}
are built up from smaller boolean or arithmetic \emph{gates} that
operate on encodings of input secrets. These circuits can become quite
large in practice \cite{kreuter2012billion}, but are built from
well-defined components. Brute force verification techniques as
described in Section \ref{section-bruteforce} are not scalable to
large circuits, but in this Section we will demonstrate that correctly
designed circuit components enjoy compositional properties
that can be automatically certification on small program components
using brute force methods. We prove that these properties preserve
invariants throughout circuit definition that yield $\NIMO$ and thus
scale to arbitrary programs. Furthermore, the certifications can be
used on arbitrary gate definitions, allowing new gate implementations
to be added soundly to libraries.

In this Section we develop and use implementations of both 2-party GMW
and YGC to illustrate our approach and demonstrate its generality.
Main results are in Theorems \ref{theorem-gmw-NIMO} and
\ref{theorem-ygc-NIMO}.

\subsection{Compositional Properties of GMW}

The GMW protocol uses secret sharing to represent data flowing through
circuits. In the 2-party case, clients 1 and 2 each share their input
secrets, and use those shares to represent inputs to gates. Outputs
are also represented as shares. We refer to the pair of shares
representing any particular value as a \emph{wire value}, and
we represent them via records of the form
$
\ttt{\{c1 = } v_1\ttt{;c2 = } v_2\ttt{\}} 
$
where $v_1$ and $v_2$ are client 1 and 2's shares respectively.

For full details of the GMW protocol the reader is referred to
\cite{evans2018pragmatic}. Our implementation libary is shown in
Figure \ref{fig-gmw}, with type signatures for the library functions
shown in Figure \ref{fig-gmw-types}. We show the And gate since it is
an interesting component. The Figure includes the
following top-level functions:
\begin{itemize}
\item \ttt{encodegmw}: This function encodes client 1's and client 2's
  secret bits called $\ttt{s[1,s1]}$ and $\ttt{s[1,s2]}$ into two
  distinct wire values (pairs of shares).
\item \ttt{andgmw}: This function defines the gate $\ttt{g}$, the
  identifier $\ttt{g}$ being used to distinguish randomness used
  within.  In our version client 1 builds the output table (using
  \ttt{andtablegmw})and transfers the correct output share to client 2
  using 1-out-of-4 OT as per standard GMW protocol.
\item \ttt{decode}: This function decodes and publishes a wire value
  by $\ttt{xor}$ing the shares. Note that this requires both client 1
  and 2 to publicize their shares.
\end{itemize}
\begin{example}
  \label{example-gmw-andcircuit}
The following program uses our GMW library to define
a circuit with a single And gate and input secrets $\ttt{s1}$ and
$\ttt{s2}$ from client's 1 and 2 respectively. 
\begin{verbatimtab}
  let ss = encodegmw(s1,s2) in
  v[0,output] := decode(andgmw(0,ss.shares1,ss.shares2))
\end{verbatimtab}
\end{example}

\subsubsection{Gate Certification}

Our certification techniques are defined generally wrt implementations
of input encoding, decoding, and internal gates. In the following we
will refer to arbitrary decode, encode, and gate functions with the
restriction that any function in each of these categories has the same
valid type signature as $\ttt{decodegmw}$, $\ttt{encodegmw}$, and
$\ttt{andgmw}$ respectively. When gates are used in ciruits they are
parameterized by wire values-- in certification we will use carefully
chosen witness parameters. 

When we consider internal components and hence intensional properties
of circuits, we will consider behaviors with respect to particular bits
of randomness. Hence we will consider basic program distributions. 
\begin{definition}
  We write $\progtt(\prog)$ to denote the \emph{basic distribution} of
  $\prog$ as defined in Definition \ref{def-progd}. For any program $\prog =
  (\eassign{v_1}{\be_1};\ldots;\eassign{v_n}{\be_n})$, define
  $\vdefs(\prog) \defeq \{ v_1,\ldots,v_n \}$.
\end{definition}
We also need to formalize notions of probabilistic independence, aka
\emph{separation}, and correlation. We borrow the symbols $*$ and $\sim$
from \cite{barthe2019probabilistic} with the same denotations.
\begin{definition}
  We write $\vc{\pmf}{x}{y}$ iff $\pmf(\{ x \mapsto 0\}\ |\ \{ y \mapsto 0 \}) =
  \pmf(\{ x \mapsto 1\}\ |\ \{ y \mapsto 1 \}) = 1$.
  We write $\sep{\pmf}{Y}{Z}$ iff for all
    $\store \in \mems(X \cup Y)$:
  $$\margd{\pmf}{Y \cup Z}(\store) =
  \pmf(\store_Y) * \pmf(\store_Z)$$ 
\end{definition}

\begin{fpfig}[t]{GMW library type annotations.}{fig-gmw-types}
{\footnotesize
  \begin{verbatimtab}
   encodegmw   : string(gid) * string(gid)
    
   andtablegmw : { k = bool[i]; p = bool[i] }
    
   andgmw      : string(gid) *  { c1 = bool[1]; c2 = bool[2] } * { c1 = bool[1]; c2 = bool[2] }
    
   decodegmw   : { c1 = bool[1]; c2 = bool[2] }  \end{verbatimtab}
}
\end{fpfig}

Now we can define GMW gate certification as follows. Intuitively, we certify
that GMW gates satisfy a probabilistic noninterference property wrt each output
share considered individually. This ensures that input dependencies remain
encrypted from each client's perspective during circuit evaluation.
\begin{definition}[GMW Gate Certification]
  \label{definition-gmwgate-certification}
  Let $\mathit{gate}$ be a gate function and let $e$ be the following
  $\metaprot$ program:
  $$
  \begin{array}{l}
    \ttt{let in1 = \{c1 = flip[1,s11];c2 = flip[2,s12]\} in}\\
    \ttt{let in2 = \{c1 = flip[1,s21];c2 = flip[2,s22]\} in}\\
    \mathit{gate}\ttt{(g,in1,in2)}
  \end{array}
  $$
  and let $
  \config{\varnothing}{e}\redxs\config{\prog}{\ttt{\{c1 = v[1,gout]; c2 = v[2,gout]\}}}
  $.
  Then $\mathit{gate}$ is \emph{certified} iff the following conditions hold:
  \begin{enumerate}[\hspace{5mm}i.]
  \item $\sep{\progtt(\prog)}{\ttt{\{flip[1,s11],flip[1,s21],flip[2,s12],flip[2,s22]\}}}{\ttt{\v[1,gout]\}}}$
  \item $\sep{\progtt(\prog)}{\ttt{\{flip[1,s11],flip[1,s21],flip[2,s12],flip[2,s22]\}}}{\ttt{\v[2,gout]\}}}$
  % \item $\sep{\progtt(\prog)}{\ttt{\{v[1,gout]\}}}{\ttt{\{v[2,gout]\}}}$
  \item All variables in:$$\vars(\prog) - \{ \ttt{flip[1,s11],flip[1,s21],flip[2,s12],flip[2,s22]}\}$$
    are distinguished by $\ttt{g}$ (i.e., contain $\ttt{g}$ as an identifier substring). 
  \end{enumerate}
\end{definition}

Encoding certification establishes the desired property in the protocol-- that
input secrets of each client are independent of the other client's views. 
\begin{definition}[GMW Encode Certification]
  \label{definition-gmwencode-certification}
  Let $\mathit{encode}$ be an encoding function, and let:
  $$
  \config{\varnothing}{\ttt{encode(s1,s2)}} \redxs
  \config{\prog}{\ttt{\{shares1 = }v_1\ttt{; shares2 = }v_2\ttt{\}}}
  $$
  where:
  $$
  \begin{array}{rcl}
    v_1 &=& \ttt{\{ c1 = v[1,s1out]; c2 = v[2,s1out] \}}\\
    v_2 &=& \ttt{\{ c1 = v[1,s2out]; c2 = v[2,s2out] \}}
  \end{array}
  $$
  Then $\mathit{encode}$ is \emph{certified} iff each of the following conditions hold:
  \begin{enumerate}[\hspace{5mm}i.]
    %\item $\sep{\progtt(\prog)}{\ttt{\{ s[1,s1],  v[1,s1out], v[1,s2out] \}}}{\ttt{\{ s[2,s2], v[2,s1out], v[2,s2out] \}}}$
  \item $\sep{\progtt(\prog)}{\ttt{\{s[1,s1]\}}}{\ttt{\{v[2,s1out],v[2,s2out]\}}}$
  \item $\sep{\progtt(\prog)}{\ttt{\{s[2,s2]\}}}{\ttt{\{v[1,s1out],v[1,s2out]\}}}$
  \item All $x \in \vars(\prog)$ are distinguished by $\ttt{in1}$ or $\ttt{in2}$. 
  \end{enumerate}
  \end{definition}

We have certified both $\ttt{encodegmw}$ and $\ttt{andgmw}$. Note that
an Or or Xor gate, for example, could be certified with this same
method to extend our library.

\subsubsection{Gate Invariants} We can show that composition maintains an
important invariant in a circuit $\prog$- namely that client 1's
secrets remain independent of client 2's views, and vice-versa, prior
to decoding.  That is, where $\iov(\prog) = S \cup V$, we show that
$\sep{\progd(\prog)}{S_{\{1\}}}{V_{\{2\}}}$ and
$\sep{\progd(\prog)}{S_{\{2\}}}{V_{\{1\}}}$ are preserved prior to
decoding.

We observe the following properties of separation which are borrowed from prior work
\cite{barthe2019probabilistic} and which will be used frequently in proofs. 
\begin{lemma}
  \label{lemma-separation}
  The following properties hold:
  \begin{enumerate}
  \item $\sep{\pmf}{Y}{Z}$ iff $\sep{\pmf}{Z}{Y}$
  \item $\vc{\pmf}{x}{y}$ if  $\vc{\pmf}{y}{x}$
  \item $\vc{\pmf}{x}{y}$ and $\vc{\pmf}{y}{z}$ imply $\vc{\pmf}{x}{z}$
  \item $\sep{\pmf}{X}{(Y \cup Z)}$ implies $(\sep{\pmf}{X}{Y}$ and $\sep{\pmf}{X}{Z})$
  \item $(\sep{\pmf}{X}{Y}$ and $\sep{\pmf}{(X \cup Y)}{Z})$ implies $\sep{\pmf}{X}{(Y \cup Z)}$ 
  %\item If $\prog_1;\prog_2$ is safe and $\vars(\prog_1) \cap \vars(\prog_2) = \varnothing$
  %  then $\sep{\progtt(\prog_1;\prog_2)}{\vars(\prog_1)}{\vars(\prog_2)}$.
  \end{enumerate}
\end{lemma}
The following property also follows by results in \cite{barthe2019probabilistic} and will
be useful to make constructions that demonstrate variable separation.
\begin{lemma}
  \label{lemma-sepjoin}
  $\sep{\progtt(\prog)}{X}{Y}$ iff for all 
  $\store^1, \store^2 \in \runs(\prog)$ there exists
  $\store \in \runs(\prog)$ with
  $\store^1_{X} \cap \store^2_{Y} \subseteq \store$.
\end{lemma}

%The following Lemma allows us to separate variables in programs that may not occur
%in the same subprogram, and hence consider gate certification properties in various
%contexts. It is reminiscent of the \emph{frame} rule of \cite{barthe2019probabilistic}.
%\begin{lemma}
%  \label{lemma-gmw-frame}
%  If $\sep{\progtt(\prog_1;\prog_2)}{Y}{V}$ for $Y = \vars(\prog_1) \cap \vars(\prog_2)$ and
%  $V \in \vdefs(\prog_2)$ and $\sep{\progtt(\prog_1)}{X}{Y}$ for
%  $X \not\in \vars(\prog_2)$ then $\sep{\progtt(\prog_1;\prog_2)}{X}{V}$. 
%\end{lemma}
%\begin{proof}
%  By assumption and Lemma \ref{lemma-sepjoin}, for all
%  $\store^1,\store^2 \in \runs(\prog_1;\prog_2)$ we have $\store^1_Y
%  \cap \store^2_V \subseteq \store_a$ for some $\store_a \in
%  \runs(\prog_1;\prog_2)$ and $\store^1_X \cap \store^2_Y \subseteq
%  \store_b$ for some $\store_b \in \runs(\prog_1;\prog_2)$.  Thus, since
%  we assume $X \not\in \vars(\prog_2)$ we have $\store_a \cap \store_b
%  \in \runs(\prog_1;\prog_2)$ and $\store^1_X \cap \store^2_V =
%  \store_a \cap \store_b$, obtaining the result by
%  \ref{lemma-sepjoin}.
%\end{proof}

The next ``noninterference'' Lemma is important to prove the desired invariance
preservation property of GMW gates, given certification.
\begin{lemma}[Noninterference]
  \label{lemma-noninterference}
  Given $\prog_1;\prog_2$ and $X = \iov(\prog_2) - \vdefs(\prog_2)$ and
  $Y \subseteq \vdefs(\prog_2)$. If $\sep{\progd(\prog_1;\prog_2)}{X}{Y}$
  then $\sep{\progd(\prog_1;\prog_2)}{\iov(\prog_1)}{Y}$.
\end{lemma}

The next two Lemmas allow us to use the properties of gates established by
certification and replace the ``dummy'' parameters used there with actual
wire values in full circuit contexts. Intuitively we show that substitution
for dummy flips preserves separation.
\begin{lemma}
  \label{lemma-presub}
  Given $\be_1$, $\be_2$, and $i$ where $\vars(\be_1) \cap
  \vars(\be_2) = \varnothing$ and $\lcod{\store_1,\be_1}{i} = \beta$
  with $\store_1(x) = \lcod{\store_2,\be_2}{i}$ and
  $\dom(\store_1) = \vars(\be_1)$ and $\dom(\store_2) = \vars(\be_2)$.
  Then $\lcod{\store_1\cap\store_2,\be_1[\be_2/x]}{i} = \beta$.
\end{lemma}
\begin{proof}
By straightforward structural induction on $\be$.
\end{proof}

\begin{lemma}[Substitution]
  \label{lemma-substitution}
  If $\sep{\progtt(\prog_2)}{\{ f \}}{X}$ and
  $\prog_1;\prog_2[\be/f]$ is safe with $\vars(\prog_1,\be) \cap
  X = \varnothing$ then
  $\sep{\progtt(\prog_1;\prog_2[\be/f])}{\vars(\be)}{X}$.
\end{lemma}
\begin{proof}
  Suppose on the contrary it was not the case that
  $\sep{\progtt(\prog_1;\prog_2[\be/f])}{\vars(\be)}{X}$.
  Then by Lemma \ref{lemma-sepjoin} there exists $\store_1,\store_2
  \in \runs(\prog_1;\prog_2[\be/f])$ with no $\store \in
  \runs(\prog_1;\prog_2[\be/f])$ such that $\store^1_{\vars(\be)}
  \cap \store^2_X \subseteq \store$.  But also by
  assumption and Lemma \ref{lemma-sepjoin} for any $\beta$ there
  exists $\store' \in \runs(\prog_2)$ with $\{ f \mapsto \beta \} \cap
  \store^2_X \subseteq \store'$. So in particular, we
  have $\{ f \mapsto \lcod{\store^1_{\vars(\be)},\be}{i} \} \cap
  \store^2_X \subseteq \store'$ for any $i$. This, the
  assumption $\vars(\prog_1,\be) \cap \vars(\prog_2) = \varnothing$,
  and application of Lemma \ref{lemma-presub} leads to the consequence
  that there exists $\store \in \runs(\prog_1;\prog_2[\be/f])$ with
  $\store \supseteq \store^1_{\vars(\be)} \cap
  \store^2_X$ given the assumption $\vars(\prog_1,\be)
  \cap \vars(\prog_2) = \varnothing$, which is a contradiction.
\end{proof}

Now we can prove that certified gates preserve the desired invariant
during execution in arbitrary circuit contexts.
We formulate this in terms of pre-~and post-conditions of gate
evaluation.
\begin{lemma}[GMW Gate Invariant]
  \label{lemma-gmw-preservation}
  Given well-typed $\prog_1;E[\mathit{gate}(\ttt{g},v_1,v_2)]$ for certified $\mathit{gate}$ where:
  $$
  \config{\prog_1}{E[\mathit{gate}(\ttt{g},v_1,v_2)]} \redxs \config{\prog_1;\prog_2}{E[v]}
  $$
  If we have the following preconditions, where $\iov(\prog_1) = S \cup V$:
  \begin{enumerate}[\hspace{5mm}i.]
  \item $\sep{\progd(\prog_1)}{S_{\{2\}}}{V_{\{1\}}}$
  \item $\sep{\progd(\prog_1)}{S_{\{1\}}}{V_{\{2\}}}$
  \end{enumerate}
  then we have as a postconditions, where $\iov(\prog_1;\prog_2) = S \cup V'$:
  \begin{enumerate}[\hspace{5mm}i.]
  \item $\sep{\progd(\prog_1;\prog_2)}{S_{\{2\}}}{V'_{\{1\}}}$
  \item $\sep{\progd(\prog_1;\prog_2)}{S_{\{1\}}}{V'_{\{2\}}}$
  \end{enumerate}
\end{lemma}

\begin{proof}
  Given well-typedness of $\prog_1;E[\mathit{gate}(\ttt{g},v_1,v_2)]$, we have for some $\ttt{g1}$ and $\ttt{g2}$:
  $$
  \begin{array}{lcl}
   v_1 &=& \ttt{\{c1 = v[1,g1out]; c2 = v[2,g1out]\}}\\
   v_2 &=& \ttt{\{c1 = v[1,g2out]; c2 = v[2,g2out]\}}\\
    v &=& \ttt{\{c1 = v[1,gout]; c2 = v[2,gout]\}}
  \end{array}
  $$
  Let $\prog$ be as defined in Definition \ref{definition-gmwgate-certification}.
  We observe:
  $$
  {\footnotesize
    \begin{array}{c}
      \prog_2 = \\
      \prog[\ttt{v[1,g1out]}/\ttt{flip[1,s11]}][\ttt{v[2,g1out]}/\ttt{flip[2,s12]}][\ttt{v[1,g2out]}/\ttt{flip[1,s21]}][\ttt{v[2,g2out]}/\ttt{flip[2,s22]}]
    \end{array}
  }
  $$
  Now, the assumption of well-typedness also assures that $\ttt{g}$
  has not been previously used as a gate identifier in $\prog_1$,
  since otherwise the views $\ttt{v[1,gout]}$ and $\ttt{v[2,gout]}$
  would have been previously defined. This ensures $\vars(\prog_1,x)
  \cap \vars(\prog) = \varnothing$ for:
  $$x \in \{ \ttt{v[1,g1out]}, \ttt{v[2,g1out]}, \ttt{v[1,g2out]}, \ttt{v[2,g2out]}\}$$
  given condition (iii) of Definition \ref{definition-gmwgate-certification}.
  Thus by Lemma \ref{lemma-substitution} we have:
  $$\sep{\progtt(\prog_1;\prog_2)}{\ttt{\{v[1,g1out],v[1,g2out],v[2,g1out],v[2,g2out]\}}}{\ttt{\{v[1,gout]\}}}$$
  and
  $$\sep{\progtt(\prog_1;\prog_2)}{\ttt{\{v[1,g1out],v[1,g2out],v[2,g1out],v[2,g2out]\}}}{\ttt{\{v[2,gout]\}}}$$
  Now, by Lemmas \ref{lemma-noninterference} and \ref{lemma-separation} and condition (iii) of Definition
  \ref{definition-gmwgate-certification} we then have:
  $$\sep{\progtt(\prog_1;\prog_2)}{(S_{\{1\}} \cup V_{\{2\}})}{\ttt{\{v[2,gout]\}}} \quad \text{and} \quad
  \sep{\progtt(\prog_1;\prog_2)}{(S_{\{2\}} \cup V_{\{1\}})}{\ttt{\{v[1,gout]\}}}$$
  so by precondition assumptions and Lemma \ref{lemma-separation} we have:
  $$\sep{\progtt(\prog_1;\prog_2)}{S_{\{1\}}}{V'_{\{2\}}} \quad \text{and} \quad
  \sep{\progtt(\prog_1;\prog_2)}{S_{\{2\}}}{V'_{\{1\}}}$$
\end{proof}

Here we show that encoding also preserves the desired invariant.
The result here is more straightforward than it is for gates, because
the variables used during encoding are guaranteed to be distinct from
the rest of the program.
\begin{lemma}[GMW Encode Invariant]
  \label{lemma-gmw-encode}
  Given well-typed $\prog_1;E[\mathit{encode}\ttt{(s1,s2)}]$ for certified $\mathit{encode}$, let:
  $$
  \config{\prog_1}{E[\mathit{encode}(\ttt{s1},\ttt{s2})]} \redxs
  \config{\prog_1;\prog_2}{E[v]}
  $$
  If we have the following preconditions, where $\iov(\prog_1) = S' \cup V$:
  \begin{enumerate}[\hspace{5mm}i.]
  \item $\sep{\progd(\prog_1)}{S_{\{2\}}}{V_{\{1\}}}$
  \item $\sep{\progd(\prog_1)}{S_{\{1\}}}{V_{\{2\}}}$
  \end{enumerate}
  then we have as a postconditions, where $\iov(\prog_1;\prog_2) = S' \cup V'$: 
  \begin{enumerate}[\hspace{5mm}i.]
  \item $\sep{\progd(\prog_1;\prog_2)}{S'_{\{2\}}}{V'_{\{1\}}}$
  \item $\sep{\progd(\prog_1;\prog_2)}{S'_{\{1\}}}{V'_{\{2\}}}$
  \end{enumerate}
\end{lemma}

\begin{proof}
  By assumptions of well-typedness we have:
  $$
  \begin{array}{rcl}
    v &=&  \ttt{\{shares1 = }v_1\ttt{; shares2 = }v_2\ttt{\}}\\
    v_1 &=& \ttt{\{ c1 = v[1,s1out]; c2 = v[2,s1out] \}}\\
    v_2 &=& \ttt{\{ c1 = v[1,s2out]; c2 = v[2,s2out] \}}
  \end{array}
  $$
  And by Definition \ref{definition-gmwencode-certification} we have:
  \begin{mathpar}
    \sep{\progtt(\prog)}{\ttt{\{s[1,s1]\}}}{\ttt{\{v[2,s1out],v[2,s2out]\}}}
    
    \sep{\progtt(\prog)}{\ttt{\{s[2,s2]\}}}{\ttt{\{v[1,s1out],v[1,s2out]\}}}
  \end{mathpar}
  Since we assume that secrets are in uniform and independent marginal
  distributions a priori, and $\vars(\prog_1) \cap \vars(\prog_2) =
  \varnothing$ by condition (iii) of Definition 
  \ref{definition-gmwencode-certification} and assumptions of well-typedness, 
  the result follows by preconditions and Lemma \ref{lemma-separation}.
\end{proof}

On the basis of the preceding we can prove our main result as
follows. Note that we assume a normal form of programs where the last
instruction is a decoding and public reveal of the output.
\begin{theorem}
  \label{theorem-gmw-NIMO}
  If $\eassign{\outv}{\ttt{decode}(e)}$ is a well-typed GMW circuit
  definition using certified components and
  $\config{\varnothing}{\eassign{\outv}{\ttt{decode}(e)}} \redxs
  \config{\prog}{\varnothing}$, then $\NIMO(\prog)$.
\end{theorem}
\begin{proof}
  Let $\iov(\prog) = S \cup (V \cup \outv)$. By definition of $\ttt{decode}$
  and assumptions of well-typedness we have for some $\ttt{g}$:
  $$
  \config{\varnothing}{\eassign{\outv}{\ttt{decode}(e)}} \redxs
  \config{\prog;\eassign{\outv}{\ttt{v[1,gout] xor v[2,gout]}}}{\varnothing}
  $$
  where by Lemmas \ref{lemma-gmw-encode} and \ref{lemma-gmw-preservation} we
  have $\sep{\progd(\prog)}{S_{\{1\}}}{V_{\{2\}}}$ and
  $\sep{\progd(\prog)}{S_{\{2\}}}{V_{\{1\}}}$.
  Let $\prog' \defeq \prog;\eassign{\outv}{\ttt{v[1,gout] xor v[2,gout]}}$. The
  preceding then implies
  $\sep{\progd(\prog')}{S_{\{1\}}}{V_{\{2\}}}$ and
  $\sep{\progd(\prog')}{S_{\{2\}}}{V_{\{1\}}}$,
  so that for all $\beta$:
  $$\sep{\condd{\progd(\prog')}{S_{\{1\}} \cup V_{\{2\}}}{\{ \outv \mapsto \beta \}}}{S_{\{1\}}}{V_{\{2\}}}
  \quad \text{and} \quad
    \sep{\condd{\progd(\prog')}{S_{\{2\}} \cup V_{\{1\}}}{\{ \outv \mapsto \beta \}}}{S_{\{2\}}}{V_{\{1\}}}$$
  thus for all $\store \in \mems(V_2 \cup \{ \outv \})$:
  $$\condd{\progd(\prog')}{S_{\{1\}}}{\store}
  = \condd{\progd(\prog')}{S_{\{1\}}}{\store_{\{\outv\}}}$$
  and for all $\store \in \mems(V_1 \cup \{ \outv \})$:
  $$\condd{\progd(\prog')}{S_{\{2\}}}{\store}
  = \condd{\progd(\prog')}{S_{\{2\}}}{\store_{\{\outv\}}}$$
  establishing the result by Lemma \ref{lemma-nimo} and Definition \ref{definition-NIMO}. 
\end{proof}
