\section{Certification and Invariants for Circuit Security}
\label{section-composition}

Here we show how to apply the certification method described in
Section \ref{section-automation-intensional} to obtain passive
security for any 2-party GMW or YGC circuits built using the libraries
described in Sections \ref{section-metalang-gmw} and
\ref{section-metalang-ygc}.  Main results are in Theorems
\ref{theorem-gmw-NIMO} and \ref{theorem-ygc-NIMO}.

Our certification techniques are defined generally wrt implementations
of input encoding, decoding, and internal gates for both GMW and YGC
circuits. In the following we will refer to arbitrary decode, encode,
and gate functions with the restriction that any function in each of
these categories has the same valid type signatures as their existing
instantiations in our GMW and YGC libaries. All certifications can be
automatically verified as a consequence of Lemmas \ref{lemma-cprogtt}
and \ref{lemma-autosep}, and complete implementations of the
certifications are available in our repository \cite{jpdf-github}. 

We also demonstrate that intenstional properties of certified
components maintain a crucial separation invariant within circuits
using methods developed in Section \ref{section-automation-logic}.
Specifically, we show that within a circuit $\prog$, the views of each
client remain probabilistically independent of the other client's
secrets, i.e., $\sep{\progd(\prog)}{S_{\{1\}}}{V_{\{2\}}}$ and
%\quad \text{and} \quad
$\sep{\progd(\prog)}{S_{\{2\}}}{V_{\{1\}}}$.
Through certification of decodings we can finally obtain $\NIMO$ in
circuits as in Theorems \ref{theorem-gmw-NIMO} and
\ref{theorem-ygc-NIMO}.

\subsection{GMW Component Certification}

In keeping with the standard definition of GMW, we can certify that
GMW gates satisfy a probabilistic noninterference property wrt each
output share considered individually. This ensures that input
dependencies remain encrypted from each client's perspective during
circuit evaluation.
\begin{definition}[GMW Gate Certification]
  \label{definition-gmwgate-certification}
  Let $\mathit{gate}$ be a gate function and let $e$ be the following
  $\metaprot$ program:
  $$
  \begin{array}{l}
    \ttt{let in1 = \{c1 = flip[1,"s11"];c2 = flip[2,"s12"]\} in}\\
    \ttt{let in2 = \{c1 = flip[1,"s21"];c2 = flip[2,"s22"]\} in}\\
    \mathit{gate}\ttt{("g",in1,in2)}
  \end{array}
  $$
  and let $
  \config{\varnothing}{e}\redxs\config{\prog}{\ttt{\{c1 = v[1,"gout"]; c2 = v[2,"gout"]\}}}
  $.
  Then $\mathit{gate}$ is \emph{certified} iff the following conditions hold:
  \begin{enumerate}[\hspace{5mm}i.]
  \item {\small$\sep{\progtt(\prog)}{\ttt{\{flip[1,"s11"],flip[1,"s21"],flip[2,"s12"],flip[2,"s22"]\}}}{\ttt{v[1,"gout"]\}}}$}
  \item {\small$\sep{\progtt(\prog)}{\ttt{\{flip[1,"s11"],flip[1,"s21"],flip[2,"s12"],flip[2,"s22"]\}}}{\ttt{v[2,"gout"]\}}}$}
  % \item $\sep{\progtt(\prog)}{\ttt{\{v[1,gout]\}}}{\ttt{\{v[2,gout]\}}}$
  \item All variables in:$$\vars(\prog) - \{ \ttt{flip[1,"s11"],flip[1,"s21"],flip[2,"s12"],flip[2,"s22"]}\}$$
    are distinguished by $\ttt{"g"}$ (i.e., contain $\ttt{"g"}$ as an identifier substring). 
  \end{enumerate}
\end{definition}

Encoding certification establishes the desired property in the protocol-- that
input secrets of each client are independent of the other client's views. 
\begin{definition}[GMW Encode Certification]
  \label{definition-gmwencode-certification}
  Let $\mathit{encode}$ be an encoding function, and let:
  $$
  \config{\varnothing}{\mathit{encode}\ttt{("s1","s2")}} \redxs
  \config{\prog}{\ttt{\{shares1 = }v_1\ttt{; shares2 = }v_2\ttt{\}}}
  $$
  where:
  $$
  \begin{array}{rcl}
    v_1 &=& \ttt{\{ c1 = v[1,"s1out"]; c2 = v[2,"s1out"] \}}\\
    v_2 &=& \ttt{\{ c1 = v[1,"s2out"]; c2 = v[2,"s2out"] \}}
  \end{array}
  $$
  Then $\mathit{encode}$ is \emph{certified} iff each of the following conditions hold:
  \begin{enumerate}[\hspace{5mm}i.]
    %\item $\sep{\progtt(\prog)}{\ttt{\{ s[1,s1],  v[1,s1out], v[1,s2out] \}}}{\ttt{\{ s[2,s2], v[2,s1out], v[2,s2out] \}}}$
  \item $\sep{\progtt(\prog)}{\ttt{\{s[1,"s1"]\}}}{\ttt{\{v[2,"s1out"],v[2,"s2out"]\}}}$
  \item $\sep{\progtt(\prog)}{\ttt{\{s[2,"s2"]\}}}{\ttt{\{v[1,"s1out"],v[1,"s2out"]\}}}$
  \item All $x \in \vars(\prog)$ are distinguished by $\ttt{"s1"}$ or $\ttt{"s2"}$. 
  \end{enumerate}
  \end{definition}

We have certified both $\ttt{encodegmw}$ and $\ttt{andgmw}$. Note that
an Or or Xor gate, for example, could be certified with this same
method to extend our library.
\begin{lemma}
  \label{lemma-gmw-certification}
  Each of $\ttt{encodegmw}$ and $\ttt{andgmw}$ are certified.
\end{lemma}

\subsection{GMW Gate Invariants and Circuit Security}
We can show that composition maintains an important invariant in a
circuit $\prog$- namely that client 1's secrets remain independent of
client 2's views, and vice-versa, prior to decoding.  That is, where
$\iov(\prog) = S \cup V$, we show that
$\sep{\progd(\prog)}{S_{\{1\}}}{V_{\{2\}}}$ and
$\sep{\progd(\prog)}{S_{\{2\}}}{V_{\{1\}}}$ are preserved prior to
decoding.

Now we can prove that certified gates preserve the desired invariant
during execution in arbitrary circuit contexts.
We formulate this in terms of pre-~and post-conditions of gate
evaluation.
\begin{lemma}[GMW Gate Invariant]
  \label{lemma-gmw-preservation}
  Given well-typed $\prog_1;E[\mathit{gate}(\ttt{"g"},v_1,v_2)]$ for certified $\mathit{gate}$ where:
  $$
  \config{\prog_1}{E[\mathit{gate}(\ttt{"g"},v_1,v_2)]} \redxs \config{\prog_1;\prog_2}{E[v]}
  $$
  If we have the following preconditions, where $\iov(\prog_1) = S \cup V$:
  \begin{enumerate}[\hspace{5mm}i.]
  \item $\sep{\progd(\prog_1)}{S_{\{2\}}}{V_{\{1\}}}$
  \item $\sep{\progd(\prog_1)}{S_{\{1\}}}{V_{\{2\}}}$
  \end{enumerate}
  then we have as a postconditions, where $\iov(\prog_1;\prog_2) = S \cup V'$:
  \begin{enumerate}[\hspace{5mm}i.]
  \item $\sep{\progd(\prog_1;\prog_2)}{S_{\{2\}}}{V'_{\{1\}}}$
  \item $\sep{\progd(\prog_1;\prog_2)}{S_{\{1\}}}{V'_{\{2\}}}$
  \end{enumerate}
\end{lemma}

\begin{proof}
  Given well-typedness of $\prog_1;E[\mathit{gate}(\ttt{g},v_1,v_2)]$, we have for some $\ttt{"g1"}$ and $\ttt{"g2"}$:
  $$
  \begin{array}{lcl}
   v_1 &=& \ttt{\{c1 = v[1,"g1out"]; c2 = v[2,"g1out"]\}}\\
   v_2 &=& \ttt{\{c1 = v[1,"g2out"]; c2 = v[2,"g2out"]\}}\\
    v &=& \ttt{\{c1 = v[1,"gout"]; c2 = v[2,"gout"]\}}
  \end{array}
  $$
  Let $\prog$ be as defined in Definition \ref{definition-gmwgate-certification}.
  We observe:
  $$
  {\footnotesize
    \begin{array}{c}
      \prog_2 = \\
      \prog[\ttt{v[1,"g1out"]}/\ttt{flip[1,"s11"]}][\ttt{v[2,"g1out"]}/\ttt{flip[2,"s12"]}][\ttt{v[1,"g2out"]}/\ttt{flip[1,"s21"]}][\ttt{v[2,"g2out"]}/\ttt{flip[2,"s22"]}]
    \end{array}
  }
  $$
  Now, the assumption of well-typedness also assures that $\ttt{"g"}$
  has not been previously used as a gate identifier in $\prog_1$,
  since otherwise the views $\ttt{v[1,"gout"]}$ and $\ttt{v[2,"gout"]}$
  would have been previously defined. This ensures $\vars(\prog_1,x)
  \cap \vars(\prog) = \varnothing$ for:
  $$x \in \{ \ttt{v[1,"g1out"]}, \ttt{v[2,"g1out"]}, \ttt{v[1,"g2out"]}, \ttt{v[2,"g2out"]}\}$$
  given condition (iii) of Definition \ref{definition-gmwgate-certification}.
  Thus by Lemma \ref{lemma-substitution} we have:
  $$\sep{\progtt(\prog_1;\prog_2)}{\ttt{\{v[1,"g1out"],v[1,"g2out"],v[2,"g1out"],v[2,"g2out"]\}}}{\ttt{\{v[1,"gout"]\}}}$$
  and
  $$\sep{\progtt(\prog_1;\prog_2)}{\ttt{\{v[1,"g1out"],v[1,"g2out"],v[2,"g1out"],v[2,"g2out"]\}}}{\ttt{\{v[2,"gout"]\}}}$$
  Now, by Lemmas \ref{lemma-noninterference} and \ref{lemma-separation} and condition (iii) of Definition
  \ref{definition-gmwgate-certification} we then have:
  $$\sep{\progtt(\prog_1;\prog_2)}{(S_{\{1\}} \cup V_{\{2\}})}{\ttt{\{v[2,"gout"]\}}} \quad \text{and} \quad
  \sep{\progtt(\prog_1;\prog_2)}{(S_{\{2\}} \cup V_{\{1\}})}{\ttt{\{v[1,"gout"]\}}}$$
  so by precondition assumptions and Lemma \ref{lemma-separation} we have:
  $$\sep{\progtt(\prog_1;\prog_2)}{S_{\{1\}}}{V'_{\{2\}}} \quad \text{and} \quad
  \sep{\progtt(\prog_1;\prog_2)}{S_{\{2\}}}{V'_{\{1\}}}$$
\end{proof}

Here we show that encoding also preserves the desired invariant.
The result here is more straightforward than it is for gates, because
the variables used during encoding are guaranteed to be distinct from
the rest of the program.
\begin{restatable}[GMW Encode Invariant]{lemma}{gmwencode}
  \label{lemma-gmw-encode}
  Given well-typed $\prog_1;E[\mathit{encode}\ttt{(s1,s2)}]$ for certified $\mathit{encode}$, let:
  $$
  \config{\prog_1}{E[\mathit{encode}(\ttt{s1},\ttt{s2})]} \redxs
  \config{\prog_1;\prog_2}{E[v]}
  $$
  If we have the following preconditions, where $\iov(\prog_1) = S' \cup V$:
  \begin{enumerate}[\hspace{5mm}i.]
  \item $\sep{\progd(\prog_1)}{S_{\{2\}}}{V_{\{1\}}}$
  \item $\sep{\progd(\prog_1)}{S_{\{1\}}}{V_{\{2\}}}$
  \end{enumerate}
  then we have as a postconditions, where $\iov(\prog_1;\prog_2) = S' \cup V'$: 
  \begin{enumerate}[\hspace{5mm}i.]
  \item $\sep{\progd(\prog_1;\prog_2)}{S'_{\{2\}}}{V'_{\{1\}}}$
  \item $\sep{\progd(\prog_1;\prog_2)}{S'_{\{1\}}}{V'_{\{2\}}}$
  \end{enumerate}
\end{restatable}

On the basis of the preceding, we can prove our main result as
follows. Note that we assume a normal form of programs where the last
instruction is a decoding and public reveal of the output.
\begin{theorem}
  \label{theorem-gmw-NIMO}
  If $\eassign{\outv}{\ttt{decode}(e)}$ is a well-typed GMW circuit
  definition using certified components and
  $\config{\varnothing}{\eassign{\outv}{\ttt{decode}(e)}} \redxs
  \config{\prog}{\varnothing}$, then $\NIMO(\prog)$.
\end{theorem}
\begin{proof}
  Let $\iov(\prog) = S \cup (V \cup \outv)$. By definition of $\ttt{decode}$
  and assumptions of well-typedness we have for some $\ttt{g}$:
  $$
  \config{\varnothing}{\eassign{\outv}{\ttt{decode}(e)}} \redxs
  \config{\prog;\eassign{\outv}{\ttt{v[1,"gout"] xor v[2,"gout"]}}}{\varnothing}
  $$
  where by Lemmas \ref{lemma-gmw-encode} and \ref{lemma-gmw-preservation} we
  have $\sep{\progd(\prog)}{S_{\{1\}}}{V_{\{2\}}}$ and
  $\sep{\progd(\prog)}{S_{\{2\}}}{V_{\{1\}}}$.
  Let $\prog' \defeq \prog;\eassign{\outv}{\ttt{v[1,"gout"] xor v[2,"gout"]}}$. The
  preceding then implies
  $\sep{\progd(\prog')}{S_{\{1\}}}{V_{\{2\}}}$ and
  $\sep{\progd(\prog')}{S_{\{2\}}}{V_{\{1\}}}$,
  so that for all $\beta$:
  $$\sep{\condd{\progd(\prog')}{S_{\{1\}} \cup V_{\{2\}}}{\{ \outv \mapsto \beta \}}}{S_{\{1\}}}{V_{\{2\}}}
  \quad \text{and} \quad
    \sep{\condd{\progd(\prog')}{S_{\{2\}} \cup V_{\{1\}}}{\{ \outv \mapsto \beta \}}}{S_{\{2\}}}{V_{\{1\}}}$$
  thus for all $\store \in \mems(V_2 \cup \{ \outv \})$:
  $$\condd{\progd(\prog')}{S_{\{1\}}}{\store}
  = \condd{\progd(\prog')}{S_{\{1\}}}{\store_{\{\outv\}}}$$
  and for all $\store \in \mems(V_1 \cup \{ \outv \})$:
  $$\condd{\progd(\prog')}{S_{\{2\}}}{\store}
  = \condd{\progd(\prog')}{S_{\{2\}}}{\store_{\{\outv\}}}$$
  establishing the result by Lemma \ref{lemma-nimo} and Definition \ref{definition-NIMO}. 
\end{proof}

\subsection{YGC Component Certification}
\label{section-ygc-certification}

Our certification techniques are defined generally wrt implementations
of input encoding, decoding, and internal gates. In the following we
will refer to arbitrary decode, encode, and gate functions with the
restriction that any function in each of these categories has the same
valid type signature as $\ttt{decode}$, $\ttt{encode}$, and
$\ttt{andgg}$ respectively.

It is necessary to define notions of correlation with wire
labels, since wire values flow into and out of gates in either
positive of negative correlation with them.
\begin{definition}
  \label{definition-gc}
  We write $\gc{\prog}{\{ \ttt{k = }\be_{k}; \ttt{p = }\be_{p} \}}{g}$ iff one of the
  following hold:
  \begin{enumerate}[\hspace{5mm}i.]
  \item $\vc{\progtt(\prog,\be_k)}{\itv}{\ttt{flip[2,gate:}g\ttt{.k]}}\quad \text{and} \quad
    \vc{\progtt(\prog,\be_p)}{\itv}{\ttt{flip[2,gate:}g\ttt{.p]}}$
  \item $\vc{\progtt(\prog,\enot\ \be_k)}{\itv}{\ttt{flip[2,gate:}g\ttt{.k]}}\quad \text{and}\quad
    \vc{\progtt(\prog,\enot\ \be_p)}{\itv}{\ttt{flip[2,gate:}g\ttt{.p]}}$
  \end{enumerate}
\end{definition}
Thus, if a value is in correlation with a wire label, it is either
correlated with the wire label or with the inverse of both the key and
pointer bits. For subsequent discussion, we make the following addition
to the codebase:
$$
\ttt{invert(\{ k = bk; p = bp \}) \{ \{ k = not bk; p = not bp \} \}}
$$
Gate certification establishes separation of input labels from the
gate table and output value, and correlation of the output with
the output wire label under any input wire value combination.
\begin{definition}[YGC Gate Certification]
  \label{definition-ygcgate-certification}
  Let $\mathit{gate}$ be a gate function.
  For $1 \le i \le 4$  let
  $
  \config{\varnothing}{e_i}\redxs\config{\prog}{v_i}
  $
  where:
  \begin{enumerate}[\hspace{5mm}$e_1 \defeq$]
    \item $\mathit{gate}\ttt{(c,a,b,owl(a),owl(b));}$
    \item $\mathit{gate}\ttt{(c,a,b,owl(a),invert(owl(b)));}$
    \item $\mathit{gate}\ttt{(c,a,b,invert(owl(a)),owl(b));}$
    \item $\mathit{gate}\ttt{(c,a,b,invert(owl(a)),invert(owl(b)))}$
  \end{enumerate}
  Define:
  $${\small X \defeq \ttt{\{ flip[2,gate:a.k], flip[2,gate:a.p], flip[2,gate:b.k], flip[2,gate:b.p] \}}}$$
  Then $\mathit{gate}$ is \emph{certified} iff the following conditions hold:
  \begin{enumerate}[\hspace{5mm}i.]
  \item $\sep{\progtt(\prog)}{X}{\vdefs(\prog)}$
  \item For all $1 \le i \le 4$, $\gc{\prog}{v_i}{\ttt{c}}$
  \end{enumerate}
\end{definition}
Encode certification establishes correlation of the encoded input values with
the secret wire labels, and separation of client 2's secret from client 1's views. 
\begin{definition}[YGC Encode Certification]
  \label{definition-ygcencode-certification}
  Let $\mathit{encode}$ be an encoding function, and let:
  $$
  \config{\varnothing}{\ttt{encode(s1,s2)}} \redxs
  \config{\prog}{\ttt{\{wv1=}v_1\ttt{;wv2=}v_2\ttt{\}}}
  $$
  Then $\mathit{encode}$ is \emph{certified} iff each of the following conditions hold:
  \begin{enumerate}[\hspace{5mm}i.]
  \item $\gc{\prog}{v_1}{\ttt{s1}} \quad \text{and} \quad \gc{\prog}{v_2}{\ttt{s2}}$
  \item $\sep{\progd(\prog)}{\{\ttt{s[1,"s1"]\}}}{\vdefs(\prog)_{\{2\}}}$
  \item $\sep{\progd(\prog)}{\{\ttt{s[2,"s2"]\}}}{\vdefs(\prog)_{\{1\}}}$    
  \item All $x \in \vars(\prog)$ are distinguished by $\ttt{in1}$ or $\ttt{in2}$. 
  \end{enumerate}
\end{definition}
Decode certification establishes separation of the input wire
label from the decoding tables under any input value condition. 
\begin{definition}[YGC Decode Certification]
  \label{definition-ygdecode-certification}
  Let $\mathit{decode}$ be a decoding function, and let:
  $$
  \config{\varnothing}{\mathit{decode}(\ttt{c}, \ttt{owl(c)})}
  \redxs \config{\prog}{\be}
  %\quad \text{and} \quad
  %\config{\varnothing}{\mathit{decode}(\ttt{c}, \ttt{invert(owl(c))})}
  %\redxs \config{\prog}{\be_0}
  $$
  Then $\mathit{decode}$ is certified iff:
  $$
  \sep{\progtt(\prog)}{\ttt{\{flip[2,gate:c.k],flip[2,gate:c.p]\}}}{\vdefs(\prog)}
  $$
  %Then $\mathit{decode}$ is certified iff both of the following conditions hold:
  %\begin{enumerate}[\hspace{5mm}i.]
  %\item $\progd(\prog_1,\be_1)(\{ \itv \mapsto 1 \}) = 1$ and
  %  $\progd(\prog_0,\be_0)(\{ \itv \mapsto 0 \}) = 1$
  %\item $\sep{\progtt(\prog_i)}{\ttt{\{flip[2,gate:c.k],flip[2,gate:c.p]\}}}{\vdefs(\prog_i)}$
  %\end{enumerate}
\end{definition}
We have certified each of the YGC components detailed above. 
\begin{lemma}
  \label{lemma-ygc-certification}
  Each of $\ttt{decode}$, $\ttt{encode}$, and $\ttt{andgg}$ are certified.
\end{lemma}

\subsection{YGC Gate Invariants and Circuit Security}
\label{section-composition-metatheory}

To demonstrate correctness of certification, we need to show that
isolated component certificates preserve relevant properties
when integrated into larger programs. 
%First, we show that if a flip $f$ is independent from the views
%in a protocol (as is the case with input labels used to
%encode a garbled table), then separation from views that
%use $f$ (e.g., in an output label) is preserved. 
%\begin{lemma}[Wire Framing]
%  \label{lemma-wire-framing}
%  If $\sep{\progtt(\prog)}{\{f\}}{\vdefs(\prog)}$ for flip $f$ and
%  $\vars(\prog) \cap \vars(\prog') = \{ f \}$ then
%  $\sep{\progtt(\prog';\prog)}{\vdefs(\prog')}{\vdefs(\prog)}$.
%\end{lemma}
%Next,

Now we can show how to splice gates into arbitrary circuits in an
invariant-preserving manner, assuming that outputs are not wired to
multiple inputs \cite{tate2003garbled,nieminen2023breaking}.
\begin{restatable}[YGC Gate Invariant]{lemma}{ygcgate}
  \label{lemma-ygc-preservation}
Given well-typed $\prog_1;E[\mathit{gate}(g,g_1,g_2,v_1,v_2)]$ and certified $\mathit{gate}$ where:
$$
\config{\prog_1}{E[\mathit{gate}(g,g_1,g_2,v_1,v_2)]} \redxs \config{\prog_1;\prog_2}{E[v]}
$$
If the following preconditions hold where $\iov(\prog_1) = S \cup V$:
\begin{enumerate}
\item $\{ g_1 \} \cap \{ g_2 \} \cap \wired(\prog_1) = \varnothing$ and $g \not\in \prog_1$
\item $\gc{\prog_1}{v_1}{g_1}$ and $\gc{\prog_1}{v_2}{g_2}$
\item $\sep{\progd(\prog_1)}{S_{\{1\}}}{V_{\{2\}}}$
\item $\sep{\progd(\prog_1)}{S_{\{2\}}}{V_{\{1\}}}$
\end{enumerate}
then we have as postconditions where $\iov(\prog_1) = S \cup V'$:
\begin{enumerate}
\item $\gc{\prog_1;\prog_2}{v}{g_2}$
\item $\sep{\progd(\prog_1,\prog_2)}{S_{\{1\}}}{V'_{\{2\}}}$
\item $\sep{\progd(\prog_1,\prog_2)}{S_{\{2\}}}{V'_{\{1\}}}$
\end{enumerate}
\end{restatable}

\begin{restatable}[YGC Encode Invariant]{lemma}{ygcencode}
  \label{lemma-ygc-encode}
Given well-typed $\prog_1;E[\mathit{encode}(s_1,s_2)]$ and certified $\mathit{encode}$ where 
$$
\config{\prog_1}{E[\ttt{encode}(s_1,s_2)]} \redxs \config{\prog_1;\prog_2}{E[\ttt{\{wv1=}v_1\ttt{;wv2=}v_2\ttt{\}}]}
$$
If the following preconditions hold where $\iov(\prog_1) = S \cup V$:
\begin{enumerate}
\item $\sep{\progd(\prog_1)}{S_{\{1\}}}{V_{\{2\}}}$
\item $\sep{\progd(\prog_1)}{S_{\{2\}}}{V_{\{1\}}}$
\end{enumerate}
then we have as postconditions where $\iov(\prog_1) = S' \cup V'$:
\begin{enumerate}
\item $\gc{\prog_1;\prog_2}{v_1}{s_1}$ and $\gc{\prog_1}{v_2}{s_2}$
\item $\sep{\progd(\prog_1;\prog_2)}{S'_{\{1\}}}{V'_{\{2\}}}$
\item $\sep{\progd(\prog_1;\prog_2)}{S'_{\{2\}}}{V'_{\{1\}}}$
\end{enumerate}
\end{restatable}

Given the above, we can now establish that any circuit built with
certified components is passive secure.  Note that we assume a normal
form of programs where the last instruction is a decoding and public
reveal of the output, and we require that no gate is wired more than once.
\cnote{There is another subtlety here, that gates are wired correctly
  in the sense that if the garbler generates a table with input wire label
  $g_1$ then only $g_1$ is wired to that input. }
\begin{restatable}{theorem}{ygcnimo}
  \label{theorem-ygc-NIMO}
  If $\eassign{\outv}{\ttt{decode}(g,e)}$ is a well-typed YGC circuit
  definition using certified components where no gate output is wired
  more than once, and
  $\config{\varnothing}{\eassign{\outv}{\ttt{decode}(g,e)}} \redxs
  \config{\prog}{\varnothing}$, then $\NIMO(\prog)$.
\end{restatable}


\begin{comment}

\begin{fpfig}[t]{YGC copy gate definitions.}{fig-ygc-copy}
  {\footnotesize
    \begin{verbatimtab}
      sharetab2(gid, tid, k, p, b)
      {
        let r1 = k xor b in
        let r0 = (not k) xor (not b) in
        v[1,gate: || gid || tid || 1] := select(p,r1,r0);
        v[1,gate: || gid || tid || 2] := select(not p,r1,r0);
      }
      
      copygate(ca,cb,g)
      {
        let wl = owl(g) in
        let owl1 = owl(ca) in
        let owl2 = owl(cb) in
        sharetab2(ca,tt,wl.k,wl.p,owl1.k); sharetab2(ca,pt,wl.k,wl.p,owl1.p);
        sharetab2(cb,tt,wl.k,wl.p,owl2.k); sharetab2(cb,pt,wl.k,wl.p,owl2.p)
      }
      
      evalcopy(ca,cb,wv)
      {
        let wv1k = wv.k xor select(wv.p,v[1,gate: || ca || tt1], v[1,gate: || ca || tt2]) in
        let wv1p = wv.k xor select(wv.p,v[1,gate: || ca || pt1], v[1,gate: || ca || pt1]) in
        let wv2k = wv.k xor select(wv.p,v[1,gate: || cb || tt1], v[1,gate: || cb || tt2]) in
        let wv2p = wv.k xor select(wv.p,v[1,gate: || cb || pt1], v[1,gate: || cb || pt1]) in
        { wv1 = { k = wv1k; p = wv1p }; wv2 = { k = wv2k; p = wv2p } }  
      }

      copy(ca,cb,g,wv) { copygate(ca,cb,g); evalcopy(ca,cb,wv) } 
    \end{verbatimtab}
  }
\end{fpfig}

\subsection{Building and Extending Automatically Secure Circuits}
\label{section-composition-copy}

Theorems \ref{theorem-gmw-NIMO} \ref{theorem-ygc-NIMO} means that any
well-typed program using certified components will generate a circuit
that is passive secure. To add an $\ttt{or}$ gate to our YGC library,
for example, we can use $\ttt{andgg}$ as a template, in fact simple
modification of $\ttt{andtable}$ is all that would be needed. Thus,
this library embodies a Fairplay-like language where well-typedness of
metaprograms guarantees safety and security of generated protocols.

The requirements that output wire labels are not used more than once
in YGC reflects a known result, that naive copy gate output is unsound
in YGC- i.e., passive security fails. In fact, we are unable to
automatically verify $\gNIMO$ for $\ttt{andgg}$ without the assumption
that input wire labels are in independent uniform distributions. We
can capture the necessary fix as in Figure \ref{fig-ygc-copy}. The
garbler creates a copy gate with two new output wire labels in
independent uniform distributions (in \ttt{copygate}), and permutes
them using the input wire label pointer after encrypting with the
input wire label keys. The evaluator then recovers the encrypted
output wire values during evaluation (in \ttt{evalcopy}). Given an
$\ttt{orgg}$ gate defined as outlined above, we can then use secure
$\ttt{copy}$ as follows:
\begin{verbatimtab}
  let s = encode(s1,s2) in
  let s1c = copy(c1s1,c2s1,s1,s.wv1) in
  let s2c = copy(c1s2,c2s2,s2,s.wv2) in
  let wv1 = andgg(1,c1s1,c1s2,s1c.wv1,s2c.wv1) in
  let wv2 = orgg(2,c2s1,c2s2,s1c.wv2,s2c.wv2) in
  decode(3,andgg(3,1,2,wv1,wv2))
\end{verbatimtab}
These definitions of $\ttt{copy}$ and $\ttt{orgg}$ have been
automatically verified as in Section \ref{section-pre-post} for
$\ttt{andgg}$. In general, any new gate form- such as optimized gates
\cite{XXX}- can be verified automatically in a similar way.

While the copy gate supports secure programming patterns, security
still relies on programmer discipline, i.e., not wiring a gate more
than once. However, we can keep track of wired gates in type effects,
similar to the manner in which we keep track of defined views- type
dependence in our system is able to accurately precisely track
gate identifiers. This is also reminiscent of the use of linearity
in other related systems for enforcing obliviousness \cite{darais2019language}. 

\end{comment}
