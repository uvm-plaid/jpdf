\section{Compositional Properties: Garbled $\NIMO$}
\label{section-composition}

Using YGC as a case study, in this Section we will discuss
compositional properties that can be automatically verified on small
program components using techniques developed in Section
\ref{section-bruteforce}. These properties compose to yield $\NIMO$
and scale to arbitrary programs. We will express these properties
manually as pre-~and post-conditions on components, so our
general approach is in a proof assistance style. The properties
we use are refinements of $\NIMO$, and the overall technique is
enabled by the intensional character of our program distribution
formulation.

\subsection{A Library for Yao's Garbled Circuits}
\label{section-metalang-ygc}

\begin{fpfig}[t]{Yao's Garbled Circuits, auxiliary functions.}{fig-ygc-aux}
{\footnotesize
\begin{verbatimtab}
  keygen(gid, b1, b2) { select4(b1,b2,H[gid || "1"],H[gid || "2"],H[gid || "3"],H[gid || "4"]) }
  
  keysgen(gid, b1, b2)
  {
    let k11 = keygen(gid,b1,b2) in
    let k10 = keygen(gid,b1,not b2) in
    let k01 = keygen(gid,not b1,b2) in
    let k00 = keygen(gid,not b1,not b2) in
    {k11 = k11; k10 = k10; k01 = k01; k00 = k00}
  }
  
  andtable(keys, bt, ap, bp)
  {
    let r11 = (keys.k11 xor bt) in 
    let r10 = (keys.k10 xor (not bt)) in
    let r01 = (keys.k01 xor (not bt)) in
    let r00 = (keys.k00 xor (not bt)) in
    permute4(ap,bp,r11,r10,r01,r00)
  }
  
  sharetable(gid, vid, table)
  {   
    v[1, "gate:" || gid || vid || "1"] := table.v1;
    v[1, "gate:" || gid || vid || "2"] := table.v2;
    v[1, "gate:" || gid || vid || "3"] := table.v3;
    v[1, "gate:" || gid || vid || "4"] := table.v4
  }
\end{verbatimtab}
}
\end{fpfig}

\begin{fpfig}[t]{Yao's Garbled Circuits, garbled gates and evaluation code.}{fig-ygc-gates}
{\footnotesize
\begin{verbatimtab}
  garbledecode(wl)
  {
    let r1 = wl.k xor true in
    let r0 = (not wl.k) xor false in
    v[1,"OUTtt1"] := select[wl.p,r1,r0];
    v[1,"OUTtt2"] := select[not wl.p,r1,r0]
  }
  
  evaldecode(wl, p) { wl.k xor select[wl.p,v[1,"OUTtt1"],v[1,"OUTtt2"]] }
  
  evalgate(gid, wla, wlb)
  {
    let k = keygen(gid,wla.k,wlb.k) in
    let ct = select4(wla.p,wlb.p,
               v[1,gid || "tt1"],v[1,gid || "tt2"],v[1,gid || "tt3"],v[1,gid || "tt4"]) in
    let cp = select4(wla.p,wlb.p,
               v[1,gid || "pt1"],v[1,gid || "pt2"],v[1,gid || "pt3"],v[1,gid || "pt4"]) in
    { k = k xor ct; p = k xor cp }
  }
  
  andgate(gid, wla, wlb, wlc) 
  {
    let keys = keysgen(gid,wla.k,wlb.k) in
    sharetable(gid,"tt",andtable(keys,wlc.k,wla.p,wlb.p));
    sharetable(gid,"pt",andtable(keys,wlc.p,wla.p,wlb.p))
  }
  
  encode(gid, wla,wlb)
  {
    let wla = { k = flip[2,"fwl1"]; p = flip[2,"pwl1"] } in
    let wlb = { k = flip[2,"fwl2"]; p = flip[2,"pwl2"] } in
    { wv1 = { k = OT[s[1,"0"],wla.k,(not wla.k)]; p = OT[s[1,"0"],wla.p,(not wla.p)]}; 
      wv2 = { k = select[s[2,"0"],wlb.k,(not wlb.k)]; p = select[s[2,"0"],wlb.p,(not wlb.p)] } }
  }
\end{verbatimtab}
}
\end{fpfig}

%  andtable(keys, bt, ap, bp)
%  {
%    let r11 = (keys.k11 xor bt) in 
%    let r10 = (keys.k10 xor (not bt)) in
%    let r01 = (keys.k01 xor (not bt)) in
%    let r00 = (keys.k00 xor (not bt)) in
%    permute4(ap,bp,r11,r10,r01,r00)
%  }


In Figures \ref{fig-ygc-aux} and \ref{fig-ygc-gates} we define a
codebase for garbled circuits. This definition follows the
\emph{point-and-permute} method described in \cite{XXX} and elsewhere,
to which the reader is referred for more in-depth discussion.
In this implementation client 2 is the \emph{garbler} and
client 1 is the \emph{evaluator}. The garbler builds the garbled
tables and shares them with the evaluator, who then evaluates
the gate in an oblivious fashion until the final public output is
generated through decryption. This definition is well-typed,
with input type annotations for top-level functions listed in
Figure \ref{fig-ygc-types}. Well-typed programs using these
libraries are therefore guaranteed to yield safe $\minifed$
programs. 

\emph{Wire labels} are fundamental to YGC, and essentially represent
gate output values in an encrypted form. In our definition, wire
labels are representated by records $\ttt{\{ k = }\beta_1\ttt{; p =
}\beta_2\ttt{ \}}$, where $\ttt{k}$ is the \emph{key bit} and
$\ttt{p}$ is the \emph{pointer bit}, and $\beta_1$ and $\beta_2$ are
flips. Flips in each output wire label are owned by the garbler and
are unique per gate by definition of their identifying string, and the
representation of $0$ is the negation of $1$. For example, here is the
representation of 1 and 0 respectively in the output wire label for a
hypothetical gate 6:
\begin{mathpar}
  \ttt{\{ k = flip[2,gate:6.k]; p =  flip[2,gate:6.p]] \}}
    
  \ttt{\{ k = not flip[2,gate:6.k]; p =  not flip[2,gate:6.p]] \}}
\end{mathpar}
The pointer bits in wire labels are used to select permuted rows in
table garblings. The key bits are used to identify a unique key for
table row in each garbled gate. Intuitively, if $\beta_1$ and
$\beta_2$ are either key or pointer bits encoding 1 on two input wire
labels to a binary gate, rows and keys in the gate are enumerated in
the order:
$$
\neg\beta_1\neg\beta_2,\ \neg\beta_1\beta_2,\ \beta_1\neg\beta_2,\ \beta_1\beta_2
$$

In our implementation, gates are wired together using gate
identifiers, which are strings $w$. Top-level functionality in Figures
\ref{fig-ygc-aux} and \ref{fig-ygc-gates} includes the following:
\begin{itemize}
\item \ttt{andgate}: This defines a subprotocol for the garbler
  to define a garbled gate $\ttt{gid}$ with input wires from gates
  $\ttt{ga}$ and $\ttt{gb}$. The garbler generates keys and garbles
  the rows in YGC fashion, them with client $1$ in
  views in a standard form. For example, the view for
  a hypothetical gate 6, row 2 garbled truth table is $\ttt{v[1,gate:6tt2]}$.
  We note that garbled gates of other binary operators can be obtained with
  replacement of $\ttt{andtable}$ with appropriate garbled  table definitions. 
\item \ttt{evalgate}: This defines a subprotocol for the evaluator to
  evaluate gate $\ttt{gid}$ given input wire values $\ttt{wva}$ and
  $\ttt{wvb}$.
\item \ttt{garbledecode} and \ttt{evaldecode}: The former function
  defines the garbler's protocol for encrypting the circuit
  output from final gate $\ttt{gid}$, and the latter defines
  the evaluator's output decryption protocol.
\item \ttt{encode}: This defines the initial phase of the protocol,
  where the evaluator receives the wire value from their own
  secret $\sx{1}{sa}$ via $\ttt{OT}$, and the garbler communicates
  the wire value for their own secret $\sx{2}{sb}$ directly.
\end{itemize}
\begin{example}
  \label{example-andcircuit}
The following program uses our YGC library to define
a ciruit with a single and gate and input secrets $\ttt{s1}$ and
$\ttt{s2}$ from client's 1 and 2 respectively. 
\begin{verbatimtab}
  andgate(0,s1,s2);
  garbledecode(0);
  let secrets = encode(s1,s2) in
  v[0,output] := decode(evalgate(0, secrets.wv1, secrets.wv2))
\end{verbatimtab}
\end{example}
We have verified passive security of the $\fedprot$ protocol
generated by this and other small circuits using the
technique described in Lemma \ref{lemma-bruteforce-nimo}.
But obviously large circuits with thousands of gates would be
intractable to verify with this method. In the next Section
we discuss compositional methods to address this issue.

\subsection{YGC Gate Pre-~and Post-Conditions}

A basic intuition about how garbled gates work is that they are
``nearly'' passive secure, modulo the feature that outputs of
individual logic gates in circuits are encryped. That is, each gate
enjoys $\NIMO$, except the output is not in the clear, but correlated
with the flips that the garbler sampled for the gate's output wire
label, both key and pointer bits. We call this \emph{Garbled $\NIMO$},
or $\gNIMO$. To clarify this definition, recall that Definition \ref{def-progd}
posits a variable $\itv$ to refer to the expression $\be$ when
considering the distribution $\progd(\prog,\be)$.
\begin{definition}[Garbled $\NIMO$]
  We write $\gNIMO(\prog,\{ \ttt{k} = \be_{\ttt{k}}; \ttt{p} =
  \be_{\ttt{p}}\}, g)$ with $\iov(\prog) = S \cup V$ iff for all $H$,
  $C$ with $|C| \le |H|$, for all $\store \in \mems(S_C \cup V_C)$,
  for both $w \in \{\ttt{p},\ttt{k}\}$, and for all $\stores$ with
  domain $X = S_C \cup V_C \cup \{ \itv, \flip{2}{\ttt{gate:}g.w} \}$
  such that:
  $$
  \stores = \{ \store' \mid \store \subset \store' \wedge 
  \store'(\itv) = \store'(\flip{2}{\ttt{gate:}g.w}) \} 
  $$
  or
  $$
  \stores = \{ \store' \mid \store \subset \store' \wedge 
  \store'(\itv) = \neg\store'(\flip{2}{\ttt{gate:}g.w}) \} 
  $$
  %where for any $\beta$:
  %$$\store^2 = \store\{ \flip{2}{\ttt{gate:}g.w} \mapsto \beta \}\{ \itv \mapsto \beta \}$$
  %or
  %$$
  %\store^2 = \store\{ \flip{2}{\ttt{gate:}g.w} \mapsto \beta \}\{ \itv \mapsto \neg\beta \}
  %$$
  we have:
  $$
  \margd{(\condd{\progd(\prog,\be_w)}{\stores_{X-{V_C}}})}{S_H} =
  \margd{(\condd{\progd(\prog,\be_w)}{\stores})}{S_H}
  $$
\end{definition}

\begin{lemma}
  \label{lemma-scope}
  If $(\margd{\progd(\prog_1;\prog_2)}{\iov(\prog_1)}) * (\margd{\progd(\prog_1;\prog_2)}{\iov(\prog_2)})$
  and $\flips(\prog_1) \cap \flips(\prog_2) = \varnothing$ then
  $(\margd{\progd(\prog_1';\prog_2)}{\iov(\prog_1)}) * (\margd{\progd(\prog_1';\prog_2)}{\iov(\prog_2)})$
  for any $\prog_1'$ where $\flips(\prog_1') \cap \flips(\prog_2) = \varnothing$.
\end{lemma}

Our aim now is to express that $\gNIMO$ is an invariant preserved throughout
the circuit by demonstrating pre-~and post-conditions for component execution.
Our formulation will assume a protocol definition where the garbler and
evaluator coordinate garbling, sharing, and evaluation of gates iteratively,
although the generated $\fedprot$ program could be rewritten so the garbler
shares the entire circuit before execution by the evaluator. As a sanity
condition we will assume all $\metaprot$ programs we consider in this
section are well-typed. 

First, we show that encoding of input secrets for use by the evaluator
establishes the $\gNIMO$ invariant.

\begin{verbatimtab}
andgg(g, wla, wlb, wlc, wva, wvb) {
andgate(g, wla, wlb, wlc);
evalgate(g, wva, wvb)
}
\end{verbatimtab}


\begin{definition}
  We write $\owl(\prog, \{ \ttt{k} = \be_{\ttt{k}}; \ttt{p} = \be_{\ttt{p}}\}, g)$
w  iff for all $b$ and both $w \in \{\ttt{p},\ttt{k}\}$:
  $$
  (\margd{(\condd{\progd(\prog,\be_{w})}{\{ \flip{2}{g.w} \mapsto b \}})}{\itv})(\{ \itv \mapsto b\}) = 1
  $$
\end{definition}

\begin{lemma}[Input Encoding]
Assume given the following evaluation relation:
$$
\config{\prog_1}{E[\ttt{encode}(w,v_1,v_2)]} \redxs \config{\prog_1;\prog_2}{E[v]}
$$
If there exists gate identifiers $g_1, g_2$ such that:
\begin{enumerate}
  \item $g_1 \ne g_2 \wedge g_1,g_2 \not\in \gates(\prog_1)$
  \item $\owl(\prog_1,v_1,g_1) \wedge \owl(\prog_1,v_2,g_2)$
\end{enumerate}
then:
\begin{enumerate}
\item $g_1,g_2 \in  \gates(\prog_2)$
\item $\gNIMO(\prog_1;\prog_2,v.\ttt{wv1},g_1) \wedge \gNIMO(\prog_1;\prog_2,v.\ttt{wv2},g_2)$
\item $(\margd{\progd(\prog_1;\prog_2)}{\iov(\prog_1)}) * (\margd{\progd(\prog_1;\prog_2)}{\iov(\prog_2)})$
\end{enumerate}
\end{lemma}

\begin{lemma}[Composition ($\eand$)]
Assume given the following evaluation relation:
$$
\config{\prog_1}{E[\ttt{andgg}(g,v_1,\ldots,v_5)]} \redxs \config{\prog_1;\prog_2}{E[v_6]}
$$
If there exist gate identifiers $g_1, g_2$ such that:
\begin{enumerate}
  \item $g_1 \ne g_2 \ne g \wedge g_1,g_2 \in \gates(\prog_1) \wedge g \not\in \gates(\prog_1)$
  \item $\owl(\prog_1,v_1,g_1) \wedge \owl(\prog_1,v_2,g_2)$
  \item $\gNIMO(\prog_1,v_4,g_1) \wedge \gNIMO(\prog_1,v_5,g_2)$
\end{enumerate}
then:
\begin{enumerate}
  \item $g \in \gates(\prog_2) \wedge \owl(\prog_1;\prog_2,v_3,g)$
  \item $\gNIMO(\prog_1;\prog_2,v_6,g)$
  \item $(\margd{\progd(\prog_1;\prog_2)}{\iov(\prog_1)}) * (\margd{\progd(\prog_1;\prog_2)}{\iov(\prog_2)})$
\end{enumerate}
\end{lemma}

\begin{lemma}[Decoding]
Assume given the following evaluation relation:
$$
\config{\prog_1}{E[\ttt{decoder}(v_1,v_2)]} \redxs \config{\prog_1;\prog_2}{E[\be]}
$$
If there exists gate identifier $g \in \gates(\prog_1)$ such that:
\begin{enumerate}
  \item $\owl(\prog_1,v_1,g)$
  \item $\gNIMO(\prog_1,v_2,g)$
\end{enumerate}
Then $\NIMO(\prog_1;\prog_2;\eassign{x}{\be})$.
\end{lemma}
