\section{Component Certification and Composition}
\label{section-composition}

\begin{fpfig}[t]{2-Party GMW circuit library with And gate (T), and type annotations (B).}{fig-gmw}
{\footnotesize
  \begin{verbatimtab}
    encodegmw(in1,in2) {
      v[1, in1++"out"] := s[1,in1] xor flip[1,in1];
      v[2, in1++"out"] := flip[1,in1];
      v[1, in2++"out"] := s[2,in2] xor flip[2,in2];
      v[2, in2++"out"] := flip[2,in2]
      { shares1 = { c1 = v[1, in1++"out"]; c2 = v[2, in1++"out"] };
        shares2 = { c1 = v[1, in2++"out"]; c2 = v[2, in2++"out"]} } 
    }
    
    andtablegmw(b1, b2) {
      let r11 = (b1 xor true) and (b2 xor true) in
      let r10 = (b1 xor true) and (b2 xor false) in
      let r01 = (b1 xor false) and (b2 xor true) in
      let r00 = (bl xor false) and (b2 xor false) in
      { v1 = r11; v2 = r10; v3 = r01; v4 = r00 }
    }
    
    andgmw(g, shares1, shares2) {
      let r = flip[1,g++".r"] in
      let table = andtablegmw(shares1.c1, shares2.c1) in
      let r11 =  r xor table.v1 in
      let r10 =  r xor table.v2 in
      let r01 =  r xor table.v3 in
      let r00 =  r xor table.v4 in
        v[2,g++"out"] := OT4(shares1.c2, shares2.c2, r11, r10, r01, r00);
      v[1,g++"out"] := r;
      { c1 = v[1,g++"out"]; c2 = v[2,g++"out"]}
    }
    
    decodegmw(shares) { v[0,"output"] := shares.c1 xor shares.c2 }   \end{verbatimtab}
}
\end{fpfig}


The Yao's Garbled Circuits (YGC) and GMW protocols are examples of a
common idiom in MPC protocol design- arbitrarily large \emph{circuits}
are built up from smaller boolean or arithmetic \emph{gates} that
operate on encodings of input secrets. These circuits can become quite
large in practice \cite{kreuter2012billion}, but are built from
well-defined components. Brute force verification techniques as
described in Section \ref{section-bruteforce} are not scalable to
large circuits, but in this Section, we will demonstrate that correctly
designed circuit components enjoy compositional properties
that can be automatically certification on small program components
using brute force methods. We prove that these properties preserve
invariants throughout circuit definition that yield $\NIMO$ and thus
scale to arbitrary programs. Furthermore, the certifications can be
used on arbitrary gate definitions, allowing new gate implementations
to be added soundly to libraries.

In this Section, we develop and use implementations of both 2-party GMW
and YGC to illustrate our approach and demonstrate its generality.
Main results are in Theorems \ref{theorem-gmw-NIMO} and
\ref{theorem-ygc-NIMO}.

\subsection{Compositional Properties of GMW}

The GMW protocol uses secret sharing to represent data flowing through
circuits. In the 2-party case, clients 1 and 2 each share their input
secrets, and use those shares to represent inputs to gates. Outputs
are also represented as shares. We refer to the pair of shares
representing any particular value as a \emph{wire value}, and
we represent them via records of the form
$
\ttt{\{c1 = } v_1\ttt{;c2 = } v_2\ttt{\}} 
$
where $v_1$ and $v_2$ are client 1 and 2's shares respectively.

For full details of the GMW protocol, the reader is referred to
\cite{evans2018pragmatic}. Our implementation library is shown in
Figure \ref{fig-gmw}, with type signatures for the library functions
shown in Figure \ref{fig-gmw-types}. We show the And gate since it is
an interesting component. The Figure includes the
following top-level functions:
\begin{itemize}
\item \ttt{encodegmw}: This function encodes client 1's and client 2's
  secret bits called $\ttt{s[1,s1]}$ and $\ttt{s[1,s2]}$ into two
  distinct wire values (pairs of shares).
\item \ttt{andgmw}: This function defines the gate $\ttt{g}$, the
  identifier $\ttt{g}$ being used to distinguish randomness used
  within.  In our version, client 1 builds the output table (using
  \ttt{andtablegmw})and transfers the correct output share to client 2
  using 1-out-of-4 OT as per standard GMW protocol.
\item \ttt{decode}: This function decodes and publishes a wire value
  by $\ttt{xor}$ing the shares. Note that this requires both client 1
  and 2 to publicize their shares.
\end{itemize}
\begin{example}
  \label{example-gmw-andcircuit}
The following program uses our GMW library to define
a circuit with a single And gate and input secrets $\ttt{s1}$ and
$\ttt{s2}$ from clients 1 and 2 respectively. 
\begin{verbatimtab}
  let ss = encodegmw(s1,s2) in
  v[0,output] := decode(andgmw(0,ss.shares1,ss.shares2))
\end{verbatimtab}
\end{example}

\subsubsection{Gate Certification}

Our certification techniques are defined generally wrt implementations
of input encoding, decoding, and internal gates. In the following we
will refer to arbitrary decode, encode, and gate functions with the
restriction that any function in each of these categories has the same
valid type signature as $\ttt{decodegmw}$, $\ttt{encodegmw}$, and
$\ttt{andgmw}$ respectively. When gates are used in circuits they are
parameterized by wire values-- in certification we will use carefully
chosen witness parameters. 

When we consider internal components and hence intensional properties
of circuits, we will consider behaviors with respect to particular bits
of randomness. Hence we will consider basic program distributions. 
\begin{definition}
  We write $\progtt(\prog)$ to denote the \emph{basic distribution} of
  $\prog$ as defined in Definition \ref{def-progd}. For any program $\prog =
  (\eassign{v_1}{\be_1};\ldots;\eassign{v_n}{\be_n})$, define
  $\vdefs(\prog) \defeq \{ v_1,\ldots,v_n \}$.
\end{definition}
We also need to formalize notions of probabilistic independence, aka
\emph{separation}, and correlation. We borrow the symbols $*$ and $\sim$
from \cite{barthe2019probabilistic} with the same denotations.
\begin{definition}
  We write $\vc{\pmf}{x}{y}$ iff $\pmf(\{ x \mapsto 0\}\ |\ \{ y \mapsto 0 \}) =
  \pmf(\{ x \mapsto 1\}\ |\ \{ y \mapsto 1 \}) = 1$.
  We write $\sep{\pmf}{Y}{Z}$ iff for all
    $\store \in \mems(X \cup Y)$:
  $$\margd{\pmf}{Y \cup Z}(\store) =
  \pmf(\store_Y) * \pmf(\store_Z)$$ 
\end{definition}

\begin{fpfig}[t]{GMW library type annotations.}{fig-gmw-types}
{\footnotesize
  \begin{verbatimtab}
   encodegmw   : string(gid) * string(gid)
    
   andtablegmw : { k = bool[i]; p = bool[i] }
    
   andgmw      : string(gid) *  { c1 = bool[1]; c2 = bool[2] } * { c1 = bool[1]; c2 = bool[2] }
    
   decodegmw   : { c1 = bool[1]; c2 = bool[2] }  \end{verbatimtab}
}
\end{fpfig}

Now we can define GMW gate certification as follows. Intuitively, we certify
that GMW gates satisfy a probabilistic noninterference property wrt each output
share considered individually. This ensures that input dependencies remain
encrypted from each client's perspective during circuit evaluation.
\begin{definition}[GMW Gate Certification]
  \label{definition-gmwgate-certification}
  Let $\mathit{gate}$ be a gate function and let $e$ be the following
  $\metaprot$ program:
  $$
  \begin{array}{l}
    \ttt{let in1 = \{c1 = flip[1,s11];c2 = flip[2,s12]\} in}\\
    \ttt{let in2 = \{c1 = flip[1,s21];c2 = flip[2,s22]\} in}\\
    \mathit{gate}\ttt{(g,in1,in2)}
  \end{array}
  $$
  and let $
  \config{\varnothing}{e}\redxs\config{\prog}{\ttt{\{c1 = v[1,gout]; c2 = v[2,gout]\}}}
  $.
  Then $\mathit{gate}$ is \emph{certified} iff the following conditions hold:
  \begin{enumerate}[\hspace{5mm}i.]
  \item $\sep{\progtt(\prog)}{\ttt{\{flip[1,s11],flip[1,s21],flip[2,s12],flip[2,s22]\}}}{\ttt{\v[1,gout]\}}}$
  \item $\sep{\progtt(\prog)}{\ttt{\{flip[1,s11],flip[1,s21],flip[2,s12],flip[2,s22]\}}}{\ttt{\v[2,gout]\}}}$
  % \item $\sep{\progtt(\prog)}{\ttt{\{v[1,gout]\}}}{\ttt{\{v[2,gout]\}}}$
  \item All variables in:$$\vars(\prog) - \{ \ttt{flip[1,s11],flip[1,s21],flip[2,s12],flip[2,s22]}\}$$
    are distinguished by $\ttt{g}$ (i.e., contain $\ttt{g}$ as an identifier substring). 
  \end{enumerate}
\end{definition}

Encoding certification establishes the desired property in the protocol-- that
input secrets of each client are independent of the other client's views. 
\begin{definition}[GMW Encode Certification]
  \label{definition-gmwencode-certification}
  Let $\mathit{encode}$ be an encoding function, and let:
  $$
  \config{\varnothing}{\ttt{encode(s1,s2)}} \redxs
  \config{\prog}{\ttt{\{shares1 = }v_1\ttt{; shares2 = }v_2\ttt{\}}}
  $$
  where:
  $$
  \begin{array}{rcl}
    v_1 &=& \ttt{\{ c1 = v[1,s1out]; c2 = v[2,s1out] \}}\\
    v_2 &=& \ttt{\{ c1 = v[1,s2out]; c2 = v[2,s2out] \}}
  \end{array}
  $$
  Then $\mathit{encode}$ is \emph{certified} iff each of the following conditions hold:
  \begin{enumerate}[\hspace{5mm}i.]
    %\item $\sep{\progtt(\prog)}{\ttt{\{ s[1,s1],  v[1,s1out], v[1,s2out] \}}}{\ttt{\{ s[2,s2], v[2,s1out], v[2,s2out] \}}}$
  \item $\sep{\progtt(\prog)}{\ttt{\{s[1,s1]\}}}{\ttt{\{v[2,s1out],v[2,s2out]\}}}$
  \item $\sep{\progtt(\prog)}{\ttt{\{s[2,s2]\}}}{\ttt{\{v[1,s1out],v[1,s2out]\}}}$
  \item All $x \in \vars(\prog)$ are distinguished by $\ttt{in1}$ or $\ttt{in2}$. 
  \end{enumerate}
  \end{definition}

We have certified both $\ttt{encodegmw}$ and $\ttt{andgmw}$. Note that
an Or or Xor gate, for example, could be certified with this same
method to extend our library.
\begin{lemma}
  \label{lemma-gmw-certification}
  Each of $\ttt{encodegmw}$ and $\ttt{andgmw}$ are certified.
\end{lemma}

\subsubsection{Gate Invariants and Circuit Security}
We can show that composition maintains an important invariant in a
circuit $\prog$- namely that client 1's secrets remain independent of
client 2's views, and vice-versa, prior to decoding.  That is, where
$\iov(\prog) = S \cup V$, we show that
$\sep{\progd(\prog)}{S_{\{1\}}}{V_{\{2\}}}$ and
$\sep{\progd(\prog)}{S_{\{2\}}}{V_{\{1\}}}$ are preserved prior to
decoding.

We observe the following properties of separation which are borrowed from prior work
\cite{barthe2019probabilistic} and which will be used frequently in proofs. 
\begin{lemma}
  \label{lemma-separation}
  The following properties hold:
  \begin{enumerate}
  \item $\sep{\pmf}{Y}{Z}$ iff $\sep{\pmf}{Z}{Y}$
  \item $\vc{\pmf}{x}{y}$ if  $\vc{\pmf}{y}{x}$
  \item $\vc{\pmf}{x}{y}$ and $\vc{\pmf}{y}{z}$ imply $\vc{\pmf}{x}{z}$
  \item $\sep{\pmf}{X}{(Y \cup Z)}$ implies $(\sep{\pmf}{X}{Y}$ and $\sep{\pmf}{X}{Z})$
  \item $(\sep{\pmf}{X}{Y}$ and $\sep{\pmf}{(X \cup Y)}{Z})$ implies $\sep{\pmf}{X}{(Y \cup Z)}$ 
  %\item If $\prog_1;\prog_2$ is safe and $\vars(\prog_1) \cap \vars(\prog_2) = \varnothing$
  %  then $\sep{\progtt(\prog_1;\prog_2)}{\vars(\prog_1)}{\vars(\prog_2)}$.
  \end{enumerate}
\end{lemma}
The following property also follows by results in \cite{barthe2019probabilistic} and will
be useful to make constructions that demonstrate variable separation.
\begin{lemma}
  \label{lemma-sepjoin}
  $\sep{\progtt(\prog)}{X}{Y}$ iff for all 
  $\store^1, \store^2 \in \runs(\prog)$ there exists
  $\store \in \runs(\prog)$ with
  $\store^1_{X} \cap \store^2_{Y} \subseteq \store$.
\end{lemma}

%The following Lemma allows us to separate variables in programs that may not occur
%in the same subprogram, and hence consider gate certification properties in various
%contexts. It is reminiscent of the \emph{frame} rule of \cite{barthe2019probabilistic}.
%\begin{lemma}
%  \label{lemma-gmw-frame}
%  If $\sep{\progtt(\prog_1;\prog_2)}{Y}{V}$ for $Y = \vars(\prog_1) \cap \vars(\prog_2)$ and
%  $V \in \vdefs(\prog_2)$ and $\sep{\progtt(\prog_1)}{X}{Y}$ for
%  $X \not\in \vars(\prog_2)$ then $\sep{\progtt(\prog_1;\prog_2)}{X}{V}$. 
%\end{lemma}
%\begin{proof}
%  By assumption and Lemma \ref{lemma-sepjoin}, for all
%  $\store^1,\store^2 \in \runs(\prog_1;\prog_2)$ we have $\store^1_Y
%  \cap \store^2_V \subseteq \store_a$ for some $\store_a \in
%  \runs(\prog_1;\prog_2)$ and $\store^1_X \cap \store^2_Y \subseteq
%  \store_b$ for some $\store_b \in \runs(\prog_1;\prog_2)$.  Thus, since
%  we assume $X \not\in \vars(\prog_2)$ we have $\store_a \cap \store_b
%  \in \runs(\prog_1;\prog_2)$ and $\store^1_X \cap \store^2_V =
%  \store_a \cap \store_b$, obtaining the result by
%  \ref{lemma-sepjoin}.
%\end{proof}

The next ``noninterference'' Lemma is important to prove the desired invariance
preservation property of GMW gates, given certification.
\begin{lemma}[Noninterference]
  \label{lemma-noninterference}
  Given $\prog_1;\prog_2$ and $X = \iov(\prog_2) - \vdefs(\prog_2)$ and
  $Y \subseteq \vdefs(\prog_2)$. If $\sep{\progd(\prog_1;\prog_2)}{X}{Y}$
  then $\sep{\progd(\prog_1;\prog_2)}{\iov(\prog_1)}{Y}$.
\end{lemma}

The next two Lemmas allow us to use the properties of gates established by
certification and replace the ``dummy'' parameters used there with actual
wire values in full circuit contexts. Intuitively we show that substitution
for dummy flips preserves separation. 
In these results and subsequently, we will use $f$ to range over flips.
\begin{lemma}
  \label{lemma-presub}
  Given $\be_1$, $\be_2$, and $i$ where $\vars(\be_1) \cap
  \vars(\be_2) = \varnothing$ and $\lcod{\store_1,\be_1}{i} = \beta$
  with $\store_1(x) = \lcod{\store_2,\be_2}{i}$ and
  $\dom(\store_1) = \vars(\be_1)$ and $\dom(\store_2) = \vars(\be_2)$.
  Then $\lcod{\store_1\cap\store_2,\be_1[\be_2/x]}{i} = \beta$.
\end{lemma}
\begin{proof}
By straightforward structural induction on $\be$.
\end{proof}

\begin{lemma}[Substitution$*$]
  \label{lemma-substitution}
  If $\sep{\progtt(\prog_2)}{\{ f \}}{X}$ and
  $\prog_1;\prog_2[\be/f]$ is safe with $\vars(\prog_1,\be) \cap
  X = \varnothing$ then
  $\sep{\progtt(\prog_1;\prog_2[\be/f])}{\vars(\be)}{X}$.
\end{lemma}
\begin{proof}
  Suppose on the contrary it was not the case that
  $\sep{\progtt(\prog_1;\prog_2[\be/f])}{\vars(\be)}{X}$.
  Then by Lemma \ref{lemma-sepjoin} there exists $\store_1,\store_2
  \in \runs(\prog_1;\prog_2[\be/f])$ with no $\store \in
  \runs(\prog_1;\prog_2[\be/f])$ such that $\store^1_{\vars(\be)}
  \cap \store^2_X \subseteq \store$.  But also by
  assumption and Lemma \ref{lemma-sepjoin} for any $\beta$ there
  exists $\store' \in \runs(\prog_2)$ with $\{ f \mapsto \beta \} \cap
  \store^2_X \subseteq \store'$. So in particular, we
  have $\{ f \mapsto \lcod{\store^1_{\vars(\be)},\be}{i} \} \cap
  \store^2_X \subseteq \store'$ for any $i$. This, the
  assumption $\vars(\prog_1,\be) \cap \vars(\prog_2) = \varnothing$,
  and application of Lemma \ref{lemma-presub} leads to the consequence
  that there exists $\store \in \runs(\prog_1;\prog_2[\be/f])$ with
  $\store \supseteq \store^1_{\vars(\be)} \cap
  \store^2_X$ given the assumption $\vars(\prog_1,\be)
  \cap \vars(\prog_2) = \varnothing$, which is a contradiction.
\end{proof}

Now we can prove that certified gates preserve the desired invariant
during execution in arbitrary circuit contexts.
We formulate this in terms of pre-~and post-conditions of gate
evaluation.
\begin{lemma}[GMW Gate Invariant]
  \label{lemma-gmw-preservation}
  Given well-typed $\prog_1;E[\mathit{gate}(\ttt{g},v_1,v_2)]$ for certified $\mathit{gate}$ where:
  $$
  \config{\prog_1}{E[\mathit{gate}(\ttt{g},v_1,v_2)]} \redxs \config{\prog_1;\prog_2}{E[v]}
  $$
  If we have the following preconditions, where $\iov(\prog_1) = S \cup V$:
  \begin{enumerate}[\hspace{5mm}i.]
  \item $\sep{\progd(\prog_1)}{S_{\{2\}}}{V_{\{1\}}}$
  \item $\sep{\progd(\prog_1)}{S_{\{1\}}}{V_{\{2\}}}$
  \end{enumerate}
  then we have as a postconditions, where $\iov(\prog_1;\prog_2) = S \cup V'$:
  \begin{enumerate}[\hspace{5mm}i.]
  \item $\sep{\progd(\prog_1;\prog_2)}{S_{\{2\}}}{V'_{\{1\}}}$
  \item $\sep{\progd(\prog_1;\prog_2)}{S_{\{1\}}}{V'_{\{2\}}}$
  \end{enumerate}
\end{lemma}

\begin{proof}
  Given well-typedness of $\prog_1;E[\mathit{gate}(\ttt{g},v_1,v_2)]$, we have for some $\ttt{g1}$ and $\ttt{g2}$:
  $$
  \begin{array}{lcl}
   v_1 &=& \ttt{\{c1 = v[1,g1out]; c2 = v[2,g1out]\}}\\
   v_2 &=& \ttt{\{c1 = v[1,g2out]; c2 = v[2,g2out]\}}\\
    v &=& \ttt{\{c1 = v[1,gout]; c2 = v[2,gout]\}}
  \end{array}
  $$
  Let $\prog$ be as defined in Definition \ref{definition-gmwgate-certification}.
  We observe:
  $$
  {\footnotesize
    \begin{array}{c}
      \prog_2 = \\
      \prog[\ttt{v[1,g1out]}/\ttt{flip[1,s11]}][\ttt{v[2,g1out]}/\ttt{flip[2,s12]}][\ttt{v[1,g2out]}/\ttt{flip[1,s21]}][\ttt{v[2,g2out]}/\ttt{flip[2,s22]}]
    \end{array}
  }
  $$
  Now, the assumption of well-typedness also assures that $\ttt{g}$
  has not been previously used as a gate identifier in $\prog_1$,
  since otherwise the views $\ttt{v[1,gout]}$ and $\ttt{v[2,gout]}$
  would have been previously defined. This ensures $\vars(\prog_1,x)
  \cap \vars(\prog) = \varnothing$ for:
  $$x \in \{ \ttt{v[1,g1out]}, \ttt{v[2,g1out]}, \ttt{v[1,g2out]}, \ttt{v[2,g2out]}\}$$
  given condition (iii) of Definition \ref{definition-gmwgate-certification}.
  Thus by Lemma \ref{lemma-substitution} we have:
  $$\sep{\progtt(\prog_1;\prog_2)}{\ttt{\{v[1,g1out],v[1,g2out],v[2,g1out],v[2,g2out]\}}}{\ttt{\{v[1,gout]\}}}$$
  and
  $$\sep{\progtt(\prog_1;\prog_2)}{\ttt{\{v[1,g1out],v[1,g2out],v[2,g1out],v[2,g2out]\}}}{\ttt{\{v[2,gout]\}}}$$
  Now, by Lemmas \ref{lemma-noninterference} and \ref{lemma-separation} and condition (iii) of Definition
  \ref{definition-gmwgate-certification} we then have:
  $$\sep{\progtt(\prog_1;\prog_2)}{(S_{\{1\}} \cup V_{\{2\}})}{\ttt{\{v[2,gout]\}}} \quad \text{and} \quad
  \sep{\progtt(\prog_1;\prog_2)}{(S_{\{2\}} \cup V_{\{1\}})}{\ttt{\{v[1,gout]\}}}$$
  so by precondition assumptions and Lemma \ref{lemma-separation} we have:
  $$\sep{\progtt(\prog_1;\prog_2)}{S_{\{1\}}}{V'_{\{2\}}} \quad \text{and} \quad
  \sep{\progtt(\prog_1;\prog_2)}{S_{\{2\}}}{V'_{\{1\}}}$$
\end{proof}

Here we show that encoding also preserves the desired invariant.
The result here is more straightforward than it is for gates because
the variables used during encoding are guaranteed to be distinct from
the rest of the program.
\begin{lemma}[GMW Encode Invariant]
  \label{lemma-gmw-encode}
  Given well-typed $\prog_1;E[\mathit{encode}\ttt{(s1,s2)}]$ for certified $\mathit{encode}$, let:
  $$
  \config{\prog_1}{E[\mathit{encode}(\ttt{s1},\ttt{s2})]} \redxs
  \config{\prog_1;\prog_2}{E[v]}
  $$
  If we have the following preconditions, where $\iov(\prog_1) = S' \cup V$:
  \begin{enumerate}[\hspace{5mm}i.]
  \item $\sep{\progd(\prog_1)}{S_{\{2\}}}{V_{\{1\}}}$
  \item $\sep{\progd(\prog_1)}{S_{\{1\}}}{V_{\{2\}}}$
  \end{enumerate}
  then we have as a postconditions, where $\iov(\prog_1;\prog_2) = S' \cup V'$: 
  \begin{enumerate}[\hspace{5mm}i.]
  \item $\sep{\progd(\prog_1;\prog_2)}{S'_{\{2\}}}{V'_{\{1\}}}$
  \item $\sep{\progd(\prog_1;\prog_2)}{S'_{\{1\}}}{V'_{\{2\}}}$
  \end{enumerate}
\end{lemma}

\begin{proof}
  By assumptions of well-typedness we have:
  $$
  \begin{array}{rcl}
    v &=&  \ttt{\{shares1 = }v_1\ttt{; shares2 = }v_2\ttt{\}}\\
    v_1 &=& \ttt{\{ c1 = v[1,s1out]; c2 = v[2,s1out] \}}\\
    v_2 &=& \ttt{\{ c1 = v[1,s2out]; c2 = v[2,s2out] \}}
  \end{array}
  $$
  And by Definition \ref{definition-gmwencode-certification} we have:
  \begin{mathpar}
    \sep{\progtt(\prog)}{\ttt{\{s[1,s1]\}}}{\ttt{\{v[2,s1out],v[2,s2out]\}}}
    
    \sep{\progtt(\prog)}{\ttt{\{s[2,s2]\}}}{\ttt{\{v[1,s1out],v[1,s2out]\}}}
  \end{mathpar}
  Since we assume that secrets are in uniform and independent marginal
  distributions a priori, and $\vars(\prog_1) \cap \vars(\prog_2) =
  \varnothing$ by condition (iii) of Definition 
  \ref{definition-gmwencode-certification} and assumptions of well-typedness, 
  the result follows by preconditions and Lemma \ref{lemma-separation}.
\end{proof}

On the basis of the preceding, we can prove our main result as
follows. Note that we assume a normal form of programs where the last
instruction is a decoding and public reveal of the output.
\begin{theorem}
  \label{theorem-gmw-NIMO}
  If $\eassign{\outv}{\ttt{decode}(e)}$ is a well-typed GMW circuit
  definition using certified components and
  $\config{\varnothing}{\eassign{\outv}{\ttt{decode}(e)}} \redxs
  \config{\prog}{\varnothing}$, then $\NIMO(\prog)$.
\end{theorem}
\begin{proof}
  Let $\iov(\prog) = S \cup (V \cup \outv)$. By definition of $\ttt{decode}$
  and assumptions of well-typedness we have for some $\ttt{g}$:
  $$
  \config{\varnothing}{\eassign{\outv}{\ttt{decode}(e)}} \redxs
  \config{\prog;\eassign{\outv}{\ttt{v[1,gout] xor v[2,gout]}}}{\varnothing}
  $$
  where by Lemmas \ref{lemma-gmw-encode} and \ref{lemma-gmw-preservation} we
  have $\sep{\progd(\prog)}{S_{\{1\}}}{V_{\{2\}}}$ and
  $\sep{\progd(\prog)}{S_{\{2\}}}{V_{\{1\}}}$.
  Let $\prog' \defeq \prog;\eassign{\outv}{\ttt{v[1,gout] xor v[2,gout]}}$. The
  preceding then implies
  $\sep{\progd(\prog')}{S_{\{1\}}}{V_{\{2\}}}$ and
  $\sep{\progd(\prog')}{S_{\{2\}}}{V_{\{1\}}}$,
  so that for all $\beta$:
  $$\sep{\condd{\progd(\prog')}{S_{\{1\}} \cup V_{\{2\}}}{\{ \outv \mapsto \beta \}}}{S_{\{1\}}}{V_{\{2\}}}
  \quad \text{and} \quad
    \sep{\condd{\progd(\prog')}{S_{\{2\}} \cup V_{\{1\}}}{\{ \outv \mapsto \beta \}}}{S_{\{2\}}}{V_{\{1\}}}$$
  thus for all $\store \in \mems(V_2 \cup \{ \outv \})$:
  $$\condd{\progd(\prog')}{S_{\{1\}}}{\store}
  = \condd{\progd(\prog')}{S_{\{1\}}}{\store_{\{\outv\}}}$$
  and for all $\store \in \mems(V_1 \cup \{ \outv \})$:
  $$\condd{\progd(\prog')}{S_{\{2\}}}{\store}
  = \condd{\progd(\prog')}{S_{\{2\}}}{\store_{\{\outv\}}}$$
  establishing the result by Lemma \ref{lemma-nimo} and Definition \ref{definition-NIMO}. 
\end{proof}

\subsection{Compositional Properties of YGC}
\label{section-composition-ygc}

\begin{fpfig}[t]{Yao's Garbled Circuits, auxiliary functions.}{fig-ygc-aux}
{\footnotesize
\begin{verbatimtab}
  keygen(gid, b1, b2) { select4(b1,b2,H[gid || "1"],H[gid || "2"],H[gid || "3"],H[gid || "4"]) }
  
  keysgen(gid, b1, b2)
  {
    let k11 = keygen(gid,b1,b2) in
    let k10 = keygen(gid,b1,not b2) in
    let k01 = keygen(gid,not b1,b2) in
    let k00 = keygen(gid,not b1,not b2) in
    {k11 = k11; k10 = k10; k01 = k01; k00 = k00}
  }
  
  andtable(keys, bt, ap, bp)
  {
    let r11 = (keys.k11 xor bt) in 
    let r10 = (keys.k10 xor (not bt)) in
    let r01 = (keys.k01 xor (not bt)) in
    let r00 = (keys.k00 xor (not bt)) in
    permute4(ap,bp,r11,r10,r01,r00)
  }
  
  sharetable(gid, vid, table)
  {   
    v[1, "gate:" || gid || vid || "1"] := table.v1;
    v[1, "gate:" || gid || vid || "2"] := table.v2;
    v[1, "gate:" || gid || vid || "3"] := table.v3;
    v[1, "gate:" || gid || vid || "4"] := table.v4
  }
\end{verbatimtab}
}
\end{fpfig}

\begin{fpfig}[t]{Yao's Garbled Circuits, garbled gates and evaluation code.}{fig-ygc-gates}
{\footnotesize
\begin{verbatimtab}
  garbledecode(wl)
  {
    let r1 = wl.k xor true in
    let r0 = (not wl.k) xor false in
    v[1,"OUTtt1"] := select[wl.p,r1,r0];
    v[1,"OUTtt2"] := select[not wl.p,r1,r0]
  }
  
  evaldecode(wl, p) { wl.k xor select[wl.p,v[1,"OUTtt1"],v[1,"OUTtt2"]] }
  
  evalgate(gid, wla, wlb)
  {
    let k = keygen(gid,wla.k,wlb.k) in
    let ct = select4(wla.p,wlb.p,
               v[1,gid || "tt1"],v[1,gid || "tt2"],v[1,gid || "tt3"],v[1,gid || "tt4"]) in
    let cp = select4(wla.p,wlb.p,
               v[1,gid || "pt1"],v[1,gid || "pt2"],v[1,gid || "pt3"],v[1,gid || "pt4"]) in
    { k = k xor ct; p = k xor cp }
  }
  
  andgate(gid, wla, wlb, wlc) 
  {
    let keys = keysgen(gid,wla.k,wlb.k) in
    sharetable(gid,"tt",andtable(keys,wlc.k,wla.p,wlb.p));
    sharetable(gid,"pt",andtable(keys,wlc.p,wla.p,wlb.p))
  }
  
  encode(gid, wla,wlb)
  {
    let wla = { k = flip[2,"fwl1"]; p = flip[2,"pwl1"] } in
    let wlb = { k = flip[2,"fwl2"]; p = flip[2,"pwl2"] } in
    { wv1 = { k = OT[s[1,"0"],wla.k,(not wla.k)]; p = OT[s[1,"0"],wla.p,(not wla.p)]}; 
      wv2 = { k = select[s[2,"0"],wlb.k,(not wlb.k)]; p = select[s[2,"0"],wlb.p,(not wlb.p)] } }
  }
\end{verbatimtab}
}
\end{fpfig}

%  andtable(keys, bt, ap, bp)
%  {
%    let r11 = (keys.k11 xor bt) in 
%    let r10 = (keys.k10 xor (not bt)) in
%    let r01 = (keys.k01 xor (not bt)) in
%    let r00 = (keys.k00 xor (not bt)) in
%    permute4(ap,bp,r11,r10,r01,r00)
%  }


In Figures \ref{fig-ygc-aux} and \ref{fig-ygc-gates} we define a
codebase for Yao's garbled circuits (YGC). This definition follows the
\emph{point-and-permute} method described in \cite{evans2018pragmatic}
and elsewhere, to which the reader is referred for more in-depth discussion.
In this implementation, client 2 is the \emph{garbler} and
client 1 is the \emph{evaluator}. The garbler builds the garbled
tables and shares them with the evaluator, who then evaluates
the gate in an oblivious fashion until the final public output is
generated through decryption. This definition is well-typed,
with input type annotations for top-level functions listed in
Figure \ref{fig-ygc-types}. Well-typed programs using these
libraries are therefore guaranteed to yield safe $\minifed$
programs. 

\emph{Wire labels} are fundamental to YGC and essentially represent
gate output values in an encrypted form. In our definition, wire
labels are represented by records $\ttt{\{ k = }\beta_1\ttt{; p =
}\beta_2\ttt{ \}}$, where $\ttt{k}$ is the \emph{key bit} and
$\ttt{p}$ is the \emph{pointer bit}, and $\beta_1$ and $\beta_2$ are
flips. Flips in each output wire label are owned by the garbler and
are unique per gate by definition of their identifying string, and the
representation of $0$ is the negation of $1$. For example, here is the
representation of 1 and 0 respectively in the output wire label for a
hypothetical gate 6:
\begin{mathpar}
  \ttt{\{ k = flip[2,gate:6.k]; p =  flip[2,gate:6.p]] \}}
    
  \ttt{\{ k = not flip[2,gate:6.k]; p =  not flip[2,gate:6.p]] \}}
\end{mathpar}
The pointer bits in wire labels are used to select permuted rows in
table garblings. The key bits are used to identify a unique key for
table row in each garbled gate. Intuitively, if $\beta_1$ and
$\beta_2$ are either key or pointer bits encoding 1 on two input wire
labels to a binary gate, rows and keys in the gate are enumerated in
the order:
$$
\neg\beta_1\neg\beta_2,\ \neg\beta_1\beta_2,\ \beta_1\neg\beta_2,\ \beta_1\beta_2
$$

In our implementation, gates are wired together using gate
identifiers, which are strings $w$. Top-level functionality in Figures
\ref{fig-ygc-aux} and \ref{fig-ygc-gates} includes the following:
\begin{itemize}
\item \ttt{andgate}: This defines a subprotocol for the garbler
  to define a garbled gate $\ttt{gid}$ with input wires from gates
  $\ttt{ga}$ and $\ttt{gb}$. The garbler generates keys and garbles
  the rows in YGC fashion, them with client $1$ in
  views in a standard form. For example, the view for
  a hypothetical gate 6, row 2 garbled truth table is $\ttt{v[1,gate:6tt2]}$.
  We note that garbled gates of other binary operators can be obtained with
  replacement of $\ttt{andtable}$ with appropriate garbled table definitions. 
\item \ttt{evalgate}: This defines a subprotocol for the evaluator to
  evaluate gate $\ttt{gid}$ given input wire values $\ttt{wva}$ and
  $\ttt{wvb}$.
\item \ttt{garbledecode} and \ttt{evaldecode}: The former function
  defines the garbler's protocol for encrypting the circuit
  output from final gate $\ttt{gid}$, and the latter defines
  the evaluator's output decryption protocol.
\item \ttt{encode}: This defines the initial phase of the protocol,
  where the evaluator receives the wire value from their own
  secret $\sx{1}{sa}$ via $\ttt{OT}$, and the garbler communicates
  the wire value for their own secret $\sx{2}{sb}$ directly.
\end{itemize}
\begin{example}
  \label{example-andcircuit}
The following program uses our YGC library to define
a circuit with a single and gate and input secrets $\ttt{s1}$ and
$\ttt{s2}$ from clients 1 and 2 respectively. 
\begin{verbatimtab}
  andgate(0,s1,s2);
  garbledecode(0);
  let secrets = encode(s1,s2) in
  v[0,output] := decode(evalgate(0, secrets.wv1, secrets.wv2))
\end{verbatimtab}
\end{example}
We have verified passive security of the $\fedprot$ protocol
generated by this and other small circuits using the
technique described in Lemma \ref{lemma-bruteforce-nimo}.
But large circuits with thousands of gates would be
intractable to verify with this method. In the next Section
we discuss compositional methods to address this issue.

\subsubsection{Gate Certification}
\label{section-ygc-certification}

Our certification method relies on a definition of compositional units comprising
complete related subprotocols-- i.e., both the garbler's
construction of the gates and the evaluator's evaluation of them. We
therefore make the following definitions. However, we note that our
formal results are applicable to the YGC style where the garbler
shares the entire circuit prior to evaluation.
\begin{definition}
  \label{ygc-modules}
  We define a composable internal \ttt{and} gate sharing and evaluation function
  called $\ttt{andgg}$, and an output decoder gate share and evaluation function called
  $\ttt{decode}$, as follows. 
\begin{verbatimtab}
  andgg(g, ga, gb, wva, wvb) { andgate(g, ga, gb); evalgate(g, wva, wvb) }
  decode(g, v) { garbledecode(g); evaldecode(v) }
\end{verbatimtab}
\end{definition}

Our certification techniques are defined generally wrt implementations
of input encoding, decoding, and internal gates. In the following we
will refer to arbitrary decode, encode, and gate functions with the
restriction that any function in each of these categories has the same
valid type signature as $\ttt{decode}$, $\ttt{encode}$, and
$\ttt{andgg}$ respectively.

It is necessary to define notions of correlation with wire
labels, since wire values flow into and out of gates in either
positive or negative correlation with them.
\begin{definition}
  \label{definition-gc}
  We write $\gc{\prog}{\{ \ttt{k = }\be_{k}; \ttt{p = }\be_{p} \}}{g}$ iff one of the
  following hold:
  \begin{enumerate}[\hspace{5mm}i.]
  \item $\vc{\progtt(\prog,\be_k)}{\itv}{\ttt{flip[2,gate:}g\ttt{.k]}}\quad \text{and} \quad
    \vc{\progtt(\prog,\be_p)}{\itv}{\ttt{flip[2,gate:}g\ttt{.p]}}$
  \item $\vc{\progtt(\prog,\enot\ \be_k)}{\itv}{\ttt{flip[2,gate:}g\ttt{.k]}}\quad \text{and}\quad
    \vc{\progtt(\prog,\enot\ \be_p)}{\itv}{\ttt{flip[2,gate:}g\ttt{.p]}}$
  \end{enumerate}
\end{definition}
Thus, if a value is in correlation with a wire label, it is either
correlated with the wire label or with the inverse of both the key and
pointer bits. For subsequent discussion, we make the following addition
to the codebase:
$$
\ttt{invert(\{ k = bk; p = bp \}) \{ \{ k = not bk; p = not bp \} \}}
$$
Gate certification establishes separation of input labels from the
gate table and output value, and correlation of the output with
the output wire label under any input wire value combination.
\begin{definition}[YGC Gate Certification]
  \label{definition-ygcgate-certification}
  Let $\mathit{gate}$ be a gate function.
  For $1 \le i \le 4$  let
  $
  \config{\varnothing}{e_i}\redxs\config{\prog}{v_i}
  $
  where:
  \begin{enumerate}[\hspace{5mm}$e_1 \defeq$]
    \item $\mathit{gate}\ttt{(c,a,b,owl(a),owl(b));}$
    \item $\mathit{gate}\ttt{(c,a,b,owl(a),invert(owl(b)));}$
    \item $\mathit{gate}\ttt{(c,a,b,invert(owl(a)),owl(b));}$
    \item $\mathit{gate}\ttt{(c,a,b,invert(owl(a)),invert(owl(b)))}$
  \end{enumerate}
  Define:
  $${\small X \defeq \ttt{\{ flip[2,gate:a.k], flip[2,gate:a.p], flip[2,gate:b.k], flip[2,gate:b.p] \}}}$$
  Then $\mathit{gate}$ is \emph{certified} iff the following conditions hold:
  \begin{enumerate}[\hspace{5mm}i.]
  \item $\sep{\progtt(\prog)}{X}{\vdefs(\prog)}$
  \item For all $1 \le i \le 4$, $\gc{\prog}{v_i}{\ttt{c}}$
  \end{enumerate}
\end{definition}
Encode certification establishes correlation of the encoded input values with
the secret wire labels, and separation of client 2's secret from client 1's views. 
\begin{definition}[YGC Encode Certification]
  \label{definition-ygcencode-certification}
  Let $\mathit{encode}$ be an encoding function, and let:
  $$
  \config{\varnothing}{\ttt{encode(s1,s2)}} \redxs
  \config{\prog}{\ttt{\{wv1=}v_1\ttt{;wv2=}v_2\ttt{\}}}
  $$
  Then $\mathit{encode}$ is \emph{certified} iff each of the following conditions hold:
  \begin{enumerate}[\hspace{5mm}i.]
  \item $\gc{\prog}{v_1}{\ttt{s1}} \quad \text{and} \quad \gc{\prog}{v_2}{\ttt{s2}}$
  \item $\sep{\progd(\prog)}{\{\ttt{s[1,s1]\}}}{\vdefs(\prog)_{\{2\}}}$
  \item $\sep{\progd(\prog)}{\{\ttt{s[2,s2]\}}}{\vdefs(\prog)_{\{1\}}}$    
  \item All $x \in \vars(\prog)$ are distinguished by $\ttt{in1}$ or $\ttt{in2}$. 
  \end{enumerate}
\end{definition}
Decode certification establishes separation of the input wire
label from the decoding tables under any input value condition. 
\begin{definition}[YGC Decode Certification]
  \label{definition-ygdecode-certification}
  Let $\mathit{decode}$ be a decoding function, and let:
  $$
  \config{\varnothing}{\mathit{decode}(\ttt{c}, \ttt{owl(c)})}
  \redxs \config{\prog}{\be}
  %\quad \text{and} \quad
  %\config{\varnothing}{\mathit{decode}(\ttt{c}, \ttt{invert(owl(c))})}
  %\redxs \config{\prog}{\be_0}
  $$
  Then $\mathit{decode}$ is certified iff:
  $$
  \sep{\progtt(\prog)}{\ttt{\{flip[2,gate:c.k],flip[2,gate:c.p]\}}}{\vdefs(\prog)}
  $$
  %Then $\mathit{decode}$ is certified iff both of the following conditions hold:
  %\begin{enumerate}[\hspace{5mm}i.]
  %\item $\progd(\prog_1,\be_1)(\{ \itv \mapsto 1 \}) = 1$ and
  %  $\progd(\prog_0,\be_0)(\{ \itv \mapsto 0 \}) = 1$
  %\item $\sep{\progtt(\prog_i)}{\ttt{\{flip[2,gate:c.k],flip[2,gate:c.p]\}}}{\vdefs(\prog_i)}$
  %\end{enumerate}
\end{definition}
We have certified each of the YGC components detailed above. 
\begin{lemma}
  \label{lemma-ygc-certification}
  Each of $\ttt{decode}$, $\ttt{encode}$, and $\ttt{andgg}$ are certified.
\end{lemma}

\subsubsection{Gate Invariants and Circuit Security}
\label{section-composition-metatheory}

To demonstrate correctness of certification, we need to show that
isolated component certificates preserve relevant properties
when integrated into larger programs. 
%First, we show that if a flip $f$ is independent from the views
%in a protocol (as is the case with input labels used to
%encode a garbled table), then separation from views that
%use $f$ (e.g., in an output label) is preserved. 
%\begin{lemma}[Wire Framing]
%  \label{lemma-wire-framing}
%  If $\sep{\progtt(\prog)}{\{f\}}{\vdefs(\prog)}$ for flip $f$ and
%  $\vars(\prog) \cap \vars(\prog') = \{ f \}$ then
%  $\sep{\progtt(\prog';\prog)}{\vdefs(\prog')}{\vdefs(\prog)}$.
%\end{lemma}
%Next,
First, we establish another substitution result, showing that
correlations (e.g., between output wire values and labels) are
preserved by substitution of correlated values for flips (e.g.,
between input wire values and labels).
\begin{lemma}[Substitution$\sim$]
  \label{lemma-substitution-sim}
  If $\vc{\progtt(\prog_2,\be)}{\itv}{f}$ and $\vc{\progtt(\prog_1,\be')}{\itv}{f'}$
  and $\vars(\be') \cap (\vars(\be) - \{ f \}) = \varnothing$
  then $\vc{\progtt(\prog_1;\prog_2,\be[\be'/f'])}{\itv}{f}$.
\end{lemma}
Now we can show how to splice gates into arbitrary circuits in an
invariant-preserving manner, assuming that outputs are not wired to
multiple inputs \cite{tate2003garbled,nieminen2023breaking}.
\begin{lemma}[YGC Gate Invariant]
  \label{lemma-ygc-preservation}
Given well-typed $\prog_1;E[\mathit{gate}(g,g_1,g_2,v_1,v_2)]$ and certified $\mathit{gate}$ where:
$$
\config{\prog_1}{E[\mathit{gate}(g,g_1,g_2,v_1,v_2)]} \redxs \config{\prog_1;\prog_2}{E[v]}
$$
If the following preconditions hold where $\iov(\prog_1) = S \cup V$:
\begin{enumerate}
\item $\{ g_1 \} \cap \{ g_2 \} \cap \wired(\prog_1) = \varnothing$ and $g \not\in \prog_1$
\item $\gc{\prog_1}{v_1}{g_1}$ and $\gc{\prog_1}{v_2}{g_2}$
\item $\sep{\progd(\prog_1)}{S_{\{1\}}}{V_{\{2\}}}$
\item $\sep{\progd(\prog_1)}{S_{\{2\}}}{V_{\{1\}}}$
\end{enumerate}
then we have as postconditions where $\iov(\prog_1) = S \cup V'$:
\begin{enumerate}
\item $\gc{\prog_1;\prog_2}{v}{g_2}$
\item $\sep{\progd(\prog_1,\prog_2)}{S_{\{1\}}}{V'_{\{2\}}}$
\item $\sep{\progd(\prog_1,\prog_2)}{S_{\{2\}}}{V'_{\{1\}}}$
\end{enumerate}
\end{lemma}
\begin{proof}
  Given preconditions we have $v_1 =  \{ \ttt{k = }\be^1_{k}; \ttt{p = }\be^1_{p} \}$ and
  $v_2 = \{ \ttt{k = }\be^2_{k}; \ttt{p = }\be^2_{p} \}$ for some
  $\be^1_{k}$,$\be^2_{k}$,$\be^1_{p}$,$\be^2_{p}$ with correlations as per Definition
  \ref{definition-gc}.
  %\begin{eqnarray*}
  %  v_1 &=& \{ \ttt{k = }\be^1_{k}; \ttt{p = }\be^1_{p} \} \\
  %  v_2 &=& \{ \ttt{k = }\be^2_{k}; \ttt{p = }\be^2_{p} \}
  %\end{eqnarray*}
  Let $\prog$ be as defined in Definition \ref{definition-ygcgate-certification}.
  We observe:
  \begin{eqnarray*}
    &\prog_2 = \\
    &{\small \prog[\be^1_{k}/\ttt{flip[2,gate:a.k]}][\be^1_{p}/\ttt{flip[2,gate:a.p]}][\be^2_{k}/\ttt{flip[2,gate:b.k]}][\be^1_{p}/\ttt{flip[2,gate:b.p]}]}
  \end{eqnarray*}
  Given that $g_1$ and $g_2$ are distinct and not wired in $\prog_1$
  we are assured that $\ttt{owl}(g_1)$ and $\ttt{owl}(g_1)$ are in
  independent uniform distributions, and given that $g \not\in
  \prog_1$ we are assured that $\ttt{owl}(g)$ contains entirely fresh
  flips. Thus by condition (i) of Definition \ref{definition-ygcgate-certification} and
  Lemma \ref{lemma-substitution} we have:
  $$
  \sep{\prog_1;\prog_2}{\vars(\be^1_{k},\be^2_{k},\be^1_{p},\be^2_{p})}{\vdefs(\prog_2)}
  $$
  Thus by Lemmas \ref{lemma-noninterference} and \ref{lemma-separation} we establish
  postconditions (ii) and (iii).

  Also since $v_1$ and $v_2$ are correlated either positively or negatively with
  $\ttt{owl}(g_1)$ and $\ttt{owl}(g_2)$ respectively by precondition (ii),
  by Definition \ref{definition-gc}, precondition (i), and Lemma \ref{lemma-substitution-sim}
  we establish postcondition (i), since Definition \ref{definition-ygcgate-certification}
  requires gate output correlation with $\ttt{owl}(g)$ given any input valence conditions.
\end{proof}

\begin{lemma}[YGC Encode Invariant]
  \label{lemma-ygc-encode}
Given well-typed $\prog_1;E[\mathit{encode}(s_1,s_2)]$ and certified $\mathit{encode}$ where 
$$
\config{\prog_1}{E[\ttt{encode}(s_1,s_2)]} \redxs \config{\prog_1;\prog_2}{E[\ttt{\{wv1=}v_1\ttt{;wv2=}v_2\ttt{\}}]}
$$
If the following preconditions hold where $\iov(\prog_1) = S \cup V$:
\begin{enumerate}
\item $\sep{\progd(\prog_1)}{S_{\{1\}}}{V_{\{2\}}}$
\item $\sep{\progd(\prog_1)}{S_{\{2\}}}{V_{\{1\}}}$
\end{enumerate}
then we have as postconditions where $\iov(\prog_1) = S' \cup V'$:
\begin{enumerate}
\item $\gc{\prog_1;\prog_2}{v_1}{s_1}$ and $\gc{\prog_1}{v_2}{s_2}$
\item $\sep{\progd(\prog_1;\prog_2)}{S'_{\{1\}}}{V'_{\{2\}}}$
\item $\sep{\progd(\prog_1;\prog_2)}{S'_{\{2\}}}{V'_{\{1\}}}$
\end{enumerate}
\end{lemma}
\begin{proof}
  By Definition \ref{definition-ygcencode-certification} we have:
  \begin{mathpar}
    \sep{\progtt(\prog_2)}{\ttt{\{s[1,s1]\}}}{\vdefs(\prog_2)_{\{2\}}}
    
    \sep{\progtt(\prog_2)}{\ttt{\{s[2,s2]\}}}{\vdefs(\prog_2)_{\{1\}}}
  \end{mathpar}
  Since we assume that secrets are in uniform and independent marginal
  distributions a priori, and $\vars(\prog_1) \cap \vars(\prog_2) =
  \varnothing$ by condition (iv) of Definition
  \ref{definition-ygcencode-certification} and assumptions of
  well-typedness, conditions (ii-iii) follow by Lemmas
  \ref{lemma-noninterference} and \ref{lemma-separation}. Also by
  Definition \ref{definition-ygcencode-certification} we have
  $\gc{\prog_2}{v_1}{s_1}$ and $\gc{\prog_2}{v_1}{s_1}$, so also
  $\gc{\prog_1;\prog_2}{v_1}{s_1}$ and
  $\gc{\prog_1;\prog_2}{v_1}{s_1}$ by condition (iv) of Definition
  \ref{definition-ygcencode-certification} and Lemmas
  \ref{lemma-noninterference} and \ref{lemma-separation}.
\end{proof}

Given the above, we can now establish that any circuit built with
certified components is passive secure.  Note that we assume a normal
form of programs where the last instruction is a decoding and public
reveal of the output, and we require that no gate is wired more than once.
\cnote{There is another subtlety here, that gates are wired correctly
  in the sense that if the garbler generates a table with input wire label
  $g_1$ then only $g_1$ is wired to that input. I'll modify the
  typing to enforce this.}
\begin{theorem}
  \label{theorem-ygc-NIMO}
  If $\eassign{\outv}{\ttt{decode}(g,e)}$ is a well-typed YGC circuit
  definition using certified components where no gate output is wired
  more than once, and
  $\config{\varnothing}{\eassign{\outv}{\ttt{decode}(g,e)}} \redxs
  \config{\prog}{\varnothing}$, then $\NIMO(\prog)$.
\end{theorem}

\begin{proof}
  By assumptions of well-typedness we
  have for some $\be_k$, $\be_p$, and $\be$:
  $$
  \config{\varnothing}{\eassign{\outv}{\ttt{decode}(g,e)}} \redxs
  \config{\prog_1}{\eassign{\outv}{\ttt{decode}(g,\{ \ttt{k = }\be_k;  \ttt{p = }\be_p\})}}
  \redxs \config{\prog_1;\prog_2}{\eassign{\outv}{\be}}
  $$
  Let $\iov(\prog_1) = S \cup V$.
  By Lemmas \ref{lemma-ygc-encode} and \ref{lemma-ygc-preservation} we
  have $\sep{\progd(\prog_1)}{S_{\{1\}}}{V_{\{2\}}}$ and
  $\sep{\progd(\prog_1)}{S_{\{2\}}}{V_{\{1\}}}$.
  Let $\prog$ be as defined in Definition \ref{definition-ygdecode-certification}.
  We observe:
  \begin{eqnarray*}
    & \prog_2 = \\ 
    & \prog[\be_k/\ttt{\{flip[2,gate:c.k]}][\be_p/\ttt{flip[2,gate:c.p]}]
  \end{eqnarray*}
  so also by  Definition \ref{definition-ygdecode-certification} and
  Lemma \ref{lemma-substitution} we have:
  $$
  \sep{\progtt(\prog_1;\prog_2)}{\vars(\be_k,\be_p)}{\vdefs(\prog_2)}
  $$
  so, letting $\iov(\prog_1;\prog_2) = S \cup V'$ by
  Lemmas \ref{lemma-noninterference} \ref{lemma-separation} we have
  $\sep{\progd(\prog_1;\prog_2)}{S_{\{1\}}}{V'_{\{2\}}}$ and
  $\sep{\progd(\prog_1;\prog_2)}{S_{\{2\}}}{V'_{\{1\}}}$.
  Thus, letting $\prog' \defeq (\prog_1;\prog_2;\eassign{\outv}{\be})$, for all $\beta$:
  $$\sep{\condd{\progd(\prog')}{S_{\{1\}} \cup V_{\{2\}}}{\{ \outv \mapsto \beta \}}}{S_{\{1\}}}{V_{\{2\}}}
  \quad \text{and} \quad
    \sep{\condd{\progd(\prog')}{S_{\{2\}} \cup V_{\{1\}}}{\{ \outv \mapsto \beta \}}}{S_{\{2\}}}{V_{\{1\}}}$$
  thus for all $\store \in \mems(V_2 \cup \{ \outv \})$:
  $$\condd{\progd(\prog')}{S_{\{1\}}}{\store}
  = \condd{\progd(\prog')}{S_{\{1\}}}{\store_{\{\outv\}}}$$
  and for all $\store \in \mems(V_1 \cup \{ \outv \})$:
  $$\condd{\progd(\prog')}{S_{\{2\}}}{\store}
  = \condd{\progd(\prog')}{S_{\{2\}}}{\store_{\{\outv\}}}$$
  establishing the result by Lemma \ref{lemma-nimo} and Definition \ref{definition-NIMO}.
\end{proof}


\begin{fpfig}[t]{YGC copy gate definitions.}{fig-ygc-copy}
  {\footnotesize
    \begin{verbatimtab}
      sharetab2(gid, tid, k, p, b)
      {
        let r1 = k xor b in
        let r0 = (not k) xor (not b) in
        v[1,gate: || gid || tid || 1] := select(p,r1,r0);
        v[1,gate: || gid || tid || 2] := select(not p,r1,r0);
      }
      
      copygate(ca,cb,g)
      {
        let wl = owl(g) in
        let owl1 = owl(ca) in
        let owl2 = owl(cb) in
        sharetab2(ca,tt,wl.k,wl.p,owl1.k); sharetab2(ca,pt,wl.k,wl.p,owl1.p);
        sharetab2(cb,tt,wl.k,wl.p,owl2.k); sharetab2(cb,pt,wl.k,wl.p,owl2.p)
      }
      
      evalcopy(ca,cb,wv)
      {
        let wv1k = wv.k xor select(wv.p,v[1,gate: || ca || tt1], v[1,gate: || ca || tt2]) in
        let wv1p = wv.k xor select(wv.p,v[1,gate: || ca || pt1], v[1,gate: || ca || pt1]) in
        let wv2k = wv.k xor select(wv.p,v[1,gate: || cb || tt1], v[1,gate: || cb || tt2]) in
        let wv2p = wv.k xor select(wv.p,v[1,gate: || cb || pt1], v[1,gate: || cb || pt1]) in
        { wv1 = { k = wv1k; p = wv1p }; wv2 = { k = wv2k; p = wv2p } }  
      }

      copy(ca,cb,g,wv) { copygate(ca,cb,g); evalcopy(ca,cb,wv) } 
    \end{verbatimtab}
  }
\end{fpfig}

\subsection{Building and Extending Automatically Secure Circuits}
\label{section-composition-copy}

Theorems \ref{theorem-gmw-NIMO} \ref{theorem-ygc-NIMO} means that any
well-typed program using certified components will generate a circuit
that is passive secure. To add an $\ttt{or}$ gate to our YGC library,
for example, we can use $\ttt{andgg}$ as a template, in fact simple
modification of $\ttt{andtable}$ is all that would be needed. Thus,
this library embodies a Fairplay-like language where well-typedness of
metaprograms guarantees safety and security of generated protocols.

The requirement that output wire labels are not used more than once
in YGC reflects a known result, that naive copy gate output is unsound
in YGC- i.e., passive security fails. In fact, we are unable to
automatically verify $\gNIMO$ for $\ttt{andgg}$ without the assumption
that input wire labels are in independent uniform distributions. We
can capture the necessary fix as in Figure \ref{fig-ygc-copy}. The
garbler creates a copy gate with two new output wire labels in
independent uniform distributions (in \ttt{copygate}), and permutes
them using the input wire label pointer after encrypting with the
input wire label keys. The evaluator then recovers the encrypted
output wire values during evaluation (in \ttt{evalcopy}). Given an
$\ttt{orgg}$ gate defined as outlined above, we can then use secure
$\ttt{copy}$ as follows:
\begin{verbatimtab}
  let s = encode(s1,s2) in
  let s1c = copy(c1s1,c2s1,s1,s.wv1) in
  let s2c = copy(c1s2,c2s2,s2,s.wv2) in
  let wv1 = andgg(1,c1s1,c1s2,s1c.wv1,s2c.wv1) in
  let wv2 = orgg(2,c2s1,c2s2,s1c.wv2,s2c.wv2) in
  decode(3,andgg(3,1,2,wv1,wv2))
\end{verbatimtab}
These definitions of $\ttt{copy}$ and $\ttt{orgg}$ have been
automatically verified as in Section \ref{section-pre-post} for
$\ttt{andgg}$. In general, any new gate form- such as optimized gates
\cite{XXX}- can be verified automatically in a similar way.

While the copy gate supports secure programming patterns, security
still relies on programmer discipline, i.e., not wiring a gate more
than once. However, we can keep track of wired gates in type effects,
similar to the manner in which we keep track of defined views- type
dependence in our system is able to accurately precisely track
gate identifiers. This is also reminiscent of the use of linearity
in other related systems for enforcing obliviousness \cite{darais2019language}. 

