\documentclass[acmsmall,screen,review]{acmart}

\usepackage{amsmath}
\usepackage{amstext}
\usepackage{fp-frame}
\usepackage[latin1]{inputenc}
\usepackage{mathpartir}
\usepackage{fancyvrb}
\usepackage{moreverb}
\usepackage{stmaryrd}
\usepackage{enumerate}
\usepackage{thmtools,thm-restate}
\usepackage{comment}

\newcommand{\note}[1]{\noindent\textit{(\textbf{$\star$note$\star$:}\ \ #1)}}

\newcommand{\evals}{\Downarrow}
\newcommand{\diverges}{\Uparrow}
\newcommand{\intt}{\mathrm{int}}
\newcommand{\unitt}{\mathrm{unit}}
\newcommand{\boolt}{\mathrm{bool}}
\newcommand{\floatt}{\mathrm{float}}
\newcommand{\stringt}{\mathrm{string}}
\newcommand{\chart}{\mathrm{char}}
\newcommand{\vb}[1]{\verb+#1+}
\newcommand{\evalexmp}[2]{\texttt{#1}\ \ensuremath{\evals}\ \texttt{#2}}
\newcommand{\texmp}[2]{\texttt{#1\ :\ #2}}
\newcommand{\skipper}{\bigskip\\}
\newcommand{\fyi}{\noindent\textbf{\textit{fyi:}}\ }
\newcommand{\NB}{\noindent\textbf{NB:\ }}
\newcommand{\const}{\ensuremath{\mathbf{c}}}
\newcommand{\defn}{\heading{definition}}
\newcommand{\defeq}{\triangleq}
\newcommand{\nat}{\mathbb{N}}
\newcommand{\atom}{\texttt{const}}
\def\squareforqed{\hbox{\rlap{$\sqcap$}$\sqcup$}}
\def\qed{\ifmmode\squareforqed\else{\unskip\nobreak\hfil
\penalty50\hskip1em\null\nobreak\hfil\squareforqed
\parfillskip=0pt\finalhyphendemerits=0\endgraf}\fi}
\newcommand{\exampletab}[1]{\skipper\begin{tabular}{lll}#1\end{tabular}\skipper}
\newcommand{\verbtab}[1]{\skipper\begin{verbatimtab}{#1}\end{verbatimtab}\skipper}
\newcommand{\eqntab}[1]{\skipper\begin{tabular}{rcl}#1\end{tabular}\skipper}
\newcommand{\recdefn}[1]{\{#1\}}
\newcommand{\ttt}[1]{\texttt{#1}}
\newcommand{\gdesc}[1]{\text{\textit{#1}}}
\newcommand{\true}{\mathrm{true}}
\newcommand{\false}{\mathrm{false}}
\newcommand{\etrue}{\texttt{true}}
\newcommand{\efalse}{\texttt{false}}
\newcommand{\reval}{\Rightarrow}
\newcommand{\Dand}{\ \mathrm{and}\ }
\newcommand{\Dor}{\ \mathrm{or}\ }
\newcommand{\Dxor}{\ \mathrm{xor}\ }
\newcommand{\Dnot}{\mathrm{not}\ }
\newcommand{\cod}[1]{\llbracket #1 \rrbracket}
\newcommand{\lcod}[2]{\llbracket #1 \rrbracket_{#2}}
\newcommand{\Dplus}{\mathrm{Plus}}
\newcommand{\Dminus}{\mathrm{Minus}}
\newcommand{\Dequal}{\mathrm{=}}
\newcommand{\Dabs}[2]{(\mathrm{Function}\ #1 \rightarrow #2)}
\newcommand{\Dfix}[3]{(\mathrm{Fix}\ #1 . #2 \rightarrow #3)}
\newcommand{\Dite}[3]{\mathrm{If}\ #1\ \mathrm{Then}\ #2\ \mathrm{Else}\ #3}
\newcommand{\dotminus}{\stackrel{.}{-}}
\newcommand{\Dlet}[3]{\mathrm{Let}\ #1 = #2\ \mathrm{In}\ #3}
\newcommand{\Dletrec}[4]{\mathrm{Let\ Rec}\ #1\ #2 = #3\ \mathrm{In}\ #4}
\newcommand{\Dfst}{\mathrm{left}}
\newcommand{\Dsnd}{\mathrm{right}}
\newcommand{\labset}{\mathit{Lab}}
\newcommand{\Drec}[1]{\{ #1 \}}
\newcommand{\linfer}[3]{\inferrule*[right=(\TirName{#1})]{#2}{#3}}
\newcommand{\lab}[1]{\mathrm{#1}}
\newcommand{\loc}{\ell}
\newcommand{\Dref}[1]{\mathrm{Ref}\,#1}
%\newcommand{\store}{\mathcal{M}}
\newcommand{\store}{m}
%\newcommand{\stores}{\overline{\store}}
\newcommand{\stores}{M}
\newcommand{\config}[2]{\langle #1,#2 \rangle}
\newcommand{\configf}[2]{\begin{array}[t]{l}\langle #1\\,\\ #2 \rangle \end{array}}
\newcommand{\extend}[3]{#1\{#2 \mapsto #3\}}
\newcommand{\emptystore}{\{\}}
\newcommand{\storedefn}[1]{\{#1\}}
\newcommand{\Dret}[1]{\mathrm{Return}\,#1}
\newcommand{\Draise}[1]{\mathrm{Raise}\,#1}
\newcommand{\Dexn}[2]{\#\!#1\,#2}
\newcommand{\Dtry}[3]{\mathrm{Try}\,#1\,\mathrm{With}\,#2 \rightarrow #3}
\newcommand{\xname}{\mathit{exn}}
\newcommand{\Dboolt}{\mathrm{Bool}}
\newcommand{\Dreft}[1]{#1\,\mathrm{ref}}
\newcommand{\reft}[1]{#1\,\mathrm{ref}}
\newcommand{\Dintt}{\mathrm{Int}}
%\newcommand{\tjudge}[3]{#1 \vdash #2 : #3}
\newcommand{\textend}[3]{#1;#2:#3}
\newcommand{\fnty}[2]{#1 \rightarrow #2}
\newcommand{\TDabs}[3]{(\mathrm{Function}\ (#1 : #2) \rightarrow #3)}
\newcommand{\TDfix}[5]{(\mathrm{Fix}\ #1 . (#2 : #3) : #4 \rightarrow #5)}
\newcommand{\TDletrec}[6]{\Dletrec{#1}{#2 : #3}{#4 : #5}{#6}}
\newcommand{\emptyenv}{\varnothing}
\newcommand{\tfail}{\mathbf{fail}}
\newcommand{\tcheck}{\mathrm{TC}}
\newcommand{\tcheckfail}{\mathbf{TypeMismatch}}
\newcommand{\algtab}[1]
{
\vspace*{-3mm}
\begin{tabbing}
\hspace*{12mm}\=\hspace{9mm}\=\hspace{9mm}\=\hspace{6mm}\=\hspace{6mm}\=
\hspace{6mm}\=
#1
\end{tabbing}
}
\newcommand{\eassign}[2]{#1 := #2}
\newcommand{\ederef}[1]{\,!#1}
\newcommand{\declass}[2]{\mathrm{declassify}_{#2}(#1)}
\newcommand{\eendorse}[2]{\mathrm{endorse}_{#2}(#1)}
\newcommand{\lt}{\left\{}
\newcommand{\rt}{\right\}}
\newcommand{\Lt}{\left\{\!\!\right.}
\newcommand{\Rt}{\left.\!\!\right\}}
\newcommand{\tinfer}{\mathit{PT}}
\newcommand{\unify}{\mathit{unify}}
\newcommand{\tsubn}{\varphi}
\newcommand{\scheme}[2]{\forall #1 . #2}
\newcommand{\Dself}{\mathrm{this}}
\newcommand{\Dsuper}{\mathrm{super}}
\newcommand{\Dsend}[3]{#1.#2(#3)}
\newcommand{\Dselect}[2]{#1.#2}
\newcommand{\Demptyclass}{\mathrm{EmptyClass}}
%\newcommand{\Dclass}[3]{\mathrm{Class}\ \mathrm{Extends}\ #1\ \mathrm{Inst}
%\ #2\ \mathrm{Meth}\ #3}
\newcommand{\Dclass}[2]{\mathrm{Class}\ \mathrm{Inst} \ #1\ \mathrm{Meth}\ #2}
\newcommand{\Dobj}[2]{\mathrm{Object}\ \mathrm{Inst}\ #1\ \mathrm{Meth}\ #2}
%\newcommand{\Dclassf}[3]{
%\begin{array}[t]{l}
%\mathrm{Class}\ \mathrm{Extends}\ #1 \\
%\quad \mathrm{Inst}\\
%\qquad #2 \\ 
%\quad \mathrm{Meth}\\
%\qquad #3
%\end{array}
%}
\newcommand{\Dclassf}[3]{
\begin{array}[t]{l}
\mathrm{Class}\\
\quad \mathrm{Inst}\\
\qquad #1 \\ 
\quad \mathrm{Meth}\\
\qquad #2
\end{array}
}
\newcommand{\Dobjf}[2]{
\begin{array}[t]{l}
\mathrm{Object}\\
\quad \mathrm{Inst}\\
\qquad #1 \\ 
\quad \mathrm{Meth}\\
\qquad #2
\end{array}
}
\newcommand{\Dnew}[1]{\mathrm{New}\ #1}
\newcommand{\vtab}[1]{\begin{verbatimtab}[4]#1\end{verbatimtab}}

\newcounter{topiccounter}
\setcounter{topiccounter}{1}
\newcommand{\topic}[1]
    {\noindent \textbf{Topic \arabic{topiccounter}.\ \textit{#1}. } \stepcounter{topiccounter}}


\newcommand{\lcalc}{$\lambda$-calculus}
\newcommand{\redx}{\rightarrow}
\newcommand{\redxs}{\redx^*}
\newcommand{\idfn}{\mathit{ID}}
\newcommand{\mlfn}[2]{\mathrm{fun}\, #1 \rightarrow #2}
\newcommand{\mlrecfn}[3]{\mathrm{fix}\,#1.#2 \rightarrow #3}
\newcommand{\mlfix}{\mathrm{fix}}
\newcommand{\eite}[3]{\mathrm{if}\ #1\ \mathrm{then} \ #2\ \mathrm{else} \ #3\ }
\newcommand{\esucc}[1]{\texttt{succ}\ #1}
\newcommand{\epred}[1]{\texttt{pred}\ #1}
\newcommand{\eiszero}[1]{\texttt{iszero}\ #1}
\newcommand{\ezero}{\texttt{0}}
\newcommand{\elet}[3]{\mathrm{let}\ #1 = #2\ \mathrm{in}\ #3}
\newcommand{\eletrec}[3]{\mathrm{letrec}\ #1 = #2\ \mathrm{in}\ #3}
\newcommand{\fv}{\mathrm{fv}}
\newcommand{\ourml}{\mathit{ML}_{\mathit{Cat}}}
\newcommand{\raisexn}{\mathrm{raise}}
\newcommand{\handler}[3]{\mathrm{try}\, #1\, \mathrm{with}\, \exn(#2) \Rightarrow #3}
\newcommand{\exn}{\mathit{exn}}
\newcommand{\dom}{\mathrm{dom}}
\newcommand{\efst}{\mathrm{fst}}
\newcommand{\esnd}{\mathrm{snd}}
\newcommand{\natt}{\textrm{Nat}}
\newcommand{\earray}{\mathrm{array}}
\newcommand{\varray}{\alpha}
\newcommand{\length}{\mathit{length}}
\newcommand{\arrayml}{\ourml^{\earray}}
\newcommand{\stackml}{\ourml^{\mathit{stack}}}
\newcommand{\flowml}{\ourml^{\mathit{flow}}}
\newcommand{\taintml}{\ourml^{\mathit{taint}}}
\newcommand{\secfail}{\mathbf{secfail}}
\newcommand{\tr}{\theta}
\newcommand{\rewrite}[1]{\mathcal{R}(#1)}
%\newcommand{\secprop}{\mathcal{P}}
%\newcommand{\trprop}{\hat{\secprop}}
\newcommand{\Prop}{\mathbf{P}}
\newcommand{\Hprop}{\mathbf{H}}
\newcommand{\secprop}{\phi}
\newcommand{\hyprop}{\eta}
\newcommand{\trprop}{\gamma}
\newcommand{\tracess}{\Sigma}
\newcommand{\trsprop}{\sigma}
\newcommand{\traces}{\Psi}
\newcommand{\fpkeyword}[1]{\mathrm{#1}}
\newcommand{\ebinop}[2]{#1\,\mathit{binop}\,#2}
\newcommand{\eenablepriv}[2]{\fpkeyword{enable}\ #1\ \fpkeyword{for}\ #2}
\newcommand{\echeckpriv}[2]{\fpkeyword{check}\ #1\ \fpkeyword{then}\ #2}
\newcommand{\esigned}[2]{#1.#2}
\newcommand{\enabled}{\mathit{enabledprivs}}
%\newcommand{\acl}{\mathcal{A}}
\newcommand{\priv}{\pi}
\newcommand{\privs}{\mathit{R}}
\newcommand{\prin}{p}
\newcommand{\nobody}{\mathit{nobody}}
\newcommand{\po}{\preceq}
\newcommand{\seclattice}{\mathcal{S}}
\newcommand{\binsl}{\mathcal{S}_{\mathrm{bin}}}
\newcommand{\seclevs}{\mathcal{L}}
\newcommand{\latel}{\varsigma}
\newcommand{\hilab}{H}
\newcommand{\lolab}{L}
\newcommand{\hiloc}{\mathit{hi}}
\newcommand{\loloc}{\mathit{low}}
\newcommand{\labty}[2]{#1 \cdot #2}
\newcommand{\labval}[2]{#1 \cdot #2}
\newcommand{\mi}[1]{\mathit{#1}}
\newcommand{\pc}{\latel_{\mathit{pc}}}
\newcommand{\cfnty}[3]{#1 \rightarrow_{#2} #3}
\newcommand{\pow}{\mathrm{pow}}

%\newcommand{\tr}{\theta}


\newcommand{\problemheading}[1]{\noindent\textbf{#1}\ }
\newcounter{problemcounter}
\setcounter{problemcounter}{1}

\newcommand{\EC}[1]
    {\problemheading{Extra Credit \textit{(#1 points)}.}}

\newcommand{\pproblem}[1]
    {\problemheading{Problem \arabic{problemcounter} \textit{(#1 points)}.} \stepcounter{problemcounter}}
           
\newcommand{\gproblem}[1]
    {\problemheading{Problem \arabic{problemcounter} \textit{(Graduate Students Only; #1 points)}.} \stepcounter{problemcounter}}

\newcommand{\problem}
    {\problemheading{Problem \arabic{problemcounter}.} \stepcounter{problemcounter}}

\newcounter{solncounter}
\setcounter{solncounter}{1}
\newcommand{\solution}
    {\problemheading{Solution to Problem \arabic{solncounter}.} \stepcounter{solncounter}}

\newcommand{\chash}{\mathcal{H}}
\newcommand{\acl}{\mathit{Auth}}
\newcommand{\opn}{\mathit{op}}
\newcommand{\egid}{\mathit{egid}}
\newcommand{\euid}{\mathit{euid}}
\newcommand{\suid}{\ttt{suid}}
\newcommand{\sgid}{\ttt{sgid}}
\newcommand{\uxroot}{\ttt{root}}
\newcommand{\fowner}[1]{\mathit{owner}_{#1}}
\newcommand{\fgroup}[1]{\mathit{group}_{#1}}
\newcommand{\gprivs}[1]{\mathit{Privs_{#1}}.\mathit{group}}
\newcommand{\uprivs}[1]{\mathit{Privs_{#1}}.\mathit{owner}}
\newcommand{\oprivs}[1]{\mathit{Privs_{#1}}.\mathit{other}}
\newcommand{\uxprivs}[1]{\mathit{Privs_{#1}}}

\newcommand{\seclab}{\mathcal{L}}
\newcommand{\sle}{\preceq}
\newcommand{\ile}{\preceq_I}
\newcommand{\ilab}{\seclab_I}

\newcommand{\minifed}{\mathit{Overture}}
\newcommand{\minicat}{\minifed}
\newcommand{\fedprot}{\minifed}
\newcommand{\metaprot}{\mathit{Prelude}}
\newcommand{\mlscat}{\mathit{mlscat}}
\newcommand{\flowcat}{\mathit{flowcat}}
\newcommand{\dflowcat}{\mathit{dflowcat}}
\newcommand{\minicatde}{\mathit{minicat}_{\mathit{de}}}
\newcommand{\minicatexp}{\mathit{minicat}_{\mathit{taint}}}
%\newcommand{\prog}{\mathcal{P}}
\newcommand{\prog}{\pi}
\newcommand{\main}{\mathit{main}}
\renewcommand{\reval}{\redx}
\renewcommand{\Dite}{\eite}


%\renewcommand{\labty}[2]{#2}
\newcommand{\fnsty}{\Sigma}
\newcommand{\secty}{\latel}

\newcommand{\tc}{\mathrm{TC}}
\newcommand{\validate}{\mathrm{validate}}


\newcommand{\mlsid}[1]{\mathrm{mls}(#1)}
\newcommand{\mlsredx}[1]{\redx_{\mlsid{#1}}}
\newcommand{\confid}{\mathit{flow}}
\newcommand{\taintid}{\mathit{dflow}}
\newcommand{\credx}{\redx_{\confid}}
\newcommand{\tredx}{\redx_{\taintid}}
\newcommand{\ccod}[1]{\lcod{\confid}{#1}}
\newcommand{\tcod}[1]{\lcod{\taintid}{#1}}
\renewcommand{\mod}{\ \textrm{mod}\ }


\newcommand{\mtrace}[1]{\mathit{trace}_{#1}}
\newcommand{\mtraces}[1]{\mathit{traces}_{#1}}
\newcommand{\head}{\mathit{hd}}
\newcommand{\memt}{\mathit{mems}}

\newcommand{\bop}{\ \mathit{binop}\ }
\newcommand{\ak}{K}
\newcommand{\ik}{K_i}
%\newcommand{\deassign}[2]{\eassign{#1}{\mathrm{declassify}(#2)}
\newcommand{\deassign}[2]{#1 :=  [#2]_\wedge }
%\newcommand{\deassign}[2]{#1\ \wedge\!\,=  #2}

\newcommand{\mems}{\mathit{mems}}
\newcommand{\mto}{\mapsto}
\newcommand{\pdf}[1]{D_{#1}}
\newcommand{\margd}[2]{{#1}_{#2}}
\newcommand{\condd}[3]{#1_{({#2}|{#3})}}
\newcommand{\progd}{\mathrm{PD}}
\newcommand{\progtt}{\mathrm{BD}}
\newcommand{\vars}{\mathit{vars}}
\newcommand{\iov}{\mathit{iovars}}
\newcommand{\flips}{\mathit{flips}}
\newcommand{\keys}{\mathit{keys}}
\newcommand{\fedcat}{\minifed}

\newcommand{\sx}[2]{\texttt{s[#1,"#2"]}}
\newcommand{\fx}[2]{\texttt{f[#1,"#2"]}}
\newcommand{\vx}[2]{\texttt{v[#1,"#2"]}}

\newcommand{\IF}[1]{#1_{\mathit{i}}}
\newcommand{\idealf}{\mathcal{F}}
\newcommand{\SIM}{\mathrm{Sim}}
\newcommand{\prob}{\mathrm{Pr}}
\newcommand{\dist}{\mathrm{D}}

\def\TirName#1{\text{\sc #1}}

\newcommand{\srct}{\tau}
\newcommand{\cidty}[1]{\ttt{cid(}#1\ttt{)}}
\newcommand{\stringty}[1]{\ttt{string(}#1\ttt{)}}
\newcommand{\unity}{\mathtt{unit}}
\newcommand{\jpdty}[2]{\mathtt{jpd}(#1,#2)}
\newcommand{\viewst}{\mathcal{V}}
\newcommand{\tjudge}[5]{#1, #2 \vdash #3 : #4,#5}
\newcommand{\bet}[1]{\ttt{bool[}#1\ttt{]}}
\newcommand{\tas}{\mathcal{A}}


\newcommand{\flip}[2]{\ttt{flip[}#1\ttt{,}#2\ttt{]}}
\newcommand{\secret}[2]{\ttt{s[}#1\ttt{,}#2\ttt{]}}
\newcommand{\view}[2]{\ttt{v[}#1\ttt{,}#2\ttt{]}}
\newcommand{\oracle}[1]{\ttt{H[}#1\ttt{]}}
\newcommand{\Oracle}{H}
\renewcommand{\etrue}{\ttt{true}}
\renewcommand{\efalse}{\ttt{false}}
\newcommand{\enot}{\ttt{not}}
\newcommand{\eand}{\ttt{and}}
\newcommand{\eor}{\ttt{or}}
\newcommand{\exor}{\ttt{xor}}
\renewcommand{\elet}[3]{\ttt{let}\ #1\ \ttt{=}\ #2\ \ttt{in}\ #3}
\newcommand{\vloc}[2]{#1@#2}
\renewcommand{\redx}{\xrightarrow{}}
\renewcommand{\redxs}{\xrightarrow{}^{*}}
\newcommand{\lredx}[1]{\xrightarrow{#1}}
\newcommand{\mem}{M}
\newcommand{\randos}{R}
\newcommand{\tape}{\randos}
\newcommand{\secrets}{S}
\newcommand{\clients}{C}
\newcommand{\views}{V}
\newcommand{\str}{\varsigma}
\newcommand{\cid}{\iota}
\newcommand{\send}[2]{#1\ \ttt{:=}\ #2}
\newcommand{\OT}[3]{\ttt{OT(} #1 \ttt{,}\ #2 \ttt{,}\ #3 \ttt{)}}
\newcommand{\select}[3]{\ttt{select(} #1 \ttt{,}\ #2 \ttt{,}\ #3 \ttt{)}}
\newcommand{\codebase}{\mathcal{C}}
\newcommand{\interp}[1]{\llbracket #1 \rrbracket}
\newcommand{\finterp}[2]{\llbracket #1 \rrbracket_{#2}}
\newcommand{\prot}{\rho}
\newcommand{\Tapes}{\mathcal{R}}
\newcommand{\outloc}{\mathit{output}}
\newcommand{\pdist}{\mathit{pd}}
\newcommand{\genpdf}{\mathrm{PD}}
\newcommand{\card}[1]{|#1|}
\newcommand{\setdefn}[2]{\{#1\ |\ #2 \}}
\newcommand{\tapes}{\mathit{tapes}}
\newcommand{\nimo}{\mathit{NIMO}}
\newcommand{\pni}{\mathit{PNI}}
\newcommand{\passec}{PS}
\newcommand{\parties}{\mathcal{P}}
\newcommand{\iout}{\mathit{output}}
\newcommand{\kideal}{k_i}
\newcommand{\jpdf}{\mathrm{pdf}}
\newcommand{\leakproof}{\mathit{LP}}
\newcommand{\flab}{\ell}
\newcommand{\be}{\varepsilon}
\newcommand{\instr}{\mathbf{c}}
\newcommand{\solve}[2]{\mathit{models}\ #1\ #2}
\newcommand{\itv}{\mathit{it}}
\newcommand{\outv}{\mathit{out}}
\newcommand{\NIMO}{\mathit{NIMO}}
\newcommand{\gNIMO}{\mathit{gNIMO}}
\newcommand{\gates}{\mathit{gates}}
\newcommand{\owl}{\mathit{owl}}
\newcommand{\logit}[1]{\lfloor #1 \rfloor}
\newcommand{\runs}{\mathit{runs}}
\newcommand{\cruns}{\hat{\mathit{runs}}}
\newcommand{\cprogd}{\hat{\progd}}
\newcommand{\cprogtt}{\hat{\progtt}}
\newcommand{\datalog}{\mathit{datalog}}
\newcommand{\concat}{\ttt{|\!|}}
\newcommand{\wired}{\mathit{wired}}
\newcommand{\gc}[3]{\mathit{goc}(#1,#2,#3)}
\newcommand{\vc}[3]{#1 \vdash #2 \sim #3}
\newcommand{\sep}[3]{#1 \vdash #2 * #3}
\newcommand{\gtab}{\mathit{table}}
\newcommand{\vdefs}{\mathit{vdefs}}
\newcommand{\funcVar}{\$}
%\newcommand{\pmf}{\mathrm{Pr}}
\newcommand{\pmf}{\mathit{P}}

%%%% REVISION DEFS

\renewcommand{\flip}[1]{\ttt{r[}#1\ttt{]}}
\newcommand{\locflip}{\ttt{r[}\mathit{local}\ttt{]}}
\renewcommand{\secret}[1]{\ttt{s[}#1\ttt{]}}
\newcommand{\key}[1]{\ttt{k[}#1\ttt{]}}
\newcommand{\mesg}[1]{\ttt{m[}#1\ttt{]}}
\newcommand{\out}[1]{\elab{\ttt{out}}{#1}}
\newcommand{\rvl}[1]{\ttt{p[}#1\ttt{]}}
\renewcommand{\oracle}[1]{\ttt{H[}#1\ttt{]}}
\newcommand{\elab}[2]{#1_{#2}}
\renewcommand{\eassign}[4]{\elab{#1}{#2} := \elab{#3}{#4}}
\newcommand{\pubout}[3]{\out{#1} := \elab{#2}{#3}}
\newcommand{\reveal}[3]{\rvl{#1} := \elab{#2}{#3}}
\newcommand{\sk}[1]{\mathrm{sk}[#1]}
\newcommand{\pk}[2]{\mathrm{pk}[#1,#2]}
\newcommand{\kgen}[1]{\mathit{kgen}(#1)}
\newcommand{\adversary}{\mathcal{A}}
\newcommand{\aredx}{\redx_{\adversary}}
\newcommand{\aredxs}{\redxs_{\adversary}}
\newcommand{\arewrite}{\mathit{rewrite}_{\adversary}}
\newcommand{\cinputs}{V_{C \rhd H}}
\newcommand{\houtputs}{V_{H \rhd C}}
\newcommand{\aruns}{\mathit{runs}_\adversary}
\newcommand{\att}{\mathrm{AD}}
\newcommand{\support}{\mathit{support}}
\renewcommand{\store}{\sigma}
\newcommand{\ctxt}[2]{\{ #1 \}_{#2}}
\newcommand{\cpub}{\mathit{pub}}
\renewcommand{\runs}{\mathit{runs}}
\newcommand{\pattern}[1]{\lfloor #1 \rfloor}
\newcommand{\fcod}[1]{\lcod{#1}{}}
\renewcommand{\flips}{\mathit{rands}}
\newcommand{\kmat}{\kappa}
\renewcommand{\Oracle}{\mathbb{O}}
\newcommand{\afilter}{\mathit{afilter}}
\renewcommand{\select}[3]{\mathtt{if}\ #1\ \mathtt{then}\ #2\ \mathtt{else}\ #3}
\newcommand{\fp}{\mathit{P}}
\newcommand{\ftimes}{*}
\newcommand{\fplus}{+}
\newcommand{\fminus}{-}
\newcommand{\mactimes}{\,\hat{\ftimes}\,}%{\otimes}
\newcommand{\macplus}{\,\hat{\fplus}\,}%\oplus}
\newcommand{\macminus}{\,\hat{\fminus}\,}%{\ominus}
\newcommand{\macv}[1]{\langle #1 \rangle}
\newcommand{\mack}[2]{\langle #1 \rangle.\ttt{k}_{#2}}
\newcommand{\macshare}[1]{\langle #1 \rangle.\ttt{share}}
\newcommand{\macauth}{\mathrm{auth}}
\newcommand{\fieldty}{\mathrm{F}}
\newcommand{\cipherty}{\mathit{c}}
\newcommand{\macty}{\hat{\fieldty}}%_{\mathit{mac}}}}
\renewcommand{\unity}[1]{\mathit{U}(#1)}
\renewcommand{\labty}[3]{#1^{#2}_{#3}}
\newcommand{\memenv}{\mathcal{M}}
\newcommand{\tensor}{\multimap}
\newcommand{\lib}{\mathcal{L}}
\newcommand{\okt}{\mathit{OK}}
\newcommand{\vty}{t}
\newcommand{\disty}{\dot{\vty}}
\newcommand{\tlev}[1]{\mathcal{T}(#1)}
\newcommand{\otp}{\mathrm{sum}}
\newcommand{\macotp}{\hat{\otp}}

\long\def\cnote#1{{\small\textbf{\textit{\color{violet}(*#1 -- Chris*)}}}}
\long\def\jnote#1{{\small\textbf{\textit{\color{brown}(*#1 -- Joe*)}}}}



\begin{document}

\title{Automating Verification of MPC Security: $\metaprot$ to $\minifed$}

\author{Author names withheld for double-blind reviewing}

\begin{abstract}
Secure Multi-Party Computation (MPC) protocols support data privacy in
important modern distributed applications. Security for MPC protocols
is superficially similar to probabilistic noninterference, but differs
in a subtle but fundamental way, and approaches for verifying
noninterference cannot naturally extend to MPC security.  Currently,
proof methods for MPC protocols are well-studied but manual
and thus tedious and error-prone, and are also non-standardized and
unfamiliar to most PL theorists.  Our goal is to strengthen
connections between the security model of MPC and trace-based
hyperproperties, and to leverage them to obtain automated
enforcement mechanisms for MPC protocol development.  We define a
novel hyperproperty that is provably sound for MPC security, as a
foundation for fully and partially automated verification of passive
MPC security in protocols including arbitrarily large YGC and GMW
circuits.
\end{abstract}

\maketitle

\section{Introduction}

Secure Multi-Party Computation (MPC) protocols support data privacy in
important modern, distributed applications such as privacy-preserving
machine learning and Zero-Knowledge proofs in blockchains \cite{XXX}
\cnote{Need some banger cites here, maybe more MPC hype.}. MPC methods
have been developed by the cryptography community for many years,
while receiving the attention of the programming languages community
only relatively recently. In contrast, information flow security has
received significant attention in PL theory and practice especially
since the turn of the century \cite{1159651}, including a menagerie of
variants, enforcement mechanisms, and programming frameworks. Much of
this has been enabled by the unified metatheory of
\emph{hyperproperties} \cite{10.5555/1891823.1891830} that establishes
a common conceptual framework for reasoning about and implementing
systems with information flow security.

Our goal is to explore and establish connections between the security
model of MPC-- the \emph{real/ideal} aka \emph{simulator security}
model -- and trace-based hyperproperties, and to leverage these
connections to obtain automated enforcement mechanisms for MPC
protocol development. Currently, proof methods for MPC protocol
development are well-studied \cite{Lindell2017} but manual and 
thus tedious and error-prone, and are also non-standardized and
unfamiliar to most PL theorists. Therefore our exploration will
make both theoretical and practical contributions by bridging a gap between
information flow and simulator security methodologies.

\subsection{The Analysis Challenge of MPC} MPC protocols involve communication
between a group of distributed participants called a \emph{federation}
that collaboratively compute and publish the result of some known
\emph{ideal functionality} $\idealf$, while keeping each party's input
``secret'', without the use of a trusted third party. This last part
is critical. For example, if we take $\idealf$ to be the majority vote
function, a protocol for computing $\idealf$ is MPC-secure if, given
any set of input votes, it correctly computes and publishes the voting
result but reveals no other information to the public or to other
participants. However, by publishing the result, some information
about individual votes may be implicitly declassified.  For example,
in the case of a majority vote in a federation of size 3, if the
motion carries and party 1 has voted no, then party 1 knows exactly
the votes of parties 2 and 3. This cannot be avoided due to the nature
of $\idealf$.

Security in the MPC setting thus means that protocols cannot reveal
any secret information other than what is impicitly declassified by
the publicized output of the ideal functionality. The security model
also assumes that some subset of participants can be corrupted and
collude adversarially to possibly infer more secret information. The
accepted method of demonstrating protocol security in this setting is
to define a \emph{simulation} algorithm that runs in the ``ideal''
world which, given just the inputs of corrupted parties and the output
of the ideal functionality, is able to reconstruct information that
corrupted parties receive in their so-called \emph{views} of the
protocol running in the real world.  This implies that adversarial
views provide no information beyond what is provided by the ideal
output alone. Simulation is defined probabilistically since MPC
protocols typically rely on cryptographic and probabilistic methods.

In Section \ref{section-hyperprop-passive} we formalize real/ideal
security, and we define and discuss simple examples of MPC protocols
in Section \ref{section-minicat-examples} and more complex ones in
Section \ref{section-composition}. But an immediate and main point in
relation to security hyperproperties is that, due to the potential for
information release in MPC, simulator security is \emph{not} a strict
probabilistic noninterference or trace obliviousness property, as we
show in Section \ref{section-hyperprop-ni}- rather, the public output
allows and sets an upper bound on declassification.

\subsection{Related Work}

We distinguish between \emph{extensional} vs.~\emph{intensional}
properties and analysis of MPC protocols. By extensional, we mean
simulator security itself as well as analysis concerned with the use
and interaction of complete secure protocols. By intensional, we mean
analysis and inner workings of protocols themselves. While the
probabilistic formulations we develop could conceivably used
extensionally, our main focus is on intensional properties of
protocols, both through whole-program analysis and compositional
properties of program components that support security.

Previous work on analysis for the SecreC language
\cite{almeida2018enforcing,10.1145/2637113.2637119} is concerned with
extensional properties of MPC, in particular the specification and
enforcement of declassification bounds in programs that use MPC in
subprograms. This work is explicitly reminiscent of information flow
approaches such as delimited information release
\cite{10.1007/978-3-540-37621-7_9} and downgrading policies and
relaxed noninterference \cite{10.1145/1040305.1040319}. However, their
program logic assumes correctness of the underlying MPC protocols.
The Wys$^\star$ language \cite{wysstar}, based on Wysteria
\cite{rastogi2014wysteria}, has similar goals and includes a
trace-based semantics for reasoning about the extensional interactions
of MPC protocols. Their compiler also enforces some intensional
properties, in particular that underlying multi-threaded protocols
enforce the single-threaded source language semantics.



\subsection{Contributions}

In this paper, we make the following contributions.
\begin{itemize}
\item A new probabilistic programming language $\minifed$ for defining
  synchronous distributed protocols over the binary field. We consider example
  protocols including oblivious transfer, communications encryption with one-time
  pads, and additive secret sharing (Section \ref{section-minicat}).
  \cnote{I'm hoping $\minifed$ hasn't been used already.}
\item A formulation of $\minifed$ program distributions, including
  distributions of secrets and views, supporting expression of security
  properties (Section \ref{section-pmf}). This intensional characterization
  is useful for expressing internal characteristics of protocols, and
  is needed for MPC due to the fact that obliviousness alone is not sufficient
  to capture characteristics of the real/ideal model, in particular, the
  allowance of some information leakage through public outputs. 
\item A novel hyperproperty of program execution traces, called
  \emph{noninterference modulo output ($\NIMO$)}, based on our
  formulation of program distributions, that implies passive security
  (Section \ref{section-hyperprop}).
\item A brute force mechanism for verifying passive security in
  $\minifed$ protocols, that is also amenable to HPC optimizations via
  translation of $\minifed$ protocols into Datalog (Section \ref{section-bruteforce}). 
\item A new metaprogramming language $\metaprot$ that dynamically
  generates $\minifed$ protocols. It includes control and data structures
  and is able to express logical protocol components, and enjoys a
  type safety result that guarantees that generated protocols
  are semantically well-defined (Section \ref{section-metalang}).
\item A formulation of Yao's Garbled Circuits, and a formulation of a
  compositional property of an extensible library of garbled gate
  components that is verifiable automatically. This property
  guarantees that any well-formed circuits using the library are
  passive secure (Section \ref{section-composition}).
\end{itemize}



\section{The $\minifed$ Language}
\label{section-minicat}

The $\minifed$ language provides a simple model of synchronous
protocols between a federation of \emph{clients} exchanging values in
the binary field. We will identify clients by natural numbers, and
federations- finite sets of clients- are always given statically.
As we will see, our threat model assumes a partition of the federation
into \emph{honest} $H$ and \emph{corrupt} $C$ subsets.

We model probabilistic programming via the \emph{random tape}
approach. That is, we will assume that programs can make reference to
values chosen from a uniform random distribution- coin ``flips''- via
values arbitrarily assigned in the initial program memory.  Programs
execute deterministically given the random tape. The random tape
formulation supports our automated analysis as we discuss in
section 

\subsection{Syntax} The syntax of $\minifed$, defined in
Figure \ref{fig-minifed-syntax}, includes a standard boolean algebra
with $\eand$, $\eor$, $\exor$, and $\enot$ as primitives. In addition
programs have different kinds of variables, including \emph{secrets}
$\secret{\cid}{w}$, \emph{flips} $\flip{\cid}{w}$, and \emph{views}
$\view{\cid}{w}$.  Each of these variables are indexed by an ``owner''
client $\cid$ and distinguishing string $w$. So for example,
$\secret{\ttt{1}}{\ttt{foo}}$ is client $\ttt{1}$'s secret called
$\ttt{foo}$. All clients can make reference to the \emph{oracle}
through variables $\oracle{w}$- this is a form of shared randomness
that is standard in the MPC setting \cite{evans2018pragmatic}.  Each client can only
compute on their own variables in \emph{expressions} $\be$, and share
values with other clients by assignment to their views as in
$\eassign{\view{\cid}{w}}{\be}$.  A \emph{protocol} $\prog$ is a
possibly empty sequence of view assignments. We will generally omit
$\varnothing$ from example code, writing $\instr_1;\ldots;\instr_n$
instead of $\instr_1;\ldots;\instr_n;\varnothing$, and abusing notation we will
write $\prog_1;\prog_2$ to denote the concatenation of $\prog_1$
and $\prog_2$.



\begin{fpfig}[t]{$\minifed$ source code syntax.}{fig-minifed-syntax}
$$
\begin{array}{rcl@{\hspace{8mm}}r}
b &\in& \{ \etrue, \efalse \} \\
w &\in& \mathrm{String} \\ 
\cid &\in& \mathrm{Clients} \subset  \mathbb{N} \\[2mm]%\qquad  (\mathrm{Clients} \subset \mathbb{N} \text{\ in\ this\ presentation})\\[2mm]
\bop &\in& \{ \eand, \eor, \exor \} \\[2mm]
\be &::=& b \mid \flip{\cid}{w} \mid \secret{\cid}{w} \mid \view{\cid}{w} \mid \oracle{w} \mid & \textit{boolean expressions}\\
& &  \enot\ \be \mid \be\ \bop\ \be \mid \select{\be}{\be}{\be} \mid \OT{\be}{\be}{\be} \\[2mm]
\instr &::=& \eassign{\view{\cid}{w}}{\be} & \textit{view assignments} \\[2mm]
\prog &::=& \varnothing \mid \instr; \prog & \textit{protocols}
\end{array}
$$ 
\end{fpfig}

We let $X$ range over sets of variables and $S,V,F$ to range over sets
of secrets, views, and flips (including invocations of the oracle)
respectively. Given a program $\prog$, we write $\iov(\prog)$ to
denote $S \cup V$ where $S$ is the set of secret variables in $\prog$
and $V$ is the set of views in $\prog$, we write $\flips(\prog)$
to denote the set $F$ of flip variables in $\prog$, and we
write $\vars(\prog)$ to denote $\iov(\prog) \cup \flips(\prog)$. For any set of
variables $X$ and parties $P$, we write $X_P$ to denote the subset of
$X$ owned by any party in $P$, and we write $X_H$ and $X_C$ to denote
the subsets belonging to honest and corrupt parties, respectively.

Expressions forms include the convenience
$\select{\be_1}{\be_2}{\be_3}$ which is essentially a conditional
expression with:
$$
\select{\be_1}{\be_2}{\be_3} \equiv (\be_1\ \eand\ \be_2)\ \eor\ (\enot\ \be_1\ \eand\ \be_3)
$$
We also include \emph{oblivious transfer} $\OT{\be_1}{\be_2}{\be_3}$ as a primitive,
with semantics similar to $\ttt{select}$ but with important nuances
related to communication between clients as we discuss more below.
As we will demonstrate, it is not necessary to include $\ttt{OT}$
as a primitive since we can implement it in a provably secure
fashion. However its inclusion simplifies our presentation and is a useful
convenience.

\subsection{Semantics}

\emph{Memories} are fundamental to the semantics of $\fedcat$ and
provide the random tape and secret inputs to protocols, and record
view assignments. Memories $\store$ are finite (partial) mapping from
variables to binary values $\beta \in \{0,1\}$. The \emph{domain} of a
memory is written $\dom(\store)$ and is the finite set of variables on
which the memory is defined. We write $\store\{ x \mapsto \beta\}$ for
$x\not\in\dom(\store)$ to denote the memory $\store'$ such that
$\store'(x) = \beta$ and otherwise $\store'(y) = \store(y)$ for all $y
\in \dom(\store)$. We write $\store_1 \subseteq \store_2$ iff
$\dom(\store_1) \subseteq \dom(\store_2)$ and $\store_1(x) =
\store_2(x)$ for all $x \in \dom(\store_1)$. We write $\store_1 \cap
\store_2$ to denote the combination of $\store_1$ and $\store_2$
assuming $\store_1(x) = \store_2(x)$ for all $x \in \dom(\store_1)
\cap \dom(\store_2)$, otherwise $\store_1 \cap \store_2$ is undefined.

Given a set of variables $X$, we write $\store_X$ to denoted the
memory $\store$ restricted to the domain $X$, and and we define
$\mems(X)$ as the set of all memories with domain $X$:
$$
\mems(X) \defeq \{ \store \mid \dom(\store) = X \}
$$
Thus, given a protocol $\prog$, the set of all random tapes for
$\prog$ is $\mems(\flips(\prog))$. We let $\stores$ range
over sets of memories with the same domain, and abusing notation
we write $\dom(\stores)$ to denote the common domain,
and $\stores_X \defeq \{ \store_X | \store \in \stores \}$.

Given a variable-free boolean expression $\be$, we write $\cod{\be}$
to denote the standard interpretation of $\be$ in the binary field.
With the introduction of variables to expressions, we have two
concerns. First, we need to interpret variables with respect to a
specific memory, and second, we need to ensure that variables are used
``legally''. That is, since expressions define local computation, all
variables used in an expression must belong to the same client.  Thus,
we denote interpretation of expressions $\be$ possibly containing
variables as $\lcod{\store,\be}{\cid}$, where $\store$ associates
variables with values and all variables must be owned by client
$\cid$. This is defined in Figure \ref{fig-minifed-interp}.

\begin{fpfig}[t]{$\minifed$ expression interpretation.}{fig-minifed-interp}
\begin{eqnarray*}
\lcod{\store, \etrue}{\cid} &=& 1\\
\lcod{\store, \efalse}{\cid} &=& 0\\
\lcod{\store, \flip{\cid}{w}}{\cid} &=& \store(\flip{\cid}{w})\\
\lcod{\store, \secret{\cid}{w}}{\cid} &=& \store(\secret{\cid}{w})\\
\lcod{\store, \view{\cid}{w}}{\cid} &=& \store(\view{\cid}{w})\\
\lcod{\store, \oracle{w}}{\cid} &=& \store(\oracle{w})\\
\lcod{\store, \enot\ \be}{\cid} &=& \cod{\enot\ \lcod{\store,\be}{\cid}}\\
\lcod{\store, \be_1\ \mathit{binop}\ \be_2}{\cid} &=&
    \cod{\lcod{\store,\be_1}{\cid}\ \mathit{binop}\ \lcod{\store,\be_2}{\cid}}\\
\lcod{\store, \select{\be_1}{\be_2}{\be_3}}{\cid} &=&
             \begin{cases}
                \lcod{\store,\be_2}{\cid} & \text{if\ } \lcod{\store,\be_1}{\cid}\\
                \lcod{\store,\be_3}{\cid} & \text{if\ } \neg\lcod{\store,\be_1}{\cid}
             \end{cases}%\\
%\lcod{\store, \OT{\be_1}{\be_2}{\be_3}}{\cid} &=&
%             \begin{cases}
%                \lcod{\store,\be_2}{\cid'} & \text{if\ } \lcod{\store,\be_1}{\cid}\\
%                \lcod{\store,\be_3}{\cid'} & \text{if\ } \neg\lcod{\store,\be_1}{\cid}
%             \end{cases}
\end{eqnarray*}
\end{fpfig}

Evaluation of configurations is then defined via a small-step reduction relation $\redx$.
This is defined in the obvious manner for view assignments other than through
$\ttt{OT}$- however note that reduction requires that views are never reassigned. 
\begin{mathpar}
  (\store, \eassign{\view{\cid}{w}}{\be};\prog) \redx (\extend{\store}{\view{\cid}{w}}{\lcod{\store,\be}{\cid'}}, \prog)
\end{mathpar}
For view assignments through $\ttt{OT}$, we define a different reduction rule that
captures the appropriate semantics- that is, the selection bit is \emph{not} communicated
to the sender (through a view assignment), and only the selected bit is sent to the receiver:
\begin{mathpar}
  \inferrule
  {\beta = \text{if\ } \lcod{\store,\be_1}{\cid}  \text{\ then\ } \lcod{\store,\be_2}{\cid'} \text{\ else\ } \lcod{\store,\be_3}{\cid'}}
      {(\store, \eassign{\view{\cid}{w}}{\OT{\be_1}{\be_2}{\be_3}};\prog) \redx (\extend{\store}{\view{\cid}{w}}{\beta}, \prog)}
\end{mathpar}
We define $\redxs$ as the reflexive, transitive closure of $\redx$.

Given $\prog$ with $\iov(\prog) = S \cup V$ and $\flips(\prog) = F$,
any execution of $\prog$ is assumed to be an evaluation of a
configuration $\config{\store}{\prog}$ where $\store \in \mems(S \cup
F)$- that is, with secrets and the random tape as inputs- and where
$\config{\store}{\prog} \redxs \config{\store'}{\varnothing}$ it is
the case that $\store' \in \mems(S \cup F \cup V)$. We define a
program as \emph{safe} iff for all $\store \in \mems(S \cup F)$ there
exists $\store'$ where $\config{\store}{\prog} \redxs
\config{\store'}{\varnothing}$. Unsafe programs are those where views
are used before they're defined, or where expressions mix owners of
variables. As a sanity condition we will only consider safe 
programs in our presentation of $\fedprot$, though in Section
\ref{section-metalang} we define a type system that guarantees
program safety by construction, as demonstrated in Theorem
\ref{theorem-metalang-safety}.

\subsection{Examples}
\label{section-minicat-examples}

Here we introduce some examples to illustrate $\minifed$ and how it
can be used to model secure communication protocols for values in the
binary field. Most of these are translations (into $\minifed$) of
object examples in previous related work
\cite{barthe2019probabilistic,darais2019language}.  We will return to
these as running examples throughout the paper to illustrate various
concepts and advantages of our methods. In Section
\ref{section-metalang} we will formulate a larger example
implementation of Yao's Garbled Circuits (Example \ref{example-ygc}).
\begin{example}[One-Time Pad]
  \label{example-otp}
As observed in \cite{barthe2019probabilistic} and elsewhere, coin flips with $\exor$ can
be used to model one-time pad encryption in the binary field. On the
basis of this we can model secure communication between over an
insecure channel using symmetric key encryption. In the following
program, we assume $H = \{ 1,2 \}$ and $C = \{ 0 \}$. Client 1 first
shares the randomly generated ``key'' $\fx{1}{0}$ with client 2
through view $\view{2}{0}$. Then, client 1 encrypts its secret
$\sx{1}{0}$ and sends it to party 0, which relays the ciphertext in
$\vx{0}{0}$ to client 2. Finally, client 2 decrypts the the ciphertext
and stores the plaintext in $\vx{2}{2}$.
\begin{verbatimtab}
v[2,0] := f[1,0];
v[0,0] := f[1,0] xor s[1,0];
v[2,1] := v[0,0];
v[2,2] := v[2,1] xor v[2,0]
\end{verbatimtab}
Intuitively, this protocol is \emph{correct} because $\vx{2}{2}$ is
exactly correlated with the secret $\sx{1}{0}$, and it is
\emph{secure} because the corrupt view $\vx{0}{0}$ in isolation is
probabilistically independent of the secret $\sx{1}{0}$.
\end{example}

\begin{example}[$\lambda_{\mathrm{obliv}}$]
  \label{example-lambda-obliv}
  In \cite{darais2019language}, Figure 3, two examples are introduced that illustrate
  subtleties of probabilistic programming in
  $\lambda_{\mathrm{obliv}}$, a language for oblivious memory
  management. We consider their translations here where we
  assume $H = \{ 1 \}$ and $C = \{ 0 \}$. First, the following
  example shows that the reuse of coin flips can result in release of
  information, even though flips are chosen from a uniform
  distribution. This is because the adversary can deduce that
  $\secret{1}{s}$ is more likely to be 1 in case their views are the
  same value. We call this subexample $(a)$:
  \begin{verbatimtab}
    v[0,0] := select(s[1,s], flip[1,sx], flip[1,sy]);
    v[0,1] := flip[1,sx] \end{verbatimtab}
  In the next example, since the nested select is more likely to evaluate
  to $1$ than $0$, the adversary can deduce that $\secret{1}{s}$ is
  more likely to be 1 in case $\view{0}{0}$ is 1. We call this subexample $(b)$:
  \begin{verbatimtab}
  v[0,0] := select(s[1,s],select(flip[1,sx], flip[1,sx], flip[1,sy]),flip[1,z]) \end{verbatimtab}
  We note that these examples are considered ``bad'' in \cite{darais2019language} since there
  no information about honest secrets should be revealed to the adversary in
  their threat model. For this reason they disallow reuse of flips through a
  linear type discipline, which is a culprit in both of these examples. However,
  as we have shown in Example \ref{example-otp} the reuse of flips is useful in
  the modeling of symmetric key encryption, and generally we will not enforce
  the same linear type discipline on the use of flips.
\end{example}

\begin{example}[Oblivious Transfer]
  \label{example-OT}
  In \cite{barthe2019probabilistic} a protocol for $\ttt{OT}$ is defined that we recreate as
  follows, where client 2 is the receiver and client 1 is the sender. Whereas
  in \cite{barthe2019probabilistic} a trusted third party provides the sender and receiver with
  shared randomness, in our version both clients refer to the oracle $\ttt{H}$
  to obtain shared random values. 
\begin{verbatimtab}
  v[2, rd] := select(flip[2, d], H[r1], H[r0]);
  v[1, e] := s[2, c] xor flip[2, d];
  v[2, f0] := s[1, m0] xor select(v[1, e], H[r1], H[r0]);
  v[2, f1] := s[1, m1] xor select(not v[1, e], H[r1], H[r0]);
  v[2, mc] := select(s[2, c], v[2, f1], v[2, f0]) xor v[2, rd]
\end{verbatimtab}
Below we will discuss how to automatically verify security properties
of this protocol that correspond to the properties proven manually in
\cite{barthe2019probabilistic}, and additionally how to automatically verify its passive
security in the simulator model. With respect to our language primitives,
we note that this protocol is probabilistically equivalent to the
following:
\begin{verbatimtab}  
  v[2, mc] := OT(s[2, c], s[1, m1], s[1, m0])
\end{verbatimtab}
\end{example}

\begin{example}[Additive Secret Sharing]
    \label{example-he}
Additive secret sharing is an MPC protocol where $k$ parties split up
their secrets into \emph{shares}, with the property that all $k$
shares are required to reconstruct the secret and any fewer reveals
nothing. Additionally, this sharing enjoys additive homomorphism. In
the binary field it is well-known that we can extend our use
of $\ttt{xor}$ and flips to generate an arbitrary number of shares
with additive homomorphism. 

For example, to generate shares for three-party additive sharing,
clients 1,2, and 3 can break up and distribute their shares to
each other as follows:
\begin{verbatimtab}
  v[2,s1] := flip[1,1] xor flip[1,s1] xor s[1,1];
  v[3,s1] := flip[1,1];

  v[1,s2] := flip[2,1] xor flip[2,s2] xor s[2,1];
  v[3,s2] := flip[2,1];

  v[1,s3] := flip[3,1] xor flip[3,s3] xor s[3,1];
  v[2,s3] := flip[3,1];
\end{verbatimtab}
Then, assuming that party 0 is a ``public'' client, each party sums
its own shares, and then the sum of sum of shares is revealed
in the output view $\vx{0}{output}$. By additive homomorphism of
$\ttt{xor}$, this output is the sum of the 3 secrets.
\begin{verbatimtab}
  v[0,ss1] := v[1,s2] xor v[1,s3] xor flip[1,s1];
  v[0,ss2] := v[2,s1] xor v[2,s3] xor flip[2,s2];
  v[0,ss3] := v[3,s1] xor v[3,s2] xor flip[3,s3];

  v[0,output] := v[0,ss1] xor v[0,ss2] xor v[0,ss3]
\end{verbatimtab}
We note that, as for MPC protocols generally, the notion of
``secrecy'' here is subtle and not the same as for encryption as in
Example \ref{example-otp}, particularly when the threat model allows
protocol participants to be corrupt. For example, suppose that $1 \in C$, and
$\sx{1}{1}$ is $1$, and after running the protocol suppose that
$\vx{0}{output}$ is 0. Then the adversary knows that 2 and 3's secrets
are either 1 and 0, or 0 and 1, with 50/50 probability, and definitely
neither 0 and 0, nor 1 and 1. 
\end{example}


\section{A Protocol Metalanguage}
\label{section-metalang}

\begin{fpfig}[t]{Syntax of $\metaprot$.}{fig-metaprot-syntax}
$$
\begin{array}{rcl@{\hspace{8mm}}r}
\flab &\in& \mathrm{Field}\\
x &\in& \mathrm{EVar}\\
f &\in& \mathrm{FName}\\[2mm]
e &::=& b \mid \flip{e}{e} \mid \secret{e}{e} \mid \view{e}{e} \mid \oracle{e} \mid \enot\ e \mid e\ \eand\ e \mid e\ \exor\ e \mid & \textit{expressions}\\[0mm]
& & \select{e}{e}{e} \mid 
\send{\view{e}{e}}{e} \mid \send{\view{e}{e}}{\OT{e}{e}{e}} \mid e;e \mid \\[0mm]
& & x \mid \elet{x}{e}{e} \mid f(e,\ldots,e) \mid \{ \flab = e; \ldots; \flab = e \}
\mid e.\flab \mid e\concat e \mid (e) \\[2mm]
v &::=& w \mid \cid \mid \be \mid \{ \flab = v;\ldots;\flab = v \} 
\mid \ttt{()} & \textit{values}\\[2mm]
{fn} &::=& f(x,\ldots,x) \{ e \} & \textit{functions}
\end{array}
$$
\end{fpfig}

Large practical MPC computations are based on much larger protocols
than the examples we've considered so far. These larger protocols are
typically based on compositional units. An example of this is Yao's
Garbled Circuits (YGC), which are composed of so-called garbled gates.
Languages for defined garbled circuits, beginning with Fairplay \cite{269581},
treat gates as compositional units that are wired together by the programmer
to generate a complete circuit. The $\fedprot$ language is low-level
and does not include abstractions for defining composable elements. 

In this Section we introduce the $\metaprot$ language which includes
structured data and function definitions, which are sufficiently
expressive to define composable protocol elements such as garbled
gates. The $\metaprot$ language is a \emph{metalanguage}, in the sense
that it produces $\fedprot$ protocols as a result of computation. That
is, $\metaprot$ is a high-level language that generates low-level
protocol code.

\subsection{Syntax}

The syntax of $\metaprot$ is defined in Figure
\ref{fig-metaprot-syntax}.  It includes a syntax of function
definitions and records, and values include client ids, identifier
strings, and boolean expressions. Expression forms allow dynamic
construction of boolean expression forms and view assignments. When
$\metaprot$ programs construct a $\fedprot$ assignment, a side effect
occurs whereby the assignment is added to the end of the $\fedprot$
program accumulated during evaluation.

Formally, we consider a complete metaprogram to include both a
codebase and a ``main'' program that uses the codebase. 
\begin{definition}
A \emph{codebase} $\codebase$ is a list of function 
declarations. We write $ \codebase(f) = x_1,\ldots,x_n,\ e$
iff $f(x_1,\ldots,x_n) \{ e \} \in \codebase$.
A \emph{metaprogram}, aka \emph{mataprotocol} is a pair of a 
codebase and expression $\codebase, e$. We may omit
$\codebase$ if it is clear from context.  
\end{definition}

When we consider the example of YGC in detail below, our focus will be
on developing a codebase that can be used to define arbitrary
circuits, i.e., complete and concrete protocols. Since strings and
identifiers can be constructed manually, and expressions can occur
inside assignments and boolean expression forms, function definitions
can generalize over $\fedprot$-level patterns to obtain composable
program units. As a simple example, consider 3 party secret
sharing as illustrated in Example \ref{example-he}. We can
define a function $\ttt{share3}$ that abstracts the process
of splitting a given client's secret into 3 separate shares.
\begin{example} \label{example-share3} The function $\ttt{share3}$ 
  splits a client's secret into 3 shares returned as a record
  with fields $\ttt{s1-3}$:
  \begin{verbatimtab}
    share3(client, secretid)
    {
      let s1 = flip[client, share1] in
      let s2 = flip[client, share2] in
      let s3 = (s1 xor s2) xor s[client, s: || secretid] in
      {s1 = s1;s2 = s2;s3 = s3}
    } \end{verbatimtab}
  Here is $\metaprot$ program that uses this this function definition:
  \begin{verbatimtab}
    let shares = share3(1, mysecret) in
    v[2,s1] := shares.s2;
    v[3,s1] := shares.s3 \end{verbatimtab}
  which generates the following $\minifed$ program, as we formalize in Example \ref{example-share3-eval}
  below:
  \begin{verbatimtab}
    v[2,s1] := flip[1, share2];
    v[3,s1] := flip[1, share1] xor flip[1, share2] xor s[1, s:mysecret] \end{verbatimtab}
\end{example}

\subsection{Semantics}

\begin{fpfig}[t]{Evaluation contexts and operational semantics of $\metaprot$.}{fig-metaprot-semantics}
$$
\begin{array}{rcl@{\hspace{3mm}}r}
E &::=& [\,] \mid \enot\ E \mid E\ \bop\ e \mid v\ \bop\ E \mid  \flip{E}{e} \mid \secret{E}{e} \mid \view{E}{e} \mid \oracle{E} \mid  \\[1mm]
& & \flip{\cid}{E} \mid \secret{\cid}{E} \mid \view{\cid}{E} \mid \send{E}{e} \mid \send{\view{\cid}{w}}{E} \mid \OT{E}{e}{e} \\[1mm]
& & \mid \OT{v}{E}{e} \mid \OT{v}{v}{E} \mid \select{E}{e}{e} \mid \select{v}{E}{e} \mid \\[1mm]
& & \select{v}{v}{E} \mid \elet{x}{E}{e} \mid f(v,\ldots,v,E,e,\ldots,e) \mid \\[1mm]
& & \{ \flab = v;\ldots;\flab = v;\flab = E;\flab = e;\ldots;\flab = e \} \mid E.\flab \mid E\concat e \mid v \concat E
\end{array}
$$
\medskip
$$
\begin{array}{rcl@{\hspace{10mm}}r}
\config{\prog}{\elet{x}{v}{e}} &\redx& \config{\prog}{e[v/x]}\\
\config{\prog}{f(v_1,...,v_n)} &\redx&
\config{\prog}{e[v_1/x_1,\ldots,v_n/x_n]} & 
 \codebase(f) = x_1,\ldots,x_n,\ e\\
\config{\prog}{\{\ldots; \flab = v; \ldots\}.\flab} &\redx&
 \config{\prog}{v}\\
 \config{\prog}{w_1\concat w_2} &\redx& \config{\prog}{w_1w_2}\\
 \config{\prog}{v;e} &\redx& \config{\prog}{e}\\
\config{\prog}{\instr} &\redx& \config{\prog;\instr}{()}\\
\config{\prog}{E[e]} &\redx& \config{\prog'}{E[e']} & \text{if}\ \config{\prog}{e} \redx \config{\prog'}{e'} 
\end{array}
$$
\end{fpfig}

We define a small-step evaluation aka reduction relation $\redx$ in
Figure \ref{fig-metaprot-semantics}.  We write $\redxs$ to denote the
reflexive, transitive closure of $\redx$. Reduction is defined on
\emph{configurations} which are pairs of the form $\config{\prog}{e}$,
where $\prog$ is the $\minifed$ program accumulated during evaluation.
In this definition we write $e[v/x]$ to denote the substitution of $v$
for free occurences of $x$ in $e$. The rules are mostly standard,
except why a concrete $\minifed$ assignment is encountered it is added
to the end of $\prog$.

The rules rely on a definition of \emph{evaluation contexts} $E$
allowing computation within a larger program context, where $E[e]$
denotes an expression with $e$ in the hole $[]$ of $E$. Evaluation
contexts include boolean expression forms, allowing generalization
and instantiation of compositional program elements.
\begin{example}
  \label{example-share3-eval}
  Let $\codebase,e_{\ref{example-share3}}$ be the $\metaprot$ program and let 
  $\prog_{\ref{example-share3}}$ be the  $\minifed$ program defined
  in Example \ref{example-share3}. We refer to the latter as ``accumulated''
  by evaluation of the former in the sense that $\config{\varnothing}{e_{\ref{example-share3}}}
  \redxs \config{\prog_{\ref{example-share3}}}{\ttt{()}}$.
\end{example}

\subsection{Type Theory and Static Type Safety}

It is desirable to statically enforce safety of both $\metaprot$
programs and the safety of the $\fedprot$ programs they
generate. Although safety of the latter could be enforced
post-generation by a direct analysis, for large programs this can be
much more expensive and it is also better to not wast time on
resource-intensive compilation of programs with known errors
\cite{kreuter2012billion}. Some consequences of safety errors, for example accidental
reuse of one-time pads, can also undermine security.

The type syntax of $\metaprot$ is defined in Figure
\ref{fig-metaprot-tsyntax}. It includes a weak form of dependency:
string types $\stringty{e}$ and client types $\cidty{e}$
are parameterized by expressions $e$ that precisely reflect the
type of the value. Boolean expression forms have the type
$\bet{e}$ indexed by expressions $e$ indicating the client
id type of the expression. The dependency is weak in the
sense that expressions in types are a strict subset of
expression forms- for any $\stringty{e}$ the expression $e$
is either a variable, a string, or a concatenation form,
and for any $\cidty{e}$ or $\bet{e}$ the expression $e$
is either a variable or a client id $\cid$. 

Type judgements for expressions are of the form
$\tjudge{\viewst_1}{\gamma}{e}{\viewst_2}$ where the \emph{view
effect} $\viewst_1$ denotes the views that have been defined so far,
and $\viewst_2$ records new views defined as the effect of the
expression on the residual $\minifed$ program.  Type judgements are
syntax-directed- selected rules are shown in Figure
\ref{fig-metaprot-tjudge}. The $\TirName{AssignT}$ rule
captures the effect of new view definitions. The
$\TirName{And}$ rule illustrates how program safety is enforced,
by ensuring that subexpressions of boolean expressions have the
same owner.

The $\TirName{FnT}$ and $\TirName{Appt}$ rules apply to function
definition and application respectively, and rely on the
definition of function input type annotations $\tas$ and
type term substitutions $\sigma$. 
\begin{definition}
  A \emph{function input type annotation} $\tas$ is a mapping from
  function names $f$ to type products $\tau_1 * \cdots * \tau_n$.
  A \emph{type term substitution} $\sigma$ is a mapping from
  $\minifed$ variables $x$ to values, where $\sigma(\tau)$ denotes
  the replacement of occurences of $x$ in $\tau$ with $\sigma(x)$. 
\end{definition}
We assume that input type annotations $\tas$ are provided by the
programmer for all function definitions. This guarantees that
$\metaprot$ type checking is straightforward and efficient.
Function types are of the form:
$$
\tau_1 * \cdots * \tau_n \rightarrow \tau,\viewst
$$
where $\viewst$ denotes the effect of the function on the residual
program.  The function type can be understood as a dependent $\Pi$
type, with every term variable bound. When applied, these variables
are instantiated with a type term substitution $\sigma$. In our
implementation, we adapt \emph{synthesis} as defined in Dependent ML \cite{10.1145/292540.292560}
to obtain this $\sigma$- essentially this is a match on the syntactic
structure of types and expressions. 
\begin{example}
  Given $\ttt{share3}$ as defined in Example \ref{example-share3} and
  annotation:
  $$\tas(\ttt{share3}) =  \ttt{cid(client)}\ *\ \ttt{string(sid)}$$
  the type of $\ttt{share3}$ is $\tas(\ttt{share3}) \rightarrow \tau,\varnothing$
  where $\tau$ is:
  $$
  \ttt{\{ s1 : bool[client]; s2 : bool[client]; s3 : bool[client] \}}
  $$
\end{example}

\begin{fpfig}[t]{Type Syntax of $\metaprot$.}{fig-metaprot-tsyntax}
$$
\begin{array}{rcl@{\hspace{2mm}}r}
\srct &::=& \cidty{e} \mid \stringty{e} \mid \bet{e} \mid  & \gdesc{types}\\ 
 &&  \{ \flab : \srct;\ldots;\flab : \srct \} \mid \tau * \cdots * \tau \rightarrow \tau,\viewst \\[1mm]
\viewst  &::=& \view{e}{e};\viewst \mid \varnothing   & \gdesc{view effects}\\[1mm]
\Gamma &::=& \Gamma; x : \tau \mid \varnothing & \gdesc{type environments}    
\end{array}
$$
\end{fpfig}

\begin{fpfig}[t]{Selected $\metaprot$ type judgement rules.}{fig-metaprot-tjudge}
\begin{mathpar}
\inferrule[\TirName{VarT}]
{}
{\tjudge{\viewst}{\Gamma}{x}{\Gamma(x)}{\viewst}}

\inferrule[\TirName{CidT}]
{}
{\tjudge{\viewst}{\Gamma}{\cid}{\cidty{\cid}}{\viewst}}

\inferrule[\TirName{StringT}]
{}
{\tjudge{\viewst}{\Gamma}{w}{\stringty{w}}{\viewst}}

\inferrule[\TirName{ConcatT}]
{\tjudge{\viewst}{\Gamma}{e_1}{\stringty{e_1'}}{\viewst_1}\\
\tjudge{\viewst_1}{\Gamma}{e_2}{\stringty{e_2'}}{\viewst_2}
}
{\tjudge{\viewst}{\Gamma}{e_1||e_2}{\stringty{e_1' ||e_2'}}{\viewst_2}}

\inferrule[\TirName{BoolT}]
{}
{\tjudge{\viewst}{\Gamma}{\etrue}{\bet{\cid}}{\viewst}}

\inferrule[\TirName{OracleT}]
{\tjudge{\viewst}{\Gamma}{e}{\stringty{e'}}{\viewst'}}
{\tjudge{\viewst}{\Gamma}{\oracle{e}}{\bet{\cid}}{\viewst'}}

\inferrule[\TirName{SecretT}]
{\tjudge{\viewst}{\Gamma}{e_1}{\cidty{e_1'}}{\viewst_1}\\
\tjudge{\viewst_1}{\Gamma}{e_2}{\stringty{e_2'}}{\viewst_2}}
{\tjudge{\viewst}{\Gamma}{\secret{e_1}{e_2}}{\bet{e_1'}}{\views_2}}

\inferrule[\TirName{AndT}]
{
\tjudge{\viewst}{\Gamma}{e_1}{\bet{e}}{\viewst_1}\\
\tjudge{\viewst_1}{\Gamma}{e_2}{\bet{e}}{\viewst_2}
}
{\tjudge{\viewst}{\Gamma}{e_1\ \eand\ e_2}{\bet{e}}{\viewst_2}}

\inferrule[\TirName{AssignT}]
{
\tjudge{\viewst}{\Gamma}{e_1}{\cidty{e_1'}}{\viewst_1}\\
\tjudge{\viewst_1}{\Gamma}{e_2}{\stringty{e_2'}}{\viewst_2}\\
\tjudge{\viewst_2}{\Gamma}{e_3}{\bet{e_3'}}{\viewst_3}
}
{
\tjudge{\viewst}{\Gamma}{\eassign{\view{e_1}{e_2}}{e_3}}{\unity}{(\viewst_3 ; \view{e_1'}{e_2'} )}
}

\inferrule[AppT]
{\Gamma(f) =  
 \tau_1 * \cdots * \tau_n \rightarrow \tau, \viewst_f \\ 
 \tjudge{\viewst}{\Gamma}{e_1}{\sigma(\tau_1)}{\viewst_1}
 \ \cdots\  
 \tjudge{\viewst_{n-1}}{\Gamma}{e_n}{\sigma(\tau_n)}{\viewst_n}}
{\tjudge{\viewst}{\Gamma}{f(e_1,\ldots,e_n)}{\sigma(\tau)}{(\viewst_n ; \sigma(\viewst_f))}}

\inferrule[FnT]
{
  \codebase(f) = x_1,\ldots,x_n,\ e \\ \tas(f) = \tau_1 * \cdots * \tau_n \\
  \tjudge{\varnothing}{\Gamma; x_1 : \tau_1; \ldots; x_n : \tau_n}{e}{\tau}{\viewst_f}
}
{ \Gamma \vdash f : \tau_1 * \cdots * \tau_n \rightarrow \tau, \viewst_f }

\inferrule[ProgT]
{
\forall f \in \dom(\codebase)\ .\ \Gamma \vdash f : \Gamma(f) \\ \Gamma \vdash e : \tau, \view{\cid_1}{w_1};\ldots;\view{\cid_1}{w_1}
}
{
\Gamma \vdash \codebase,e : \tau,\{\view{\cid_1}{w_1}\} \sqcup \cdots \sqcup \{ \view{\cid_n}{w_n}\}
}
\end{mathpar}
\end{fpfig}

Top-level type judgements are of the form $\Gamma \vdash \codebase, e
: \tau, V$, where all the functions in $\codebase$ are well-typed in
$\Gamma$, the top level view effect $V$ is a set of concrete
$\fedprot$ views which are constructed by disjoint union of the views
in the effect of $e$- the disjointness requirement guarantees that
views are uniquely defined. Our $\metaprot$ type safety result is
formulated as follows. In addition to safe execution of the
metaprogram, it also guarantees safety of the residual $\fedprot$
program. Of course, it is important to note that ``safety'' here
means the usual type safety property that programs do not get stuck,
it does not imply any security hyperproperties. 
\begin{theorem}[$\metaprot$ Type Safety]
  \label{theorem-metalang-safety}
  Given $\codebase$, $e$, and $\Gamma$ with $\Gamma \vdash \codebase,e : \unity : V$,
  then $\config{\varnothing}{e} \redxs \config{\prog}{\ttt{()}}$ where
  $\prog$ is safe with $\iov(\prog) = S \cup V$ for some $S$.
\end{theorem}


\section{Probability Mass Functions and Program Distributions}

\begin{definition}
  A \emph{memory probability distribution} $\pdf{X}$ is a function mapping
  memories in $\mems(X)$ to values in $[0..1]$, such that:
  $$
  \sum_{\store \in \mems(X)} \pdf{X}(\store) \  = \ 1
  $$
  We call $X$ the \emph{domain} of $\pdf{X}$.
\end{definition}

\begin{definition}
  Given $X$ the \emph{marginal distribution} of $Y \subseteq X$
  in a distribution $\pdf{X}$, denoted $\margd{\pdf{X}}{Y}$,
  is a distribution with domain $Y$ where for all
  $\store \in \mems(Y)$:
  $$
  (\margd{\pdf{X}}{Y})(\store) =
  \sum_{\store' \in \mems(X-Y)} \pdf{X}(\store \cup \store')
  $$
\end{definition}

\begin{definition}
  Given $\pdf{X}$, its \emph{conditional distribution given
  $\store$} with $\dom(\store) = Y$ and $Y \subseteq X$, denoted
  $\condd{\pdf{X}}{\store}$, is a distribution with domain $X - Y$ where for all
  $\store' \in \mems(X - Y)$:
  $$
  (\condd{\pdf{X}}{\store})(\store') =
      \pdf{X}(\store \cup \store') / ((\margd{\pdf{X}}{Y})(\store))
  $$
\end{definition}

Now we can define the probability distribution of a program $\prog$,
that we denote $\progd(\prog)$. Recall that $\fedcat$ is
deterministic, so a run is determined by the input secrets together
with random flip assignments. And since we constrain programs to not
overwrite memories, we can be sure that final memories resulting from
computation contain both a complete record of all initial secrets and
flips as well as views of communicated information. Thus a final
memory represents a run of the program, and since we consider and set
of \emph{input} secrets and flips to be equally likely, we consider
each memory in the set of final memories of a program to be equally
likely, which establishes the basic probability distribution for the
program. And since we don't care about flip values which are only
obversable insofar as they affect views, the probability distribution
we are really interested in is the marginal distribution of secrets
and views in this basic program distribution.
\begin{definition}
  Given progam $\prog$ with $\iov(\prog) = (S,V)$ and $\flips(\prog) = F$, define $\stores$ as:
  $$
  \stores \defeq \{ \store \mid (\dom(\store) = S \cup V \cup F) \wedge (\store_{S \cup F},\prog) \redxs (\store,\varnothing) \}
  $$
  Define also $\pdf{S \cup V \cup F}$ as the program's \emph{basic distribution} such that for all
  $\store \in \mems(S \cup V \cup F)$:
  $$
  \pdf{S \cup V \cup F}(\store) =
  \begin{cases}
    1 / |\stores| & \text{if}\ \store \in \stores\\
    0 & \text{otherwise}
  \end{cases}
  $$
  Then the \emph{program distribution of $\prog$}, denoted $\progd(\prog)$, is the
  marginal distribution of $S \cup V$ in $\prog$'s basic distribution:
  $$
  \progd(\prog) =  \margd{\pdf{S \cup V \cup F}}{S\cup V}
  $$
\end{definition}


\section{Hyperproperties of MPC Passive Security}
\label{section-hyperprop}

As we've discussed in our section on Related Work
(\ref{section-related}), previous authors have noted connections
between the traditional formulation of the MPC security model- aka the
simulator security model- and hyperproperty-style formulations of
program trace behavior \cite{XXX} including probabilistic
noninterference. This formulation of hyperproperties that are at least
sound with respect to simulator security bridges a conceptual gap
between foundational approaches to security, and also motivates and
supports automated analysis \cite{XXX}. 

In this section, we formulate a new hyperproperty that we show is sound
for passive security in the real/ideal model, called
\emph{noninterference modulo output ($\NIMO$)}. We do not claim
completeness with respect to passive security, but soundness
guarantees that enforcement of $\NIMO$ in our analyses implies passive
security. In Section \ref{section-bruteforce} we show how $\NIMO$ can
be automatically enforced in a brute-force manner, and in Section
\ref{section-composition} we show how $\NIMO$ can be refined to obtain
compositional properties supporting efficient, scalable enforcement of
complex protocols that incorporate modular subprotocols, with a
specific use-case of Yao's Garbled Circuits (YGC) as an example where
garbled gates are the compositional units.  The latter methodology
demonstrates a particular advantage of our hyperproperty formulation.

\subsection{Passive Security and the Real/Ideal Model} \dnote{would shortening the background info on how simulator model works be a bad idea?} Here we
briefly summarize the formal definition of MPC security. For a
more detailed discussion the reader is referred to \cite{XXX}.
In this paper, we will consider the \emph{passive} aka
semi-honest aka honest-but-curious variant. In this threat
model, protocol participants can be corrupted, but all clients
follow the rules of the protocol- as opposed to the
\emph{active} aka malicious variant where participants
are allowed to break these rules. In the passive model, the
vulnerability is through information leaked to the adversary
through the views of corrupted participants. While the passive
model also assumes that any clients can potentially be
corrupted, it assumes that no more than half will be-
i.e., it assumes an \emph{honest majority}. 

The real/ideal model posits a \emph{simulator} that exists in an ideal
world, whose goal is to reconstruct observable information in a real
world run of a protocol $\prog$ that correctly implements an ideal
functionality $\idealf$- that is, execution of a configuration
$\config{\store}{\prog}$. We will assume that any protocol $\prog$
includes a public \emph{output view} $\outv$ where the result is
published, where $0$ is the ``public'' client and thus $\outv$
is of the form $\view{0}{w}$ for some $w$, i.e., is owned by 0.
Note that the protocol is \emph{correct} iff the resulting
memory contains $\outv \mapsto \idealf(\store_S)$, where $S$ are the
secret inputs (honest and corrupt) in $\prog$.

The simulator is given just the corrupt inputs to the run- that is,
$\store_{S_C}$- and the output of the protocol- that is,
$\idealf(\store)$- and aims to reconstruct the distribution of
adversarial views observed in the real world. The intuition here is
that the simulator is given the information that is necessarily
available to the adversary by observing a run of the protocol, and if
the simulator can reconstruct whatever information appears in the real
world adversarial views, by reconstructing their distributions, then
those views leak no additional information, i.e., they have no
additional dependencies on honest input secrets than what is revealed
by the output and conditionings on corrupt inputs.

The simulator is represented by a probabilistic algorithm $\SIM_C$,
aka a \emph{simulation}, that is parameterized by a corrupt inputs and
the output of an ideal functionality, and that returns a set of
adversarial views (as a memory) with some probability. Given
corrupt inputs $\store$ and ideal functionality output $v$,  
we write:
$$
\prob(\SIM_C(\store,v) = \store')
$$
to denote the probability that $\SIM_C(\store,v)$
returns corrupt views $\store'$ as a result. We can then define the
probability distribution of corrupt views reconstructed
by the simulator as follows:
\begin{definition}
  Given $C$, $\store$, and $v$, we write $\progd(\SIM_C(\store,v))$ to
  denote the distribution of corrupt views reconstructed by the
  simulation, which is a distribution $\pdf{V}$ where for
  all $\store' \in \mems(V)$:
  $$
  \pdf{V}(\store')\ \defeq\ \prob(\SIM_C(\store,v) = \store') 
  $$
\end{definition}

Then we can define passive security in the real/ideal
model as follows. 
\begin{definition}[Passive Security]
  Assume given a program $\prog$ that correctly implements an ideal
  functionality $\idealf$, with $\iov(\prog) = (S,V)$.  Then $\prog$
  is \emph{passive secure in the simulator model} iff for all
  partitions of the federation into honest and corrupt sets $H$ and $C$
  with $|C| < |H|$ and for all $\store \in \mems(S)$ there exists a
  simulation $\SIM_C$ such that:
  $$
  \progd(\SIM_C(\store_{S_C},\idealf(\store))) = \margd{(\condd{\progd(\prog)}{\store})}{V_C}
  $$
\end{definition}

\subsection{Probabilistic Noninterference}
\label{section-ni}
Previous work has noted
relations between probabilistic noninterference and security
properties such as memory trace obliviousness \cite{XXX} and passive
security \cite{XXX}. And above (Section \ref{section-pmf-examples}) we
have demonstrated that our basic encryption scheme enjoys a noninterference
property. To formulate probabilistic noninterference in
our setting we begin by defining low equivalence of memories.
\begin{definition}[Memory Low Equivalence]
  Given $\store^1$ and $\store^2$ with $\dom(\store^1) = \dom(\store^2) = X$,
  we write $\store^1 =_C \store^2$ iff $\store^1_{X_C} = \store^2_{X_C}$.
\end{definition}
Now, we can define probabilistic noninterference as a property of
programs ensuring that low-equivalent initial memories produce
the same low-observable trace (i.e., corrupt views) \emph{with the
same probability}. 
\begin{definition}[Probabilistic Noninterference]
  Given a program $\prog$ with $\iov(\prog) = (S,V)$, we say that
  $\prog$ satisfies \emph{probabilistic noninterference} iff for all
  $\store_1, \store_2 \in \mems(S)$:
  $$\store_1 =_C \store_2 \implies
    \margd{(\condd{\progd(\prog)}{\store_1})}{V_C} =
    \margd{(\condd{\progd(\prog)}{\store_2})}{V_C}$$
\end{definition}

Despite the intuitive resonance between noninterference
and passive security, since MPC
protocols implement an ideal functionality, they cannot generally satisfy
noninterference. We can easily demonstrate this by using
Example \ref{example-he} as a failure witness, where we'll
assume $C = \{0,2\}$ and $H = \{1,3\}$:
$$
(\progd(\prog_{\ref{example-he}}))({\outv \mapsto 1}|\{\{\sx{1}{1} \mapsto 0, \sx{2}{1} \mapsto 1, \sx{3}{1} \mapsto 0 \} \})
= 1
$$
whereas:
$$
(\progd(\prog_{\ref{example-he}}))({\outv \mapsto 1}|\{\{\sx{1}{1} \mapsto 1, \sx{2}{1} \mapsto 1, \sx{3}{1} \mapsto 0 \} \})
= 0
$$

Probabilistic noninterference can also be restated to formally capture a
basic intuition that noninterfering programs do not provide
the adversary with any clues about honest secrets through corrupt
views, as in the following Lemma. This is similar to asserting that
adversarial knowledge is not increased by the execution of a program as in formulations
of so-called gradual release \cite{XXX}. 
\begin{lemma}
  Given a program $\prog$ with $\iov(\prog) = (S,V)$, 
  $\prog$ satisfies probabilistic noninterference iff for all
  $\store \in \mems(S \cup V)$:
  $$\margd{(\condd{\progd(\prog)}{\store_{S_C}})}{S_H} =
    \margd{(\condd{\progd(\prog)}{\store_{(S\cup V)_C}})}{S_H} $$
\end{lemma}

\subsection{A Sound Hyperproperty for Passive Security}
\label{section-nimo}
The intuition that passive security means that the protocol releases
no more information than what is released by conditioning on the
public output and corrupt inputs suggests a
refinement of noninterference. That is, the set of low equivalent
initial memories \emph{that can produce the same output} should generate the
same distributions of adversarial views.
\begin{definition}[Noninterference Modulo Output]
  We say that a program  $\prog$ satisfies \emph{noninterference modulo output},
  written $\NIMO(\prog)$, iff for all $H$ and $C$ with $|C|\le|H|$ and 
  $\store_1,\store_2 \in \mems(S)$ we have:
  \begin{eqnarray*}
    & \store_1 =_C \store_2 \wedge
     (\margd{(\condd{\progd(\prog)}{\store_1})}{\{ \outv \}} =
     \margd{(\condd{\progd(\prog)}{\store_2})}{\{ \outv \}}) \\
    & \implies \\
    & \margd{(\condd{\progd(\prog)}{\store_1})}{V_C} =
    \margd{(\condd{\progd(\prog)}{\store_2})}{V_C}
  \end{eqnarray*}
where $\iov(\prog) = (S,V)$.
\end{definition}

From an information flow perspective, we can say that a $\NIMO$
protocol does not release information beyond what is declassified
via the public output. From a real/ideal perspective, we can
observe that $\NIMO$ implies that the simulator can use
the protocol itself to reconstruct the distribution of adversarial
views. That is, given the output of the ideal functionality
$\idealf(\store_S)$, the simulator can randomly choose any
$\store'$ such that $\store' =_C \store$ and $\idealf(\store'_S) =
\idealf(\store_S)$ and run the protocol with initial memory $\store'$-
$\NIMO$ guarantees ideal reconstruction of the real-world corrupt views. 
 However, there is a subtlety here in that the adversary
must be able to tractably reconstruct this knowledge. This is not
generally true, in particular, if $\idealf$ is a 1-way
function. Define:
\begin{definition}[Ideal Knowledge]
  Given a functionality $\idealf$ and output value $v$, the associated
  \emph{ideal knowledge}, denoted $\ik(\idealf,v)$ is:
  $$
  \{ \store\ |\ \idealf(\store) = v \}
  $$
  We say that $\idealf$ is \emph{invertible} iff $\ik(\idealf, v)$ for all
  $v$ can be computed tractably.
\end{definition}
Now we can show our main result, that noninterference modulo output,
together with the assumption that $\idealf$ is invertible, implies
passive security of a protocol.
\begin{theorem}
  \label{theorem-nimo}
  Assume given invertible $\idealf$ and a protocol $\prog$ that
  correctly implements $\idealf$.  If $\NIMO(\prog)$
  then $\prog$ is passive secure.
\end{theorem}

\begin{proof}
  Let $H$ and $C$ be an arbitrary partition of the federation and
  suppose $\prog$ satisfies noninterference modulo output. Let
  $\iov(\prog) = (V,S)$ and $\flips(\prog) = F$ with output view $x
  \in V$ and let $\store$ be an arbitrary member of $\mems(S \cup
  F)$. Then the distribution of adversarial views in the real world
  is, by definition:
  $$\margd{(\condd{\progd(\prog)}{\store_S})}{V_C}$$

  The simulator is given both $\store_{S_C}$ and
  $\idealf(\store_S)$.  The simulation $\SIM_C(\store_{S_C},
  \idealf(\store_S))$ can be defined as follows. First, some $\store'
  \in \ik(\idealf,\idealf(\store))$ is randomly chosen such that
  $\store' =_C \store_S$, as is a random tape $\store'' \in
  \mems(F)$\footnote{The real/ideal model allows consultation of a
  ``Random Oracle''.}. Then, the run of $(\store' \cup \store'',
  \prog)$ is evaluated in simulation, yielding $(\store^{\SIM},\varnothing)$.
  The simulation returns $\store^{\SIM}_{V_C}$ as a result.

  Now, since the random tape is selected in simulation from the same distribution
  that we assume for the real world, after selection of $\store'$ the
  probability that any particular $\store^{\SIM}_{V_C}$ is returned is by definition:
  $$
   (\margd{(\condd{\progd(\prog)}{\store'})}{V_C})(\store^{\SIM}_{V_C})
  $$
  Furthermore, since we assume that $\prog$ correctly implements $\idealf$, this
  means:
  $$
  \store_S =_C \store' \wedge
     (\margd{(\condd{\progd(\prog)}{\store_S})}{\{ x \}} =
      \margd{(\condd{\progd(\prog)}{\store'})}{\{ x \}})
  $$
  so by the definition of noninterference modulo output the choice
  any choice of $\store'$ yields the same distribution, i.e.:
  $$\margd{(\condd{\progd(\prog)}{\store_S})}{V_C}$$
  Thus, by definition:
  $$
   \progd(\SIM_C(\store_{S_C},\idealf(\store))) = \margd{(\condd{\progd(\prog)}{\store_S})}{V_C}
  $$
\end{proof}

As for noninterference, we can show that $\NIMO$ is equivalent
to saying that a given program does not change the probability
of honest secrets conditioned on corrupt secrets and views
(including the output view), as compared to conditioning on
just the corrupt secrets and input. In addition to providing
insights about $\NIMO$, this form is helpful for
verification techniques. 
\begin{lemma}
  \label{lemma-nimo}
  $\NIMO(\prog)$ iff for all $H$ and $C$ with $|C| \le |H|$ and
  $\store \in \mems(S \cup V)$:
  $$\margd{(\condd{\progd(\prog)}{\store_{(S \cup \{\outv\})_C}})}{S_H} =
  \margd{(\condd{\progd(\prog)}{\store_{(S\cup V)_C}})}{S_H} $$
  where  $\iov(\prog) = (S,V)$.
\end{lemma}


\section{Automating Security Verification}
\label{section-automation}

%We first explore a brute force approach to automatically verifying
%program behaviour, the first step of which is to compute the basic
%distribution of a program $\prog$. The program distribution
%$\progd(\prog)$ can obviously be automatically derived from that, as
%can be any of its marginal or conditional distributions. In this
%Section, we consider two automated mechanisms for deriving
%$\progd(\prog)$ in this way. The first is through a straightforward
%computation of all final program memories.  The second approach
%rewrites $\minifed$ programs to stratified Datalog programs, that are
%amenable HPC optimizations such as parallelization and GPU matrix
%computations as shown in recent work
%\cite{sakama2017linear,aspis2018linear,nguyen2022enhancing,nguyen2021efficient}.

%Here we describe a straightforward technique for obtaining program
%distributions $\progd(\prog)$ through direct computation of
%$\runs(\prog)$. Although we could obtain $\runs(\prog)$ simply by
%executing $\prog$ given all random tapes and input secrets, the method
%we describe here is empirically more efficient (though still
%exponential in the length of the random tape), and also supports the
%translation to Datalog described in Section
%\ref{section-bruteforce-datalog}. An implementation of these
%techniques along with Examples from Sections
%\ref{section-minicat-examples} and \ref{section-metalang-ygc} is
%available online\footnote{URL redacted for double-blind review.}.

Our first step in automating security analysis is to show how to
compute the basic program distribution. Recall that basic
distributions are based on the memories resulting from execution of an
$\minicat$ program, which contain a record of all client views and the
public output.  Given $\prog$ we denote this as $\runs(\prog)$, and we
denote the basic distribution of a program as $\progtt(\prog)$.
\begin{definition}
  Given $\prog$ with $\iov(\prog) = S \cup V$ and $\flips(\prog) = F$:
  $$
  \runs(\prog) \defeq \{ \store \in \mems(S\cup F \cup V) \mid \config{\store_{S \cup F}}{\prog} \redxs \config{\store}{\varnothing} \}
  $$
  We write $\progtt(\prog)$ to denote the \emph{basic distribution} of
  $\prog$, where given $\iov(\prog) = S \cup V$ and $\flips(\prog) = F$
  we have for all $\store \in \mems(S \cup V \cup F)$:
  $$
  \progtt(\prog)(\store) =  1 / 2^{|S\cup F|} \ \text{if}\ \store \in \runs(\prog), \text{otherwise}\ 0
  $$
\end{definition}

Any marginalization or conditioning of $\progtt(\prog)$ can be easily
obtained algorithmically, in particular $\progd(\prog) =
\margd{\progtt(\prog)}{\iov(\prog)}$.  As already discussed in Section
\ref{section-hyperprop} and as we will discuss more below, extensional
and intensional security properties are predicated on marginalizations
and conditionings of basic distributions. So calculating
$\runs(\prog)$ establishes the foundation for our analysis.

\begin{fpfig}[t]{Truth table visualizations of $\runs(\prog_{\ref{example-otp}})$ (L) and
    $\runs(\prog_{\ref{example-lambda-obliv}(a)})$ (R).}{fig-basic-distributions}
{\footnotesize
  $$
  \begin{array}{cc}
    \begin{array}{cccccc}
      \verb+s[1,0]+ & \verb+f[1,0]+ & \verb+v[2,0]+  & \verb+v[2,1]+ & \verb+v[2,2]+ & \verb+v[0,0]+\\
      \hline
      0 & 0 & 0 & 0 & 0 & 0 \\ 
      0 & 1 & 1 & 1 & 0 & 1 \\ 
      1 & 0 & 0 & 1 & 1 & 1 \\ 
      1 & 1 & 1 & 0 & 1 & 0
    \end{array}
    & 
    \begin{array}{ccccc}
      \verb+s[1,s]+ & \verb+f[1,sx]+ & \verb+f[1,sy]+ & \verb+v[0,0]+ & \verb+v[0,1]+ \\
      \hline
      0 & 0 & 0 & 0 & 0 \\ 
      0 & 0 & 1 & 1 & 0 \\ 
      0 & 1 & 0 & 0 & 1 \\ 
      0 & 1 & 1 & 1 & 1 \\
      1 & 0 & 0 & 0 & 0 \\ 
      1 & 0 & 1 & 0 & 0 \\ 
      1 & 1 & 0 & 1 & 1 \\ 
      1 & 1 & 1 & 1 & 1  
    \end{array}
  \end{array}
  $$
}
\end{fpfig}

Intuitively, we can represent $\runs(\prog)$ in tabular form, as
illustrated in Figure \ref{fig-basic-distributions} for Examples
\ref{example-otp} and \ref{example-lambda-obliv}(a). So a basic
strategy is to directly compute this truth table for any given
program. We first describe an approach where we compute
$\runs(\instr_1,\ldots,\instr_n)$ iteratively for the subprograms
$\instr_1,\ldots,\instr_i$ for $1 \le i \le n$, by individually
computing the truth table of expressions $\be_i$ for each $\instr_i =
(\eassign{v_i}{\be_i})$, and joining this with
$\runs(\instr_1,\ldots,\instr_{i-1})$.

In Figure \ref{fig-solve} we define the algorithm
$\solve{\stores}{\be}$ which filters a given $\stores$ to obtain the
subset whose elements satisfy $\be$ (make it true). It's correctness
is characterized as follows.
\begin{lemma}
  \label{lemma-solves}
  For all $\stores$ and $\be$ with $\vars(\be) \subseteq \dom(\stores)$,
  $(\solve{\stores}{\be}) = \{ \store \in \stores \ \mid\ \lcod{\store,\be}{\cid} = 1 \}$
  for some $\cid$.
\end{lemma}
\begin{proof}
  Since we assume safety of programs we can assume that all variables in $\be$ have the
  same owner $\cid$. The result otherwise follows in a straightforward manner by induction
  on $\be$. 
\end{proof}

\begin{fpfig}[t]{Filtering memories that satisfy a boolean expression.}{fig-solve}
{\small
\begin{eqnarray*}
\solve{\stores}{\etrue} &=& \stores\\
\solve{\stores}{\efalse} &=& \varnothing\\
\solve{\stores}{\flip{\cid}{w}} &=& \{ \store \in \stores \mid \store(\flip{\cid}{w}) \} \\
\solve{\stores}{\secret{\cid}{w}} &=& \{ \store \in \stores \mid \store(\secret{\cid}{w}) \} \\
\solve{\stores}{\view{\cid}{w}} &=& \{ \store \in \stores \mid \store(\view{\cid}{w}) \} \\
\solve{\stores}{\oracle{w}} &=& \{ \store \in \stores \mid \store(\oracle{w}) \} \\
\solve{\stores}{(\enot\ \be)} &=& \stores - (\solve{\stores}{\be})\\
\solve{\stores}{(\be_1\ \eand\ \be_2)} &=& (\solve{\stores}{\be_1}) \cap (\solve{\stores}{\be_2}) \\
\solve{\stores}{(\be_1\ \eor\ \be_2)} &=& (\solve{\stores}{\be_1}) \cup (\solve{\stores}{\be_2}) \\
\solve{\stores}{(\be_1\ \exor\ \be_2)} &=&
 ((\solve{\stores}{\be_1}) \cap (\stores - \solve{\stores}{\be_2})) \cup\\
 && ((\stores - \solve{\stores}{\be_1}) \cap (\solve{\stores}{\be_2})) \\
\solve{\stores}{\select{\be_1}{\be_2}{\be_3}} &=&
 ((\solve{\stores}{\be_1}) \cap (\solve{\stores}{\be_2})) \cup \\
 && ((\stores - \solve{\stores}{\be_1}) \cap (\solve{\stores}{\be_3})) \\
\solve{\stores}{\OT{\be_1}{\be_2}{\be_3}} &=&
 ((\solve{\stores}{\be_1}) \cap (\solve{\stores}{\be_2})) \cup\\
 && ((\stores - \solve{\stores}{\be_1}) \cap (\solve{\stores}{\be_3}))
\end{eqnarray*}
}
\end{fpfig}

To obtain $\runs(\prog)$, we can perform a left folding of $\solve$
across $\prog$ and each view definition, inductively extending
memories with valid view assignments in the order of their definition.
We denote this computation as $\cruns(\prog)$.
\begin{lemma}
  \label{lemma-cruns}
  Given $\prog$ where $\iov(\prog) = S \cup V$ and $\flips(\prog) = F$. Define:
  \begin{eqnarray*}
    {tt}\ \ \stores\ (\eassign{\view{\cid}{w}}{\be}) &\defeq& \begin{array}{l}
      \mathrm{let}\ \stores' = \solve{\stores}{\be} \ \mathrm{in}\\
      \ \ \{\extend{\store}{\view{\cid}{w}}{1} \mid \store \in \stores' \}\ \cup\\
      \ \ \{\extend{\store}{\view{\cid}{w}}{0} \mid \store \in \stores - \stores' \}\end{array}\\[2mm]
    \cruns(\prog) &\defeq& \mathit{foldl}\ {tt}\ \mems(S \cup F)\ \prog
  \end{eqnarray*}
  Then $\cruns(\prog) = \runs(\prog)$.
\end{lemma}
\begin{proof}
  By Lemma \ref{lemma-solves} and induction on the length of $\prog$. 
\end{proof}
The above yields an algorithm for computing basic distributions and
thus program distributions. 
\begin{lemma}
  \label{lemma-cprogtt}
  Given $\prog$ where $\iov(\prog) = S \cup V$ and $\flips(\prog) = F$. Define
  $\cprogtt(\prog)$ as the distribution $\pmf$
  where for all $\store \in \mems(S\cup F \cup V)$:
  $$
  \pmf(\store) = 1 / 2^{|S\cup F|} \ \text{if}\ \store \in \cruns(\prog), \text{otherwise}\ 0
  %\begin{cases}1/2^{|S \cup F|} & \text{if\ } \store \in
  %  \cruns(\prog) \\ 0 & \text{otherwise} \end{cases} 
  $$
  Then $\cprogtt(\prog) = \progtt(\prog)$ and $\margd{\cprogtt(\prog)}{S\cup V} = \progd(\prog)$.
\end{lemma}
\begin{proof}
  Immediately by Lemma \ref{lemma-cruns} and Definition \ref{def-progd}.
\end{proof}
A full implementation of these methods is available in our online
repository \cite{jpdf-github}. To
avoid notational confusion we will continue to refer to $\progtt$ and
$\cruns$ in our subsequent discussion of automated verification but
with the understanding that both can be computed using the above
algorithms.

Since $|\runs(\prog)|$ grows exponentially in the size of
$\flips(\prog)$, scalability to larger programs is a concern.  However
computation of $\runs$ is amenable to optimizations, especially
parallelization- rather than computing the truth table
``horizontally'', it can be computed vertically. We can rewrite
$\minifed$ programs $\prog$ to stratified Datalog programs, where
$\runs(\prog)$ is equivalent to set of least Herbrand models obtained
by representing different random tapes and input secrets as different
fact bases. Details of this rewrite are provided in Appendix
\ref{section-bruteforce}. The model for each fact base can be computed
independently and in parallel, with the potential to apply other
matrix optimizations developed in recent work on HPC for Datalog
\cite{sakama2017linear,aspis2018linear,nguyen2022enhancing,nguyen2021efficient}.

\subsection{Certification of Extensional Properties}
\label{section-automation-extensional}

We will use \emph{certification} to refer to algorithmic methods of
verifying properties of $\minifed$ protocols, vs.~manual methods. For
the sake of clarity we make the following definition:
\begin{definition}
  A predicate $\phi$ on $\minifed$ protocols is \emph{certifiable} iff
  there exists an algorithm $A$ such that $A(\prog)$ iff $\phi(\prog)$,
  and a safe protocol $\prog$ is \emph{certified for $\phi$} iff
  $A(\prog)$ holds.
\end{definition}

Given computation of $\prog(\prog)$ for programs $\prog$, we can
immediately observe that the extensional hyperproperties discussed in
Section \ref{section-hyperprop} are certifiable.
\begin{lemma}
  Both $\pni$ (Definition \ref{definition-PNI}) and $\NIMO$ (Definition \ref{definition-NIMO}) are certifiable. 
\end{lemma}

\begin{proof}
  Let $\progd(\prog)$ be computed as described above, and let $\iov(\prog) = S
  \cup V$. To verify $\pni(\prog)$ for
  some given $H$ and $C$, we enumerate $\runs(\prog)$ and check:
  $$
  \progd(\prog)(\store_{S_H}|\store_{S_C}) =
  \progd(\prog)(\store_{S_H}|\store_{S_C \cup V_C})
  $$
  for all $\store \in \mems(S \cup V)$, and return success $\pni(\prog)$ iff every such
  check succeeds. The result follows by Lemmas \ref{lemma-cprogtt} and \ref{lemma-pni}.
  To verify $\nimo(\prog)$, we first enumerate all partitions of clients
  into set $H$ and $C$ where $0 \in C$, then enumerate $\runs(\prog)$, and
  then check:
  $$
  \progd(\prog)(\store_{S_H}|\store_{S_C \cup \{\outv\}}) =
  \progd(\prog)(\store_{S_H}|\store_{S_C \cup V_C})
  $$
  The result follows by Lemma \ref{lemma-cprogtt} and Lemma \ref{lemma-nimo}.
\end{proof}

We can thus automatically verify security properties of examples considered
in previous sections, observing that passive security of $\prog_{\ref{example-OT}}$
has not been previously demonstrated.
\begin{lemma}
  $\prog_{\ref{example-he}}$, $\prog_{\ref{example-gmw-andcircuit}}$,
  $\prog_{\ref{example-ygc-andcircuit}}$ are all passive secure,
  $\prog_{\ref{example-OT}}$ is passive secure assuming client 3 is
  honest, and $\prog_{\ref{example-otp}}$ satisfies probabilistic
  noninterference and hence perfect secrecy (as defined in
  \cite{barthe2019probabilistic}).
\end{lemma}
\begin{proof}
  We have certified $\nimo(prog_{\ref{example-he}})$, $\nimo(\prog_{\ref{example-gmw-andcircuit}})$,
  $\nimo(\prog_{\ref{example-ygc-andcircuit}})$, and $\nimo(prog_{\ref{example-OT}})$ assuming
  $3 \in H$, and $\pni(\prog_{\ref{example-otp}})$. The result follows by Lemmas
  \ref{lemma-pni} and \ref{lemma-nimo}.
\end{proof}

\subsection{Certification of Composable Intensional Properties}
\label{section-automation-intensional}

While the brute force method is feasible for smaller programs,
including small circuits such as Examples
$\prog_{\ref{example-gmw-andcircuit}}$ and
$\prog_{\ref{example-ygc-andcircuit}}$, it is not for larger programs
such as YGC or GMW circuits since the size of $\runs(\prog)$ is
exponential in the size of the random tape plus the number of input
secrets, and circuits can be quite large in practice
\cite{kreuter2012billion}.  However, the YGC and GMW protocols are
examples of a common idiom in MPC protocol design- arbitrarily large
circuits are built up from smaller boolean or arithmetic gates that
operate on encodings of input secrets.

Our solution is to design gate certifications that demonstrably
guarantee security for circuits built from them. Intuitively, we can
consider any gate as a mini-program and certify intensional properties
of its isolated program distribution. These intensional properties are
guaranteed secure in circuit compositions, as we demonstrate using
methods based on probabilistic separation logic
\ref{barthe2019probabilistic}. While this last step is manual, our
gate certifications are defined generally and can be used to certify
new gate implementations, and verification applies to a circuit of any
size. In Section \ref{section-composition} we will demonstrate our
approach using YGC and GMW as examples.

When we consider intensional properties
of gates in isolation, we will explicitly consider flip conditions in
distributions. Thus we will be directly concerned with basic distributions,
which we denote $\progtt(\prog)$, and we will often focus on views
defined in subprotocols. 
Notions of probabilistic independence, aka \emph{separation}, as well
as correlation, are important to certify. We borrow the symbols $*$
and $\sim$ from \cite{barthe2019probabilistic} with the same
denotations.
\begin{definition}
  We write $\vc{\pmf}{x}{y}$ iff $\pmf(\{ x \mapsto 0\}\ |\ \{ y \mapsto 0 \}) =
  \pmf(\{ x \mapsto 1\}\ |\ \{ y \mapsto 1 \}) = 1$.
  We write $\sep{\pmf}{X}{Y}$ iff for all
    $\store \in \mems(X \cup Y)$ we have
  $\margd{\pmf}{X \cup Y}(\store) =
  \pmf(\store_X) * \pmf(\store_Y)$
\end{definition}
We immediately observe that separation and correlation are certifiable,
since they are based on marginalization and conditioning of program
distributions. 
\begin{lemma}
  \label{lemma-autosep}
  Both of the following are true:
  \begin{enumerate}
  \item $\sep{\progtt(\prog)}{X}{Y}$ is certifiable for any $X$ and $Y$.
  \item $\vc{\progtt(\prog)}{x}{y}$ is certifiable for any $x$ and $y$.
  \end{enumerate}
\end{lemma}

\subsection{Compositional Metatheory}
\label{section-automation-logic}

Since we certify properties of gates in isolation with witness inputs,
our metatheory is concerned with the preservation of important invariants
when certified components are ``spliced'' into larger program
contexts. This technique often requires us to focus on views defined
in subprograms, so we define the following.
\begin{definition}
For any program $\prog =
  (\eassign{v_1}{\be_1};\ldots;\eassign{v_n}{\be_n})$, define
$\vdefs(\prog) \defeq \{ v_1,\ldots,v_n \}$.
\end{definition}

An important reasoning principle is related to component
noninterference- specifically, if we show that gate views enjoy a
noninterference property wrt their free (input) variables,
probabilistic independence propagates to other variables in the
preceding program.
\begin{restatable}[Noninterference]{lemma}{noninterference}
  \label{lemma-noninterference}
  Given $\prog_1;\prog_2$ and $X = \iov(\prog_2) - \vdefs(\prog_2)$ and
  $Y \subseteq \vdefs(\prog_2)$. If $\sep{\progd(\prog_1;\prog_2)}{X}{Y}$
  then $\sep{\progd(\prog_1;\prog_2)}{\iov(\prog_1)}{Y}$.
\end{restatable}

Also, when we certify gates in isolation with witness inputs, we can
``splice'' these properties into larger program contexts using the
following substitution Lemmas, showing that the substitution of
wire inputs for witness inputs preserves both separation and correlation.
\begin{restatable}[Substitution$*$]{lemma}{substar}
  \label{lemma-substitution}
  If $\sep{\progtt(\prog_2)}{\{ f \}}{X}$ and
  $\prog_1;\prog_2[\be/f]$ is safe with $\vars(\prog_1,\be) \cap
  X = \varnothing$ then
  $\sep{\progtt(\prog_1;\prog_2[\be/f])}{\vars(\be)}{X}$.
\end{restatable}

\begin{lemma}[Substitution$\sim$]
  \label{lemma-substitution-sim}
  If $\vc{\progtt(\prog_2,\be)}{\itv}{f}$ and $\vc{\progtt(\prog_1,\be')}{\itv}{f'}$
  and $\vars(\be') \cap (\vars(\be) - \{ f \}) = \varnothing$
  then $\vc{\progtt(\prog_1;\prog_2,\be[\be'/f'])}{\itv}{f}$.
\end{lemma}

We note that while the Frame rule of \cite{barthe2019probabilistic}
supports local reasoning in larger program contexts, it does not
support this splicing method, and our Lemmas \ref{lemma-substitution}
and \ref{lemma-substitution-sim} are both novel. Supplementary proofs
are demonstrated in Appendix \ref{section-proofs}.


\section{Component Certification and Composition}
\label{section-composition}

\begin{fpfig}[t]{2-Party GMW circuit library with And gate.}{fig-gmw}
{\footnotesize
  \begin{verbatimtab}
    encodegmw(in, i1, i2) {
      m[in]@i2 := (s[in] xor r[in])@i2;
      m[in]@i1 := r[in1]@2;
      m[in]
    }
    
    andtablegmw(b1, b2, r) {
      let r11 = r xor (b1 xor true) and (b2 xor true) in
      let r10 = r xor (b1 xor true) and (b2 xor false) in
      let r01 = r xor (b1 xor false) and (b2 xor true) in
      let r00 = r xor (bl xor false) and (b2 xor false) in
      { v1 = r11; v2 = r10; v3 = r01; v4 = r00 }
    }
    
    andgmw(g, v1, v2) {
      let r = r[g] in
      let table = andtablegmw(v1,v2,r) in
      m[g]@2 := OT4(v1,v2,table,2,1);
      m[g]@1 := r;
      m[g]
    }
    
    decodegmw(v) {
      p[1] := v@1; p[2] := v@2;
      out@1 := (p[1] + [2])@1;
      out@2 :=(p[1] + [2])@2
    }
  \end{verbatimtab}
}
\end{fpfig}


The Yao's Garbled Circuits (YGC) and GMW protocols are examples of a
common idiom in MPC protocol design- arbitrarily large \emph{circuits}
are built up from smaller boolean or arithmetic \emph{gates} that
operate on encodings of input secrets. These circuits can become quite
large in practice \cite{kreuter2012billion}, but are built from
well-defined components. Brute force verification techniques as
described in Section \ref{section-bruteforce} are not scalable to
large circuits, but in this Section we will demonstrate that correctly
designed circuit components enjoy compositional properties
that can be automatically certification on small program components
using brute force methods. We prove that these properties preserve
invariants throughout circuit definition that yield $\NIMO$ and thus
scale to arbitrary programs. Furthermore, the certifications can be
used on arbitrary gate definitions, allowing new gate implementations
to be added soundly to libraries.

In this Section we develop and use implementations of both 2-party GMW
and YGC to illustrate our approach and demonstrate its generality.
Main results are in Theorems \ref{theorem-gmw-NIMO} and
\ref{theorem-ygc-NIMO}.

\subsection{Compositional Properties of GMW}

The GMW protocol uses secret sharing to represent data flowing through
circuits. In the 2-party case, clients 1 and 2 each share their input
secrets, and use those shares to represent inputs to gates. Outputs
are also represented as shares. We refer to the pair of shares
representing any particular value as a \emph{wire value}, and
we represent them via records of the form
$
\ttt{\{c1 = } v_1\ttt{;c2 = } v_2\ttt{\}} 
$
where $v_1$ and $v_2$ are client 1 and 2's shares respectively.

For full details of the GMW protocol the reader is referred to
\cite{evans2018pragmatic}. Our implementation libary is shown in
Figure \ref{fig-gmw}, with type signatures for the library functions
shown in Figure \ref{fig-gmw-types}. We show the And gate since it is
an interesting component. The Figure includes the
following top-level functions:
\begin{itemize}
\item \ttt{encodegmw}: This function encodes client 1's and client 2's
  secret bits called $\ttt{s[1,s1]}$ and $\ttt{s[1,s2]}$ into two
  distinct wire values (pairs of shares).
\item \ttt{andgmw}: This function defines the gate $\ttt{g}$, the
  identifier $\ttt{g}$ being used to distinguish randomness used
  within.  In our version client 1 builds the output table (using
  \ttt{andtablegmw})and transfers the correct output share to client 2
  using 1-out-of-4 OT as per standard GMW protocol.
\item \ttt{decode}: This function decodes and publishes a wire value
  by $\ttt{xor}$ing the shares. Note that this requires both client 1
  and 2 to publicize their shares.
\end{itemize}
\begin{example}
  \label{example-gmw-andcircuit}
The following program uses our GMW library to define
a circuit with a single And gate and input secrets $\ttt{s1}$ and
$\ttt{s2}$ from client's 1 and 2 respectively. 
\begin{verbatimtab}
  let ss = encodegmw(s1,s2) in
  v[0,output] := decode(andgmw(0,ss.shares1,ss.shares2))
\end{verbatimtab}
\end{example}

\subsubsection{Gate Certification}

Our certification techniques are defined generally wrt implementations
of input encoding, decoding, and internal gates. In the following we
will refer to arbitrary decode, encode, and gate functions with the
restriction that any function in each of these categories has the same
valid type signature as $\ttt{decodegmw}$, $\ttt{encodegmw}$, and
$\ttt{andgmw}$ respectively. When gates are used in ciruits they are
parameterized by wire values-- in certification we will use carefully
chosen witness parameters. 

When we consider internal components and hence intensional properties
of circuits, we will consider behaviors with respect to particular bits
of randomness. Hence we will consider basic program distributions. 
\begin{definition}
  We write $\progtt(\prog)$ to denote the \emph{basic distribution} of
  $\prog$ as defined in Definition \ref{def-progd}. For any program $\prog =
  (\eassign{v_1}{\be_1};\ldots;\eassign{v_n}{\be_n})$, define
  $\vdefs(\prog) \defeq \{ v_1,\ldots,v_n \}$.
\end{definition}
We also need to formalize notions of probabilistic independence, aka
\emph{separation}, and correlation. We borrow the symbols $*$ and $\sim$
from \cite{barthe2019probabilistic} with the same denotations.
\begin{definition}
  We write $\vc{\pmf}{x}{y}$ iff $\pmf(\{ x \mapsto 0\}\ |\ \{ y \mapsto 0 \}) =
  \pmf(\{ x \mapsto 1\}\ |\ \{ y \mapsto 1 \}) = 1$.
  We write $\sep{\pmf}{Y}{Z}$ iff for all
    $\store \in \mems(X \cup Y)$:
  $$\margd{\pmf}{Y \cup Z}(\store) =
  \pmf(\store_Y) * \pmf(\store_Z)$$ 
\end{definition}

\begin{fpfig}[t]{GMW library type annotations.}{fig-gmw-types}
{\footnotesize
  \begin{verbatimtab}
   encodegmw   : string(gid) * string(gid)
    
   andtablegmw : { k = bool[i]; p = bool[i] }
    
   andgmw      : string(gid) *  { c1 = bool[1]; c2 = bool[2] } * { c1 = bool[1]; c2 = bool[2] }
    
   decodegmw   : { c1 = bool[1]; c2 = bool[2] }  \end{verbatimtab}
}
\end{fpfig}

Now we can define GMW gate certification as follows. Intuitively, we certify
that GMW gates satisfy a probabilistic noninterference property wrt each output
share considered individually. This ensures that input dependencies remain
encrypted from each client's perspective during circuit evaluation.
\begin{definition}[GMW Gate Certification]
  \label{definition-gmwgate-certification}
  Let $\mathit{gate}$ be a gate function and let $e$ be the following
  $\metaprot$ program:
  $$
  \begin{array}{l}
    \ttt{let in1 = \{c1 = flip[1,s11];c2 = flip[2,s12]\} in}\\
    \ttt{let in2 = \{c1 = flip[1,s21];c2 = flip[2,s22]\} in}\\
    \mathit{gate}\ttt{(g,in1,in2)}
  \end{array}
  $$
  and let $
  \config{\varnothing}{e}\redxs\config{\prog}{\ttt{\{c1 = v[1,gout]; c2 = v[2,gout]\}}}
  $.
  Then $\mathit{gate}$ is \emph{certified} iff the following conditions hold:
  \begin{enumerate}[\hspace{5mm}i.]
  \item $\sep{\progtt(\prog)}{\ttt{\{flip[1,s11],flip[1,s21],flip[2,s12],flip[2,s22]\}}}{\ttt{\v[1,gout]\}}}$
  \item $\sep{\progtt(\prog)}{\ttt{\{flip[1,s11],flip[1,s21],flip[2,s12],flip[2,s22]\}}}{\ttt{\v[2,gout]\}}}$
  % \item $\sep{\progtt(\prog)}{\ttt{\{v[1,gout]\}}}{\ttt{\{v[2,gout]\}}}$
  \item All variables in:$$\vars(\prog) - \{ \ttt{flip[1,s11],flip[1,s21],flip[2,s12],flip[2,s22]}\}$$
    are distinguished by $\ttt{g}$ (i.e., contain $\ttt{g}$ as an identifier substring). 
  \end{enumerate}
\end{definition}

Encoding certification establishes the desired property in the protocol-- that
input secrets of each client are independent of the other client's views. 
\begin{definition}[GMW Encode Certification]
  \label{definition-gmwencode-certification}
  Let $\mathit{encode}$ be an encoding function, and let:
  $$
  \config{\varnothing}{\ttt{encode(s1,s2)}} \redxs
  \config{\prog}{\ttt{\{shares1 = }v_1\ttt{; shares2 = }v_2\ttt{\}}}
  $$
  where:
  $$
  \begin{array}{rcl}
    v_1 &=& \ttt{\{ c1 = v[1,s1out]; c2 = v[2,s1out] \}}\\
    v_2 &=& \ttt{\{ c1 = v[1,s2out]; c2 = v[2,s2out] \}}
  \end{array}
  $$
  Then $\mathit{encode}$ is \emph{certified} iff each of the following conditions hold:
  \begin{enumerate}[\hspace{5mm}i.]
    %\item $\sep{\progtt(\prog)}{\ttt{\{ s[1,s1],  v[1,s1out], v[1,s2out] \}}}{\ttt{\{ s[2,s2], v[2,s1out], v[2,s2out] \}}}$
  \item $\sep{\progtt(\prog)}{\ttt{\{s[1,s1]\}}}{\ttt{\{v[2,s1out],v[2,s2out]\}}}$
  \item $\sep{\progtt(\prog)}{\ttt{\{s[2,s2]\}}}{\ttt{\{v[1,s1out],v[1,s2out]\}}}$
  \item All $x \in \vars(\prog)$ are distinguished by $\ttt{in1}$ or $\ttt{in2}$. 
  \end{enumerate}
  \end{definition}

We have certified both $\ttt{encodegmw}$ and $\ttt{andgmw}$. Note that
an Or or Xor gate, for example, could be certified with this same
method to extend our library.
\begin{lemma}
  \label{lemma-gmw-certification}
  Each of $\ttt{encodegmw}$ and $\ttt{andgmw}$ are certified.
\end{lemma}

\subsubsection{Gate Invariants and Circuit Security}
We can show that composition maintains an important invariant in a
circuit $\prog$- namely that client 1's secrets remain independent of
client 2's views, and vice-versa, prior to decoding.  That is, where
$\iov(\prog) = S \cup V$, we show that
$\sep{\progd(\prog)}{S_{\{1\}}}{V_{\{2\}}}$ and
$\sep{\progd(\prog)}{S_{\{2\}}}{V_{\{1\}}}$ are preserved prior to
decoding.

We observe the following properties of separation which are borrowed from prior work
\cite{barthe2019probabilistic} and which will be used frequently in proofs. 
\begin{lemma}
  \label{lemma-separation}
  The following properties hold:
  \begin{enumerate}
  \item $\sep{\pmf}{Y}{Z}$ iff $\sep{\pmf}{Z}{Y}$
  \item $\vc{\pmf}{x}{y}$ if  $\vc{\pmf}{y}{x}$
  \item $\vc{\pmf}{x}{y}$ and $\vc{\pmf}{y}{z}$ imply $\vc{\pmf}{x}{z}$
  \item $\sep{\pmf}{X}{(Y \cup Z)}$ implies $(\sep{\pmf}{X}{Y}$ and $\sep{\pmf}{X}{Z})$
  \item $(\sep{\pmf}{X}{Y}$ and $\sep{\pmf}{(X \cup Y)}{Z})$ implies $\sep{\pmf}{X}{(Y \cup Z)}$ 
  %\item If $\prog_1;\prog_2$ is safe and $\vars(\prog_1) \cap \vars(\prog_2) = \varnothing$
  %  then $\sep{\progtt(\prog_1;\prog_2)}{\vars(\prog_1)}{\vars(\prog_2)}$.
  \end{enumerate}
\end{lemma}
The following property also follows by results in \cite{barthe2019probabilistic} and will
be useful to make constructions that demonstrate variable separation.
\begin{lemma}
  \label{lemma-sepjoin}
  $\sep{\progtt(\prog)}{X}{Y}$ iff for all 
  $\store^1, \store^2 \in \runs(\prog)$ there exists
  $\store \in \runs(\prog)$ with
  $\store^1_{X} \cap \store^2_{Y} \subseteq \store$.
\end{lemma}

%The following Lemma allows us to separate variables in programs that may not occur
%in the same subprogram, and hence consider gate certification properties in various
%contexts. It is reminiscent of the \emph{frame} rule of \cite{barthe2019probabilistic}.
%\begin{lemma}
%  \label{lemma-gmw-frame}
%  If $\sep{\progtt(\prog_1;\prog_2)}{Y}{V}$ for $Y = \vars(\prog_1) \cap \vars(\prog_2)$ and
%  $V \in \vdefs(\prog_2)$ and $\sep{\progtt(\prog_1)}{X}{Y}$ for
%  $X \not\in \vars(\prog_2)$ then $\sep{\progtt(\prog_1;\prog_2)}{X}{V}$. 
%\end{lemma}
%\begin{proof}
%  By assumption and Lemma \ref{lemma-sepjoin}, for all
%  $\store^1,\store^2 \in \runs(\prog_1;\prog_2)$ we have $\store^1_Y
%  \cap \store^2_V \subseteq \store_a$ for some $\store_a \in
%  \runs(\prog_1;\prog_2)$ and $\store^1_X \cap \store^2_Y \subseteq
%  \store_b$ for some $\store_b \in \runs(\prog_1;\prog_2)$.  Thus, since
%  we assume $X \not\in \vars(\prog_2)$ we have $\store_a \cap \store_b
%  \in \runs(\prog_1;\prog_2)$ and $\store^1_X \cap \store^2_V =
%  \store_a \cap \store_b$, obtaining the result by
%  \ref{lemma-sepjoin}.
%\end{proof}

The next ``noninterference'' Lemma is important to prove the desired invariance
preservation property of GMW gates, given certification.
\begin{lemma}[Noninterference]
  \label{lemma-noninterference}
  Given $\prog_1;\prog_2$ and $X = \iov(\prog_2) - \vdefs(\prog_2)$ and
  $Y \subseteq \vdefs(\prog_2)$. If $\sep{\progd(\prog_1;\prog_2)}{X}{Y}$
  then $\sep{\progd(\prog_1;\prog_2)}{\iov(\prog_1)}{Y}$.
\end{lemma}

The next two Lemmas allow us to use the properties of gates established by
certification and replace the ``dummy'' parameters used there with actual
wire values in full circuit contexts. Intuitively we show that substitution
for dummy flips preserves separation. 
In these results and subsequently we will use $f$ to range over flips.
\begin{lemma}
  \label{lemma-presub}
  Given $\be_1$, $\be_2$, and $i$ where $\vars(\be_1) \cap
  \vars(\be_2) = \varnothing$ and $\lcod{\store_1,\be_1}{i} = \beta$
  with $\store_1(x) = \lcod{\store_2,\be_2}{i}$ and
  $\dom(\store_1) = \vars(\be_1)$ and $\dom(\store_2) = \vars(\be_2)$.
  Then $\lcod{\store_1\cap\store_2,\be_1[\be_2/x]}{i} = \beta$.
\end{lemma}
\begin{proof}
By straightforward structural induction on $\be$.
\end{proof}

\begin{lemma}[Substitution$*$]
  \label{lemma-substitution}
  If $\sep{\progtt(\prog_2)}{\{ f \}}{X}$ and
  $\prog_1;\prog_2[\be/f]$ is safe with $\vars(\prog_1,\be) \cap
  X = \varnothing$ then
  $\sep{\progtt(\prog_1;\prog_2[\be/f])}{\vars(\be)}{X}$.
\end{lemma}
\begin{proof}
  Suppose on the contrary it was not the case that
  $\sep{\progtt(\prog_1;\prog_2[\be/f])}{\vars(\be)}{X}$.
  Then by Lemma \ref{lemma-sepjoin} there exists $\store_1,\store_2
  \in \runs(\prog_1;\prog_2[\be/f])$ with no $\store \in
  \runs(\prog_1;\prog_2[\be/f])$ such that $\store^1_{\vars(\be)}
  \cap \store^2_X \subseteq \store$.  But also by
  assumption and Lemma \ref{lemma-sepjoin} for any $\beta$ there
  exists $\store' \in \runs(\prog_2)$ with $\{ f \mapsto \beta \} \cap
  \store^2_X \subseteq \store'$. So in particular, we
  have $\{ f \mapsto \lcod{\store^1_{\vars(\be)},\be}{i} \} \cap
  \store^2_X \subseteq \store'$ for any $i$. This, the
  assumption $\vars(\prog_1,\be) \cap \vars(\prog_2) = \varnothing$,
  and application of Lemma \ref{lemma-presub} leads to the consequence
  that there exists $\store \in \runs(\prog_1;\prog_2[\be/f])$ with
  $\store \supseteq \store^1_{\vars(\be)} \cap
  \store^2_X$ given the assumption $\vars(\prog_1,\be)
  \cap \vars(\prog_2) = \varnothing$, which is a contradiction.
\end{proof}

Now we can prove that certified gates preserve the desired invariant
during execution in arbitrary circuit contexts.
We formulate this in terms of pre-~and post-conditions of gate
evaluation.
\begin{lemma}[GMW Gate Invariant]
  \label{lemma-gmw-preservation}
  Given well-typed $\prog_1;E[\mathit{gate}(\ttt{g},v_1,v_2)]$ for certified $\mathit{gate}$ where:
  $$
  \config{\prog_1}{E[\mathit{gate}(\ttt{g},v_1,v_2)]} \redxs \config{\prog_1;\prog_2}{E[v]}
  $$
  If we have the following preconditions, where $\iov(\prog_1) = S \cup V$:
  \begin{enumerate}[\hspace{5mm}i.]
  \item $\sep{\progd(\prog_1)}{S_{\{2\}}}{V_{\{1\}}}$
  \item $\sep{\progd(\prog_1)}{S_{\{1\}}}{V_{\{2\}}}$
  \end{enumerate}
  then we have as a postconditions, where $\iov(\prog_1;\prog_2) = S \cup V'$:
  \begin{enumerate}[\hspace{5mm}i.]
  \item $\sep{\progd(\prog_1;\prog_2)}{S_{\{2\}}}{V'_{\{1\}}}$
  \item $\sep{\progd(\prog_1;\prog_2)}{S_{\{1\}}}{V'_{\{2\}}}$
  \end{enumerate}
\end{lemma}

\begin{proof}
  Given well-typedness of $\prog_1;E[\mathit{gate}(\ttt{g},v_1,v_2)]$, we have for some $\ttt{g1}$ and $\ttt{g2}$:
  $$
  \begin{array}{lcl}
   v_1 &=& \ttt{\{c1 = v[1,g1out]; c2 = v[2,g1out]\}}\\
   v_2 &=& \ttt{\{c1 = v[1,g2out]; c2 = v[2,g2out]\}}\\
    v &=& \ttt{\{c1 = v[1,gout]; c2 = v[2,gout]\}}
  \end{array}
  $$
  Let $\prog$ be as defined in Definition \ref{definition-gmwgate-certification}.
  We observe:
  $$
  {\footnotesize
    \begin{array}{c}
      \prog_2 = \\
      \prog[\ttt{v[1,g1out]}/\ttt{flip[1,s11]}][\ttt{v[2,g1out]}/\ttt{flip[2,s12]}][\ttt{v[1,g2out]}/\ttt{flip[1,s21]}][\ttt{v[2,g2out]}/\ttt{flip[2,s22]}]
    \end{array}
  }
  $$
  Now, the assumption of well-typedness also assures that $\ttt{g}$
  has not been previously used as a gate identifier in $\prog_1$,
  since otherwise the views $\ttt{v[1,gout]}$ and $\ttt{v[2,gout]}$
  would have been previously defined. This ensures $\vars(\prog_1,x)
  \cap \vars(\prog) = \varnothing$ for:
  $$x \in \{ \ttt{v[1,g1out]}, \ttt{v[2,g1out]}, \ttt{v[1,g2out]}, \ttt{v[2,g2out]}\}$$
  given condition (iii) of Definition \ref{definition-gmwgate-certification}.
  Thus by Lemma \ref{lemma-substitution} we have:
  $$\sep{\progtt(\prog_1;\prog_2)}{\ttt{\{v[1,g1out],v[1,g2out],v[2,g1out],v[2,g2out]\}}}{\ttt{\{v[1,gout]\}}}$$
  and
  $$\sep{\progtt(\prog_1;\prog_2)}{\ttt{\{v[1,g1out],v[1,g2out],v[2,g1out],v[2,g2out]\}}}{\ttt{\{v[2,gout]\}}}$$
  Now, by Lemmas \ref{lemma-noninterference} and \ref{lemma-separation} and condition (iii) of Definition
  \ref{definition-gmwgate-certification} we then have:
  $$\sep{\progtt(\prog_1;\prog_2)}{(S_{\{1\}} \cup V_{\{2\}})}{\ttt{\{v[2,gout]\}}} \quad \text{and} \quad
  \sep{\progtt(\prog_1;\prog_2)}{(S_{\{2\}} \cup V_{\{1\}})}{\ttt{\{v[1,gout]\}}}$$
  so by precondition assumptions and Lemma \ref{lemma-separation} we have:
  $$\sep{\progtt(\prog_1;\prog_2)}{S_{\{1\}}}{V'_{\{2\}}} \quad \text{and} \quad
  \sep{\progtt(\prog_1;\prog_2)}{S_{\{2\}}}{V'_{\{1\}}}$$
\end{proof}

Here we show that encoding also preserves the desired invariant.
The result here is more straightforward than it is for gates, because
the variables used during encoding are guaranteed to be distinct from
the rest of the program.
\begin{lemma}[GMW Encode Invariant]
  \label{lemma-gmw-encode}
  Given well-typed $\prog_1;E[\mathit{encode}\ttt{(s1,s2)}]$ for certified $\mathit{encode}$, let:
  $$
  \config{\prog_1}{E[\mathit{encode}(\ttt{s1},\ttt{s2})]} \redxs
  \config{\prog_1;\prog_2}{E[v]}
  $$
  If we have the following preconditions, where $\iov(\prog_1) = S' \cup V$:
  \begin{enumerate}[\hspace{5mm}i.]
  \item $\sep{\progd(\prog_1)}{S_{\{2\}}}{V_{\{1\}}}$
  \item $\sep{\progd(\prog_1)}{S_{\{1\}}}{V_{\{2\}}}$
  \end{enumerate}
  then we have as a postconditions, where $\iov(\prog_1;\prog_2) = S' \cup V'$: 
  \begin{enumerate}[\hspace{5mm}i.]
  \item $\sep{\progd(\prog_1;\prog_2)}{S'_{\{2\}}}{V'_{\{1\}}}$
  \item $\sep{\progd(\prog_1;\prog_2)}{S'_{\{1\}}}{V'_{\{2\}}}$
  \end{enumerate}
\end{lemma}

\begin{proof}
  By assumptions of well-typedness we have:
  $$
  \begin{array}{rcl}
    v &=&  \ttt{\{shares1 = }v_1\ttt{; shares2 = }v_2\ttt{\}}\\
    v_1 &=& \ttt{\{ c1 = v[1,s1out]; c2 = v[2,s1out] \}}\\
    v_2 &=& \ttt{\{ c1 = v[1,s2out]; c2 = v[2,s2out] \}}
  \end{array}
  $$
  And by Definition \ref{definition-gmwencode-certification} we have:
  \begin{mathpar}
    \sep{\progtt(\prog)}{\ttt{\{s[1,s1]\}}}{\ttt{\{v[2,s1out],v[2,s2out]\}}}
    
    \sep{\progtt(\prog)}{\ttt{\{s[2,s2]\}}}{\ttt{\{v[1,s1out],v[1,s2out]\}}}
  \end{mathpar}
  Since we assume that secrets are in uniform and independent marginal
  distributions a priori, and $\vars(\prog_1) \cap \vars(\prog_2) =
  \varnothing$ by condition (iii) of Definition 
  \ref{definition-gmwencode-certification} and assumptions of well-typedness, 
  the result follows by preconditions and Lemma \ref{lemma-separation}.
\end{proof}

On the basis of the preceding we can prove our main result as
follows. Note that we assume a normal form of programs where the last
instruction is a decoding and public reveal of the output.
\begin{theorem}
  \label{theorem-gmw-NIMO}
  If $\eassign{\outv}{\ttt{decode}(e)}$ is a well-typed GMW circuit
  definition using certified components and
  $\config{\varnothing}{\eassign{\outv}{\ttt{decode}(e)}} \redxs
  \config{\prog}{\varnothing}$, then $\NIMO(\prog)$.
\end{theorem}
\begin{proof}
  Let $\iov(\prog) = S \cup (V \cup \outv)$. By definition of $\ttt{decode}$
  and assumptions of well-typedness we have for some $\ttt{g}$:
  $$
  \config{\varnothing}{\eassign{\outv}{\ttt{decode}(e)}} \redxs
  \config{\prog;\eassign{\outv}{\ttt{v[1,gout] xor v[2,gout]}}}{\varnothing}
  $$
  where by Lemmas \ref{lemma-gmw-encode} and \ref{lemma-gmw-preservation} we
  have $\sep{\progd(\prog)}{S_{\{1\}}}{V_{\{2\}}}$ and
  $\sep{\progd(\prog)}{S_{\{2\}}}{V_{\{1\}}}$.
  Let $\prog' \defeq \prog;\eassign{\outv}{\ttt{v[1,gout] xor v[2,gout]}}$. The
  preceding then implies
  $\sep{\progd(\prog')}{S_{\{1\}}}{V_{\{2\}}}$ and
  $\sep{\progd(\prog')}{S_{\{2\}}}{V_{\{1\}}}$,
  so that for all $\beta$:
  $$\sep{\condd{\progd(\prog')}{S_{\{1\}} \cup V_{\{2\}}}{\{ \outv \mapsto \beta \}}}{S_{\{1\}}}{V_{\{2\}}}
  \quad \text{and} \quad
    \sep{\condd{\progd(\prog')}{S_{\{2\}} \cup V_{\{1\}}}{\{ \outv \mapsto \beta \}}}{S_{\{2\}}}{V_{\{1\}}}$$
  thus for all $\store \in \mems(V_2 \cup \{ \outv \})$:
  $$\condd{\progd(\prog')}{S_{\{1\}}}{\store}
  = \condd{\progd(\prog')}{S_{\{1\}}}{\store_{\{\outv\}}}$$
  and for all $\store \in \mems(V_1 \cup \{ \outv \})$:
  $$\condd{\progd(\prog')}{S_{\{2\}}}{\store}
  = \condd{\progd(\prog')}{S_{\{2\}}}{\store_{\{\outv\}}}$$
  establishing the result by Lemma \ref{lemma-nimo} and Definition \ref{definition-NIMO}. 
\end{proof}

\subsection{Compositional Properties of YGC}
\label{section-composition-ygc}

\begin{fpfig}[t]{Yao's Garbled Circuits, selected auxiliary functions.}{fig-ygc-aux}
{\footnotesize
\begin{verbatimtab}
  keygen(gid, b1, b2) { select4(b1,b2,H[gid || 1],H[gid || 2],H[gid || 3],H[gid || 4]) }
  
  keysgen(gid, b1, b2)
  {
    let k11 = keygen(gid,b1,b2) in
    let k10 = keygen(gid,b1,not b2) in
    let k01 = keygen(gid,not b1,b2) in
    let k00 = keygen(gid,not b1,not b2) in
    {k11 = k11; k10 = k10; k01 = k01; k00 = k00}
  }
  
  andtable(keys, bt, ap, bp)
  {
    let r11 = (keys.k11 xor bt) in 
    let r10 = (keys.k10 xor (not bt)) in
    let r01 = (keys.k01 xor (not bt)) in
    let r00 = (keys.k00 xor (not bt)) in
    permute4(ap,bp,r11,r10,r01,r00)
  }
  
  sharetable(gid, tid, table)
  {   
    v[1, gate: || gid || tid || 1] := table.v1;
    v[1, gate: || gid || tid || 2] := table.v2;
    v[1, gate: || gid || tid || 3] := table.v3;
    v[1, gate: || gid || tid || 4] := table.v4
  }

  owl(gid) {  { k = flip[2,gate: || gid || .k]; p = flip[2,gate: || gid || .p] }  }
\end{verbatimtab}
}
\end{fpfig}

\begin{fpfig}[t]{Yao's Garbled Circuits, garbled gates and evaluation code.}{fig-ygc-gates}
{\footnotesize
\begin{verbatimtab}
  garbledecode(gid)    
  {
    let wl = owl(gid) in
    let r1 = wl.k xor true in
    let r0 = (not wl.k) xor false in
    v[1,OUTtt1] := select(wl.p,r1,r0);
    v[1,OUTtt2] := select(not wl.p,r1,r0)
  }
  
  evaldecode(wv, p) { wv.k xor select[wv.p,v[1,OUTtt1],v[1,OUTtt2]] }
  
  evalgate(gid, wva, wvb)  
  {
    let k = keygen(gid,wva.k,wvb.k) in
    let ct = select4(wva.p,wvb.p,
               v[1,gid || tt1],v[1,gid || tt2],v[1,gid || tt3],v[1,gid || tt4]) in
    let cp = select4(wva.p,wvb.p,
               v[1,gid || pt1],v[1,gid || pt2],v[1,gid || pt3],v[1,gid || pt4]) in
    { k = k xor ct; p = k xor cp }
  }
  
  andgate(gid, ga, gb) 
  {
    let wla = owl(ga) in
    let wlb = owl(gb) in
    let wlc = owl(gid) in
    let keys = keysgen(gid,wla.k,wlb.k) in
    sharetable(gid,tt,andtable(keys,wlc.k,wla.p,wlb.p));
    sharetable(gid,pt,andtable(keys,wlc.p,wla.p,wlb.p))
  }

  encode(sa, sb)
  {
    let owl1 = owl(sa) in
    let owl2 = owl(sb) in
    v[1,gate: || sa || 1.k] := OT(s[1,sa],owl1.k,(not owl1.k));
    v[1,gate: || sa || 1.p] := OT(s[1,sa],owl1.p,(not owl1.p));
    v[1,gate: || sb || 2.k] := select(s[2,sb],owl2.k,(not owl2.k));
    v[1,gate: || sb || 2.p] := select(s[2,sb],owl2.p,(not owl2.p));
    { wv1 = { k = v[1,gate: || sa || 1.k]; p = v[1,gate: || sa || 1.p] };
      wv2 = { k = v[1,gate: || sb || 2.k]; p = v[1,gate: || sb || 2.p] } }
  }
\end{verbatimtab}
}
\end{fpfig}

%  andtable(keys, bt, ap, bp)
%  {
%    let r11 = (keys.k11 xor bt) in 
%    let r10 = (keys.k10 xor (not bt)) in
%    let r01 = (keys.k01 xor (not bt)) in
%    let r00 = (keys.k00 xor (not bt)) in
%    permute4(ap,bp,r11,r10,r01,r00)
%  }

  
%  encode(gid, wla,wlb)
%  {
%    let wla = { k = flip[2,fwl1]; p = flip[2,pwl1] } in
%    let wlb = { k = flip[2,fwl2]; p = flip[2,pwl2] } in
%    
%    { wv1 = { k = OT[s[1,0],wla.k,(not wla.k)]; p = OT[s[1,0],wla.p,(not wla.p)]}; 
%      wv2 = { k = select[s[2,0],wlb.k,(not wlb.k)]; p = select[s[2,0],wlb.p,(not wlb.p)] } }
%  }


In Figures \ref{fig-ygc-aux} and \ref{fig-ygc-gates} we define a
codebase for Yao's garbled circuits (YGC). This definition follows the
\emph{point-and-permute} method described in \cite{evans2018pragmatic}
and elsewhere, to which the reader is referred for more in-depth discussion.
In this implementation client 2 is the \emph{garbler} and
client 1 is the \emph{evaluator}. The garbler builds the garbled
tables and shares them with the evaluator, who then evaluates
the gate in an oblivious fashion until the final public output is
generated through decryption. This definition is well-typed,
with input type annotations for top-level functions listed in
Figure \ref{fig-ygc-types}. Well-typed programs using these
libraries are therefore guaranteed to yield safe $\minifed$
programs. 

\emph{Wire labels} are fundamental to YGC, and essentially represent
gate output values in an encrypted form. In our definition, wire
labels are represented by records $\ttt{\{ k = }\beta_1\ttt{; p =
}\beta_2\ttt{ \}}$, where $\ttt{k}$ is the \emph{key bit} and
$\ttt{p}$ is the \emph{pointer bit}, and $\beta_1$ and $\beta_2$ are
flips. Flips in each output wire label are owned by the garbler and
are unique per gate by definition of their identifying string, and the
representation of $0$ is the negation of $1$. For example, here is the
representation of 1 and 0 respectively in the output wire label for a
hypothetical gate 6:
\begin{mathpar}
  \ttt{\{ k = flip[2,gate:6.k]; p =  flip[2,gate:6.p]] \}}
    
  \ttt{\{ k = not flip[2,gate:6.k]; p =  not flip[2,gate:6.p]] \}}
\end{mathpar}
The pointer bits in wire labels are used to select permuted rows in
table garblings. The key bits are used to identify a unique key for
table row in each garbled gate. Intuitively, if $\beta_1$ and
$\beta_2$ are either key or pointer bits encoding 1 on two input wire
labels to a binary gate, rows and keys in the gate are enumerated in
the order:
$$
\neg\beta_1\neg\beta_2,\ \neg\beta_1\beta_2,\ \beta_1\neg\beta_2,\ \beta_1\beta_2
$$

In our implementation, gates are wired together using gate
identifiers, which are strings $w$. Top-level functionality in Figures
\ref{fig-ygc-aux} and \ref{fig-ygc-gates} includes the following:
\begin{itemize}
\item \ttt{andgate}: This defines a subprotocol for the garbler
  to define a garbled gate $\ttt{gid}$ with input wires from gates
  $\ttt{ga}$ and $\ttt{gb}$. The garbler generates keys and garbles
  the rows in YGC fashion, them with client $1$ in
  views in a standard form. For example, the view for
  a hypothetical gate 6, row 2 garbled truth table is $\ttt{v[1,gate:6tt2]}$.
  We note that garbled gates of other binary operators can be obtained with
  replacement of $\ttt{andtable}$ with appropriate garbled table definitions. 
\item \ttt{evalgate}: This defines a subprotocol for the evaluator to
  evaluate gate $\ttt{gid}$ given input wire values $\ttt{wva}$ and
  $\ttt{wvb}$.
\item \ttt{garbledecode} and \ttt{evaldecode}: The former function
  defines the garbler's protocol for encrypting the circuit
  output from final gate $\ttt{gid}$, and the latter defines
  the evaluator's output decryption protocol.
\item \ttt{encode}: This defines the initial phase of the protocol,
  where the evaluator receives the wire value from their own
  secret $\sx{1}{sa}$ via $\ttt{OT}$, and the garbler communicates
  the wire value for their own secret $\sx{2}{sb}$ directly.
\end{itemize}
\begin{example}
  \label{example-andcircuit}
The following program uses our YGC library to define
a circuit with a single and gate and input secrets $\ttt{s1}$ and
$\ttt{s2}$ from client's 1 and 2 respectively. 
\begin{verbatimtab}
  andgate(0,s1,s2);
  garbledecode(0);
  let secrets = encode(s1,s2) in
  v[0,output] := decode(evalgate(0, secrets.wv1, secrets.wv2))
\end{verbatimtab}
\end{example}
We have verified passive security of the $\fedprot$ protocol
generated by this and other small circuits using the
technique described in Lemma \ref{lemma-bruteforce-nimo}.
But large circuits with thousands of gates would be
intractable to verify with this method. In the next Section
we discuss compositional methods to address this issue.

\subsubsection{Gate Certification}
\label{section-ygc-certification}

Our certification method relies on a definition of compositional units comprising
complete related subprotocols-- i.e., both the garbler's
construction of the gates and the evaluator's evaluation of them. We
therefore make the following definitions. However, we note that our
formal results are applicable to the YGC style where the garbler
shares the entire circuit prior to evaluation.
\begin{definition}
  \label{ygc-modules}
  We define a composable internal \ttt{and} gate sharing and evaluation function
  called $\ttt{andgg}$, and an output decoder gate share and evaluation function called
  $\ttt{decode}$, as follows. 
\begin{verbatimtab}
  andgg(g, ga, gb, wva, wvb) { andgate(g, ga, gb); evalgate(g, wva, wvb) }
  decode(g, v) { garbledecode(g); evaldecode(v) }
\end{verbatimtab}
\end{definition}

Our certification techniques are defined generally wrt implementations
of input encoding, decoding, and internal gates. In the following we
will refer to arbitrary decode, encode, and gate functions with the
restriction that any function in each of these categories has the same
valid type signature as $\ttt{decode}$, $\ttt{encode}$, and
$\ttt{andgg}$ respectively.

It is necessary to define notions of correlation with wire
labels, since wire values flow into and out of gates in either
positive of negative correlation with them.
\begin{definition}
  \label{definition-gc}
  We write $\gc{\prog}{\{ \ttt{k = }\be_{k}; \ttt{p = }\be_{p} \}}{g}$ iff one of the
  following hold:
  \begin{enumerate}[\hspace{5mm}i.]
  \item $\vc{\progtt(\prog,\be_k)}{\itv}{\ttt{flip[2,gate:}g\ttt{.k]}}\quad \text{and} \quad
    \vc{\progtt(\prog,\be_p)}{\itv}{\ttt{flip[2,gate:}g\ttt{.p]}}$
  \item $\vc{\progtt(\prog,\enot\ \be_k)}{\itv}{\ttt{flip[2,gate:}g\ttt{.k]}}\quad \text{and}\quad
    \vc{\progtt(\prog,\enot\ \be_p)}{\itv}{\ttt{flip[2,gate:}g\ttt{.p]}}$
  \end{enumerate}
\end{definition}
Thus, if a value is in correlation with a wire label, it is either
correlated with the wire label or with the inverse of both the key and
pointer bits. For subsequent discussion we make the following addition
to the codebase:
$$
\ttt{invert(\{ k = bk; p = bp \}) \{ \{ k = not bk; p = not bp \} \}}
$$
Gate certification establishes separation of input labels from the
gate table and output value, and correlation of the output with
the output wire label under any input wire value combination.
\begin{definition}[YGC Gate Certification]
  \label{definition-ygcgate-certification}
  Let $\mathit{gate}$ be a gate function.
  For $1 \le i \le 4$  let
  $
  \config{\varnothing}{e_i}\redxs\config{\prog}{v_i}
  $
  where:
  \begin{enumerate}[\hspace{5mm}$e_1 \defeq$]
    \item $\mathit{gate}\ttt{(c,a,b,owl(a),owl(b));}$
    \item $\mathit{gate}\ttt{(c,a,b,owl(a),invert(owl(b)));}$
    \item $\mathit{gate}\ttt{(c,a,b,invert(owl(a)),owl(b));}$
    \item $\mathit{gate}\ttt{(c,a,b,invert(owl(a)),invert(owl(b)))}$
  \end{enumerate}
  Define:
  $${\small X \defeq \ttt{\{ flip[2,gate:a.k], flip[2,gate:a.p], flip[2,gate:b.k], flip[2,gate:b.p] \}}}$$
  Then $\mathit{gate}$ is \emph{certified} iff the following conditions hold:
  \begin{enumerate}[\hspace{5mm}i.]
  \item $\sep{\progtt(\prog)}{X}{\vdefs(\prog)}$
  \item For all $1 \le i \le 4$, $\gc{\prog}{v_i}{\ttt{c}}$
  \end{enumerate}
\end{definition}
Encode certification establishes correlation of the encoded input values with
the secret wire labels, and separation of client 2's secret from client 1's views. 
\begin{definition}[YGC Encode Certification]
  \label{definition-ygcencode-certification}
  Let $\mathit{encode}$ be an encoding function, and let:
  $$
  \config{\varnothing}{\ttt{encode(s1,s2)}} \redxs
  \config{\prog}{\ttt{\{wv1=}v_1\ttt{;wv2=}v_2\ttt{\}}}
  $$
  Then $\mathit{encode}$ is \emph{certified} iff each of the following conditions hold:
  \begin{enumerate}[\hspace{5mm}i.]
  \item $\gc{\prog}{v_1}{\ttt{s1}} \quad \text{and} \quad \gc{\prog}{v_2}{\ttt{s2}}$
  \item $\sep{\progd(\prog)}{\{\ttt{s[1,s1]\}}}{\vdefs(\prog)_{\{2\}}}$
  \item $\sep{\progd(\prog)}{\{\ttt{s[2,s2]\}}}{\vdefs(\prog)_{\{1\}}}$    
  \item All $x \in \vars(\prog)$ are distinguished by $\ttt{in1}$ or $\ttt{in2}$. 
  \end{enumerate}
\end{definition}
Decode certification establishes separation of the input wire
label from the decoding tables under any input value condition. 
\begin{definition}[YGC Decode Certification]
  \label{definition-ygdecode-certification}
  Let $\mathit{decode}$ be a decoding function, and let:
  $$
  \config{\varnothing}{\mathit{decode}(\ttt{c}, \ttt{owl(c)})}
  \redxs \config{\prog}{\be}
  %\quad \text{and} \quad
  %\config{\varnothing}{\mathit{decode}(\ttt{c}, \ttt{invert(owl(c))})}
  %\redxs \config{\prog}{\be_0}
  $$
  Then $\mathit{decode}$ is certified iff:
  $$
  \sep{\progtt(\prog)}{\ttt{\{flip[2,gate:c.k],flip[2,gate:c.p]\}}}{\vdefs(\prog)}
  $$
  %Then $\mathit{decode}$ is certified iff both of the following conditions hold:
  %\begin{enumerate}[\hspace{5mm}i.]
  %\item $\progd(\prog_1,\be_1)(\{ \itv \mapsto 1 \}) = 1$ and
  %  $\progd(\prog_0,\be_0)(\{ \itv \mapsto 0 \}) = 1$
  %\item $\sep{\progtt(\prog_i)}{\ttt{\{flip[2,gate:c.k],flip[2,gate:c.p]\}}}{\vdefs(\prog_i)}$
  %\end{enumerate}
\end{definition}
We have certified each of the YGC components detailed above. 
\begin{lemma}
  \label{lemma-ygc-certification}
  Each of $\ttt{decode}$, $\ttt{encode}$, and $\ttt{andgg}$ are certified.
\end{lemma}

\subsubsection{Gate Invariants and Circuit Security}
\label{section-composition-metatheory}

To demonstrate correctness of certification, we need to show that
isolated component certificates preserve relevant properties
when integrated into larger programs. 
%First, we show that if a flip $f$ is independent from the views
%in a protocol (as is the case with input labels used to
%encode a garbled table), then separation from views that
%use $f$ (e.g., in an output label) is preserved. 
%\begin{lemma}[Wire Framing]
%  \label{lemma-wire-framing}
%  If $\sep{\progtt(\prog)}{\{f\}}{\vdefs(\prog)}$ for flip $f$ and
%  $\vars(\prog) \cap \vars(\prog') = \{ f \}$ then
%  $\sep{\progtt(\prog';\prog)}{\vdefs(\prog')}{\vdefs(\prog)}$.
%\end{lemma}
%Next,
First we establish another substitution result, showing that
correlations (e.g., between output wire values and labels) are
preserved by substitution of correlated values for flips (e.g.,
between input wire values and labels).
\begin{lemma}[Substitution$\sim$]
  \label{lemma-substitution-sim}
  If $\vc{\progtt(\prog_2,\be)}{\itv}{f}$ and $\vc{\progtt(\prog_1,\be')}{\itv}{f'}$
  and $\vars(\be') \cap (\vars(\be) - \{ f \}) = \varnothing$
  then $\vc{\progtt(\prog_1;\prog_2,\be[\be'/f'])}{\itv}{f}$.
\end{lemma}
Now we can show how to splice gates into arbitrary circuits in an
invariant-preserving manner, assuming that outputs are not wired to
multiple inputs \cite{tate2003garbled,nieminen2023breaking}.
\begin{lemma}[YGC Gate Invariant]
  \label{lemma-ygc-preservation}
Given well-typed $\prog_1;E[\mathit{gate}(g,g_1,g_2,v_1,v_2)]$ and certified $\mathit{gate}$ where:
$$
\config{\prog_1}{E[\mathit{gate}(g,g_1,g_2,v_1,v_2)]} \redxs \config{\prog_1;\prog_2}{E[v]}
$$
If the following preconditions hold where $\iov(\prog_1) = S \cup V$:
\begin{enumerate}
\item $\{ g_1 \} \cap \{ g_2 \} \cap \wired(\prog_1) = \varnothing$ and $g \not\in \prog_1$
\item $\gc{\prog_1}{v_1}{g_1}$ and $\gc{\prog_1}{v_2}{g_2}$
\item $\sep{\progd(\prog_1)}{S_{\{1\}}}{V_{\{2\}}}$
\item $\sep{\progd(\prog_1)}{S_{\{2\}}}{V_{\{1\}}}$
\end{enumerate}
then we have as postconditions where $\iov(\prog_1) = S \cup V'$:
\begin{enumerate}
\item $\gc{\prog_1;\prog_2}{v}{g_2}$
\item $\sep{\progd(\prog_1,\prog_2)}{S_{\{1\}}}{V'_{\{2\}}}$
\item $\sep{\progd(\prog_1,\prog_2)}{S_{\{2\}}}{V'_{\{1\}}}$
\end{enumerate}
\end{lemma}
\begin{proof}
  Given preconditions we have $v_1 =  \{ \ttt{k = }\be^1_{k}; \ttt{p = }\be^1_{p} \}$ and
  $v_2 = \{ \ttt{k = }\be^2_{k}; \ttt{p = }\be^2_{p} \}$ for some
  $\be^1_{k}$,$\be^2_{k}$,$\be^1_{p}$,$\be^2_{p}$ with correlations as per Definition
  \ref{definition-gc}.
  %\begin{eqnarray*}
  %  v_1 &=& \{ \ttt{k = }\be^1_{k}; \ttt{p = }\be^1_{p} \} \\
  %  v_2 &=& \{ \ttt{k = }\be^2_{k}; \ttt{p = }\be^2_{p} \}
  %\end{eqnarray*}
  Let $\prog$ be as defined in Definition \ref{definition-ygcgate-certification}.
  We observe:
  \begin{eqnarray*}
    &\prog_2 = \\
    &{\small \prog[\be^1_{k}/\ttt{flip[2,gate:a.k]}][\be^1_{p}/\ttt{flip[2,gate:a.p]}][\be^2_{k}/\ttt{flip[2,gate:b.k]}][\be^1_{p}/\ttt{flip[2,gate:b.p]}]}
  \end{eqnarray*}
  Given that $g_1$ and $g_2$ are distinct and not wired in $\prog_1$
  we are assured that $\ttt{owl}(g_1)$ and $\ttt{owl}(g_1)$ are in
  independent uniform distributions, and given that $g \not\in
  \prog_1$ we are assured that $\ttt{owl}(g)$ contains entirely fresh
  flips. Thus by condition (i) of Definition \ref{definition-ygcgate-certification} and
  Lemma \ref{lemma-substitution} we have:
  $$
  \sep{\prog_1;\prog_2}{\vars(\be^1_{k},\be^2_{k},\be^1_{p},\be^2_{p})}{\vdefs(\prog_2)}
  $$
  Thus by Lemmas \ref{lemma-noninterference} and \ref{lemma-separation} we establish
  postconditions (ii) and (iii).

  Also since $v_1$ and $v_2$ are correlated either positively or negatively with
  $\ttt{owl}(g_1)$ and $\ttt{owl}(g_2)$ respectively by precondition (ii),
  by Definition \ref{definition-gc}, precondition (i), and Lemma \ref{lemma-substitution-sim}
  we establish postcondition (i), since Definition \ref{definition-ygcgate-certification}
  requires gate output correlation with $\ttt{owl}(g)$ given any input valence conditions.
\end{proof}

\begin{lemma}[YGC Encode Invariant]
  \label{lemma-ygc-encode}
Given well-typed $\prog_1;E[\mathit{encode}(s_1,s_2)]$ and certified $\mathit{encode}$ where 
$$
\config{\prog_1}{E[\ttt{encode}(s_1,s_2)]} \redxs \config{\prog_1;\prog_2}{E[\ttt{\{wv1=}v_1\ttt{;wv2=}v_2\ttt{\}}]}
$$
If the following preconditions hold where $\iov(\prog_1) = S \cup V$:
\begin{enumerate}
\item $\sep{\progd(\prog_1)}{S_{\{1\}}}{V_{\{2\}}}$
\item $\sep{\progd(\prog_1)}{S_{\{2\}}}{V_{\{1\}}}$
\end{enumerate}
then we have as postconditions where $\iov(\prog_1) = S' \cup V'$:
\begin{enumerate}
\item $\gc{\prog_1;\prog_2}{v_1}{s_1}$ and $\gc{\prog_1}{v_2}{s_2}$
\item $\sep{\progd(\prog_1;\prog_2)}{S'_{\{1\}}}{V'_{\{2\}}}$
\item $\sep{\progd(\prog_1;\prog_2)}{S'_{\{2\}}}{V'_{\{1\}}}$
\end{enumerate}
\end{lemma}
\begin{proof}
  By Definition \ref{definition-ygcencode-certification} we have:
  \begin{mathpar}
    \sep{\progtt(\prog_2)}{\ttt{\{s[1,s1]\}}}{\vdefs(\prog_2)_{\{2\}}}
    
    \sep{\progtt(\prog_2)}{\ttt{\{s[2,s2]\}}}{\vdefs(\prog_2)_{\{1\}}}
  \end{mathpar}
  Since we assume that secrets are in uniform and independent marginal
  distributions a priori, and $\vars(\prog_1) \cap \vars(\prog_2) =
  \varnothing$ by condition (iv) of Definition
  \ref{definition-ygcencode-certification} and assumptions of
  well-typedness, conditions (ii-iii) follow by Lemmas
  \ref{lemma-noninterference} and \ref{lemma-separation}. Also by
  Definition \ref{definition-ygcencode-certification} we have
  $\gc{\prog_2}{v_1}{s_1}$ and $\gc{\prog_2}{v_1}{s_1}$, so also
  $\gc{\prog_1;\prog_2}{v_1}{s_1}$ and
  $\gc{\prog_1;\prog_2}{v_1}{s_1}$ by condition (iv) of Definition
  \ref{definition-ygcencode-certification} and Lemmas
  \ref{lemma-noninterference} and \ref{lemma-separation}.
\end{proof}

Given the above, we can now establish that any circuit built with
certified components is passive secure.  Note that we assume a normal
form of programs where the last instruction is a decoding and public
reveal of the output, and we require that no gate is wired more than once.
\cnote{There is another subtlety here, that gates are wired correctly
  in the sense that if the garbler generates a table with input wire label
  $g_1$ then only $g_1$ is wired to that input. I'll modify the
  typing to enforce this.}
\begin{theorem}
  \label{theorem-ygc-NIMO}
  If $\eassign{\outv}{\ttt{decode}(g,e)}$ is a well-typed YGC circuit
  definition using certified components where no gate output is wired
  more than once, and
  $\config{\varnothing}{\eassign{\outv}{\ttt{decode}(g,e)}} \redxs
  \config{\prog}{\varnothing}$, then $\NIMO(\prog)$.
\end{theorem}

\begin{proof}
  By assumptions of well-typedness we
  have for some $\be_k$, $\be_p$, and $\be$:
  $$
  \config{\varnothing}{\eassign{\outv}{\ttt{decode}(g,e)}} \redxs
  \config{\prog_1}{\eassign{\outv}{\ttt{decode}(g,\{ \ttt{k = }\be_k;  \ttt{p = }\be_p\})}}
  \redxs \config{\prog_1;\prog_2}{\eassign{\outv}{\be}}
  $$
  Let $\iov(\prog_1) = S \cup V$.
  By Lemmas \ref{lemma-ygc-encode} and \ref{lemma-ygc-preservation} we
  have $\sep{\progd(\prog_1)}{S_{\{1\}}}{V_{\{2\}}}$ and
  $\sep{\progd(\prog_1)}{S_{\{2\}}}{V_{\{1\}}}$.
  Let $\prog$ be as defined in Definition \ref{definition-ygdecode-certification}.
  We observe:
  \begin{eqnarray*}
    & \prog_2 = \\ 
    & \prog[\be_k/\ttt{\{flip[2,gate:c.k]}][\be_p/\ttt{flip[2,gate:c.p]}]
  \end{eqnarray*}
  so also by  Definition \ref{definition-ygdecode-certification} and
  Lemma \ref{lemma-substitution} we have:
  $$
  \sep{\progtt(\prog_1;\prog_2)}{\vars(\be_k,\be_p)}{\vdefs(\prog_2)}
  $$
  so, letting $\iov(\prog_1;\prog_2) = S \cup V'$ by
  Lemmas \ref{lemma-noninterference} \ref{lemma-separation} we have
  $\sep{\progd(\prog_1;\prog_2)}{S_{\{1\}}}{V'_{\{2\}}}$ and
  $\sep{\progd(\prog_1;\prog_2)}{S_{\{2\}}}{V'_{\{1\}}}$.
  Thus, letting $\prog' \defeq (\prog_1;\prog_2;\eassign{\outv}{\be})$, for all $\beta$:
  $$\sep{\condd{\progd(\prog')}{S_{\{1\}} \cup V_{\{2\}}}{\{ \outv \mapsto \beta \}}}{S_{\{1\}}}{V_{\{2\}}}
  \quad \text{and} \quad
    \sep{\condd{\progd(\prog')}{S_{\{2\}} \cup V_{\{1\}}}{\{ \outv \mapsto \beta \}}}{S_{\{2\}}}{V_{\{1\}}}$$
  thus for all $\store \in \mems(V_2 \cup \{ \outv \})$:
  $$\condd{\progd(\prog')}{S_{\{1\}}}{\store}
  = \condd{\progd(\prog')}{S_{\{1\}}}{\store_{\{\outv\}}}$$
  and for all $\store \in \mems(V_1 \cup \{ \outv \})$:
  $$\condd{\progd(\prog')}{S_{\{2\}}}{\store}
  = \condd{\progd(\prog')}{S_{\{2\}}}{\store_{\{\outv\}}}$$
  establishing the result by Lemma \ref{lemma-nimo} and Definition \ref{definition-NIMO}.
\end{proof}


\begin{fpfig}[t]{YGC copy gate definitions.}{fig-ygc-copy}
  {\footnotesize
    \begin{verbatimtab}
      sharetab2(gid, tid, k, p, b)
      {
        let r1 = k xor b in
        let r0 = (not k) xor (not b) in
        v[1,gate: || gid || tid || 1] := select(p,r1,r0);
        v[1,gate: || gid || tid || 2] := select(not p,r1,r0);
      }
      
      copygate(ca,cb,g)
      {
        let wl = owl(g) in
        let owl1 = owl(ca) in
        let owl2 = owl(cb) in
        sharetab2(ca,tt,wl.k,wl.p,owl1.k); sharetab2(ca,pt,wl.k,wl.p,owl1.p);
        sharetab2(cb,tt,wl.k,wl.p,owl2.k); sharetab2(cb,pt,wl.k,wl.p,owl2.p)
      }
      
      evalcopy(ca,cb,wv)
      {
        let wv1k = wv.k xor select(wv.p,v[1,gate: || ca || tt1], v[1,gate: || ca || tt2]) in
        let wv1p = wv.k xor select(wv.p,v[1,gate: || ca || pt1], v[1,gate: || ca || pt1]) in
        let wv2k = wv.k xor select(wv.p,v[1,gate: || cb || tt1], v[1,gate: || cb || tt2]) in
        let wv2p = wv.k xor select(wv.p,v[1,gate: || cb || pt1], v[1,gate: || cb || pt1]) in
        { wv1 = { k = wv1k; p = wv1p }; wv2 = { k = wv2k; p = wv2p } }  
      }

      copy(ca,cb,g,wv) { copygate(ca,cb,g); evalcopy(ca,cb,wv) } 
    \end{verbatimtab}
  }
\end{fpfig}

\subsection{Building and Extending Automatically Secure Circuits}
\label{section-composition-copy}

Theorems \ref{theorem-gmw-NIMO} \ref{theorem-ygc-NIMO} means that any
well-typed program using certified components will generate a circuit
that is passive secure. To add an $\ttt{or}$ gate to our YGC library,
for example, we can use $\ttt{andgg}$ as a template, in fact simple
modification of $\ttt{andtable}$ is all that would be needed. Thus,
this library embodies a Fairplay-like language where well-typedness of
metaprograms guarantees safety and security of generated protocols.

The requirements that output wire labels are not used more than once
in YGC reflects a known result, that naive copy gate output is unsound
in YGC- i.e., passive security fails. In fact, we are unable to
automatically verify $\gNIMO$ for $\ttt{andgg}$ without the assumption
that input wire lables are in independent uniform distributions. We
can capture the necessary fix as in Figure \ref{fig-ygc-copy}. The
garbler creates a copy gate with two new output wire labels in
independent uniform distributions (in \ttt{copygate}), and permutes
them using the input wire label pointer after encrypting with the
input wire label keys. The evaluator then recovers the encrypted
output wire values during evaluation (in \ttt{evalcopy}). Given an
$\ttt{orgg}$ gate defined as outlined abotve, we can then use secure
$\ttt{copy}$ as follows:
\begin{verbatimtab}
  let s = encode(s1,s2) in
  let s1c = copy(c1s1,c2s1,s1,s.wv1) in
  let s2c = copy(c1s2,c2s2,s2,s.wv2) in
  let wv1 = andgg(1,c1s1,c1s2,s1c.wv1,s2c.wv1) in
  let wv2 = orgg(2,c2s1,c2s2,s1c.wv2,s2c.wv2) in
  decode(3,andgg(3,1,2,wv1,wv2))
\end{verbatimtab}
These definitions of $\ttt{copy}$ and $\ttt{orgg}$ have been
automatically verified as in Section \ref{section-pre-post} for
$\ttt{andgg}$. In general, any new gate form- such as optimized gates
\cite{XXX}- can be verified automatically in a similar way.

While the copy gate supports secure programming patterns, security
still relies on programmer discipline, i.e., not wiring a gate more
than once. However, we can keep track of wired gates in type effects,
similar to the manner in which we keep track of defined views- type
dependence in our system is able to accurately precisely track
gate identifiers. This is also reminiscent of the use of linearity
in other related systems for enforcing obliviousness \cite{darais2019language}. 



%% This is scratch pad to develop ideas before integration.
%\newcommand{\sx}[2]{\elab{\secret{#1}}{#2}}
\newcommand{\mx}[2]{\elab{\mesg{#1}}{#2}} 
%\newcommand{\px}[2]{\elab{\rvl{#1}}{#2}} 
\newcommand{\rx}[2]{\elab{\flip{#1}}{#2}}
\newcommand{\ox}[2]{\elab{\out{#1}}{#2}}
\newcommand{\signals}{\leadsto}

\newcommand{\tj}[5]{#1,#2 \vdash_{#3} #4 : #5}
\newcommand{\cty}[2]{c(#1,#2)}
\newcommand{\setit}[1]{\{ #1 \}}
\newcommand{\ty}{T}
\newcommand{\eqs}{\mathit{E}}
\newcommand{\toeq}[1]{\lfloor #1 \rfloor}
\newcommand{\autheq}[1]{\phi_{\mathrm{auth}}(#1)}
\newcommand{\upgrade}[1]{\uparrow #1}

\renewcommand{\redx}{\Rightarrow}
\renewcommand{\redxs}{\redx}
\newcommand{\abort}{\bot}
\newcommand{\pre}[1]{\ttt{pre}(#1)}
\newcommand{\post}[1]{\ttt{post}(#1)}
\newcommand{\eqflag}{\mathit{sw}}
\newcommand{\eqon}{\ttt{on}}
\newcommand{\eqoff}{\ttt{off}}
\newcommand{\eqtrans}[1]{\lfloor #1 \rfloor}
\newcommand{\mc}[4]{(#1,#2,#3,#4)}
\newcommand{\cmd}{\instr}

$$
    \begin{array}{rcl@{\hspace{2mm}}r}
      \multicolumn{4}{l}{v \in \mathbb{F}_p,\ w \in \mathrm{String},\ \cid \in \mathrm{Clients} \subset  \mathbb{N} }\\[2mm] %, \bop \in \{ \eand, \eor, \exor \}} \\[2mm]
      \be &::=& \flip{w} \mid \secret{w} \mid \mesg{w} \mid \rvl{w} \mid & \textit{expressions}\\
      & & v \mid \be \fminus \be \mid \be \fplus \be \mid \be \ftimes \be \\[2mm]
      x &::=& \elab{\flip{w}}{\cid} \mid \elab{\secret{w}}{\cid} \mid \elab{\mesg{w}}{\cid} \mid \rvl{w} \mid \out{\cid} & \textit{variables} \\[2mm]
      %& &  \select{\be}{\be}{\be} \mid \ctxt{v}{k} \mid \key{w} \mid \sk{\be}(\be) \mid \pk{\be}{\be}(\be) \mid \pk{\be}{\be} \\[2mm]
      %& &  \select{\fp(\be)}{\be}{\be} \ctxt{v,\be}{k}  \mid \sk{\be}(\be) \mid \pk{\be}{\be}(\be) \mid \pk{\be}{\be} \\[2mm]
      \prog &::=& \eassign{\mesg{w}}{\cid}{\be}{\cid} \mid
      \reveal{w}{e}{\cid} \mid \pubout{\cid}{\be}{\cid} \mid \prog;\prog & \textit{protocols} 
    \end{array}
$$

\bigskip
    
 $$
  %\begin{array}{c@{\hspace{5mm}}c}
  \begin{array}{rcl}
    \lcod{\store, v}{\cid} &=& v\\
    \lcod{\store, \be_1 \fplus \be_2}{\cid} &=& \fcod{\lcod{\store, \be_1}{\cid} \fplus \lcod{\store, \be_2}{\cid}}\\ 
    \lcod{\store, \be_1 \fminus \be_2}{\cid} &=& \fcod{\lcod{\store, \be_1}{\cid} \fminus \lcod{\store, \be_2}{\cid}}\\ 
    \lcod{\store, \be_1 \ftimes \be_2}{\cid} &=& \fcod{\lcod{\store, \be_1}{\cid} \ftimes \lcod{\store, \be_2}{\cid}}\\
  %\end{array} 
  %\begin{array}{rcl}
    \lcod{\store, \flip{w}}{\cid} &=& \store(\elab{\flip{w}}{\cid})\\
    \lcod{\store, \secret{w}}{\cid} &=& \store(\elab{\secret{w}}{\cid})\\
    \lcod{\store, \mesg{w}}{\cid} &=& \store(\elab{\mesg{w}}{\cid})\\
    \lcod{\store, \rvl{w}}{\cid} &=& \store(\rvl{w})\\
    %\lcod{\store, f\,\be_1\,\cdots\, \be_n}{\cid} &=& \delta(f,\lcod{\store, \be_1}{\cid},\ldots,\lcod{\store,\be_n}{\cid})
  \end{array}
  %\end{array}
  $$

\bigskip

  \begin{mathpar}
    (\store, \xassign{x}{\be}{\cid}) \redx \extend{\store}{x}{\lcod{\store,\be}{\cid}}

    \inferrule
    {(\store_1,\be_1) \redx \store_2 \\ (\store_2,\be_2) \redx \store_3 }
    {(\store_1,\be_1;\be_2) \redx \store_3}
    %(\store, \eassign{\mesg{w}}{\cid_1}{\be}{\cid_2};\prog) \redx (\extend{\store}{\mesg{w}_{\cid_1}}{\lcod{\store,\be}{\cid_2}}, \prog)    
    %(\store, \reveal{w}{\be}{\cid};\prog) \redx (\extend{\store}{\rvl{w}}{\lcod{\store,\be}{\cid}}, \prog)   
    %(\store, \pubout{\cid}{\be}{\cid};\prog) \redx (\extend{\store}{\out{\cid}}{\lcod{\store,\be}{\cid}}, \prog)
  \end{mathpar}


$$
\begin{array}{rclr}
  (\store, \xassign{x}{\be}{\cid}) &\aredx&
  \extend{\store}{x}{\lcod{\store,\be}{\cid}} & \cid \in H\\
  (\store, \xassign{x}{\be}{\cid}) &\aredx&
  \extend{\store}{x}{\lcod{\arewrite(\store_C,\be)}{\cid}} & \cid \in C
\end{array}
$$

$$
\begin{array}{rcl@{\qquad}r}
  (\store,\elab{\assert{\be_1 = \be_2}}{\cid}) &\aredx& \store & \text{if\ }
  \lcod{\store,\be_1}{\cid} = \lcod{\store,\be_2}{\cid}  \text{\ or\ } \cid \in C\\
  (\store,\elab{\assert{\phi(\be)}}{\cid}) &\aredx& \abort & \text{if\ } \neg\phi(\store,\lcod{\store,\be}{\cid})
\end{array}
$$

\begin{mathpar}
  (\store, \xassign{x}{\be}{\cid}) \redx \extend{\store}{x}{\lcod{\store,\be}{\cid}}
  
  \inferrule
      {(\store_1,\be_1) \redx \abort}
      {(\store_1,\be_1;\be_2) \redx \abort}
\end{mathpar}

$$
\begin{array}{rcl}
  \multicolumn{3}{l}{\flab \in \mathrm{Field},\   y \in \mathrm{EVar}, \  f \in \mathrm{FName}}\\[1mm]
  %x &\in& \mathrm{EVar}\\
  %f &\in& \mathrm{FName}\\[2mm]
  e &::=& \mv \mid \flip{e} \mid \secret{e} \mid \mesg{e} \mid \rvl{e} \mid e \bop e \mid
  \elet{y}{e}{e} \mid \\
  & & f(e,\ldots,e) \mid \{ \flab = e; \ldots; \flab = e \} \mid e.\flab \\
  %  & \textit{expressions}\\
  \cmd &::=& \msend{e}{e}{e}{e} \mid \reveal{e}{e}{e} \mid \pubout{e}{e}{e} \mid
      \elab{\assert{e = e}}{e} \mid \\
  & & f(e,\ldots,e) \mid  \cmd;\cmd \mid \pre{\eqs} \mid \post{\eqs} \\[1mm]
  \bop &::=& \fplus \mid \fminus \mid \ftimes \mid \concat  \\[1mm]% \textit{operators}\\[2mm]
  \mv &::=& w \mid \cid \mid \be \mid \{ \flab = \mv;\ldots;\flab = \mv \} 
  \\ % \mid \ttt{()} \\[1mm] %& \textit{values}\\[2mm]
  \mathit{fn} &::=& f(y,\ldots,y) \{ e \} \mid  f(y,\ldots,y) \{ \cmd \} \\[1mm]%& \textit{functions}
  \phi &::=& \elab{\flip{e}}{e} \mid \elab{\secret{e}}{e} \mid \elab{\mesg{e}}{e} \mid \rvl{e} \mid \out{e} \mid \phi \fplus \phi \mid \phi \fminus \phi \mid \phi \ftimes \phi \\
  \eqs &::=& \phi = \phi \mid \eqs \wedge \eqs 
\end{array}
$$

\begin{mathpar}
  \inferrule
      {e[\mv/y] \redx \mv'}
      {\elet{y}{\mv}{e} \redx \mv'}

  \inferrule
      {\codebase(f) = y_1,\ldots,y_n,\ e \\ e_1 \redx \mv_1 \cdots e_n \redx \mv_n \\
        e[\mv_1/y_1]\cdots[\mv_n/y_n] \redx \mv}
      {f(e_1,\ldots,e_n) \redx \mv}

  \inferrule
      {e_1 \redx \mv_1 \cdots e_n \redx \mv_n }
      {\{ \flab_1 = e_1; \ldots; \flab_n = e_n \} \redx \{ \flab_1 = \mv_1; \ldots; \flab_n = \mv_n \} }

  \inferrule
      {e \redx \{\ldots; \flab = \mv; \ldots\}}
      {e.\flab \redx \mv}

  \inferrule
      {e_1 \redx w_1 \\ e_2 \redx w_2}
      {e_1 \concat e_2 \redx w_1w_2}
\end{mathpar}

\begin{mathpar}
  \inferrule
      {e_1 \redx \be_1 \\ e_2 \redx \be_2 \\ e \redx \cid}
      {\mc{\prog}{(\eqs_1,\eqs_2)}{\eqon}{\elab{\assert{e_1 = e_2}}{e}} \redx
        (\prog,(\eqs_1, \eqs_2 \wedge \eqtrans{\elab{\be_1}{\cid}} = \eqtrans{\elab{\be_2}{\cid}},\eqon)}
\end{mathpar}


\bibliographystyle{ACM-Reference-Format}
\bibliography{logic-bibliography,secure-computation-bibliography}

\appendix

\section{Examples and Proof Methods}
\label{section-examples}

\subsection{2-Party GMW}
\label{section-metalang-gmw}

\begin{fpfig}[t]{2-Party GMW circuit library with And gate.}{fig-gmw}
{\footnotesize
  \begin{verbatimtab}
    encodegmw(in, i1, i2) {
      m[in]@i2 := (s[in] xor r[in])@i2;
      m[in]@i1 := r[in1]@2;
      m[in]
    }
    
    andtablegmw(b1, b2, r) {
      let r11 = r xor (b1 xor true) and (b2 xor true) in
      let r10 = r xor (b1 xor true) and (b2 xor false) in
      let r01 = r xor (b1 xor false) and (b2 xor true) in
      let r00 = r xor (bl xor false) and (b2 xor false) in
      { v1 = r11; v2 = r10; v3 = r01; v4 = r00 }
    }
    
    andgmw(g, v1, v2) {
      let r = r[g] in
      let table = andtablegmw(v1,v2,r) in
      m[g]@2 := OT4(v1,v2,table,2,1);
      m[g]@1 := r;
      m[g]
    }
    
    decodegmw(v) {
      p[1] := v@1; p[2] := v@2;
      out@1 := (p[1] + [2])@1;
      out@2 :=(p[1] + [2])@2
    }
  \end{verbatimtab}
}
\end{fpfig}


The GMW protocol is a garbled binary circuit protocol. In this section
we will assume the 2-party version, though it generalizes to $n$
parties\cite{XXX}. GMW uses a common technique in MPC, which is to
represent values $v$ as distributed shares $v_1$ and $v_2$ with
$v = v_1 \fplus v_2$. This trick maintains secrecy of $v$
from both parties, and in GMW it is used to maintain the intermediate
values of internal gate outputs in circuits. In related literature
the notation $\macgv{x}$ is used to represent the ``true''
value of $x$ and $[x]$ is often used to represent the share of
given party.

To capture this convention, which is used in many other protocols, we
introduce a new naming convention for ``global view'' elements
$\macgv{\mesg{w}}$, which denote the summed value of
$\elab{\mesg{w}}{1}$ and $\elab{\mesg{w}}{2}$ in a protocol
run. This concept integrates program distributions in the
usual manner, as the probability of the outcome of summation
of two variables in the distribution.
\begin{definition}
  For all $\mesg{w}$ define:
  $$\pmf(\macgv{\mesg{w}} = v) \defeq \sum_{\sigma \in A} \pmf(\sigma)$$
  and define:
  $$\condd{\pmf}{X}{\macgv{\mesg{w}} = v}(\sigma) \defeq  \sum_{\sigma' \in A} \condd{\pmf}{X}{\sigma'}(\sigma)$$
  where $A = \{ \store \in \mems(\{ \elab{\alpha}{1},\elab{\alpha}{2} \} ) \mid
      \cod{\store(\elab{\mesg{w}}{1}) + \store(\elab{\mesg{w}}{2})} = v \}$.
\end{definition}

For full details of the GMW protocol the reader is referred to
\cite{evans2018pragmatic}. Our implementation libary is shown in
Figure \ref{fig-gmw}, and includes encoding functions, where
input secrets are split into shares, $\eand$ and $\exor$ gate
functions, and a decoding function. Note that $\exor$ requires
no interaction between parties, while $\eand$ necessitates and
1-of-4 oblivious transfer. The gate computation is
done entirely in secrect, and the decoding function
is where the declassification occurs-- both parties reveal
their shares of the final gate output $\macgv{z}$.

For example, the following program uses our GMW library to define
a circuit with a single \eand gate and input secrets $\ttt{s1}$ and
$\ttt{s2}$ from client's 1 and 2 respectively:
\begin{verbatimtab}
         let s1 = encodegmw("s1") in
         let s2 = encodegmw("s2") in
         decodegmw(andgmw("z",s1,s2))
\end{verbatimtab}
By convention we will assume that all gates are assigned unique output
identifiers $\ttt{"z"}$, and that all programs are in the form
of a sequence of let-bindings followed by a call to $\decodegmw$
wrapping a circuit.

\paragraph{Oblivious Transfer} A passive secure OT protocol
based on previous work \cite{XXX} can be defined in $\metaprot$,
however this protocol assumes some shared randomness. Alternatively,
a simple passive secure OT can be defined with the addition of
public key cryptography as a primitive. But given the diversity
of approaches to OT, we instead assume that OT is abstract with
respect to its implementation, where calls to OT in $\mathbb{F}_2$
are of the following form-- given a \emph{choice bit}
$\be_1$ provided by a receiver $\cid$, the sender
sends either $\be_2$ or $\be_3$.
$$
\OT{\elab{\be_1}{\cid}}{\be_2}{\be_3}
$$
Critically, the sender learns nothing about $\be_1$ and the
receiver learns nothing about the unselected value, so we interpret
these calls in our implementation in $\mathbb{F}_2$ as follows.
$$
\begin{array}{l}
\solve{\stores}{\OT{\elab{\be_1}{\cid_1}}{\be_2}{\be_3}}{\cid_2} = \\
\qquad ((\solve{\stores}{\be_1}{\cid_1}) \cap 
(\solve{\stores}{\be_2}{\cid_2})) \cup \\
\qquad ((\stores - (\solve{\stores}{\be_1}{\cid_1})) \cap
(\solve{\stores}{\be_3}{\cid_2})
\end{array}
$$

\subsubsection{Correctness Proof with Verification Tactics}

As discussed above and in related work \cite{XXX}, probabilistic
separation conditional on certain variables-- e.g., secret inputs or
public outputs-- is a key mechanism for reasoning about MPC protocol
security. Following \cite{XXX}, we define a conditional separation
relation $\condsep{\pmf}{X_1}{X_2}{X_3}$ to mean that
under the condition of all value assignments for
$X_2$ and $X_3$ are separable under $\pmf$ on the condition
of any value assignment of $X_1$-- i.e., conditionally
on any $\store \in \mems(X_1)$. 
\begin{definition}[Conditional Separation]
  We write $\condsep{\pmf}{X_1}{X_2}{X_3}$ iff for all
  $\store' \in \mems(X_1)$ and $\store \in \mems(X_2 \cup X_3)$
  and letting $X = X_1 \cup X_2 \cup X_3$ we have:
  $$\condd{\pmf}{X}{\store'}(\store) =
  \condd{\pmf}{X}{\store'}(\store_{X_2}) *
  \condd{\pmf}{X}{\store'}(\store_{X_3})$$
\end{definition}
Another key concept needed especially for reasoning about
circuits is conditional determinism. For example, if
$\macgv{z}$ is an output of an internal gate, it will
definitely be computed using random variables, however,
it \emph{should} be determistic under any set of input
secrets $S$, since we assume that $\idealf$ is
deterministic.
\begin{definition}[Conditional Determinism]  
  We write $\conddetx{\pmf}{X_1}{X_2}$ iff for all
  $\store' \in \mems(X_1)$ there exists 
  $\store \in \mems(X_2)$ such that
  $\condd{\pmf}{X_1 \cup X_2}{\store'}(\store) = 1$.
\end{definition}

Given these definitions, we can formulate an invariant
for circuit computation with respect to internal gates
as follows. It says that the output of any gate is
deterministic wrt inputs $S$, and conditionally
on $S$ the output $\macgv{\mesg{z}}$ remains
separable from corrupt views and both shares of
$\macgv{\mesg{z}}$. This last nuance is critical,
since those shares will in fact be revealed if
$\macgv{\mesg{z}}$ is decoded as the circuit output. 
\begin{lemma}[GMW Invariant]
  \label{lemma-gmwinvariant}
  Given:
  $$ (\varnothing,e) \redxs (\prog,\decodegmw(E[\mesg{z}])) $$
  Then both of the following conditions hold for all $H$ and $C$ where $\iov(\prog) = S \cup M$:
  \begin{enumerate}
    \item $\conddetx{\progtt(\prog)}{S}{\{\macgv{\mesg{z}}\}}$
    \item $\condsep{\progtt(\prog)}{S}{\{\macgv{\mesg{z}}\}}{(M_C \cup \{ \elab{\mesg{z}}{1}, \elab{\mesg{z}}{2} \})}$
  \end{enumerate}
\end{lemma}
To prove this, we can formulate and automatically prove gate-level
versions of the invariant. This serves as a proof tactic
that simplifies the proof of the GMW invariant. It establishes
that the $\eand$ gate output is deterministic conditional on
the inputs, and is separable from the output shares and
either parties' received messages. 
\begin{lemma}[And Gate Tactic]
  \label{lemma-gmwtactic}
  %Define:
  %$$
  %\begin{array}{c}
  %  \prog_{i} \defeq \\
  %  \eassign{\mesg{x}}{1}{\flip{x}}{1}; \eassign{\mesg{x}}{2}{\flip{x}}{2}; \\
  %  \eassign{\mesg{y}}{1}{\flip{y}}{1}; \eassign{\mesg{y}}{2}{\flip{y}}{2} 
  %\end{array}
  %$$
  Given:
  $$
  \begin{array}{c}
  (\varnothing,\andgmw(z,\mesg{x},\mesg{y}) \redxs %\\
  (\prog,\mesg{z})
  \end{array}
  $$
  Then both of the following conditions hold for both $\cid \in \{ 1,2 \}$ where $\iov(\prog) = M$:
  \begin{enumerate}
  \item
    $\conddetx{\progtt(\prog)}{\{ \macgv{\mesg{x}},\macgv{\mesg{y}} \}}{\{ \macgv{\mesg{z}} \}}$
   \item $\condsep
      {\progtt(\prog)}
      {\{ \macgv{\mesg{x}},\macgv{\mesg{y}} \}}
      {\{ \macgv{\mesg{z}} \}}
      {M_{\{\cid\}} \cup \{ \elab{\mesg{z}}{1},\elab{\mesg{z}}{2} \}}$
  \end{enumerate}
\end{lemma}
\begin{proof}
Verified automatically using techniques described in Section \ref{section-bruteforce}.  
\end{proof}

To properly integrate the local reasoning of Lemma \ref{lemma-gmwtactic} with
the global reasoning of Lemma \ref{lemma-gmwinvariant}, we can demonstrate
the following.
\begin{lemma}
  \label{lemma-conditioning}
  Each of the following hold:
  \begin{enumerate}
    \item Given $\condp{\pmf}{X_1}{\detx{X_2}}$ and
      $\condp{\pmf}{X_2}{\detx{X_3}}$, then $\condp{\pmf}{X_1}{\detx{X_3}}$.
    \item Given $\condp{\pmf}{X_1}{\detx{X_2}}$ and
      $\condsep{\pmf}{X_2}{X_3}{X_4}$, then $\condsep{\pmf}{X_1}{X_3}{X_4}$.
    \item Given $\condsep{\pmf}{X_1}{X_2}{X_3}$ and $\condp{\pmf}{X_1}{\detx{X_2}}$
      and $\condp{\pmf}{X_2}{\detx{X_4}}$, then $\condsep{\pmf}{X_1}{X_4}{X_3}$.
  \end{enumerate}
\end{lemma}

Then we can put the pieces together to prove the invariant, using automated tactics
for gate-level reasoning.  
\begin{proof}[Proof of Lemma \ref{lemma-gmwinvariant}]
  By induction on the length of $(\varnothing,e) \redxs (\prog,\decodegmw(E[\mesg{z}]))$.
  Encoding establishes the invarian in the basis. The most interesting inductive
  case is the $\andgmw$ gate. 
  \paragraph{Case $\andgmw$.} In this case we have:
  $$
  \begin{array}{c}
  (\varnothing,e) \redxs (\prog',\decodegmw(E[\andgmw(z,\mesg{x},\mesg{y})])) \redxs \\
    (\prog,\decodegmw(E[\mesg{z}]))
  \end{array}
  $$
  The induction hypothesis, together with the assumed uniquenenss of $z$, gives:
  \begin{mathpar}
  \condsep{\progtt(\prog)}{S^1}{\{ \macgv{\mesg{x}} \}}{(M^1_C \cup \{ \elab{\mesg{x}}{1}, \elab{\mesg{x}}{2} \})}
  
  \condsep{\progtt(\prog)}{S^2}{\{ \macgv{\mesg{y}} \}}{(M^2_C \cup \{ \elab{\mesg{y}}{1}, \elab{\mesg{y}}{2} \})}

  \conddetx{\progtt(\prog)}{S^1}{\{ \macgv{\mesg{x}} \}}
  
  \conddetx{\progtt(\prog)}{S^1}{\{ \macgv{\mesg{x}} \}}    
  \end{mathpar}
  This, together with Lemma \ref{lemma-gmwtactic} (1) and Lemma \ref{lemma-conditioning} (1)
  give:
  $$
  \conddetx{\progtt(\prog)}{S^1 \cup S^2}{\{ \macgv{\mesg{z}} \}}
  $$
  and Lemma  \ref{lemma-gmwtactic} (2) and Lemma \ref{lemma-conditioning} (2-3) gives:
  $$
  \begin{array}{c}
    \progtt(\prog)|S^1 \cup S^2) \vdash \\
    {\{ \macgv{\mesg{z}} \}}{(M^1_C \cup M^2_C \cup \{ \elab{\mesg{x}}{2}, \elab{\mesg{y}}{2}, \elab{\mesg{z}}{1}, \elab{\mesg{z}}{2} \})}
  \end{array}
  $$
\end{proof}

\subsection{2-Party BDOZ}

\begin{fpfig}[t]{2-party BDOZ circuit library: sum gates and secure opening.}{fig-bdozsum}
{\footnotesize
\begin{verbatimtab}
  _open(x,i1,i2){
    m[x++"exts"]@i1 := m[x++"s"]@i2;
    m[x++"extm"]@i1 := m[x++"m"]@i2;
    assert(m[x++"extm"] = m[x++"k"] + (m["delta"] * m[x++"exts"]))@i1;
    m[x]@i1 := (m[x++"exts"] + m[x++"s"])@i1
  }
  
  pre: { m[x++"m"]@i2 == m[x++"k"]@i1 + (m["delta"]@i1 * m[x++"s"])@i2 /\
         m[y++"m"]@i2 == m[y++"k"]@i1 + (m["delta"]@i1 * m[y++"s"])@i2 }
  _sum(z, x, y,i1,i2) {
      m[z++"s"]@i2 := (m[x++"s"] + m[y++"s"])@i2;
      m[z++"m"]@i2 := (m[x++"m"] + m[y++"m"])@i2;
      m[z++"k"]@i1 := (m[x++"k"] + m[y++"k"])@i1
  }
  post: { m[z++"m"]@i2 == m[z++"k"]@i1 + (m["delta"]@i1 * m[z++"s"]@i2 } 
  
  sum(z,x,y) { _sum(z,x,y,1,2); _sum(z,x,y,2,1) }

  open(x) { _open(x,1,2); _open(x,2,1) }
\end{verbatimtab}
}
\end{fpfig}

\begin{fpfig}[t]{2-party BDOZ circuit library: multiplication gates.}{fig-bdozmult}
{\footnotesize
\begin{verbatimtab}
  _mult1(z, x, y) {
      m[z++"s"]@1 :=
        (m[z++"bs"] * m[z++"d"] + m[z++"as"] * m[z++"e"] + m[z++"cs"])@1;
      m[z++"m"]@1 :=
        (m[z++"bm"] * m[z++"d"] + m[z++"am"] * m[z++"e"] + m[z++"cm"])@1;
      m[z++"k"]@2 :=
        (m[z++"bk"] * m[z++"d"] + m[z++"ak"] * m[z++"e"] + m[z++"ck"])@2    
  }
  post: { m[z++"m"]@2 == m[z++"k"]@1 + (m["delta"]@1 * m[z++"s"]@1 }

  mult(z,x,y) {
      sum(z++"a", x, z++"d");
      open(z++"d");
      sum(z++"b", y, z++"e");
      open(z++"e"); 
      _mult1(z,x,y); _mult2(z,x,y)
  }
  post: {  m[z++"s"]@1 + m[z++"s"]@2 ==
          (m[x++"s"]@1 + m[x++"s"]@2) * (m[y++"s"]@1 + m[y++"s"]@2)} 
  
\end{verbatimtab}
}
\end{fpfig}

% hints for confidentiality
\begin{comment}
      m[z++"ds"]@1 as m[x++"s"]@2 + r[z++"as"]@2;
      m[z++"ds"]@2 as m[x++"s"]@1 + r[z++"as"]@1;
      m[z++"ms"]@2 as m[z++"k"]@1 + (m["delta"]@1 * m[z++"ds"]@2);
      m[z++"ms"]@1 as m[z++"k"]@2 + (m["delta"]@2 * m[z++"ds"]@1);
\end{comment}


In a malicious setting, detecting cheating by adding
information-theoretic secure MAC codes to shares is a fundamental approach
realized by protocols such as BDOZ \cite{XXX} and SPDZ \cite{XXX}.
These protocols assume a pre-processing phase that distributes shares
with MAC codes to clients.  This integrates well with pre-processed
Beaver Triples to implement malicious secure, and relatively
efficient, multiplication \cite{XXX}.  Here we show how to implement
malicious secure two party multiplication in the BDOZ style in
$\metaprot$. 

A field value $v$ is secret shared among 2 clientz in BDOZ as follows.
Each client $\cid$ gets a pair of the form $(v_\cid,m_\cid)$, where
$v_\cid$ is a secret share of $v$, i.e., $v = \fcod{v_1 \fplus v_2}$.
Further, field values $m_\cid$ are MACs of $v_\cid$ that are authenticated
by the other client's key. In particular, $m_\cid = k + (k_\Delta *
v_\cid)$ where \emph{keys} $k$ and $k_\Delta$ are known only to $\cid'
\ne \cid$ and $k_\Delta$. Further, there is a unique \emph{local key}
$k$ for each shared value, while the \emph{global key} $k_\Delta$ is
common to all MACs authenticated by $\cid'$. By extending certain
field operations to these MACed values, a semi-homorphic encryption
scheme is obtained that is adequate to compute Beaver Triples. For
example, if local keys $k_1$ and $k_2$ authenticate $\macv_1$ and
$\macv_2$, then $k_1 \fplus k_2$ authenticates $\macv_1 \macplus
\macv_2$, where $\macplus$ is addition extended to MACed
values. Semantics for summation of MACed shares, and multipication of
shares and plain field values, are given in Figure \ref{fig-bdoz}.
For more details the reader is referred to \cite{XXX}. We note
that while we restrict values $v$, $m$, and $k$ to the same
field in this presentation for simplicity, in general $m$ and
$k$ can be in extensions of $\mathbb{Z}_p$. 

We define variable forms $\macx{\secret{w}}{\cid}$ and
$\macx{\flip{w}}{\cid}$ to range over local MACed shares of a secret
and random sample owned by client $\cid$, respectively. We let
$\mack{x}{}$ denote the local key for authenticating shares of $x$.
We also assume a global key $\flip{\ttt{delta}}$ on each client.
Pre-processing is modeled by assuming that shares are provided on
input random tapes (memories) with the required properties, as defined
on the bottom of Figure \ref{fig-bdoz}.

Similarly to field values, we can use MACed shares in
uniform independent distributions as one-time-pads for
hiding other values, using subtraction over MACed shares.
So we introduce a function $\macotp$ to do this
declaratively.
$$
\delta(\macotp,v_1,v_2) \defeq \fcod{v_1 \macminus v_2}
$$

Another key idea in BDOZ is \emph{secure opening}, where distributed
shares of a value $v$ are communicated and combined to ``open'' the
same $v$ on each client. This relies on authentication to detect
cheating in the malicious setting. To implement secure opening we
define a $\macauth$ function that authenticates shares, and a
$\macopen$ function that extracts a field value from its shares.
\begin{mathpar}
\delta(\macauth, (v, m), k) \defeq
     (v, m) \text{\ if\ } m = k \fplus (\flip{\ttt{delta}} \ftimes v)
 
\delta(\macopen, (v_1, m_1), (v_2,m_2)) \defeq
\fcod{v_1 \fplus v_2}
\end{mathpar}
This authentication eliminates cheating and guarantees protocol
integrity. As we will show in Section \ref{section-types}, this
can be reflected in a security type for $\macauth$.

Putting these pieces together, we can implement 2-party multiplication with
Beaver Triples as follows. Recall that Beaver Triples are values
$(a,b,c)$ such that $c = \fcod{b \ftimes a}$- we implement them here
as a secret $\secret{c}$ and samples $\flip{a}$ and $\flip{b}$ owned
by an \emph{oracle} $\Oracle$ that we always assume is honest but is
otherwise not involved in the protocol. We define pre-processing
properties of Beaver Triples in Figure \ref{fig-bdoz}. Only one
Beaver triple can be used per multiplication to ensure secure
openings, but our scheme can be extended with more Beaver triples
and reuse of them will be rule out by type linearity as we will
discuss in Section \ref{section-types}.

First, clients 1 and 2 compute (aka \emph{open}) $\secret{x} \fplus
\flip{a}$ and $\secret{y} \fplus \flip{b}$ where $\secret{x}$ and
$\secret{y}$ are 1 and 2's secrets respectively.  Then they both
perform a secure opening of $\secret{x} \fplus \flip{a}$ and
$\secret{y} \fplus \flip{b}$, by authenticating each others' shares,
and record them as $\mesg{d}$ and $\mesg{e}$ respectively. Note
that the secrets remain hidden since $\flip{a}$ and $\flip{b}$
are random values.

\begin{definition}
  Given $\prog$ with $\iov(\prog) = S \cup V \cup O$,
  let $X_H \subseteq \{ x \mid x \in (\houtputs \cup O_H) \wedge x \in \dom(\store) \}$.
  Then the \emph{adversarial inputs to $X_H$} is the least set
  $X_C \subseteq \cinputs$ such that $\progtt(\prog) \not\vdash X_C * X_H$.
\end{definition}
\begin{definition}[Cheating Detection]
  \emph{Cheating is detected} in $\prog$ with $\iov(\prog) = S \cup V \cup O$ iff
  for all  $\store \in \aruns(\prog)$,
  letting $X_H = \{ x \mid x \in (\houtputs \cup O_H) \wedge x \in \dom(\store) \}$,
  and letting $X_C$ be the adversarial inputs to $X_H$,
  there exists $\sigma'\in \runs(\prog)$
  with $\store_{X_C} = \store'_{X_C}$.  
\end{definition}

\begin{lemma}
  If cheating is detected in $\prog$, then $\prog$ has integrity.
\end{lemma}


\begin{fpfig}[t]{Yao's Garbled Circuits, selected auxiliary functions.}{fig-ygc-aux}
{\footnotesize
\begin{verbatimtab}
  keygen(gid, b1, b2) { select4(b1,b2,H[gid || 1],H[gid || 2],H[gid || 3],H[gid || 4]) }
  
  keysgen(gid, b1, b2)
  {
    let k11 = keygen(gid,b1,b2) in
    let k10 = keygen(gid,b1,not b2) in
    let k01 = keygen(gid,not b1,b2) in
    let k00 = keygen(gid,not b1,not b2) in
    {k11 = k11; k10 = k10; k01 = k01; k00 = k00}
  }
  
  andtable(keys, bt, ap, bp)
  {
    let r11 = (keys.k11 xor bt) in 
    let r10 = (keys.k10 xor (not bt)) in
    let r01 = (keys.k01 xor (not bt)) in
    let r00 = (keys.k00 xor (not bt)) in
    permute4(ap,bp,r11,r10,r01,r00)
  }
  
  sharetable(gid, tid, table)
  {   
    v[1, gate: || gid || tid || 1] := table.v1;
    v[1, gate: || gid || tid || 2] := table.v2;
    v[1, gate: || gid || tid || 3] := table.v3;
    v[1, gate: || gid || tid || 4] := table.v4
  }

  owl(gid) {  { k = flip[2,gate: || gid || .k]; p = flip[2,gate: || gid || .p] }  }
\end{verbatimtab}
}
\end{fpfig}

\begin{fpfig}[t]{Yao's Garbled Circuits, garbled gates and evaluation code.}{fig-ygc-gates}
{\footnotesize
\begin{verbatimtab}
  garbledecode(gid)    
  {
    let wl = owl(gid) in
    let r1 = wl.k xor true in
    let r0 = (not wl.k) xor false in
    v[1,OUTtt1] := select(wl.p,r1,r0);
    v[1,OUTtt2] := select(not wl.p,r1,r0)
  }
  
  evaldecode(wv, p) { wv.k xor select[wv.p,v[1,OUTtt1],v[1,OUTtt2]] }
  
  evalgate(gid, wva, wvb)  
  {
    let k = keygen(gid,wva.k,wvb.k) in
    let ct = select4(wva.p,wvb.p,
               v[1,gid || tt1],v[1,gid || tt2],v[1,gid || tt3],v[1,gid || tt4]) in
    let cp = select4(wva.p,wvb.p,
               v[1,gid || pt1],v[1,gid || pt2],v[1,gid || pt3],v[1,gid || pt4]) in
    { k = k xor ct; p = k xor cp }
  }
  
  andgate(gid, ga, gb) 
  {
    let wla = owl(ga) in
    let wlb = owl(gb) in
    let wlc = owl(gid) in
    let keys = keysgen(gid,wla.k,wlb.k) in
    sharetable(gid,tt,andtable(keys,wlc.k,wla.p,wlb.p));
    sharetable(gid,pt,andtable(keys,wlc.p,wla.p,wlb.p))
  }

  encode(sa, sb)
  {
    let owl1 = owl(sa) in
    let owl2 = owl(sb) in
    v[1,gate: || sa || 1.k] := OT(s[1,sa],owl1.k,(not owl1.k));
    v[1,gate: || sa || 1.p] := OT(s[1,sa],owl1.p,(not owl1.p));
    v[1,gate: || sb || 2.k] := select(s[2,sb],owl2.k,(not owl2.k));
    v[1,gate: || sb || 2.p] := select(s[2,sb],owl2.p,(not owl2.p));
    { wv1 = { k = v[1,gate: || sa || 1.k]; p = v[1,gate: || sa || 1.p] };
      wv2 = { k = v[1,gate: || sb || 2.k]; p = v[1,gate: || sb || 2.p] } }
  }
\end{verbatimtab}
}
\end{fpfig}

%  andtable(keys, bt, ap, bp)
%  {
%    let r11 = (keys.k11 xor bt) in 
%    let r10 = (keys.k10 xor (not bt)) in
%    let r01 = (keys.k01 xor (not bt)) in
%    let r00 = (keys.k00 xor (not bt)) in
%    permute4(ap,bp,r11,r10,r01,r00)
%  }

  
%  encode(gid, wla,wlb)
%  {
%    let wla = { k = flip[2,fwl1]; p = flip[2,pwl1] } in
%    let wlb = { k = flip[2,fwl2]; p = flip[2,pwl2] } in
%    
%    { wv1 = { k = OT[s[1,0],wla.k,(not wla.k)]; p = OT[s[1,0],wla.p,(not wla.p)]}; 
%      wv2 = { k = select[s[2,0],wlb.k,(not wlb.k)]; p = select[s[2,0],wlb.p,(not wlb.p)] } }
%  }


\section{Rewriting $\minicat$ to Stratified Datalog}
\label{section-bruteforce}

Here we show how to rewrite any $\prog$ to an equivalent Datalog
program, which supports application of recent work in linear algebraic
interpretation of Datalog and optimizations of model computation on
high performance computers \cite{sakama2017linear}. The method here also enumerates
$\runs(\prog)$ memory-by-memory, rather than in a ``batched'' manner
as in our first method, allowing parallelization of model
computation. The rewriting we describe here is to Datalog with
negation, with a negation-as-failure model, though we can also
use techniques in \cite{sakama2017linear} to eliminate negation from
resulting programs. \emph{Atoms} are $\minifed$ variables, \emph{literals}
are atoms or negated atoms, and clause bodies are conjunctions of literals.
A \emph{Datalog program} is a list of clauses.
$$
\begin{array}{rclr}
  \alpha &::=& \view{\cid}{w} \mid  \secret{\cid}{w} \mid \flip{\cid}{w} \mid \oracle{w} & \qquad \textit{(atoms)}\\
  \mathit{body} &::=&  \alpha \mid \neg \alpha \mid \alpha \wedge \mathit{body} \mid \neg\alpha \wedge \mathit{body} \mid \varnothing \\
  \mathit{clause} &::=& \alpha \gets \mathit{body}
\end{array}
$$

The first step in converting a protocol $\prot$ to a Datalog
program is to apply $\solve$ ``locally'' to each view definition
in $\prog$, obtaining constraints on memories that satisfy each
view in isolation.  
\begin{lemma} Let $\vars(\be)$ be the variables in $\be$, and define:
$$
{vtt}(\eassign{\view{\cid}{w}}{\be}) \defeq (\view{\cid}{w}, (\solve{(\mems(\vars\ \be))}{\be}))
$$
Then ${vtt}(\eassign{\view{\cid}{w}}{\be}) = (\view{\cid}{w},\stores)$ where $\stores
  = \{ \store \in \mems(\vars\ \be) \mid \lcod{\store,\be}{\cid} = 1 \}$ for some $\cid$.
\end{lemma}
Given this definition, the mapping of ${vtt}$ across a program
$\prog$- i.e., $(\mathit{map}\ {vtt}\ \prog)$- essentially defines the
logic program for ``view atoms'' modulo syntactic conversion. We can
accomplish the latter as follows, where $\datalog(\prog)$ defines the
full conversion.
\begin{definition} We define the conversion from memories to
  literals and clause bodies as follows:
\begin{mathpar}
  \logit{\alpha \mapsto 1} = \alpha

  \logit{\alpha \mapsto 0} = \neg \alpha

  \logit{\{ \alpha_1 \mapsto \beta_1, \ldots, \alpha_n \mapsto \beta_n\}} =
  \logit{\alpha_1 \mapsto \beta_1} \wedge \cdots \wedge \logit{\alpha_n \mapsto \beta_n}
\end{mathpar}
Given pairs $(\alpha,\stores)$ in the range of ${vtt}$, we define the conversion
to clauses as follows:
\begin{mathpar}
  \mathit{clauses}(\alpha,\{ \store_1,...,\store_n \}) = \alpha \gets \logit{\store_1} \vee \cdots \vee \alpha \gets \logit{\store_n}
\end{mathpar}
The $\minifed$-to-Datalog conversion is then defined as:
$$
\datalog(\prog) \defeq  \mathit{map}\ \mathit{clauses}\ (\mathit{map}\ {vtt}\ \prog)
$$
\end{definition}

In addition to converting view definitions to logic clauses, we also need to convert
secrets and random tapes. Since we assume given values for these in an arbitrary run of
the program, we can capture these as a ``fact base'' in our program, where
a fact is a clause of the form $\alpha \gets \varnothing$ and means that $\alpha$
is true in any model of the program. 
\begin{definition}
  Given $\store$, let $\{\alpha_1,\ldots,\alpha_n \} =
  \{ \alpha \in \dom(\store) \mid \store(\alpha) = 1 \}$.
  Then define:
  $$
  \mathit{facts}(\store) = \alpha_1 \gets \varnothing, \ldots, \alpha_n \gets \varnothing
  $$
\end{definition}

For any safe $\prog$ with $\iov(\prog) = S \cup V$ and $\flips(\prog) = F$, and
$\store \in \mems(S \cup F)$, it is the case
that $(\mathit{facts}(\store),\datalog(\prog))$ is a \emph{normal}, \emph{stratified}
(in fact, non-recursive) Datalog program, and so has a unique Least Herbrand Model
and is thus amenable to HPC optimization techniques \cite{aspis2018linear}. 
To compute $\runs(\prog)$, and thus $\progd(\prog)$, we compute
the Least Herbrand Model $\store$ of $(\mathit{facts}(\store'),\datalog(\prog))$
for all $\store' \in \mems(S \cup F)$, observing that model computation for
each $\store'$ can be done in parallel. The following establishes
correctness of this approach.
\begin{lemma}
  For all $\prog$ with $\iov(\prog) = S \cup V$ and $\flips(\prog) = F$,
  $\datalog(\prog)$ is a \emph{normal}, \emph{stratified}
  program \cite{aspis2018linear}, and $\store$ is the unique Least Herbrand
  Model of $(\mathit{facts}(\store_{S \cup F}),\datalog(\prog))$
  iff $\store \in \runs(\prog)$.
\end{lemma}
A full empirical exploration of the application of HPC optimizations
is beyond the scope of this paper but is a compelling topic for future
work. The reader is referred to \cite{nguyen2022enhancing} for
empirical results of HPC optimizations in other logic programming
contexts.


\section{Supplementary Proofs}
\label{section-proofs}

We observe the following properties of separation which are borrowed from prior work
\cite{barthe2019probabilistic} and which will be used frequently in proofs. 
\begin{lemma}
  \label{lemma-separation}
  The following properties hold:
  \begin{enumerate}
  \item $\sep{\pmf}{Y}{Z}$ iff $\sep{\pmf}{Z}{Y}$
  \item $\vc{\pmf}{x}{y}$ if  $\vc{\pmf}{y}{x}$
  \item $\vc{\pmf}{x}{y}$ and $\vc{\pmf}{y}{z}$ imply $\vc{\pmf}{x}{z}$
  \item $\sep{\pmf}{X}{(Y \cup Z)}$ implies $(\sep{\pmf}{X}{Y}$ and $\sep{\pmf}{X}{Z})$
  \item $(\sep{\pmf}{X}{Y}$ and $\sep{\pmf}{(X \cup Y)}{Z})$ implies $\sep{\pmf}{X}{(Y \cup Z)}$ 
  %\item If $\prog_1;\prog_2$ is safe and $\vars(\prog_1) \cap \vars(\prog_2) = \varnothing$
  %  then $\sep{\progtt(\prog_1;\prog_2)}{\vars(\prog_1)}{\vars(\prog_2)}$.
  \end{enumerate}
\end{lemma}
The following property also follows by results in \cite{barthe2019probabilistic} and will
be useful to make constructions that demonstrate variable separation.
\begin{lemma}
  \label{lemma-sepjoin}
  $\sep{\progtt(\prog)}{X}{Y}$ iff for all 
  $\store^1, \store^2 \in \runs(\prog)$ there exists
  $\store \in \runs(\prog)$ with
  $\store^1_{X} \cap \store^2_{Y} \subseteq \store$.
\end{lemma}

\begin{lemma}
  \label{lemma-presub}
  Given $\be_1$, $\be_2$, and $i$ where $\vars(\be_1) \cap
  \vars(\be_2) = \varnothing$ and $\lcod{\store_1,\be_1}{i} = \beta$
  with $\store_1(x) = \lcod{\store_2,\be_2}{i}$ and
  $\dom(\store_1) = \vars(\be_1)$ and $\dom(\store_2) = \vars(\be_2)$.
  Then $\lcod{\store_1\cap\store_2,\be_1[\be_2/x]}{i} = \beta$.
\end{lemma}
\begin{proof}
By straightforward structural induction on $\be$.
\end{proof}

\substar*

\begin{proof}
  Suppose on the contrary it was not the case that
  $\sep{\progtt(\prog_1;\prog_2[\be/f])}{\vars(\be)}{X}$.
  Then by Lemma \ref{lemma-sepjoin} there exists $\store_1,\store_2
  \in \runs(\prog_1;\prog_2[\be/f])$ with no $\store \in
  \runs(\prog_1;\prog_2[\be/f])$ such that $\store^1_{\vars(\be)}
  \cap \store^2_X \subseteq \store$.  But also by
  assumption and Lemma \ref{lemma-sepjoin} for any $\beta$ there
  exists $\store' \in \runs(\prog_2)$ with $\{ f \mapsto \beta \} \cap
  \store^2_X \subseteq \store'$. So in particular, we
  have $\{ f \mapsto \lcod{\store^1_{\vars(\be)},\be}{i} \} \cap
  \store^2_X \subseteq \store'$ for any $i$. This, the
  assumption $\vars(\prog_1,\be) \cap \vars(\prog_2) = \varnothing$,
  and application of Lemma \ref{lemma-presub} leads to the consequence
  that there exists $\store \in \runs(\prog_1;\prog_2[\be/f])$ with
  $\store \supseteq \store^1_{\vars(\be)} \cap
  \store^2_X$ given the assumption $\vars(\prog_1,\be)
  \cap \vars(\prog_2) = \varnothing$, which is a contradiction.
\end{proof}


\gmwencode*

\begin{proof}
  By assumptions of well-typedness we have:
  $$
  \begin{array}{rcl}
    v &=&  \ttt{\{shares1 = }v_1\ttt{; shares2 = }v_2\ttt{\}}\\
    v_1 &=& \ttt{\{ c1 = v[1,s1out]; c2 = v[2,s1out] \}}\\
    v_2 &=& \ttt{\{ c1 = v[1,s2out]; c2 = v[2,s2out] \}}
  \end{array}
  $$
  And by Definition \ref{definition-gmwencode-certification} we have:
  \begin{mathpar}
    \sep{\progtt(\prog)}{\ttt{\{s[1,s1]\}}}{\ttt{\{v[2,s1out],v[2,s2out]\}}}
    
    \sep{\progtt(\prog)}{\ttt{\{s[2,s2]\}}}{\ttt{\{v[1,s1out],v[1,s2out]\}}}
  \end{mathpar}
  Since we assume that secrets are in uniform and independent marginal
  distributions a priori, and $\vars(\prog_1) \cap \vars(\prog_2) =
  \varnothing$ by condition (iii) of Definition 
  \ref{definition-gmwencode-certification} and assumptions of well-typedness, 
  the result follows by preconditions and Lemma \ref{lemma-separation}.
\end{proof}

\ygcgate*

\begin{proof}
  Given preconditions we have $v_1 =  \{ \ttt{k = }\be^1_{k}; \ttt{p = }\be^1_{p} \}$ and
  $v_2 = \{ \ttt{k = }\be^2_{k}; \ttt{p = }\be^2_{p} \}$ for some
  $\be^1_{k}$,$\be^2_{k}$,$\be^1_{p}$,$\be^2_{p}$ with correlations as per Definition
  \ref{definition-gc}.
  %\begin{eqnarray*}
  %  v_1 &=& \{ \ttt{k = }\be^1_{k}; \ttt{p = }\be^1_{p} \} \\
  %  v_2 &=& \{ \ttt{k = }\be^2_{k}; \ttt{p = }\be^2_{p} \}
  %\end{eqnarray*}
  Let $\prog$ be as defined in Definition \ref{definition-ygcgate-certification}.
  We observe:
  \begin{eqnarray*}
    &\prog_2 = \\
    &{\small \prog[\be^1_{k}/\ttt{flip[2,gate:a.k]}][\be^1_{p}/\ttt{flip[2,gate:a.p]}][\be^2_{k}/\ttt{flip[2,gate:b.k]}][\be^1_{p}/\ttt{flip[2,gate:b.p]}]}
  \end{eqnarray*}
  Given that $g_1$ and $g_2$ are distinct and not wired in $\prog_1$
  we are assured that $\ttt{owl}(g_1)$ and $\ttt{owl}(g_1)$ are in
  independent uniform distributions, and given that $g \not\in
  \prog_1$ we are assured that $\ttt{owl}(g)$ contains entirely fresh
  flips. Thus by condition (i) of Definition \ref{definition-ygcgate-certification} and
  Lemma \ref{lemma-substitution} we have:
  $$
  \sep{\prog_1;\prog_2}{\vars(\be^1_{k},\be^2_{k},\be^1_{p},\be^2_{p})}{\vdefs(\prog_2)}
  $$
  Thus by Lemmas \ref{lemma-noninterference} and \ref{lemma-separation} we establish
  postconditions (ii) and (iii).

  Also since $v_1$ and $v_2$ are correlated either positively or negatively with
  $\ttt{owl}(g_1)$ and $\ttt{owl}(g_2)$ respectively by precondition (ii),
  by Definition \ref{definition-gc}, precondition (i), and Lemma \ref{lemma-substitution-sim}
  we establish postcondition (i), since Definition \ref{definition-ygcgate-certification}
  requires gate output correlation with $\ttt{owl}(g)$ given any input valence conditions.
\end{proof}

\ygcencode*

\begin{proof}
  By Definition \ref{definition-ygcencode-certification} we have:
  \begin{mathpar}
    \sep{\progtt(\prog_2)}{\ttt{\{s[1,s1]\}}}{\vdefs(\prog_2)_{\{2\}}}
    
    \sep{\progtt(\prog_2)}{\ttt{\{s[2,s2]\}}}{\vdefs(\prog_2)_{\{1\}}}
  \end{mathpar}
  Since we assume that secrets are in uniform and independent marginal
  distributions a priori, and $\vars(\prog_1) \cap \vars(\prog_2) =
  \varnothing$ by condition (iv) of Definition
  \ref{definition-ygcencode-certification} and assumptions of
  well-typedness, conditions (ii-iii) follow by Lemmas
  \ref{lemma-noninterference} and \ref{lemma-separation}. Also by
  Definition \ref{definition-ygcencode-certification} we have
  $\gc{\prog_2}{v_1}{s_1}$ and $\gc{\prog_2}{v_1}{s_1}$, so also
  $\gc{\prog_1;\prog_2}{v_1}{s_1}$ and
  $\gc{\prog_1;\prog_2}{v_1}{s_1}$ by condition (iv) of Definition
  \ref{definition-ygcencode-certification} and Lemmas
  \ref{lemma-noninterference} and \ref{lemma-separation}.
\end{proof}


\ygcnimo*

\begin{proof}
  By assumptions of well-typedness we
  have for some $\be_k$, $\be_p$, and $\be$:
  $$
  \config{\varnothing}{\eassign{\outv}{\ttt{decode}(g,e)}} \redxs
  \config{\prog_1}{\eassign{\outv}{\ttt{decode}(g,\{ \ttt{k = }\be_k;  \ttt{p = }\be_p\})}}
  \redxs \config{\prog_1;\prog_2}{\eassign{\outv}{\be}}
  $$
  Let $\iov(\prog_1) = S \cup V$.
  By Lemmas \ref{lemma-ygc-encode} and \ref{lemma-ygc-preservation} we
  have $\sep{\progd(\prog_1)}{S_{\{1\}}}{V_{\{2\}}}$ and
  $\sep{\progd(\prog_1)}{S_{\{2\}}}{V_{\{1\}}}$.
  Let $\prog$ be as defined in Definition \ref{definition-ygdecode-certification}.
  We observe:
  \begin{eqnarray*}
    & \prog_2 = \\ 
    & \prog[\be_k/\ttt{\{flip[2,gate:c.k]}][\be_p/\ttt{flip[2,gate:c.p]}]
  \end{eqnarray*}
  so also by  Definition \ref{definition-ygdecode-certification} and
  Lemma \ref{lemma-substitution} we have:
  $$
  \sep{\progtt(\prog_1;\prog_2)}{\vars(\be_k,\be_p)}{\vdefs(\prog_2)}
  $$
  so, letting $\iov(\prog_1;\prog_2) = S \cup V'$ by
  Lemmas \ref{lemma-noninterference} \ref{lemma-separation} we have
  $\sep{\progd(\prog_1;\prog_2)}{S_{\{1\}}}{V'_{\{2\}}}$ and
  $\sep{\progd(\prog_1;\prog_2)}{S_{\{2\}}}{V'_{\{1\}}}$.
  Thus, letting $\prog' \defeq (\prog_1;\prog_2;\eassign{\outv}{\be})$, for all $\beta$:
  $$\sep{\condd{\progd(\prog')}{S_{\{1\}} \cup V_{\{2\}}}{\{ \outv \mapsto \beta \}}}{S_{\{1\}}}{V_{\{2\}}}
  \quad \text{and} \quad
    \sep{\condd{\progd(\prog')}{S_{\{2\}} \cup V_{\{1\}}}{\{ \outv \mapsto \beta \}}}{S_{\{2\}}}{V_{\{1\}}}$$
  thus for all $\store \in \mems(V_2 \cup \{ \outv \})$:
  $$\condd{\progd(\prog')}{S_{\{1\}}}{\store}
  = \condd{\progd(\prog')}{S_{\{1\}}}{\store_{\{\outv\}}}$$
  and for all $\store \in \mems(V_1 \cup \{ \outv \})$:
  $$\condd{\progd(\prog')}{S_{\{2\}}}{\store}
  = \condd{\progd(\prog')}{S_{\{2\}}}{\store_{\{\outv\}}}$$
  establishing the result by Lemma \ref{lemma-nimo} and Definition \ref{definition-NIMO}.
\end{proof}



\end{document}
\endinput
