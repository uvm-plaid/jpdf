\begin{fpfig}{Yao's Garbled Circuits, selected auxiliary functions.}{fig-ygc-aux}
{\footnotesize
\begin{verbatimtab}
  keygen(gid, b1, b2) { select4(b1,b2,H[gid || 1],H[gid || 2],H[gid || 3],H[gid || 4]) }
  
  keysgen(gid, b1, b2)
  {
    let k11 = keygen(gid,b1,b2) in
    let k10 = keygen(gid,b1,not b2) in
    let k01 = keygen(gid,not b1,b2) in
    let k00 = keygen(gid,not b1,not b2) in
    {k11 = k11; k10 = k10; k01 = k01; k00 = k00}
  }
  
  andtable(keys, bt, ap, bp)
  {
    let r11 = (keys.k11 xor bt) in 
    let r10 = (keys.k10 xor (not bt)) in
    let r01 = (keys.k01 xor (not bt)) in
    let r00 = (keys.k00 xor (not bt)) in
    permute4(ap,bp,r11,r10,r01,r00)
  }
  
  sharetable(gid, tid, table)
  {   
    v[1, gate: || gid || tid || 1] := table.v1;
    v[1, gate: || gid || tid || 2] := table.v2;
    v[1, gate: || gid || tid || 3] := table.v3;
    v[1, gate: || gid || tid || 4] := table.v4
  }

  owl(gid) {  { k = flip[2,gate: || gid || .k]; p = flip[2,gate: || gid || .p] }  }
\end{verbatimtab}
}
\end{fpfig}

\begin{fpfig}{Yao's Garbled Circuits, garbled gates and evaluation code.}{fig-ygc-gates}
{\footnotesize
\begin{verbatimtab}
  garbledecode(gid)    
  {
    let wl = owl(gid) in
    let r1 = wl.k xor true in
    let r0 = (not wl.k) xor false in
    v[1,OUTtt1] := select(wl.p,r1,r0);
    v[1,OUTtt2] := select(not wl.p,r1,r0)
  }
  
  evaldecode(wv) { wv.k xor select(wv.p,v[1,OUTtt1],v[1,OUTtt2]) }
  
  evalgate(gid, wva, wvb)  
  {
    let k = keygen(gid,wva.k,wvb.k) in
    let g = gate: || gid in
    let ct = select4(wva.p,wvb.p,
               v[1,g || tt1],v[1,g || tt2],v[1,g || tt3],v[1,g || tt4]) in
    let cp = select4(wva.p,wvb.p,
               v[1,g || pt1],v[1,g || pt2],v[1,g || pt3],v[1,g || pt4]) in
    { k = k xor ct; p = k xor cp }
  }
  
  andgate(gid, ga, gb) 
  {
    let wla = owl(ga) in
    let wlb = owl(gb) in
    let wlc = owl(gid) in
    let keys = keysgen(gid,wla.k,wlb.k) in
    sharetable(gid,tt,andtable(keys,wlc.k,wla.p,wlb.p));
    sharetable(gid,pt,andtable(keys,wlc.p,wla.p,wlb.p))
  }

  encode(sa, sb)
  {
    let owl1 = owl(sa) in
    let owl2 = owl(sb) in
    v[1,gate: || sa || 1.k] := OT(s[1,sa],owl1.k,(not owl1.k));
    v[1,gate: || sa || 1.p] := OT(s[1,sa],owl1.p,(not owl1.p));
    v[1,gate: || sb || 2.k] := select(s[2,sb],owl2.k,(not owl2.k));
    v[1,gate: || sb || 2.p] := select(s[2,sb],owl2.p,(not owl2.p));
    { wv1 = { k = v[1,gate: || sa || 1.k]; p = v[1,gate: || sa || 1.p] };
      wv2 = { k = v[1,gate: || sb || 2.k]; p = v[1,gate: || sb || 2.p] } }
  }
\end{verbatimtab}
}
\end{fpfig}

\begin{fpfig}{Yao's Garbled Circuits, selected gate library type annotations.}{fig-ygc-types}
{\footnotesize
\begin{verbatimtab}
  garbledecode : string(gid)
  
  evaldecode   : { k = bool[1]; p = bool[1] }
  
  evalgate     : string(gid) *  { k = bool[1]; p = bool[1] } * { k = bool[1]; p = bool[1] }
  
  andgate      : string(gid) * string(ga) * string(gb) 

  encode       : string(sa) * string(sb)
\end{verbatimtab}
}
\end{fpfig}
%  andtable(keys, bt, ap, bp)
%  {
%    let r11 = (keys.k11 xor bt) in 
%    let r10 = (keys.k10 xor (not bt)) in
%    let r01 = (keys.k01 xor (not bt)) in
%    let r00 = (keys.k00 xor (not bt)) in
%    permute4(ap,bp,r11,r10,r01,r00)
%  }

  
%  encode(gid, wla,wlb)
%  {
%    let wla = { k = flip[2,fwl1]; p = flip[2,pwl1] } in
%    let wlb = { k = flip[2,fwl2]; p = flip[2,pwl2] } in
%    
%    { wv1 = { k = OT[s[1,0],wla.k,(not wla.k)]; p = OT[s[1,0],wla.p,(not wla.p)]}; 
%      wv2 = { k = select[s[2,0],wlb.k,(not wlb.k)]; p = select[s[2,0],wlb.p,(not wlb.p)] } }
%  }

\section{YGC Code in Detail}
\label{section-ygc}

In Figures \ref{fig-ygc-aux} and \ref{fig-ygc-gates} we define a
codebase for Yao's garbled circuits (YGC). This definition follows the
\emph{point-and-permute} method described in \cite{evans2018pragmatic}
and elsewhere, to which the reader is referred for more in-depth
discussion.  In this implementation client 2 is the \emph{garbler} and
client 1 is the \emph{evaluator}. The garbler builds the garbled
tables and shares them with the evaluator, who then evaluates the gate
in an oblivious fashion until the final public output is generated
through decryption. Our certification method leverages this definition
of compositional units comprising complete related subprotocols--
i.e., both the garbler's construction of the gates and the evaluator's
evaluation of them. However, we note that our formal results are applicable
to the YGC style where the garbler shares the entire circuit prior to
evaluation. Our implementation is well-typed, with input type annotations
for top-level functions listed in Figure \ref{fig-ygc-types}.

The pointer bits in wire labels are used to select permuted rows in
table garblings. The key bits are used to identify a unique key for
table row in each garbled gate. Intuitively, if $\beta_1$ and
$\beta_2$ are either key or pointer bits encoding 1 on two input wire
labels to a binary gate, rows and keys in the gate are enumerated in
the order:
$$
\neg\beta_1\neg\beta_2,\ \neg\beta_1\beta_2,\ \beta_1\neg\beta_2,\ \beta_1\beta_2
$$
