\begin{fpfig}[t]{Yao's Garbled Circuits, auxiliary functions.}{fig-ygc-aux}
{\footnotesize
\begin{verbatimtab}
  keygen(gid, b1, b2) { select4(b1,b2,H[gid || 1],H[gid || 2],H[gid || 3],H[gid || 4]) }
  
  keysgen(gid, b1, b2)
  {
    let k11 = keygen(gid,b1,b2) in
    let k10 = keygen(gid,b1,not b2) in
    let k01 = keygen(gid,not b1,b2) in
    let k00 = keygen(gid,not b1,not b2) in
    {k11 = k11; k10 = k10; k01 = k01; k00 = k00}
  }
  
  andtable(keys, bt, ap, bp)
  {
    let r11 = (keys.k11 xor bt) in 
    let r10 = (keys.k10 xor (not bt)) in
    let r01 = (keys.k01 xor (not bt)) in
    let r00 = (keys.k00 xor (not bt)) in
    permute4(ap,bp,r11,r10,r01,r00)
  }
  
  sharetable(gid, tid, table)
  {   
    v[1, gate: || gid || tid || 1] := table.v1;
    v[1, gate: || gid || tid || 2] := table.v2;
    v[1, gate: || gid || tid || 3] := table.v3;
    v[1, gate: || gid || tid || 4] := table.v4
  }

  owl(gid) {  { k = flip[2,gate: || gid || .k]; p = flip[2,gate: || gid || .p] }  }
\end{verbatimtab}
}
\end{fpfig}

\begin{fpfig}[t]{Yao's Garbled Circuits, garbled gates and evaluation code.}{fig-ygc-gates}
{\footnotesize
\begin{verbatimtab}
  garbledecode(wl)
  {
    let r1 = wl.k xor true in
    let r0 = (not wl.k) xor false in
    v[1,OUTtt1] := select[wl.p,r1,r0];
    v[1,OUTtt2] := select[not wl.p,r1,r0]
  }
  
  evaldecode(wl, p) { wl.k xor select[wl.p,v[1,OUTtt1],v[1,OUTtt2]] }
  
  evalgate(gid, wla, wlb)
  {
    let k = keygen(gid,wla.k,wlb.k) in
    let ct = select4(wla.p,wlb.p,
               v[1,gid || tt1],v[1,gid || tt2],v[1,gid || tt3],v[1,gid || tt4]) in
    let cp = select4(wla.p,wlb.p,
               v[1,gid || pt1],v[1,gid || pt2],v[1,gid || pt3],v[1,gid || pt4]) in
    { k = k xor ct; p = k xor cp }
  }
  
  andgate(gid, wla, wlb, wlc) 
  {
    let keys = keysgen(gid,wla.k,wlb.k) in
    sharetable(gid,tt,andtable(keys,wlc.k,wla.p,wlb.p));
    sharetable(gid,pt,andtable(keys,wlc.p,wla.p,wlb.p))
  }

  encode(sid, owl1, owl2)
  {
    v[1,gate: || sid || 1.k] := OT(s[1,0],owl1.k,(not owl1.k));
    v[1,gate: || sid || 1.p] := OT(s[1,0],owl1.p,(not owl1.p));
    v[1,gate: || sid || 2.k] := select(s[2,0],owl2.k,(not owl2.k));
    v[1,gate: || sid || 2.p] := select(s[2,0],owl2.p,(not owl2.p));
    { wv1 = { k = v[1,gate: || sid || 1.k]; p = v[1,gate: || sid || 1.p] };
      wv2 = { k = v[1,gate: || sid || 2.k]; p = v[1,gate: || sid || 2.p] } }
  }
\end{verbatimtab}
}
\end{fpfig}

%  andtable(keys, bt, ap, bp)
%  {
%    let r11 = (keys.k11 xor bt) in 
%    let r10 = (keys.k10 xor (not bt)) in
%    let r01 = (keys.k01 xor (not bt)) in
%    let r00 = (keys.k00 xor (not bt)) in
%    permute4(ap,bp,r11,r10,r01,r00)
%  }

  
%  encode(gid, wla,wlb)
%  {
%    let wla = { k = flip[2,fwl1]; p = flip[2,pwl1] } in
%    let wlb = { k = flip[2,fwl2]; p = flip[2,pwl2] } in
%    
%    { wv1 = { k = OT[s[1,0],wla.k,(not wla.k)]; p = OT[s[1,0],wla.p,(not wla.p)]}; 
%      wv2 = { k = select[s[2,0],wlb.k,(not wlb.k)]; p = select[s[2,0],wlb.p,(not wlb.p)] } }
%  }
