\section{The $\minifed$ Language}

The $\minifed$ language provides a simple model of synchronous
protocols between a federation of \emph{clients} exchanging values in
the binary field. We will identify clients by natural numbers, and
federations- finite sets of clients- are always given statically.
As we will see, our threat model assumes a partition of the federation
into \emph{honest} $H$ and \emph{corrupt} $C$ subsets.

We model probabilistic programming via the \emph{random tape}
approach. That is, we will assume that programs can make reference to
values chosen from a uniform random distribution- coin ``flips''- via
values arbitrarily assigned in the initial program memory.  Programs
execute deterministically given the random tape. The random tape
formulation supports our automated analysis as we discuss in
section 

\subsection{Syntax} The syntax of $\minifed$, defined in
Figure \ref{fig-minifed-syntax}, includes a standard boolean algebra
with $\eand$, $\eor$, $\exor$, and $\enot$ as primitives. In addition
programs have different kinds of variables, including \emph{secrets}
$\secret{\cid}{w}$, \emph{flips} $\flip{\cid}{w}$, and \emph{views}
$\view{\cid}{w}$.  Each of these variables are indexed by an ``owner''
client $\cid$ and distinguishing string $w$. So for example,
$\secret{\ttt{1}}{\ttt{foo}}$ is client $\ttt{1}$'s secret called
$\ttt{foo}$. All clients can make reference to the \emph{oracle}
through variables $\oracle{w}$- this is a form of shared randomness
that is standard in the MPC setting \cite{XXX}.  Each client can only
compute on their own variables in \emph{expressions} $\be$, and share
values with other clients by assignment to their views as in
$\eassign{\view{\cid}{w}}{\be}$.  A \emph{protocol} $\prog$ is a
possibly empty sequence of view assignments. We will generally omit
$\varnothing$ from example code, i.e., writing $c_1;\ldots;\c_n$
instead of $c_1;\ldots;\c_n;\varnothing$, and abusing notation we will
write $\prog_1;\prog_2$ to denote the concatenation of $\prog_1$
and $\prog_2$.

We let $X$ range over sets of variables and $S,V,F$ to range over sets
of secrets, views, and flips respectively. Given a program $\prog$, we
write $\iov(\prog)$ to denote $S \cup V$ where $S$ is the set of
secret variables in $\prog$ and $V$ is the set of views in $\prog$,
and we write $\flips(\prog)$ to denote the set $F$ of flip variables
in $\prog$. For any set of variables $X$ and parties $P$, we write
$X_P$ to denote the subset of $X$ owned by any party in $P$, and we
write $X_H$ and $X_C$ to denote the subsets belonging to honest and
corrupt parties, respectively.

Expressions forms include the convenience
$\select{\be_1}{\be_2}{\be_3}$ which is essentially a conditional
expression with:
$$
\select{\be_1}{\be_2}{\be_3} \equiv (\be_1\ \eand\ \be_2)\ \eor\ (\enot\ \be_1\ \eand\ \be_3)
$$
We also include \emph{oblivious transfer} $\OT{\be_1}{\be_2}{\be_3}$ as a primitive,
with semantics similar to $\ttt{select}$ but with important nuances
related to communication between clients as we discuss more below.
As we will demonstrate, it is not necessary to include $\ttt{OT}$
as a primitive since we can implement it in a provably secure
fashion. However its inclusion simplifies our presentation and is a useful
convenience.

\subsection{Semantics}

\begin{fpfig}[t]{$\minifed$ source code syntax.}{fig-minifed-syntax}
$$
\begin{array}{rcl@{\hspace{8mm}}r}
b &\in& \{ \etrue, \efalse \} \\
w &\in& \mathrm{String} \\ 
\cid &\in& \mathrm{Clients} \subset  \mathbb{N} \\[2mm]%\qquad  (\mathrm{Clients} \subset \mathbb{N} \text{\ in\ this\ presentation})\\[2mm]
\bop &\in& \{ \eand, \eor, \exor \} \\[2mm]
\be &::=& b \mid \flip{\cid}{w} \mid \secret{\cid}{w} \mid \view{\cid}{w} \mid \oracle{w} \mid & \textit{boolean expressions}\\
& &  \enot\ \be \mid \be\ \bop\ \be \mid \select{\be}{\be}{\be} \mid \OT{\be}{\be}{\be} \\[2mm]
\instr &::=& \eassign{\view{\cid}{w}}{\be} & \textit{view assignments} \\[2mm]
\prog &::=& \varnothing \mid \instr; \prog & \textit{protocols}
\end{array}
$$ 
\end{fpfig}

\emph{Memories} are fundamental to the semantics of $\fedcat$ and
provide the random tape and secret inputs to protocols, and record
view assignments. Memories $\store$ are finite (partial) mapping from
variables to binary values $\beta \in \{0,1\}$. The \emph{domain} of a
memory is written $\dom(\store)$ and is the finite set of variables on
which the memory is defined. We write $\store\{ x \mapsto \beta\}$ for
$x\not\in\dom(\store)$ to denote the memory $\store'$ such that
$\store'(x) = \beta$ and otherwise $\store'(y) = \store(y)$ for all $y \in
\dom(\store)$.

Given a set of variables $X$, we write $\store_X$ to denoted the
memory $\store$ restricted to the domain $X$, and and we define
$\mems(X)$ as the set of all memories with domain $X$:
$$
\mems(X) \defeq \{ \store \mid \dom(\store) = X \}
$$
Thus, given a protocol $\prog$, the set of all random tapes for
$\prog$ is $\mems(\flips(\prog))$. We let $\stores$ range
over sets of memories.

Given a variable-free boolean expression $\be$, we write $\cod{\be}$
to denote the standard interpretation of $\be$ in the binary field.
With the introduction of variables to expressions, we have two
concerns. First, we need to interpret variables with respect to a
specific memory, and second, we need to ensure that variables are used
``legally''. That is, since expressions define local computation, all
variables used in an expression must belong to the same client.  Thus,
we denote interpretation of expressions $\be$ possibly containing
variables as $\lcod{\store,\be}{\cid}$, where $\store$ associates
variables with values and all variables must be owned by client
$\cid$. This is defined in Figure \ref{fig-minifed-interp}.

\begin{fpfig}[t]{$\minifed$ expression interpretation.}{fig-minifed-interp}
\begin{eqnarray*}
\lcod{\store, \etrue}{\cid} &=& 1\\
\lcod{\store, \efalse}{\cid} &=& 0\\
\lcod{\store, \flip{\cid}{w}}{\cid} &=& \store(\flip{\cid}{w})\\
\lcod{\store, \secret{\cid}{w}}{\cid} &=& \store(\secret{\cid}{w})\\
\lcod{\store, \view{\cid}{w}}{\cid} &=& \store(\view{\cid}{w})\\
\lcod{\store, \oracle{w}}{\cid} &=& \store(\oracle{w})\\
\lcod{\store, \enot\ \be}{\cid} &=& \cod{\enot\ \lcod{\store,\be}{\cid}}\\
\lcod{\store, \be_1\ \mathit{binop}\ \be_2}{\cid} &=&
    \cod{\lcod{\store,\be_1}{\cid}\ \mathit{binop}\ \lcod{\store,\be_2}{\cid}}\\
\lcod{\store, \select{\be_1}{\be_2}{\be_3}}{\cid} &=&
             \begin{cases}
                \lcod{\store,\be_2}{\cid} & \text{if\ } \lcod{\store,\be_1}{\cid}\\
                \lcod{\store,\be_3}{\cid} & \text{if\ } \neg\lcod{\store,\be_1}{\cid}
             \end{cases}%\\
%\lcod{\store, \OT{\be_1}{\be_2}{\be_3}}{\cid} &=&
%             \begin{cases}
%                \lcod{\store,\be_2}{\cid'} & \text{if\ } \lcod{\store,\be_1}{\cid}\\
%                \lcod{\store,\be_3}{\cid'} & \text{if\ } \neg\lcod{\store,\be_1}{\cid}
%             \end{cases}
\end{eqnarray*}
\end{fpfig}

Evaluation of configurations is then defined via a small-step reduction relation $\redx$.
This is defined in the obvious manner for view assignments other than through
$\ttt{OT}$- however note that reduction requires that views are never reassigned. 
\begin{mathpar}
  (\store, \eassign{\view{\cid}{w}}{\be};\prog) \redx (\extend{\store}{\view{\cid}{w}}{\lcod{\store,\be}{\cid'}}, \prog)
\end{mathpar}
For view assignments through $\ttt{OT}$, we define a different reduction rule that
captures the appropriate semantics- that is, the selection bit is \emph{not} communicated
to the sender (through a view assignment), and only the selected bit is sent to the receiver:
\begin{mathpar}
  \inferrule
  {\beta = \text{if\ } \lcod{\store,\be_1}{\cid}  \text{\ then\ } \lcod{\store,\be_2}{\cid'} \text{\ else\ } \lcod{\store,\be_3}{\cid'}}
      {(\store, \eassign{\view{\cid}{w}}{\OT{\be_1}{\be_2}{\be_3}};\prog) \redx (\extend{\store}{\view{\cid}{w}}{\beta}, \prog)}
\end{mathpar}
We define $\redxs$ as the reflexive, transitive closure of $\redx$.

Given $\prog$ with $\iov(\prog) = S \cup V$ and $\flips(\prog) = F$,
any execution of $\prog$ is assumed to be an evaluation of a
configuration $\config{\store}{\prog}$ where $\store \in \mems(S \cup
F)$- that is, with secrets and the random tape as inputs- and where
$\config{\store}{\prog} \redxs \config{\store'}{\varnothing}$ it is
the case that $\store' \in \mems(S \cup F \cup V)$.

\subsection{Examples}

\begin{example}[One-Time Pad]
  \label{example-otp}
We can model secure communication between two parties
using symmetric key encryption over an insecure channel, for
example. In the following program, we assume $H = \{ 1,2 \}$ and
$C = \{ 0 \}$. Party 1 first shares the randomly generated
key $\fx{1}{0}$ with party 2. Then, party 1 encrypts
its secret $\sx{1}{0}$ and sends it to party 0,
which relays the ciphertext in $\vx{0}{0}$ to
party 2. Finally, party 2 decrypts\footnote{Recall we are using
a random tape approach, so the value of $\fx{1}{0}$ remains
consistent during any run the program, though that value
is randomly determined by the initial memory in a given
run} the ciphertext and stores the plaintext in $\vx{2}{2}$. 
\begin{verbatimtab}
v[2,0] := f[1,0];
v[0,0] := f[1,0] xor s[1,0];
v[2,1] := v[0,0];
v[2,2] := v[2,1] xor v[2,0]
\end{verbatimtab}
This protocol is \emph{correct} because $\vx{2}{2}$ is exactly
correlated with the secret $\sx{1}{0}$, and it is \emph{secure}
because the corrupt view $\vx{0}{0}$ in isolation is probabilistically
independent of the secret $\sx{1}{0}$, while serving as a
communication channel for its encryption. 
\end{example}

\begin{example}[Lambda-obliv]
  \label{example-lambda-obliv}

  \begin{verbatimtab}
v[0,0] := select(s[1,0], flip[1,0], flip[1,1] );
v[0,1] := flip[1,0]
\end{verbatimtab}

\begin{verbatimtab}
v[0,0] := select(s[1,0],select(flip[1,0], flip[1,0], flip[1,1]),flip[1,2])
\end{verbatimtab}
\end{example}

\begin{example}[Oblivious Transfer]
  \label{example-lambda-OT}

\begin{verbatimtab}
v[2, rd] := select[flip[2, d], H[r1], H[r0]];
v[1, e] := s[2, c] xor flip[2, d];
v[2, f0] := s[1, m0] xor select[v[1, e], H[r1], H[r0]];
v[2, f1] := s[1, m1] xor select[not v[1, e], H[r1], H[r0]];
v[2, mc] := (select[s[2, c], v[2, f1], v[2, f0]] xor v[2, rd])
\end{verbatimtab}
\end{example}

\begin{example}[Additive Secret Sharing]
    \label{example-lambda-he}
Another 
example of an MPC protocol is additive secret sharing, where $k$
parties split up their secrets into \emph{shares}, with the property
that all $k$ shares are required to reconstruct the secret and any
fewer reveals nothing. In $\fedcat$, we can use $\ttt{xor}$, again, to
generate shares. Here, party 1 generates 3 shares of the secret
$\sx{1}{1}$- one is given to party 2 through view $\vx{2}{s1}$, one is
given to party 3 through view $\vx{3}{s1}$, and party 1 keeps
$\fx{1}{s1}$ for itself:
\begin{verbatimtab}
v[2,s1] := flip[1,1] xor flip[1,s1] xor s[1,1];
v[3,s1] := flip[1,1];
\end{verbatimtab}
Note that $\vx{2}{s1}$ and $\vx{3}{s1}$ are both in uniform
random distributions.

Another property of $\ttt{xor}$ is that it is additively homomorphic,
while functioning as addition in the boolean algebra (i.e., the binary
field). That is, given:
\begin{mathpar}
  x_1\ \ttt{xor}\ y_1 = z_1

  x_2\ \ttt{xor}\ y_2 = z_2
\end{mathpar}
then we have:
$$
(x_1\ \ttt{xor}\ x_2)\ \ttt{xor}\ (y_1\ \ttt{xor}\ y_2) = z_1\ \ttt{xor}\ z_2
$$
Thus, we can obtain secret 3-party summation (xor) by continuing
the protocol begun above- parties 2 and 3 can share their secrets
in the same fashion:
\begin{verbatimtab}
v[1,s2] := flip[2,1] xor flip[2,s2] xor s[2,1];
v[3,s2] := flip[2,1];

v[1,s3] := flip[3,1] xor flip[3,s3] xor s[3,1];
v[2,s3] := flip[3,1];
\end{verbatimtab}
Then, assuming that party 0 is the ``public party'', each party sums
its own shares, and then the sum of sum of shares is revealed
in the output view $\vx{0}{output}$. By additive homomorphism of
$\ttt{xor}$, this output is the sum of the 3 secrets, but
the protocol reveals nothing to the adversary, even if one
(or even two, in fact) of $\{1,2,3\}$ are corrupt. 
\begin{verbatimtab}
v[0,ss1] := v[1,s2] xor v[1,s3] xor flip[1,s1];
v[0,ss2] := v[2,s1] xor v[2,s3] xor flip[2,s2];
v[0,ss3] := v[3,s1] xor v[3,s2] xor flip[3,s3];

v[0,output] := v[0,ss1] xor v[0,ss2] xor v[0,ss3]
\end{verbatimtab}

Because this protocol correctly implements 3-party addition in the
binary field, it can reveal information to the adversary. For example,
suppose that $1 \in C$, and $\sx{1}{1}$ is $1$, and after running the
protocol suppose that $\vx{0}{output}$ is 0. Then the adversary knows
that 2 and 3's secrets are either 1 and 0, or 0 and 1, with 50/50
probability, and definitely neither 0 and 0, nor 1 and 1. However, the
protocol does not reveal anything more through adversarial views than
what the output reveals.
\end{example}
