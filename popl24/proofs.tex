\section{Supplementary Proofs}
\label{section-proofs}

\nimosecure*

\begin{proof}
  Let $H$ and $C$ be an arbitrary partition of the federation and
  suppose $\prog$ satisfies noninterference modulo output. Let
  $\iov(\prog) = (V,S)$ and $\flips(\prog) = F$ with output view $\outv
  \in V$ and let $\store$ be an arbitrary element of $\mems(S \cup
  F)$. Then the distribution of adversarial views in the real world
  is, by definition,
  $\condd{\progd(\prog)}{V_C}{\store_S}$.

  The simulator is given both $\store_{S_C}$ and
  $\idealf(\store_S)$.  The simulation $\SIM_C(\store_{S_C},
  \idealf(\store_S))$ can be defined as follows. First, some $\store'
  \in \ik(\idealf,\idealf(\store))$ is randomly chosen such that
  $\store' =_C \store_S$, as is a random tape $\store'' \in
  \mems(F)$\footnote{The real/ideal model allows consultation of a
  ``Random Oracle''.}. Then, the run of $(\store' \cup \store'',
  \prog)$ is evaluated in simulation, yielding $(\store^{\SIM},\varnothing)$.
  The simulation returns $\store^{\SIM}_{V_C}$ as a result.

  Now, since the random tape is selected in simulation from the same distribution
  that we assume for the real world, after selection of $\store'$ the
  probability that any particular $\store^{\SIM}_{V_C}$ is returned is by definition
  $
  \progd(\prog)(\store^{\SIM}_{V_C} | \store')
  $.
  Furthermore, since we assume that $\prog$ correctly implements $\idealf$, this
  means
  $
  \store_S =_C \store' \wedge
     \condd{\progd(\prog)}{\{ \outv \}}{\store_S} =
     \condd{\progd(\prog)}{\{ \outv \}}{\store'}
  $,
  so by the definition of noninterference modulo output,
  any choice of $\store'$ yields the same distribution $\condd{\progd(\prog)}{V_C}{\store_S}$.
  Therefore by definition
  $
   \progd(\SIM_C(\store_{S_C},\idealf(\store))) = \condd{\progd(\prog)}{V_C}{\store_S}
  $.
\end{proof}

We observe the following properties of separation which are borrowed from prior work
\cite{barthe2019probabilistic} and which will be used frequently in proofs. 
\begin{lemma}
  \label{lemma-separation}
  The following properties hold:
  \begin{enumerate}
  \item $\sep{\pmf}{Y}{Z}$ iff $\sep{\pmf}{Z}{Y}$
  \item $\vc{\pmf}{x}{y}$ if  $\vc{\pmf}{y}{x}$
  \item $\vc{\pmf}{x}{y}$ and $\vc{\pmf}{y}{z}$ imply $\vc{\pmf}{x}{z}$
  \item $\sep{\pmf}{X}{(Y \cup Z)}$ implies $(\sep{\pmf}{X}{Y}$ and $\sep{\pmf}{X}{Z})$
  \item $(\sep{\pmf}{X}{Y}$ and $\sep{\pmf}{(X \cup Y)}{Z})$ implies $\sep{\pmf}{X}{(Y \cup Z)}$ 
  %\item If $\prog_1;\prog_2$ is safe and $\vars(\prog_1) \cap \vars(\prog_2) = \varnothing$
  %  then $\sep{\progtt(\prog_1;\prog_2)}{\vars(\prog_1)}{\vars(\prog_2)}$.
  \end{enumerate}
\end{lemma}
The following property also follows by results in \cite{barthe2019probabilistic} and will
be useful to make constructions that demonstrate variable separation.
\begin{lemma}
  \label{lemma-sepjoin}
  $\sep{\progtt(\prog)}{X}{Y}$ iff for all 
  $\store^1, \store^2 \in \runs(\prog)$ there exists
  $\store \in \runs(\prog)$ with
  $\store^1_{X} \cap \store^2_{Y} \subseteq \store$.
\end{lemma}

\begin{lemma}
  \label{lemma-presub}
  Given $\be_1$, $\be_2$, and $i$ where $\vars(\be_1) \cap
  \vars(\be_2) = \varnothing$ and $\lcod{\store_1,\be_1}{i} = \beta$
  with $\store_1(x) = \lcod{\store_2,\be_2}{i}$ and
  $\dom(\store_1) = \vars(\be_1)$ and $\dom(\store_2) = \vars(\be_2)$.
  Then $\lcod{\store_1\cap\store_2,\be_1[\be_2/x]}{i} = \beta$.
\end{lemma}
\begin{proof}
By straightforward structural induction on $\be$.
\end{proof}

\substar*

\begin{proof}
  Suppose on the contrary it was not the case that
  $\sep{\progtt(\prog_1;\prog_2[\be/f])}{\vars(\be)}{X}$.
  Then by Lemma \ref{lemma-sepjoin} there exists $\store_1,\store_2
  \in \runs(\prog_1;\prog_2[\be/f])$ with no $\store \in
  \runs(\prog_1;\prog_2[\be/f])$ such that $\store^1_{\vars(\be)}
  \cap \store^2_X \subseteq \store$.  But also by
  assumption and Lemma \ref{lemma-sepjoin} for any $\beta$ there
  exists $\store' \in \runs(\prog_2)$ with $\{ f \mapsto \beta \} \cap
  \store^2_X \subseteq \store'$. So in particular, we
  have $\{ f \mapsto \lcod{\store^1_{\vars(\be)},\be}{i} \} \cap
  \store^2_X \subseteq \store'$ for any $i$. This, the
  assumption $\vars(\prog_1,\be) \cap \vars(\prog_2) = \varnothing$,
  and application of Lemma \ref{lemma-presub} leads to the consequence
  that there exists $\store \in \runs(\prog_1;\prog_2[\be/f])$ with
  $\store \supseteq \store^1_{\vars(\be)} \cap
  \store^2_X$ given the assumption $\vars(\prog_1,\be)
  \cap \vars(\prog_2) = \varnothing$, which is a contradiction.
\end{proof}


\gmwencode*

\begin{proof}
  By assumptions of well-typedness we have:
  $$
  \begin{array}{rcl}
    v &=&  \ttt{\{shares1 = }v_1\ttt{; shares2 = }v_2\ttt{\}}\\
    v_1 &=& \ttt{\{ c1 = v[1,"s1out"]; c2 = v[2,"s1out"] \}}\\
    v_2 &=& \ttt{\{ c1 = v[1,"s2out"]; c2 = v[2,"s2out"] \}}
  \end{array}
  $$
  And by Definition \ref{definition-gmwencode-certification} we have:
  \begin{mathpar}
    \sep{\progtt(\prog)}{\ttt{\{s[1,"s1"]\}}}{\ttt{\{v[2,"s1out"],v[2,"s2out"]\}}}
    
    \sep{\progtt(\prog)}{\ttt{\{s[2,"s2"]\}}}{\ttt{\{v[1,"s1out"],v[1,"s2out"]\}}}
  \end{mathpar}
  Since we assume that secrets are in uniform and independent marginal
  distributions a priori, and $\vars(\prog_1) \cap \vars(\prog_2) =
  \varnothing$ by condition (iii) of Definition 
  \ref{definition-gmwencode-certification} and assumptions of well-typedness, 
  the result follows by preconditions and Lemma \ref{lemma-separation}.
\end{proof}

\ygcgate*

\begin{proof}
  Given preconditions we have $v_1 =  \{ \ttt{k = }\be^1_{k}; \ttt{p = }\be^1_{p} \}$ and
  $v_2 = \{ \ttt{k = }\be^2_{k}; \ttt{p = }\be^2_{p} \}$ for some
  $\be^1_{k}$,$\be^2_{k}$,$\be^1_{p}$,$\be^2_{p}$ with correlations as per Definition
  \ref{definition-gc}.
  %\begin{eqnarray*}
  %  v_1 &=& \{ \ttt{k = }\be^1_{k}; \ttt{p = }\be^1_{p} \} \\
  %  v_2 &=& \{ \ttt{k = }\be^2_{k}; \ttt{p = }\be^2_{p} \}
  %\end{eqnarray*}
  Let $\prog$ be as defined in Definition \ref{definition-ygcgate-certification}.
  We observe:
  \begin{eqnarray*}
    &\prog_2 = \\
    &{\small \prog[\be^1_{k}/\ttt{flip[2,gate:a.k]}][\be^1_{p}/\ttt{flip[2,gate:a.p]}][\be^2_{k}/\ttt{flip[2,gate:b.k]}][\be^1_{p}/\ttt{flip[2,gate:b.p]}]}
  \end{eqnarray*}
  Given that $g_1$ and $g_2$ are distinct and not wired in $\prog_1$
  we are assured that $\ttt{owl}(g_1)$ and $\ttt{owl}(g_1)$ are in
  independent uniform distributions, and given that $g \not\in
  \prog_1$ we are assured that $\ttt{owl}(g)$ contains entirely fresh
  flips. Thus by condition (i) of Definition \ref{definition-ygcgate-certification} and
  Lemma \ref{lemma-substitution} we have:
  $$
  \sep{\prog_1;\prog_2}{\vars(\be^1_{k},\be^2_{k},\be^1_{p},\be^2_{p})}{\vdefs(\prog_2)}
  $$
  Thus by Lemmas \ref{lemma-noninterference} and \ref{lemma-separation} we establish
  postconditions (ii) and (iii).

  Also since $v_1$ and $v_2$ are correlated either positively or negatively with
  $\ttt{owl}(g_1)$ and $\ttt{owl}(g_2)$ respectively by precondition (ii),
  by Definition \ref{definition-gc}, precondition (i), and Lemma \ref{lemma-substitution-sim}
  we establish postcondition (i), since Definition \ref{definition-ygcgate-certification}
  requires gate output correlation with $\ttt{owl}(g)$ given any input valence conditions.
\end{proof}

\ygcencode*

\begin{proof}
  By Definition \ref{definition-ygcencode-certification} we have:
  \begin{mathpar}
    \sep{\progtt(\prog_2)}{\ttt{\{s[1,"s1"]\}}}{\vdefs(\prog_2)_{\{2\}}}
    
    \sep{\progtt(\prog_2)}{\ttt{\{s[2,"s2"]\}}}{\vdefs(\prog_2)_{\{1\}}}
  \end{mathpar}
  Since we assume that secrets are in uniform and independent marginal
  distributions a priori, and $\vars(\prog_1) \cap \vars(\prog_2) =
  \varnothing$ by condition (iv) of Definition
  \ref{definition-ygcencode-certification} and assumptions of
  well-typedness, conditions (ii-iii) follow by Lemmas
  \ref{lemma-noninterference} and \ref{lemma-separation}. Also by
  Definition \ref{definition-ygcencode-certification} we have
  $\gc{\prog_2}{v_1}{s_1}$ and $\gc{\prog_2}{v_1}{s_1}$, so also
  $\gc{\prog_1;\prog_2}{v_1}{s_1}$ and
  $\gc{\prog_1;\prog_2}{v_1}{s_1}$ by condition (iv) of Definition
  \ref{definition-ygcencode-certification} and Lemmas
  \ref{lemma-noninterference} and \ref{lemma-separation}.
\end{proof}


\ygcnimo*

\begin{proof}
  By assumptions of well-typedness we
  have for some $\be_k$, $\be_p$, and $\be$:
  $$
  \config{\varnothing}{\eassign{\outv}{\ttt{decode}(g,e)}} \redxs
  \config{\prog_1}{\eassign{\outv}{\ttt{decode}(g,\{ \ttt{k = }\be_k;  \ttt{p = }\be_p\})}}
  \redxs \config{\prog_1;\prog_2}{\eassign{\outv}{\be}}
  $$
  Let $\iov(\prog_1) = S \cup V$.
  By Lemmas \ref{lemma-ygc-encode} and \ref{lemma-ygc-preservation} we
  have $\sep{\progd(\prog_1)}{S_{\{1\}}}{V_{\{2\}}}$ and
  $\sep{\progd(\prog_1)}{S_{\{2\}}}{V_{\{1\}}}$.
  Let $\prog$ be as defined in Definition \ref{definition-ygdecode-certification}.
  We observe:
  \begin{eqnarray*}
    & \prog_2 = \\ 
    & \prog[\be_k/\ttt{\{flip[2,gate:c.k]}][\be_p/\ttt{flip[2,gate:c.p]}]
  \end{eqnarray*}
  so also by  Definition \ref{definition-ygdecode-certification} and
  Lemma \ref{lemma-substitution} we have:
  $$
  \sep{\progtt(\prog_1;\prog_2)}{\vars(\be_k,\be_p)}{\vdefs(\prog_2)}
  $$
  so, letting $\iov(\prog_1;\prog_2) = S \cup V'$ by
  Lemmas \ref{lemma-noninterference} \ref{lemma-separation} we have
  $\sep{\progd(\prog_1;\prog_2)}{S_{\{1\}}}{V'_{\{2\}}}$ and
  $\sep{\progd(\prog_1;\prog_2)}{S_{\{2\}}}{V'_{\{1\}}}$.
  Thus, letting $\prog' \defeq (\prog_1;\prog_2;\eassign{\outv}{\be})$, for all $\beta$:
  $$\sep{\condd{\progd(\prog')}{S_{\{1\}} \cup V_{\{2\}}}{\{ \outv \mapsto \beta \}}}{S_{\{1\}}}{V_{\{2\}}}
  \quad \text{and} \quad
    \sep{\condd{\progd(\prog')}{S_{\{2\}} \cup V_{\{1\}}}{\{ \outv \mapsto \beta \}}}{S_{\{2\}}}{V_{\{1\}}}$$
  thus for all $\store \in \mems(V_2 \cup \{ \outv \})$:
  $$\condd{\progd(\prog')}{S_{\{1\}}}{\store}
  = \condd{\progd(\prog')}{S_{\{1\}}}{\store_{\{\outv\}}}$$
  and for all $\store \in \mems(V_1 \cup \{ \outv \})$:
  $$\condd{\progd(\prog')}{S_{\{2\}}}{\store}
  = \condd{\progd(\prog')}{S_{\{2\}}}{\store_{\{\outv\}}}$$
  establishing the result by Lemma \ref{lemma-nimo} and Definition \ref{definition-NIMO}.
\end{proof}
