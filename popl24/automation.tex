\section{Automating Security Verification}
\label{section-automation}

%We first explore a brute force approach to automatically verifying
%program behaviour, the first step of which is to compute the basic
%distribution of a program $\prog$. The program distribution
%$\progd(\prog)$ can obviously be automatically derived from that, as
%can be any of its marginal or conditional distributions. In this
%Section, we consider two automated mechanisms for deriving
%$\progd(\prog)$ in this way. The first is through a straightforward
%computation of all final program memories.  The second approach
%rewrites $\minifed$ programs to stratified Datalog programs, that are
%amenable HPC optimizations such as parallelization and GPU matrix
%computations as shown in recent work
%\cite{sakama2017linear,aspis2018linear,nguyen2022enhancing,nguyen2021efficient}.

%Here we describe a straightforward technique for obtaining program
%distributions $\progd(\prog)$ through direct computation of
%$\runs(\prog)$. Although we could obtain $\runs(\prog)$ simply by
%executing $\prog$ given all random tapes and input secrets, the method
%we describe here is empirically more efficient (though still
%exponential in the length of the random tape), and also supports the
%translation to Datalog described in Section
%\ref{section-bruteforce-datalog}. An implementation of these
%techniques along with Examples from Sections
%\ref{section-minicat-examples} and \ref{section-metalang-ygc} is
%available online\footnote{URL redacted for double-blind review.}.

Our first step in automating security analysis is to show how to
compute the basic program distribution. Recall that basic
distributions are based on the memories resulting from execution of an
$\minicat$ program, which contain a record of all client views and the
public output.  Given $\prog$ we denote this as $\runs(\prog)$, and we
denote the basic distribution of a program as $\progtt(\prog)$.
\begin{definition}
  Given $\prog$ with $\iov(\prog) = S \cup V$ and $\flips(\prog) = F$:
  $$
  \runs(\prog) \defeq \{ \store \in \mems(S\cup F \cup V) \mid \config{\store_{S \cup F}}{\prog} \redxs \config{\store}{\varnothing} \}
  $$
  We write $\progtt(\prog)$ to denote the \emph{basic distribution} of
  $\prog$, where given $\iov(\prog) = S \cup V$ and $\flips(\prog) = F$
  we have for all $\store \in \mems(S \cup V \cup F)$:
  $$
  \progtt(\prog)(\store) =  1 / 2^{|S\cup F|} \ \text{if}\ \store \in \runs(\prog), \text{otherwise}\ 0
  $$
\end{definition}

Any marginalization or conditioning of $\progtt(\prog)$ can be easily
obtained algorithmically, in particular $\progd(\prog) =
\margd{\progtt(\prog)}{\iov(\prog)}$.  As already discussed in Section
\ref{section-hyperprop} and as we will discuss more below,
crucial properties are predicated on marginalizations
and conditionings of basic distributions. So calculating
$\runs(\prog)$ establishes the foundation for our analysis.

\begin{fpfig}[t]{Truth table visualizations of $\runs(\prog_{\ref{example-otp}})$ (L) and
    $\runs(\prog_{\ref{example-lambda-obliv}(a)})$ (R).}{fig-basic-distributions}
{\footnotesize
  $$
  \begin{array}{cc}
    \begin{array}{cccccc}
      \verb+s[1,0]+ & \verb+f[1,0]+ & \verb+v[2,0]+  & \verb+v[2,1]+ & \verb+v[2,2]+ & \verb+v[0,0]+\\
      \hline
      0 & 0 & 0 & 0 & 0 & 0 \\ 
      0 & 1 & 1 & 1 & 0 & 1 \\ 
      1 & 0 & 0 & 1 & 1 & 1 \\ 
      1 & 1 & 1 & 0 & 1 & 0
    \end{array}
    & 
    \begin{array}{ccccc}
      \verb+s[1,s]+ & \verb+f[1,sx]+ & \verb+f[1,sy]+ & \verb+v[0,0]+ & \verb+v[0,1]+ \\
      \hline
      0 & 0 & 0 & 0 & 0 \\ 
      0 & 0 & 1 & 1 & 0 \\ 
      0 & 1 & 0 & 0 & 1 \\ 
      0 & 1 & 1 & 1 & 1 \\
      1 & 0 & 0 & 0 & 0 \\ 
      1 & 0 & 1 & 0 & 0 \\ 
      1 & 1 & 0 & 1 & 1 \\ 
      1 & 1 & 1 & 1 & 1  
    \end{array}
  \end{array}
  $$
}
\end{fpfig}

Intuitively, we can represent $\runs(\prog)$ in tabular form, as
illustrated in Figure \ref{fig-basic-distributions} for Examples
\ref{example-otp} and \ref{example-lambda-obliv}(a). So a basic
strategy is to directly compute this truth table for any given
program. We first describe an approach where we compute
$\runs(\instr_1,\ldots,\instr_n)$ iteratively for the subprograms
$\instr_1,\ldots,\instr_i$ for $1 \le i \le n$, by individually
computing the truth table of expressions $\be_i$ for each $\instr_i =
(\eassign{v_i}{\be_i})$, and joining this with
$\runs(\instr_1,\ldots,\instr_{i-1})$.

In Figure \ref{fig-solve} we define the algorithm
$\solve{\stores}{\be}$ which filters a given $\stores$ to obtain the
subset whose elements satisfy $\be$ (make it true). It's correctness
is characterized as follows.
\begin{lemma}
  \label{lemma-solves}
  For all $\stores$ and $\be$ with $\vars(\be) \subseteq \dom(\stores)$,
  $(\solve{\stores}{\be}) = \{ \store \in \stores \ \mid\ \lcod{\store,\be}{\cid} = 1 \}$
  for some $\cid$.
\end{lemma}
\begin{proof}
  Since we assume safety of programs we can assume that all variables in $\be$ have the
  same owner $\cid$. The result otherwise follows in a straightforward manner by induction
  on $\be$. 
\end{proof}

\begin{fpfig}[t]{Filtering memories that satisfy a boolean expression.}{fig-solve}
{\small
\begin{eqnarray*}
\solve{\stores}{\etrue} &=& \stores\\
\solve{\stores}{\efalse} &=& \varnothing\\
\solve{\stores}{\flip{\cid}{w}} &=& \{ \store \in \stores \mid \store(\flip{\cid}{w}) \} \\
\solve{\stores}{\secret{\cid}{w}} &=& \{ \store \in \stores \mid \store(\secret{\cid}{w}) \} \\
\solve{\stores}{\view{\cid}{w}} &=& \{ \store \in \stores \mid \store(\view{\cid}{w}) \} \\
\solve{\stores}{\oracle{w}} &=& \{ \store \in \stores \mid \store(\oracle{w}) \} \\
\solve{\stores}{(\enot\ \be)} &=& \stores - (\solve{\stores}{\be})\\
\solve{\stores}{(\be_1\ \eand\ \be_2)} &=& (\solve{\stores}{\be_1}) \cap (\solve{\stores}{\be_2}) \\
\solve{\stores}{(\be_1\ \eor\ \be_2)} &=& (\solve{\stores}{\be_1}) \cup (\solve{\stores}{\be_2}) \\
\solve{\stores}{(\be_1\ \exor\ \be_2)} &=&
 ((\solve{\stores}{\be_1}) \cap (\stores - \solve{\stores}{\be_2})) \cup\\
 && ((\stores - \solve{\stores}{\be_1}) \cap (\solve{\stores}{\be_2})) \\
\solve{\stores}{\select{\be_1}{\be_2}{\be_3}} &=&
 ((\solve{\stores}{\be_1}) \cap (\solve{\stores}{\be_2})) \cup \\
 && ((\stores - \solve{\stores}{\be_1}) \cap (\solve{\stores}{\be_3})) \\
\solve{\stores}{\OT{\be_1}{\be_2}{\be_3}} &=&
 ((\solve{\stores}{\be_1}) \cap (\solve{\stores}{\be_2})) \cup\\
 && ((\stores - \solve{\stores}{\be_1}) \cap (\solve{\stores}{\be_3}))
\end{eqnarray*}
}
\end{fpfig}

To obtain $\runs(\prog)$, we can perform a standard left folding of $\solve{\!}{\!}$
across $\prog$ as a list of view definitions, inductively extending
memories with valid view assignments in the order of their definition.
We denote this computation as $\cruns(\prog)$.
\begin{lemma}
  \label{lemma-cruns}
  Given $\prog$ where $\iov(\prog) = S \cup V$ and $\flips(\prog) = F$. Define:
  \begin{eqnarray*}
    {tt}\ \ \stores\ (\eassign{\view{\cid}{w}}{\be}) &\defeq& \begin{array}{l}
      \mathrm{let}\ \stores' = \solve{\stores}{\be} \ \mathrm{in}\\
      \ \ \{\extend{\store}{\view{\cid}{w}}{1} \mid \store \in \stores' \}\ \cup\\
      \ \ \{\extend{\store}{\view{\cid}{w}}{0} \mid \store \in \stores - \stores' \}\end{array}\\[2mm]
    \cruns(\prog) &\defeq& \mathit{foldl}\ {tt}\ \mems(S \cup F)\ \prog
  \end{eqnarray*}
  Then $\cruns(\prog) = \runs(\prog)$.
\end{lemma}
\begin{proof}
  By Lemma \ref{lemma-solves} and induction on the length of $\prog$. 
\end{proof}
The above yields an algorithm for computing basic distributions and
thus program distributions. 
\begin{lemma}
  \label{lemma-cprogtt}
  Given $\prog$ where $\iov(\prog) = S \cup V$ and $\flips(\prog) = F$. Define
  $\cprogtt(\prog)$ as the distribution $\pmf$
  where for all $\store \in \mems(S\cup F \cup V)$:
  $$
  \pmf(\store) = 1 / 2^{|S\cup F|} \ \text{if}\ \store \in \cruns(\prog), \text{otherwise}\ 0
  %\begin{cases}1/2^{|S \cup F|} & \text{if\ } \store \in
  %  \cruns(\prog) \\ 0 & \text{otherwise} \end{cases} 
  $$
  Then $\cprogtt(\prog) = \progtt(\prog)$ and $\margd{\cprogtt(\prog)}{S\cup V} = \progd(\prog)$.
\end{lemma}
\begin{proof}
  Immediately by Lemma \ref{lemma-cruns} and Definition \ref{def-progd}.
\end{proof}
A full implementation of these methods is available in our online
repository \cite{jpdf-github}. To
avoid notational confusion we will continue to refer to $\progtt$ and
$\cruns$ in our subsequent discussion of automated verification but
with the understanding that both can be computed using the above
algorithms.

Since $|\runs(\prog)|$ grows exponentially in the size of
$\flips(\prog)$, scalability to larger programs is a concern.  However
computation of $\runs$ is amenable to optimizations, especially
parallelization- rather than computing the truth table
``horizontally'', it can be computed ``vertically''. We can rewrite
$\minifed$ programs $\prog$ to stratified Datalog programs, where
$\runs(\prog)$ is equivalent to set of least Herbrand models obtained
by representing different random tapes and input secrets as different
fact bases. Details of this rewrite are provided in Appendix
\ref{section-bruteforce}. The model for each fact base can be computed
independently and in parallel, with the potential to apply other
matrix optimizations developed in recent work on HPC for Datalog
\cite{sakama2017linear,aspis2018linear,nguyen2022enhancing,nguyen2021efficient}.

\subsection{Certification of Extensional Properties}
\label{section-automation-extensional}

We will use \emph{certification} to refer to algorithmic methods of
verifying properties of $\minifed$ protocols, vs.~manual methods. For
the sake of clarity we make the following definition:
\begin{definition}
  A predicate $\phi$ on $\minifed$ protocols is \emph{certifiable} iff
  there exists an algorithm $A$ such that $A(\prog)$ iff $\phi(\prog)$,
  and a safe protocol $\prog$ is \emph{certified for $\phi$} iff
  $A(\prog)$ holds.
\end{definition}

Given computation of $\prog(\prog)$ for programs $\prog$, we can
immediately observe that the security hyperproperties discussed in
Section \ref{section-hyperprop} are certifiable.
\begin{lemma}
  Both $\pni$ (Definition \ref{definition-PNI}) and $\NIMO$ (Definition \ref{definition-NIMO}) are certifiable. 
\end{lemma}

\begin{proof}
  Let $\progd(\prog)$ be computed as described above, and let $\iov(\prog) = S
  \cup V$. To verify $\pni(\prog)$ for
  some given $H$ and $C$, we enumerate $\runs(\prog)$ and check:
  $$
  \progd(\prog)(\store_{S_H}|\store_{S_C}) =
  \progd(\prog)(\store_{S_H}|\store_{S_C \cup V_C})
  $$
  for all $\store \in \mems(S \cup V)$, and return success $\pni(\prog)$ iff every such
  check succeeds. The result follows by Lemmas \ref{lemma-cprogtt} and \ref{lemma-pni}.
  To verify $\nimo(\prog)$, we first enumerate all partitions of clients
  into set $H$ and $C$ where $0 \in C$, then enumerate $\runs(\prog)$, and
  then check:
  $$
  \progd(\prog)(\store_{S_H}|\store_{S_C \cup \{\outv\}}) =
  \progd(\prog)(\store_{S_H}|\store_{S_C \cup V_C})
  $$
  The result follows by Lemma \ref{lemma-cprogtt} and Lemma \ref{lemma-nimo}.
\end{proof}

We can thus automatically verify security properties of examples considered
in previous sections, observing that passive security of $\prog_{\ref{example-OT}}$
has not been previously demonstrated.
\begin{lemma}
  $\prog_{\ref{example-he}}$, $\prog_{\ref{example-gmw-andcircuit}}$,
  $\prog_{\ref{example-ygc-andcircuit}}$ are all passive secure,
  $\prog_{\ref{example-OT}}$ is passive secure assuming client 3 is
  honest, and $\prog_{\ref{example-otp}}$ satisfies probabilistic
  noninterference and hence perfect secrecy (as defined in
  \cite{barthe2019probabilistic}).
\end{lemma}
\begin{proof}
  We have certified $\nimo(prog_{\ref{example-he}})$, $\nimo(\prog_{\ref{example-gmw-andcircuit}})$,
  $\nimo(\prog_{\ref{example-ygc-andcircuit}})$, and $\nimo(prog_{\ref{example-OT}})$ assuming
  $3 \in H$, and $\pni(\prog_{\ref{example-otp}})$. The result follows by Lemmas
  \ref{lemma-pni} and \ref{lemma-nimo}.
\end{proof}

\subsection{Certification of Composable Properties}
\label{section-automation-intensional}

While the brute force method is feasible for smaller programs,
including small circuits such as Examples
$\prog_{\ref{example-gmw-andcircuit}}$ and
$\prog_{\ref{example-ygc-andcircuit}}$, it is not for larger programs
such as YGC or GMW circuits since the size of $\runs(\prog)$ is
exponential in the size of the random tape plus the number of input
secrets, and circuits can be quite large in practice
\cite{kreuter2012billion}.  However, the YGC and GMW protocols are
examples of a common idiom in MPC protocol design- arbitrarily large
circuits are built up from smaller boolean or arithmetic gates that
operate on encodings of input secrets.

Our solution is to design gate certifications that demonstrably
guarantee security for circuits built from them. Intuitively, we can
consider any gate as a mini-program and certify properties
of its isolated program distribution. These component properties are
guaranteed secure in circuit compositions, as we demonstrate using
methods based on probabilistic separation logic
\ref{barthe2019probabilistic}. While this last step is manual, our
gate certifications are defined generally and can be used to certify
new gate implementations, and verification applies to a circuit of any
size. In Section \ref{section-composition} we will demonstrate our
approach using YGC and GMW as examples.

When we consider properties
of gates in isolation, we will explicitly consider flip conditions in
distributions. Thus we will be directly concerned with basic distributions,
which we denote $\progtt(\prog)$, and we will often focus on views
defined in subprotocols. 
Notions of probabilistic independence, aka \emph{separation}, as well
as correlation, are important to certify. We borrow the symbols $*$
and $\sim$ from \cite{barthe2019probabilistic} with the same
denotations.
\begin{definition}
  We write $\vc{\pmf}{x}{y}$ iff $\pmf(\{ x \mapsto 0\}\ |\ \{ y \mapsto 0 \}) =
  \pmf(\{ x \mapsto 1\}\ |\ \{ y \mapsto 1 \}) = 1$.
  We write $\sep{\pmf}{X}{Y}$ iff for all
    $\store \in \mems(X \cup Y)$ we have
  $\margd{\pmf}{X \cup Y}(\store) =
  \pmf(\store_X) * \pmf(\store_Y)$
\end{definition}
We immediately observe that separation and correlation are certifiable,
since they are based on marginalization and conditioning of program
distributions. 
\begin{lemma}
  \label{lemma-autosep}
  Both of the following are true:
  \begin{enumerate}
  \item $\sep{\progtt(\prog)}{X}{Y}$ is certifiable for any $X$ and $Y$.
  \item $\vc{\progtt(\prog)}{x}{y}$ is certifiable for any $x$ and $y$.
  \end{enumerate}
\end{lemma}

\subsection{Compositional Metatheory}
\label{section-automation-logic}

Since we certify properties of gates in isolation with witness inputs,
our metatheory is concerned with the preservation of important invariants
when certified components are ``spliced'' into larger program
contexts. This technique often requires us to focus on views defined
in subprograms, so we define the following.
\begin{definition}
For any program $\prog =
  (\eassign{v_1}{\be_1};\ldots;\eassign{v_n}{\be_n})$, define
$\vdefs(\prog) \defeq \{ v_1,\ldots,v_n \}$.
\end{definition}

An important reasoning principle is related to component
noninterference- specifically, if we show that gate views enjoy a
noninterference property wrt their free (input) variables,
probabilistic independence propagates to other variables in the
preceding program.
\begin{restatable}[Noninterference]{lemma}{noninterference}
  \label{lemma-noninterference}
  Given $\prog_1;\prog_2$ and $X = \iov(\prog_2) - \vdefs(\prog_2)$ and
  $Y \subseteq \vdefs(\prog_2)$. If $\sep{\progd(\prog_1;\prog_2)}{X}{Y}$
  then $\sep{\progd(\prog_1;\prog_2)}{\iov(\prog_1)}{Y}$.
\end{restatable}

Also, when we certify gates in isolation with witness inputs, we can
``splice'' these properties into larger program contexts using the
following substitution Lemmas, showing that the substitution of
wire inputs for witness inputs preserves both separation and correlation.
\begin{restatable}[Substitution$*$]{lemma}{substar}
  \label{lemma-substitution}
  If $\sep{\progtt(\prog_2)}{\{ f \}}{X}$ and
  $\prog_1;\prog_2[\be/f]$ is safe with $\vars(\prog_1,\be) \cap
  X = \varnothing$ then
  $\sep{\progtt(\prog_1;\prog_2[\be/f])}{\vars(\be)}{X}$.
\end{restatable}

\begin{lemma}[Substitution$\sim$]
  \label{lemma-substitution-sim}
  If $\vc{\progtt(\prog_2,\be)}{\itv}{f}$ and $\vc{\progtt(\prog_1,\be')}{\itv}{f'}$
  and $\vars(\be') \cap (\vars(\be) - \{ f \}) = \varnothing$
  then $\vc{\progtt(\prog_1;\prog_2,\be[\be'/f'])}{\itv}{f}$.
\end{lemma}

We note that while the Frame rule of \cite{barthe2019probabilistic}
supports local reasoning in larger program contexts, it does not
support this splicing method, and our Lemmas \ref{lemma-substitution}
and \ref{lemma-substitution-sim} are both novel. Supplementary proofs
are demonstrated in Appendix \ref{section-proofs}.
