\section{Automating Security Verification}

We will use \emph{certification} to refer to algorithmic methods of
verifying properties of $\minifed$ protocols, vs.~manual methods. For
the sake of clarity we make the following definition:
\begin{definition}
  A predicate $\phi$ on $\minifed$ protocols is \emph{certifiable} iff
  there exists an algorithm $A$ such that $A(\prog)$ iff $\phi(\prog)$,
  and a safe protocol $\prog$ is \emph{certified for $\phi$} iff
  $A(\prog)$ holds.
\end{definition}

We base certification of protocols $\prog$ on a calculation of its
program distribution. Recall from Definition \ref{definition-progd} that
program distributions are calculated from runs of a protocol given
any random tape and input secrets. That is:
\begin{definition}
  Given $\prog$ with $\iov(\prog) = S \cup V$ and $\flips(\prog) = F$, define:
  $$
  \runs(\prog) \defeq \{ \store \in \mems(S\cup F \cup V) \mid \config{\store_{S \cup F}}{\prog} \redxs \config{\store}{\varnothing} \}
  $$
\end{definition}

Given any $\prog$, the set $\runs(\prog)$ can be computed via brute
force methods. One is to fold a standard interpretation of logical
formulas across $\prog$, obtaining a truth table representation of
$\runs(\prog)$. We can also interpret $\prog$ as a stratified Datalog
program, with $\runs(\prog)$ being the set of least Herbrand models
obtained by representing different random tapes and input secrets as
different fact bases. This representation is amenable HPC
optimizations such as parallelization and GPU matrix computations as
shown in recent work
\cite{sakama2017linear,aspis2018linear,nguyen2022enhancing,nguyen2021efficient}.
We provide more details of brute force computation of $\runs(\prog)$ in
Appendix \ref{section-bruteforce} and an implementation of
the tabular method is available in our online repository\footnote{Link
redacted for double-blind reviewing.}.

Given a calculation of $\runs(\prog)$, we can easily calculate its
basic distribution as in Definition \ref{definition-progd)}, as well
as any marginalization (such as $\progd(\prog)$) or conditioning of it.

\subsection{Certification of Extensional Properties}

Given computation of $\prog(\prog)$ for programs $\prog$, we can
immediately observe that the extensional hyperproperties discussed in
Section \ref{section-hyperprop} are certifiable.
\begin{lemma}
  Both $\pni$ (Definition \ref{definition-PNI}) and $\NIMO$ (Definition \ref{definition-NIMO}) are certifiable. 
\end{lemma}

\begin{proof}
  Let $\progd(\prog)$ be computed as described above, and let $\iov(\prog) = S
  \cup V$. To verify $\pni(\prog)$ for
  some given $H$ and $C$, we enumerate $\runs(\prog)$ and check:
  $$
  \progd(\prog)(\store_{S_H}|\store_{S_C}) =
  \progd(\prog)(\store_{S_H}|\store_{S_C \cup V_C})
  $$
  for all $\store \in \mems(S \cup V)$, and return success $\pni(\prog)$ iff every such
  check succeeds. The result follows by Lemmas \ref{lemma-cprogd} and \ref{lemma-pni}.
  To verify $\nimo(\prog)$, we first enumerate all partitions of clients
  into set $H$ and $C$ where $0 \in C$, then enumerate $\runs(\prog)$, and
  then check:
  $$
  \progd(\prog)(\store_{S_H}|\store_{S_C \cup \{\outv\}}) =
  \progd(\prog)(\store_{S_H}|\store_{S_C \cup V_C})
  $$
  The result follows by Lemma \ref{lemma-cprogd} and Lemma \ref{lemma-nimo}.
\end{proof}

We can thus automatically verify security properties of examples considered
in previous sections, observing that passive security of $\prog_{\ref{example-OT}}$
has not been previously demonstrated.
\begin{lemma}
  $\prog_{\ref{example-he}}$, $\prog_{\ref{example-gmw-andcircuit}}$,
  $\prog_{\ref{example-ygc-andcircuit}}$ are all passive secure,
  $\prog_{\ref{example-OT}}$ is passive secure assuming client 3 is
  honest, and $\prog_{\ref{example-otp}}$ satisfies probabilistic
  noninterference and hence perfect secrecy as defined in
  \ref{barthe2019probabilistic}.
\end{lemma}
\begin{proof}
  We have certified $\nimo(prog_{\ref{example-he}})$, $\nimo(\prog_{\ref{example-gmw-andcircuit}})$,
  $\nimo(\prog_{\ref{example-ygc-andcircuit}})$, and $\nimo(prog_{\ref{example-OT}})$ assuming
  $3 \in H$, and $\pni(\prog_{\ref{example-otp}})$. The result follows by Lemmas
  \ref{lemma-pni} and \ref{lemma-nimo}.
\end{proof}

\subsection{Certification of Composable Intensional Properties}

While the brute force method is feasible for smaller programs,
including small circuits such as Examples
$\prog_{\ref{example-gmw-andcircuit}}$ and
$\prog_{\ref{example-ygc-andcircuit}}$, it is not for larger programs
such as YGC or GMW circuits since the size of $\runs(\prog)$ is
exponential in the size of the random tape plus the number of input
secrets, and circuits can be quite large in practice
\cite{kreuter2012billion}.  However, the YGC and GMW protocols are
examples of a common idiom in MPC protocol design- arbitrarily large
circuits are built up from smaller boolean or arithmetic gates that
operate on encodings of input secrets.

Our approach is thus to design gate certifications that demonstrably
guarantee security for circuits built from them. Intuitively, we can
consider any gate as a mini-program and certify intensional properties
of its isolated program distribution. These intensional properties are
guaranteed secure in circuit compositions, as we demonstrate using
methods based on probabilistic separation logic
\ref{barthe2019probabilistic}. While this last step is manual, our
gate certifications are defined generally and can be used to certify
new gate implementations, and verification applies to a circuit of any
size. In Section \ref{section-composition} we will demonstrate our
approach using YGC and GMW as examples.

When we consider intensional properties
of gates in isolation, we will explicitly consider flip conditions in
distributions. Thus we will be directly concerned with basic distributions,
which we denote $\progtt(\prog)$, and we will often focus on views
defined in subprotocols. 
\begin{definition}
  We write $\progtt(\prog)$ to denote the \emph{basic distribution} of
  $\prog$ as defined in Definition \ref{def-progd}. For any program $\prog =
  (\eassign{v_1}{\be_1};\ldots;\eassign{v_n}{\be_n})$, define
  $\vdefs(\prog) \defeq \{ v_1,\ldots,v_n \}$.
\end{definition}
Notions of probabilistic independence, aka \emph{separation}, as well
as correlation, are important to certify. We borrow the symbols $*$
and $\sim$ from \cite{barthe2019probabilistic} with the same
denotations.
\begin{definition}
  We write $\vc{\pmf}{x}{y}$ iff $\pmf(\{ x \mapsto 0\}\ |\ \{ y \mapsto 0 \}) =
  \pmf(\{ x \mapsto 1\}\ |\ \{ y \mapsto 1 \}) = 1$.
  We write $\sep{\pmf}{X}{Y}$ iff for all
    $\store \in \mems(X \cup Y)$ we have
  $\margd{\pmf}{X \cup Y}(\store) =
  \pmf(\store_X) * \pmf(\store_Y)$
\end{definition}
We immediately observe that separation and correlation are certifiable,
since they are based on marginalization and conditioning of program
distributions. 
\begin{lemma}
  Both of the following are true:
  \begin{enumerate}
  \item $\sep{\progtt(\prog)}{X}{Y}$ is certifiable for any $X$ and $Y$.
  \item $\vc{\progtt(\prog)}{x}{y}$ is certifiable for any $x$ and $y$.
  \end{enumerate}
\end{lemma}

\subsection{Compositional Metatheory}

Since we certify properties of gates in isolation with witness inputs,
our metatheory is concerned with preservation of important invariants
when certified components are ``spliced'' into larger program
contexts.  One important reasoning principle is related to
noninterference properties- specifically, if we show that gate views
enjoy a noninterference property wrt their free (input) variables,
probabilistic independence propagates to other variables in the
preceding program.
\begin{restatable}[Noninterference]{lemma}{noninterference}
  \label{lemma-noninterference}
  Given $\prog_1;\prog_2$ and $X = \iov(\prog_2) - \vdefs(\prog_2)$ and
  $Y \subseteq \vdefs(\prog_2)$. If $\sep{\progd(\prog_1;\prog_2)}{X}{Y}$
  then $\sep{\progd(\prog_1;\prog_2)}{\iov(\prog_1)}{Y}$.
\end{restatable}

Also, when we certify gates in isolation with witness inputs, we can
``splice'' these properties into larger program contexts using the
following substitution Lemmas, showing that the substitution of
wire inputs for witness inputs preserves both separation and correlation.
\begin{restatable}[Substitution$*$]{lemma}{substar}
  \label{lemma-substitution}
  If $\sep{\progtt(\prog_2)}{\{ f \}}{X}$ and
  $\prog_1;\prog_2[\be/f]$ is safe with $\vars(\prog_1,\be) \cap
  X = \varnothing$ then
  $\sep{\progtt(\prog_1;\prog_2[\be/f])}{\vars(\be)}{X}$.
\end{restatable}

\begin{lemma}[Substitution$\sim$]
  \label{lemma-substitution-sim}
  If $\vc{\progtt(\prog_2,\be)}{\itv}{f}$ and $\vc{\progtt(\prog_1,\be')}{\itv}{f'}$
  and $\vars(\be') \cap (\vars(\be) - \{ f \}) = \varnothing$
  then $\vc{\progtt(\prog_1;\prog_2,\be[\be'/f'])}{\itv}{f}$.
\end{lemma}

We note that while the Frame rule of \cite{barthe2019probabilistic}
supports local reasoning in larger program contexts, it does not
support this splicing method, and our Lemmas \ref{lemma-substitution}
and \ref{lemma-substitution-sim} are both novel. Supplementary proofs
are demonstrated in Appendix \ref{section-proofs}.
