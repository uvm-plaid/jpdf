\section{A Protocol Metalanguage}
\label{section-metalang}

\begin{fpfig}[t]{Syntax of $\metaprot$.}{fig-metaprot-syntax}
$$
\begin{array}{rcl@{\hspace{8mm}}r}
\flab &\in& \mathrm{Field}\\
x &\in& \mathrm{EVar}\\
f &\in& \mathrm{FName}\\[2mm]
e &::=& b \mid \flip{e}{e} \mid \secret{e}{e} \mid \view{e}{e} \mid \oracle{e} \mid \enot\ e \mid e\ \eand\ e \mid e\ \exor\ e \mid & \textit{expressions}\\[0mm]
& & \select{e}{e}{e} \mid 
\send{\view{e}{e}}{e} \mid \send{\view{e}{e}}{\OT{e}{e}{e}} \mid e;e \mid \\[0mm]
& & x \mid \elet{x}{e}{e} \mid f(e,\ldots,e) \mid \{ \flab = e; \ldots; \flab = e \}
\mid e.\flab \mid e\concat e \mid (e) \\[2mm]
v &::=& w \mid \cid \mid \be \mid \{ \flab = v;\ldots;\flab = v \} 
\mid (\,) & \textit{values}\\[2mm]
{fn} &::=& f(x,\ldots,x) \{ e \} & \textit{functions}
\end{array}
$$
\end{fpfig}

Large practical MPC computations are based on much larger protocols
than the examples we've considered so far. These larger protocols are
typically based on compositional units. An example of this is Yao's
Garbled Circuits (YGC), which are composed of so-called garbled gates.
Languages for defined garbled circuits, beginning with Fairplay \cite{XXX},
treat gates as compositional units that are wired together by the programmer
to generate a complete circuit. The $\fedprot$ language is low-level
and does not include abstractions for defining composable elements. 

In this Section we introduce the $\metaprot$ language which includes
structured data and function definitions, which are sufficiently
expressive to define composable protocol elements such as garbled
gates. The $\metaprot$ language is a \emph{metalanguage}, in the sense
that it produces $\fedprot$ protocols as a result of computation. That
is, $\metaprot$ is a high-level language that generates low-level
protocol code.

\subsection{Syntax}

The syntax of $\metaprot$ is defined in Figure
\ref{fig-metaprot-syntax}.  It includes a syntax of function
definitions and records, and values include client ids, identifier
strings, and boolean expressions. Expression forms allow dynamic
construction of boolean expression forms and view assignments. When
$\metaprot$ programs construct a $\fedprot$ assignment, a side effect
occurs whereby the assignment is added to the end of the $\fedprot$
program accumulated during evaluation.

Formally, we consider a complete metaprogram to include both a
codebase and a ``main'' program that uses the codebase. 
\begin{definition}
A \emph{codebase} $\codebase$ is a list of function 
declarations. We write $ \codebase(f) = x_1,\ldots,x_n,\ e$
iff $f(x_1,\ldots,x_n) \{ e \} \in \codebase$.
A \emph{metaprogram}, aka \emph{mataprotocol} is a pair of a 
codebase and expression $\codebase, e$. We may omit
$\codebase$ if it is clear from context.  
\end{definition}

When we consider the example of YGC in detail below, our focus will be
on developing a codebase that can be used to define arbitrary
circuits, i.e., complete and concrete protocols. Since strings and
identifiers can be constructed manually, and expressions can occur
inside assignments and boolean expression forms, function definitions
can generalize over $\fedprot$-level patterns to obtain composable
program units. As a simple example, consider 3 party secret
sharing as illustrated in Example \ref{example-he}. We can
define a function $\ttt{share3}$ that abstracts the process
of splitting a given client's secret into 3 separate shares.
\begin{example} \label{example-share3} The function $\ttt{share3}$ 
  splits a client's secret into 3 shares returned as a record
  with fields $\ttt{s1-3}$:
  \begin{verbatimtab}
    share3(client, secretid)
    {
      let s1 = flip[client, share1] in
      let s2 = flip[client, share2] in
      let s3 = (s1 xor s2) xor s[client, s: || secretid] in
      {s1 = s1;s2 = s2;s3 = s3}
    } \end{verbatimtab}
  Here is $\metaprot$ program that uses this this function definition:
  \begin{verbatimtab}
    let shares = share3(1, mysecret) in
    v[2,s1] := shares.s2;
    v[3,s1] := shares.s3 \end{verbatimtab}
  which generates the following $\minifed$ program, as we formalize in Example \ref{example-share3-eval}
  below:
  \begin{verbatimtab}
    v[2,s1] := flip[1, share2];
    v[3,s1] := flip[1, share1] xor flip[1, share2] xor s[client, s:mysecret] \end{verbatimtab}
\end{example}

\subsection{Semantics}

\begin{fpfig}[t]{Evaluation contexts and operational semantics of $\metaprot$.}{fig-metaprot-semantics}
$$
\begin{array}{rcl@{\hspace{3mm}}r}
E &::=& [\,] \mid \enot\ E \mid E\ \bop\ e \mid v\ \bop\ E \mid  \flip{E}{e} \mid \secret{E}{e} \mid \view{E}{e} \mid \oracle{E} \mid  \\[1mm]
& & \flip{\cid}{E} \mid \secret{\cid}{E} \mid \view{\cid}{E} \mid \send{E}{e} \mid \send{\view{\cid}{w}}{E} \mid \OT{E}{e}{e} \\[1mm]
& & \mid \OT{v}{E}{e} \mid \OT{v}{v}{E} \mid \select{E}{e}{e} \mid \select{v}{E}{e} \mid \\[1mm]
& & \select{v}{v}{E} \mid \elet{x}{E}{e} \mid f(v,\ldots,v,E,e,\ldots,e) \mid \\[1mm]
& & \{ \flab = v;\ldots;\flab = v;\flab = E;\flab = e;\ldots;\flab = e \} \mid E.\flab \mid E\concat e \mid v \concat E
\end{array}
$$
\medskip
$$
\begin{array}{rcl@{\hspace{10mm}}r}
\config{\prog}{\elet{x}{v}{e}} &\redx& \config{\prog}{e[v/x]}\\
\config{\prog}{f(v_1,...,v_n)} &\redx&
\config{\prog}{e[v_1/x_1,\ldots,v_n/x_n]} & 
 \codebase(f) = x_1,\ldots,x_n,\ e\\
\config{\prog}{\{\ldots; \flab = v; \ldots\}.\flab} &\redx&
 \config{\prog}{v}\\
 \config{\prog}{w_1\concat w_2} &\redx& \config{\prog}{w_1w_2}\\
 \config{\prog}{v;e} &\redx& \config{\prog}{e}\\
\config{\prog}{\instr} &\redx& \config{\prog;\instr}{()}\\
\config{\prog}{E[e]} &\redx& \config{\prog'}{E[e']} & \text{if}\ \config{\prog}{e} \redx \config{\prog'}{e'} 
\end{array}
$$
\end{fpfig}

We define a small-step evaluation aka reduction relation $\redx$ in
Figure \ref{fig-metaprot-semantics}.  We write $\redxs$ to denote the
reflexive, transitive closure of $\redx$. Reduction is defined on
\emph{configurations} which are pairs of the form $\config{\prog}{e}$,
where $\prog$ is the $\minifed$ program accumulated during evaluation.
In this definition we write $e[v/x]$ to denote the substitution of $v$
for free occurences of $x$ in $e$. The rules are mostly standard,
except why a concrete $\minifed$ assignment is encountered it is added
to the end of $\prog$.

The rules rely on a definition of \emph{evaluation contexts} $E$
allowing computation within a larger program context, where $E[e]$
denotes an expression with $e$ in the hole $[]$ of $E$. Evaluation
contexts include boolean expression forms, allowing generalization
and instantiation of compositional program elements.
\begin{example}
  \label{example-share3-eval}
  Let $\codebase,e_{\ref{example-share3}}$ be the $\metaprot$ program, and let 
  $\prog_{\ref{example-share3}}$ be the  $\minifed$ program defined
  in Example \ref{example-share3}. We refer to the latter as ``accumulated''
  by evaluation in the sense that $\config{\varnothing}{e_{\ref{example-share3}}}
  \redxs \config{\prog_{\ref{example-share3}}}{\varnothing}$.
\end{example}

\subsection{An Implementation of Yao's Garbled Circuits}
\label{section-metalang-ygc}

\begin{fpfig}[t]{Yao's Garbled Circuits, auxiliary functions.}{fig-ygc-aux}
{\footnotesize
\begin{verbatimtab}
  keygen(gid, b1, b2) { select4(b1,b2,H[gid || "1"],H[gid || "2"],H[gid || "3"],H[gid || "4"]) }
  
  keysgen(gid, b1, b2)
  {
    let k11 = keygen(gid,b1,b2) in
    let k10 = keygen(gid,b1,not b2) in
    let k01 = keygen(gid,not b1,b2) in
    let k00 = keygen(gid,not b1,not b2) in
    {k11 = k11; k10 = k10; k01 = k01; k00 = k00}
  }
  
  andtable(keys, bt, ap, bp)
  {
    let r11 = (keys.k11 xor bt) in 
    let r10 = (keys.k10 xor (not bt)) in
    let r01 = (keys.k01 xor (not bt)) in
    let r00 = (keys.k00 xor (not bt)) in
    permute4(ap,bp,r11,r10,r01,r00)
  }
  
  sharetable(gid, vid, table)
  {   
    v[1, "gate:" || gid || vid || "1"] := table.v1;
    v[1, "gate:" || gid || vid || "2"] := table.v2;
    v[1, "gate:" || gid || vid || "3"] := table.v3;
    v[1, "gate:" || gid || vid || "4"] := table.v4
  }
\end{verbatimtab}
}
\end{fpfig}

\begin{fpfig}[t]{Yao's Garbled Circuits, garbled gates and evaluation code.}{fig-ygc-gates}
{\footnotesize
\begin{verbatimtab}
  garbledecode(wl)
  {
    let r1 = wl.k xor true in
    let r0 = (not wl.k) xor false in
    v[1,"OUTtt1"] := select[wl.p,r1,r0];
    v[1,"OUTtt2"] := select[not wl.p,r1,r0]
  }
  
  evaldecode(wl, p) { wl.k xor select[wl.p,v[1,"OUTtt1"],v[1,"OUTtt2"]] }
  
  evalgate(gid, wla, wlb)
  {
    let k = keygen(gid,wla.k,wlb.k) in
    let ct = select4(wla.p,wlb.p,
               v[1,gid || "tt1"],v[1,gid || "tt2"],v[1,gid || "tt3"],v[1,gid || "tt4"]) in
    let cp = select4(wla.p,wlb.p,
               v[1,gid || "pt1"],v[1,gid || "pt2"],v[1,gid || "pt3"],v[1,gid || "pt4"]) in
    { k = k xor ct; p = k xor cp }
  }
  
  andgate(gid, wla, wlb, wlc) 
  {
    let keys = keysgen(gid,wla.k,wlb.k) in
    sharetable(gid,"tt",andtable(keys,wlc.k,wla.p,wlb.p));
    sharetable(gid,"pt",andtable(keys,wlc.p,wla.p,wlb.p))
  }
  
  encode(gid, wla,wlb)
  {
    let wla = { k = flip[2,"fwl1"]; p = flip[2,"pwl1"] } in
    let wlb = { k = flip[2,"fwl2"]; p = flip[2,"pwl2"] } in
    { wv1 = { k = OT[s[1,"0"],wla.k,(not wla.k)]; p = OT[s[1,"0"],wla.p,(not wla.p)]}; 
      wv2 = { k = select[s[2,"0"],wlb.k,(not wlb.k)]; p = select[s[2,"0"],wlb.p,(not wlb.p)] } }
  }
\end{verbatimtab}
}
\end{fpfig}

%  andtable(keys, bt, ap, bp)
%  {
%    let r11 = (keys.k11 xor bt) in 
%    let r10 = (keys.k10 xor (not bt)) in
%    let r01 = (keys.k01 xor (not bt)) in
%    let r00 = (keys.k00 xor (not bt)) in
%    permute4(ap,bp,r11,r10,r01,r00)
%  }


In Figures \ref{fig-ygc-aux} and \ref{fig-ygc-gates} we define a
codebase for garbled circuits. This definition follows the
\emph{point-and-permute} method described in \cite{XXX} and elsewhere,
to which the reader is referred for more in-depth discussion.
In this implementation client 2 is the \emph{garbler} and
client 1 is the \emph{evaluator}. The garbler builds the garbled
tables and shares them with the evaluator, who then evaluates
the gate in an oblivious fashion until the final public output is
generated through decryption.



