\section{The $\metaprot$ Metalanguage}
\label{section-metalang}

Large practical MPC computations are based on much larger protocols
than the examples we've considered so far. These larger protocols are
typically based on compositional units. Examples include GMW circuits
and Yao's Garbled Circuits (YGC), which are composed of so-called
garbled gates.  Languages for defined garbled circuits, beginning with
Fairplay \cite{269581}, treat gates as compositional units that are
wired together by the programmer to generate a complete circuit. The
$\fedprot$ language is low-level and does not include abstractions for
defining composable elements.

In this Section, we introduce the $\metaprot$ language which includes
structured data and function definitions, which are sufficiently
expressive to define composable protocol elements such as garbled
gates. The $\metaprot$ language is a \emph{metalanguage}, in the sense
that it produces $\fedprot$ protocols as a result of computation. That
is, $\metaprot$ is a high-level language that generates low-level
protocol code. We endow $\metaprot$ with a type system that enjoys a
type safety result guaranteeing safety of generated protocols. We
consider library implementations for GMW and YGC circuits as
examples. Additional examples are provided in Appendix
\ref{section-examples-lang}.

\subsection{Syntax}

The syntax of $\metaprot$ is defined in Figure
\ref{fig-metaprot}.  It includes a syntax of function
definitions and records, and values include client ids, identifier
strings, and boolean expressions. Expression forms allow dynamic
construction of boolean expression forms and view assignments. When
$\metaprot$ programs construct an $\fedprot$ assignment, a side effect
occurs whereby the assignment is added to the end of the $\fedprot$
program accumulated during evaluation.

Formally, we consider a complete metaprogram to include both a
codebase and a ``main'' program that uses the codebase. 
\begin{definition}
A \emph{codebase} $\codebase$ is a list of function 
declarations. We write $ \codebase(f) = x_1,\ldots,x_n,\ e$
iff $f(x_1,\ldots,x_n) \{ e \} \in \codebase$.
A \emph{metaprogram}, aka \emph{metaprotocol}  is a pair of a 
codebase and expression $\codebase, e$. We may omit
$\codebase$ if it is clear from context.  
\end{definition}

When we consider the examples of GMW and YGC in detail below, our
focus will be on developing a codebase that can be used to define
arbitrary circuits, i.e., complete and concrete protocols. Since
strings and identifiers can be constructed manually, and expressions
can occur inside assignments and boolean expression forms, function
definitions can generalize over $\fedprot$-level patterns to obtain
composable program units.

\subsection{Semantics}

\begin{fpfig}[t]{$\metaprot$ syntax (T), evaluation contexts (M), and operational semantics (B).}{fig-metaprot}
    {\small
    $$
    \begin{array}{rcl@{\hspace{8mm}}r}
      \multicolumn{3}{l}{\flab \in \mathrm{Field},\   x \in \mathrm{EVar}, \  f \in \mathrm{FName}}\\[2mm]
      %x &\in& \mathrm{EVar}\\
      %f &\in& \mathrm{FName}\\[2mm]
      e &::=& b \mid \flip{e}{e} \mid \secret{e}{e} \mid \view{e}{e} \mid \oracle{e} \mid \enot\ e \mid e\ \eand\ e \mid e\ \exor\ e \mid & \textit{expressions}\\[0mm]
      & & \select{e}{e}{e} \mid 
      \send{\view{e}{e}}{e} \mid \send{\view{e}{e}}{\OT{e}{e}{e}} \mid e;e \mid \\[0mm]
      & & x \mid \elet{x}{e}{e} \mid f(e,\ldots,e) \mid \{ \flab = e; \ldots; \flab = e \}
      \mid e.\flab \mid e\concat e \mid (e) \\[2mm]
      v &::=& w \mid \cid \mid \be \mid \{ \flab = v;\ldots;\flab = v \} 
      \mid \ttt{()} & \textit{values}\\[2mm]
           {fn} &::=& f(x,\ldots,x) \{ e \} & \textit{functions}
    \end{array}
    $$
    }

    \rule{130mm}{0.5pt}

    {\small
    $$
    \begin{array}{rcl@{\hspace{3mm}}r}
      E &::=& [\,] \mid \enot\ E \mid E\ \bop\ e \mid v\ \bop\ E \mid  \flip{E}{e} \mid \secret{E}{e} \mid \view{E}{e} \mid \oracle{E} \mid  \\[1mm]
      & & \flip{\cid}{E} \mid \secret{\cid}{E} \mid \view{\cid}{E} \mid \send{E}{e} \mid \send{\view{\cid}{w}}{E} \mid \OT{E}{e}{e} \\[1mm]
      & & \mid \OT{v}{E}{e} \mid \OT{v}{v}{E} \mid \select{E}{e}{e} \mid \select{v}{E}{e} \mid \\[1mm]
      & & \select{v}{v}{E} \mid \elet{x}{E}{e} \mid f(v,\ldots,v,E,e,\ldots,e) \mid \\[1mm]
      & & \{ \flab = v;\ldots;\flab = v;\flab = E;\flab = e;\ldots;\flab = e \} \mid E.\flab \mid E\concat e \mid v \concat E
    \end{array}
    $$
    \vspace{.5mm}
    
    \rule{130mm}{0.5pt}
    
    $$
    \begin{array}{rcl@{\hspace{10mm}}r}
      \config{\prog}{\elet{x}{v}{e}} &\redx& \config{\prog}{e[v/x]}\\
      \config{\prog}{f(v_1,...,v_n)} &\redx&
      \config{\prog}{e[v_1/x_1,\ldots,v_n/x_n]} & 
      \codebase(f) = x_1,\ldots,x_n,\ e\\
      \config{\prog}{\{\ldots; \flab = v; \ldots\}.\flab} &\redx&
      \config{\prog}{v}\\
      \config{\prog}{w_1\concat w_2} &\redx& \config{\prog}{w_1w_2}\\
      \config{\prog}{v;e} &\redx& \config{\prog}{e}\\
      \config{\prog}{\instr} &\redx& \config{\prog;\instr}{()}\\
      \config{\prog}{E[e]} &\redx& \config{\prog'}{E[e']} & \text{if}\ \config{\prog}{e} \redx \config{\prog'}{e'} 
    \end{array}
    $$
  }
\end{fpfig}

We define a small-step evaluation aka reduction relation $\redx$ in
Figure \ref{fig-metaprot}.  We write $\redxs$ to denote the
reflexive, transitive closure of $\redx$. Reduction is defined on
\emph{configurations} which are pairs of the form $\config{\prog}{e}$,
where $\prog$ is the $\minifed$ program accumulated during evaluation.
In this definition we write $e[v/x]$ to denote the substitution of $v$
for free occurences of $x$ in $e$. The rules are mostly standard,
except when a concrete $\minifed$ assignment is encountered it is added
to the end of $\prog$.

The rules rely on a definition of \emph{evaluation contexts} $E$
allowing computation within a larger program context, where $E[e]$
denotes an expression with $e$ in the hole $[]$ of $E$. Evaluation
contexts include boolean expression forms, allowing generalization
and instantiation of compositional program elements.

\subsection{Type Theory and Static Type Safety}

It is desirable to statically enforce safety of both $\metaprot$
programs and the safety of the $\fedprot$ programs they
generate. Although safety of the latter could be enforced
post-generation by a direct analysis, for large programs this can be
much more expensive and it is also better to not waste time on
resource-intensive compilation of programs with known errors
\cite{kreuter2012billion}. Some consequences of safety errors, for example, accidental
reuse of one-time pads can also undermine security.

The type syntax of $\metaprot$ is defined in Figure
\ref{fig-metaprot-types}. It includes a weak form of dependency:
string types $\stringty{e}$ and client types $\cidty{e}$
are parameterized by expressions $e$ that precisely reflect the
type of the value. Boolean expression forms have the type
$\bet{e}$ indexed by expressions $e$ indicating the client
id type of the expression. The dependency is weak in the
sense that expressions in types are a strict subset of
expression forms- for any $\stringty{e}$ the expression $e$
is either a variable, a string, or a concatenation form,
and for any $\cidty{e}$ or $\bet{e}$ the expression $e$
is either a variable or a client id $\cid$. We define
the equivalence $w_1 \concat w_2 \equiv w_1w_2$ in typings. 

Type judgements for expressions are of the form
$\tjudge{\viewst_1}{\gamma}{e}{\viewst_2}$ where the \emph{view
effect} $\viewst_1$ denotes the views that have been defined so far,
and $\viewst_2$ records new views defined as the effect of the
expression on the residual $\minifed$ program.  Type judgements are
syntax-directed- selected rules are shown in Figure
\ref{fig-metaprot-types}
%, with a more complete set provided in
%Appendix Figure \ref{fig-metaprot-tjudge}
and a full implementation is available in our online repository \cite{jpdf-github}. The
$\TirName{AssignT}$ rule captures the effect of new view
definitions. The $\TirName{And}$ rule illustrates how program safety
is enforced, by ensuring that subexpressions of boolean expressions
have the same owner.

The $\TirName{Appt}$ rule applies to function
definition and application respectively. Function typings rely on the
definition of function input type annotations $\tas$ and
type term substitutions $\sigma$. 
\begin{definition}
  A \emph{function input type annotation} $\tas$ is a mapping from
  function names $f$ to type products $\tau_1 * \cdots * \tau_n$.
  A \emph{type term substitution} $\sigma$ is a mapping from
  $\minifed$ variables $x$ to values, where $\sigma(\tau)$ denotes
  the replacement of occurences of $x$ in $\tau$ with $\sigma(x)$. 
\end{definition}
We assume that input type annotations $\tas$ are provided by the
programmer for all function definitions. This guarantees that
$\metaprot$ type checking is straightforward and efficient.
Function types are of the form
$
(\tau_1 * \cdots * \tau_n \rightarrow \tau,\viewst)
$,
where $\viewst$ denotes the effect of the function on the residual
program.  The function type can be understood as a dependent $\Pi$
type, with every term variable bound. When applied, these variables
are instantiated with a type term substitution $\sigma$. In our
implementation, we adapt \emph{synthesis} as defined in Dependent ML \cite{10.1145/292540.292560}
to obtain this $\sigma$- essentially this is a match on the syntactic
structure of types and expressions. 

\begin{fpfig}[t]{Type syntax of $\metaprot$ (T) and selected type judgement rules (B).}{fig-metaprot-types}
{\small
  $$
  \begin{array}{rcl@{\hspace{2mm}}r}
    \srct &::=& \cidty{e} \mid \stringty{e} \mid \bet{e} \mid  \{ \flab : \srct;\ldots;\flab : \srct \} \mid \tau * \cdots * \tau \rightarrow \tau,\viewst & \gdesc{types}\\ [1mm]
    %&&  \{ \flab : \srct;\ldots;\flab : \srct \} \mid \tau * \cdots * \tau \rightarrow \tau,\viewst \\[1mm]
    \viewst  &::=& \view{e}{e};\viewst \mid \varnothing   & \gdesc{view effects}\\[1mm]
    \Gamma &::=& \Gamma; x : \tau \mid \varnothing & \gdesc{type environments}    
  \end{array}
  $$
}

\rule{130mm}{0.5pt}

{\small
\begin{mathpar}
\inferrule[\TirName{SecretT}]
{\tjudge{\viewst}{\Gamma}{e_1}{\cidty{e_1'}}{\viewst_1}\\
\tjudge{\viewst_1}{\Gamma}{e_2}{\stringty{e_2'}}{\viewst_2}}
{\tjudge{\viewst}{\Gamma}{\secret{e_1}{e_2}}{\bet{e_1'}}{\views_2}}

\inferrule[\TirName{AndT}]
{
\tjudge{\viewst}{\Gamma}{e_1}{\bet{e}}{\viewst_1}\\
\tjudge{\viewst_1}{\Gamma}{e_2}{\bet{e}}{\viewst_2}
}
{\tjudge{\viewst}{\Gamma}{e_1\ \eand\ e_2}{\bet{e}}{\viewst_2}}

\inferrule[\TirName{AssignT}]
{
\tjudge{\viewst}{\Gamma}{e_1}{\cidty{e_1'}}{\viewst_1}\\
\tjudge{\viewst_1}{\Gamma}{e_2}{\stringty{e_2'}}{\viewst_2}\\
\tjudge{\viewst_2}{\Gamma}{e_3}{\bet{e_3'}}{\viewst_3}
}
{
\tjudge{\viewst}{\Gamma}{\eassign{\view{e_1}{e_2}}{e_3}}{\unity}{(\viewst_3 ; \view{e_1'}{e_2'} )}
}

\inferrule[AppT]
{\Gamma(f) =  
 \tau_1 * \cdots * \tau_n \rightarrow \tau, \viewst_f \\ 
 \tjudge{\viewst}{\Gamma}{e_1}{\sigma(\tau_1)}{\viewst_1}
 \ \cdots\  
 \tjudge{\viewst_{n-1}}{\Gamma}{e_n}{\sigma(\tau_n)}{\viewst_n}}
{\tjudge{\viewst}{\Gamma}{f(e_1,\ldots,e_n)}{\sigma(\tau)}{(\viewst_n ; \sigma(\viewst_f))}}
\end{mathpar}
}
\end{fpfig}

Top-level type judgements are of the form $\Gamma \vdash \codebase, e
: \tau, V$, where all the functions in $\codebase$ are well-typed in
$\Gamma$, the top level view effect $V$ is a set of concrete
$\fedprot$ views which are constructed by a disjoint union of the views
in the effect of $e$- the disjointness requirement guarantees that
views are uniquely defined. Our $\metaprot$ type safety result is
formulated as follows. In addition to safe execution of the
metaprogram, it also guarantees safety of the residual $\fedprot$
program (Definition \ref{definition-safety}).
\begin{theorem}[$\metaprot$ Type Safety]
  \label{theorem-metalang-safety}
  Given $\codebase$, $e$, and $\Gamma$ with $\Gamma \vdash \codebase,e : \unity : V$,
  then $\config{\varnothing}{e} \redxs \config{\prog}{\ttt{()}}$ where
  $\prog$ is safe with $\iov(\prog) = S \cup V$ for some $S$.
\end{theorem}

\subsection{Example: GMW}
\label{section-metalang-gmw}

\begin{fpfig}[t]{2-Party GMW circuit library with And gate (T), and type annotations (B).}{fig-gmw}
{\footnotesize
  \begin{verbatimtab}
    encodegmw(in1,in2) {
      v[1, in1++"out"] := s[1,in1] xor flip[1,in1];
      v[2, in1++"out"] := flip[1,in1];
      v[1, in2++"out"] := s[2,in2] xor flip[2,in2];
      v[2, in2++"out"] := flip[2,in2]
      { shares1 = { c1 = v[1, in1++"out"]; c2 = v[2, in1++"out"] };
        shares2 = { c1 = v[1, in2++"out"]; c2 = v[2, in2++"out"]} } 
    }
    
    andtablegmw(b1, b2) {
      let r11 = (b1 xor true) and (b2 xor true) in
      let r10 = (b1 xor true) and (b2 xor false) in
      let r01 = (b1 xor false) and (b2 xor true) in
      let r00 = (bl xor false) and (b2 xor false) in
      { v1 = r11; v2 = r10; v3 = r01; v4 = r00 }
    }
    
    andgmw(g, shares1, shares2) {
      let r = flip[1,g++".r"] in
      let table = andtablegmw(shares1.c1, shares2.c1) in
      let r11 =  r xor table.v1 in
      let r10 =  r xor table.v2 in
      let r01 =  r xor table.v3 in
      let r00 =  r xor table.v4 in
        v[2,g++"out"] := OT4(shares1.c2, shares2.c2, r11, r10, r01, r00);
      v[1,g++"out"] := r;
      { c1 = v[1,g++"out"]; c2 = v[2,g++"out"]}
    }
    
    decodegmw(shares) { v[0,"output"] := shares.c1 xor shares.c2 }   \end{verbatimtab}
}
\end{fpfig}


\begin{fpfig}[t]{GMW library type annotations.}{fig-gmw-types}
{\footnotesize
  \begin{verbatimtab}
   encodegmw   : string(gid) * string(gid)
    
   andtablegmw : { k = bool[i]; p = bool[i] }
    
   andgmw      : string(gid) *  { c1 = bool[1]; c2 = bool[2] } * { c1 = bool[1]; c2 = bool[2] }
    
   decodegmw   : { c1 = bool[1]; c2 = bool[2] }  \end{verbatimtab}
}
\end{fpfig}


The GMW protocol uses secret sharing to represent data flowing through
circuits. In the 2-party case, clients 1 and 2 each share their input
secrets, and use those shares to represent inputs to gates. Outputs
are also represented as shares. We refer to the pair of shares
representing any particular value as a \emph{wire value}, and
we represent them via records of the form
$
\ttt{\{c1 = } v_1\ttt{;c2 = } v_2\ttt{\}} 
$
where $v_1$ and $v_2$ are client 1 and 2's shares respectively.

For full details of the GMW protocol the reader is referred to
\cite{evans2018pragmatic}. Our implementation libary is shown in
Figure \ref{fig-gmw}, with type signatures for the library functions
shown in Figure \ref{fig-gmw-types}. We show the And gate since it is
an interesting component. The Figure includes the
following top-level functions:
\begin{itemize}
\item \ttt{encodegmw}: This function encodes client 1's and client 2's
  secret bits called $\ttt{s[1,"s1"]}$ and $\ttt{s[1,"s2"]}$ into two
  distinct wire values (pairs of shares).
\item \ttt{andgmw}: This function defines the gate $\ttt{"g"}$, the
  identifier $\ttt{"g"}$ being used to distinguish randomness used
  within.  In our version client 1 builds the output table (using
  \ttt{andtablegmw})and transfers the correct output share to client 2
  using 1-out-of-4 OT as per standard GMW protocol.
\item \ttt{decode}: This function decodes and publishes a wire value
  by $\ttt{xor}$ing the shares. Note that this requires both client 1
  and 2 to publicize their shares.
\end{itemize}
\begin{example}
  \label{example-gmw-andcircuit}
The following program uses our GMW library to define
a circuit with a single And gate and input secrets $\ttt{s1}$ and
$\ttt{s2}$ from client's 1 and 2 respectively. 
{\small
  \begin{verbatimtab}
  let ss = encodegmw("s1","s2") in v[0,"output"] := decode(andgmw(0,ss.shares1,ss.shares2)) \end{verbatimtab}
}
\end{example}

\subsection{Example: YGC}
\label{section-metalang-ygc}

\begin{fpfig}[t]{Yao's Garbled Circuit component functions.}{fig-ygc-lib}
{\footnotesize
\begin{verbatimtab}
  owl(gid) { { k = flip[2,"gate:" || gid || ".k"]; p = flip[2,"gate:" || gid || ".p"] } }

  encodeygc(sa, sb) {
    let owl1 = owl(sa) in
    let owl2 = owl(sb) in
    v[1,"gate:" || sa || "1.k"] := OT(s[1,sa],owl1.k,(not owl1.k));
    v[1,"gate:" || sa || "1.p"] := OT(s[1,sa],owl1.p,(not owl1.p));
    v[1,"gate:" || sb || "2.k"] := select(s[2,sb],owl2.k,(not owl2.k));
    v[1,"gate:" || sb || "2.p"] := select(s[2,sb],owl2.p,(not owl2.p));
    { wv1 = { k = v[1,"gate:" || sa || "1.k"]; p = v[1,"gate:" || sa || "1.p"] };
      wv2 = { k = v[1,"gate:" || sb || "2.k"]; p = v[1,"gate:" || sb || "2.p"] } }
  }

  andygc(g, ga, gb, wva, wvb) { andgate(g, ga, gb); evalgate(g, wva, wvb) }

  decodeygc(g, wv) { garbledecode(g); evaldecode(wv) }
\end{verbatimtab}
}
\end{fpfig}

Our YGC implementation follows the \emph{point-and-permute} method
described in \cite{evans2018pragmatic} and elsewhere, to which the
reader is referred for more in-depth discussion.  In this
implementation client 2 is the \emph{garbler} and client 1 is the
\emph{evaluator}. The garbler builds the garbled tables and shares
them with the evaluator, who then evaluates the gate in an oblivious
fashion until the final public output is generated through
decryption. 

\emph{Wire labels} are fundamental to YGC, and essentially represent
gate output values in an encrypted form. In our definition, wire
labels are represented by records $\ttt{\{ k = }\beta_1\ttt{; p =
}\beta_2\ttt{ \}}$, where $\ttt{k}$ is the \emph{key bit} and
$\ttt{p}$ is the \emph{pointer bit}, and $\beta_1$ and $\beta_2$ are
flips. Flips in each output wire label are owned by the garbler and
are unique per gate by definition of their identifying string, and the
representation of $0$ is the negation of $1$. For example, here is the
representation of 1 and 0 respectively in the output wire label for a
hypothetical gate 6:
{\small
  \begin{mathpar}
    \ttt{\{ k = flip[2,"gate:6.k"]; p =  flip[2,"gate:6.p"]] \}}

    \ttt{\{ k = not flip[2,"gate:6.k"]; p =  not flip[2,"gate:6.p"]] \}}
  \end{mathpar}
}
In our implementation, gates are wired together using gate
identifiers, which are strings $w$. Top-level functionality
shown in Figure \ref{fig-ygc-lib} includes:
\begin{itemize}
\item \ttt{owl}: This function returns the output wire label for any
  gate identified by \ttt{gid}.
\item \ttt{andgate}: This defines a subprotocol for the garbler
  to define a garbled gate $\ttt{gid}$ with input wires from gates
  $\ttt{ga}$ and $\ttt{gb}$. The garbler generates keys and garbles
  the rows in YGC fashion, them with client $1$ in
  views in a standard form. For example, the view for
  a hypothetical gate 6, row 2 garbled truth table is $\ttt{v[1,"gate:6tt2"]}$.
  We note that garbled gates of other binary operators can be obtained with
  replacement of $\ttt{andtable}$ with appropriate garbled table definitions. 
\item \ttt{evalgate}: This defines a subprotocol for the evaluator to
  evaluate gate $\ttt{gid}$ given input wire values $\ttt{wva}$ and
  $\ttt{wvb}$.
\item \ttt{andygc}: This composes the garbler and evaluator subprotocols
  for construction and evaluation of an And gate. 
\item \ttt{garbledecode} and \ttt{evaldecode}: The former function
  defines the garbler's protocol for encrypting the circuit
  output from final gate $\ttt{gid}$, and the latter defines
  the evaluator's output decryption protocol. These are composed with \ttt{decodeygc}.
\item \ttt{encodegmw}: This defines the initial phase of the protocol,
  where the evaluator receives the wire value from their own
  secret $\sx{1}{sa}$ via $\ttt{OT}$, and the garbler communicates
  the wire value for their own secret $\sx{2}{sb}$ directly.
\end{itemize}
We provide more code details in Appendix \ref{section-ygc}, Figures
\ref{fig-ygc-aux} and \ref{fig-ygc-gates}, and complete code in our
online repository \cite{jpdf-github}.
\begin{example}
  \label{example-ygc-andcircuit}
The following program uses our YGC library to define
a circuit with a single and gate and input secrets $\ttt{s1}$ and
$\ttt{s2}$ from client's 1 and 2 respectively. 
{\small
\begin{verbatimtab}
  let secrets = encode("s1","s2") in
  v[0,output] := decode(andgg("0", secrets.wv1, secrets.wv2)) \end{verbatimtab}
}
%\begin{verbatimtab}
%  andgate(0,s1,s2);
%  garbledecode(0);
%  let secrets = encode(s1,s2) in
%  v[0,output] := decode(evalgate(0, secrets.wv1, secrets.wv2))
%\end{verbatimtab}
\end{example}
