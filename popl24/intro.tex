\section{Introduction}

Secure Multi-Party Computation (MPC) protocols support data privacy in
important modern, distributed applications such as privacy-preserving
machine learning \cite{li2021privacy, knott2021crypten,
  koch2020privacy, liu2020privacy} and Zero-Knowledge proofs in
blockchains \cite{ishai2009zero, lu2019honeybadgermpc,
  gao2022symmeproof, tomaz2020preserving}. MPC methods have been
developed by the cryptography community for many years, while
receiving the attention of the programming languages community more
recently. In contrast, information flow security has received
significant attention in PL theory and practice especially since the
turn of the century \cite{1159651}, including a menagerie of variants,
enforcement mechanisms, and programming frameworks. Much of this has
been enabled by the unified metatheory of \emph{hyperproperties}
\cite{10.5555/1891823.1891830}, including noninterference, that
establishes a common conceptual framework for reasoning about and
implementing systems with information flow security.

On its surface, security for MPC protocols seems similar to
probabilistic noninterference. However, MPC security differs from
noninterference in a subtle but fundamental way, and approaches for
verifying noninterference cannot naturally extend to MPC security. Our
goal is to explore and establish connections between the security
model of MPC-- the \emph{real/ideal} aka \emph{simulator security}
model -- and trace-based hyperproperties, and to leverage these
connections to obtain automated enforcement mechanisms for MPC
protocol development. Currently, proof methods for MPC protocol
development are well-studied \cite{Lindell2017} but manual and thus
tedious and error-prone, and are also non-standardized and unfamiliar
to most PL theorists. Therefore our exploration will make both
theoretical and practical contributions by bridging a gap between
information flow and simulator security methodologies.

\subsection{The Analysis Challenge of MPC}
MPC protocols involve communication
between a group of distributed participants called a \emph{federation}
that collaboratively compute and publish the result of some known
\emph{ideal functionality} $\idealf$, while keeping each party's input
``secret'', without the use of a trusted third party. This last part
is critical. For example, if we take $\idealf$ to be the majority vote
function, a protocol for computing $\idealf$ is MPC-secure if, given
any set of input votes, it correctly computes and publishes the voting
result but reveals no other information to the public or to other
participants. However, by publishing the result, some information
about individual votes may be implicitly declassified.  For example,
in the case of a majority vote in a federation of size 3, if the
motion carries and party 1 has voted no, then party 1 knows exactly
the votes of parties 2 and 3. This cannot be avoided due to the nature
of $\idealf$.

Security in the MPC setting thus means that protocols cannot reveal
any secret information other than what is implicitly declassified by
the publicized output of the ideal functionality. The security model
also assumes that some subset of participants can be corrupted and
collude adversarially to possibly infer more secret information. The
accepted method of demonstrating protocol security in this setting is
to define a \emph{simulation} algorithm that runs in the ``ideal''
world which, given just the inputs of corrupted parties and the output
of the ideal functionality, is able to reconstruct information that
corrupted parties receive in their so-called \emph{views} of the
protocol running in the real world.  This implies that adversarial
views provide no information beyond what is provided by the ideal
output alone. Simulation is defined probabilistically since MPC
protocols typically rely on cryptographic and probabilistic methods.

In Section \ref{section-hyperprop-passive} we formalize real/ideal
security, and we define and discuss simple examples of MPC protocols
in Section \ref{section-minicat-examples} and more complex ones in
Section \ref{section-metalang}. But an immediate and main point in
relation to security hyperproperties is that, due to the potential for
information release in MPC, simulator security is \emph{not} a strict
probabilistic noninterference or trace obliviousness property, as we
show in Section \ref{section-hyperprop-ni}- rather, the public output
allows and sets an upper bound on declassification.

\subsection{Related Work}

We distinguish between \emph{extensional} vs.~\emph{intensional}
properties and analysis of MPC protocols. By extensional, we mean
simulator security itself as well as analysis concerned with the use
and interaction of complete secure protocols. By intensional, we mean
analysis and inner workings of protocols themselves. While the
probabilistic formulations we develop could conceivably be used
extensionally, our main focus is on automated and semi-automated
reasoning about intensional properties of protocols, both through
whole-program analysis and compositional properties of program
components that support security. 

Previous work on analysis for the SecreC language
\cite{almeida2018enforcing,10.1145/2637113.2637119} is concerned with
extensional properties of MPC, in particular the specification and
enforcement of declassification bounds in programs that use MPC in
subprograms. This work is explicitly reminiscent of information flow
approaches such as delimited information release
\cite{10.1007/978-3-540-37621-7_9}, downgrading policies \cite{li2005downgrading}, and
relaxed noninterference \cite{10.1145/1040305.1040319}. However, their
program logic assumes correctness of the underlying MPC protocols.
The Wys$^\star$ language \cite{wysstar}, based on Wysteria
\cite{rastogi2014wysteria}, has similar goals and includes a
trace-based semantics for reasoning about the extensional interactions
of MPC protocols. Their compiler also guarantees that underlying
multi-threaded protocols enforce the single-threaded source language
semantics which is an intensional property. \mnote{We need a clearer statement here of the flaws with Wys-star specifically} All of these systems
leverage type systems to obtain automated security analyses.

In the realm of intensional analysis, previous work has shown how to
reduce proofs of simulator security to program equivalence assertions,
which supports the application of known techniques in automated verification
\cite{8429300}. However, they focus on noninterference properties,
which are arguably sound but notably incomplete wrt MPC functionality
as we discuss above and in Section
\ref{section-hyperprop-ni}. Probabilistic Separation Logic
\cite{barthe2019probabilistic} develops a logical framework for
reasoning about probabilistic independence in programs.  They consider
several MPC protocols and show how to systematically deduce, e.g.,
probabilistic independence of input secrets and adversarial views that
are critical intensional properties of simulator security.  However,
their focus on independence does not address the subtleties of
dependence that arise during MPC protocol execution, and the authors
do not formulate or prove simulator security properties in
\cite{barthe2019probabilistic}. The $\lambda_{\mathrm{obliv}}$
language \cite{darais2019language} is perhaps most closely related to
our work. Here, a type system is used to automatically enforce
so-called probabilistic trace obliviousness and they show how to use
their system to implement several variants of oblivious RAM. However,
obliviousness is related to noninterference and does not address
allowable information leakage as in MPC, and their use of type
linearity disallows the reuse of sampled probabilistic values which is
common in the MPC idiom, as our examples will illustrate. Similarly,
previous work on oblivious data structures \cite{10.1145/3498713},
while intended for sound integration into MPC protocols, focus on
expression and enforcement of obliviousness and do not account
for allowable information leakage.

Our work also shares many ideas with probabilistic programming
languages designed to perform (exact or approximate)
inference~\cite{holtzen2020scaling, carpenter2017stan, wood2014new,
  bingham2019pyro, albarghouthi2017fairsquare, de2007problog,
  pfeffer2009figaro, saad2021sppl}. Our setting, however, requires
verifying properties beyond inference, including conditional
independence. Recent work by Li et al.~\cite{li2023lilac} proposes a
manual approach for proving such properties, but does not provide
automation.


\subsection{Overview and Contributions}

\paragraph{Language design.} In Section \ref{section-minicat} we
develop a new probabilistic programming language $\minifed$ for
defining synchronous distributed protocols over the binary field. The
syntax and semantics provides a succinct account of \emph{views} and
(synchronous) messaging between protocol \emph{clients}. In Section
\ref{section-metalang} we define a new metaprogramming language
$\metaprot$ that dynamically generates $\minifed$ protocols. It
includes control and data structures and is able to express logical
protocol components, and enjoys a type safety result that guarantees
that generated protocols are semantically well-defined.

\paragraph{Hyperproperty formulation.} In Section \ref{section-pmf} we
formulate \emph{program distributions}, a formalism for expressing
dependencies between secrets and views, useful for expressing
intensional characteristics of protocols and precisely quantifying
allowable information leakage in terms of protocol outputs. On this
basis we then formulate a novel hyperproperty of program execution
traces, called \emph{noninterference modulo output ($\NIMO$)} that we
demonstrate implies passive simulator security.

\paragraph{Fully and partially automated verification.} In Section
\ref{section-automation} we develop an automated technique, aka
certification, that verifies extensional properties of protocols such
as $\NIMO$, and also intensional properties of protocol
components. The latter include properties of protocol components that
provably guarantee security through composition in large
circuits. These compositional proofs are based on separation logic but
include novel results to support our semi-automated method for large
circuits where we (1) define certification for components, (2) certify
components, (3) prove $\NIMO$ based on the certifiedness of
components. We develop novel methods partially based on separation
logic \cite{barthe2019probabilistic} to enable (3).  In Section
\ref{section-composition} we illustrate our method to obtain
semi-automated verification methods for the GMW and Yao's Garbled
Circuit protocols, where certification can be re-used to validate new
gate implementations.
