\section{Rewriting $\minicat$ to Stratified Datalog}
\label{section-bruteforce}

Here we show how to rewrite any $\prog$ to an equivalent Datalog
program, which supports application of recent work in linear algebraic
interpretation of Datalog and optimizations of model computation on
high performance computers \cite{sakama2017linear}. The method here also enumerates
$\runs(\prog)$ memory-by-memory, rather than in a ``batched'' manner
as in our first method, allowing parallelization of model
computation. The rewriting we describe here is to Datalog with
negation, with a negation-as-failure model, though we can also
use techniques in \cite{sakama2017linear} to eliminate negation from
resulting programs. \emph{Atoms} are $\minifed$ variables, \emph{literals}
are atoms or negated atoms, and clause bodies are conjunctions of literals.
A \emph{Datalog program} is a list of clauses.
$$
\begin{array}{rclr}
  \alpha &::=& \view{\cid}{w} \mid  \secret{\cid}{w} \mid \flip{\cid}{w} \mid \oracle{w} & \qquad \textit{(atoms)}\\
  \mathit{body} &::=&  \alpha \mid \neg \alpha \mid \alpha \wedge \mathit{body} \mid \neg\alpha \wedge \mathit{body} \mid \varnothing \\
  \mathit{clause} &::=& \alpha \gets \mathit{body}
\end{array}
$$

The first step in converting a protocol $\prot$ to a Datalog
program is to apply $\solve$ ``locally'' to each view definition
in $\prog$, obtaining constraints on memories that satisfy each
view in isolation.  
\begin{lemma} Let $\vars(\be)$ be the variables in $\be$, and define:
$$
{vtt}(\eassign{\view{\cid}{w}}{\be}) \defeq (\view{\cid}{w}, (\solve{(\mems(\vars\ \be))}{\be}))
$$
Then ${vtt}(\eassign{\view{\cid}{w}}{\be}) = (\view{\cid}{w},\stores)$ where $\stores
  = \{ \store \in \mems(\vars\ \be) \mid \lcod{\store,\be}{\cid} = 1 \}$ for some $\cid$.
\end{lemma}
Given this definition, the mapping of ${vtt}$ across a program
$\prog$- i.e., $(\mathit{map}\ {vtt}\ \prog)$- essentially defines the
logic program for ``view atoms'' modulo syntactic conversion. We can
accomplish the latter as follows, where $\datalog(\prog)$ defines the
full conversion.
\begin{definition} We define the conversion from memories to
  literals and clause bodies as follows:
\begin{mathpar}
  \logit{\alpha \mapsto 1} = \alpha

  \logit{\alpha \mapsto 0} = \neg \alpha

  \logit{\{ \alpha_1 \mapsto \beta_1, \ldots, \alpha_n \mapsto \beta_n\}} =
  \logit{\alpha_1 \mapsto \beta_1} \wedge \cdots \wedge \logit{\alpha_n \mapsto \beta_n}
\end{mathpar}
Given pairs $(\alpha,\stores)$ in the range of ${vtt}$, we define the conversion
to clauses as follows:
\begin{mathpar}
  \mathit{clauses}(\alpha,\{ \store_1,...,\store_n \}) = \alpha \gets \logit{\store_1} \vee \cdots \vee \alpha \gets \logit{\store_n}
\end{mathpar}
The $\minifed$-to-Datalog conversion is then defined as:
$$
\datalog(\prog) \defeq  \mathit{map}\ \mathit{clauses}\ (\mathit{map}\ {vtt}\ \prog)
$$
\end{definition}

In addition to converting view definitions to logic clauses, we also need to convert
secrets and random tapes. Since we assume given values for these in an arbitrary run of
the program, we can capture these as a ``fact base'' in our program, where
a fact is a clause of the form $\alpha \gets \varnothing$ and means that $\alpha$
is true in any model of the program. 
\begin{definition}
  Given $\store$, let $\{\alpha_1,\ldots,\alpha_n \} =
  \{ \alpha \in \dom(\store) \mid \store(\alpha) = 1 \}$.
  Then define:
  $$
  \mathit{facts}(\store) = \alpha_1 \gets \varnothing, \ldots, \alpha_n \gets \varnothing
  $$
\end{definition}

For any safe $\prog$ with $\iov(\prog) = S \cup V$ and $\flips(\prog) = F$, and
$\store \in \mems(S \cup F)$, it is the case
that $(\mathit{facts}(\store),\datalog(\prog))$ is a \emph{normal}, \emph{stratified}
(in fact, non-recursive) Datalog program, and so has a unique Least Herbrand Model
and is thus amenable to HPC optimization techniques \cite{aspis2018linear}. 
To compute $\runs(\prog)$, and thus $\progd(\prog)$, we compute
the Least Herbrand Model $\store$ of $(\mathit{facts}(\store'),\datalog(\prog))$
for all $\store' \in \mems(S \cup F)$, observing that model computation for
each $\store'$ can be done in parallel. The following establishes
correctness of this approach.
\begin{lemma}
  For all $\prog$ with $\iov(\prog) = S \cup V$ and $\flips(\prog) = F$,
  $\datalog(\prog)$ is a \emph{normal}, \emph{stratified}
  program \cite{aspis2018linear}, and $\store$ is the unique Least Herbrand
  Model of $(\mathit{facts}(\store_{S \cup F}),\datalog(\prog))$
  iff $\store \in \runs(\prog)$.
\end{lemma}
A full empirical exploration of the application of HPC optimizations
is beyond the scope of this paper but is a compelling topic for future
work. The reader is referred to \cite{nguyen2022enhancing} for
empirical results of HPC optimizations in other logic programming
contexts.
