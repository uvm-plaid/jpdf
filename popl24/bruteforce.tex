\section{Brute Force Verification}
\label{section-bruteforce}

We have shown how our formulation of program distributions, a precise
definition of intensional protocol behavior, can be used to enforce
the extensional property of MPC security. This intensional description
also has the benefit of being flexible- we can use the same mechanisms
to desribe $\NIMO$, noninterference, perfect secrecy, violations of
obliviousness, etc. These mechanisms can also be exploited to support
automated program analysis.

We initially explore a brute force approach to automatically verifying
program behaviour, the first step of which is to compute the basic
distribution of a program $\prog$. The program distribution
$\progd(\prog)$ can obviously be automatically derived from that, as
can be any of its marginal or conditional distributions. In this
Section we consider two automated mechanisms for deriving
$\progd(\prog)$ in this way. The first is through a straightforward
computation of all final program memories.  The second approach
rewrites $\minifed$ programs to stratified Datalog programs, that are
amenable HPC optimizations such as parallelization and GPU matrix
computations as shown in recent work \cite{XXX}.

The basic distribution of a program $\prog$ is based on the
final memories resulting from any run of $\prog$, denoted
$\runs(\prog)$.
\begin{definition}
  Given $\prog$ with $\iov(\prog) = S \cup V$ and $\flips(\prog) = F$:
  $$
  \runs(\prog) \defeq \{ \store \in \mems(S\cup F \cup V) \mid \config{\store_{S \cup F}}{\prog} \redxs \config{\store}{\varnothing} \}
  $$
\end{definition}
As we've observed, $|\runs(\prog)| = 2^{|S \cup F|}$, so the size of
$\runs(\prog)$ is exponential in the amount of randomness used in the
protocol which generally can be used as a proxy for the protocol
size. In Section \ref{section-composition} we will show how the
methods described here can be used to individually verify
compositional properties of small program components, that guarantee
security properties in larger protocols that combine them. Thus, while
scalability to program size is a clear challenge for our brute force
techniques alone, our ultimate goal is to use them for verification of
compositional properties in larger programs.

\begin{fpfig}[t]{Filtering memories that satisfy a boolean expression.}{fig-solve}
\begin{eqnarray*}
\solve{\stores}{\etrue} &=& \stores\\
\solve{\stores}{\efalse} &=& \varnothing\\
\solve{\stores}{\flip{\cid}{w}} &=& \{ \store \in \stores \mid \store(\flip{\cid}{w}) \} \\
\solve{\stores}{\secret{\cid}{w}} &=& \{ \store \in \stores \mid \store(\secret{\cid}{w}) \} \\
\solve{\stores}{\view{\cid}{w}} &=& \{ \store \in \stores \mid \store(\view{\cid}{w}) \} \\
\solve{\stores}{\oracle{w}} &=& \{ \store \in \stores \mid \store(\oracle{w}) \} \\
\solve{\stores}{(\enot\ \be)} &=& \stores - (\solve{\stores}{\be})\\
\solve{\stores}{(\be_1\ \eand\ \be_2)} &=& (\solve{\stores}{\be_1}) \cap (\solve{\stores}{\be_2}) \\
\solve{\stores}{(\be_1\ \eor\ \be_2)} &=& (\solve{\stores}{\be_1}) \cup (\solve{\stores}{\be_2}) \\
\solve{\stores}{(\be_1\ \exor\ \be_2)} &=&
 ((\solve{\stores}{\be_1}) \cap (\stores - \solve{\stores}{\be_2})) \cup\\
 && ((\stores - \solve{\stores}{\be_1}) \cap (\solve{\stores}{\be_2})) \\
\solve{\stores}{\select{\be_1}{\be_2}{\be_3}} &=&
 ((\solve{\stores}{\be_1}) \cap (\solve{\stores}{\be_2})) \cup \\
 && ((\stores - \solve{\stores}{\be_1}) \cap (\solve{\stores}{\be_3})) \\
\solve{\stores}{\OT{\be_1}{\be_2}{\be_3}} &=&
 ((\solve{\stores}{\be_1}) \cap (\solve{\stores}{\be_2})) \cup\\
 && ((\stores - \solve{\stores}{\be_1}) \cap (\solve{\stores}{\be_3}))
\end{eqnarray*}
\end{fpfig}

\subsection{Method 1: Enumerating $\runs(\prog)$}

Here we describe a straightforward technique for obtaining program
distributions $\progd(\prog)$ through direct computation of
$\runs(\prog)$. An implementation of these techniques along with
Examples from Sections \ref{section-minicat-examples} and
\ref{section-metalang-ygc} is available online \cite{XXX}.

In Figure \ref{fig-solve} we define the algorithm $\solve{\stores}{\be}$
which filters $\stores$ based on satisfaction of $\be$. This algorithm
is fundamental to our brute force methods. It's correctness is characterized
as follows. 
\begin{lemma}
  For all $\stores$ and $\be$ with $\vars(\be) \subseteq \dom(\stores)$,
  $(\solve{\stores}{\be}) = \{ \store \in \stores \ \mid\ \lcod{\store,\be}{\cid} = 1 \}$
  for some $\cid$.
\end{lemma}
\begin{proof}
  Since we assume safety of programs we can assume that all variables in $\be$ have the
  same owner $\cid$. The result otherwise follows in a straightforward manner by induction
  on $\be$. 
\end{proof}

\begin{lemma}
  Given $\prog$ where $\iov(\prog) = S \cup V$ and $\flips(\prog) = F$. Define:
  \begin{eqnarray*}
    {tt}\ \ \stores\ (\eassign{\view{\cid}{w}}{\be}) &\defeq& \begin{array}{l}
      \mathrm{let}\ \stores' = \solve{\stores}{\be} \ \mathrm{in}\\
      \ \ \{\extend{\store}{\view{\cid}{w}}{\etrue} \mid \store \in \stores' \}\ \cup\\
      \ \ \{\extend{\store}{\view{\cid}{w}}{\efalse} \mid \store \in \stores - \stores' \}\end{array}\\[2mm]
    \cruns(\prog) &\defeq& \mathit{foldl}\ {tt}\ \mems(S \cup F)\ \prog
  \end{eqnarray*}
  Then $\cruns(\prog) = \runs(\prog)$.
\end{lemma}

\begin{lemma}
  Given $\prog$ where $\iov(\prog) = S \cup V$ and $\flips(\prog) = F$. Define:
  $$
  \pdf{S \cup F \cup V}(\store) = \begin{cases}1/2^{|S \cup F|} & \text{if\ } \store \in
    \cruns(\prog) \\ 0 & \text{otherwise} \end{cases} 
  $$
  Then:
  $$
  \margd{\pdf{S \cup F \cup V}}{S \cup V} = \progd(\prog)
  $$
\end{lemma}

\subsection{Method 2: Rewriting to Datalog}

$$
f(\eassign{\view{\cid}{w}}{\be}) = (\view{\cid}{w}, (\solve{(\mems(\vars\ \be))}{\be}))
$$

$$
\mathit{map}\ f\ \prog
$$

\begin{mathpar}
  \logit{x \mapsto 1} = x

  \logit{x \mapsto 0} = \neg x

  \logit{\{ x_1 \mapsto \beta_1, \ldots, x_n \mapsto \beta_n\}} =
  \logit{x_1 \mapsto \beta_1} \wedge \cdots \wedge \logit{x_n \mapsto \beta_n}

  \mathit{clauses}(x,\{ \store_1,...,\store_n \}) = x \gets \logit{\store_1} \vee \cdots \vee x \gets \logit{\store_n}
\end{mathpar}

$$
\mathit{map}\ \mathit{clauses}\ (\mathit{map}\ f\ \prog)
$$
