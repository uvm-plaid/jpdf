\section{Supplemental Examples}

\subsection{$\metaprot$ and $\minicat$ Protocol Examples}

\begin{example}[Additive Secret Sharing]
    \label{example-he}
Additive secret sharing is an MPC protocol where $k$ parties split up
their secrets into \emph{shares}, with the property that all $k$
shares are required to reconstruct the secret and any fewer reveals
nothing. Additionally, this sharing enjoys additive homomorphism. In
the binary field it is well-known that we can extend our use
of $\ttt{xor}$ and flips to generate an arbitrary number of shares
with additive homomorphism. 

For example, to generate shares for three-party additive sharing,
clients 1,2, and 3 can break up and distribute their shares to
each other as follows:
{\small
\begin{verbatimtab}
     v[2,"s1"] := flip[1,"1"] xor flip[1,"s1"] xor s[1,"1"];
     v[3,"s1"] := flip[1,"1"];
     v[1,"s2"] := flip[2,"1"] xor flip[2,"s2"] xor s[2,"1"];
     v[3,"s2"] := flip[2,"1"];
     v[1,"s3"] := flip[3,"1"] xor flip[3,"s3"] xor s[3,"1"];
     v[2,"s3"] := flip[3,"1"] \end{verbatimtab}
}
Then, assuming that party 0 is a ``public'' client, each party sums
its own shares, and then the sum of the sum of shares is revealed
in the output view $\vx{0}{output}$. By additive homomorphism of
$\ttt{xor}$, this output is the sum of the 3 secrets.
{\small
\begin{verbatimtab}
     v[0,"ss1"] := v[1,"s2"] xor v[1,"s3"] xor flip[1,"s1"];
     v[0,"ss2"] := v[2,"s1"] xor v[2,"s3"] xor flip[2,"s2"];
     v[0,"ss3"] := v[3,"s1"] xor v[3,"s2"] xor flip[3,"s3"];
     v[0,"output"] := v[0,"ss1"] xor v[0,"ss2"] xor v[0,"ss3"] \end{verbatimtab}
}
We note that, as for MPC protocols generally, the notion of
``secrecy'' here is subtle and not the same as for encryption as in
Example \ref{example-otp}, particularly when the threat model allows
protocol participants to be corrupt. For example, suppose that $1 \in C$, and
$\sx{1}{1}$ is $1$, and after running the protocol suppose that
$\vx{0}{output}$ is 0. Then the adversary knows that 2 and 3's secrets
are either 1 and 0, or 0 and 1, with 50/50 probability, and definitely
neither 0 and 0, nor 1 and 1. 
\end{example}

Buliding on Example \ref{example-he}, we can define a function
$\ttt{share3}$ that abstracts the process of splitting a given
client's secret into 3 separate shares.
\begin{example} \label{example-share3} The function $\ttt{share3}$ 
  splits a client's secret into 3 shares returned as a record
  with fields $\ttt{s1-3}$:
  {\small
  \begin{verbatimtab}
      share3(client, secretid) {
        let s1 = flip[client, "share1"] in
        let s2 = flip[client, "share2"] in
        let s3 = (s1 xor s2) xor s[client, "s:" || secretid] in
        {s1 = s1;s2 = s2;s3 = s3}
      } \end{verbatimtab}
  }
  Here is a $\metaprot$ program that uses this function definition:
  {\small
    \begin{verbatimtab}
      let shares = share3(1, "mysecret") in
      v[2,s1] := shares.s2;
      v[3,s1] := shares.s3 \end{verbatimtab}
  }
  which generates the following $\minifed$ program, as we formalize in Example \ref{example-share3-eval}
  below:
  {\small
  \begin{verbatimtab}
      v[2,s1] := flip[1, "share2"];
      v[3,s1] := flip[1, "share1"] xor flip[1, "share2"] xor s[1, "s:mysecret"] \end{verbatimtab}
  }
\end{example}

\begin{example}
  \label{example-share3-eval}
  Let $\codebase,e_{\ref{example-share3}}$ be the $\metaprot$ program and let 
  $\prog_{\ref{example-share3}}$ be the  $\minifed$ program defined
  in Example \ref{example-share3}. We refer to the latter as ``accumulated''
  by evaluation of the former in the sense that $\config{\varnothing}{e_{\ref{example-share3}}}
  \redxs \config{\prog_{\ref{example-share3}}}{\ttt{()}}$.
\end{example}

\begin{example}
  Given $\ttt{share3}$ as defined in Example \ref{example-share3} and
  annotation:
  $$\tas(\ttt{share3}) =  \ttt{cid(client)}\ *\ \ttt{string(sid)}$$
  the type of $\ttt{share3}$ is $\tas(\ttt{share3}) \rightarrow \tau,\varnothing$
  where $\tau$ is:
  $$
  \ttt{\{ s1 : bool[client]; s2 : bool[client]; s3 : bool[client] \}}
  $$
\end{example}

\subsection{Program Distribution Examples}
\label{section-pmf-examples}

In the following, we will refer to program examples defined in
Section \ref{section-minicat-examples} as $\prog_x$, where $x$
is the Example number. 



\paragraph{Example \ref{example-otp} revisited}
%For small programs it is
%helpful to visualize their basic distributions graphically as their
%truth tables, where each row corresponds to an equally likely final
%memory for the given program. In Figure \ref{fig-basic-distributions}
%we show this for Example \ref{example-otp}. In this truth table we can
%see that $\sx{1}{0}$ and $\vx{2}{2}$ are perfectly correlated.
In this example, 
from the adversarial view, $\sx{1}{0}$ is 1 or 0 with equal
probability given any value of $\vx{0}{0}$. More precisely, for any
$\beta_1,\beta_2$ we have:
\begin{mathpar}
(\progd(\prog_{\ref{example-otp}}))(\{\sx{1}{0} \mapsto \beta_1 \}|\{ \vx{0}{0} \mapsto \beta_2 \}) = .5
\end{mathpar}
Note that this corresponds to \emph{perfect secrecy} as described
in \cite{barthe2019probabilistic}.

\paragraph{Example \ref{example-lambda-obliv}(a) revisited} As demonstrated
by Example \ref{example-lambda-obliv} taken from \cite{darais2019language}, it is possible
to leak information probabilistically through observable dependencies
in views.
%In the basic distribution for Example \ref{example-lambda-obliv}(a)
%illustrated in Figure \ref{fig-basic-distributions},
We can see that
the adversarial views $\vx{0}{0}$ and $\vx{0}{1}$ are both 1 in
three different runs of the protocol, and in 2 out of 3 the
honest secret $\sx{1}{0}$ is 1. More precisely:
$$
(\progd(\prog_{\ref{example-lambda-obliv}(a)}))(\{\sx{1}{0} \mapsto 1 \}|\{ \vx{0}{0} \mapsto 1, \vx{0}{1} \mapsto 1 \}) = 2/3
$$
An important point illustrated by this example is that the
probability of values in views, and not just the values themselves,
can release information to the adversary.

\paragraph{Example \ref{example-he} revisited} This standard example
of an MPC secure protocol illustrates how a functionality- in this
case addition (in the binary field)- can be implemented securely. As
mentioned previously, it is \emph{not} the case that \emph{no}
information about input secrets can be revealed to the adversary in
the MPC threat model, since the result of the functionality can itself
reveal information. For example, if the result of addition is $0$,
the adversary knows that two of the clients' secrets must be 0
and the third's must be 1, though not whose is whose- so we can
observe:
\begin{mathpar}
  (\progd(\prog_{\ref{example-he}}))(\{\sx{1}{1} \mapsto 0, \sx{2}{1} \mapsto 1, \sx{3}{1} \mapsto 0 \}|\{ \vx{0}{output} \mapsto 0 \}) = .33
\end{mathpar}
Further, since the standard MPC threat model allows participating
clients to be corrupted we can formally consider the impacts on
information released to the adversary by conditioning the program
distribution on corrupt secrets.  So if we assume client 2
with input secret 1 is corrupted, and we assume the output is 0, then
the adversary knows with certainty that clients 1 and 3 both have input
secrets 0. We can express this as:
$$
\progd(\prog_{\ref{example-he}}))(\{\sx{1}{1} \mapsto 0, \sx{3}{1} \mapsto 0 \}|\{ \sx{2}{1} \mapsto 1, \vx{0}{output} \mapsto 0 \})
= 1
$$ A critical point to note is that this information revealed to the
adversary does not violate MPC security- under the assumption of
public release of the functionality result, and under the assumption
of the possible corruption of clients, this is the cost of doing
business. Rather, the concern is that the protocol does not release
any additional information through information communicated to views
of corrupted parties. Additive secret sharing as realized in
$\prog_{\ref{example-he}}$ is MPC secure in this sense- so for
example, for any $\beta_1$ and $\beta_2$ we have:
$$
\begin{array}{c}
\progd(\prog_{\ref{example-he}}))(\{\sx{1}{1} \mapsto 0 \}|\{ \sx{2}{1} \mapsto 0, \vx{0}{output} \mapsto 0 \})
= \\
\progd(\prog_{\ref{example-he}}))(\{\sx{1}{1} \mapsto 0 \}|\{ \vx{2}{s1} \mapsto \beta_1, \vx{2}{s3} \mapsto \beta_2, \sx{2}{1} \mapsto 0, \vx{0}{output} \mapsto 0 \})
= \\
.5
\end{array}
$$
